\documentclass[a4paper,12pt,fleqn]{article}
\usepackage[utf8]{inputenc}
\usepackage[T1]{fontenc}
\usepackage{lmodern}
\usepackage{ngerman}
\usepackage{amsmath}
\usepackage{amssymb}
\usepackage{color}
\definecolor{c1}{RGB}{00,40,80}
\usepackage[colorlinks=true,linkcolor=c1]{hyperref}
\begin{document}

\noindent
{\huge\textbf{Kombinatorik}}
\vspace{2mm}

\noindent
Februar 2016

\tableofcontents
\vspace{4mm}

\noindent
\textbf{Zusammenfassung}. Dieser Text soll die elementare
Mengenlehre, das Konzept der Funktion und die elementare
Kombinatorik zu einem Formalismus zusammenfassen, so dass sich
die einzelnen Theorien gegenseitig bereichern. Um kombinatorische
Einsichten zu gewinnen muss man sich natürlich auf endliche
Mengen beschränken.

\section{Grundlegende Prinzipien}
\subsection{Permutationen}
Eine Permutation kann als Umordnung eines Tupels verstanden
werden. Die identische Umordnung (die alles so lässt wie es ist),
wollen wir auch als Umordnung zulässt ansehen. Ein Tupel aus
einem einzigen Element hat daher die identische Permutation als
einzige Umordnung. Ein Tupel mit zwei Elementen hat zwei
Permutationen. Ein Tupel mit drei Elementen hat schon sechs.

Spätestens bei vier Elementen wird das Zählen umständlich.
Bezeichnen wir die Anzahl der Permutationen eines Tupels
mit $n$ Elementen als $n!$. Fixiert man nun aber ein Element,
so gibt es für die restlichen Elemente noch $n-1$ Permutationen.
Bedenkt man, dass das Element an jeder der $n$ Stellen stehen
kann, so ergibt sich die Rekursionsformel
\begin{equation}
n! = n(n-1)!.
\end{equation}
Benutzt man diese Rekursionsformel wiederholt, so ist
\begin{equation}
n! = \prod_{k=1}^n k = 1\times 2\times\ldots\times(n-1)\times n
\end{equation}
unmittelbar einsichtig. Eine Permutation ist nun aber nichts
anderes als eine bijektive Selbstabbildung. Das heißt folgendes.
Alle Elemente des Tupels bilden eine Menge $M$ mit $|M|=n$.
Eine Permutation ist nun eine bijektive Funktion aus
$\mathrm{Abb}(M,M)$. Von den Funktionen in $\mathrm{Abb}(M,M)$
sind also genau $|M|!$ bijektiv. Bezeichnet man die Menge
der Bijektionen aus $\mathrm{Abb(M,M)}$ mit
$\mathrm{Bij}(M,M)$, so ergibt sich die Formel
\begin{equation}
|M|! = \mathrm{Bjn}(M,M).
\end{equation}
Daher kann man
\begin{equation}
M!:=\mathrm{Bjn}(M,M)
\end{equation}
definieren.

\subsection{Variationen}
Wir wollen jetzt die Zusammenhänge aus dem letzten Abschnitt
verallgemeinern. Anstelle von Bijektionen betrachtet man
nun Injektionen aus $\mathrm{Abb}(D,Z)$. Setzt man $D=Z$, so
ergibt sich wieder die Menge der Bijektionen, weil injektive
Selbstabbildungen auf einer endlichen Menge zwingend bijektiv
sind. Man betrachtet nun die Zielmenge $Z$, die stets größer sein
muss als der Definitionsbereich $D$. Von der Zielmenge zurück
zum Definitionsbereich ist wie das Ablegen von $n=|Z|$
unterschiedlichen Objekten auf $k=|D|$ freie Plätze. Die elementare
Kombinatorik lehrt, dass es dafür
\begin{equation}
V(n,k) = \frac{n!}{(n-k)!} = n^{\underline k}
\end{equation}
Möglichkeiten gibt. Setzt man jetzt $n=k$ so ergibt sich wegen
$0!=1$ wieder $n!$. Daher gilt auch $n^{\underline n}=n!$.

Wenn man Wiederholungen erlaubt, so gilt stattdessen die
Formel
\begin{equation}
V_w(n,k) = n^k.
\end{equation}
Wenn man das gleiche Element aus $Z$ nun aber auf mehrere
frei Plätze von $D$ legen kann, so bedeutet das umgekehrt,
dass die Funktion nicht mehr injektiv zu sein braucht.
Stattdessen ergibt sich die Menge aller Funktionen. Daher
gilt
\begin{equation}
|Z|^{|D|} = |\mathrm{Abb}(D,Z)|.
\end{equation}
Aus diesem Grund definiert man
\begin{equation}
Z^D := \mathrm{Abb}(D,Z)
\end{equation}
sowie
\begin{equation}
Z^{\underline D}:=\mathrm{Ijn}(D,Z).
\end{equation}

\subsection{Zahlen}
Es ist sinnvoll, auch die natürlichen Zahlen durch Mengen
auszudrücken. Man definiert
\begin{gather*}
0:=\{\},\\
1:=\{0\} = \{\{\}\},\\
2:=\{0,1\} = \{\{\},\{\{\}\}\},\\
3:=\{0,1,2\}
\end{gather*}
usw.

\subsection{Produktmenge}
Das kartesische Produkt von $A,B$, auch Produktmenge genannt,
bezeichnet man mit $A\times B$. Nun gilt die Formel
\[|A\times B| = |A||B|.\]
Bezeichnet man mit $A^2$ die Produktmenge von $A$ mit sich selbst
und mit $A^n$ die Potenz, so gilt $|A^n|=|A|^n$. Nun gilt aber
$2=\{0,1\}$. Daher lässt sich $A^2$ auch als $\mathrm{Abb}(\{0,1\},A)$
auffassen. Wie soll man das interpretieren?

Nun, ein geordnetes Paar aus $A^2$ ist nichts anderes als eine
endliche Folge $(a_0,a_1)$ mit Werten $a_k\in A$. Besteht der
Definitionsbereich einer Funktion aus den ersten natürlichen Zahlen,
so handelt es sich um eine endliche Folge.

\subsection{Potenzmenge}
Die Potenzmenge $P(A)$ einer endlichen Menge $A$ hat
$2^{|A|}$ Elemente. Somit kann man auch $P(A)=2^A$ schreiben.
Wegen $2=\{0,1\}$ muss es sich bei $2^A$ um die Menge
$\mathrm{Abb}(A,\{0,1\})$ handeln. Wie kann man das interpretieren?

Nun, eine Teilmenge $T\subseteq A$ kann als eine
Indikatorfunktion \(\chi\) mit Werten in $\{0,1\}$ interpretiert werden.
Für jedes Element $x\in A$ sagt die Indikatorfunktion ob $x$
zu $T$ gehört (wahr=1) oder nicht (falsch=0). Es ist also
\[\chi(x) = [x\in A].\]
Zu jeder Teilmenge gibt es genau eine Indikatorfunktion.

Was versteht man nun unter $3^A$? Nun, hier liegen die Werte der
Indikatorfunktion in $\{0,1,2\}$. Die Indikatorfunktion zählt
sozusagen, wie oft ein Element vorkommt. Eine solche
Indikatorfunktion kann als Multimenge interpretiert werden.
Bei $3^A$ handelt es sich also um die Multimengen über $A$, wobei
jedes Element aus $A$ maximal doppelt vorkommen darf.

Somit ist $n^A$ die Menge aller Multimengen, bei denen jedes
Element aus $A$ maximal $n-1$ mal vorkommen darf.

\subsection{Zahlen als Mengen von Folgen}
Jede natürliche Zahl $n$ lässt sich als $n^1$ schreiben. Und $n^1$
kann man als
\[\mathrm{Abb}(\{0\},\{0,1,\ldots,n-1\})\]
betrachten. Die Zahl $b^k$ lässt sich als Menge aller Folgen
\[(a_0,\ldots,a_{k-1}),\qquad a_k\in\{0,\ldots,b-1\}\]
interpretieren. Das lässt sich als Darstellung der Zahlen von null
bis $b^k-1$ betrachten, was $b^k$ ja auch ist. Die Zahlen sind
dabei mit $k$ Ziffern zur Basis $b$ dargestellt.

\subsection{Kombinationen}
Beachtet man bei den Variationen von $n$ Elementen aus der
Grundgesamtheit auf $k$ freien Plätzen die Reihenfolge nicht mehr,
so erhält man die Kombinationen $C(n,k)$. Wir wissen aber, dass
es auf $k$ Plätzen $k!$ mögliche Reihenfolgen gibt. Verschiedene
Reihenfolgen werden jetzt als äquivalent betrachtet und zu
einer Äquivalenzklasse zusammengefasst. Über der Menge der
injektiven Funktionen ergibt sich eine Zerlegung in disjunkte
Teilmengen. Man spricht von einer Partition, auch Quotientenmenge
genannt. Die Zahl der Äquivalenzklassen ist
\begin{equation}
C(n,k) = \frac{V(n,k)}{k!} = \frac{n!}{k!\,(n-k)!}
= \frac{n^{\underline k}}{k!}.
\end{equation}
So definiert man auch den Binomialkoeffizient, es ist
\begin{equation}
\binom{n}{k} := \frac{n^{\underline k}}{k!}.
\end{equation}
Jetzt stellt sich noch die Frage, wie das Vergessen der Reihenfolge
als Formel dargestellt werden kann, um eine klare
Berechnungsvorschrift für die Quotientenmenge zu erhalten.

Die injektive Funktion ordnet ja nun jedem Platz genau eine Karte
zu, wobei jeder Pfeil auf eine andere Karte zeigt. Wenn die
Reihenfolge der Plätze keine Rolle spielen soll, so muss man
alle Funktionen mit permutiertem Definitionsbereich als gleich
ansehen. Bezeichnen wir mit $\sigma$ eine Permutation
aus $D!$. Nun definieren man die Äquivalenzrelation
\begin{equation}
f\sim g  := \exists \sigma{:}\; f\circ\sigma = g.
\end{equation}
Mit $\circ$ ist die Komposition gemeint, welche durch
\begin{equation}
(f\circ \sigma)(x) := f(\sigma(x))
\end{equation}
definiert ist.

Alle Permutationen bilden bezüglich der Komposition eine Gruppe,
die symmetrische Gruppe $S_k$ genannt wird. Es gilt also
$D!=S_k$. Weiterhin handelt es sich bei $f\circ\sigma$ um eine
Gruppenaktion. Die Permutation $\sigma$ aus der symmetrischen
Gruppe wirkt hierbei auf $f$. Man definiert nun die Menge
\[f\circ S_k := \{f\circ\sigma\;|\;\sigma\in S_k\}.\]
Diese Menge wird als Orbit oder Bahn von $f$ bezeichnet.
Bei den zuvor besprochenen Äquivalenzklassen handelt es sich
um ebendiese Orbits.

Die Quotientenmenge ist nun genau die Menge unterschiedlicher
Bahnen und wird daher als Bahnenraum bezeichnet. Die Kardinalität
des Bahnenraumes ist die Anzahl der möglichen Kombinationen.
Das heißt es ist
\begin{equation}
C(n,k) = |\{f\circ S_k\;|\;f\in Z^{\underline D}\}|.
\end{equation}
An dieser Stelle tut sich eine höchst interessante Frage auf,
auf welche wir jetzt noch nicht eingehen wollen. Die symmetrische
Gruppe $S_k$ hat Untergruppen $U$, die Permutationsgruppen
genannt werden. Nun stellt sich aber die Frage, welche Formeln
sich für
\begin{equation}
|\{f\circ U\;|\;f\in Z^{\underline D}\}|
\end{equation}
ergeben. Die Antwort ist, dass diese Formeln wahrscheinlich
beliebig kompliziert werden. Nach dem Satz von Cayley lässt sich
nämlich jede Gruppe als Permutationsgruppe interpretieren. Und
es gibt unvorstellbar komplizierte Gruppen.

Daher würde man gerne erst einmal $|U|$ für eine Untergruppe
$U$ berechnen. Der Satz von Lagrange sagt aus, dass $|U|$ schon einmal
ein Teiler von $|S_k|=k!$ sein muss.

Allgemeiner gibt es beliebige Teilmengen $T$ von $S_k$ wo man
eine Formel für
\[|\{f\circ T\;|\;f\in Z^{\underline D}\}|\]
angeben möchte, sofern $T$ auch durch eine Formel
beschrieben ist. Wir diskutieren nun, in welchen Fällen die
Formel
\begin{equation}
|\{f\circ T\;|\;f\in Z^{\underline D}\}|
= \frac{|Z^{\underline D}|}{|T|}
\end{equation}
gilt, wobei wir uns auf Gruppen beschränken.
Bei diesen gilt die Bahnformel
\begin{equation}
|U| = |f\circ U|\,|U_f|,
\end{equation}
wobei mit $U_f$ die Fixgruppe (auch Stabilisator genannt) von $U$
gemeint ist. Diese ist definiert durch
\begin{equation}
U_f := \{\sigma\in U\;|\; f\circ\sigma = f\}.
\end{equation}
Nun sei bei $U$ die Fixgruppe trivial. So sagt man,
wenn diese nur aus der identischen Permutation $\mathrm{id}$,
dem neutralen Element, besteht. In diesem Fall gilt $|f\circ U|=|U|$.
Das bedeutet, jeder Orbit ist gleich groß und hat $|U|$ Elemente.
Die Gesamtzahl ist aber die Anzahl der Orbits mal die Anzahl der
Elemente im Orbit. Daher ergibt sich
\begin{equation}
|\{f\circ U\;|\;f\in Z^{\underline D}\}|\,|U|
= |Z^{\underline D}|.
\end{equation}
Damit die Formel gültig ist, reicht es also aus, wenn die
Fixgruppe für jedes beliebige $f$ trivial ist. Bei injektiven
Funktionen ist das aber immer der Fall.

Die Quotientenmenge wird auch kurz als $X/U$ bezeichnet,
wobei $X$ die Menge sein soll, auf welche die Gruppenaktion wirkt.
Daher ergibt sich nun die hübsche Formel
\begin{equation}
|Z^{\underline D}/U| = \frac{|Z^{\underline D}|}{|U|}
= \frac{n^{\underline k}}{|U|}.\qquad(k=|D|,\;n=|Z|)
\end{equation}
%
Betrachten wir nun Kombinationen mit zurücklegen.
Hier wir die Bedingung aufgehoben, dass die Funktionen injektiv
sein müssen. Keinesfalls gilt jetzt $C_w(n,k)=n^k/k!$. Dazu
muss man bedenken, dass bei mehrfachen Vorkommen von Elementen
auf den $k$ freien Plätzen die Funktion für einige Permutationen
übereinstimmt. Die Fixgruppe ist also nicht mehr trivial.

Mit der Methode {\glqq}Stars and bars{\grqq} findet man vielmehr
die Formel
\begin{equation}
C_w(n,k) = \binom{n+k-1}{k}.
\end{equation}

\subsection{Vereinigungsmenge}
Ein grundlegendes Prinzip ist das Prinzip von Inklusion und
Exklusion. Für die Anzahl der Elemente einer Vereinigung
gilt
\begin{equation}
|A\cup B| = |A|+|B|+|A\cap B|.
\end{equation}
Bei einer disjunkten Vereinigung gilt daher
\[|A\sqcup B| = |A|+|B|.\]
Mit diesem Prinzip zerlegt man nun auch
\[|A\cup B\cup C| = |A\cup B|+|C|-|(A\cup B)\cap C|.\]
Jetzt beachtet man noch
\[(A\cup B)\cap C = (A\cap C)\cup(B\cap C).\]
Hiermit ergibt sich
\begin{gather*}
|A\cup B\cup C| = |A|+|B|+|C|-|A\cap B|-|A\cap C|-|B\cap C|\\
+|(A\cap B)\cap (B\cap C)|.
\end{gather*}
Beachtet man nun noch
\[(A\cap B)\cap (B\cap C) = A\cap B\cap C\]
so erhält man
\begin{equation}
\begin{split}
|A\cup B\cup C| &= |A|+|B|+|C|\\
&-|A\cap B|-|A\cap C|-|B\cap C|\\
&+|A\cap B\cap C|.
\end{split}
\end{equation}
Man erkennt das folgende allgemeine Muster. Es lautet
\begin{equation}
\bigg|\bigcup_{i=1}^n A_i\bigg|
= \sum_{k=1}^n (-1)^{k-1} \sum_{i_1<\ldots<i_k}
|A_{i_1}\cap\ldots\cap A_{i_k}|.
\end{equation}

\subsection{Partitionen}

Eine Menge mit $n$ Elementen lässt sich in disjunkte Teilmengen
zerlegen. Wie viele Möglichkeiten gibt es dafür? Und wie viele sind
es, wenn die Anzahl der Teilmengen fix sein soll? Ebenso lässt sich
eine natürliche Zahl aus $\mathbb N_1$ in eine Summe von natürlichen
Zahlen aus $\mathbb N_1$ zerlegen. Aber wie viele solcher Summen gibt
es? Und wie viele sind es, wenn die Anzahl der Summanden fix sein
soll?

Zunächst wollen wir die Beziehung zwischen den Zerlegungen
von Mengen und Zahlen klarstellen. Eine Summe ist gleichbedeutend mit
der Vereinigung disjunkter Mengen. Hat man aber nun die Menge
$\{a,b,c\}$ so gibt es folgende Zerlegungen in zwei Teilmengen:
\begin{gather*}
\{a\}\cup\{b,c\},\\
\{b\}\cup\{a,c\},\\
\{c\}\cup\{a,b\}.
\end{gather*}
Für die Zahl drei gibt es aber nur die eine Zerlegung $1+2$.
Offensichtlich bekommt man die Anzahl der Möglichkeiten für
die Partition von Zahlen, wenn man bei der Partition von Mengen
die Reihenfolge vergisst.

Um virtuoser rechnen zu können wollen wir Partitionen und die
Menge der Partitionen als eigenständige mathematische
Objekte betrachten, so wie es bei den Variationen und Kombinationen
bereits erfolgt ist.

Man nimmt nun die Menge $\{a,b,c\}$ als Definitionsbereich einer
Funktion. Als Zielmenge nimmt man zwei Elemente $v,w$.
Die Funktion soll surjektiv sein. Daher ergeben sich die
sechs Funktionen
\begin{gather*}
f_1(a)=v,\quad f_1(b)=v,\quad f_1(c)=w,\\
f_2(a)=v,\quad f_2(b)=w,\quad f_2(c)=v,\\
f_3(a)=w,\quad f_3(b)=v,\quad f_3(c)=v,\\
f_4(a)=w,\quad f_4(b)=w,\quad f_4(c)=v,\\
f_5(a)=w,\quad f_5(b)=v,\quad f_5(c)=w,\\
f_6(a)=v,\quad f_6(b)=w,\quad f_6(c)=w.
\end{gather*}
Bei der Zielmenge vergisst man nun die Reihenfolge. D.h. man
vertauscht $v$ und $w$. Die verbleibenden drei Möglichkeiten
entsprechen genau den Partitionen in zwei Teilmengen.

Sei $n:=|Z|$ und $k:=|D|$. Bezeichne mit $\mathrm{Sjn}(D,Z)$
die Menge der Surjektionen von $D$ nach $Z$. Eine Partition
lässt sich nun als Orbit
\begin{equation}
S_n\circ f := \{\sigma\circ f\;|\;\sigma\in S_n\}
\end{equation}
ausdrücken, wobei $f$ eine Surjektion sein soll. 
Die Menge der Partitionen ist daher beschrieben durch
\begin{equation}
\{S_n\circ f\;|\;f\in\mathrm{Sjn}(D,Z)\}.
\end{equation}
Die Zahlen
\begin{equation}
\begin{Bmatrix}
k\\ n
\end{Bmatrix}
:=|\{S_n\circ f\;|\;f\in\mathrm{Sjn}(D,Z)\}|
\end{equation}
heißen Stirling-Zahlen zweiter Art. Wir wollen sie
als Partitionszahlen bezeichnen. Die Zahlen
\begin{equation}
B_k:=\sum_{n=0}^k \begin{Bmatrix}
k\\ n
\end{Bmatrix}
\end{equation}
werden als Bellzahlen bezeichnet. Die Bellzahl $B_k$ ist
die gesamte Anzahl an Möglichkeiten, eine Menge in disjunkte
Teilmengen zu zerlegen. 

Bei den Partitionen von Zahlen muss die Reihenfolge des
Definitionsbereiches auch noch vergessen werden. Somit ergibt
sich die Darstellung
\begin{equation}
p\begin{pmatrix}k\\ n\end{pmatrix}
:=|\{S_n\circ f\circ S_k\;|\; f\in\mathrm{Sjn}(D,Z)\}|.
\end{equation}
Die Funktion $p(k,n)$ gibt die Anzahl der Möglichkeiten an, die
Zahl $k$ in $n$ Summanden zu zerlegen. Die Funktion
\begin{equation}
p(k) := \sum_{n=1}^k p\begin{pmatrix}
k\\ n
\end{pmatrix}
\end{equation}
wird als Partitionsfunktion bezeichnet. Die Zahl $p(k)$ ist
die gesamte Anzahl an Möglichkeiten, die Zahl $k$ in Summanden
zu zerlegen.

\subsection{Twelvefold way}
Die bisher dargestellten Prinzipien lassen sich übersichtlich
zu einer Tabelle zusammenfassen, die als \textit{Twelvefold way}
bezeichnet wird. Es handelt sich um eine Tabelle mit $4\times 3$
Einträgen, die noch Lücken enthält, welche sich sinnvoll ausfüllen
lassen. Man kann von einer Art Periodensystem, dem Periodensystem
der Kombinatorik sprechen.

\section{Grundlegende Werkzeuge}
\subsection{Dreieckszahlen}
Dreieckszahlen sind Zahlen die man wie beim pascalschen Dreieck
als Dreiecksschema anordnen kann. Dies beruht im wesentlichen
auf einer Rekursionsformel in zwei Variablen.

Binomialkoeffizienten:
\begin{equation}
\binom{n}{k} = \binom{n-1}{k-1}+\binom{n-1}{k}
\end{equation}
Zykelzahlen (Stirling-Zahlen erster Art):
\begin{equation}
\begin{bmatrix} n\\ k\end{bmatrix}
= \begin{bmatrix} n-1\\ k-1\end{bmatrix}
+ (n-1)\begin{bmatrix} n-1\\ k \end{bmatrix}
\end{equation}
Partitionszahlen (Stirling-Zahlen zweiter Art):
\begin{equation}
\begin{Bmatrix} n\\ k \end{Bmatrix}
= \begin{Bmatrix} n-1\\ k-1 \end{Bmatrix}
+ k\begin{Bmatrix} n-1\\ k \end{Bmatrix}
\end{equation}
Zahlpartitionen:
\begin{equation}
p\begin{pmatrix} n\\ k \end{pmatrix}
= p\begin{pmatrix} n-1\\ k-1 \end{pmatrix}
+ p\begin{pmatrix} n-k\\ k \end{pmatrix}
\end{equation}
Es gibt ziemlich viele Formeln für diese Zahlen. Einige davon
drücken Beziehungen zwischen den unterschiedlichen Dreieckszahlen
aus. In \textit{Concrete Mathematics} beschäftigt sich ein
90-seitiges Kapitel allein mit Binomialkoeffizienten.

\subsection{Erzeugende Funktionen}
Viele Probleme der Kombinatorik lassen sich zunächst als
rekursiv definierte Folge formulieren. Betrachten wir nun erst einmal
die endliche Folge
\[(a_0,a_1,a_2,a_3).\]
Diese Folge ist ein Tupel mit vier Elementen und lässt sich
daher auch als Punkt im Koordinatenraum der Dimension vier darstellen.
Die kanonische Basis lässt sich mit der Polynombasis identifizieren.
Daher lässt sich das Tupel als Linearkombination
\[a_0X^0+a_1X^1+a_2X^2+a_3X^3\]
schreiben. Die endliche Folge lässt damit als ein Element
des Polynomringes $R[X]$ mit $a_k\in R$ interpretieren.

Diese Linearkombination wird nun ins unendliche fortgesetzt.
D.h. man bildet
\begin{equation}
G\{a_k\}(X) := \sum_{k=0}^\infty a_kX^k.
\end{equation}
Die Menge der Folgen wird damit als Menge der formalen
Potenzreihen $R[[X]]$ interpretiert.

Jede endliche Folge lässt sich auch darin einbetten.
Geht die Folge z.B. bis zum Index $N=3$, so lässt sich die Folge
für alle $k>N$ einfach durch $a_k=0$ erweitern.

Man kann nun die formale Variable $X$ gegen eine tatsächliche
Variable $x$ ersetzen und die entstehende Funktion betrachten.
Diese nennt man dann erzeugende Funktion der Folge. Macht man anstelle
von $X=x$ die Substitution $X=1/z$, so ergibt sich die
Z-Transformation. Behandelt man $X$ jedoch formal und
benutzt sich formale Identitäten, so spielen Konvergenzfragen
keine rolle.

Tatsächlich gelten viele Identitäten auch formal, so dass eine
solche Substitution nicht notwendig ist. Dazu muss man aber erst
einmal klären wie man formale Potenzreihen multipliziert.

Die Multiplikation ergibt sich in völliger Analogie zur
Multiplikation von Polynomen und ist mit dieser kompatibel.
Diese Erweiterung der Multiplikation auf Potenzreihen wird
als Cauchy-Produkt bezeichnet.

Weitere Informationen findet man im englischen Wikipediaartikel
\textit{Formal power series}.


\subsection{Operatoren}
Es gibt einige Operatoren, die man auf Folgen bzw. formale
Potenzreihen wirken lassen kann. Mit diesen lassen sich höhere
Identitäten formulieren, die gewinnbringed tieferliegende
Zusammenhänge herausstellen. Anstelle von $a_k$ wollen wir
hier $f(k)$ schreiben.

Der Identitätsoperator ist definiert durch
\begin{equation}
(If)(k) := f(k).
\end{equation}
Der Translationsoperator ist definiert durch
\begin{equation}
(T_n f)(k) := f(k+n).
\end{equation}
Der Differenzenoperator ist definiert durch
\begin{equation}
(\Delta_n f)(k) := f(k+n)-f(k).
\end{equation}
Arithmetische Operationen mit Operatoren erklärt man Punktweise.
Sind z.B. $A,B$ zwei Operatoren, so definiert man die Summe $A+B$
durch
\begin{equation}
((A+B)f)(k):=(A f)(k)+(B f)(k).
\end{equation}
Somit ergibt sich die Operatorengleichung
\begin{equation}
\Delta_n = T_n-I.
\end{equation}
Ferner gilt nun
\begin{equation}
\Delta_n^2 = (T_n-I)^2
= T_n^2-T_nI-IT_n+I^2
= T_{2n}-2T_n+I.
\end{equation}
Für jeden Operator $A$ setzt man $A^0:=I$. So ist auch $T_n^0=T_0=I$.
Da Translationsoperator und Identitätsoperator kommutieren, kann
man den binomischen Lehrsatz verwenden. Es ergibt sich
\[\Delta_n^m = (T_n-I)^m
= \sum_{k=0}^m \binom{m}{k} T_n^k (-I)^{m-k}
= \sum_{k=0}^m (-1)^{m-k}\binom{m}{k} T_{kn}.\]
Den formalen Ableitungsoperator erklärt man durch
\begin{equation}
D\Big(\sum_{k=0}^\infty a_kX^k\Big)
:= \sum_{k=0}^\infty a_k kX^{k-1}.
\end{equation}


% \renewcommand{\refname}{Literatur}
\begin{thebibliography}{10}
\setlength{\itemsep}{0pt}
\bibitem{CM} Graham, Knuth, Patashnik: Concrete Mathematics.
\bibitem{EC} Richard P. Stanley: Enumerative Combinatorics. (online)
\bibitem{AK} Michael Stoll: Abzählende Kombinatorik. (online)
\bibitem{FPS} Wikipedia: Formal power series.
\end{thebibliography}

\end{document}


