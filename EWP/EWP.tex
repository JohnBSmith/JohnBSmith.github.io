\documentclass[a4paper,11pt,fleqn,twocolumn]{article}
\usepackage[utf8]{inputenc}
\usepackage{ngerman}
\usepackage{amsmath}
\usepackage{amssymb}
\usepackage{color}
\definecolor{c1}{RGB}{00,40,80}
\usepackage[colorlinks=true,linkcolor=c1]{hyperref}
\usepackage{geometry}
\geometry{a4paper,left=20mm,right=20mm,top=20mm,bottom=28mm}
\setlength{\columnsep}{6mm}
\begin{document}
%\setlength{\abovedisplayskip}{2.4mm}
%\setlength{\belowdisplayskip}{2.4mm}

\section*{Extremwertprobleme}

%\tableofcontents

\subsection*{Ein einfaches Beispiel}

Ein Hirte möchte mit einen Zaun eine Schafsweide umspannen.
Der Zaun hat einen festen Umfang \(u\). Gesucht ist das Rechteck,
für das der Flächeninhalt \(A\) maximal ist. Wie muss das Verhältnis
der Seitenlängen sein?

Die Formel für den Flächeninhalt ist \(A=ab\). Diese Formel stellt
die Hauptbedingung dar, da der Flächeninhalt ja maximiert werden soll.

Bezeichnet man die Seitenlängen mit \(a\) und \(b\), so ergibt sich
für den Umfang \(u=2a+2b\). Diese zusätzliche Bedingung heißt
Nebenbedingung.

Man setzt nun die umgeformte Nebenbedingung \(b=(u-2a)/2\) in die
Hauptbedingung ein und erhält
\[A = \frac{1}{2}a(u-2a) = -a^2+\frac{1}{2}u.\]
Gesucht ist das Maximum der Funktion \(A(a)\). Da diese Funktion
eine quadratische Funktion ist, ist das Maximum der Scheitelpunkt
der Parabel. Somit kann die Scheitelpunktformel benutzt werden.
Wir wollen stattdessen die allgemeinere Differentialrechnung zum
auffinden des Maximums benutzen. Die Ableitungen sind
\begin{gather*}
A'(a) = -2a+\frac{1}{2}u,\\
A''(a) = -2.
\end{gather*}
Das notwendige Kriterium ist \(A'(a)=0\). Damit erhält  man die
Gleichung
\[0=-2a+\frac{u}{2}.\]
Äquivalenzumformung bringt \(a=u/4\). Das hinreichende Kriterium
für ein Maximum ist \(A''(a)<0\). Man erhält die Ungleichung \(-2<0\),
die für alle \(a\) erfüllt ist. Einsetzen von \(a=u/4\) in die Nebenbedingung
und anschließendes Umformen bringt dann \(b=u/4\). Das Seitenverhältnis
ist
\[\frac{a}{b} = \frac{u/4}{u/4} = 1.\]
Bei der Fläche handelt es sich also um ein Quadrat.

Man kann die Rolle von Hauptbedingung und Nebenbedingung in diesem
Fall jedoch auch vertauschen. Anstelle den Flächeninhalt zu maximieren
kann man bei gegebenem Flächeninhalt ja auch den Umfang minimieren.
Man müsste auch bei diesem Ansatz auf das selbe Resultat kommen.

Sei jetzt \(u=2a+2b\) die Hauptbedingung. Umformen der Nebenbedingung
bringt \(b=A/a\). Einsetzen in die Hauptbedingung bringt
\[u=2a+\frac{2A}{a}.\]
Interessanterweise ist \(u(a)\) keine quadratische Funktion.
Die Verwendung der Scheitelpunktformel ist also nicht mehr möglich.
Die Ableitungen sind
\begin{gather*}
u'(a) = 2-\frac{2A}{a^2},\\
u''(a) = 2+\frac{4A}{a^3}.
\end{gather*}
Aus dem notwendigen Kriterium \(u'(a)=0\) ergib sich
\(A=a^2\). Das hinreichende Kriterium für ein Minimum ist
\(u''(a)>0\). Es ist immer erfüllt, wenn \(A>0\) und \(a>0\) sind.
Einsetzen von \(A=a^2\) in die Hauptbedingung \(A=ab\) liefert
\[a=b.\]
Als Optimum erhält man wieder ein Quadrat.


\subsection*{Kriterien}

Bei Extremwertaufgaben wird mit dem notwendigen Kriterium und
dem hinreichenden (aber nicht notwendigen) Kriterium gearbeitet.
Da ein solcher Sprachgebrauch etwas unpräzise ist, sollen diese
Kriterien noch einmal in mathematischer Sprache formuliert werden.

Bei den Kriterien werden alle Funktionen als differenzierbar
vorrausgesetzt.

Das notwendige Kriterium ist
\[f(x)\;\text{ist ein Extremum}\Rightarrow f'(x)=0.\]
Das hinreichende Kriterium ist
\begin{gather*}
f'(x)=0\wedge f''(x)<0\\
\Rightarrow f(x)\;\text{ist ein Maximum}
\end{gather*}
bzw.
\begin{gather*}
f'(x)=0\wedge f''(x)>0\\
\Rightarrow f(x)\;\text{ist ein Minimum}.
\end{gather*}
Aus den letzten beiden kann ein gemeinsames gebildet werden.
Es ergibt sich
\begin{gather*}
f'(x)=0\wedge f''(x)\ne 0\\
\Rightarrow f(x)\;\text{ist ein Extremum}.
\end{gather*}
Von allen Implikationen kann die Kontraposition gebildet werden.
Die Kontraposition ist die Tautologie
\[(A\Rightarrow B) \Leftrightarrow
(\overline B\Rightarrow\overline A).\]
Wir stellen uns nun die Frage, warum das notwendige Kriterium
{\glqq}notwendig{\grqq} genannt wird. Dazu muss zunächst die Kontraposition
gebildet werden. Das notwendige Kriterium wird mit der Kontraposition
äquivalent umgeformt zu
\[f'(x)\ne 0\Rightarrow f(x)\;\text{ist kein Extremwert}.\]
In dieser Form steht die Notwendigkeit direkt da.
Das hinreichende Kriterium ist nicht notwendig, da \(A\Rightarrow B\)
wahr sein kann, obwohl \(A\) falsch ist.

Der ungenaue Sprachegebrauch kann allgmein auf Implikationen
übertragen werden. Ist die Implikation
\(A\Rightarrow B\)
wahr, so ist \(B\) eine notwendige Bedingung für \(A\).
Gleichzeitig ist \(A\) eine hinreichende Bedingung für \(B\).

Ist die Äquivalenz \(A\Leftrightarrow B\) eine wahre Aussage, so ist
\(A\) eine hinreichende und notwendige Bedingung für \(B\). Da die
Äquivalenz symmetrisch ist, ist umgekehrt \(B\) eine hinreichende
und notwendige Bedingung für \(A\).

Wenn an einer Stelle \(x\) die Gleichung \(f'(x)=0\) erfüllt ist,
so wird diese Stelle als kritisch bezeichnet. Das notwendige
Kriterium kann damit ausschließlich mit Worten formuliert werden:
Extremstellen sind immer kritische Stellen. Jedoch muss eine kritische
Stelle keine Extremstelle sein. Das einfachste Gegenbeispiel ist
die Funktion \(f(x)=x^3\). Hier ist \(x=0\) eine kritsche Stelle,
jedoch keine Extremstelle.

\subsection*{Lagrangesche Multiplikatoren}

Auffällig beim Lösen der Extremwertprobleme ist, dass die
Nebenbedingung häufig umgestellt werden muss. Aber es kann ja sein,
dass diese Umformung unnötig schwierig ist. Um dieses Problem zu
umgehen führen wir eine allgemeinere Methode ein, die auf das
Umstellen verzichtet.

Man stellt sich die
Hauptbedigung
\[A=f(a,b)=ab\]
dabei zunächst als Funktion von den zwei
Variablen \(a\) und \(b\) vor. Die Nebenbedingung kann nun als
implizite Funktion
\[g(a,b)=2a+2b-u=0\]
interpretiert werden. Durch die
Nebenbedingung wird eine Kurve in der Ebene mit den Koordinaten
\((a,b)\) gewählt. Die Kurve wird auf die Fläche \(A(a,b)\)
hinaufprojeziert. Eben genau auf dieser Projektion ist der
Extrempunkt gesucht.

Dabei berühren sich die Isolinien, die durch \(g(a,b)=0\) und
\(f(a,b)=A_\mathrm{max}\) beschrieben werden. Da Gradiente immer
rechtwinklig auf den Isolinien stehen, sind dort die Gradiente
kollinear. Es ergibt sich die Bedingung
\[\nabla f = -\lambda\nabla g\]
wobei das Minuszeichen nur Konvention ist und keine mathematische
Bedeutung hat. Zunächst ist
\begin{gather*}
\nabla f = \frac{\partial f}{\partial a}e_1
+\frac{\partial f}{\partial b}e_2\\
= be_1+ae_2
\end{gather*}
und
\begin{gather*}
\nabla g = \frac{\partial g}{\partial a}e_1
+\frac{\partial g}{\partial b}e_2\\
= (2+2b)e_1+(2a+2)e_2.
\end{gather*}
Damit ergibt sich das Gleichungssystem
\begin{gather*}
b=-\lambda (2+2b),\\
a=-\lambda (2a+2).
\end{gather*}
Die drei gesuchten Variablen sind \(a,b\) und \(\lambda\).
Die dritte Gleichung des Gleichungssystems ist die Nebenbedingung
\(g(a,b)=0\).

Zunächst kann \(\lambda\) eliminiert werden, es ergibt sich
\[a=\frac{b}{2+2b}(2a+2).\]
Mit der Nebenbedingung erhält man
\[a=\frac{u/2-a}{u-2a+2}(2a+2).\]
Multiplikation mit dem Divisor bringt
\[a(u-2a+2)=(u/2-a)(2a+2).\]
Durch Ausmultiplizieren erhält man
\[au-2a^2+2a = au+u-2a^2-2a.\]
Kürzen und Umformen liefert dann \(u=4a\). Einsetzen
in die Nebenbedingung bringt wieder \(a=b\).

Die Methode der lagrangeschen Multiplikatoren kann also auch auf
ganz gewöhnliche Extremwertprobleme mit Nebenbedingungen angewendet
werden, auch wenn es ein wenig mit Kanonen auf Spatzen schießen ist.

Ausgelassen wurden an dieser Stelle hinreichende Kriterien,
das würde den Rahmen etwas sprengen.

\subsection*{Differentialformen}

Die Bedingung \(\nabla f=-\lambda\nabla g\) für die Kollinarität von \(\nabla f\) und \(\nabla g\) ist zu der Bedingung \(\nabla f\wedge\nabla g=0\) äquivalent. Wenn man eine Orthonormalbasis vorliegen hat, ist diese Bedingung auch gleichbedeutend mit \(\mathrm df\wedge\mathrm dg=0\).
Daraus ergibt sich ein alternativer Formalismus
zur Aufstellung notwendiger Bedingungen. Man berechnet zunächst
\[\mathrm df(a,b) = \mathrm d(ab) = b\mathrm da+a\mathrm db.\]
und
\[\mathrm dg(a,b) = \mathrm d(2a+2b-u) = 2\mathrm da+2\mathrm db.\]
Man multipliziert nun aus und beachtet dabei die Rechenregeln
für das äußere Produkt. Es ergibt sich
\[\mathrm df\wedge\mathrm dg
= (2b-2a)\mathrm da\wedge\mathrm db.\]
Somit erhält man
\[2b-2a=0\]
womit sich sofort \(a=b\) ergibt. Man sieht hier, dass dieser Rechenweg für manche Fälle besonders elegant ist.

\subsection*{Funktionen in zwei Variablen}

Bei Funktionen in zwei Variablen sind Extremwertprobleme ohne
Nebenbedingungen etwas komplizierter aber immer noch analog.
Man stelle sich den Graph
als einen Hügel vor. An der höchsten Stelle des Hügels kann man
zweie Schnitte machen, die jeweils parallel zu einer Koordinatenachse
sind. Die Kurven sind Funktionen einer Variablen (da man
die jeweils andere Variable konstant hält). Diese Funktionen lassen
sich ganz normal ableiten.

Sei \(f(x,y)\) die Funktion, welche den Hügel beschreibt und sei
\((x_0,y_0)\) die höchste Stelle. Dort ist dann
\[\frac{\partial f}{\partial x}(x_0,y_0)=0,\quad
\frac{\partial f}{\partial y}(x_0,y_0)=0.\]
Die beiden Gleichungen lassen sich mit dem Gradient zusammenfassen
zu
\[(\nabla f)(x_0,y_0)=0.\]
Das ist das notwendige Kriterium: Bei einer kritischen Stelle
muss der Tangentialraum waagerecht sein. Das ist der Fall, wenn
die Tangenten beider Schnittkurven waagerecht sind.

Äquivalent dazu ist, dass jede Richtungsableitung verschwindet.
Wenn \(v\) ein Richtungsvektor ist, so ist die Richtungsableitung
\(D_vf = \langle v,\nabla f\rangle\). Wenn aber \(\nabla f=0\),
dann ist auch
\[D_v f = \langle v,0\rangle = 0.\]
An einer kritischen Stelle muss sich kein Extrempunkt befinden,
es kann sich z.B. auch um einen Sattelpunkt handeln.

\subsection*{Aufgaben}

a) Eine Dose wird durch einen Zylinder modelliert.
Der Mantelflächeninhalt der Dose ist fest vorgegeben.
In die Dose soll ein möglichst großes Volumen passen.
Welches Volumen und welchen Radius hat der Zylinder, wenn seine
Höhe 20cm beträgt?

b) Formuliere die Aufgabe mit dem Schafszaun, wenn die Schafe von
einer Seite durch einen Fluss eingesperrt sind.

c) Formuliere die Aufgabe mit dem Schafszaun, wenn die Schafe
von zwei rechtwinklig aneinander liegenden Seiten durch einen
Teich eingesperrt sind.

d) Kanonische Schachtel-Aufgabe. Diese Aufgabe ist in den meisten
Lehrbüchern zu finden.

\subsection*{Literatur}

\begin{verbatim}
[1] Jürgen Köller: "Extremwertaufgaben
mit Nebenbedingung", http://www.mathema
tische basteleien.de/extremwert.htm

[2] "Extremwertaufgaben mit Nebenbeding
ung", http://www.mathematik.de/ger/fragen
antworten/erstehilfe/extremwertaufgaben
/extremwertaufgaben.html

[3] mathe online: "Anwendungen der
Differentialrechnung", http://www.mathe-
online.at/mathint/anwdiff/i.html

[4] Madipedia (2013): "Extremwertauf-
gaben", http://wiki.dzlm.de/wiki/Extrem
wertaufgaben

[5] Garrett P.: "Local minima and maxima
(First Derivative Test)." from Math
Insight, http://mathinsight.org/local_
minima_maxima_refresher

[6] Garrett P. "Minimization and maxi
mization refresher." from Math Insight,
http://mathinsight.org/minimization_
maximization_refresher

[7] "MAXIMUM/MINIMUM PROBLEMS", https://
www.math.ucdavis.edu/~kouba/CalcOneDIREC
TORY/maxmindirectory/MaxMin.html
\end{verbatim}

\end{document}


