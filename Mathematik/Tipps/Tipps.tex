\documentclass[a4paper,10pt,fleqn,twocolumn,twoside,dvipdfmx]{scrartcl}
\usepackage[utf8]{inputenc}
\usepackage[T1]{fontenc}
\usepackage[ngerman]{babel}
\usepackage{microtype}

\usepackage{libertine}
\usepackage[libertine,smallerops]{newtxmath}
\usepackage[scaled=0.8]{DejaVuSans}

\usepackage{amsmath}
\usepackage{amssymb}
\usepackage{booktabs}

\usepackage[automark]{scrlayer-scrpage}
\ohead{\pagemark}
\ihead{\headmark}
\chead{}
\cfoot{}
\ofoot{}

\usepackage{color}
\definecolor{c1}{RGB}{00,00,00}
\usepackage[colorlinks=true,linkcolor=c1]{hyperref}

\usepackage{geometry}
\geometry{a4paper,left=23mm,right=12mm,top=23mm,bottom=27mm}
\setlength{\columnsep}{5mm}
% \addtokomafont{disposition}{\rmfamily}

\newcommand{\strong}[1]{\textbf{#1}}
\newcommand{\bvec}[1]{\mathbf{#1}}
\newcommand{\ee}{\mathrm{e}}
\newcommand{\ui}{\mathrm{i}}
\newcommand{\id}{\operatorname{id}}
\newcommand{\N}{\mathbb N}
\newcommand{\Z}{\mathbb Z}
\newcommand{\Q}{\mathbb Q}
\newcommand{\R}{\mathbb R}
\newcommand{\C}{\mathbb C}

\begin{document}

\section*{\LARGE Tipps zur Mathematik}

\tableofcontents

\section{Differentialrechnung}

\subsection{Ermitteln der Ableitung}

Die Ableitung einer reellen Funktion $f$ ist definiert als%
\[f'(x) = \lim_{h\to 0}\frac{f(x+h)-f(x)}{h}.\]
Möchte man in Erfahrung bringen, ob sie unter Anwendung
der Ableitungsregeln richtig ermittelt wurde, kann man die Probe
machen, indem der Differenzenquotient%
\[D_h f(x) = \frac{f(x+h)-f(x)}{h}\]
an einer konkreten Stelle $x$ für ein kleines $h$ numerisch
berechnet und mit $f'(x)$ verglichen wird.

Da dies recht umständlich ist, ist es sinnvoll, diese Aufgabe einem
Funktionenplotter zu überlassen. Numerisch günstiger ist es, den
Differenzenquotient nicht naiv gemäß der Definition zu berechnen,
sondern als%
\[D_h f(x) = \frac{f(x+h)-f(x-h)}{2h}.\]
Mit dem Plotter kann man schließlich
\[10^n (D_h f(x) - f'(x)) \approx 0\]
für $n\in\{0,1,2,\ldots\}$ prüfen.

\section{Komplexe Zahlen}

Jeder komplexen Zahl ist gemäß
\[\Phi(a+b\ui) = \begin{pmatrix}a & -b\\ b & a\end{pmatrix}\]
oder äquivalent
\[\Phi(r\ee^{\ui\varphi}) = r\begin{pmatrix}
\cos\varphi & -\sin\varphi\\
\sin\varphi & \cos\varphi
\end{pmatrix}\]
genau eine Matrix zugeordnet. Die Abbildung $\Phi$ ist ein
Isomorphismus vom Körper der komplexen Zahlen in einen Körper,
der eine Unterstruktur des Matrizenrings darstellt.

Diese Beziehung schafft eine Verbindung zwischen dem
Rechnen mit komplexen Zahlen und Konzepten der linearen Algebra.

\section{Wahrscheinlichkeitsrechnung}

\subsection{Zufallsgrößen}

Was ist eine Zufallsgröße? Eine Zufallsgröße kann man sich zunächst
einfach als eine Abbildung $X\colon\Omega\to\Omega'$ zwischen
Ergebnismengen vorstellen. Sei bspw.
\[\Omega := \{(w_1,w_2)\mid w_1,w_2\in\{1,\ldots,6\}\}\]
die Menge der Ergebnisse des Wurfs zweier gewöhnlicher
Würfel. Das heißt, wurde mit dem ersten Würfel eine Drei
gewürfelt, und mit dem zweiten eine Fünf, ist das Ergebnis $(3, 5)$.
Jedes elementare Ereignis $\{(w_1,w_2)\}$ besitzt offenbar
dieselbe Wahrscheinlichkeit%
\[P(\{(w_1,w_2)\}) = \frac{1}{|\Omega|} = \frac{1}{36}.\]
Für ein beliebiges Ereignis $A$ gilt daher%
\[P(A) = \frac{|A|}{|\Omega|}.\]
Ein gutes Beispiel für eine Zufallsgröße ist die Summe der
Augenzahlen, also%
\[X((w_1,w_2)) := w_1 + w_2.\]
Des Pudels Kern liegt nun in der Beantwortung der Frage, wie
wahrscheinlich ein aus Funktionswerten von $X$ bestehendes Ereignis
ist.

Ein elementares Ereignis $\{x\}$ tritt doch genau dann ein,
wenn $x$ der Funktionswert $x=X(\omega)$ zum Ergebnis
$\omega$ ist. Wurde bspw. das Ergebnis $\omega=(3, 5)$
gewürfelt, ist das elementare Ereignis%
\[\{X(\omega)\} = \{3 + 5\} = \{8\}\]
eingetreten.

Das Ereignis $\{x\}$ tritt also genau dann ein, wenn das Ergebnis
$\omega$ im Urbild $X^{-1}(x)$ liegt, für das sich die Schreibweise%
\[X^{-1}(x)=\{X=x\}\]
eingebürgert hat. Demnach stimmt die Wahrscheinlichkeit von $\{x\}$
mit der des Urbildereignisses $\{X=x\}$ überein. Das heißt, es gilt%
\[P_X(\{x\}) = P(X^{-1}(x)) = P(\{X=x\}) = P(X=x).\]
Bspw. ist
\[\{X=8\} = \{(2,6), (6,2), (3,5), (5,3), (4,4)\}.\]
Damit ergibt sich
\[P(X=8) = \frac{|\{X=8\}|}{|\Omega|} = \frac{5}{36}\]
als Wahrscheinlichkeit der Augensumme acht.

% Dieses Heft steht unter der Lizenz\\
% Creative Commons CC0 1.0.

\end{document}

