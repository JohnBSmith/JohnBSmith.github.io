\documentclass[a4paper,11pt,fleqn,twoside]{scrartcl}
\usepackage[utf8]{inputenc}
\usepackage[T1]{fontenc}
\usepackage{lmodern}

\usepackage{ngerman}
\usepackage{amsmath}
\usepackage{amssymb}
\usepackage[all]{xy}
\usepackage{enumitem}

\usepackage{color}
\definecolor{c1}{RGB}{0,40,80}
\definecolor{gray1}{RGB}{80,80,80}
\usepackage[colorlinks=true,linkcolor=c1]{hyperref}
\usepackage{geometry}
\geometry{a4paper,left=38mm,right=22mm,top=24mm,bottom=40mm}
\setlength{\columnsep}{4mm}
\numberwithin{equation}{section}
\setcounter{tocdepth}{2}

\newcommand{\N}{\mathbb N}
\newcommand{\Z}{\mathbb Z}
\newcommand{\R}{\mathbb R}
\newcommand{\C}{\mathbb C}
\newcommand{\id}{\operatorname{id}}
\renewcommand{\Re}{\operatorname{Re}}
\renewcommand{\Im}{\operatorname{Im}}
\newcommand{\ui}{\mathrm i}
\newcommand{\ee}{\mathrm e}
\newcommand{\sur}{\operatorname{sur}}
\newcommand{\defiff}{\;:\Longleftrightarrow\;}
\newcommand{\strong}[1]{{\normalfont\sffamily\bfseries #1}}

\newenvironment{Definition}{\par\noindent\strong{Definition.}}{\par}
\newcommand{\emdef}[1]{\emph{#1}}
\newcommand{\pstrut}[1]{\rule{0pt}{\dimexpr 8pt+#1}}
\newcommand{\bitem}{\item[\scriptsize\color{gray1}$\blacksquare$]}

\begin{document}
\thispagestyle{empty}

\noindent
{\huge\sffamily\bfseries
Skript zum\hfill {\normalsize\rmfamily\mdseries Juni 2018}
\par\noindent
Mathematik-Studium
\par}

\tableofcontents

\newpage
\section*{Formeln}
\strong{Additionstheoreme}
\begin{align*}
\sin(x+y) &= \sin x\cos y + \cos x\sin y,\\
\cos(x+y) &= \cos x\cos y - \sin x\sin y
\end{align*}
\strong{Polarkoordinaten}
\begin{align*}
x &= r\cos\varphi,\\
y &= r\sin\varphi,\\
r &= \sqrt{x^2+y^2},\\
\varphi &= \operatorname{sgn}(y)\arccos\Big(\frac{x}{r}\Big)\qquad (\varphi\ne\pi)
\end{align*}
\strong{Inverse Matrix}
\[
\begin{bmatrix}
a & b\\
c & d
\end{bmatrix}^{-1}
= \frac{1}{ad-bc} \begin{bmatrix}
d & -b\\
-c & a
\end{bmatrix}
\]

\newpage

\section{Grundlagen}
\subsection{Formale Systeme}
\subsubsection{Formale Sprachen}
Zum Ausdrücken von mathematischen Sachverhalten wird eine Sprache
benötigt. Die Sprache soll jeder Repräsentation einen bestimmten
Sachverhalt zuordnen. Hierbei ist ein schrittweises Vorgehen
sinnvoll.

Zunächst wird eine Menge von \emdef{Symbolen} festgelegt, aus denen
\emdef{Symbolewörter} durch Aneinanderreihung der Symbole aufgebaut
werden dürfen. Die Menge der zulässigen Symbole wird als
\emdef{Alphabet} $\Sigma$ bezeichnet. Die Menge aller möglichen
Aneinanderreihungen aus Symbolewörtern aus dem Alphabet heißt
\emdef{kleensche Hülle} $\Sigma^*$. Dabei ist auch das
\emdef{leere Wort} $\varepsilon$ zugelassen.

Beispiel. $\Sigma:=\{0,1\}$. Es ergibt sich
\begin{equation}
\Sigma^* = \{\varepsilon,0,1,00,01,10,11,000,001,010,011,100,101,110,111,0000,\ldots\}.
\end{equation}
Jede Teilmenge $L\subseteq\Sigma^*$ heißt \emdef{formale Sprache}.

\subsection{Aussagenlogik}
\subsubsection{Syntax der Aussagenlogik}
Zunächst wird eine Menge von Variablen definiert:
\begin{equation}
V:=\{A_1,A_2,A_3,\ldots\}.
\end{equation}
Alternativ ist z.\,B. auch $A,B,C$ anstelle von $A_1,A_2,A_3$ möglich.
Es sollen bloß abzählbar unendlich viele Variablen sein, damit
für jede endliche Formel immer genügend Variablen zur Verfügung
stehen. Wie diese Variablen benannt sind, ist unwichtig.

Die Aussagenlogik besitzt das Alphabet
\begin{equation}
\Sigma := V\cup\{0,1,(,),\neg,\land,\lor,\rightarrow,\leftrightarrow\}.
\end{equation}
Nun definieren wir Produktionsregeln, da die Menge der wohlgeformten
Formeln unendlich groß ist und daher nicht als endliche Liste
aufgelistet werden kann. Die Produktionsregeln sind:
\begin{itemize}[noitemsep,topsep=2pt]
\setlength\itemsep{4pt}
\bitem 0 und 1 sind Formeln.
\bitem Jede Variable aus $V$ ist eine Formel.
\bitem Sind $\varphi$ und $\psi$ Formeln, dann sind es auch
   $(\neg \varphi)$, $(\varphi\land\psi)$, $(\varphi\lor\psi)$,
   $(\varphi\rightarrow\psi)$, $(\varphi\leftrightarrow\psi)$.
\end{itemize}
Auf die Klammerung kann verzichtet werden, die Produktionsregeln
werden hierdurch aber komplizierter. Eine für die Programmierung
praktische Angabe ist die \emdef{Backus-Naur-Form}. Die
Produktionsregeln für die Aussagenlogik mit optionalen Klammern sind:
\begin{itemize}[noitemsep,topsep=2pt]
\setlength\itemsep{4pt}
\bitem $0$, $1$ und jede Variable aus $V$ sind Atome.
\bitem R = Atom | $(\varphi)$.
\bitem N = R | $\neg$ R.
\bitem K = N | N $\land$ N.
\bitem D = K | K $\lor$ K.
\bitem I = D | D $\rightarrow$ D.
\bitem $\varphi$ = I | I $\leftrightarrow$ I.
\end{itemize}
\pagebreak[2]
Hier muss zwischen unterschiedlichen Formelvariablen unterschieden
werden, welche als \emdef{Nonterminalsymbole} bezeichnet werden.
Die Nonterimalsymbole sind $\varphi$, Atom, N, K, D, I.
Symbole aus dem Alphabet $\Sigma$ werden \emdef{Terminalsymbole}
genannt. Die Terminalsymbole können auch nach belieben in einfache
oder doppelte Anführungszeichen gesetzt werden, um sie besser
von den Nonterminalsymbolen zu unterscheiden.

Man wendet nun ausgehend vom \emdef{Startsymbol} $\varphi$
solange die Produktionsregeln an, bis sich eine Formel ergibt,
die nur noch aus Terminalsymbolen besteht.

Die Produktionsregeln können auch durch eine andere Art von
Relation beschrieben werden. Aber die Beschreibung mittels
Backus-Naur-Form lässt sich am einfachsten als
\emdef{rekursiver Abstieg} umsetzen, da sie diesem entspricht.

Die Technik des rekursiven Abstiegs ermöglicht es uns für
eine große Klasse von formalen Sprachen auf systematische
und übersichtliche Art eine syntaktische Analyse umzusetzen.
Die Aufgabe der syntaktischen Analyse ist zum einen die Überprüfung,
ob eine Formel wohlgeformt ist und zum anderen die Überführung
einer Formel in einen abstrakten Syntaxbaum (engl. AST für
abstract syntax tree).

Eine Spitzfindigkeit ist nun noch, dass die Konjunktion $A\land B$
und die Disjunktion $A\lor B$ das Assoziativgesetz erfüllen.
Die Produktionsregeln lassen sich so formulieren, dass ungeklammerte
Operationen links- oder rechtsassoziativ sind. Die formale Sprache
erfährt hierdurch keine Änderung, da für diese nur die
Wohlgeformtheit der Symbolketten wichtig ist. Jedoch gehören zu
den unterschiedlichen Formulierungen unterschiedliche abstrakte
Syntaxbäume. D.\,h., praktische formulierte Produktionsregeln können
auch die Gestalt der abstrakten Syntaxbäume wiedergeben.

An den Produktionsregeln werden folgende Modifikationen vorgenommen:
\begin{itemize}[noitemsep,topsep=4pt]
\setlength\itemsep{4pt}
\bitem K = N | K $\land$ N.
\bitem D = K | D $\lor$ K.
\end{itemize}

\subsubsection{Interpretation der Aussagenlogik}
\begin{Definition}
Sei $\varphi$ eine aussagenlogische Formel. Mit $V$ ist die
Menge aller Variablen gemeint, mit $\operatorname{V}(\varphi)$ die
Menge der Variablen, die in der Formel $\varphi$ vorkommen.
Jede Abbildung%
\begin{equation}
I\colon V\to\{0,1\}\quad\text{bzw.}\quad
I\colon\operatorname{V}(\varphi)\to\{0,1\}
\end{equation}
heißt \emdef{Interpretation} von $\varphi$.

Die Interpretation lässt sich zu $I\colon L\to\{0,1\}$
erweitern, wobei mit $L$ die Menge der aussagenlogischen Formeln
ist. Man definiert:
\begin{align}
I(0) &= 0,\\
I(1) &= 1,\\
I(\neg\varphi) &= (\neg I(\varphi)),\\
I(\varphi\land\psi) &= (I(\varphi)\land I(\psi)),\\
I(\varphi\lor\psi) &= (I(\varphi)\lor I(\psi)),\\
I(\varphi\rightarrow\psi) &= (I(\varphi)\rightarrow I(\psi)),\\
I(\varphi\leftrightarrow\psi) &= (I(\varphi)\leftrightarrow I(\psi)),
\end{align}
wobei die rechten Seiten gemäß der Wertetabelle zu berechnen sind.
\end{Definition}

\subsubsection*{Wertetabelle}
\begin{tabular}{c|c|c|c|c|c|c|c|c}
& & UND & ODER & impl. & gdw. & XOR & NAND & NOR\\
$A$ & $B$ & $A\land B$ & $A\lor B$
& $A\rightarrow B$ & $A\leftrightarrow B$
& $A\oplus B$ & $A\uparrow B$ & $A\downarrow B$\\
\hline\pstrut{4pt}%
0 & 0 & 0 & 0 & 1 & 1 & 0 & 1 & 1\\
0 & 1 & 0 & 1 & 1 & 0 & 1 & 1 & 0\\
1 & 0 & 0 & 1 & 0 & 0 & 1 & 1 & 0\\
1 & 1 & 1 & 1 & 1 & 1 & 0 & 0 & 0
\end{tabular}

\vspace{10pt}
\begin{Definition}
\emdef{Modellrelation}:
\begin{equation}
(I\models\varphi) \defiff (I(\varphi)=1).
\end{equation}
Man sagt, $I$ \emdef{modelliert} $\varphi$ oder
$I$ \emdef{ist ein Modell von} $\varphi$.
\end{Definition}

\subsection{Prädikatenlogik}
Sei $M$ eine endliche Menge und $(x_k)_{k=1}^n$ eine bijektive
Abzählung von $M$, die jedem $k$ ein Element von $M$ zuordnet.
Man definiert nun\index{Allquantor}
\begin{equation}
\forall x{\in}M\;[P(x)] :\Longleftrightarrow \bigwedge_{k=1}^n P(x_k) :\Longleftrightarrow x_1\land x_2\land\ldots\land x_n
\end{equation}
und\index{Existenzquantor}
\begin{equation}
\exists x{\in}M\;[P(x)] :\Longleftrightarrow \bigvee_{k=1}^n P(x_k) :\Longleftrightarrow x_1\lor x_2\lor\ldots\lor x_n.
\end{equation}
Hierbei ist zu beachten, dass es nicht auf die Reihenfolge der
Abzählung ankommt, da für UND und ODER das Kommutativgesetz gilt.

Formale Beweise lassen sich induktiv führen. Hierfür ist es notwendig,
die rechten Seiten wie beim Summenzeichen rekursiv zu formalisieren:
\begin{equation}
\bigwedge_{k=1}^1 P(x_k) :\Longleftrightarrow P(x_1),\qquad
\bigwedge_{k=1}^n P(x_k) :\Longleftrightarrow P(x_n)\land\bigwedge_{k=1}^{n-1} P(x_k).
\end{equation}
Nun lässt sich die Regel $C\lor (A\land B)\iff (C\lor A)\land(C\lor B)$
auf den Allquantor übertragen. Der Induktionsanfang ist
\begin{equation}
C\lor\bigwedge_{k=1}^1 P(x_k) \iff C\lor P(x_1) \iff \bigwedge_{k=1}^1 (C\lor P(x_1)).
\end{equation}
Der Induktionsschritt ist
\begin{align}
& C\lor\!\bigwedge_{k=1}^n P(x_k)
\iff C\lor\bigg(P(x_n)\land\bigwedge_{k=1}^{n-1} P(x_k)\bigg)\\
&\iff (C\lor P(x_n))\land\bigg(C\lor\!\bigwedge_{k=1}^{n-1} P(x_k)\bigg)\\
&\iff (C\lor P(x_n))\land\bigwedge_{k=1}^{n-1} (C\lor P(x_k))
\iff \bigwedge_{k=1}^{n} (C\lor P(x_k)).
\end{align}
\subsection{Mengenlehre}

\begin{Definition}
\emdef{Aufzählende Schreibweise}\index{aufzählende Schreibweise}:
\begin{equation}
a\in\{x_1\,\ldots,x_n\} \quad:\Longleftrightarrow\quad
a=x_1\lor\ldots\lor a=x_n.
\end{equation}
\end{Definition}

\noindent
Ein Prädikat $P(x)$ ist das selbe wie eine Aussageform, das ist eine
Funktion, die jedem $x$ aus einer beliebigen Definitionsmenge
einen Wert aus der Zielmenge $\{0,1\}$ zuordnet.

\begin{Definition}
\emph{Mengenbau-Schreibweise}\index{Mengenbau-Schreibweise}:
\begin{align}
& a\in\{x\mid P(x)\} :\Longleftrightarrow P(a)\qquad(\text{unbeschränkter Mengenbau}),\\
& \{x\in M\mid P(x)\} := \{x\mid x\in M\land P(x)\}\qquad(\text{beschränkter Mengenbau}),\\
& \{f(x)\mid P(x)\} := \{y\mid y=f(x)\land P(x)\}.
\end{align}
\end{Definition}
\begin{Definition}
\emph{Gleichheit} und \emdef{Teilmengenrelation}\index{Teilmenge}
von Mengen:
\begin{align}
A=B &\;:\Longleftrightarrow\;\forall x\,[x\in A\iff x\in B],\\
A\subseteq B &\;:\Longleftrightarrow\;\forall x\,[x\in A\implies x\in B].
\end{align}
\end{Definition}
\noindent
\begin{Definition}
Grundlegende Operationen:
\index{Vereinigung}\index{Schnittmenge}
\index{Differenzmenge}\index{symmetrische Differenz}
\begin{align}
A\cup B &:= \{x\mid x\in A\lor x\in B\},\qquad (\text{\emdef{Vereinigung}})\\
A\cap B &:= \{x\mid x\in A\land x\in B\},\qquad (\text{\emdef{Schnitt}})\\
A\setminus B &:= \{x\mid x\in A\land x\notin B\},\qquad (\text{\emdef{Differenz}})\\
A\triangle B &:= \{x\mid x\in A\oplus x\in B\}.\qquad (\text{\emdef{symmetrische Differenz}})
\end{align}
\end{Definition}

\subsection{Funktionen}
\begin{Definition}\index{Bildmenge}
\emph{Bildmenge} von $M\subseteq A$ unter $f\colon A\to B$:
\begin{equation}
f(M) := \{f(x)\mid x\in M\} := \{y\mid \exists x(x\in M\land y=f(x))\}.
\end{equation}
Die maximale Bildmenge $f(A)$ wird einfach \emdef{Bildmenge} von $f$ genannt.
\end{Definition}
Die Regel $f(M\cup N)=f(M)\cup f(N)$ ist allgemeingültig, denn:
\begin{align}
&y\in f(M\cup N) \iff \exists x[x\in M\cup N\land y=f(x)]\\
&\iff \exists x[(x\in M\lor x\in N)\land y=f(x)]\\
&\iff \exists x[x\in M\land y=f(x)\lor x\in N\land y=f(x)]\\
&\iff \exists x[x\in M\land y=f(x)]\lor\exists x[x\in N\land y=f(x)]\\
&\iff y\in f(M) \lor y\in f(N)
\iff y\in f(M)\cup f(N).
\end{align}
Ein wesentlicher Schritt ist dabei die Nutzung der tautologischen
Formel:
\begin{equation}
\exists x(P(x)\lor Q(x)) \iff \exists x(P(x))\lor\exists x(Q(x)).
\end{equation}
Für die Schnittmenge gilt aber nur noch
$f(M\cap N)\subseteq f(M)\cap f(N)$. Es gilt:
\begin{align}
& y\in f(M\cap N) \iff \exists x[x\in M\cap N\land y=f(x)]\\
& \iff \exists x[x\in M\land x\in N\land y=f(x)]\\
& \iff \exists x[x\in M\land y=f(x)\land x\in N\land y=f(x)]\\
& \implies \exists x[x\in M\land y=f(x)]\land\exists x[x\in N\land y=f(x)]\\
& \iff y\in f(M)\land y\in f(N)
\iff y\in f(M)\cap f(N).
\end{align}
Hierbei wurde $A\Leftrightarrow A\land A$ sowie das Kommutativ- und das
Assoziativgesetz zunutze gemacht. Die Äquivalenzkette ist durch
eine Implikation unterbrochen, weil nur gilt:
\begin{equation}
\exists x(P(x)\land Q(x)) \implies \exists x(P(x))\land\exists x(q(x)).
\end{equation}
Für die Umkehrung lassen sich leicht Gegenbeispiele finden.

Es gilt sogar die allgemeine Regel
\begin{equation}
f(\bigcup_{i\in I} M_i) = \bigcup_{i\in I} f(M_i).
\end{equation}
Man rechnet nach:
\begin{align}
& y\in f(\bigcup_{i\in I} M_i)
\iff \exists x[x\in\bigcup_{i\in I} M_i\land y=f(x)]\\
& \iff \exists x[\exists i(i\in I\land x\in M_i)\land y=f(x)]\\
& \iff \exists x\exists i(i\in I\land x\in M_i\land y=f(x))\\
& \iff \exists i\exists x(i\in I\land x\in M_i\land y=f(x))\\
& \iff \exists i[i\in I\land\exists x(x\in M_i\land y=f(x))]\\
& \iff \exists i[i\in I\land y\in f(M_i)]
\iff y\in \bigcup_{i\in I} f(M_i).
\end{align}
Analog zum Schnitt von zwei Mengen gilt nur:
\begin{equation}
f(\bigcap_{i\in I} M_i) \subseteq \bigcap_{i\in I} f(M_i).
\end{equation}
Für die Umkehrung lassen sich die selben Gegenbeispiele benutzen.

\begin{Definition}\index{Injektion}
Eine Funktion $f\colon A\to B$ heißt \emph{surjektiv}, wenn $f(A)=B$ ist.
\end{Definition}

\begin{Definition}\index{Surjektion}
Eine Funktion $f\colon A\to B$ heißt \emph{injektiv}, wenn
\begin{equation}
\forall x_1,x_2\in A\,[f(x_1)=f(x_2)\implies x_1=x_2]
\end{equation}
gilt.
\end{Definition}

\begin{Definition}\index{Verkettung}\index{Komposition}
Für zwei Funktionen $f\colon A\to B$ und $g\colon B\to C$ wird
die Funktion
\begin{equation}
(g\circ f)\colon A\to C,\quad (g\circ f)(x):=g(f(x))
\end{equation}
als \emph{Verkettung} oder \emph{Komposition} von $g$ und $f$
bezeichnet. Man spricht \emph{$g$ nach $f$}.
\end{Definition}

Zu beachten ist, dass die Zielmenge von $f$ mit der Definitionsmenge
von $g$ übereinstimmen muss. Für die Situation $f\colon A\to B$
und $g\colon D\to C$ mit $B\subseteq D$ kann $g$ natürlich auf
$B$ eingeschränkt werden. Die korrekte Notation lautet somit
$g|_B\circ f$.

Die Verkettung von zwei Surjektionen ist wieder surjektiv
und die Verkettung von zwei Injektionen ist wieder injektiv. Somit
ist die Verkettung von zwei Bijektionen wieder bijektiv.

Die Surjektivität können wir für $f\colon A\to B$ und $g\colon B\to C$
einfach nachrechnen:
\begin{align}
(g\circ f)(A) &= \{g(f(x))\mid x\in A\} = \{g(y)\mid y=f(x)\wedge x\in A\}\\
&= \{g(y)\mid y\in\{f(x)\mid x\in A\}\} = \{g(y)\mid y\in f(A)\}\\
&= \{g(y)\mid y\in B\} = g(B) = C.
\end{align}
Die Injektivität lässt sich auch einfach nachrechnen:
\begin{align}
&(g\circ f)(x_1) = (g\circ f)(x_2)
\iff g(f(x_1)) = g(f(x_2))\\
& \implies f(x_1) = f(x_2)
\implies x_1=x_2.
\end{align}

\section{Lineare Algebra}
\subsection{Vektorräume}
\strong{Untervektorraum-Kriterium.}\index{Untervektorraum-Kriterium}
Ist $V$ ein Vektorraum über
dem Körper $K$ und $U\subseteq V$, so ist $U$ auch ein Vektorraum,
falls
\begin{equation}
v,w\in U\implies v+w\in U
\quad\text{und}\quad
v\in U\implies \lambda v\in U
\end{equation}
für $\lambda\in K$ gilt. Die beiden Bedingungen sind die Bausteine von
Linearkombinationen. Das Kriterium sagt also aus, dass $U$
abgeschlossen bezüglich jeder Linearkombination sein muss.
Das heißt, dass sich das Ergebnis jeder Linearkombination wieder
in $U$ befinden muss.

Wenn $K$ ein Körper und
$M$ eine nichtleere Menge ist, so ist $K^M$ ein Vektorraum bezüglich
punktweiser Addition als Vektoraddition und punktweiser Multiplikation
mit einer Konstanten als Ska"-larmultiplikation. Ein Spezialfall davon
ist also $\R^\R$. Es ist nun so, dass $C^1(\R\to\R)$ ein Untervektorraum
von $C(\R\to\R)$ und dieser ein Untervektorraum von $\R^\R$ ist.

Betrachte dazu das Untervektorraum"=Kriterium. Wenn zwei Funktionen
stetig sind, so ist auch deren Summe und Produkt stetig. Die Skalare
können als konstante Funktionen betrachtet werden. Wenn zwei
Funktionen stetig differenzierbar sind, so auch deren Summe und
Produkt. Wenn eine Funktion stetig differenzierbar ist, so ist sie
erst recht stetig.

Bemerkung: Die Menge $M$ darf nicht leer sein, weil $K^M$ mindestens
die Nullabbildung enthalten muss, welche als Nullvektor verwendet
wird. Wäre $M$ leer, so wäre $|K^M|=|K|^0=0$, wobei $|K|\ne 0$
vorausgesetzt werden kann, weil ein Körper nicht leer sein darf.
Nun würde $K^M$ keinen einzigen Vektor enthalten, was nicht sein darf,
weil jeder Vektorraum mindestens den Nullvektor enthalten muss.

\subsection{Lineare Abhängigkeit}
\begin{Definition}\index{linear unabhängig}
Eine Menge $\{v_k\}_{k=1}^n$ von Vektoren $v_k\in V$ heißt
\emph{linear unabhängig}, wenn $\sum_{k=1}^n \lambda_k v_k=0$ nur
gilt, falls alle $\lambda_k=0$ sind. Andernfalls heißen die Vektoren
\emph{linear abhängig}.
\end{Definition}

Das Definition ist äquivalent dazu, dass sich keiner der Vektoren
als Linearkombination der anderen darstellen lässt.

Sei nun zunächst $n=2$. Sind $v_1,v_2$ linear abhängig, so gilt
\begin{equation}
\lambda_1 v_1+\lambda_2 v_2 = 0
\end{equation}
und nach Umformung gilt $v_2=\lambda v_1$ mit
$\lambda=-\lambda_1/\lambda_2$.

Da die Vektoren dem gleichen Vektorraum entstammen, kann ihr
äußeres Produkt gebildet werden. Es gilt nun
\begin{equation}
v_1\wedge v_2 = v_1\wedge (\lambda v_1) = \lambda v_1\wedge v_1 = 0.
\end{equation}
Tatsächlich ist die Menge $\{v_k\}_{k=1}^n$ genau dann linear
abhängig, wenn
\begin{equation}
v_1\wedge v_2\wedge\ldots\wedge v_n = 0
\end{equation}
gilt. Es genügt, dies am Beispiel $n=3$ zu verifizieren. Sei also
ohne Beschränkung der Allgemeinheit $v_3=\lambda_1 v_1+\lambda_2 v_2$.
Nun ist:
\begin{align}
&v_1\wedge v_2\wedge v_3
= v_1\wedge v_2\wedge(\lambda_1 v_1+\lambda_2 v_2)
= \lambda_1 v_1\wedge v_2\wedge v_1
+ \lambda_2 v_1\wedge v_2\wedge v_2\\
&= -\lambda_1 (\underbrace{v_1\wedge v_1}_{=0})\wedge v_2
+ \lambda_2 v_1\wedge (\underbrace{v_2\wedge v_2}_{=0})
= 0.
\end{align}
Das Prinzip ist klar: Jeder Vektor der Linearkombination steckt
im Faktor $A=v_1\wedge v_2$. Weil sowohl $A\wedge v_1=0$ als auch
$A\wedge v_2=0$ gilt, verschwindet auch die gesamte Linearkombination.
Die Umkehrung zu zeigen, ist etwas schwieriger.

\subsection{Lineare Abbildungen}

\begin{Definition}\index{lineare Abbildung}
Sind $V$ und $W$ Vektorräume über dem selben Körper $K$, so heißt
eine Abbildung $f\colon V\to W$ \emph{linear}, wenn sie additiv
und homogen ist. \emph{Additiv} bedeutet, dass für alle $v,w\in V$
gilt:
\begin{equation}
f(v+w) = f(v)+f(w)
\end{equation}
und \emph{homogen} bedeutet, dass für alle $\lambda\in K$ und $v\in V$
gilt:
\begin{equation}
f(\lambda v) = \lambda f(v).
\end{equation}
\end{Definition}

\strong{Beispiel.} $f(x):=2x$, $f\colon\R\to\R$. Bei $\R$ handelt
es sich natürlich um einen Vektorraum, denn $\R^n$ ist ja ein
Vektorraum und $\R^1=\R$.

\strong{Beispiel.} $D\colon C^1(\R\to\R)\to C(\R\to\R)$ mit
\begin{equation}
D(f):=x\mapsto\frac{\mathrm df(x)}{\mathrm dx}.
\end{equation}

\strong{Beispiel.} $I\colon C([a,b]\to\R)\to C^1([a,b]\to\R)$ mit
\begin{equation}
I(f):=x\mapsto\int_a^x f(x)\,\mathrm dx,
\end{equation}
wobei $a$ fest aber beliebig ist.

Die Menge der linearen Abbildungen:
\begin{equation}
\operatorname{Hom}_K(V,W)
:= \{f\mid f\colon V\to W, f\;\text{ist linear}\}.
\end{equation}
Bei $\operatorname{Hom}_K(V,W)$ handelt es sich selbst wieder
um einen Vektorraum über dem Körper $K$. Man definiert dafür
\begin{equation}
(f+g)(v) := f(v)+g(v),\qquad
(\lambda f)(v) := \lambda f(v).
\end{equation}
Die Beweisskizze hierfür:
\[\begin{split}
&(f+g)(v+w) = f(v+w)+g(v+w) = f(v)+f(w)+g(v)+g(w)\\
&= f(v)+g(v)+f(w)+g(w) = (f+g)(v)+(f+g)(w).
\end{split}\]
Die restlichen Axiome können als kleine Übung vom Leser
überprüft werden.

Bemerkung: Wegen $f(v)\in W$ muss $\lambda$ bei $\lambda f(v)$ aus
dem Körper von $W$ sein. Somit sind alle betrachteten Vektorräume
über dem Körper $K$.

\strong{Frage.} \emph{Ist eine additive Abbildung auch homogen?}

Man bemerkt zunächst
\begin{equation}
f(2v) = f(v+v) = f(v)+f(v) = 2f(v).
\end{equation}
Allgemein ergibt sich bei dieser Betrachtung $f(nv) = nf(v)$ für jede
natürliche Zahl $n\ge 1$. Dies soll zur Übung noch mal formal
verifiziert werden. Man definiert dazu für eine gegebene Folge
von Vektoren $(v_k)_{k=1}^n$ das Summenzeichen rekursiv:
\begin{equation}
\sum\nolimits_{k=1}^1 v_k := v_1,\qquad
\sum\nolimits_{k=1}^n v_k := \sum\nolimits_{k=1}^{n-1} v_k+v_n.
\end{equation}
Der Induktionsanfang ist einfach
\begin{equation}
f\Big(\sum\nolimits_{k=1}^1 v_k\Big) = f(v_k).
\end{equation}
Der Induktionsschritt ist
\begin{equation}
\begin{split}
&f\Big(\sum\nolimits_{k=1}^n v_k\Big)
= f\Big(\sum\nolimits_{k=1}^{n-1} v_k\Big)+f(v_n)\\
&=\sum\nolimits_{k=1}^{n-1} f(v_k)+f(v_n)
= \sum\nolimits_{k=1}^n f(v_k).
\end{split}
\end{equation}
Setzt man nun $v_k=v$ für alle $k$, so ergibt sich
\begin{equation}
f(nv) = f\bigg(\sum_{k=1}^n v\bigg) = \sum_{k=1}^n f(v) = nf(v).
\end{equation}
Weiterhin gilt
\begin{equation}
f(0v) = f(0) = f(0+0) = f(0) + f(0).
\end{equation}
Aus $f(0)=f(0)+f(0)$ folgt $f(0)=0=0f(v)$.

Beachte nun
\begin{equation}
0 = f(\underbrace{-v+v}_{=0}) = f(-v)+f(v).
\end{equation}
Daraus folgt $f(-v) = -f(v)$.
Nach den bisherigen Ausführungen ergibt sich $f(nv) = nf(v)$ für alle
$n\in\Z$.

Die Fragestellung lässt sich auch für rationale Zahlen bejahen.
Die grundlegende Feststellung dazu ist
\begin{equation}
f(v) = f(\tfrac{1}{2}v+\tfrac{1}{2}v) = f(\tfrac{1}{2}v) + f(\tfrac{1}{2}v)
= 2f(\tfrac{1}{2}v).
\end{equation}
Division durch zwei bringt $\tfrac{1}{2}f(v) = f(\tfrac{1}{2}v)$.
Allgemein gilt wieder $\frac{1}{n}f(v) = f(\tfrac{1}{n}v)$ für
$n\in\N$ mit $n\ge 1$. Ist $q$ nun eine rationale Zahl, so gibt
es die Darstellung $q=\frac{m}{n}$ mit $m\in\Z$ und $n\in\N, n\ge 1$.
Es gilt nun
\begin{equation}
f(qv) = f(m\cdot\tfrac{1}{n}\cdot v) = m\cdot f(\tfrac{1}{n}\cdot v)
= m\cdot\tfrac{1}{n}\cdot f(v) = qf(v).
\end{equation}
Was ist nun mit reellen Zahlen? Sei dazu $s_n:=\sum_{k=0}^n q_k$
eine Reihe von rationalen Zahlen $q_k$, welche gegen eine reelle
Zahl $r$ konvergiert. Am besten $q_k = \frac{d_k}{10^k}$ mit
$q_0\in\Z$ und $q_{k\ne 0}\in\{0\ldots 9\}$.

Nun gilt
\begin{equation}
\begin{split}
&f(rv) = f\bigg(\bigg(\lim_{n\to\infty}\sum_{k=0}^n q_k\bigg)v\bigg)
\stackrel{?}= f\bigg(\lim_{n\to\infty}\bigg(\sum_{k=0}^n q_k\bigg)v\bigg)
= f\bigg(\lim_{n\to\infty}\sum_{k=0}^n q_k v\bigg)\\
&\stackrel{??}= \lim_{n\to\infty} f\bigg(\sum_{k=0}^n q_k v\bigg)
= \lim_{n\to\infty}\sum_{k=0}^n q_k f(v)
= \lim_{n\to\infty}\bigg(\sum_{k=0}^n q_k\bigg) f(v)\\
&\stackrel{?}= \bigg(\lim_{n\to\infty}\sum_{k=0}^n q_k\bigg) f(v)
= r f(v).
\end{split}
\end{equation}
Um die Fragezeichen zu klären, nimmt man nun an, dass $V$ und $W$
mit einer Norm ausgestattet und somit metrische Räume sind.
Der folgende elementare Satz ist jetzt aufschlussgebend.

\textbf{Satz.} Eine Abbildung $f\colon X\to Y$ zwischen metrischen
Räumen $(X,d_x)$ und $(Y,d_y)$ ist genau dann stetig, wenn für
jede konvergente Folge $(x_n)$ mit $x_n\in X$ die Eigenschaft
\begin{equation}
f\Big(\lim_{n\to\infty} x_n\Big) = \lim_{n\to\infty} f(x_n)
\end{equation}
gilt.

\subsection{Darstellungsmatrizen}\index{Darstellungsmatrix}
Wird in $V$ die Basis $B$ ausgewählt und in $W$ die Basis $B'$,
so ist dadurch der eindeutige Isomorphismus
\begin{equation}
M_{B'}^B\colon \operatorname{Hom}_K(V,W)\to K^{\dim W\times\dim V}
\end{equation}
bestimmt. Man nennt $M_{B'}^B(f)$ die \emph{Darstellungsmatrix}
von $f$. Bei $M_{B'}^B$ handelt es sich als Isomorphismus zwischen
Vektorräumen um eine bijektive lineare Abbildung.

Das Problem mit dem Isomorphismus $M_{B'}^B$ ist, dass dieser nicht
kanonisch ist, weil er von den Basen $B$ und $B'$ abhängt, die
willkürlich gewählt werden können.
Nun gibt es aber den Koordinatenraum $K^n$, welcher die Standardbasis
als kanonische Basis besitzt.

Die Tatsache, dass die Verkettung zweier linearer Abbildungen
selbst wieder linear ist, kann dazu verwendet werden, eine
lineare Abbildung in eine Verkettung zu zerlegen. Das veranlasst
uns dazu, bei der Bildung der Darstellungsmatrix einen
Zwischenschritt einzufügen. Somit ist
\begin{equation}\label{eq:Zwischenschritt}
\operatorname{Hom}_K(V,W)
\xrightarrow{\displaystyle\quad F_{B'}^B\quad}\operatorname{Hom}_K (K^{\dim V},K^{\dim W})
\xrightarrow{\displaystyle\quad\varphi\quad}K^{\dim W\times\dim V},
\end{equation}
sodass $M_{B'}^B = \varphi\circ F_{B'}^B$ gilt. Bei $\varphi$
handelt es sich nun um einen kanonischen Isomorphismus, sodass
eine lineare Abbildung zwischen Koordinatenräumen mit ihrer
Darstellungsmatrix identifiziert werden kann. Man kann also
bedenkenlos
\begin{equation}\label{eq:Matrizenraum-Identifikation}
\operatorname{Hom}_K (K^n,K^m) = K^{m\times n}
\end{equation}
setzen.

Kleine Bauchschmerzen bereitet \eqref{eq:Matrizenraum-Identifikation}
aber doch, denn es können ja trotzdem andere Basen als die
Standardbasen gewählt werden. Es genügt hierfür aber,
\eqref{eq:Zwischenschritt} zu spezialisieren. Man erhält
\begin{equation}
\operatorname{Hom}_K(K^n,K^m)
\xrightarrow{\displaystyle\quad F_{B'}^B\quad}\operatorname{Hom}_K (K^n,K^m)
\xrightarrow{\displaystyle\quad\varphi\quad}K^{m\times n}.
\end{equation}
Bei $F_{B'}^B$ handelt es sich jetzt um einen Automorphismus.
Bei einem Automorphismus muss es sich aber zwangsweise um einen
Basiswechsel handeln. Hierbei werden $B$ und $B'$ zu einer
gemeinsamen Basis $(B,B')$ zusammengefasst. Von dieser gemeinsamen
Basis wird nun in die gemeinsame Standardbasis $(E,E')$
gewechselt.

Zuweilen findet man oft Angaben wie
\begin{equation}
f\begin{bmatrix}x\\ y\end{bmatrix}
:=\begin{bmatrix}2x+4y\\ 4x\end{bmatrix}
\end{equation}
vor. Solche Angaben sind eigentlich recht sinnfrei, weil $f$ wegen
\eqref{eq:Matrizenraum-Identifikation}
auch gleich als Matrix beschrieben werden kann:
\begin{equation}
f\begin{bmatrix}x\\ y\end{bmatrix}
:=\begin{bmatrix}2x+4y\\ 4x\end{bmatrix}
= \begin{bmatrix}2 & 4\\ 4 & 0\end{bmatrix}
\begin{bmatrix}x\\ y\end{bmatrix}.
\end{equation}
Die Tupelschreibweise und die Koordinaten $(x,y)$ suggerieren, dass
wir es hier mit dem Koordinatenraum und seiner Standardbasis
zu tun haben. Man wollte eine lineare Abbildung direkt, aber
ohne Festlegung auf Basen beschreiben. Das ist jedoch gar nicht
möglich, weil die direkte Beschreibung immer die rechte Seite
von \eqref{eq:Matrizenraum-Identifikation} bedingt.

Die Festlegung auf andere Basen bedeutet nun, dass die
Darstellungsmatrix transformiert werden muss. Viele Aufgaben der
linearen Algebra mit konkreten linearen Abbildungen können somit
vollständig durch Matrizenrechnung gelöst werden.

\subsection{Gruppen}\label{sub:LA-Gruppen}
Betrachte die Menge der Selbstabbildungen $X^X:=\{f\mid f\colon X\to X\}$.
Die Menge der bijektiven Selbstabbildungen wird
\emph{symmetrische Gruppe}\index{symmetrische Gruppe}
$S(X)$ genannt und bildet bezüglich der
Verkettung von Abbildungen die Gruppe $(S(X),\circ)$.

\strong{(Axiom E)} Abgeschlossenheit gilt, weil die Verkettung zweier
bijektiver Abbildungen auch wieder bijektiv ist und $S(X)$ alle
bijektiven Selbstabbildungen umfassen soll.

\strong{(Axiom A)} Das Assoziativgesetz
$(h\circ g)\circ f = h\circ (g\circ f)$
gilt für \emph{alle} Abbildungen.

\strong{(Axiom N)} Auf \emph{jeder} Menge $X$ gibt es die identische
Abbildung $\id_X$, so dass $f\circ\id_X = f$ und $\id_X\circ f=f$.

\strong{(Axiom I)} Da die Abbildungen als bijektiv vorausgesetzt sind,
gibt es zu jeder Abbildung $f$ ein $g$, so dass
$f\circ g = \id_X$ und $g\circ f=\id_X$.

Betrachte nun $K^{n\times n}$, den Matrizenraum der quadratischen
Matrizen mit Einträgen aus dem Körper $K$. Man definiert nun
die Menge der regulären quadratischen Matrizen:
\begin{equation}
\mathrm{GL}(n,K) := \{A\mid A\in K^{n\times n}\wedge\det(A)\ne 0\}.
\end{equation}
Bei $\mathrm{GL}(n,K)$ handelt es sich auch um eine Gruppe. Das
neutrale Element ist die Einheitsmatrix $E_n$, die durch
\begin{equation}
E_2 := \begin{bmatrix}
1 & 0\\
0 & 1
\end{bmatrix},\quad
E_3 := \begin{bmatrix}
1 & 0 & 0\\
0 & 1 & 0\\
0 & 0 & 1
\end{bmatrix},\quad\text{u.s.w.}
\end{equation}
definiert ist.

Nun lässt sich aber ein Vektorraum der Dimension $n$ über dem
Körper $K$ betrachten. Nach Wahl einer Basis $B$ gibt es nun
\begin{equation}
M_B^B\colon \operatorname{Hom}_K(V,V)\to K^{\dim V\times\dim V}.
\end{equation}
Eine linearen Selbstabbildung wird als
\emph{Endomorphismus}\index{Endomorphismus}
bezeichnet. Eine bijektive lineare Selbstabbildung wird
\emph{Automorphismus}\index{Automorphismus}
genannt.

Wenn $M_B^B$ auch ein Gruppenisomorphismus bezüglich der
multiplikativen Struktur ist, so ergibt sich
\begin{equation}
M_B^B\colon \operatorname{Aut}_K(V)\to\operatorname{GL}(\dim V,K).
\end{equation}
Dann ist $\operatorname{Aut}_K(V)$ aber eine Gruppe, und
zwar eine Untergruppe von $S(V)$. Man spricht bei
$\operatorname{Aut}_K(V)$ von der \emph{Automorphismengruppe}.
Zur Übersicht ergibt sich folgendes Diagramm:
\begin{equation}
\begin{matrix}
\operatorname{Aut}_K(V) & \subseteq & \operatorname{Hom}_K(V,V)\\
M_B^B\bigg\downarrow\qquad & & M_B^B\bigg\downarrow\qquad\\
\mathrm{GL}(\dim V,K) & \subseteq & K^{\dim V\times\dim V}
\end{matrix}
\end{equation}

\subsection{Quadratische Matrizen}
Die Multiplikation von Matrizen kann nach dem \emph{falkschen Schema}
ausgeführt werden:\\

\qquad\begin{tabular}{rr|ll}
& & $b_{11}$ & $b_{12}$\\
& & $b_{21}$ & $b_{22}$\\
\hline
$a_{11}$ & $a_{12}$ & $\sum$ & $\sum$\\
$a_{21}$ & $a_{22}$ & $\sum$ & $\sum$
\end{tabular}\\
\phantom{x}

\noindent
Auch die Multiplikation von mehr als zwei Matrizen kann nach diesem
Schema notiert werden. Das Produkt $Z=ABCD$ wird in die
Multiplikationen $X=AB$, $Y=XC$ und $Z=YD$ aufgetrennt, womit
sich eine horizontales Schema ergibt. Alternativ setzt man $X=CD$,
$Y=BX$ und $Z=AY$, womit sich eine vertikales Schema ergibt.

Das sind:

\begin{tabular}{@{\hspace{2em}}l@{\hspace{4em}}l}
\begin{tabular}{l|l|l|l}
  & $B$ & $C$ & $D$\\
\hline
$A$ & $X$ & $Y$ & $Z$
\end{tabular}
&
\begin{tabular}{l|l}
  & D\\
\hline
C & X\\
\hline
B & Y\\
\hline
A & Z
\end{tabular}
\end{tabular}

\strong{Problem.} \emph{Zu einer gegebenen regulären
Matrix $A$ soll die Matrix
$B$ gefunden werden, sodass $AB=E_2$ gilt.}

Es gibt drei Ansätze zur Lösung.

Der \strong{erste Ansatz}
ist die Trennung der Matrizengleichung in vier Gleichungen, wodurch
sich zwei LGS ergeben. Zunächst lautet die Matrizengleichung:%
\begin{equation}
\begin{bmatrix}
a_{11} & a_{12}\\
a_{21} & a_{22}
\end{bmatrix}
\begin{bmatrix}
b_{11} & b_{12}\\
b_{21} & b_{22}
\end{bmatrix}
= \begin{bmatrix}
a_{11}b_{11}+a_{12}b_{21} & a_{11}b_{12}+a_{12}b_{22}\\
a_{21}b_{11}+a_{22}b_{21} & a_{21}b_{12}+a_{22}b_{22}
\end{bmatrix}
= \begin{bmatrix}
1 & 0\\
0 & 1
\end{bmatrix}.
\end{equation}
Hierdurch ergeben sich die beiden LGS:
\begin{equation}
\bigg|\,\begin{matrix}
a_{11}b_{11}+a_{12}b_{21} &=& 1\\
a_{21}b_{11}+a_{22}b_{21} &=& 0
\end{matrix}\,\bigg|,\qquad
\bigg|\,\begin{matrix}
a_{11}b_{12}+a_{12}b_{22} &=& 0\\
a_{21}b_{12}+a_{22}b_{22} &=& 1
\end{matrix}\,\bigg|.
\end{equation}

Der \strong{zweite Ansatz}
ist die Verwendung des Gauß"=Jordan"=Verfahrens
um $[E_2|B]$ aus $[A|E_2]$ zu gewinnen.

Der \strong{dritte Ansatz} ist die Anwendung der Lösungsformel
\begin{equation}
\begin{bmatrix}
a_{11} & a_{12}\\
a_{21} & a_{22}
\end{bmatrix}^{-1}
= \frac{1}{a_{11}a_{22}-a_{12}a_{21}}\begin{bmatrix}
a_{22} & -a_{12}\\
-a_{21} & a_{11}
\end{bmatrix}.
\end{equation}
Bei Matrizen vom Typ $3\times 3$ oder $4\times 4$ ist nur noch
das Gauß"=Jordan"=Verfahren sinnvoll. Ab $5\times 5$ oder bei
$\C^{n\times n}$ anstelle von $\R^{n\times n}$ wird man Matlab/Octave
verwenden.

\section{Algebra}
\subsection{Gruppen}
Man betrachte die Menge $M=\{A,B,C\}$, wobei $\triangle ABC$ ein
gleichseitiges Dreieck sein soll. Außerdem definiert man
\begin{equation}
S(M) := \{f\mid f\colon M\to M,\;\text{$f$ ist bijektiv}\}.
\end{equation}
In Abschnitt \ref{sub:LA-Gruppen} wurde schon festgestellt,
dass $(S(M),\circ)$ eine Gruppe ist. Außerdem muss $S(M)$ eine
Anzahl von $3!=6$ Elementen besitzen, wie aus der elementaren
Kombinatorik bekannt ist. Nun definieren wir
\begin{equation}
d_1 := (A \mapsto B \mapsto C\mapsto A),\qquad d_2:=(A\mapsto C\mapsto B\mapsto A).
\end{equation}
Wir finden nun folgende Gesetzmäßigkeiten heraus:
\begin{equation}
d_1^2 = d_1\circ d_1 = d_2, \qquad d_1^3 = d_2\circ d_1 = \id, \qquad d_1^{-1}=d_2,\qquad d_2^{-1}=d_1.
\end{equation}
Weiteres Herumspielen führt uns aber nicht zu den drei anderen
Elementen in $S(M)$, egal wie sehr wir uns bemühen. Trotzdem
haben wir zu jedem Element ein inverses, wobei man beachtet,
dass $\id$ zu sich selbst invers ist. Auch $\id$ ist enthalten,
und die restlichen Gruppenaxiome gelten auch. Somit handelt es sich
bei
\begin{equation}\label{eq:Drehungen}
H := \{\id,d_1,d_2\}
\end{equation}
um eine Untergruppe von $S(X)$.

\begin{Definition}
Ist $(G,*)$ eine Gruppe und $H\subseteq G$, so heißt $H$
\emph{Untergruppe} von $G$, wenn $(H,*)$ auch wieder eine Gruppe
bildet. Man schreibt dann $H\le G$.
\end{Definition}
Drei Elemente haben wir aus dem Teich $S(M)$ gefischt, drei
stecken noch im Schlamm. Aber der Teich ist nicht besonders tief,
mehr eine Pfütze als ein Teich. Was aus dem Wasser gezogen wird,
möchten wir aber sogleich beschauen.

\strong{Korollar aus den Gruppenaxiomen.} \emph{Jede Gruppe
besitzt genau ein neutrales Element. Zu jedem Element gibt
es genau ein inverses.}

Nähere Überlegung führt zu der Erkenntnis, dass es ausgeschlossen
ist dass zwei unterschiedliche Elemente das selbe inverse besitzen.
Wenn $b$ nämlich
zu $a_1$ und zu $a_2$ invers ist, so sind $a_1$ und $a_2$ auch invers
zu $b$. Weil das inverse Element zu $b$ aber eindeutig bestimmt sein
muss, gilt zwingend $a_1=a_2$.

Jedes Element besitzt also seinen inversen Partner,
wobei ein Element aber auch zu sich selbst invers sein darf.

Die restlichen Elemente von $S(M)$ müssen sich zu solchen
Partnerschaften gruppieren lassen. Dann müssen zwei Elemente
ein Paar bilden und eines ist zu sich selbst invers? Falsch.
Alle drei sind zu sich selbst invers. Das sind
\begin{align}\label{eq:Spiegelung1}
s_1&:= (A)(BC) = (A\mapsto A)(B\mapsto C\mapsto B),\\
s_2&:= (B)(AC) = (B\mapsto B)(A\mapsto C\mapsto A),\\
s_3&:= (C)(AB) = (C\mapsto C)(A\mapsto B\mapsto A).
\end{align}
Alle Operation lassen sich bei $\triangle ABC$ deuten.
Es handelt sich um Kongruenzabbildungen, welche das Dreieck
auf sich selbst abbilden, so dass man keinen Unterschied
bemerkt, außer dass zwei oder alle drei Ecken vertauscht sind.
Das Element $d_1$ ist eine Drehung um $120^\circ$ und $d_2$ ist
eine Drehung um $240^\circ = -120^\circ$. Das Element $\id$
ist eine Drehung um $0^\circ = 360^\circ$. Bei $s_1,s_2,s_3$
handelt es sich um Spiegelungen mit den Winkelhalbierenden
als Spiegelaxen.

Man könnte nun auf die Idee kommen, dass auch bei einem Quadrat
die Symmetrien genau die Elemente der symmetrischen Gruppe sind.
Die Gruppe hat aber $4!=24$ Elemente, eine solche Anzahl von
Symmetrien scheint fragwürdig. Ein Gegenbeispiel ist sofort
ersichtlich:
\begin{equation}
f := (A)(B)(CD) = (A\mapsto A)(B\mapsto B)(C\mapsto D\mapsto C).
\end{equation}
Da müsste das Quadrat auseinandergeschnitten, eines der Stücke
umgedreht, und dann beide Stücke wieder zusammengeklebt werden.

Tatsächlich bilden die Symmetrien des Quadrates aber eine Gruppe,
die aus diesem Grund als \emph{Symmetriegruppe} bezeichnet wird.
Die Übereinstimmung der Symmetriegruppe des Dreiecks mit der
symmetrischen Gruppe der Ecken stellt sich als Koinzidenz heraus.

\subsubsection{Konjugation}
Sei $G$ eine Gruppe und $g,h\in G$. Unter der \emdef{Konjugation}
von $g$ mit $h$ wird die Operation $hgh^{-1}$ verstanden. Beim
Verketten von Abbildungen ist die Reihenfolge umgekehrt, das ist
$h^{-1}\circ g\circ h$. Konjugation ist bei nichtabelschen Gruppen
von besonderer Bedeutung. Für abelsche Gruppen ist die Konjugation
trivial, da sich $h$ und $h^{-1}$ kürzen lassen.

Als Beispielaufgabe betrachten wird die Gruppe aller Drehmatrizen
im Raum, die $\operatorname{SO}(3)$. Während die $\operatorname{SO}(2)$
abelsch ist, ist das bei der $\operatorname{SO}(3)$ nicht mehr
der Fall. Man betrachte nun den $\R^3$, wobei die $x$-Achse nach
rechts zeigt, die $y$-Achse in den Raum hinein und die $z$-Achse
nach oben. Die Drehmatrizen um die Koordinatenachsen sind klar:
\begin{equation}
R_{xy} = \begin{bmatrix}
\cos\varphi & -\sin\varphi & 0\\
\sin\varphi & \cos\varphi & 0\\
0 & 0 & 1
\end{bmatrix},
R_{xz} = \begin{bmatrix}
\cos\varphi & 0 & -\sin\varphi\\
0 & 1 & 0\\
\sin\varphi & 0 & \cos\varphi
\end{bmatrix},
R_{yz} = \begin{bmatrix}
1 & 0 & 0\\
0 & \cos\varphi & -\sin\varphi\\
0 & \sin\varphi & \cos\varphi
\end{bmatrix}.
\end{equation}
Hierbei ist $R_{xy}(\varphi)$ die Rotation um die $z$-Achse,
$R_{xz}(\varphi)$ die Rotation um die $y$-Achse und
$R_{yz}(\varphi)$ die Rotation um die $x$-Achse.

Die Aufgabe besteht nun darin, um die Winkelhalbierende $x=y$ zu
drehen. Man möchte also in der Ebene $\{(x,y,z)\mid x=-y\}$ drehen.
Dazu wird der Punkt auf der Achse $x=-y$ zunächst zur $x$-Achse
gedreht. Nun lässt sich die Matrix $R_{xz}(\theta)$ anwenden.
Anschließend wird die Drehung um die $z$-Achse rückgängig gemacht.
Es ergibt sich
\begin{equation}
R = R_{xy}(45^\circ)^{-1}\,R_{xz}(\theta)\,R_{xy}(45^\circ).
\end{equation}
Bei Rotationsmatrizen gilt immer $R^{-1}(\varphi)=R(-\varphi)$.
Wenn man nicht Punkte tranformieren möchte, sondern das
Koordinatensystem\footnote{Bzw. den Viewport, das ist die Aussicht
auf das Koordinatensystem.}, dann müssen die Winkel durch ihre
Negationen substituiert werden.

Das falksche Schema zur Multiplikation
$R_{xy}(-\varphi)\,R_{xz}(\theta)\,R_{xy}(\varphi)$ ist:

\begin{tabular}{l|c}
& $\begin{bmatrix}
\cos\varphi\qquad & \qquad -\sin\varphi\qquad & \qquad 0\\
\sin\varphi\qquad & \cos\varphi & \qquad 0\\
0\qquad & 0 & \qquad 1
\end{bmatrix}$\\
\hline
$\begin{bmatrix}
\cos\theta & 0 & -\sin\theta\\
0 & 1 & 0\\
\sin\theta & 0 & \cos\theta
\end{bmatrix}$ &
$\begin{bmatrix}
  \cos\theta\cos\varphi\quad
& \quad -\cos\theta\sin\varphi\quad
& \quad -\sin\theta\\
  \sin\varphi\quad
& \cos\varphi
& \quad 0\\
  \sin\theta\cos\varphi\quad
& -\sin\theta\sin\varphi
& \quad\cos\theta
\end{bmatrix}$\\
\hline
$\begin{bmatrix}
\cos\varphi & \sin\varphi & 0\\
-\sin\varphi & \cos\varphi & 0\\
0 & 0 & 1
\end{bmatrix}$
&$\begin{bmatrix}
  \cos\theta\cos^2\varphi+\sin^2\varphi
& (1-\cos\theta)\cos\varphi\sin\varphi
& -\sin\theta\cos\varphi\\
  (1-\cos\theta)\cos\varphi\sin\varphi
& \cos\theta\sin^2\varphi+\cos^2\varphi
& \sin\theta\sin\varphi\\
  \sin\theta\cos\varphi
& -\sin\theta\sin\varphi
& \cos\theta
\end{bmatrix}$
\end{tabular}

\pagebreak[4]
\subsection{Zyklische Gruppen}
Die Gruppe \eqref{eq:Drehungen} ist von besonderer Gestalt.
Sie lässt sich von nur einem einzigen Element erzeugen.

\begin{Definition}\index{zyklische Gruppe}
Eine Gruppe heißt \emph{zyklisch}, wenn sie die Form
\begin{equation}
\langle g\rangle := \{g^n\mid n\in\Z\}
\end{equation}
besitzt. Man nennt $g$ den \emph{Erzeuger}\index{Erzeuger}
und die Gruppe $\langle g\rangle$ das \emph{Erzeugnis} von $g$.
\end{Definition}
Es ergibt sich nun:
\begin{equation}
\begin{tabular}{c|c|c|c|c|c|c|c|c|c|c|c}
$n$ & $-5$ & $-4$ & $-3$ & $-2$ & $-1$
& $0$ & $1$ & $2$ & $3$ & $4$ & $5$\\
\hline
$d_1^n$ & $d_1$ & $d_2$ & $\id$ & $d_1$ & $d_2$
& $\id$ & $d_1$ & $d_2$ & $\id$ & $d_1$ & $d_2$
\end{tabular}
\end{equation}
Die Elemente wiederholen sich immer. Es gibt einen Zyklus
der Länge drei. Es ist
\begin{equation}
\langle d_1\rangle = \{\id,d_1,d_2\},\qquad\langle d_2\rangle = \{\id,d_1,d_2\}.
\end{equation}
Die Regel $\langle g^{-1}\rangle = \langle g\rangle$ gilt natürlich
immer. Für viele Gruppenelemente ist aber
$\langle g^2\rangle\ne \langle g\rangle$. Als Beispiel lässt sich die
Spiegelung $s_1$, die an der Stelle \eqref{eq:Spiegelung1} definiert
wurde, heranziehen. Es gilt $s_1^2=\id$ und daher:
\begin{equation}
\langle s_1^2\rangle = \langle\id\rangle
= \{\id\} \ne \langle s_1\rangle = \{\id,s_1\}.
\end{equation}
\subsection{Ringe}
\subsubsection{Elementare Eigenschaften.}
Ein Ring ist eine Struktur $(R,+,*)$.

Axiome: EANIK, EA, D.

\strong{Definition.} $a-b:=a+(-b)$.

Es gilt
\begin{equation}
a*0=0.
\end{equation}

Beweis.
\begin{align}
a*0 = a*(0+0) &= a*0+a*0\qquad\qquad\qquad\qquad (\text{Axiome N, D})\\
\iff \underbrace{a*0-a*0}_{=0} &= a*0+\underbrace{a*0-a*0}_{=0}.\;\Box
\qquad\quad\; (\text{Axiome A, I})
\end{align}
Es gilt
\begin{equation}\label{eq:minus-minus-plus}
a*(-b) = -(a*b),\qquad (-a)*b = -(a*b).
\end{equation}
Beweis.
\begin{align}
0 = a*0 = a*(b+(-b)) &= a*b+a*(-b)\\
\iff  -(a*b) &= a*(-b).\;\Box
\end{align}
Die Rechnung für die zweite Gleichung ist analog.

Es gilt
\begin{equation}
(-a)*(-b) = a*b.
\end{equation}
Beweis.
\begin{equation}
(-a)*(-b) \stackrel{\eqref{eq:minus-minus-plus}}= -(a*(-b))
\stackrel{\eqref{eq:minus-minus-plus}}= -(-(a*b)) = a*b.
\end{equation}
\subsubsection{Ringhomomorphismen}
\begin{Definition}
Seien $(R,+,*)$ und $(R',+',*')$ zwei Ringe. Eine Abbildung
$\varphi\colon R\to R'$ heißt \emdef{Ringhomomorphismus},
wenn gilt:
\begin{equation}
\varphi(a+b) = \varphi(a)+\varphi(b),\qquad
\varphi(a*b) = \varphi(a)*\varphi(b),\qquad
\varphi(1) = 1.
\end{equation}
\end{Definition}
\subsubsection{Beispiele}
Für jeden Ring $R$ ist $R^{n\times n}$ bezüglich
Addition $(a_{ij})+(b_{ij}):=(a_{ij}+b_{ij})$
und Multiplikation $(a_{ij})(b_{ij}):=(\sum_k a_{ik}b_{kj})$
von Matrizen wieder ein Ring.

Im Wesentlichen müssen das Assoziativgesetz und die beiden
Distributivgesetze überprüft werden. Es ergibt sich
\begin{equation}
(\sum_k a_{ik}b_{kj})(c_{ij})
= \sum_\ell \sum_k a_{ik}b_{k\ell}c_{\ell j}
= \sum_k \sum_{\ell} \ldots
= (a_{ij})(\sum_\ell b_{i\ell}c_{\ell j})
\end{equation}
und
\begin{gather}
\hspace{-2.2em}(a_{ij})(x_{ij}{+}y_{ij})
= \sum_k a_{ik} (x_{kj}{+}y_{kj})
= \sum_k a_{ik} x_{kj}{+}\sum_k a_{ik}y_{kj}
= (a_{ij})(x_{ij}){+}(a_{ij})(y_{ij}),\\
\hspace{-2.2em}(x_{ij}{+}y_{ij})(a_{ij})
= \sum_k (x_{kj}{+}y_{kj})a_{ik} 
= \sum_k x_{kj}a_{ik}{+}\sum_k y_{kj}a_{ik}
= (x_{ij})(a_{ij}){+}(y_{ij})(a_{ij}).
\end{gather}

\section{Analysis}
\subsection{Grenzwert einer Funktion}\index{Grenzwert!einer Funktion}
\begin{Definition}
Sei $D\subseteq\R$ und $a$ ein Berührpunkt von $D$.
Sei $f\colon D\to\R$. Es sei $\lim_{x\to a} f(x)=L$ genau dann,
wenn für jede Folge $(x_n)$, $x_n\in D$ mit $x_n\to a$ die
Bilderfolge $f(x_n)$ gegen $L$ geht. Wie bei Folgen wird $L$ als
\emph{Grenzwert} bezeichnet.
\end{Definition}
Diese Definition wirkt zunächst außerordentlich unhandlich, da
zunächst nicht klar ist, wie man etwas für alle beliebigen
Folgen überprüfen soll. Dies ist für einfache Beispiele aber nicht
sonderlich schwer, da man einfach eine Variable verwenden
kann, welche für alle möglichen Folgen steht.

Ein einführendes Beispiel. Betrachtet wird die Funktion $f(x):=x^2$
mit $f\colon\R\to\R$. Wir würden gerne $L=\lim_{x\to 2} f(x)$
berechnen. Nun sei $(x_n)$ eine beliebige Folge mit $x_n\to 2$. Es
gilt nun
\begin{equation}
\lim_{n\to\infty} f(x_n)
= \lim_{n\to\infty} x_n^2
= \lim_{n\to\infty} (x_n\cdot x_n)
= \Big(\lim_{n\to\infty} x_n\Big)\cdot\Big(\lim_{n\to\infty} x_n\Big)
= 2\cdot 2 = 4.
\end{equation}
Da die Folge $(x_n)$ konvergent ist, konnten wir den Grenzwertsatz
für Summen anwenden. Somit muss $L=4$ sein.

Solche Rechnungen müssen für einfache Beispiele nicht weiter
durchgeführt werden, denn die Zusammenhänge sind hier so direkt, dass
sich die Grenzwertsätze für Folgen gleich auf Grenzwerte für
Funktionen übertragen lassen.

Angenommen es gilt $\lim_{x\to a} f(x)=A$ und
$\lim_{x\to a} g(x) = B$. Sei $x_n\to a$. Dann konvergieren
$f(x_n)\to A$ und $g(x_n)\to B$. Somit ist
\begin{equation}
\lim_{n\to\infty} (f(x_n)\cdot g(x_n))
= \Big(\lim_{n\to\infty} f(x_n)\Big)\cdot \Big(\lim_{n\to a} g(x_n)\Big)
= AB.
\end{equation}
Es ergibt sich
\begin{equation}
\lim_{x\to a} (f(x)\cdot g(x)) = AB.
\end{equation}
Die anderen Grenzwertsätze lassen sich auf die gleiche Art
verifizieren.

Jetzt können wir einfach
\begin{equation}\label{eq:Grenzwert-Quadrat}
\lim_{x\to a} x^2 = \lim_{x\to a} (x\cdot x)
= \Big(\lim_{x\to a} x\Big)\cdot\Big(\lim_{x\to a} x\Big) = a^2
\end{equation}
rechnen, ohne den Umweg über eine Folge gehen zu müssen.

Diese Zusammenhänge ermöglichen es, Grenzwerte für eine große
Klasse von Funktionen sofort in die Hand geschenkt zu bekommen.

\subsection{Stetige Funktionen}
\begin{Definition}
Eine Funktion $f\colon D\to\R$ heißt \emph{stetig} an der Stelle
$a\in D$, wenn $\lim_{x\to a} f(x)=f(a)$ gilt. Man nennt $f$ stetig
in $D$, wenn $f$ an jeder Stelle von $D$ stetig ist.
\end{Definition}
\strong{Beispiel.} Die Funktion $f(x):=x^2$ mit $f\colon\R\to\R$
ist stetig. Denn es gilt
\begin{equation}
\lim_{x\to a} f(x) = \lim_{x\to a} x^2 \stackrel{\eqref{eq:Grenzwert-Quadrat}}= \Big(\lim_{x\to a} x\Big)^2
= a^2 = f(a)
\end{equation}
für jedes $a\in\R$.

Ist $f\colon D\to\R$ eine stetige Funktion, so folgt sofort
\begin{equation}
\lim_{n\to\infty} f(x_n) = f(\lim_{n\to\infty} x_n)
\end{equation}
für jede konvergente Folge $(x_n)$, $x_n\in D$ mit $x_n\to a\in D$.

\subsection{Die Richtungsableitung}
Man stelle sich einen affinen Raum $A$ vor, auf welchem eine
Funktion $f_A\colon A\to\R$ definiert ist.
Um eine solche Funktion konkret angeben zu können, benötigen
wir eine \emph{Koordinatendarstellung} von $f_A$. Dazu wählt man
eine affine Basis $(p_0,B)$, die ein Koordinatensystem%
\begin{equation}
\varphi\colon \R^n\to A,\quad\varphi(x) := p_0+\sum_{k=1}^n x_k b_k
\end{equation}
mit $x=(x_k)_{k=1}^n$ und $B=(b_k)_{k=1}^n$ induziert. Man beachte,
dass $\varphi$ eine invertierbare affine Abbildung ist. Ein
Koordinatensystem ist also eine invertierbare affine Abbildung
zwischen dem Koordinatenraum und einem affinen Raum.

Wählt man speziell $A=\R^n$, so kann $B$ als Matrix aus
Spaltenvektoren $b_k=(b_{ik})$ dargestellt werden. Dann ist
\begin{equation}
y = \varphi(x) = p_0+Bx.
\end{equation}
Die Matrix $B$ muss nun regulär sein, damit die Umformung
\begin{equation}
x = B^{-1}(y-p_0) = \varphi^{-1}(y)
\end{equation}
formuliert werden kann. Betrachtet man einen affinen Raum $A$ als
abstraktes Objekt, so ist dieser nicht über ein Koordinatensystem
$\varphi$ erfassbar, weil $A$ selbst nicht erfassbar ist. Denkt
man sich aber zwei Koordinatensysteme $\varphi,\psi$, so ist
der Koordinatensystemwechsel $T=\varphi^{-1}\circ\psi$ sehr
wohl erfassbar. Den Raum $A$ können wir nun völlig außer acht lassen.
Alle Darstellungen von Funktionen können sich somit auf
den $\R^n$ beziehen und über
\begin{equation}
T\colon\R^n\to\R^n,\quad T(x) := p+Ax
\end{equation}
mit einer Matrix $A\in\operatorname{GL}(\R,n)$ transformiert werden.
Man beachte
hierzu, dass die Inverse einer affinen Abbildung und die Verkettung
von zwei affinen Abbildungen wieder eine affine Abbildung ist. Somit
muss $T=\varphi^{-1}\circ\psi$ auch wieder eine affine Abbildung
sein.

Anstelle von $f_A$ betrachtet man stattdessen
die konkrete Funktion
\begin{equation}
f\colon\R^n\to\R,\quad f := f_A\circ\varphi.
\end{equation}
Bei einer Koordinatentransformation ergibt sich nun
\begin{equation}
f\circ T = f_A\circ\varphi\circ\varphi^{-1}\circ\psi
= f_A\circ\psi,
\end{equation}
was als gleichberechtigte Darstellung von $f_A$ angesehen werden
darf.

Bei $f$ handelt es sich um eine Funktion in mehreren Variablen.
Wir würden gerne auch für $f$ so etwas wie eine Ableitung definieren,
um auch für solche Funktionen eine Differentialrechnung entwickeln
zu können.

Betrachte dazu eine Gerade $g\in\R^n$ und einen Punkt $p\in\R^n$.
Für die Anschauung sei $n=2$. Man definiert nun die Projektion
\begin{equation}
\pi\colon\R^n\times\R\to\R^n,\quad \pi(x,y):=x,
\end{equation}
die einen Punkt $(x,y)$ senkrecht auf die Stelle $x$
herunterprojiziert.

Zu beachten ist, dass $x=(x_k)_{k=1}^n$ ein
Tupel, $y$ aber nur eine reelle Zahl ist.

Man betrachte nun die Ebene $\pi^{-1}(g)$. Die
Schnittmenge
\begin{equation}
G=\operatorname{Graph}(f)\cap\pi^{-1}(g)
\end{equation}
ist eine Kurve, die als Graph einer Funktion $s\colon\R\to\R$
dargestellt werden kann. Aber $s$ lässt sich mit dem herkömmlichen
Differentialquotient ableiten. Wählt man auf $g$ nun eine affine
Basis $(p,v)$, so wird dadurch das Koordinatensystem
\begin{equation}
\varphi\colon\R\to\R^n,\quad\varphi(t):=p+tv
\end{equation}
induziert. Ein solches Koordinatensystem für einen eindimensionalen
Raum wird auch als \emph{Lineal} bezeichnet.

Nun gibt es eine Darstellung $s=f\circ\varphi$.

\begin{Definition}
\emph{Richtungsableitung} von $f$ in Richtung $v$
an der Stelle $p$:
\begin{equation}
(D_v f)(p) := (f\circ\varphi)'(0).
\end{equation}
\end{Definition}
\noindent
Aus der Definition können wir sofort eine Analogiebetrachtung
bei den Differentialquotienten extrahieren:
\begin{align}
f'(x) = \lim_{h\to 0}\frac{f(x+h)-f(x)}{h},\qquad
(D_v f)(p) = \lim_{h\to 0}\frac{f(p+hv)-f(p)}{h}.
\end{align}
Der Koordinatenraum $\R^n$ besitzt die kanonische Basis $(e_k)_{k=1}^n$.
Die Basisvektoren lassen sich natürlich als Richtungsvektoren
verwenden, und diesen Richtungsableitungen kommt eine besondere
Bedeutung zu.

\begin{Definition}
Die Richtungsableitungen bezüglich
der kanonischen Basis werden als \emph{partielle Ableitungen}
bezeichnet:
\begin{equation}
(D_k f)(p) := (D_{e_k} f)(p).
\end{equation}
Bemerkung: Es gibt die alternativen Schreibweisen
\begin{equation}
(D_v f)(p) \equiv \frac{\partial f(x)}{\partial v}\Big|_{x=p},\qquad
(D_k f)(p) \equiv \frac{\partial f(x)}{\partial x_k}\Big|_{x=p}.
\end{equation}
\end{Definition}
\noindent
Außerdem gibt es eine zweite Vorstellung von Differenzierbarkeit.
Diese Vorstellung ist, dass an der Stelle $p$ eine Tangentialebene
an den Graphen von $f$ gelegt werden kann. Eine solche
Tangentialebene muss die Funktionsgleichung
\begin{equation}
T(x_1,x_2) = f(p)+a_1(x_1-p_1)+a_2(x_2-p_2)
\end{equation}
besitzen, wobei $a_1,a_2$ zwei zu bestimmende Parameter sind.
Die Tangentialebene soll die Funktion $f$ an der Stelle $p$ nun
möglichst gut approximieren. Für ein kleines $h$ soll also
\begin{equation}
f(p+h)-T(p+h)\approx 0
\end{equation}
sein. Schärfer soll sogar
\begin{equation}
\lim_{h\to 0}\frac{f(p+h)-T(p+h)}{\|h\|}=0
\end{equation}
gelten.

\begin{Definition}
Eine Funktion $f\colon\R^n\to\R$ heißt
\emph{total differenzierbar} an der Stelle $p$, wenn
\begin{equation}
\lim_{h\to 0}\frac{f(p+h)-f(p)-\langle a,x-p\rangle}{\|h\|}=0
\end{equation}
ist. Der Koordinatenvektor $a$ ist hierdurch eindeutig bestimmt und
wird Ableitung von $f$ genannt. Man nennt%
\begin{equation}
(\mathrm df)(p) = \sum_{k=1}^n a_k\,\mathrm dx^k(p)
\end{equation}
das \emph{totale Differenzial} von $f$. Hiernach ergibt sich
\begin{equation}
\langle a,x-p\rangle = (\mathrm df)(p)(x-p)
\end{equation}
über duale Paarung, wobei die linke Seite bezüglich Koordinatentupeln
gemeint ist und dich rechte bezüglich tatsächlichen Vektoren.
\end{Definition}

Das wirkt zunächst wie eine zweite unnötige Schreibweise. Betrachtet
man aber $f_A$ anstelle von $f$, so muss es nicht unbedingt ein
Skalarprodukt $\langle a,x-p\rangle$ geben, weil $x-p$ als
Differenz zweier Punkte $x,p\in A$ im Verschiebungsvektorraum
$V$ liegt. Und auf $V$ braucht kein Skalarprodukt definiert zu sein.
Stattdessen entstammt $(\mathrm df)(p)$ aus dem Dualraum $V^\ast$
und ist bezüglich der dualen Basis $(\mathrm dx^k)_{k=1}^n$ definiert,
so dass $(\mathrm df)(p)$ und $x-p$ dual gepaart werden können.

Noch eine Bermerkung zum Definitionsbereich von $f$: Es reicht aus,
wenn $f$ auf einer offenen Umgebung des betrachteten Punktes $p$
definiert ist. Die Umgebung sollte aber offen sein, damit man sich bei
Grenzwertbildungen dem Punkt $p$ aus jeder Richtung nähern kann.
Sonst würde man einen Ableitungsbegriff benötigen, der bestimmte
Richtungen ausschließt, was unnötigen
technischen Balast darstellt.

\section{Kombinatorik}
\subsection{Funktionen Zählen}
Man definiert
\begin{equation}
Y^X := \{f\colon X\to Y\},
\end{equation}
das ist die Menge der Funktionen mit Definitionsmenge $X$ und
Zielmenge $Y$. Sind $X$ und $Y$ endlich, so kann auch die
Anzahl der Funktionen bestimmt werden. Hierfür ergibt sich
\begin{equation}
|Y^X| = |Y|^{|X|}.
\end{equation}

\subsubsection{Permutationen Zählen}
\begin{Definition}
Eine bijektive Selbstabbildung $\sigma\colon X\to X$ heißt
\emdef{Permutation}. Die Menge aller Permutationen auf $X$ wird mit
$S(X)$ bezeichnet und \emdef{symmetrische Gruppe} genannt.
\end{Definition}
Sei $n:=|X|$. Wir haben nun $n$ freie Plätze, auf die $n$
unterschiedliche Stifte gelegt werden sollen. Bei $n=1$ gibt es
eine Möglichkeit, bei $n=2$ sind es zwei und bei $n=3$ gibt es schon
sechs Möglichkeiten. Das Zählen der Möglichkeiten wird schnell
unübersichtlich. Um Übersicht zu gewinnen, überlegt man sich das
folgende systematische Prinzip. Zunächst soll $n!$ die Anzahl der
Möglichkeiten für $n$ Stifte auf $n$ freie Plätze bezeichnen.
Für den ersten Stift gibt es nun $n$ freie Plätze, und für jede
dieser Möglichkeiten bleiben $n-1$ Plätze übrig, für die es jeweils
$(n-1)!$ Möglichkeiten gibt. Das sind insgesamt $n\cdot (n-1)!$
Möglichkeiten.

\begin{Definition}
\emdef{Fakultät}:
\begin{equation}
1!:=1,\qquad n! := n\cdot (n-1)!.
\end{equation}
\end{Definition}
\noindent
Wird die Rekursionsgleichung bis zum Rekursionsanfang ausgeführt,
so ergibt sich
\begin{equation}
n! = \prod_{k=1}^n k,
\qquad\prod_{k=m}^m a_k:=a_m,
\qquad\prod_{k=m}^n a_k := a_n\prod_{k=m}^{n-1} a_k.
\end{equation}
Der Induktionsanfang ist trivial. Der Induktionsschritt ist
\begin{equation}
n! = n\cdot (n-1)! = n\prod_{k=1}^{n-1} k = \prod_{k=1}^n k.
\end{equation}

\subsection{Endliche Summen}
\begin{Definition}
Das \emdef{Summenzeichen} (der \emdef{Summierungsoperator}) wird rekursiv
definiert:
\begin{equation}
\sum_{k=m}^m a_k := a_1,\qquad \sum_{k=m}^{n} := a_n+\sum_{k=m}^{n-1} a_k.
\end{equation}
\end{Definition}

\noindent
\strong{Rechenregel.} \emph{Die Summierung lässt sich auf Summanden verteilen}:
\begin{equation}
\sum_{k=m}^n (a_k+b_k) = \sum_{k=m}^n a_k + \sum_{k=m}^n b_k.
\end{equation}
Induktionsanfang:
\begin{equation}
\sum_{k=m}^m (a_k+b_k) = a_m+b_m = \sum_{k=m}^m a_k+\sum_{k=m}^m b_k.
\end{equation}
Induktionsschritt:
\begin{equation}
\begin{split}
&\sum_{k=m}^n (a_k+b_k) = (a_n+b_n)+\sum_{k=m}^{n-1} (a_k+b_k)
= a_n+b_n+\sum_{k=m}^{n-1} a_k+\sum_{k=m}^{n-1}b_k\\
&= \Big(a_n+\sum_{k=m}^{n-1} a_k\Big)+\Big(b_n+\sum_{k=m}^{n-1} b_k\Big)
= \sum_{k=m}^n a_k+\sum_{k=m}^n b_k.
\end{split}
\end{equation}
\strong{Rechenregel.} \emph{Ein konstanter Faktor lässt sich aus dem
aus der Summierung herausziehen (Distributivgesetz)}:
\begin{equation}
\sum_{k=m}^n ca_k = c\sum_{k=m}^n a_k.
\end{equation}
Induktionsanfang:
\begin{equation}
\sum_{k=m}^m ca_k = ca_m = c\sum_{k=m}^m a_k.
\end{equation}
Induktionsschritt:
\begin{equation}
\sum_{k=m}^n ca_k = ca_n+\sum_{k=m}^{n-1} ca_k = ca_n+c\sum_{k=m}^{n-1} a_k
= c\Big(a_n+\sum_{k=m}^{n-1} a_k\Big) = c\sum_{k=m}^n a_k.
\end{equation}
\strong{Rechenregel.} \emph{Konstante Summanden}:
\begin{equation}
\sum_{k=m}^n c = (n{-}m{+}1)\,c.
\end{equation}
Induktionsanfang:
\begin{equation}
\sum_{k=m}^m c = c = (m{-}m{+}1)\,c.
\end{equation}
Induktionsschritt:
\begin{equation}
\sum_{k=m}^n c = c+\sum_{k=m}^{n-1} c
= c+(n{-}1{-}m{+}1)\,c = c+(n{-}m)\,c = (n{-}m{+}1)\,c.
\end{equation}
\strong{Formel.} \emph{Partialsumme der geometrischen Reihe}:
\begin{equation}\label{eq:Partialsumme-geometrische-Reihe}
\sum_{k=1}^n q^k = \frac{q^{n+1}-q}{q-1}.
\end{equation}
Herleitung.
\begin{equation}
\sum_{k=1}^n q^k = q\sum_{k=1}^n q^{k-1}
= q\sum_{k=0}^{n-1} q^k
= q\cdot\bigg(1-q^n+\sum_{k=1}^n q^k\bigg).
\end{equation}
Setze nun $y:=\sum_{k=1}^n q^k$, um Schreibaufwand zu sparen.
Die Gleichung
\begin{equation}
y = q\cdot (1-q^n+y) = q-q^{n+1}+qy
\end{equation}
wird nun nach $y$ umgestellt. Es ergibt sich
\begin{equation}
y = \frac{q^{n+1}-q}{q-1}.
\end{equation}
\strong{Formel.} \emph{Es gilt}:
\begin{equation}\label{eq:diff-Partialsumme-geometrische-Reihe}
\sum_{k=m}^{n-1} k^p q^k = \bigg(q\frac{\mathrm d}{\mathrm dq}\bigg)^p\;\frac{q^n-q^m}{q-1}.
\end{equation}
Der Induktionsanfang $p=0$ ist  \eqref{eq:Partialsumme-geometrische-Reihe}.
Beim Induktionsschritt gehen wir von der Gültigkeit von
\begin{equation}
\sum_{k=m}^{n-1} k^{p-1} q^k = \bigg(q\frac{\mathrm d}{\mathrm dq}\bigg)^{p-1}\;\frac{q^n-q^m}{q-1}.
\end{equation}
aus. Auf beiden Seiten der Gleichung wird nun $q\frac{\mathrm d}{\mathrm dq}$ appliziert,
womit sich \eqref{eq:diff-Partialsumme-geometrische-Reihe} ergibt.

Bemerkung: Mit Ausnahme von Spezialfällen ist
\begin{equation}
\bigg(q\frac{\mathrm d}{\mathrm dq}\bigg)^p \ne q^p \frac{\mathrm d^p}{\mathrm dq^p},
\end{equation}
da das Produkt mit $q$ nach dem Differentialoperator vom nächsten
Differentialoperator mit abgeleitet werden muss.

Aus \eqref{eq:diff-Partialsumme-geometrische-Reihe} wird mit $p=1$ nun
z.\,B. die Formel
\begin{equation}
\sum_{k=m}^{n-1} kq^k = \frac{nq^n-mq^m}{q-1} - \frac{q^{n+1}-q^{m+1}}{(q-1)^2}
\end{equation}
gewonnen.

\subsection{Differenzenrechnung und Teleskopsummen}
Sei
\begin{align}
(\Delta f)(n) &:= f(n+1)-f(n), && \text{(\emph{Differenzenoperator})}\\
(\Sigma f)(n) &:= \sum_{k=0}^{n-1} f(k). && \text{(\emph{Summenoperator})}
\end{align}
Wir wollen zeigen:
\begin{align}
(\Delta\circ\Sigma)(f)(n) = (\Delta(\Sigma f))(n) = \id(f)(n) = f(n).
\end{align}
Und rechnen nach:
\begin{align}
&(\Delta(\Sigma f))(n)
= \Delta(n\mapsto\sum_{k=0}^{n-1} f(k))(n)
= \sum_{k=0}^{n} f(k)-\sum_{k=0}^{n-1} f(k)\\
&= f(n)+\underbrace{\sum_{k=0}^{n-1} f(k)-\sum_{k=0}^{n-1} f(k)}_{=0}
= f(n).
\end{align}
Das bedeutet, dass $\Delta$ die Linksinverse zu $\Sigma$ ist.

Somit muss $\Sigma\colon\R^\N\to\R^\N$ ein injektiver Operator
sein.

Jetzt kehren wir die Reihenfolge um und berechnen $\Sigma\circ\Delta$.
Es ergibt sich:
\begin{align}
&(\Sigma(\Delta f))(n) = (\Sigma(n\to f(n+1)-f(n)))(n)
= \sum_{k=0}^{n-1} [f(k+1)-f(k)]\\
&= \underbrace{\sum_{k=0}^{n-1} f(k+1)}_{\sum_{k=1}^n f(k)}
- \sum_{k=0}^{n-1} f(k)
= f(n)-f(0) + \underbrace{\sum_{k=0}^{n-1} f(k)
- \sum_{k=0}^{n-1} f(k)}_{=0}.
\end{align}
Bis auf die Konstante $f(0)$ ist also auch $\Sigma$ invers zu $\Delta$.
Dieser Zusammenhang nennt sich \emph{Teleskopsumme}.

\subsection{Formale Potenzreihen}
Zwischen Potenzreihen und formalen Potenzreihen besteht
ein großer Unterschied. Die Notation
\begin{equation}
\sum_{k=0}^\infty a_k x^k := \lim_{n\to\infty}\sum_{k=0}^n a_k x^k
\end{equation}
bedeutet bei Potenzreihen den Wert der Reihe, das ist der
Grenzwert der Partialsummenfolge. Die Partialsummenfolge sollte
dafür konvergent oder zumindest bestimmt divergent sein.
Die Notation
\begin{equation}
\sum_{k=0}^\infty a_k X^k := (a_k)_{k=0}^\infty = (a_0,a_1,a_2,a_3,\ldots)
\end{equation}
ist bei formalen Potenzreihen bloß eine andere
Schreibweise für die Folge $(a_k)$.

Die Addition von zwei formalen Potenzreihen wird
wie die Addition von zwei Polynomen erklärt:
\begin{equation}\label{eq:Addition-formale-Potenzreihen}
\sum_{k=0}^\infty a_k X^k + \sum_{k=0}^\infty b_k X^k
:= \sum_{k=0}^\infty (a_k+b_k)X^k.
\end{equation}
Die Multiplikation von zwei Polynomen lautet allgemein:
\begin{align}
\bigg(\sum_{i=0}^m a_i X^i\bigg)\bigg(\sum_{j=0}^n b_j X^j\bigg)
= \sum_{i=0}^m \bigg(\sum_{j=0}^n b_j X^j\bigg) a_i X^i
= \sum_{i=0}^m \sum_{j=0}^n \underbrace{a_i X^i b_j X^j}_{\displaystyle a_ib_j X^{i+j}}.
\end{align}
Bei der Summation ist die Reihenfolge aber beliebig, da im Ring
bezüglich der Addition das Kommutativgesetz und das Assoziativgesetz
gültig ist. Multipliziert man zwei Polynome aus und gruppiert
die Summanden nach dem Grad, so ergibt sich nun
\begin{equation}
\begin{split}
\sum_{i=0}^m \sum_{j=0}^n a_ib_j X^{i+j}
&= a_0 b_0 X^0 \,+\, (a_0 b_1 + a_1 b_0) X^1\,+\,
(a_0 b_2 + a_1 b_1 + a_2 b_0) X^2\\
&+\,(a_0 b_3 + a_1 b_2 + a_2 b_1 + a_3 b_0) X^3\,+\,\ldots
\end{split}
\end{equation}
Das Monom vom Grad $k$ ist offenbar immer von der Form
\begin{equation}
\bigg(\sum_{i=0}^k a_i b_{k-i}\bigg) X^k
= (a_0 b_k + a_1 b_{k-1} + \ldots + a_{k-1} b_1 + a_k b_0) X^k.
\end{equation}
Als Grad des Produktes ergibt sich $m+n$, wenn der Ring
nullteilerfrei ist, sonst maximal $m+n$. Insgesamt ergibt sich
die Formel
\begin{equation}\label{eq:Produkt-Polynome}
\bigg(\sum_{i=0}^m a_i X^i\bigg)\bigg(\sum_{j=0}^n b_j X^j\bigg)
= \sum_{k=0}^{m+n}\bigg(\sum_{i=0}^k a_i b_{k-i}\bigg) X^k.
\end{equation}
Die inneren Summen sind auch bei unendlichen Reihen endlich, was
die folgende Definition motiviert.

\begin{Definition}
\emdef{Produkt} von zwei formalen Potenzreihen:
\begin{equation}\label{eq:Produkt-formale-Potenzreihen}
\bigg(\sum_{i=0}^\infty a_i X^i\bigg)\bigg(\sum_{j=0}^\infty b_j X^j\bigg)
:= \sum_{k=0}^\infty\bigg(\sum_{i=0}^k a_i b_{k-i}\bigg) X^k.
\end{equation}
Das Produkt der formalen Potenzreihen zu $(a_k)$ und $(b_k)$ wird
auch \emdef{Faltung} der Folgen $(a_k)$ und $(b_k)$ genannt.
Zu dieser alternativen Sprechweise gehört die alternative
Schreibweise%
\begin{equation}
(a_k)\ast (b_k) := (c_k),\quad c_k:=\sum_{i=0}^k a_i b_{k-i}.
\end{equation}
\end{Definition}
\noindent
Die Polynome lassen sich in die Menge der formalen Potenzreihen
einbetten, indem $a_k:=0$ für $k>m$ und $b_k:=0$ für $k>n$
gesetzt wird. Aus Formel \eqref{eq:Produkt-formale-Potenzreihen}
wird dann \eqref{eq:Produkt-Polynome}.

Die Menge aller Potenzreihen mit Koeffizienten aus dem Ring $R$:
\begin{equation}\textstyle
R[[X]] := \{\sum_{k=0}^\infty a_k X^k\mid a_k\in R\}.
\end{equation}
Die Menge $R[[X]]$ bildet bezüglich der Addition
\eqref{eq:Addition-formale-Potenzreihen} und der Multiplikation
\eqref{eq:Produkt-formale-Potenzreihen}
einen Ring.

\section{Dynamische Systeme}
\subsection{Fixpunkt-Methode}
Betrachte eine Iteration
\begin{equation}\label{eq:Fixpunkt-Iteration}
x_{n+1}=\varphi(x_n)
\end{equation}
mit einem Startwert $x_0$. Z.\,B.
\begin{equation}
\varphi(x) = \frac{1}{a+x},\quad a\in\N
\end{equation}
mit beliebigem Startwert. Man stellt numerisch fest, dass die
Folge $(x_n)$ bei diesem Beispiel (es handelt sich um Kettenbrüche)
offenbar immer gegen eine Zahl konvergiert. Außerdem
scheint der Startwert dafür unbedeutsam zu sein. Bei $a=2$ erkennt
ein Rechenmeister mit geschultem Blick noch, dass die Folge
gegen $\sqrt{2}-1$ konvergiert, bei $a=1$ gegen den goldenen Schnitt.
Aber bei $a=3$ ist der Wert zunächst von einem dunklen Schleier
umgeben.

Eine weitere Beobachtung ist nun, dass sich die Werte der Folge
immer weniger voneinander unterscheiden, wenn $n$ wächst.
Es gilt also $x_n\approx x_{n+1}$ bzw. $x_n\approx \varphi(x_n)$
für große $n$. Für den Grenzwert $x=\lim_{n\to\infty} x_n$ müsste
dann
\begin{equation}
x = \varphi(x)
\end{equation}
gelten. Eine solche Gleichung wird als \emph{Fixpunktgleichung}
bezeichnet, die dazu gehörige Iteration \eqref{eq:Fixpunkt-Iteration}
als \emph{Fixpunkt-Iteration}.

Setzt man nun das Beispiel mit $a=2$ ein, so ergibt
sich
\begin{equation}
x = \frac{1}{2+x},
\end{equation}
was zur quadratischen Gleichung
\begin{equation}
x^2+2x-1 = 0
\end{equation}
führt, deren eine Lösung $\sqrt{2}-1$ ist. Warum es eine zweite
Lösung gibt, und diese nicht Ergebnis der Iteration ist, ist
zunächst unklar.

Die Beispiele
\begin{equation}
\varphi(x) = \frac{1}{1-bx},\quad b\in\{1,2\}
\end{equation}
scheinen niemals zu konvergierten. Für $b=1$ ergibt sich eine
periodische Folge, und bei $b=2$ scheint die Folge recht
chaotisch zu sein. Die zugehörigen Fixpunkt-Gleichungen führen
auf quadratische Gleichungen, welche keine reellen Lösungen
besitzen.

Wenn also der Fixpunkt eine komplexe Zahl ist, so kann die Folge
niemals gegen den Fixpunkt konvergieren, weil der Wert von $\varphi$
niemals eine komplexe Zahl sein wird, wenn als Argument eine
reelle Zahl gegeben wurde. Würden wir mit einem komplexen Startwert
beginnen, so hätten wir eine Chance auf Konvergenz. Ein Plotter
für die komplexe Ebene zeigt aber, dass für fast alle
Zahlen bis auf einzelne offenbar keine Konvergenz vorliegt.
Das Beispiel $b=1$ ist für fast alle komplexen Startwerte periodisch
mit Periodenlänge drei.

Beim Beispiel $a=2$ scheinen die Folgenwerte von der zweiten
Lösung der quadratischen Gleichung abgestoßen zu werden
und zur ersten Lösung zu wandern, was klar wird wenn man die
Iteration ganz in der Nähe der zweiten Lösung beginnt.
Man hat es hier mit einem abstoßenden und einem anziehenden
Fixpunkt zu tun. Das Abstoßen geschieht um so schneller, je weiter
der der Iterationswert vom abstoßenden Fixpunkt entfernt ist.

Bei den Beispielen $b=1$ und $b=2$ ergibt sich jedoch, dass sich
die Folgenwerte um die Fixpunkte herum bewegen, ohne sie in
absehbarer Zeit zu erreichen. Bei $b=2$ liegen die Werte
von $(x_n)$ auf einem Kreis um einen der Fixpunkte, dessen
Mittelpunkt aber nicht der Fixpunkt sein muss. Für
$\Im(x_0)=0$ entartet der Radius des Kreises ins unendliche. Die
beiden Fixpunkte sind zueinander komplex Konjugiert, und so kann
sich die Folge bei $\Im(x_0)=0$ nicht für die Umkreisung von einem
der beiden Fixpunkte entscheiden.

Bei $b=2+\ui/2$ ergibt sich noch ein viel seltsameres Gebilde.
Die Folgenwerte werden wieder von einem der beiden Fixpunkte
angezogen und von dem anderen abgestoßen. Dabei bilden sich
jedoch Strudel mit fünf Armen aus, wobei die Arme des abstoßenden
Fixpunktes mit denen des anziehenden Fixpunktes verbunden sind.

Doch da ist noch viel mehr. Nehmen wir das Beispiel
$\varphi(x)=x^2-1$. Damit Konvergenz vorliegt, dürfen die Folgenwerte
notwendigerweise einen bestimmten Betrag nicht überschreiten, denn
nach der umgekehrten Dreiecksungleichung gilt:
\begin{equation}
|\varphi(x)| = |x^2-1| \ge ||x^2|-|1|| = ||x|^2-1|.
\end{equation}
Für $|x|\ge 2$ ergibt sich $|x|^2\ge 4$ und $|x|^2-1\ge 3$.
Somit ist
\begin{equation}
||x|^2-1| = |x|^2-1
\end{equation}
und $|\varphi(x)|\ge 3$.
Man beachtet jetzt
\begin{equation}
|\varphi(x)|\ge |x|^2-1 \implies \sqrt{|\varphi(x)|+1}\ge |x|.
\end{equation}
Nun ist aber $|\varphi(x)|>\sqrt{|\varphi(x)|+1}$ für
$|\varphi(x)|\ge (\sqrt{5}+1)/2$, was bei $|x|\ge 2$ erfüllt ist.
Insgesamt ergibt sich $|\varphi(x)|>|x|$, was folglich die Divergenz
der Folge zur Folge hat.

Die Menge der Startwerte $x_0\in\C$, so dass
$(x_n)$ mit $x_{n+1}=x_n^2+c$ beschränkt bleibt,
heißt \emph{Julia-Menge} zur Konstanten $c$. Die Menge zu $c=-1$,
die innerhalb des Kreises $|x|=2$ liegt, ist ein außerordentlich
seltsames Gebilde, welches dem Augenschein nach
fraktale Strukturen besitzt.

Für einen Startwert $x_0$ wird die Folge $(x_n)$ nach
\eqref{eq:Fixpunkt-Iteration} auch als \emph{Orbit}
von $x_0$ bezeichnet. Im Fall, dass negative $n$ nicht zugelassen
sind, spricht man auch von einem \emph{Semiorbit}. Dieser Begriff
ist mit dem Gruppentheoretischen Begriff \emph{Orbit} deckungsgleich,
denn die Iterierten $\varphi^n$ bilden bezüglich Verkettung
eine Gruppe und erwirken eine Gruppenaktion $\varphi^n(x_0)$.
Sind negative $n$ nicht zugelassen,
so handelt es sich nicht um eine Gruppe, sondern um ein Monoid,
dessen Elemente eine Monoidaktion anstelle einer Gruppenaktion
erwirken.

\begin{Definition}
Die Struktur $(T,X,\Phi)$ mit
\begin{equation}
\Phi\colon T\times X\to X,\quad\Phi(t,x)
\end{equation}
wird als \emph{dynamisches System} bezeichnet. Das System heißt
\emph{diskret}, wenn $t\in\N$ bzw. $t\in\Z$ ist. Ein $x$ heißt
\emph{Ruhelage}, wenn $\Phi(t,x)=x$ für alle $t$ gilt. Der Orbit $\Phi(T,x)$
heißt \emph{periodisch} mit Periodenlänge $P$, wenn $\Phi(t+P,x)=\Phi(x)$
für alle $t$ gilt. 
\end{Definition}
Im Fall einer Fixpunkt-Iteration $x_{n+1}=\varphi(x_n)$ ist
\begin{equation}
\Phi(t,x):=\varphi^t(x).
\end{equation}
ein diskretes dynamisches System. Die Fixpunkte von $\varphi$ sind
die \emph{Ruhelagen} von $\Phi$.

\section{Tricks}
\subsection{Matrizen vom Typ $2\times 2$}
Matrizen der Form
\begin{equation}\label{eq:Drehstreckung}
\begin{bmatrix}
a & -b\\
b & a
\end{bmatrix} = r\begin{bmatrix}
\cos\theta & -\sin\theta\\
\sin\theta & \cos\theta
\end{bmatrix}
\end{equation}
sind isomorph zu den komplexen Zahlen $a+b\ui = r\ee^{\ui\theta}$.
Das erscheint
zunächst wie eine unwesentliche Tatsache, aber es ermöglicht,
Matrizenrechnungen gegen Rechnungen mit komplexen Zahlen
auszutauschen. Der Isomorphismus ist
\begin{equation}
\varphi\colon M\to\C,\quad
\varphi(\begin{bmatrix}
a & -b\\
b & a
\end{bmatrix}) := a+b\ui.
\end{equation}
Zunächst verifiziert man, dass $M$ und $\C$ Körper sind, indem
die Gültigkeit der Körperaxiome nachgewiesen wird. Hierbei ergibt
sich, dass $M\setminus\{0\}$ eine abelsche Gruppe ist. Es gilt
insbesondere $M\setminus\{0\}\le\operatorname{GL}(2,\R)$.

\begin{Definition}
Sind $(K,+,\bullet)$ und $(K',+',\bullet')$ zwei Körper, so wird
$\varphi\colon K\to K'$ als \emph{Körperhomomorphismus} bezeichnet,
wenn
\begin{align}
\varphi(a+b) &= \varphi(a)+'\varphi(b),\\
\varphi(a\bullet b) &= \varphi(a)\bullet'\varphi(b)
\end{align}
für alle $a,b\in K$ gilt und $\varphi(1)=1$ ist.
\end{Definition}
Ein bijektiver Körperhomomorphismus ist ein \emph{Körperisomorphismus}.

Nun kommutiert das folgende Diagramm:
\begin{equation}
\begin{matrix}
M & \xrightarrow{\quad\text{Berechnung}\quad} & M\\
\varphi\Big\downarrow\hspace{1em} & & \hspace{1.4em}\Big\uparrow\varphi^{-1}\\
\C & \xrightarrow{\quad\text{Berechnung}\quad} & \C
\end{matrix}
\end{equation}
Zunächst ergeben sich ein paar Zusammehänge, die ein wenig
verblüffend erscheinen:
\begin{equation}
\varphi(A^T) = \overline{\varphi(A)},\qquad
|\varphi(A)| = \sqrt{\det A},\qquad
\operatorname{Spur}(A) = 2\Re(\varphi(A)).
\end{equation}
Wichtiger sind Regeln wie
\begin{equation}\label{eq:Inverse-Potenz}
A^{-1} = \varphi^{-1}\bigg(\frac{1}{\varphi(A)}\bigg),\qquad
A^n = \varphi^{-1}(\varphi(A)^n).
\end{equation}
So lassen sich Matrizen einfach invertieren und große
Matrixpotenzen berechnen. Aber das bringt uns auf eine seltsame
Idee. Wir definieren nun
\begin{equation}
f(A) := \varphi^{-1}(f(\varphi(A))) = (\varphi^{-1}\circ f\circ\varphi)(A)
\end{equation}
für jede Funktion $f\colon G\to\C$ mit $\varphi(A)\in G$.
Diese Definition erweitert \eqref{eq:Inverse-Potenz}.

Für eine Matrix $A\in M$ lassen sich jetzt eine Reihe scheinbar
absurder Dinge berechnen:%
\begin{equation}
\sqrt{A},\quad A^{3/2},\quad A^\ui, \quad A^A,\quad \frac{1}{A+1},
\quad \ee^A, \quad \ln(A), \quad \sin(A).
\end{equation}
Einige würden jetzt den Nutzen solcher Operationen, ja sogar die
Operationen selbst in Frage stellen. Aber das hat man bei den
komplexen Zahlen zunächst auch getan. Tatsächlich ist $\sqrt{A}$ eine
Teilantwort auf das Problem, dass zu einer Matrix $A$ die Matrix $X$
mit $X^2=A$ gefunden werden soll.

Die nächste Frage besteht nun darin, ob sich Operationen
wie $f(x)=\ee^x$ auch auf Matrizen übertragen lassen,
die nicht in der Form \eqref{eq:Drehstreckung} vorliegen.

\vfill
\texttt{Dieses Heft steht unter der Creative"=Commons"=Lizenz CC0.}
\end{document}



