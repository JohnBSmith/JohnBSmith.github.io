\documentclass[a4paper,10pt,fleqn,twocolumn,twoside]{scrartcl}
\usepackage[utf8]{inputenc}
\usepackage[T1]{fontenc}
\usepackage[ngerman]{babel}
\usepackage{amsmath}
\usepackage{amssymb}
\usepackage{lipsum}

\usepackage[libertine,cmintegrals]{newtxmath}
\usepackage{libertine}

\usepackage{geometry}
\geometry{a4paper,left=28mm,right=10mm,top=30mm,bottom=30mm}
\setlength{\columnsep}{4mm}

\newcommand{\ee}{\mathrm e}
\newcommand{\ui}{\mathrm i}
\newcommand{\real}{\operatorname{Re}}
\newcommand{\imag}{\operatorname{Im}}
\newcommand{\uv}[1]{\underline{#1}}
\newcommand{\bv}[1]{\mathbf{#1}}

\newcommand{\N}{\mathbb N}
\newcommand{\Z}{\mathbb Z}
\newcommand{\Q}{\mathbb Q}
\newcommand{\R}{\mathbb R}
\newcommand{\C}{\mathbb C}

\newcommand{\id}{\operatorname{id}}
\newcommand{\sgn}{\operatorname{sgn}}
\newcommand{\Abb}{\operatorname{Abb}}
\newcommand{\unit}[1]{\mathrm{#1}}
\newcommand{\chem}[1]{\mathrm{#1}}
\newcommand{\strong}[1]{\textsf{\textbf{#1}}}

\begin{document}
\thispagestyle{empty}

\noindent
{\LARGE\textbf{Überblick über die\\
Mathematik}\par}

\tableofcontents

\section{Analysis}
\subsection{Warum Analysis?}
Was bringt uns die Analysis? Was erlaubt Analysis, was zuvor nicht
möglich war? Zur Beantwortung dieser Frage werfen wir einen Blick
auf wichtige Sätze, die zwar allgemeine Aussagen außerhalb
der Analysis machen, jedoch nur mittels Konzepten der Analysis
verstanden und bewiesen werden können.

Ein wichtiges Problem in der Mathematik ist das Lösen von Gleichungen.
Gleichungen sind aber eng mit Nullstellen verbunden. Eine
Gleichung in einer Variable, $T_1(x)=T_2(x)$ mit $T_1,T_2\colon G\to\R$
und $G\subseteq\R$ lässt sich über $f(x):=T_1(x)-T_2(x)$ als
$f(x)=0$ schreiben. Das Lösen der Gleichung entspricht dem Finden der
Nullstellen von $f$ bzw. dem Bestimmen des Urbilds $f^{-1}(0)$.
Hierbei stellen sich drei Probleme. Erstens, woher weiß man, ob
es überhaupt eine Nullstelle gibt? Zweitens, wie findet man eine
Nullstelle von $f$? Drittens, wie kann man sich sicher sein, alle
Nullstellen von $f$ gefunden zu haben?

Zum ersten Problem, Existenz von Nullstellen. Der Zwischenwertsatz
macht hierüber eine Aussage. Dieser setzt aber eine stetige Funktion
voraus.

Zum zweiten Problem, finden von Nullstellen. Dies lässt sich z.\,B.
Approximativ mittels Näherungsverfahren wie Bisektionsverfahren
oder Newtonverfahren bewerkstelligen. Das Newtonverfahren benötigt
aber Differentialrechnung.

Zum dritten Problem, ausschließen weiterer Nullstellen. Wenn die
Funktion auf einem Intervall injektiv ist, kann sie dort höchstens
eine Nullstelle besitzen. Jede streng monotone Funktion ist injektiv.
Eine differenzierbare Funktion ist sicher dann streng monoton, wenn
$f'(x)>0$ für alle $x$ aus dem betrachteten Intervall. Zur Bestimmung
der Ableitung $f'$ ist wieder Differentialrechnung notwendig.

Wie man sieht, erlaubt Analysis die Lösung eines praktischen Problems
außerhalb der Analysis. Hierzu muss man jedoch Funktionen auf
Eigenschaften aus der Analysis untersuchen, das sind hier
Stetigkeit und Differenzierbarkeit. Diese Begriffe sind aufs
engste mit Konvergenz verbunden, dem zentralen Konzept der Analysis.

\subsection{Was ist Analysis?}

Die zentralen Konzepte \emph{Konvergenz}, \emph{Stetigkeit}
und \emph{Differenzierbarkeit}
sind sogenannte lokale Eigenschaften. Hierbei untersucht man ein
Objekt an einer Stelle hinsichtlich beliebig klein werdender Umgebungen
dieser Stelle. Man kann sich das so vorstellen, als wenn man das
Objekt an der Stelle durch ein immer stärkeres Mikroskop betrachtet.
Damit das Objekt an der Stelle eine lokale Eigenschaft besitzt,
muss es ein bestimmtes reguläres Verhalten aufzeigen, egal wie klein
die Umgebungen werden.

Diese lokalen Eigenschaften ermöglichen eine tiefergehende,
reichhaltigere Untersuchung von Objekten wie Funktionen oder Folgen.
Mit der Zeit gelang es, diese Begriffe immer genauer zu
charakterisieren. Dabei hat sich herausgestellt, dass lokale
Eigenschaften mit einer bestimmten Art von Struktur verbunden sind,
der \emph{topologischen} Struktur. Umgebungen, Stetigkeit und
Konvergenz stellten sich als topologische Grundbegriffe heraus.

Man kann daher sagen, dass diese Analysis auf den Grundlagen der
Topologie aufbaut.

Neben lokalen Eigenschaften gibt es auch noch globale. Zum einen
ergibt sich bei manchen lokalen Eigenschaften eine entsprechende
globale Eigenschaft, wenn die lokale Eigenschaft an jeder Stelle
erfüllt wird. Das ist z.\,B. bei der Stetigkeit und der
Differenzierbarkeit der Fall. Bei der Lipschitz"=Stetigkeit
ist der Zusammenhang zwischen lokaler und globaler Version aber
ein wenig komplizierter.

Zum anderen gibt Eigenschaften, die nur global definiert sind.
Hierunter Fällt der Begriff der Integrierbarkeit und des bestimmten
Integrals. Integrierbarkeit baut wie Differenzierbarkeit auch
auf dem Konzept der Konvergenz auf. Das Integral ist ein sogenanntes
Funktional, das ist eine Abbildung die einer Funktion eine reelle
Zahl zuordnet.

Zwischen den unterschiedlichen globalen Eigenschaften gibt es auch
Beziehungen. Z.\,B. ist jede differenzierbare reelle Funktion auch
stetig und jede stetige reelle Funktion auch integrierbar.

Charakteristisch für die Analysis ist aber nicht nur die die
topologische Struktur. Die Analysis resultiert vielmehr aus einer
Synthese der topologischen Konzepte mit arithmetischen Konzepten.
Hier lässt sich interessanterweise eine genauere Unterscheidung
vornehmen. So sind Konvergenz und Stetigkeit ganz allgemeine Begriffe.
Die Begriffe Differentialquotient und bestimmtes Integral setzen
jedoch arithmetische Struktur bzw. Operationen voraus. Man könnte
nun einwenden, dass Untersuchungen zu Konvergenz und Stetigkeit eher
zur Topologie gehörten. Jedoch werden in der Analysis gerade solche
Folgen und Funktionen untersucht, die arithmetisch aufgebaut sind.

Außerdem spielt die Ordnungsstruktur der reellen Zahlen in der
Analysis eine wesentliche Rolle. Ungleichungen sind grundlegend
für die Untersuchung von Folgen und Funktionen hinsichtlich
Konvergenz, Stetigkeit, Differenzierbarkeit und Integrierbarkeit.

\subsection{Werkzeuge der Analysis}

Was sind die wesentlichen Werkzeuge der Analysis? Nun, die Werkzeuge
manifestieren sich wie immer in Begriffen und Sätzen. Die Sätze
motivieren hierbei die Begriffe. Werkzeuge dienen meist der Lösung
von bestimmten Problemen. Diese sollten wir daher auch in das Zentrum
stellen.

\strong{1. Extremwertaufgaben.} Die Bestimmung von Optimumsstellen
-- das sind Stellen an welchen eine Funktion einen größten oder
kleinsten Wert annimmt -- lässt sich darauf zurückführen, eine
waagerechte Tangente bzw. allgemeiner bei $f\colon\R^n\to\R$ eine
Nullstelle der totalen Ableitung $\mathrm df$ zu finden. Das
Verschwinden der totalen Ableitung ist nur eine notwendige Bedingung,
da es sich auch um einen Sattelpunkt handeln könnte. Aufbauend
gibt es aber auch Sätze über hinreichende Bedingungen, dann kann man
sich sicher sein, ein Optimum gefunden zu haben. Dieses Verfahren
lässt sich auch ohne Weiteres auf Mannigfaltigkeiten übertragen.
Ist $M$ eine Mannigfaltigkeit und $f\colon M\to\R$ ein Skalarfeld,
dann ist $\mathrm df(p)=0$ eine notwendige Bedingung, dass bei $p$
ein lokales Optimum vorliegt.

\strong{2. Extremwertaufgaben mit Nebenbedingungen.}\\
Das Verfahren zur Behandlung dieser Art von Optimierungsproblemen
ist von allergrößter Wichtigkeit für die Mathematik, die Physik,
die Technik und die Ökonomie. Man kann diese Art von Problem als
konzeptuelle Verallgemeinerung von gewöhnlichen Extremwertaufgaben
ansehen.

Zum Finden von Optimumsstellen einer differenzierbaren Funktion
$f(x,y)$ unter einer Nebenbedingung $g(x,y)=0$ ist eine notwendige
Bedingung, dass $\mathrm df$ kollinear zu $\mathrm dg$ ist, dass
es also ein $\lambda\in\R$ mit $\mathrm df = \lambda\mathrm dg$
gibt, bzw. dass $\mathrm df\wedge\mathrm dg=0$ gilt. Definiert
man nun $L(x,y,\lambda):=f(x,y)-\lambda g(x,y)$, dann ist
$\mathrm dL=0$ äquivalent zum Gleichungssystem aus Nebenbedingung
und notwendiger Bedigung. Nunmehr würde es sich um eine
gewöhnliche Extremwertaufgabe für $L$ handeln, jedoch gibt
es dabei noch einen Haken. Die notwendige Bedingung lässt auch
Sattelpunkte zu, und tatsächlich ist jede Lösung des ursprünglichen
Problems ein Sattelpunkt von $L$.

Das Verfahren ist ohne Schwierigkeiten verallgemeinerbar auf
Funktionen $f\colon\R^n\to\R$ mit mehreren Nebenbedingungen.
Außerdem lässt es sich auch ohne Weiteres auf Mannigfaltigkeiten
formulieren.

\begin{thebibliography}{0}
\bibitem{Princeton-Companion} Timothy Gowers (Ed.):
»The Princeton Companion of Mathematics«. Princeton University Press,
2008.
\end{thebibliography}

\newpage\mbox{}
\vfill\noindent
{\small Dieser Text steht unter der Lizenz\\
Creative Commons CC0 1.0.}

\end{document}
