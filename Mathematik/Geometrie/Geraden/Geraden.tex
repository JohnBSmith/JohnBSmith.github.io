\documentclass[a4paper,10pt,fleqn,twocolumn,twoside,dvipdfmx]{scrartcl}
\usepackage[utf8]{inputenc}
\usepackage[T1]{fontenc}
\usepackage[ngerman]{babel}
\usepackage{microtype}
\usepackage{libertine}
\usepackage[libertine]{newtxmath}
\addtokomafont{disposition}{\rmfamily}

\usepackage{amsmath}
\usepackage{amssymb}

\usepackage{color}
\definecolor{c1}{RGB}{00,40,80}
\usepackage[colorlinks=true,linkcolor=c1]{hyperref}
\usepackage{geometry}
\geometry{a4paper,left=24mm,right=14mm,top=20mm,bottom=28mm}
\setlength{\columnsep}{5mm}

\newcommand{\R}{\mathbb R}
\begin{document}

\noindent
{\huge\bfseries Geraden}

\tableofcontents

\section{Innere Geomerie}
\subsection{Geraden als affine Räume}
Stellen wir uns eine Gerade vor. Die Position dieser Geraden im Raum
ist ja unbedeutsam, und so können wir uns auch eine Gerade als Raum
selbst vorstellen. Die Gerade könnte auch in einen Raum eingebettet
werden. Aber dieser Raum könnte wiederum in einen höherdimensionalen
Raum eingebettet werden. Dieser Prozess läuft ad infinitmum weiter.
Da es keine ausgezeichnete Einbettung gibt, lässt man es am besten
gleich ganz bleiben. Und wenn eine Einbettung doch stattfinden soll,
so muss man eben eine bijektive affine Abbildung von der Geraden zu
ihrer Einbettung angeben und damit hat sich's.

Wie beschreibt man nun die Gerade? Die Gerade kann doch als affiner
Raum angesehen werden. An jedem Punkt lässt sich ein Vektorraum
anheften. Man spricht von einem sogenannten Vektorraumbündel. Durch
jeden dieser Vektorräume ist ein Ursprung festgelegt, so dass die
Gerade zu einer Ursprungsgeraden wird.

Wählt man nun einen Vektorraum und für diesen eine Orthonormalbasis,
so ergibt sich die Zahlengerade. Das ist ein eindimensionales
Koordinatensystem. Jedem Punkt auf der Geraden kann damit bijektiv
eine reelle Zahl zugeordnet werden. Die Bijektion ist nur durch den
Anheftungspunkt $P$ und den Basisvektor $e$ bestimmt. Auf der
Geraden gibt es nur zwei entgegen gesetzte Richtungen. Somit gibt es
auch nur zwei normierte Basisvektoren $v_1,v_2$ mit der
Beziehung $v_1=-v_2$. Man kann $e=v_1$ wählen oder auch $e=v_2$.

Die Bijektion $f$ kann jetzt explizit angegeben werden. Es ist
\begin{equation}
f(x) = P+xe.
\end{equation}
Der Definitionsbereich ist $\R$ und die Zielmenge ist die Gerade.
Wenn $P$ und $e$ in einen höherdimensionalen Raum eingebettet
werden, so auch die Gerade selbst. Bei $e$ handelt es sich dann
um einen normierten Richtungsvektor.

Die Zahlengerade, das ist die Menge der reellen Zahlen $\R$.
Mit der Bijektion können wir nun aber $\R$ selbst als eine
Gerade ansehen. Die lineare Funktion $f(x)=mx$ kann als lineare
Abbildung auf dem Vektorraum $V$ angesehen werden, welcher am
Punkt $P$ angeheftet ist. Es ist die Transformation, die zu einem
Basiswechsel gehört.

Die Addition einer Zahl $n$ entspricht einer Verschiebung,
bei $n$ handelt es sich um einen Verschiebungsvektor. Damit ist
die affine Funktion $f(x)=mx+n$ eine Transformation zwischen zwei
Koordinatensystemen auf dem affinen Raum.

\subsection{Geraden als offene Intervalle}
Die Gerade kann rechnerisch sehr schön durch $\R$ beschrieben
werden. Aber es gibt noch weitere Beschreibungen. Das Intervall
$(0,1)$ beschreibt nämlich ebenso die Gerade. Das klingt seltsam,
da das Intervall doch endlich ist, die Gerade jedoch unendlich weit
reicht. Als Beispiel wählt man z.\,B. die Bijektion
\begin{equation}
f(x) = 1-\frac{1}{\exp(x)+1}.
\end{equation}
Damit wird jeder reellen Zahl eine Zahl auf dem Intervall $(0,1)$
zugeordnet. Mit der Bijektion
\begin{equation}
g(x) = a+(b-a)x
\end{equation}
kann man dem Intervall $(0,1)$ dann jedes andere Intervall
$(a,b)$ zuordnen. Eine Gerade kann also durch ein beliebiges offenes
Intervall beschrieben werden. Da $f$ und $g$ Homöomorphismen sind,
ist eine Gerade sogar homöomorph zu jedem offenen Intervall.

Nun kann man aber mit einem offenen Intervall einen offenen Abschnitt
einer gekrümmten Kurve parametrisieren. Man nimmt dazu am besten
einen Homöomorphismus. Damit kann doch jeder solche Abschnitt als
Gerade aufgefasst werden. Es bietet sich daher an, auch einen Kreis
als Gerade aufzufassen. Dabei ergibt sich jedoch ein Problem: Der
Kreis kann nicht geschlossen werden, ein Punkt muss immer fehlen.
Das ist natürlich hochgradig unästhetisch. Um diesen Mangel zu
beheben, müssen wir ein neues Konzept einführen, den
\emph{projektiven Abschluss}.

\subsection{Projektive Geometrie}

Man kann die reellen Zahlen noch mit den Punkten $-\infty$
und $\infty$ erweitern. Die beiden Punkte macht man nun äquivalent,
d.h. man setzt per Definition $-\infty:=\infty$. Die erweiterte
Gerade wird dabei zu einem Kreis gebogen und bei $-\infty$ und
$\infty$ verklebt. Das Ergebnis ist der projektive Abschluss der
reellen Zahlen. Dieser projektive Raum wird als projektive Gerade
bezeichnet und kann jetzt als Kreis aufgefasst werden. Dabei soll
kein Punkt auf dem Kreis ausgezeichnet sein, so wie kein Punkt auf
dem affinen Raum ausgezeichnet ist. Alle Punkte auf dem Kreis werden
völlig gleich behandelt. Jeder Punkt auf der projektiven Geraden kann
nun mit einem Winkel $\varphi$ auf dem Intervall $[0,2\pi)$
beschrieben werden. Dabei denkt man sich jedoch wieder einen
Winkelursprung, der beliebig gewählt werden kann. Zum Winkel kann
außerdem ein Drehwinkel addiert werden, das entspricht einer Drehung
auf dem Kreis. Der Drehwinkel ist analog zum Verschiebungsvektor im
affinen Raum.

Gibt es auch eine konkrete Bijektion vom Kreis zu
$(\R,\infty)$? Ja die gibt es. Ein Beispiel ist die
stereografische Projektion. Wir nehmen zunächst die Ebene
$\R^2$. Dort sollen die Punkte $x,y$ auf einem Kreis
um den Koordinatenursprung liegen. Das kann man durch die Bijektion
\begin{gather}
x=\cos\varphi,\\
y=\sin\varphi
\end{gather}
erreichen. Dabei ist $\varphi\in{[0,2\pi)}$ der besagte Winkel.
Die stereografische Projektion ist nun die Funktion
\begin{equation}
p(x,y) := \frac{x}{1-y}.
\end{equation}
Für den Fall $\varphi=\pi/2$ definiert man $p(x,y):=\infty$.

Verkettet man die beiden Funktionen, so erhält man
\begin{equation}
p(\varphi) = \frac{\cos\varphi}{1-\sin\varphi}
= \tan\Big(\frac{\varphi}{2}+\frac{\pi}{4}\Big).
\end{equation}

\subsection{Mächtigkeiten}

Nun, wenn wir eine Bijektion von einer Geraden auf eine andere Menge
$M$ angeben können, so kann $M$ als die Gerade selbst aufgefasst
werden. Das heißt jede zu den reellen Zahlen gleichmächtige Menge
kann als Gerade aufgefasst werden. Ein offenes Intervall $I$ ist
aber gleichmächtig zum kartesischen Produkt $I\times I$. Jedes
offene Quadrat ist damit nicht von einer Geraden unterscheidbar.
Zwischen dem offenen Quadrat und $\R\times\R$ gibt
es wieder eine einfache Bijektion. Man nimmt einfach die Bijektion
zwischen $(0,1)$ und $\R$ komponentenweise.

Damit ist eine Gerade dasselbe wie die Ebene. Allgemeiner sind die
reellen Zahlen gleichmächtig zum $\R^n$. Das ist natürlich
eine außerordentlich absonderliche Vorstellung. Man wird nun nach
einer konkreten Konstruktion einer solchen Bijektion fragen, und
da geraten wir in Schwierigkeiten.

Eine Gerade ist das Gleiche wie eine Ebene. Oder nicht?
Nun, eine stärkere Forderung als eine Bijektion ist eine
bijektive lineare Abbildung. Würde es eine solche geben, so müssten
die Ebene und die Gerade die gleiche Dimension haben. Da die Dimension
unterschiedlich ist, kann es eine solche lineare Abbildung
offensichtlich nicht geben. In dieser Hinsicht unterscheidet sich
eine Gerade von einer Ebene. Eine lineare Abbildung wird auch als
Homomorphismus bezeichnet, eine bijektive lineare Abbildung heißt
Isomorphismus.

Unser neues Weltbild geht nun davon aus, dass die Gerade mehr
Struktur hat als nur die Mächtigkeit. Homomorphismen stellen
Struktur"=übertragende Abbildungen dar. Für verschiedene Strukturen
gibt es sogenannte Isomorphismen, welche zwei Mengen in einer
bestimmten Hinsicht als strukturgleich herausstellen. Homöomorphismen
sind z.\,B. Isomorphismen bezüglich der topologischen Struktur.
Bijektive lineare Abbildungen respektieren dagegen die
Vektorraumstruktur. Weiterhin gibt es auch Ordnungsisomorphismen,
welche die Anordnung erhalten. Tatsächlich ist eine bijektive
Abbildung in bestimmter Hinsicht auch ein Isomorphismus.
Eine bijektive Abbildung erhält nämlich die Mächtigkeit.


\subsection{Homöomorphismen}

Anschaulicher als allgemeine Bijektionen sind Homöomorphismen. Man
stellt sich dabei ein offenes Intervall als ein Kaugummi vor, das
auseinander gezogen werden kann. Das verlängerte Kaugummi lässt sich
dann biegen bzw. auf einen Gegenstand aufwickeln. Jedoch wird das
Kaugummi dabei an keiner Stelle zerrissen. Das entspricht eher der
Vorstellung einer Geraden als eine Zersplitterung in unendlich
viele Teile.

Man stellt sich nun alles das als Gerade vor, was homöomorph zum
Intervall $(0,1)$ ist. Wenn man die Gerade um die Punkte
$-\infty$ und $\infty$ erweitert, so ist diese Erweiterung
homöomorph zum Intervall $[0,1]$. Die projektive Gerade ist
homöomorph zum Kreis.

\subsection{Symmetrie}

Wie misst man auf der Zahlengeraden Abstände? Nun, die Punkte null
und eins sollen doch die Entfernung eins haben. Bezeichnen wir
diese Entfernung mit $d$. Es ist also $d(0,1)=1$. Die Zahl zwei
ist anschaulich doppelt so weit entfernt. Also ist $d(0,2)=2$ und
allgemeiner $d(0,x)=x$. Aber wenn $x<0$ ist, so erhält man den
gleichen Abstand. Damit gilt allgemein $d(0,x)=|x|$. Zwischen $a$
und $a+x$ soll der Abstand aber auch $|x|$ betragen. Wir verlangen
also
\begin{equation}
d(a,a+x)=|x|.
\end{equation}
Man definiert nun $b:=a+x$. Dann ist aber $x=b-a$ und somit ergibt sich
\begin{equation}
d(a,b)=|b-a|.
\end{equation}
Auf der Geraden gibt es symmetrische Teilmengen. Beschränken wir uns
dabei auf solche Symmetrien, bei denen Abstände erhalten bleiben.
Solche werden Isometrien genannt. Eine Funktion $f$ auf der Geraden
ist eine Isometrie, wenn
\begin{equation}
d(a,b)=d(f(a),f(b))
\end{equation}
ist. Die Menge aller Isometrien, die ein Objekt hat, ist immer eine
Gruppe. Diese Gruppe nennen wir folglich Isometriegruppe. Man kann nun
nach der Isometriegruppe einer diskreten Teilmenge der Zahlengerade
fragen, ja. Aber man kann auch nach der Isometriegruppe der gesamten
Zahlengerade fragen.

Zunächst betrachten wir die Verschiebung
\begin{equation}
f(x) = x+n.
\end{equation}
Diese Funktion ist eine Isometrie, denn
\begin{gather}
d(a+n,b+n) = |(b+n)-(a+n)|\\
= |b-a| = d(a,b).
\end{gather}
Für $n=0$ erhält man die identische Verschiebung.
Die Verkettung von zwei Verschiebungen ist wieder
eine Verschiebung, denn es ist
\begin{equation}
(x+n_1)+n_2 = x+(n_1+n_2) = x+n_3.
\end{equation}
Damit bilden die Verschiebungen bereits eine Gruppe.
Ist das die Isometriegruppe der Zahlengerade? Nein, denn es gibt
Isometrien, die nicht in der Gruppe der Verschiebungen enthalten
sind. Es ist ja
\begin{equation}
g(x) = -x
\end{equation}
auch eine Isometrie. Man rechnet
\begin{gather}
d(-a,-b) = |(-b)-(-a)|\\
= |a-b| = |b-a| = d(a,b).
\end{gather}
Verkettet man die Negation mit sich selbst, so erhält man die
identische Funktion. Weitere Verkettungen von diesen beiden Funktion
führen nicht aus der Menge heraus, und so erhält man wieder eine
Untergruppe. Die Negation ist die Spiegelung der Geraden am Ursprung.
Da kein Punkt ausgezeichnet ist, soll es solche Spiegelungen auch an
allen anderen Punkten geben. Eine Spiegelung am Punkt $a$
bewerkstelligt die Funktion
\begin{equation}
s_a(x) = -(x-a)+a.
\end{equation}
Diese Funktion ist die Verkettung von zwei Verschiebungen und der
Spiegelung am Ursprung. Eine Verkettung der Spiegelung am Ursprung
und einer Verschiebung reicht jedoch schon aus, denn es ist
\begin{equation}
-(x-a)+a = -x+2a.
\end{equation}
Woher wissen wir, ob wir wirklich alle Isometrien haben?
Wir brauchen die gesamte Lösungsmenge der Funktionalgleichung
\begin{equation}
|f(b)-f(a)| = |b-a|.
\end{equation}
Setzt man zunächst $a=0$, so erhält man
\begin{equation}
|f(b)-f(0)|=|b|.
\end{equation}
Daraus erhält man
\begin{gather}
f(b) = \frac{\mathrm{sgn}(b)}{\mathrm{sgn}(f(b)-f(0))} b+f(0)\\
= s(b,f)b+f(0).
\end{gather}
Damit ergibt sich
\begin{equation}
f(b)-f(a) = s(b,f)b-s(a,f)a.
\end{equation}
Aus der Bedingung
\begin{equation}
|s(b,f)b-s(a,f)a| = |b-a|
\end{equation}
erhält man nun die Forderung
\begin{equation}
s(b,f)=s(a,f).
\end{equation}
Damit ist das Vorzeichen nur von $f$ abhängig. Somit ergibt sich
\begin{equation}
f(b) = s(f)b+f(0).
\end{equation}
Somit lässt sich jede Isometrie in der Form
\begin{equation}
f(x) = sx+c
\end{equation}
darstellen, wobei $s$ ein konstantes Vorzeichen ist und $c$ eine
Konstante ist.

Betrachten wir nochmals die Untergruppe der Verschiebungen. Wenn
$P$ ein Punkt auf der Geraden ist, dann ist $P+n$ ein neuer Punkt
auf der Geraden. Aber $n$ entstammt dabei aus dem Vektorraum
$\R$, welcher ja selbst eine Gruppe ist. Die Verschiebung
\begin{equation}
f(n,P) = P+n
\end{equation}
ist eine Gruppenaktion, denn es ist $f(a+b,P)=f(a,f(b,P))$
und $f(0,P)=P$.

Die Untergruppe der Verschiebungen wird daher durch den
Vektorraum $\R$ abstrahiert.

\section{Äußere Geometrie}
\subsection{Affine Funktionen}

Anschaulich ist uns klar, was eine Gerade ist. Jedoch muss dieser
anschauliche Geradenbegriff erst einmal in eine präzise rechnerische
Form übersetzt werden. Die einfachste Beschreibung von Geraden
erfolgt in einem Koordinatensystem.

Da je eine Gerade durch zwei unterschiedliche Punkte
$(x_1,y_1)$ und $(x_2,y_2)$ verläuft, genügen diese beiden
Punkte, um eine Gerade eindeutig zu spezifizieren. Es stellt
sich nun die Frage wie man überprüft ob ein weiterer Punkt
$(x,y)$ auf der Geraden liegt. Man bezeichnet das als Inzidenztest.

Man denke sich eine Funktion $f(x)$. Die Funktion $f(x)+n$ hat
den gleichen Graphen wie $f(x)$, mit dem Unterschied einer
senkrechten Verschiebung um $n$. Bei der Funktion $f(x)=0$
handelt es sich um die $x$-Achse. Somit ist $g(x)=n$ eine
Gerade, die parallel zur $x$-Achse ist. Mit jeder Zahl $n$ lässt
sich je eine waagerechte Gerade beschreiben.

Wir wollen aber auch Geraden beschreiben, die nicht waagerecht sind.
Zunächst verlangen wir, dass die Gerade durch den Koordinatenursprung $(0,0)$
geht. Als weiterer Punkt ist $(x_2,y_2)$ gegeben. Mit dem Punkt
$(x_2,0)$ zusammen bilden die drei Punkte ein Dreieck. Beim Dreieck
mit den Punkten $(2x_2,0)$ und $(2x_2,2y_2)$ handelt es sich
um das zentrisch gestreckte Dreieck. Dann ist $(2x_2,2y_2)$ doch
aber auch ein Punkt auf der Geraden.

Formal muss man zur Überprüfung die Ähnlichkeitssätze bzw. die
Strahlensätze benutzen. Bei $x_2$ und $y_2$ handelt es sich ja
um die Seitenlängen des Dreiecks.

Allgemein liegt also für verschiedene reelle Zahlen $r$
jeder Punkt $(rx_2,ry_2)$ auf ein und derselben Geraden.
Wenn man nun $x_2\ne 0$ und $r\ne 0$ verlangt, so kann man
den Quotient
\begin{equation}
m = \frac{ry_2}{rx_2} = \frac{y_2}{x_2}
\end{equation}
bilden. Der Quotient $m$ ist nicht von $r$ abhängig.
Durch Umformen erhält man $y_2=mx_2$. Jede Gerade, die nicht
senkrecht ist, kann also durch eine Funktion der Form
\begin{equation}
f(x)=mx
\end{equation}
beschrieben werden. Senkrechte Geraden sind wegen der Forderung
$x_2\ne 0$ nicht möglich. Für je zwei verschiedene Quotienten
$m_1$ und $m_2$ erhält man außerdem unterschiedliche Geraden,
da der Quotient ja nicht vom Punkt auf der Geraden abhängig ist.

Umgekehrt wird auch durch jede
Funktion $y=mx$ eine Gerade beschrieben. Multipliziert man auf
beiden Seiten mit $r$, so erhält man $ry=mrx$. Und von den Punkten
$(x,y)$ und $(rx,ry)$ wissen wir, dass sie auf einer Geraden liegen.

Nun kann man eine solche Gerade auch senkrecht verschieben. Allgemein
wird eine Gerade also durch die affine Funktion
\begin{equation}
f(x) = mx+n
\end{equation}
beschrieben. Zum Test, ob ein Punkt $(x,y)$ auf der Geraden liegt,
überprüft man einfach die Gleichung $y=f(x)$. Der Punkt liegt
genau dann auf der Geraden, wenn diese Gleichung wahr ist. Der
Inzidenztest ist bei der Beschreibung mit affinen Funktionen also
außerordentlich trivial, was besonders schön ist.

Aber wie kommt man nun von zwei gegebenen Punkten $(x_1,y_2)$ und
$(x_2,y_2)$ auf die Zahlen $m$ und $n$? Die Gleichung der
Funktion ist $y=mx+n$. Zunächst beobachtet man, dass man das $n$
loswerden kann, indem man die Differenz von zwei Funktionswerten
bildet. Es ist
\begin{gather}
y_2-y_1 = mx_2+n - mx_1-n\\
= m(x_2-x_1).
\end{gather}
Da die Punkte unterschiedlich sein müssen und die Gerade nicht
senkrecht sein darf, ist $x_2-x_1\ne 0$. Durch Division
auf beiden Seiten erhält man den Quotient
\begin{equation}
m = \frac{y_2-y_1}{x_2-x_1}.
\end{equation}
Das $n$ bekommt nun durch Umformung einer der beiden Gleichungen
$y_1=mx_1+n$ und $y_2=mx_2+n$. Wählen wir die erste Gleichung.
Somit ist $n=y_1-mx_1$. Einsetzen in die Funktionsgleichung bringt
\begin{gather}
y = mx+n = mx+y_1-mx_1\\
= m(x-x_1)+y_1.
\end{gather}
Die Funktionsgleichung kann damit sofort in der Form
\begin{equation}
y = \frac{y_2-y_1}{x_2-x_1}(x-x_1)+y_1
\end{equation}
angegeben werden.

\subsection{Implizite Form}

Dass senkrechte Geraden nicht beschrieben werden können, ist ein
Mangel. Dieses Problem kann gelöst werden. Umformung der
Funktionsgleichung bringt
\begin{equation}
y-y_1 = \frac{y_2-y_1}{x_2-x_1}(x-x_1).
\end{equation}
Durch Multiplikation erhält man
\begin{equation}
(y-y_1)(x_2-x_1) = (y_2-y_1)(x-x_1).
\end{equation}
Eine Division durch null kommt nicht mehr vor, da kein Quotient
gebildet wird. Tatsächlich wird damit auch das Problem der
senkrechten Geraden gelöst. Diese implizite Form der Beschreibung
kann man daher als elegant betrachten. Der Nachteil ist, dass keine
explizite Funktionsgleichung vorliegt.

Die explizite Gleichung wird für einen Inzidenztest jedoch auch
nicht benötigt, und damit ist der Nachteil kein wirklicher.
Außerdem sind $x$ und $y$ jetzt völlig gleichberechtigt. Man
könnte auch nach $x$ umstellen und so eine Funktion $x=g(y)$
erhalten.

Sei $a:=y_2-y_1$ und $b:=x_1-x_2$. Durch Ausmultiplizieren
erhält man
\begin{equation}
by_1-by=ax-ax_1.
\end{equation}
Umformen bringt
\begin{equation}
ax+by=ax_1+by_1.
\end{equation}
Sei noch $c:=ax_1+by_1$. Damit erhält man
\begin{equation}
ax+by=c.
\end{equation}
Da diese Gleichung ohne weitere Voraussetzungen nur durch Umbenennung,
Ausmultiplizieren und
Äquivalenzumformung hervorgebracht wurde, kann jede Gerade auch
in dieser Form
beschrieben werden. Es können nie Gleichzeitig $a$
und $b$ null sein. Die Darstellung ist außerdem nicht eindeutig.
Man kann beide Seiten der Gleichung mit einer Zahl außer null
multiplizieren, ohne dass sich der Aussagegehalt der Gleichung
ändert. Man wird je nach Situation $a$ oder $b$ auf eins
normieren. Nach Umformung ergibt sich dann wieder die explizite
Form.

Mit der impliziten Form ist die Beschreibung von Geraden im Raum
nicht möglich. Stattdessen müsste man Ebenen im Raum
in der Form
\begin{equation}
ax+by+cz=d
\end{equation}
beschreiben und eine Gerade als Schnittmenge von zwei Ebenen
formulieren. Die vektorielle Beschreibung ist etwas einfacher
handhabbar.

\subsection{Vektorielle Form}

Ein Ort kann durch den Ortsvektor $b$ beschrieben werden.
Addiert man zu $b$ einen Richtungsvektor $a$,
so gelangt man zu einem neuen Ort. Der Richtungsvektor kann auch
vorher skaliert werden. Zeichnet man nun den Ort bezüglich allen
Skalierungen ein, so liegen die Punkte auf einer Geraden.
Eine Gerade wird also beschrieben
durch die Parameterkurve
\begin{equation}
g(t) = at+b
\end{equation}
mit dem Parameter $t$. Die Gerade ist dann einfach die Bildmenge
dieser Parameterkurve. In der Ebene lässt sich die Gleichung
der Geraden in die zwei Gleichungen
\begin{equation}
\begin{split}
x &= a_1 t+b_1,\\
y &= a_2 t+b_2
\end{split}
\end{equation}
aufspalten. Falls $a_1\ne 0$ ist, bringt Umformung der ersten
Gleichung nun $t = (x-b_1)/a_1$.
Damit ergibt sich
\begin{equation}
y = \frac{a_2}{a_1}(x-b_1)+b_2.
\end{equation}
Die Beschreibung in der expliziten Form ist also als Spezialfall
in der vektoriellen Form enthalten.

Mit der vektoriellen Form kann man Geraden in Räumen beliebiger
endlicher Dimension beschreiben. Das wird jedoch dadurch erkauft,
dass der Inzidenztest nun etwas komplizierter ist.

\subsection{Geraden als Äquivalenzklassen}

Angenommen man will überprüfen, ob zwei Geraden $g_1,g_2$ gleich sind.
Dann müsste man überprüfen ob jeder Punkt aus $g_1$ in $g_2$
enthalten ist und jeder Punkt aus $g_2$ in $g_1$ enthalten ist.
Das ist insofern problematisch, dass eine Gerade aus unendlich vielen
Punkten besteht.

Aus dieser Misere kommt man heraus, indem man die Geraden als
Äquivalenzklassen betrachtet, nachdem man den möglichen Schnittpunkt
entfernt hat. Man nimmt nun zwei unterschiedliche Repräsentanten
$p_1,p_2$ aus $g_1$ und überprüft ob beide auch Repräsentanten
von $g_2$ sind. Schon dann weiß man, dass beide Geraden gleich
sein müssen. Ein einziger Repräsentant würde nicht ausreichen, denn
es könnte zufällig der möglicherweise vorhandene Schnittpunkt sein.

Betrachten wir z.\,B. die Ursprungsgeraden, wobei wir uns den
Ursprung als problematischen Schnittpunkt wegdenken. Jede Gerade
mit Ausnahme der senkrechten wird durch genau eine Funktion
$f(x)=mx$ bzw. durch genau ein $m$ angegeben.
Will man nun überprüfen ob zwei Ursprungsgeraden $g_1,g_2$ gleich
sind, so kann stattdessen auch überprüfen, ob $m_1=m_2$ ist.

Da für $m$ keine Beschränkungen gelten, gibt es zu jeder
Ursprungsgeraden mit Ausnehme der senkrechten genau eine reelle Zahl.
Nimmt man noch den projektiven Abschluss $m=\infty$ hinzu, so
gibt es keine Ausnahme. Die Menge der Ursprungsgeraden ist also
gewissermaßen dasselbe wie der Kreis. Darauf kann man auch einfacher
kommen: zu jeder Ursprungsgeraden gibt es genau eine Richtung.


\subsection{Geodäten}

In einem flachen Raum sind Geraden die kürzeste Verbindung zwischen
zwei Punkten. Diese Eigenschaft ist dazu geeignet den Begriff der
Geraden auf Räume zu übertragen, wo es eigentlich keine Geraden
gibt. Befindet man sich z.\,B. in einem Labyrinth, so können alle
Strecken als gerade aufgefasst werden, die die kürzeste Verbindung
zwischen zwei Orten im Labyrinth darstellen. Die Geraden auf der
Kugeloberfläche sind Großkreise. Gerade im Kleinen stimmen die
Großkreise der Erdoberfläche sehr gut mit Geraden überein,
z.\,B. besser als ein Kreis mit dem Radius $1\,\mathrm{km}$.
Das rechtfertigt den Begriff kürzesten Verbindung als Gerade.

Solche Verallgemeinerungen von Geraden und Strecken werden als
\emph{Geodäten} bezeichnet. Dieses Konzept lässt sich in einem
beliegen metrischen Raum formulieren. Und die Menge der metrischen
Räume ist sehr groß, auch in praktischer Hinsicht.

\vfill\noindent
Dieser Text steht unter der Lizenz\\
Creative Commons CC0 1.0.

\end{document}



