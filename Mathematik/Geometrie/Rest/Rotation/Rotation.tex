\documentclass[a4paper,fleqn,11pt]{article}
\usepackage[utf8]{inputenc}
\usepackage[T1]{fontenc}
\usepackage[ngerman]{babel}
\usepackage{amsmath}
\usepackage{amssymb}
\usepackage{amsthm}

\usepackage{mdframed}
\usepackage{lipsum}
%\usepackage{microtype}

\usepackage{lmodern}
%\usepackage{arev}
%\usepackage{libertine}
%\usepackage[libertine,cmintegrals]{newtxmath}
%\renewcommand\ttdefault{lmvtt}

\usepackage{geometry}
\geometry{a4paper,left=34mm,right=34mm,top=38mm,bottom=48mm}

\usepackage{color}
\definecolor{c1}{RGB}{0,40,80}
\definecolor{gray1}{RGB}{80,80,80}
\usepackage[colorlinks=true,linkcolor=c1]{hyperref}

\newcommand{\strong}[1]{\textsf{\textbf{#1}}}

\newtheoremstyle{rmbox}%
  {0pt}% space above
  {0pt}% space below
  {}% bodyfont
  {}% indent
  {\bfseries}% head font
  {\\[2pt]}% punctuation between head and body
  {0pt}% space after theorem head
  {\thmname{#1}\;\thmnumber{#2}.\;\thmnote{#3.}}

\theoremstyle{rmbox}
\newtheorem{definition}{Definition}
\newtheorem{theorem}{Satz}
\newtheorem{lemma}[theorem]{Lemma}
\newtheorem{corollary}[theorem]{Korollar}
\newtheorem{Aufgabe}{Aufgabe}

\definecolor{greenblue}{rgb}{0.0,0.32,0.4}
\definecolor{grayblue}{rgb}{0.2,0.2,0.4}

\surroundwithmdframed[topline=false,rightline=false,bottomline=false,%
  linecolor=greenblue, linewidth=4pt, innerleftmargin=7pt,%
  innertopmargin=2pt, innerbottommargin=4pt,%
  innerrightmargin=0pt%
]{definition}

\newcommand{\framedtheorem}[1]{%
\surroundwithmdframed[topline=false,rightline=false,bottomline=false,%
  linecolor=grayblue, linewidth=3.5pt, innerleftmargin=6pt,%
  innertopmargin=2pt, innerbottommargin=6pt,%
  innerrightmargin=0pt%
]{#1}}

\framedtheorem{theorem}
\framedtheorem{lemma}
\framedtheorem{corollary}
\framedtheorem{Aufgabe}

\newcommand{\N}{\mathbb N}
\newcommand{\Z}{\mathbb Z}
\newcommand{\R}{\mathbb R}
\newcommand{\C}{\mathbb C}
\newcommand{\ui}{\mathrm i}
\newcommand{\ee}{\mathrm e}
\newcommand{\defiff}{\;:\Longleftrightarrow\;}

\DeclareMathOperator{\id}{id}
\DeclareMathOperator{\sur}{sur}
\DeclareMathOperator{\real}{Re}
\DeclareMathOperator{\imag}{Im}

\begin{document}
\pagestyle{empty}

\section*{Die Rotation im Raum mittels Clifford-Algebra}

Im Folgenden wird die folgende Notation benutzt:
\begin{align*}
\langle v,w\rangle &\qquad \text{Standardskalarprodukt},\\
v\wedge w &\qquad \text{äußeres Produkt},\\
vw &\qquad \text{Produkt der Clifford-Algebra}.
\end{align*}
Der Vektor $\vec r_0$ soll um die durch den normierten Vektor $\vec n$
gegebene Achse rotiert werden. Die Rotation um den Winkel $\varphi$
lässt sich berechnen gemäß
\begin{equation}\label{eq:Rotation}
\vec r(\varphi) = R\vec r_0\tilde R,\quad R=\ee^{-\hat B\varphi/2},\quad\tilde R=\ee^{\hat B\varphi/2}.
\end{equation}
Hierbei ist $\hat B$ der Einheitsbivektor der Ebene orthogonal zu
$\vec n$. Mittels Hodge"=Stern"=Operator gilt dann $\hat B=*\vec n$.
Die Clifford"=Algebra"=Elemente $R$ und $\tilde R$
werden \emph{Rotoren} genannt, und $\tilde R$ ist die \emph{Reversion}
von $R$.

Allgemein für den $\R^n$ gilt
\begin{equation}\label{eq:Pseudoskalar-Vektor}
*\vec v = \vec vI_n = (-1)^{n-1}I_n\vec v,
\end{equation}
wobei $I_n = e_1e_2\ldots e_n$ der Pseudoskalar ist.

Im $\R^3$ gilt also $\vec vI_3=I_3\vec v$. Wir schreiben
ab jetzt kurz $I:=I_3$.

Somit ist $\hat B = \vec nI = I\vec n$.

Es gilt die Verallgemeinerung der eulerschen Formel:
\begin{equation}
\ee^{\hat B x} = \cos x +\hat B\sin x.
\end{equation}
Es ergibt sich
\begin{gather}
R\vec r_0\tilde R = (\cos\tfrac{\varphi}{2}-\hat B\sin\tfrac{\varphi}{2})
\vec r_0(\cos\tfrac{\varphi}{2}+\hat B\sin\tfrac{\varphi}{2})\\
= (\cos\tfrac{\varphi}{2}-\hat B\sin\tfrac{\varphi}{2})
(\vec r_0\cos\tfrac{\varphi}{2}+\vec r_0\hat B\sin\tfrac{\varphi}{2})\\
= \vec r_0\cos^2\tfrac{\varphi}{2}-\hat B\vec r_0\hat B\sin^2\tfrac{\varphi}{2}
+ (\vec r_0\hat B-\hat B\vec r_0)\cos\tfrac{\varphi}{2}\sin\tfrac{\varphi}{2}\\
= \vec r_0\cos^2\tfrac{\varphi}{2}-\vec n I\vec r_0\vec n I\sin^2\tfrac{\varphi}{2}
+(\vec r_0\vec n I -\vec n \vec r_0 I)\cos\tfrac{\varphi}{2}\sin\tfrac{\varphi}{2}\\
= \vec r_0\cos^2\tfrac{\varphi}{2}-\vec n \vec r_0\vec n I^2\sin^2\tfrac{\varphi}{2}
+(\vec r_0\vec n-\vec n \vec r_0)I\cos\tfrac{\varphi}{2}\sin\tfrac{\varphi}{2}.
\end{gather}
Aus $ab = \langle a,b\rangle+a\wedge b$ folgt nun aber
\begin{equation}
ab-ba = 2a\wedge b,\qquad ab+ba = 2\langle a,b\rangle.
\end{equation}
Daher gilt
\begin{equation}
(\vec r_0\vec n-\vec n\vec r_0)I = 2(\vec r_0\wedge\vec n)I
= -2\vec r_0\times\vec n = 2\vec n\times\vec r_0
\end{equation}
und
\begin{equation}
\vec n \vec r_0\vec n = \vec n(2\langle\vec r_0,\vec n\rangle-\vec n\vec r_0)
= 2\langle\vec r_0,\vec n\rangle\vec n - \vec n\vec n\vec r_0
= 2\langle\vec r_0,\vec n\rangle\vec n - \langle\vec n,\vec n\rangle\vec r_0,
\end{equation}
wobei $\langle\vec n,\vec n\rangle=1$.

Man beachtet nun $I^2=-1$. Außerdem gilt
\begin{align}
\cos^2(\varphi/2)&=\tfrac{1}{2}(1+\cos\varphi),\\
\sin^2(\varphi/2)&=\tfrac{1}{2}(1-\cos\varphi),\\
\cos(\varphi/2)\sin(\varphi/2)&=\tfrac{1}{2}\sin\varphi.
\end{align}
Schließlich ergibt sich
\begin{align}
R\vec r_0\tilde R
&= \vec r_0\cos^2\tfrac{\varphi}{2}+(2\langle\vec r_0,\vec n\rangle\vec n - \vec r_0)\sin^2\tfrac{\varphi}{2}
+\vec n\times\vec r_0\sin\varphi\\
&= \tfrac{\vec r_0}{2}(1+\cos\varphi)-\tfrac{\vec r_0}{2}(1-\cos\varphi)
+(1-\cos\varphi)\langle\vec r_0,\vec n\rangle\vec n
+\vec n\times\vec r_0\sin\varphi\\
&= \vec r_0\cos\varphi
+(1-\cos\varphi)\langle\vec r_0,\vec n\rangle\vec n
+\vec n\times\vec r_0\sin\varphi.
\end{align}

%\newpage

\noindent
Der Vorteil von Formel \eqref{eq:Rotation} ist, dass hiermit
Rotationen im $\R^n$ auch für $n\ne 3$ beschrieben werden können.
Im $\R^2$ kann man den Pseudoskalar $I_2=e_1e_2$ mit der
imaginären Einheit identifizieren. Gemäß \eqref{eq:Pseudoskalar-Vektor}
gilt $I_2\vec v=-\vec vI_2$ und daher
\begin{equation}\label{eq:komplexe-Zahl-Vektor}
z\vec v = (a+bI_2)\vec v = a\vec v+bI_2\vec v
= \vec v a -\vec v\,bI_2 = \vec v(a-bI_2) = \vec v\,\overline z.
\end{equation}
In der Ebene ist $\hat B=I_2=\ui$. Mit \eqref{eq:komplexe-Zahl-Vektor}
erhält man
\begin{equation}
R\vec v\tilde R = \ee^{-\ui\varphi/2}\vec v\,\ee^{\ui\varphi/2}
= \ee^{-\ui\varphi/2}\ee^{-\ui/\varphi/2}\vec v
= \ee^{-\ui\varphi}\vec v.
\end{equation}
Die Anwendung einer komplexen Zahl auf einen Vektor ergibt
\begin{align}
(a+b\ui)\vec v &= (a+be_1e_2)\vec v = (a+be_1e_2)(v_1e_1+v_2e_2)\\
&= av_1 e_1 + av_2 e_2 - bv_1 e_2 + bv_2 e_1\\
&= (av_1+bv_2)e_1+(-bv_1+av_2)e_2\\
&= \begin{bmatrix}
av_1+bv_2\\
-bv_1+av_2
\end{bmatrix}
= \begin{bmatrix}
a & b\\
-b & a
\end{bmatrix}\begin{bmatrix}
v_1\\ v_2
\end{bmatrix}.
\end{align}
Somit ist
\begin{equation}
\ee^{-\varphi\ui}\vec v = \begin{bmatrix}
\cos\varphi & -\sin\varphi\\
\sin\varphi & \cos\varphi
\end{bmatrix}\vec v.
\end{equation}
Speziell gilt
\begin{equation}
(-\ui)\vec v = \begin{bmatrix}
0 & -1\\
1 & 0
\end{bmatrix}\vec v.
\end{equation}

\end{document}


