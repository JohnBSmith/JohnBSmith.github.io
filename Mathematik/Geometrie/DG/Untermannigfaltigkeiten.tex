
\chapter{Untermannigfaltigkeiten des euklidischen Raums}

\section{Skalarfelder}

\subsection{Die Richtungsableitung}

Sei $U\subseteq\R^n$ eine offene Menge. Sei außerdem $p\in U$ 
eine Stelle und $v\in\R^n$ ein Vektor. Man betrachte nun die Gerade
\begin{equation}\label{eq:Gerade-zum-Vektor}
G := \{p+tv\mid t\in\R\}.
\end{equation}
Die Einschränkung von $f$ auf $U\cap G$ ist
\begin{equation}
g\colon U\cap G\to\R,\quad g(t):=f(p+tv).
\end{equation}
Für $g$ ist die gewöhnliche Ableitung definiert, da $g$ eine reelle
Funktion in einer Variablen ist. Wir nennen $g'(0)$ die
\emph{Richtungsableitung} von $f$ an der Stelle $p$ in Richtung $v$.
Es ergibt sich%
\begin{equation}\label{eq:Richtungsableitung-flach}
(\mathrm d_p f)(v) := g'(0) = \lim_{h\to 0}\frac{g(h)-g(0)}{h}
= \lim_{h\to 0}\frac{f(p+hv)-f(p)}{h}.
\end{equation}
Wir betrachten nun ein Skalarfeld $f\colon M\to\R$, welches
auf einer Untermannigfaltigkeit $M$ definiert ist.
Die Richtungsableitung lässt sich nun aber nicht mehr gemäß
\eqref{eq:Richtungsableitung-flach} bestimmen, weil die Gerade
\eqref{eq:Gerade-zum-Vektor} nicht innerhalb von $M$ liegen muss.
Außerhalb von $M$ ist das Skalarfeld nicht definiert, der Ausdruck
\eqref{eq:Richtungsableitung-flach} setzt dies aber voraus.

Dieses Problem lässt sich wie folgt lösen. Sei $p\in M$ ein Punkt
auf $M$. Sei $c\colon I\to M$ eine Kurve in $M$ mit
$p=c(0)$ und $v=c'(0)$. Damit das überhaupt möglich ist, muss
der Vektor $v\in T_p M$, d.\,h. im Punkt $p$ tangential an $M$ sein.
Nun wird ähnlich wie zuvor die Einschränkung des
Skalarfeldes $f$ auf den kleineren Definitionsbereich $c(I)$
betrachtet. Diese Einschränkung lässt sich als
$g\colon I\to\R$ mit $g:=f\circ c$ angeben. Für $g$ ist die
gewöhnliche Ableitung definiert. Die Richtungsableitung lässt sich
also gemäß $g'(0)=(f\circ c)'(0)$ bestimmen.

\begin{definition}[Richtungsableitung]
Sei $M$ eine Untermannigfaltigkeit und $f\colon M\to\R$ ein Skalarfeld.
Sei $p\in M$ ein Punkt und $v$ ein Vektor, welcher am Punkt $p$
tangential an $M$ ist. Sei $c\colon I\to M$ eine glatte Kurve mit
$p=c(0)$ und $v=c'(0)$. Unter der \emdef{Richtungsableitung} von $f$ am
Punkt $p$ in Richtung $v$ versteht man die reelle Zahl
\begin{equation}
(\mathrm d_p f)(v) := (f\circ c)'(0) = \lim_{h\to 0}\frac{f(c(h))-f(p)}{h}.
\end{equation}
\end{definition}



\section{Vektorfelder}

\subsection{Die kovariante Ableitung}

Sei $M$ eine Untermannigfaltigkeit des $\R^n$ und $X\colon M\to TM$
ein Vektorfeld. Nun kann doch wie bei einem Skalarfeld die
Richtungsableitung von $X$ in Richtung
eines Vektors $v\in T_p M$ definiert werden. Sei dazu $c\colon I\to M$
eine Kurve mit $p=c(0)$ und $v=c'(0)$. Dann definiert man%
\begin{equation}
(\mathrm d_p X)(v) := (X\circ c)'(0).
\end{equation}
Nun taucht das Problem auf, dass die Ableitung an der Stelle $p$
ein Vektor ist, welcher nicht unbedingt tangential an $M$ sein muss.
Ein Ziel der Differentialgeometrie ist es aber, eine Theorie aufzubauen,
welche nur auf Tangentialräumen beruht. Aus diesem Grund projizieren
wir den Vektor $w=(\mathrm d_p X)(v)$ orthogonal auf den Tangentialraum
$T_p M$. Der zum Tangentialraum orthogonale Anteil entfällt dabei.

Die orthogonale Projektion auf $T_p M$ nennen wir $\Pi_p$. Es handelt
sich um eine lineare Abbildung.

\begin{definition}[Kovariante Ableitung]
Die \emdef{kovariante Ableitung} eines Vektorfeldes $X\colon M\to TM$
an der Stelle $p\in M$ in Richtung $v\in T_p M$ ist definiert als
\begin{equation}
(\nabla_v X)(p) := \Pi_p((X\circ c)'(0)),
\end{equation}
wobei $c\colon I\to M$ eine glatte Kurve mit $p=c(0)$ und $v=c'(0)$ ist.
\end{definition}

\noindent
Natürlich kann auch $v=Y(p)$ sein, wobei $Y$ ein zweites Vektorfeld
ist. Man notiert dazu
\begin{equation}
(\nabla_Y X)(p) := (\nabla_{Y(p)} X)(p).
\end{equation}

\noindent
Wir wollen nun eine Formel für die Richtungsableitung herleiten,
wenn das Vektorfeld in lokalen Koordinaten dargestellt ist. Sei
dazu $\varphi\colon U\to V$ mit $V\subseteq M$ eine lokale
Parametrisierung. Diese induziert den Rahmen $(g_k)$ mit
$g_k = \frac{\partial\varphi}{\partial u_k}$. Die lokale
Darstellung $\tilde X$ des Vektorfeldes $X$ sei gemäß den
Funktionen $a^k(u)$ gegeben, so dass%
\begin{equation}
\tilde X(u) = (X\circ\varphi)(u) = \sum_{k=1}^m a^k(u)g_k(u).
\end{equation}
An jedem Punkt $p=\varphi(u)$ ist das Vektorfeld also als
Linearkombination aus der Tangentialbasis an diesem Punkt dargestellt.
Der Vektor $v$ sei am Punkt $p=\varphi(u_0)$ ebenfalls als
Linearkombination aus der Tangentialbasis dargestellt:
$v = \sum_{k=1}^m v^k g_k(u_0)$.

Nun sei $\tilde c$ die lokale Darstellung der Kurve, gemäß
$c=\varphi\circ\tilde c$. Es ergibt sich
\begin{equation}
(\nabla_v X)(p) = \Pi_p((X\circ c)'(0))
= \Pi_p((X\circ\varphi\circ\tilde c)'(0))
= \Pi_p((\tilde X\circ\tilde c)'(0)).
\end{equation}
Nach der Produktregel ergibt sich
\begin{gather}
(\tilde X\circ\tilde c)'(t)
= \frac{\mathrm d}{\mathrm dt} \sum_{i=1}^m a^i g_i
= \sum_{i=1}^m \frac{\mathrm da^i}{\mathrm dt}g_i
+ \sum_{i=1}^m a^i\frac{\mathrm dg_i}{\mathrm dt}.
\end{gather}
Anwendung der Kettenregel bringt nun
\begin{equation}
\frac{\mathrm da^i}{\mathrm dt}
= \sum_{j=1}^m \frac{\partial a^i}{\partial u_j}
\frac{\mathrm d\tilde c_j}{\mathrm dt},\qquad
\frac{\mathrm dg^i}{\mathrm dt}
= \sum_{j=1}^m \frac{\partial g_i}{\partial u_j}
\frac{\mathrm d\tilde c_j}{\mathrm dt}.
\end{equation}
Nach der Kettenregel und $c(0)=\varphi(u_0)$ ergibt sich aber auch
\begin{equation}
c'(0) = (\tilde c\circ\varphi)'(0)
= \sum_{j=1}^m \tilde c_j'(0)(\partial_j\varphi)(u_0)
= \sum_{j=1}^m \tilde c_j'(0)g_j(u_0)
= \sum_{j=1}^m v^j g_j(u_0).
\end{equation}
Der Koeffizientenvergleich ergibt $\tilde c_j'(0)=v^j$. Demnach ergibt sich
\begin{equation}
(\tilde X\circ\tilde c)'(0)
= \sum_{i,j} (\partial_j a^i)(u_0) v^j g_i(u_0)
+ \sum_{i,j} a^i(u_0) (\partial_j g_i)(u_0) v^j.
\end{equation}
Wir nutzen nun aus, dass die Projektion eine lineare Abbildung ist.
Die linke Seite ist eine Linearkombination aus der Tangentialbasis,
liegt also schon im Tangentialraum. Auf die linke Seite ist die
Projektion daher wirkungslos. Somit ergibt sich
\begin{equation}\label{eq:kov-Ableitung1}
(\nabla_v X)(p) = \sum_{i,j} (\partial_j a^i)(u_0) v^j g_i(u_0)
+ \sum_{i,j} v^j a^i(u_0) \Pi_p((\partial_j g_i)(u_0)).
\end{equation}
Die übrig gebliebene Projektion stellen wir als Linearkombination
aus der Tangentialbasis dar. Die Koeffizienten $\Gamma_{ij}^k$
sind an der Stelle $u_0$ also durch%
\begin{equation}
\Pi_p((\partial_j g_i)(u_0))
= \Pi_p((\partial_i\partial_j\varphi)(u_0))
= \sum_{k=1}^m \Gamma_{ij}^k g_k(u_0)
\end{equation}
gegeben. Die $\Gamma_{ij}^k(u_0)$ nennt man \emph{Christoffel-Symbole}.
Auf der linken Seite von \eqref{eq:kov-Ableitung1} machen wir eine
Indexumbenennung $i:=k$. Es ergibt sich schließlich
\begin{equation}
(\nabla_v X)(p)
= \sum_{k=1}^m \bigg(\sum_{j=1}^m(\partial_j a^k)(u_0)v^j
+\sum_{i=1}^m\sum_{j=1}^m \Gamma_{ij}^k(u_0) v^j a^i(u_0)\bigg)g_k(u_0).
\end{equation}
Kurz
\begin{equation}
\nabla_v X = \sum_k \bigg(\sum_j v^j\partial_j a^k
+\sum_{i,j} \Gamma_{ij}^k v^j a^i\bigg)g_k.
\end{equation}
Oder kürzer $(\nabla_v X)_k = v^j \partial_j a^k + \Gamma_{ij}^k v^j a^i$.
