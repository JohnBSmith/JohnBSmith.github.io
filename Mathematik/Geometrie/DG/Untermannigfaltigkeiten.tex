
\chapter{Untermannigfaltigkeiten des Koordinatenraums}

\section{Grundlagen}

\subsection{Lokale Karten}

Sei $\varphi\colon U\to\R^m$ mit $U\subseteq\R^n$ eine lokale Karte.
Man definiert die Richtungsableitung von $\varphi$ als
\begin{equation}
\mathrm d\varphi_u(v) = (\mathrm d\varphi)(u)(v)
:= \lim_{h\to 0}\frac{\varphi(u+hv)-\varphi(u)}{h}.
\end{equation}
Im Folgenden wird der Begriff der totalen Differenzierbarkeit
aus der mehrdimensionalen Analysis als bekannt vorausgesetzt.
Ist $\varphi$ total differenzierbar, dann lässt sich
$\mathrm d\varphi_u$ als lineare Abbildung aus $\hom(\R^n,\R^m)$
darstellen. Aufgrund der kanonischen Isomorphie zwischen
$\hom(\R^n,\R^m)$ und dem Matrizenraum $\R^{m\times n}$ gibt
es für jede lineare Abbildung genau eine Darstellungsmatrix,
wobei für $\R^n$ und $\R^m$ die kanonische Basis zu wählen ist.
Dieser Zusammenhang ist so eindrücklich, dass wir die lineare
Abbildung zwischen Koordinatenräumen mit ihrer Darstellungsmatrix
identifizieren könnten.

Die kanonische Darstellungsmatrix von $\mathrm d\varphi_u$ ist die
Jacobi"=Matrix $J=D\varphi_u$. Für einen Vektor $v\in\R^n$ mit
$v=\sum_{j=1}^n v_j \mathbf e_j$ gilt dann
\begin{equation}
\mathrm d\varphi_u(v) = Jv
= \sum_{j=1}^n \frac{\partial\varphi}{\partial u^j}(u) v^j
= \sum_{i=1}^m\sum_{j=1}^n \frac{\partial\varphi^i}{\partial u^j}(u) v^j\mathbf e_i
= \sum_{i=1}^m\sum_{j=1}^n J_{ij} v^j\mathbf e_i.
\end{equation}

\subsection{Tangentialräume}

Sei $M$ eine Untermannigfaltigkeit, $p\in M$ ein Punkt und
$c\colon (-\varepsilon,\varepsilon)\to M$ eine differenzierbare Kurve
mit $p=c(0)$. Da $c'(0)$ der Tangentialvektor der Kurve am Punkt $p$
ist und die Kurve in $M$ liegt, muss $c'(0)$ auch ein Tangentialvektor
von $M$ sein. Die Menge aller Tangentialvektoren, die sich auf
diese Art am Punkt $p$ bilden lassen, nennt man Tangentialraum $T_p M$.

\begin{theorem}
Sei $M$ eine $n$"=dimensionale Untermannigfaltigkeit des $\R^m$.
Der Tangentialraum $T_p M$ ist ein $n$"=dimensionaler Untervektorraum
des $\R^m$. Sei $\varphi\colon U\to V$ mit $U\subseteq\R^n$ und
$V\subseteq M$ eine lokale Karte von $M$.  Es gilt
$T_p M = \Bild(\mathrm d\varphi_u)$, wobei $p=\varphi(u)$. 
\end{theorem}

\noindent\strong{Beweis.}
Zur Kurve $c$ in $M$ gehört genau die Kurve $\tilde c$ im $\R^n$, so
dass $c=\varphi\circ\tilde c$. Sei $u=\tilde c(0)$. Nach der
Kettenregel gilt%
\begin{equation}
c'(0) = (\varphi\circ\tilde c)'(0) = \mathrm d\varphi_u\tilde c'(0),
\end{equation}
Zu jedem Vektor $v\in\R^n$ lässt sich eine Kurve $\tilde c$
mit $u=\tilde c(0)$ und $v=\tilde c'(0)$ finden. Demnach gilt
\begin{equation}
T_p M = \mathrm d\varphi_u(\R^n) = \Bild(\mathrm d\varphi_u).
\end{equation}
Weil $\mathrm d\varphi_u$ eine injektive lineare Abbildung ist,
gilt nach dem Dimensionssatz%
\begin{equation}
\dim\Bild(d\varphi_u) = \dim(\R^n) = n.\;\qedsymbol
\end{equation}
Da $\mathrm d\varphi_u$ injektiv ist, wird einer Basis wieder eine
Basis zugeordnet. Nimmt man die Standardbasis
$(\mathbf e_k)=(\mathbf e_1,\ldots,\mathbf e_n)$, dann ergibt
sich für $T_p M$ die Basis $(g_k)$ mit
\begin{equation}
g_k(u) = \mathrm d\varphi_u(\mathbf e_k)
= \frac{\partial\varphi}{\partial u^k}(u),\quad\text{wobei}\; p=\varphi(u).
\end{equation}


\section{Skalarfelder}

\subsection{Die Richtungsableitung}

Sei $U\subseteq\R^n$ eine offene Menge. Sei außerdem $p\in U$ 
eine Stelle und $v\in\R^n$ ein Vektor. Man betrachte die Gerade
\begin{equation}\label{eq:Gerade-zum-Vektor}
G := \{p+tv\mid t\in\R\}.
\end{equation}
Das Skalarfeld $f$ wollen wir nun auf den Streifen $U\cap G$
einschränken. Parametrisiert man diesen Streifen um $p$ herum durch
$t$, dann ergibt sich die Funktion%
\begin{equation}
g\colon (-\varepsilon,\varepsilon)\to\R,\quad g(t):=f(p+tv).
\end{equation}
Für $g$ ist die gewöhnliche Ableitung definiert, da $g$
eine reelle Funktion in einer Variablen ist. Wir nennen $g'(0)$ die
\emph{Richtungsableitung} von $f$ an der Stelle $p$ in Richtung $v$.
Es ergibt sich%
\begin{equation}\label{eq:Richtungsableitung-flach}
\mathrm df_p(v) = (\mathrm df)(p)(v) := g'(0)
= \lim_{h\to 0}\frac{g(h)-g(0)}{h}
= \lim_{h\to 0}\frac{f(p+hv)-f(p)}{h}.
\end{equation}
Wir betrachten nun ein Skalarfeld $f\colon M\to\R$, welches
auf einer Untermannigfaltigkeit $M$ definiert ist.
Die Richtungsableitung lässt sich nun aber nicht mehr gemäß
\eqref{eq:Richtungsableitung-flach} bestimmen, weil die Gerade
\eqref{eq:Gerade-zum-Vektor} nicht innerhalb von $M$ liegen muss.
Außerhalb von $M$ ist das Skalarfeld nicht definiert, der Ausdruck
\eqref{eq:Richtungsableitung-flach} setzt dies aber voraus.

Dieses Problem lässt sich wie folgt lösen. Sei $p\in M$ ein Punkt
auf $M$. Sei $c\colon (-\varepsilon,\varepsilon)\to M$ eine Kurve
in $M$ mit $p=c(0)$ und $v=c'(0)$. Damit das überhaupt möglich ist, muss
der Vektor $v\in T_p M$, d.\,h. im Punkt $p$ tangential an $M$ sein.
Nun wird ähnlich wie zuvor die Funktion
$g\colon (-\varepsilon,\varepsilon)\to\R$ mit $g:=f\circ c$ betrachtet.
Für $g$ ist die gewöhnliche Ableitung definiert. Die Richtungsableitung
lässt sich also gemäß $g'(0)=(f\circ c)'(0)$ bestimmen.

\begin{definition}[Richtungsableitung]
Sei $M$ eine Untermannigfaltigkeit und $f\colon M\to\R$ ein Skalarfeld.
Sei $p\in M$ ein Punkt und $v$ ein Vektor, welcher am Punkt $p$
tangential an $M$ ist. Sei $c\colon (-\varepsilon,\varepsilon)\to M$
eine glatte Kurve mit $p=c(0)$ und $v=c'(0)$. Unter der
\emdef{Richtungsableitung} von $f$ am Punkt $p$ in Richtung $v$
versteht man die reelle Zahl
\begin{equation}
\mathrm df_p(v) = (\mathrm df)(p)(v) := (f\circ c)'(0)
= \lim_{h\to 0}\frac{f(c(h))-f(p)}{h}.
\end{equation}
\end{definition}
Sei $\varphi\colon U\to V$ mit $U\subseteq\R^m$ und $V\subseteq M$
eine lokale Karte. Sei $(g_k)$ der durch
$g_k=\frac{\partial\varphi}{\partial u^k}$ induzierte lokale
Rahmen. Demnach ist $(g_k)$ am Punkt $p$ eine
Basis von $T_p M$. Sei $(\mathrm dx^k)$ die zu $(g_k)$ eindeutig
bestimmte duale Basis. Diese spannt den Kotangentialraum $T_p^* M$ auf.

Die partiellen Ableitungen von $f$ lassen sich wie bei Skalarfeldern
auf dem $\R^n$ als die Richtungsableitungen in Richtung der
Basisvektoren sehen. Beim $\R^n$ war es die Standarbasis, hier ist
es $(g_k)$.
\begin{definition}[Partielle Ableitungen]
Die \emdef{partielle Ableitung} von $f$ nach der $k$-ten Koordinate
ist definiert als
\begin{equation}
(\partial_k f)(p) = \frac{\partial f}{\partial x^k}(p)
= \frac{\partial f(x)}{\partial x^k}\Big|_{x=p} :=
\mathrm df_p(g_k).
\end{equation}
\end{definition}
Wenn $f$ als total differenzierbar vorausgesetzt wird, dann ist
$\mathrm df_p$ eine Linearform, also ein Element des
Kotangentialraums. Es ergibt sich
\begin{equation}
\mathrm df_p = \sum_{k=1}^n \frac{\partial f}{\partial x^k}(p)\mathrm dx^k.
\end{equation}
Für einen Vektor $v=\sum_{k=1}^n v^k g_k$ ergibt
sich dann die duale Paarung
\begin{equation}
\mathrm df_p(v) = \sum_{k=1}^n \frac{\partial f}{\partial x^k}(p)v^k.
\end{equation}
Mit der lokalen Karte $\varphi$ ist eigentlich die lokale
Darstellung $\tilde f=f\circ\varphi$ gegeben. Zwischen
$f$ und $\tilde f$ gibt es aber einen besonders einfachen Zusammenhang.
\begin{corollary}
Sei $f$ ein Skalarfeld, $\varphi$ eine lokale Karte und
$\tilde f=f\circ\varphi$. Es gilt%
\begin{equation}
\frac{\partial f}{\partial x^k}(p) = \frac{\partial\tilde f}{\partial u^k}(u),
\end{equation}
wobei $p=\varphi(u)$.
\end{corollary}
\strong{Beweis.} Sei dazu $c(t)$ eine Kurve mit $p=c(0)$ und
$g_k(u)=c'(0)$. Sei außerdem $\tilde c$ eine Kurve im $\R^m$, so dass
$c=\varphi\circ \tilde c$. Nach der Kettenregel gilt
\begin{equation}
g_k(u) = c'(0) = (\varphi\circ\tilde c)'(0) = \mathrm d\varphi_u(\tilde c'(0)).
\end{equation}
Es gilt aber auch $g_k(u) = (\partial_k\varphi)(u) =
\mathrm d\varphi_u(\mathbf e_k)$. Daraus folgt
$\mathrm d\varphi_u(\tilde c'(0)) = \mathrm d\varphi_u(\mathbf e_k)$.
Da $\mathrm d\varphi_u$ eine injektive Abbildung ist, ergibt sich
$\tilde c'(0) = \mathbf e_k$.\;\qedsymbol

Für die Richtungsableitung ergibt sich
\begin{equation}
\mathrm df_p(v) = \mathrm d\tilde f_u((v^k)) =
\langle(\nabla \tilde f)(u),(v^k)\rangle.
\end{equation}
Auf der rechten Seite steht das Standardskalarprodukt, weil die
duale Paarung mit dem totalen Differential im Koordinatenraum einfach
das Standardskalarprodukt mit dem Gradient ist. Das ist ein rein
technischer Formalismus, diese Formel setzt keinenfalls eine
Metrik oder ähnlich voraus.


\section{Vektorfelder}

\subsection{Die kovariante Ableitung}

Sei $M$ eine Untermannigfaltigkeit des $\R^m$ und $X\colon M\to TM$
ein Vektorfeld. Nun kann doch wie bei einem Skalarfeld die
Richtungsableitung von $X$ in Richtung
eines Vektors $v\in T_p M$ definiert werden. Sei dazu
$c\colon (-\varepsilon,\varepsilon)\to M$ eine Kurve
mit $p=c(0)$ und $v=c'(0)$. Dann definiert man%
\begin{equation}
\mathrm dX_p(v) := (X\circ c)'(0).
\end{equation}
Nun taucht das Problem auf, dass die Ableitung an der Stelle $p$
ein Vektor ist, welcher nicht unbedingt tangential an $M$ sein muss.
Ein Ziel der Differentialgeometrie ist es aber, eine Theorie aufzubauen,
welche nur auf Tangentialräumen beruht. Aus diesem Grund projizieren
wir den Vektor $w=\mathrm dX_p(v)$ orthogonal auf den Tangentialraum
$T_p M$. Der zum Tangentialraum orthogonale Anteil entfällt dabei.

Die orthogonale Projektion auf $T_p M$ nennen wir $\Pi_p$. Es handelt
sich um eine lineare Abbildung.

\begin{definition}[Kovariante Ableitung]
Die \emdef{kovariante Ableitung} eines Vektorfeldes $X\colon M\to TM$
an der Stelle $p\in M$ in Richtung $v\in T_p M$ ist definiert als
\begin{equation}
(\nabla_v X)(p) := \Pi_p((X\circ c)'(0)),
\end{equation}
wobei $c\colon (-\varepsilon,\varepsilon)\to M$ eine glatte Kurve
mit $p=c(0)$ und $v=c'(0)$ ist.
\end{definition}

\noindent
Natürlich kann auch $v=Y(p)$ sein, wobei $Y$ ein zweites Vektorfeld
ist. Man notiert dazu
\begin{equation}
(\nabla_Y X)(p) := (\nabla_{Y(p)} X)(p).
\end{equation}

\noindent
Wir wollen nun eine Formel für die Richtungsableitung herleiten,
wenn das Vektorfeld in lokalen Koordinaten dargestellt ist. Sei
dazu $\varphi\colon U\to V$ mit $V\subseteq M$ eine lokale
Parametrisierung. Diese induziert den Rahmen $(g_k)$ mit
$g_k = \frac{\partial\varphi}{\partial u_k}$. Die lokale
Darstellung $\tilde X$ des Vektorfeldes $X$ sei gemäß den
Funktionen $a^k(u)$ gegeben, so dass%
\begin{equation}
\tilde X(u) = (X\circ\varphi)(u) = \sum_{k=1}^n a^k(u)g_k(u).
\end{equation}
An jedem Punkt $p=\varphi(u)$ ist das Vektorfeld also als
Linearkombination aus der Tangentialbasis an diesem Punkt dargestellt.
Der Vektor $v$ sei am Punkt $p=\varphi(u_0)$ ebenfalls als
Linearkombination aus der Tangentialbasis dargestellt:
$v = \sum_{k=1}^n v^k g_k(u_0)$.

Nun sei $\tilde c$ die lokale Darstellung der Kurve, gemäß
$c=\varphi\circ\tilde c$. Es ergibt sich
\begin{equation}
(\nabla_v X)(p) = \Pi_p((X\circ c)'(0))
= \Pi_p((X\circ\varphi\circ\tilde c)'(0))
= \Pi_p((\tilde X\circ\tilde c)'(0)).
\end{equation}
Nach der Produktregel ergibt sich
\begin{gather}
(\tilde X\circ\tilde c)'(t)
= \frac{\mathrm d}{\mathrm dt} \sum_{i=1}^n a^i g_i
= \sum_{i=1}^n \frac{\mathrm da^i}{\mathrm dt}g_i
+ \sum_{i=1}^n a^i\frac{\mathrm dg_i}{\mathrm dt}.
\end{gather}
Anwendung der Kettenregel bringt nun
\begin{equation}
\frac{\mathrm da^i}{\mathrm dt}
= \sum_{j=1}^n \frac{\partial a^i}{\partial u_j}
\frac{\mathrm d\tilde c_j}{\mathrm dt},\qquad
\frac{\mathrm dg^i}{\mathrm dt}
= \sum_{j=1}^n \frac{\partial g_i}{\partial u_j}
\frac{\mathrm d\tilde c_j}{\mathrm dt}.
\end{equation}
Nach der Kettenregel und $c(0)=\varphi(u_0)$ ergibt sich aber auch
\begin{equation}
c'(0) = (\tilde c\circ\varphi)'(0)
= \sum_{j=1}^n \tilde c_j'(0)(\partial_j\varphi)(u_0)
= \sum_{j=1}^n \tilde c_j'(0)g_j(u_0)
= \sum_{j=1}^n v^j g_j(u_0).
\end{equation}
Der Koeffizientenvergleich ergibt $\tilde c_j'(0)=v^j$. Demnach ergibt sich
\begin{equation}
(\tilde X\circ\tilde c)'(0)
= \sum_{i,j} (\partial_j a^i)(u_0) v^j g_i(u_0)
+ \sum_{i,j} a^i(u_0) (\partial_j g_i)(u_0) v^j.
\end{equation}
Wir nutzen nun aus, dass die Projektion eine lineare Abbildung ist.
Die linke Seite ist eine Linearkombination aus der Tangentialbasis,
liegt also schon im Tangentialraum. Auf die linke Seite ist die
Projektion daher wirkungslos. Somit ergibt sich
\begin{equation}\label{eq:kov-Ableitung1}
(\nabla_v X)(p) = \sum_{i,j} (\partial_j a^i)(u_0) v^j g_i(u_0)
+ \sum_{i,j} v^j a^i(u_0) \Pi_p((\partial_j g_i)(u_0)).
\end{equation}
Die übrig gebliebene Projektion stellen wir als Linearkombination
aus der Tangentialbasis dar. Die Koeffizienten $\Gamma_{ij}^k$
sind an der Stelle $u_0$ also durch%
\begin{equation}
\Pi_p((\partial_j g_i)(u_0))
= \Pi_p((\partial_i\partial_j\varphi)(u_0))
= \sum_{k=1}^n \Gamma_{ij}^k g_k(u_0)
\end{equation}
gegeben. Die $\Gamma_{ij}^k(u_0)$ nennt man \emph{Christoffel-Symbole}.
Auf der linken Seite von \eqref{eq:kov-Ableitung1} machen wir eine
Indexumbenennung $i:=k$. Es ergibt sich schließlich
\begin{equation}
(\nabla_v X)(p)
= \sum_{k=1}^n \bigg(\sum_{j=1}^n (\partial_j a^k)(u_0)v^j
+\sum_{i=1}^n\sum_{j=1}^n \Gamma_{ij}^k(u_0) v^j a^i(u_0)\bigg)g_k(u_0).
\end{equation}
Kurz
\begin{equation}
\nabla_v X = \sum_k \bigg(\sum_j v^j\partial_j a^k
+\sum_{i,j} \Gamma_{ij}^k v^j a^i\bigg)g_k.
\end{equation}
Oder kürzer $(\nabla_v X)^k = v^j \partial_j a^k + \Gamma_{ij}^k v^j a^i$.
