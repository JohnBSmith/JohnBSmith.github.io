
\chapter{Differenzierbare Mannigfaltigkeiten}

\section{Differentialgleichungen}

Sei $M$ eine differenzierbare Mannigfaltigkeit und sei $c\colon\R\to M$
eine differenzierbare Parameterkurve in $M$. Daher gibt es auch
$c'\colon\R\to TM$. Das bringt uns auf die folgende Idee. Hat man
ein durch $t$ parametrisiertes Vektorfeld
\begin{equation}
f\colon\R\times M\to TM,\quad f(t,p)\in T_p M,
\end{equation}
gegeben, dann lässt sich das Anfangswertproblem
\begin{equation}
c'(t) = f(t,c(t)),\quad c_0=c(t_0),\quad 
\end{equation}
formulieren. Für eine abstrakte Mannigfaltigkeit muss das
Anfangswertproblem aber nun in lokale Koordinaten übersetzt
werden. Sei dazu $\varphi$ eine lokale Karte und $c=\varphi\circ x$.
Anwendung der Kettenregel ergibt
\begin{equation}
c'(t) = (\varphi\circ x)'(t) = \mathrm d\varphi_{x(t)}(x'(t)).
\end{equation}
Aus $\mathrm d\varphi_{x(t)}(x'(t))=f(t,\varphi(x(t)))$ erhält man
\begin{equation}
x'(t) = f_\varphi(t,x(t)),\quad
f_\varphi(t,x):=\mathrm d\varphi_{x}^{-1}(f(t,\varphi(x)).
\end{equation}
In lokalen Koordinaten lässt sich das Anfangswertproblem also als
ein gewöhnliches Anfangswertproblem im Koordinatenraum darstellen.
Für eine abstrakte Mannigfaltigkeit ist auch nur $f_\varphi$ bekannt.
Wir wollen nun aber in Erfahrung bringen, welche Gestalt das
Anfangswertproblem nach einem Kartenwechsel bekommt. Sei $\psi$
dazu eine zweite Karte. Daraus ergibt sich der Vergleich
\begin{equation}
f(t,p)
= \mathrm d\varphi_x(f_\varphi(t,x))
= \mathrm d\psi_{\tilde x}(f_\psi(t,\tilde x)),\quad
p = \varphi(x) = \psi(\tilde x).
\end{equation}
Umformen bringt $x = (\varphi^{-1}\circ\psi)(\tilde x)$ und
\begin{equation}
f_\varphi(t,x) = (\mathrm d\varphi_x^{-1}\circ \mathrm d\psi_{\tilde x})(f_\psi(t,\tilde x))
= \mathrm d(\varphi^{-1}\circ \psi)_{\tilde x}(f_\psi(t,\tilde x)).
\end{equation}
Es gilt aber auch
\begin{equation}
x'(t) = (\varphi^{-1}\circ\psi\circ\tilde x)'(t)
= \mathrm d(\varphi^{-1}\circ\psi)_{\tilde x(t)}(\tilde x'(t)).
\end{equation}
Das Differential des Kartenwechsels hebt sich weg und man
erhält%
\begin{equation}
\tilde x'(t) = f_\psi(t,\tilde x(t))
= \mathrm d(\psi^{-1}\circ\varphi)_{x(t)}(f_\varphi(t,x(t))).
\end{equation}
Bei der Transformation des Anfangswertproblems werden die Werte des
parametrisierten Vektorfelds also mit dem Differential der
Kartenwechselabbildung transformiert.

\newpage
\section{Dynamische Systeme}

\begin{definition}[Dynamisches System]\mbox{}\\*
Ein \emdef{dynamisches System} ist eine Abbildung
$\Phi\colon T\times M\to M$, welche
die beiden Eigenschaften
\begin{gather}
\label{eq:dyn-Sys-1}
\Phi(0,x) = x,\\
\label{eq:dyn-Sys-2}
\Phi(t_1+t_2,x) = \Phi(t_2,\Phi(t_1,x))
\end{gather}
erfüllt. Man bezeichnet $M$ als \emdef{Zustandsraum} des Systems und
$T$ als \emdef{Menge der Zeitpunkte}.
Für $T=\R$ oder $T=\R^+$ spricht man von einem
\emdef{Zeit-kontinuierlichen} System, für $T=\Z$ oder $T=\N$ von einem
\emdef{Zeit-diskreten}.
\end{definition}

\noindent
Wir wollen uns nun auf Zeit-kontinuierliche Systeme beschränken,
bei denen der Zustandsraum $M$ eine differenzierbare Mannigfaltigkeit
ist. Man betrachte das autonome Anfangswertproblem
\begin{equation}\label{eq:AWP-M}
c'(t) = f(c(t)),\quad c(0)=c_0,
\end{equation}
wobei das Vektorfeld $f\colon M\to TM$ hinreichend gutartig sein soll,
so dass zu jedem $c_0$ eine eindeutige Lösung $c\colon\R\to M$
existiert. 

\begin{theorem}\label{autonom-fasert}
Besitzt das Anfangswertproblem \eqref{eq:AWP-M} zu jedem $c_0$ eine
eindeutige Lösung, dann verläuft durch jeden Punkt von $M$ genau
eine Lösung, d.\,h. keine zwei Lösungen können sich schneiden
oder berühren.
\end{theorem}
\noindent\strong{Beweis.}
Seien $c_1(t)$ und $c_2(t)$ zwei Lösungen der Dgl. und sei
$c_1(t_1)=c_2(t_2)$. Setze $a=t_1-t_2$, dann gilt $c_2(t_2)=c_1(t_2+a)$.
Es muss gezeigt werden, dass $c_2$ nur Parameter"=verschoben
zu $c_1$ um $a$ ist, d.\,h. $c_2(t)=c_1(t+a)$ für alle $t$. Dazu
definieren wir $c(t):=c_1(t+a)$. Für $c$ ergibt sich das
Anfangswertproblem
\begin{equation}
c'(t) = c_1'(t+a) = f(c_1(t+a)) = f(c(t)),\quad c(t_2)=c_1(t_2+a).
\end{equation}
Außerdem gilt
\begin{equation}
c_2'(t) = f(c_2(t)), \quad c_2(t_2)=c_1(t_2+a).
\end{equation}
Das ist beides das selbe Anfangswertproblem. Da die Lösung nach
Voraussetzung eindeutig ist, muss $c(t)=c_2(t)$
für alle $t$ sein.\;\qedsymbol

\begin{theorem}
Unter der Voraussetzung, dass es zu jedem $c_0$ eine eindeutige
Lösung des Anfangswertproblems \eqref{eq:AWP-M} gibt, ist durch
$\Phi(t,c_0)=c(t)$ ein dynamisches System gegeben.
\end{theorem}
\noindent\strong{Beweis.}
Die Eigenschaft \eqref{eq:dyn-Sys-1} ist gemäß $\Phi(0,c_0)=c_0$
trivial erfüllt. Die Eigenschaft \eqref{eq:dyn-Sys-2} bekommt
die Form $c(t_1+t_2)=\Phi(t_2,c(t_1))$. Nach Satz
\ref{autonom-fasert} verläuft durch den Zustand $c(t_1)$ genau
eine Bahn $p(t)=\Phi(t,c(t_1))$ und nach \eqref{eq:dyn-Sys-1} gilt
$p(0)=\Phi(0,c(t_1))=c(t_1)$. Dann muss $p$ zu $c$ Parameter"=verschoben
sein und es gilt $p(t)=c(t+t_1)$. Setze speziell $t=t_2$.\;\qedsymbol



