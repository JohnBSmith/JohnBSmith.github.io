
\chapter{Krummlinige Koordinaten}

\section{Partielle Ableitungen}

Wir betrachten die stetig differenzierbare Koordinatentransformation
$\Phi\colon U\to V$ mit offenen $U,V\subseteq\R^n$, so dass
$u$ die krummlinigen Koordinaten sind und $x = \Phi(u)$ die
kartesischen. Zudem sei $\mathrm d\Phi$ an jeder Stelle
invertierbar. Gegeben sei außerdem ein stetig differenzierbares
Skalarfeld $f(u)$. Nun stellt sich die Frage, wie $\partial_k f(u)$ zu
verstehen ist. Sei $\bar f(x)$ die Darstellung des Skalarfeldes in
kartesischen Koordinaten, so dass $f = \bar f\circ\Phi$ ist. Mit der
Kettenregel bekommen wir die Beziehung
\begin{equation}\label{eq:partial-Phi}
\partial_k f(u) = \partial_k (\bar f\circ\Phi)(u)
= \mathrm d\bar f(\Phi(u))(\partial_k\Phi(u))
= \mathrm d\bar f(x)(\mathbf g_k).
\end{equation}
Die $k$-te partielle Ableitung in krummlinigen Koordinaten ist also
die Richtungsableitung in Richtung des $k$-ten Tangentialvektors der
Koordinatenlinien.

Sei nun $\gamma(t):=u+t\mathbf e_k$ die Parametergerade. Die
partielle Ableitung ist definiert als
$\partial_k f(u):=(f\circ\gamma)'(0)$. Nun kann man das Konzept
allerdings auch für Parameterkurven anstelle von Parametergeraden
definieren. Man betrachtet daher die Linie $\Phi\circ\gamma$ und
definiert damit die partielle Linienableitung nach der $k$-ten
Koordinatenlinie als
\begin{equation}
\tilde\partial_k \bar f(x) := (\bar f\circ\Phi\circ\gamma)'(0).
\end{equation}
Damit ergibt sich aber
\begin{equation}
\tilde\partial_k \bar f(x) = (f\circ\gamma)'(0) = \partial_k f(u).
\end{equation}
Die Linienableitungen der Darstellung in kartesischen Koordinaten sind
also die gewöhnlichen partiellen Ableitungen der Darstellung in
krummlinigen Koordinaten.

Von wichtigem Interesse ist, wie die gewöhnlichen partiellen
Ableitungen $\partial_k \bar f(x)$ in Beziehung zu den partiellen
Ableitungen $\partial_k f(u)$ stehen. Analog zu \eqref{eq:partial-Phi}
bekommen wir
\begin{equation}
\partial_k \bar f(x) = \partial_k(f\circ\Phi^{-1})(x)
= \mathrm df(u)(\partial_k\Phi^{-1}(x))
= \sum\nolimits_i\partial_i f(u)\partial_k\Phi^{-1}(x)_i.
\end{equation}
gemäß der Kettenregel.

\section{Gradient}

Weil in Koordinaten gerechnet wird, ist das Differential
$J=\mathrm d\Phi(u)$ die Jacobi"=Matrix. Ihre Inverse ist die Matrix
\begin{equation}
J^{-1} = \mathrm d\Phi(u)^{-1} = \mathrm d\Phi^{-1}(x).
\end{equation}
Kurze Vorbetrachtung. Die $\mathbf e_k$ müssen als Linearkombination
der $\mathbf g_k$ dargestellt werden. Die $\mathbf g_k$ sind aber
gerade die Spalten der Jacobi"=Matrix. Somit können wir rechnen
\begin{equation}
E = JJ^{-1},\;\text{folglich}\;\, (\mathbf e_j)_i = \delta_{ij}
= \sum_k J_{ik}(J^{-1})_{kj} = \sum_k (J^{-1})_{kj} (\mathbf g_k)_i.
\end{equation}
Wir bekommen $\mathbf e_j = \sum_k (J^{-1})_{kj}\mathbf g_k$.
Außerdem ist
\begin{equation}
\partial_k\Phi^{-1}(x) = \mathrm d\Phi^{-1}(x)(\mathbf e_k)
= \mathrm d\Phi(u)^{-1}(\mathbf e_k)
= J^{-1}\mathbf e_k,
\end{equation}
womit $\partial_k\Phi^{-1}(x)_i = (J^{-1})_{ik}$ gilt.

Ausgehend von der kartesischen Darstellung rechnen wir den
Gradienten nun in die Darstellung bezüglich
krummliniger Koordinaten um. Das macht
\begin{align}
\nabla\bar f(x) &= \sum_k\partial_k \bar f(x)\mathbf e_k
= \sum_{k,i}\partial_i f(u)\partial_k\Phi^{-1}(x)_i\mathbf e_k\\
&= \sum_{k,i}\partial_i f(u) (J^{-1})_{ik}\mathbf e_k
= \sum_{k,i,j}\partial_i f(u) (J^{-1})_{ik}(J^{-1})_{jk}\mathbf g_j.
\end{align}
Da steckt eine Matrizenmultiplikation drin, nämlich
\begin{equation}
\sum_{k} (J^{-1})_{ik}(J^{-1})_{jk} = (J^{-1}(J^{-1})^T)_{ij}
= ((J^T J)^{-1})_{ij} = (G^{-1})_{ij} = g^{ij}.
\end{equation}
Man kommt zum Ergebnis
\begin{equation}
\operatorname{grad} f(u) = \nabla\bar f(x)
= \sum_{i,j}\partial_i f(u) g^{ij}\mathbf g_j.
\end{equation}
In klassischer Notation würde man schreiben
\begin{equation}
\partial_k\Phi^{-1}(x)_i = \frac{\partial u_i}{\partial x_k},\quad
\mathbf g_j = \frac{\partial x}{\partial u_j},\quad
\mathbf e_k = \frac{\partial x}{\partial x_k}.
\end{equation}
Man rechnet dann lediglich zweimal Kettenregel, das ist
\begin{equation}
\nabla\bar f(x) = \sum_k\frac{\partial\bar f}{\partial x_k}\frac{\partial x}{\partial x_k}
= \sum_{k,i}\frac{\partial f}{\partial u_i}
\frac{\partial u_i}{\partial x_k}\frac{\partial x}{\partial x_k}
= \sum_{k,i,j}\frac{\partial f}{\partial u_i}\frac{\partial u_i}{\partial x_k}
\frac{\partial u_j}{\partial x_k}\frac{\partial x}{\partial u_j},
\end{equation}
und beachtet
\begin{equation}
g^{ij} = \sum_k\frac{\partial u_i}{\partial x_k}\frac{\partial u_j}{\partial x_k}.
\end{equation}
Speziell für orthogonale Koordinaten ist $G=(g_{ij})$ eine
Diagonalmatrix. Wegen $G^{-1}=(g^{ij})$ ist dann
$g^{kk} = |\mathbf g_k|^2$ der Kehrwert von
$g_{kk} = |\mathbf g_k|^2$. Infolge ist
\begin{equation}
\operatorname{grad} f(u)
= \sum_k\partial_k f(u)g^{kk}\mathbf g_k
= \sum_k\partial_k f(u)\frac{\mathbf g_k}{|\mathbf g_k|^2}
= \sum_k\partial_k f(u)\frac{\mathbf {\hat g}_k}{|\mathbf g_k|}.
\end{equation}
Für Polarkoordinaten
\begin{equation}
\begin{pmatrix}x\\ y\end{pmatrix}
= \Phi\begin{pmatrix}r\\ \varphi\end{pmatrix}
= \begin{pmatrix}r\cos\varphi\\ r\sin\varphi\end{pmatrix},\quad
\mathbf g_r = \begin{pmatrix}\cos\varphi\\ \sin\varphi\end{pmatrix},\quad
\mathbf g_\varphi = \begin{pmatrix}-r\sin\varphi\\ r\cos\varphi\end{pmatrix}
\end{equation}
gilt beispielsweise
\begin{equation}
\operatorname{grad} f(u)
= \partial_r f(u)\frac{\mathbf g_r}{|\mathbf g_r|^2}
+ \partial_\varphi f(u)\frac{\mathbf g_\varphi}{|\mathbf g_\varphi|^2}
= \partial_r f(u)\mathbf g_r
+ \frac{1}{r^2}\partial_\varphi f(u)\mathbf g_\varphi.
\end{equation}

