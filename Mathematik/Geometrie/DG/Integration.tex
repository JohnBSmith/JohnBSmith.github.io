
\chapter{Integration auf Mannigfaltigkeiten}

\section{Vorbereitungen}

Das dreidimensionale Analogon zu einem Parallelogramm nennt man
\emph{Spat} oder \emph{Parallelepiped}. Die Verallgemeinerung dieses
Begriffs auf eine beliebige Dimension wollen wir als \emph{Parallelotop}
bezeichnen oder auch einfach \emph{Spat} nennen. 

Im Folgenden sollen ein einige Ergebnisse der multilinearen Algebra
rekapituliert werden, mit denen sich das Volumen eines Spates ermitteln
lässt.

Das äußere Produkt $\Lambda^n \R^m$ ist ein Vektorraum, deren Elemente
\emph{Multivektoren} genannt werden. Ein \emph{Blade} ist ein
Multivektor der Form $v_1\wedge\ldots\wedge v_n$ für $v_k\in\R^m$.
Jeder Multivektor lässt sich als Linearkombination von Blades
darstellen.

Auf $\Lambda^n\R^m$ lässt sich gemäß
\begin{equation}
\langle v_1\wedge\ldots\wedge v_n,w_1\wedge\ldots\wedge w_n\rangle
:= \det(\langle v_i,w_j\rangle)
\end{equation}
ein Skalarprodukt definieren. Das Skalarprodukt von Multivektoren
wird über die Bilinearität auf das von Blades zurückgeführt. Wie
bei jedem Skalarprodukt wird gemäß
\begin{equation}
\|X\| = \sqrt{\langle X,X\rangle}
\end{equation}
eine Norm induziert. Das Volumen des durch die Vektoren $v_1,\ldots,v_n$
aufgespannten Parallelotops $P$ ist
\begin{equation}
\operatorname{vol}(P) = \|v_1\wedge\ldots\wedge v_n\|.
\end{equation}
Sei $f\colon\R^n\to\R^m$ mit $f(x):=Ax+t$ eine affine Abbildung.
Ist $E$ der Einheitswürfel, dann ist $P=f(E)$, das ist das Bild des
Einheitswürfels unter der affinen Abbildung. Da sich das Volumen
bei Verschiebung nicht ändert, kann ohne Beschränkung der Allgemeinheit
$t=0$ gesetzt werden. Der Einheitswürfel $E$ wird von der kanonischen
Orthonormalbasis $(\mathbf e_k)$ aufgespannt. Genauer gesagt gilt
\begin{equation}\textstyle
E = \{v\mid v = \sum_{k=1}^n a_k\mathbf e_k\;\text{und}\;a_k\in [0,1]\}.
\end{equation}
Sind die $v_k$ die Spaltenvektoren von $A$, d.\,h. $v_k = A\mathbf e_k$,
dann ergibt sich
\begin{equation}
\operatorname{vol}(P) = \sqrt{\det(A^T A)}.
\end{equation}
Man bezeichnet $\det(A^T A)$ als \emph{gramsche Determinante}
zur Matrix $A$.

Die Vektoren $(v_1,\ldots,v_n)$ sind genau dann linear abhängig,
wenn $\operatorname{vol}(P)=0$ gilt. Das Parallelotop ist dann
flach zusammengefallen. Man stelle sich dazu den Fall $n=2$ und
$m=3$ vor, das ist ein Parallelogramm welches zu einer Strecke
zusammenfällt.

Die lineare Abbildung $f(x)=Ax$ ist also genau dann injektiv, wenn
$\det(A^T A)\ne 0$. Nach der Äquivalenz von $X=0$ und $\|X\|=0$ ist
$f$ auch genau dann injektiv, wenn
\begin{equation}
v_1\wedge\ldots\wedge v_n \ne 0.\qquad (v_k=A\mathbf e_k)
\end{equation}





