

\chapter{Weitere Konzepte und Begriffe}

\section{Faserbündel}

Seien $B$ und $E$ zunächst beliebige Mengen und sei $\pi\colon E\to B$
eine surjektive Abbildung. Eine surjektive Abbildung ist die allgemeinste
Vorstellung von dem, was man unter einer Projektion verstehen kann.
Um diese Vorstellung anschaulich zu gestalten, führen wir ein paar
zusätzliche Begriffe ein. Wir nennen $E$ den \emph{Totalraum} und $B$
die \emph{Basis}. Für $p\in B$ wird das Urbild $\pi^{-1}(\{p\})$ als
\emph{Faser} bezeichnet. Durch die Projektion $\pi$ wird der Totalraum
in disjunkte Fasern zerlegt. Allen Punkten einer Faser wird durch
die Projektion derselbe Punkt auf der Basis zugeordnet.

Jede der Fasern kann eine beliebige Punktmenge sein. Ist nämlich eine
beliebige Partition von $E$ gegeben, dann entspricht dies einer
Äquivalenzrelation, wobei die Fasern die Äquivalenzklassen sind.
Die Projektion $\pi$ ist dann nichts anderes als die kanonische
Projektion, wobei die Basis $B$ eine Indizierung der Quotientenmenge
erwirkt, dergestalt dass jedem Element von $B$ bijektiv eine Faser
zugeordnet ist.

Dieser allgemeine Umstand soll nun darin eingeschränkt werden, dass
jede Faser gleichartiger Gestalt sein muss. Jedenfalls ist das der
Gedankengang der uns als nächstes naheliegt, da wir nicht beliebige
Punktmengen betrachten wollen, sondern geometrische und topologische
Ideen verwirklichen.

Eine erste Überlegung dazu ist, dass die Fasern $\pi^{-1}(\{p\})$ alle
homöomorph zur selben prototypischen Faser $F$ sein sollen. Der
Totalraum kann aber durch die Projektion völlig zerrissen werden.
Um das zu verhindern soll die Projektion $\pi$ stetig sein. Daher muss
es sich bei $B$ und $E$ um topologische Räume handeln. Jetzt kann man
sich überlegen, ob der der Totalraum dennoch auf irgendeine Art und
Weise zerschnitten sein oder Löcher besitzen darf. Um das auszuschließen,
verlangt man dass nicht nur die Faser homöomorph zu $F$ sein soll,
sondern auf einer hinreichend kleinen Umgebung $U$ auch
$\pi^{-1}(U)$ homöomorph zu $U\times F$, wobei $\pi$ dann der
Projektion auf den ersten Faktor des kartesischen Produktes
entspricht.

\begin{definition}[Faserbündel]\index{Faserbündel}\mbox{}\\*
Seien $E,B$ und $F$ topologische Räume und sei $\pi\colon E\to B$
eine stetige surjektive Abbildung. Ein Faserbündel ist eine Struktur
$(E,B,\pi,F)$, wobei $\pi$ lokal trivialisierbar ist.

Man nennt $\pi$ lokal trivialisierbar\index{lokal trivialisierbar},
wenn es zu jedem Punkt $x\in E$ eine offene Umgebung
$U\subseteq B$ mit $p=\pi(x)\in U$ gibt,
so dass ein Homöomorphismus $\varphi\colon U\times F\to\pi^{-1}(U)$
mit $\operatorname{proj}_1 = \pi\circ\varphi$ existiert. Hierbei
ist $\operatorname{proj}_1\colon U\times F\to U$ mit
$\operatorname{proj}_1(p,y):=p$ die Projektion auf den ersten Faktor
des kartesischen Produktes.
\end{definition}

\noindent
Oft kommen solche Abbildungen vor, die jedem Punkt $p\in B$ einen Punkt
in der zugehörigen Faser $\pi^{-1}(\{p\})$ zuordnen.

\begin{definition}[Schnitt]\index{Schnitt}\mbox{}\\*
Sei $(E,B,\pi,F)$ ein Faserbündel. Eine stetige Abbildung
$f\colon B\to E$ wird Schnitt genannt, wenn $f(p)\in\pi^{-1}(\{p\})$
für jeden Punkt $p\in B$ gilt.
\end{definition}

\noindent
Für einen Schnitt $f$ gilt $\pi(f(p))=p$. Ein Schnitt $f$ ist also eine
Rechtsinverse der Projektion $\pi$.

\begin{definition}[Vektorraumbündel]\index{Vektorbündel}\mbox{}\\*
Ein Faserbündel $(E,B,\pi,F)$ mit $F=\R^n$ heißt Vektorraumbündel, wenn
jede Faser $\pi^{-1}(\{p\})$ ein Vektorraum der Dimension $n$ ist und
$\psi\colon\R^n\to\pi^{-1}(\{p\})$ mit $\psi(v):=\varphi(p,v)$
eine bijektive lineare Abbildung, wobei $\varphi$ der Homöomorphismus
zur lokalen Trivialisierung sein soll.
\end{definition}

