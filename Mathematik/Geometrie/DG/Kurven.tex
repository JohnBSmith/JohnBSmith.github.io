

\chapter{Kurven im euklidischen Raum}

\section{Vorbereitungen}

\subsection{Koordinatensysteme}\label{section:Koordinatensysteme}

Zur Darstellung von Punkten im euklidischen Raum $E_n$ wird ein
Koordinatensystem benötigt, das ist eine Abbildung
$\varphi\colon\R^n\to E_n$. Da der $E_n$ ein abstraktes mathematisches
Objekt ist, kann auch $\varphi$ nicht rechnerisch erfasst werden.
Wir bräuchten eine Darstellung von $E_n$, was aber $\varphi$ selbst
sein soll.

Was wir aber erfassen können, ist die Koordinatenwechselabbildung%
\begin{equation}
\psi\colon\R^n\to\R^n,\quad \psi := \varphi_2\circ\varphi_1^{-1},
\end{equation}
wobei $\varphi_1\colon\R^n\to E_n$ und $\varphi_2\colon\R^n\to E_n$
zwei unterschiedliche Koordinatensysteme sind.

Man beschränkt sich bei $\varphi$ zunächst
auf eine bijektive affine Abbildung. Eine affine Abbildung ist
zusammengesetzt aus einer bijektiven linearen Abbildung und einer
Verschiebung. Demnach gilt $p=\varphi(x) = p_0+L(x)$ wobei
$L\colon\R^n\to V$ eine lineare Abbildung und $V$ der
Verschiebungsvektorraum von $E_n$ ist. Umstellen nach $x$ ergibt
$x=L^{-1}(p-p_0)$. Auch bei $\psi$ muss es sich um eine affine
Abbildung handeln:%
\begin{align}
\psi(x) &= (\varphi_2^{-1}\circ\varphi_1)(x)= L_2^{-1}((p_1+L_1(x))-p_2)\\
&= L_2^{-1}(p_1-p_2)+(L_2^{-1}\circ L_1)(x) = v_0+Ax.
\end{align}
Da $A\colon\R^n\to\R^n$ mit $A:=L_2^{-1}\circ L_1$ ein Automorphismus
zwischen Koordinatenräumen ist, darf $A$ bei Wahl der kanonischen Basis
wie üblich mit seiner kanonischen Darstellungsmatrix identifiziert
werden. Bei $A$ muss es sich demnach um eine reguläre Matrix handeln.
Bei $v_0\in\R^n$ mit
$v_0:=L_2^{-1}(p_1-p_2)$ handelt es sich um einen Vektor, welcher den
Verschiebungsvektor repräsentiert, der von $p_1$ nach $p_2$ verschiebt.

Da man anstelle von $E_n$ irgendeinen affinen Raum einsetzen kann,
zeigt die Rechnung auch, dass die Verkettung $\psi_2^{-1}\circ\psi$
auch wieder ein affiner Koordinatenwechsel ist, wenn es denn $\psi_1$
und $\psi_2$ sind. Für $\psi_1(x)=v_1+A_1 x$ und
$\psi_2(x) = v_2+A_2 x$
ergibt sich:%
\begin{equation}
\psi(x) = (\psi_2^{-1}\circ\psi_1)(x)
= A_2^{-1}((v_1+A_1x)-v_2)
= A_2^{-1}(v_1-v_2) + A_2^{-1} A_1 x.
\end{equation}

\begin{definition}[Affine Koordinatentransformation]%
\index{affine Koordinatentransformation}\mbox{}\\*
Unter einer \emdef{affinen Koordinatentransformation} versteht man
die affine Abbildung
\begin{equation}
\psi\colon\R^n\to\R^n,\quad \psi(x):=v_0+Ax,
\end{equation}
wobei $v_0\in\R^n$ und $A\in\operatorname{GL}(n,\R)$ ist.
\end{definition}
Für $v_0=0$ spricht man von einer \emph{linearen
Koordinatentransformation}, wobei $A$ die \emph{Transformationsmatrix}
ist.

\subsection{Vektorwertige Funktionen}
\begin{theorem}\label{lim-vektorwertige-Folge}
Eine vektorwertige Folge $v_i = \sum_{k=1}^n v_{ki}\mathbf e_k$
ist genau dann konvergent, wenn sie komponentenweise konvergiert.
Es gilt
\begin{equation}
\lim_{i\to\infty} v_i
= \sum_{k=1}^n (\lim_{i\to\infty} v_{ki})\mathbf e_k.
\end{equation}
\end{theorem}
\strong{Beweis.} Angenommen, alle $(v_{ki})$ sind konvergent, dann gilt
nach den Grenzwertsätzen für Folgen in normierten Räumen die folgende
Rechnung:
\begin{equation}
\sum_{k=1}^n (\lim_{i\to\infty} v_{ki})\mathbf e_k
= \sum_{k=1}^n \lim_{i\to\infty} (v_{ki}\mathbf e_k)
= \lim_{i\to\infty}\sum_{k=1}^n v_{ki}\mathbf e_k = \lim_{i\to\infty} v_i.
\end{equation}
Sei nun umgekehrt $(v_i)$ konvergent mit $v = \lim\limits_{i\to\infty} v_i$.
Man betrachte nun die Projektion $p_k\colon\R^n\to\R$ mit
\begin{equation}
p_k(v) = p_k(\sum_{k=1}^n v_k\mathbf e_k) := v_k.
\end{equation}
Wenn man $v$ leicht variiert und dabei $p_k(v)$ betrachtet, ist
ersichtlich, dass es sich bei $p_k$ um eine stetige Abbildung handeln
muss. Da $p_k$ total differenzierbar ist, sollte diese Eigenschaft
evident sein. Daher gilt:
\begin{equation}
p_k(v) = p_k(\lim_{i\to\infty} v_i) = \lim_{i\to\infty}(p_k(v_i))
= \lim_{i\to\infty} v_{ik}.
\end{equation}
Somit hat jede Komponente den erwarteten Grenzwert.\;\qedsymbol

\begin{theorem}\label{lim-vektorwertige-Funktion}
Eine vektorwertige Funktion $f\colon I\to\R^n$ mit $I\subseteq\R$ und
$f(t)=(f_1(t),\ldots,f_n(t))$ ist genau dann an der Stelle $t_0$
konvergent, wenn alle Komponenten $f_k\colon I\to\R$
konvergent sind. Es gilt%
\begin{equation}
\lim_{t\to t_0} f(t) = \sum_{k=1}^n (\lim_{t\to t_0}f(t))\mathbf e_k.
\end{equation}
\end{theorem}
\begin{corollary}
Eine vektorwertige Funktion ist genau dann stetig, wenn sie in
jeder Komponente stetig ist.
\end{corollary}
\strong{Beweis.} Ist völlig analog zum vorangegangenen Beweis.
Alternativ sei $v=(v_1,\ldots,v_n)$ und $(t_i)$ eine beliebige Folge
mit $t_i\to t$. Satz \ref{lim-vektorwertige-Folge} zufolge gilt dann
\begin{gather}
v = \lim_{t\to t_0} f(t) \iff \forall (t_i)(v=\lim_{i\to\infty} f(t_i))
\iff \forall k\forall(t_i)(v_k = \lim_{i\to\infty} f_k(t_i))\\
\iff \forall k(v_k = \lim_{t\to t_0} f_k(t)).\;\qedsymbol
\end{gather}


\section{Allgemeine Begriffe}
\subsection{Parameterkurven}

Was wollen wir genau unter einer Kurve verstehen? Zunächst wollen
wir nur solche Kurven betrachten, die in den euklidische Raum $E_n$
eingebettet sind. Es ist nicht abträglich, sich dabei zunächst immer
die euklidische Ebene $E_2$ vorzustellen.

Wir haben also ein unendlich weit ausgedehntes leeres Blatt Papier
vor uns. Auf dieses Blatt wird nun mit einem Stift eine Linie gezogen.
Jedem Zeitpunkt $t$ wird dabei ein Punkt $p=c(t)$ zugeordnet.
Man bezeichnet $c$ als \emph{Parameterkurve} mit Parameter $t$.
Es ist nun so, dass beim Ziehen der Linie keine instantanen Sprünge
gemacht werden. Daher kann gefordert werden, dass $c$ eine
stetige Abbildung sein soll.

\begin{definition}[Parameterkurve, Kurve]%
\index{Parameterkurve}\index{Kurve}\mbox{}\\*
Sei $I$ ein reelles Intervall und $X$ ein topologischer Raum.
Eine stetige Abbildung%
\begin{equation}
c\colon I\to X
\end{equation}
wird als \emdef{Parameterkurve} bezeichnet. Die Bildmenge
$c(I)\subseteq X$ wird \emdef{Kurve} genannt.
\end{definition}

\noindent
Zunächst betrachten wir nur $X=E_n$, speziell $X=E_2$. Als Intervall
sind z.\,B.%
\[I=\R,\quad I=(a,b),\quad I=[a,b],\quad
I=[a,b),\quad I=[a,\infty)\]
erlaubt.

Zur Angabe einer Parameterkurve im euklidischen Punktraum $E_n$ kann
aufgrund der in Abschnitt \ref{section:Koordinatensysteme} erläuterten
Zusammenhänge einfach äquivalent eine Kurve $c\colon I \to\R^n$
betrachtet werden. Dann ergibt sich die Parameterkurve
$(\varphi\circ c)\colon I\to E_n$,
wobei $\varphi\colon \R^n\to E_n$ eine bijektive affine Abbildung ist.

Beim Ziehen der Linie kann es doch sein, dass wir diese mit sich selbst
überschneiden. Bei einer injektiven Parameterkurve wird das niemals
der Fall sein. Trotzdem soll es aber erlaubt sein, wenn Anfangspunkt
und Endpunkt der Kurve übereinstimmen. Eine solche Kurve nennt man
\emph{einfach}.

\begin{definition}[Einfache Parameterkurve]\mbox{}\\*
Eine Parameterkurve $c\colon I\to X$ heißt \emdef{einfach} oder
\emdef{doppelpunktfrei}, wenn $c$ injektiv ist. Für ein kompaktes
Intervall $I=[a,b]$ ist auch $c(a)=c(b)$ erlaubt.
\end{definition}

\noindent
Das Ziehen der Linie mit einem Stift geschieht normalerweise in einer
endlichen Zeit. Eine Kurve, auf der man nach einer endlichen Zeit
von einem Anfangspunkt zu einem Zielpunkt kommt, wird auch als
Weg oder Pfad bezeichnet. Aus diesem Gedanken heraus
ergibt sich die folgende Definition.

\begin{definition}[Weg, Anfangspunkt, Endpunkt]\mbox{}\\*
Eine Parameterkurve $c\colon I\to\R$ wird für ein kompaktes
Intervall $I=[a,b]$ als \emdef{Weg} bezeichnet. Man nennt dann
$c(a)$ den \emdef{Anfangspunkt} und $c(b)$ den
\emdef{Endpunkt} des Weges.
\end{definition}

\noindent
Eine Kurve kann natürlich geschlossen sein, in dem Sinn, dass man
wieder dort ankommt, woher man kommt.

\begin{definition}[Geschlossener Weg]%
\index{geschlossener Weg}\mbox{}\\*
Ein Weg $c\colon [a,b]\to X$ heißt \emdef{geschlossen},
wenn $c(a)=c(b)$ ist, wenn also Anfangspunkt und
Endpunkt übereinstimmen.
\end{definition}

\subsection{Differenzierbarkeit}
Die Ableitung einer Kurve können wir uns als Information über die
Tangente der Kurve an einem Punkt auf dieser Kurve vorstellen. Dieses
Konzept lässt sich leicht von reellen Funktionen auf Parameterkurven
übertragen.

Sei $I$ ein offenes Intervall, welches $t_0$ enthält. Wenn die einseitige
Ableitung betrachtet wird, darf $t_0$ natürlich auch Randpunkt von $I$
sein. Sei $c\colon I\to E_n$ eine Parameterkurve im euklidischen
Raum. Ist nun $t$ eine weitere Stelle, dann ist die Strecke von
$c(t_0)$ nach $c(t)$ eine Sekante der Kurve.

Ist $p_0\in E_n$ ein Punkt im euklidischen Raum und $v\in V$ ein Vektor
aus dem dazugehörigen Verschiebungsvektorraum, dann ergibt sich
gemäß $p=p_0+v$ ein neuer Punkt, die Verschiebung von $p_0$ um $v$.
Um präzise zu sein, die Addition eines Punktes und eines Vektors ist
eine Gruppenaktion der Gruppe $(V,+)$ auf $E_n$. 

Umgekehrt kann die »Differenz der Punkte« als dieser Vektor $v$
verstanden werden:%
\begin{equation}
v = p-p_0 \iff p = p_0+v.
\end{equation}
Betrachtet man die Kurve am Punkt $c(t_0)$ unter einer hinreichend
starken Vergrößerung, sollte sie sich dort gut durch
eine Gerade approximieren lassen. D.\,h. es muss einen Vektor $v$
geben, so dass
\begin{equation}
c(t_0+h) \approx c(t_0)+hv.
\end{equation}
Umformung liefert $v \approx \tfrac{c(t_0+h)-c(t_0)}{h}$.
Die Approximation sollte umso genauer werden, je kleiner $h$ ist.
Nach der vorherigen Ausführung wissen wir auch, dass es sich bei
$v$ um einen Sekantenvektor handelt. Für $h\to 0$ bzw.
$t\to t_0$ mit $t:=t_0+h$ müsste $v$ dann die Richtung der Tangente
haben. Die in der Berechnung von $v$ enthaltene Skalarmultiplikation
mit $h^{-1}$ ändert nur die Länge. Ohne diese Änderung würde bei $h\to 0$
immer der Fall $|v|=0$ eintreten.
\begin{definition}[Differenzierbare Parameterkurve]%
\index{Tangentialvektor}\label{diff-Parameterkurve}\mbox{}\\*
Sei $c\colon I\to E_n$ eine Parameterkurve. Wenn der Grenzwert
\begin{equation}
c'(t_0) := \lim_{h\to 0} \frac{c(t_0+h)-c(t_0)}{h}
= \lim_{t\to t_0} \frac{c(t)-c(t_0)}{t-t_0}
\end{equation}
existiert, dann heißt $c$ an der Stelle $t_0$
\emdef{differenzierbar} und $c'(t_0)$ wird \emdef{Ableitung}
oder \emdef{Tangentialvektor} genannt. Eine Parameterkurve heißt
\emdef{differenzierbar}, wenn sie an jeder Stelle differenzierbar ist.
\end{definition}
Betrachtet man $t$ als die Zeit, dann handelt es sich beim
Tangentialvektor um die Momentangeschwindigkeit, mit der sich ein
Punkt auf der Kurve bewegt. Es kann nun sein, dass die Bewegung für
einen Zeitpunkt oder eine Weile lang zum stehen kommt, dass der
Tangentialvektor also verschwindet, d.\,h. zum Nullvektor wird.
Für wichtige Anwendungen der Differentialgeometrie muss dieser
Fall aber ausgeschlossen werden.
\begin{definition}[Reguläre Parameterkurve]%
\index{reguläre Parameterkurve}\mbox{}\\*
Eine Parameterkurve $c$ heißt \emdef{regulär} an der Stelle $t_0$,
wenn $c'(t_0)\ne 0$ ist. Eine Parameterkurve heißt \emdef{regulär},
wenn sie an jeder Stelle regulär ist.
\end{definition}

\noindent
Die Analogie zwischen reellen Funktionen und Parameterkurven
verschärft sich unter dem folgenden Satz.

\pagebreak
\begin{theorem}
Eine Parameterkurve $c\colon I\to\R^n$ ist genau dann
differenzierbar, wenn sie komponentenweise differenzierbar
ist. Es gilt
\begin{equation}
c'(t) = \sum_{k=1}^n x_k'(t)\,\mathbf e_k,
\quad\text{wobei}\; c(t)=(x_1,\ldots,x_n)=\sum_{k=1}^n x_k(t)\,\mathbf e_k.
\end{equation}
\end{theorem}

\noindent\strong{Beweis.}
Für den Differenzenquotient gilt:
\begin{gather}
\frac{c(t+h)-c(t)}{h}
= \frac{1}{h}\bigg(\sum_{k=1}^n x_k(t+h)\mathbf e_k-\sum_{k=1}^n x_k(t)\mathbf e_k\bigg)\\
= \frac{1}{h}\sum_{k=1}^n (x_k(t+h)-x_k(t))\mathbf e_k
= \sum_{k=1}^n \frac{x_k(t+h)-x_k(t)}{h}\mathbf e_k.
\end{gather}
Laut Definition \ref{diff-Parameterkurve} und
Satz \ref{lim-vektorwertige-Funktion} gilt dann
\begin{equation}
c'(t) = \lim_{h\to 0}\frac{c(t+h)-c(t)}{h}
= \sum_{k=1}^n \lim_{h\to 0}\frac{x_k(t+h)-x_k(t)}{h}\mathbf e_k
= \sum_{k=1}^n x_k'(t)\mathbf e_k.\;\qedsymbol
\end{equation}

\begin{theorem}
Eine Parameterkurve $(\varphi\circ c)\colon I\to E_n$ ist genau dann
differenzierbar, wenn die Darstellung $c\colon I\to\R^n$
komponentenweise differenzierbar ist. Es gilt
\begin{equation}
(\varphi\circ c)'(t_0) = L(c'(t_0))
= \sum_{k=1}^n x_k'(t) L(\mathbf e_k),
\end{equation}
wobei $c(t) = (x_1,\ldots,x_n) = \sum_{k=1}^n x_k(t)\,\mathbf e_k$.
Dabei ist $\varphi(x) = p_0+L(x)$, wobei $L$ eine bijektive lineare
Abbildung ist.
\end{theorem}

\noindent\strong{Beweis.}
Jede affine Abbildung $\varphi$ ist total differenzierbar. Die
Umkehrabbildung $\varphi^{-1}$ ist auch eine affine Abbildung und
somit ebenfalls differenzierbar.

Gemäß der Kettenregel muss mit $c$ auch $\varphi\circ c$
differenzierbar sein. Es gilt dann die Rechnung%
\begin{equation}
(\varphi\circ c)'(t) = (\mathrm d\varphi_{c(t)})(c'(t))
= L(c'(t)) = L(\sum_{k=1}^n x_k'(t)\mathbf e_k)
= \sum_{k=1}^n x_k'(t)L(\mathbf e_k).
\end{equation}
Dabei gilt $\mathrm d\varphi_x = \mathrm dL_x = L$, weil die lineare
Approximation einer linearen Abbildung einfach diese lineare Abbildung
ist.

Wenn nun $\varphi\circ c$ als differenzierbar vorausgesetzt wird,
muss gemäß Kettenregel auch $c=\varphi^{-1}\circ(\varphi\circ c)$
differenzierbar sein. Es gilt dann die Rechnung
\begin{equation}
c'(t) = (\varphi^{-1}\circ(\varphi\circ c))'(t)
= L^{-1}((\varphi\circ c)'(t)).\;\qedsymbol
\end{equation}

\pagebreak
\subsection{Parametertransformationen}

\begin{definition}[Umparametrisierung, Parametertransformation]%
\index{Parametertransformation}\index{Umparametrisierung}\mbox{}\\*
Sei $c\colon I\to X$ eine Parameterkurve und $\varphi\colon J\to I$
eine stetige und streng monotone Funktion. Man nennt
$\tilde c\colon J\to X$ mit $\tilde c:=c\circ\varphi$ dann
\emdef{Umparametrisierung} von $c$, wobei $\varphi$ die
\emdef{Parametertransformation} dazu ist. Ein streng monoton steigendes
$\varphi$ wird als \emdef{orientierungserhaltend} bezeichnet,
ein streng monoton fallendes als \emdef{orientierungsumkehrend}.

Man spricht von einer $C^k$"=Parametertransformation, wenn sowohl
$\varphi$ also auch $\varphi^{-1}$ aus $C^k$ sind. Man betrachtet
auch $C^\infty$, die glatten Parametertransformationen und
$C^\omega$, die reell-analytischen.
\end{definition}

\noindent
Wenn $\tilde c$ eine Umparametrisierung ist, gilt natürlich
$\Bild\tilde c=\Bild c$, denn
\begin{equation}
\Bild\tilde c = \tilde c(J)
= (c\circ\varphi)(J) = (c)(\varphi(J))
= c(I) = \Bild c.
\end{equation}
Die Regel $(g\circ f)(A)=g(f(A))$ gilt für beliebige Abbildungen,
wie aus den Grundlagen der Mathematik bekannt sein sollte.

Die Umkehrung ist nicht allgemeingültig: Aus
$\Bild c=\Bild\tilde c$ folgt nicht zwingend, dass
$\tilde c$ eine Umparametrisierung von $c$ ist. Sei
$c(t)=(t,0)$ und $t\in[0,1]$. Als Gegenbeispiel wählt man
$\tilde c$ nun so, dass sich auch diese Strecke ergibt, die
Bewegung aber auch rückwärts verläuft, z.\,B. $\tilde c(t)=(\sin(t),0)$
mit $t\in[0,\pi]$. Die erste Komponente von $c(t)$ ist streng
monoton steigend. Da die Verkettung von streng monotonen Funktionen
auch wieder streng monoton ist, kann $\sin(t)$ niemals das Ergebnis
einer Parametertransformation sein.

\begin{corollary}
Jede Parametertransformation ist auch ein
Homöomorphismus. Die Umkehrfunktion ist auch eine
Parametertransformation. Jede $C^k$"=Parametertransformation
ist auch ein $C^k$"=Diffeomorphismus.
\end{corollary}

\noindent\strong{Beweis.}
Folgt trivial aus dem Umkehrsatz für streng monotone
Funktionen.\;\qedsymbol

\begin{corollary}
Ein $C^k$"=Diffeomorphismus $\varphi$ ist auch eine
$C^k$"=Parametertransformation.
\end{corollary}

\noindent\strong{Beweis.}
Eine bijektive stetige reelle Funktion muss streng monoton sein.
Daher ist $\varphi$ streng monoton.\;\qedsymbol

Man beachte, dass auch der Fall $C^0$ mit eingeschlossen ist.
Insgesamt bekommen wir das folgende übersichtliche Resultat.

\begin{corollary}[Charakterisierung von Parametertransformationen]\mbox{}\\*
Die $C^k$"=Diffeomorphismen zwischen reellen Intervallen sind
genau die $C^k$"=Parametertransformationen.
\end{corollary}

\begin{corollary}
Sei $\varphi\colon J\to I$ eine stetig differenzierbare
Funktion zwischen Intervallen. Ist $\varphi'(x)\ne 0$ für alle $x$,
dann ist $\varphi$ bereits eine Parametertransformation.
\end{corollary}

\noindent\strong{Beweis.}
Da $\varphi'$ stetig ist, muss es nach dem Zwischenwertsatz beim
Vorzeichenwechsel eine Stelle $x$ mit $\varphi'(x)=0$ geben. Da
dies ausgeschlossen ist, gilt entweder $\varphi'(x)>0$ für alle $x$
oder $\varphi'(x)<0$ für alle $x$. Somit ist $\varphi$ streng monoton.
Da $\varphi$ stetig differenzierbar ist, ist es erst recht
stetig.\;\qedsymbol

Umparametrisierungen ändern die Bildmenge nicht und lassen wohl auch
andere geometrische Eigenschaften unverändert. Zwar wird sich
beim schnelleren durchlaufen der Kurve ein größerer Tangentialvektor
ergeben, die Tangente bleibt aber gleich. Auch der normierte
Tangentialvektor bleibt gleich, wenn die Parametertransformation
orientierungserhaltend ist. Diese Überlegungen motivieren das folgende
Konzept.

\begin{definition}[Geometrische Kurve]\mbox{}\\*
Zwei Parameterkurven seien in Relation, wenn die eine eine
Umparametrisierung der anderen ist, wobei die Parametertransformation
glatt sein soll. Hierdurch ist eine 
Äquivalenzrelation gegeben. Die Äquivalenzklasse nennt man
\emdef{geometrische Kurve}.
\end{definition}

\begin{definition}[Orientierte Kurve]%
\index{orientierte Kurve}\mbox{}\\*
Eine \emdef{orientierte Kurve} ist das Analogon zu einer
geometrischen Kurve, wobei man sich auf orientierungserhaltende
Parametertransformationen beschränkt.
\end{definition}

\noindent
Wie üblich schreiben wir $[c]$ für die Äquivalenzklasse zum
Repräsentanten $c$. Wir wollen eine Eigenschaft als
\emph{geometrisch} bezeichnen, wenn sie für eine geometrische Kurve
wohldefiniert ist, d.\,h. unabhängig von der Wahl des Repräsentanten.

Es folgt ein einfaches Beispiel.

\begin{corollary}
Sei $c$ doppelpunktfrei. Der Tangentialraum
\begin{equation}
T_p[c] := \{rc'(t_0)\mid r\in\R\}
\quad\text{für ein $t_0$ mit $p=c(t_0)$}
\end{equation}
ist ein geometrisches Konzept. Regularität ist eine geometrische
Eigenschaft, d.\,h. für eine reguläre Parameterkurve gilt
$\dim T_p[c]=1$.
\end{corollary}

\noindent\strong{Beweis.}
Wir zeigen einfach, dass der Tangentialvektor nach Umparametrisierung
kollinear zum ursprünglichen Tangentialvektor ist. Gemäß der
Kettenregel ergibt sich:%
\begin{equation}
w = (c\circ\varphi)'(t_0) = (c'\circ\varphi)(t_0)\cdot \varphi'(t_0).
\end{equation}
Sei $v=(c'\circ\varphi)(t_0)$. Gemäß der Definition der
Parametertransformation ist $r=\varphi'(t_0)\ne 0$. Demnach
gilt $w=rv$, was wegen $r\ne 0$ eine kollineare Beziehung zwischen
$w$ und $v$ ist. Wenn $c$ regulär ist, muss $v\ne 0$ sein.
Wegen $r\ne 0$ ist dann aber auch $w\ne 0$.\;\qedsymbol


\subsection{Rektifizierbare Wege}

Wir wollen nun versuchen, die Länge eines Weges zu ermitteln.
Der Gedankengang ist, dass sich ein Weg durch einen Polygonzug
approximieren lassen müsste. Sei also $c\colon [a,b]\to X$
ein Weg und $(X,d)$ ein metrischer Raum. Sei durch%
\begin{equation}
P:=(t_0,\ldots,t_m),\quad a=t_0<t_1<\ldots <t_m=b
\end{equation}
eine Partition (Zerlegung) von $[a,b]$ gegeben. Dann ergibt sich
über die Knoten $c(t_k)$ ein Polygonzug. Dessen Länge ergibt
sich gemäß%
\begin{equation}
L(c,P) := \sum_{k=0}^{m-1} d(c(t_{k+1}),c(t_k)).
\end{equation}
Wenn man nun die Partition immer weiter verfeinert, dann sollte
sich die Länge des Polygonzuges der Länge des Weges nähern, sofern
dieser Weg überhaupt eine Länge besitzt.
\begin{definition}[Länge, rektifizierbarer Weg]%
\index{rektifizierbarer Weg}\index{Bogenlänge}\mbox{}\\*
Die \emdef{Länge} eines Weges $c$ ist definiert als%
\begin{equation}
L_a^b(c) := \sup\nolimits_P L(c,P), \quad P=(a,\ldots,b).
\end{equation}
Ein Weg mit endlicher Länge wird \emdef{rektifizierbar} genannt.
\end{definition}
Das ist eine typische dieser unzugänglich erscheinenden Definitionen.
Ein Satz mit einer praktischen Formel zur Berechnung gelangt uns
aber sogleich in die Hände.
\begin{theorem}
Ein stetig differenzierbarer Weg $c\colon [a,b]\to\R^n$ ist
rektifizierbar und es gilt%
\begin{equation}\label{eq:Laenge}
L_a^b(c) = \int_a^b |c'(t)|\,\mathrm dt.
\end{equation}
\end{theorem}

\noindent\strong{Beweis.} Wir folgen der Argumentation
aus \cite{Kriegl}. Nach dem Hauptsatz der Analysis und der
Dreiecksungleichung gilt
\begin{gather}
L(c,P) = \sum_{k=0}^{m-1} |c(t_{k+1})-c(t_k)|
= \sum_{k=0}^{m-1} \Big|\int_{t_k}^{t_{k+1}}c'(t)\,\mathrm dt\Big|\\
\le \sum_{k=0}^{m-1} \int_{t_k}^{t_{k+1}} |c'(t)|\,\mathrm dt
= \int_a^b |c'(t)|\,\mathrm dt.
\end{gather}
Da $c'(t)$ nach Voraussetzung stetig ist, ist auch $|c'(t)|$
stetig. Demnach nimmt das Integral einen endlichen Wert an. Also ergibt
sich
\begin{gather}
L_a^b(c) \le \int_a^b |c'(t)|\,\mathrm dt < \infty.
\end{gather}
Die Länge der geraden Strecke kann gemäß Dreiecksungleichung niemals
länger sein, als die eines Polygonzuges. Das heißt, es muss
$|c(t+h)-c(t)|\le L_t^{t+h}(c)$ sein. Demnach
gilt die Abschätzung
\begin{equation}
\frac{|c(t+h)-c(t)|}{h}\le L_t^{t+h}(c)
\le\frac{1}{h}\int_t^{t+h} |c'(t)|\,\mathrm dt.
\end{equation}
Beide Seiten konvergieren gegen $|c'(t)|$ für $h\to 0$.
Gemäß dem Einschnürungssatz muss auch der mittlere Term gegen diesen
Wert konvergieren. Demnach gilt
\begin{equation}
|c'(t)| = \lim_{h\to 0}\frac{L_t^{t+h}(c)}{h}
= \lim_{h\to 0}\frac{L_a^{t+h}(c)-L_a^t(c)}{h}
= \frac{\mathrm d}{\mathrm dt} L_a^t(c).
\end{equation}
Ziehen wir nochmals den Hauptsatz der Analysis heran, dann ergibt
sich das gewünschte Resultat:
\begin{equation}
L_a^b(c) = \int_a^b \frac{\mathrm d}{\mathrm dt} L_a^t(c)\,\mathrm dt
= \int_a^b |c'(t)|\,\mathrm dt.\;\qedsymbol
\end{equation}

\begin{theorem}
Die Länge eines stetig differenzierbaren Weges ist ein geometrisches
Konzept.
\end{theorem}

\noindent\strong{Beweis.}
Sei $c\colon [a,b]\to\R^n$ ein stetig differenzierbarer Weg und
und $\tilde c\colon [\alpha,\beta]\to\R^n$ eine Umparametrisierung
gemäß $\tilde c = c\circ\varphi$, wobei $\varphi$ eine
orientierungserhaltende und stetig differenzierbare Parametertransformation
ist. Nach der Kettenregel und wegen $\varphi'(t)>0$ ist zunächst
\begin{gather}
|\tilde c'(t)| = |(c'\circ\varphi)(t)\cdot\varphi'(t)|
= |(c'\circ\varphi)(t)|\cdot\varphi'(t).
\end{gather}
Unter Bemühung der Substitutionsregel ergibt sich
\begin{gather}
L(\tilde c) = \int_\alpha^\beta |\tilde c'(t)|\,\mathrm dt
= \int_\alpha^\beta |(c'\circ\varphi)(t)|\,\varphi'(t)\,\mathrm dt
= \int_a^b |c'(\varphi)|\,\mathrm d\varphi = L(c).
\end{gather}
Die Länge ist also nicht vom gewählten Repräsentanten
abhängig.\;\qedsymbol

\subsection{Parametrisierung nach Bogenlänge}

Sei $c\colon\R\to\R^n$ eine reguläre und stetig differenzierbare
Parameterkurve. Die Bogenlängenfunktion $L(t):=L_a^t(c)$ nach
\eqref{eq:Laenge} ist als Integralfunktion erst recht stetig.
Mit $c'(t)\ne 0$ gilt auch $|c'(t)|>0$. Daher ist $L(t)$ streng monoton
steigend. Nach dem Umkehrsatz für streng monotone Funktionen gibt es
die Umkehrfunktion $\varphi(s):=L^{-1}(s)$. Diese lässt sich als
Parametertransformation verwenden, und $(c\circ\varphi)(s)$ wird
\emph{Parametrisierung nach der Bogenlänge} genannt.

\begin{corollary}
Sei $(c\circ\varphi)(s)$ nach Bogenlänge parametrisiert
und $t=\varphi(s)$. Es gilt
\begin{equation}
\frac{\mathrm dc}{\mathrm ds} =
(c\circ\varphi)'(s) = \frac{c'(t)}{|c'(t)|}.
\end{equation}
\end{corollary}
\noindent\strong{Beweis.}
Aus $t=(\varphi\circ L)(t)$ folgt
\begin{equation}
1 = (\varphi\circ L)'(t) = (\varphi'\circ L)(t)\,L'(t) = \varphi'(s)L'(t).
\end{equation}
Nach dem Hauptsitz ist $L'(t)=|c'(t)|$. Es folgt
\begin{equation}
(c\circ\varphi)'(s) = (c'\circ\varphi)(s)\,\varphi'(s) = c'(t)\varphi'(s)
= \frac{c'(t)}{|c'(t)|}.\;\qedsymbol
\end{equation}
Wir sehen hier also, dass die Tangentialvektoren von
Bogenlänge"=parametrisierten Kurven automatisch normiert sind.

\subsection{Kurvenintegrale}
\begin{definition}[Kurvenintegral eines Skalarfeldes]%
\index{Kurvenintegral}\mbox{}\\*
Sei $U\subseteq\R^n$ und $f\colon U\to\R$.
Sei $\gamma\colon [a,b]\to U$ stetig differenzierbar. Man nennt dann%
\[\int_\gamma f(x)\,\mathrm ds := \int_a^b f(\gamma(t))\cdot |\gamma'(t)|\,\mathrm dt\]
das Kurvenintegral von $f$ über $\gamma$.
Ist $\gamma$ zumindest stückweise stetig differenzierbar und
$\gamma = \gamma_1\oplus\ldots\oplus\gamma_n$ eine Zerlegung
in die stetig differenzierbaren Stücke $\gamma_k$, dann setzt man%
\[\int_\gamma f(x)\,\mathrm ds := \sum_{k=1}^n \int_{\gamma_k} f(x)\,\mathrm ds.\]
\end{definition}
Wie bei der Bogenlänge ist das Kurvenintegral unabhängig von
der gewählten Parametrisierung, dafür muss die Kurve aber regulär sein.
Die Bogenlänge erhält man als Spezialfall $f(x):=1$.

\begin{definition}[Kurvenintegral eines Vektorfeldes]\mbox{}\\*
Sei $U\subseteq\R^n$ und $F\colon U\to\R^n$.
Sei $\gamma\colon [a,b]\to U$ stetig differenzierbar. Man nennt dann
\begin{equation}\label{eq:Kurvenintegral-Vektorfeld}
\int_\gamma\langle F(x),\mathrm dx\rangle
:= \int_a^b \langle F(\gamma(t)),\gamma'(t)\rangle\,\mathrm dt
\end{equation}
das Kurvenintegral von $F$ über $\gamma$.
\end{definition}
Diese Formulierung ist speziell auf den Koordinatenraum bezogen. Zur
späteren Übertragung dieses Begriffs auf Mannigfaltigkeiten ist
es günstig, wenn die Formulierung einmal zur Veranschaulichung in
abstrakte Form gebracht wird. Sei dazu $A$ ein affiner Raum mit
Verschiebungsvektorraum $V$. Sei $U\subseteq A$. Es gibt auf
$V$ kein Skalarprodukt. Nun kann man aber formal rechnen%
\begin{equation}\label{eq:Kurvenintegral-1-Form}
\omega(x) := \langle F(x),\mathrm dx\rangle = \sum_{k=1}^n F_k(x)\,\mathrm dx_k.
\end{equation}
Das heißt, $\omega\colon U\to V^*$ ist eine 1-Form, auch Kovektorfeld
genannt. Nun definiert man%
\begin{equation}
\int_\gamma \omega
:= \int_a^b \omega(\gamma(t))(\gamma'(t))\,\mathrm dt.
\end{equation}
Diese Formulierung ist nicht mehr von einer speziellen Basis von $V$
abhängig, sofern wir vergessen dass $\omega$ als Linearkombination
aus den $\mathrm dx_k$ aufgebaut ist. Und für eine Orthonormalbasis
$(e_k)$ ergibt sich gemäß $\mathrm dx_i(e_j) = \langle e_i,e_j\rangle$
wieder Formel \eqref{eq:Kurvenintegral-Vektorfeld}, was die von der
Notation geleitete Überlegung \eqref{eq:Kurvenintegral-1-Form}
rechtfertigt. Zur Wiederholung: Die Gleichung
$\delta_{ij}=\mathrm dx_i(e_j)$ ist definitionsgemäß allgemeingültig,
die Gleichung $\delta_{ij}=\langle e_i,e_j\rangle$
jedoch nur bei einer Orthonormalbasis.

Das Kurvenintegral eines Skalarfeldes lässt sich unter einem gewissen
Vorbehalt als Kurvenintegral eines Vektorfeldes darstellen.
Ein Vergleich der rechten Seiten der Definitionen bringt uns zur
Überlegung, dass die Gleichung
\begin{equation}
\langle F(\gamma(t)),\gamma'(t)\rangle = f(\gamma(t))\, |\gamma'(t)|
\end{equation}
erfüllt sein muss. Aus einem genauen Blick auf diese Gleichung liest
man das Muster $\langle v,w\rangle = |v|\cdot|w|$ heraus, dazu muss
aber $v$ gleichsinnig parallel zu $w$ sein. Daher muss $F$ auf
$\gamma([a,b])$ die Form $F(\gamma(t)) = f(\gamma(t))T(t)$ haben,
wobei $T(t)=\frac{\gamma'(t)}{|\gamma'(t)|}$ der
Tangenteneinheitsvektor ist. Nachrechnen bringt in der Tat
\begin{equation}
\langle F(\gamma),\gamma'\rangle
= \langle f(\gamma)\,\frac{\gamma'}{|\gamma'|},\gamma'\rangle
= f(\gamma)\,\frac{|\gamma'|^2}{|\gamma'|}
= f(\gamma)\,|\gamma'|.
\end{equation}
Der Vorbehalt ist nun der Umstand, dass hier $|\gamma'|$ im
Nenner vorkommt. Demnach würde der Integrand an einer Stelle $t$ mit
$\gamma'(t)=0$ eine Singularität haben. Bei Vorhandensein von
Singularitäten ist das bestimmte Integral aber nicht definiert.
Demnach muss $\gamma$ eine reguläre Kurve sein. Außerdem sind nur
orientierungserhaltende Umparametrisierungen zugelassen, denn
das Kurvenintegral eines Vektorfeldes ist von der Orientierung
der Kurve abhängig, das Kurvenintegral eines Skalarfeldes jedoch
nicht.

\section{Krümmung}
\subsection{Krümmung ebener Kurven}

Die Krümmung einer Kurve lässt sich definieren als das Maß für
Richtungsänderung einer Kurve. Desto höher die Krümmung an einem
bestimmten Punkt ist, desto schneller ändert sich dort die
Richtung beim Durchlaufen der Kurve. Da die Richtung an einem
Punkt durch den Tangentialvektor gegeben ist und Änderung
durch die Ableitung quantifiziert wird, müsste die Krümmung
als Betrag der Ableitung des Tangentialvektorfeldes nach der Zeit
gegeben sein. Da die Durchlaufgeschwindigkeit dabei keine Rolle spielen
soll, muss die Kurve vorher nach Bogenlänge parametrisiert werden.

Alternativ lässt sich zunächst die Krümmung eines Kreises definieren.
Ein Kreis ist umso stärker gekrümmt, je kleiner er ist. Demnach lässt
sich die Krümmung einfach als Kehrwert des Radius definieren. Die
Krümmung einer Kurve an einem Punkt ist dann die Krümmung des Kreises
der sich der Kurve an dem Punkt am besten anschmiegt.

Wie im Folgenden gefunden wird, sind diese beiden Ansätze äquivalent.

Sei $\gamma\colon\R\to\R^2$ eine reguläre und zweimal stetig
differenzierbare Parameterkurve. Sei
\begin{equation}
c\colon\R\to\R^2,\quad c(t)=\begin{bmatrix}
x_0+R\cos(\omega t+\varphi_0)\\
y_0+R\sin(\omega t+\varphi_0)
\end{bmatrix}.
\end{equation}
Die Kreiskurve $c(t)$ soll nun an der Stelle $t=a$ an $\gamma(t)$
geschmiegt sein. Dazu muss notwendigerweise $c(a)=\gamma(a)$ sein.
Außerdem müssen die Tangentialvektoren dort gleich sein, also
$\gamma'(a) = c'(a)$. Das allein genügt jedoch noch nicht. Der
Kreis sollte sich zweiter Ordnung dort anschmiegen. Wenn die
Taylorpolynome zweiten Grades übereinstimmen, muss auch
$\gamma''(a) = c''(a)$ sein. Nun ist es aber so, dass die $\gamma$
und $c$ eine unterschiedliche Durchlaufgeschwindigkeit haben können.
Man könnte verlangen, dass beide Kurven nach Bogenlänge parametrisiert
sind. Für den Kreis ergibt sich dabei
\begin{equation}
s = \int_0^t |c'(t)|\,\mathrm dt = \int_0^t R\omega\,\mathrm dt
= R\omega t \implies \omega t = \frac{s}{R}.
\end{equation}
Es ergibt sich nun
\begin{align}
x'(s) &= -\sin(\tfrac{s}{R}+\varphi_0),
& x''(s) &= -\tfrac{1}{R}\cos(\tfrac{s}{R}+\varphi_0),\\
y'(s) &= \cos(\tfrac{s}{R}+\varphi_0),
& y''(s) &= -\tfrac{1}{R}\sin(\tfrac{s}{R}+\varphi_0).
\end{align}
Um an das $R$ zu gelangen, findet man $x''(s)=-\tfrac{1}{R}y'(s)$
und $y''(s)=\tfrac{1}{R}x'(s)$. Nun kann es aber sein, dass
$x'(s)=0$ oder $y'(s)=0$ ist. Dann müsste man zur Berechnung der
Krümmung $K = 1/R$ die jeweils andere Formel benutzen, um
nicht durch null zu dividieren. Eine allgemeingültige Formel
ergibt sich, wenn jeweils ein Faktor $x'(s)$ und $y'(s)$ in
die Beziehung $x'(s)^2+y'(s)^2=1$ substituiert wird. Es ergibt
sich dann $Rx'(s)y''(s)-Rx''(s)y'(s)=1$ und daher
\begin{equation}\label{eq:Kruemmung-mit-Vorzeichen}
K = \frac{1}{R} =
\begin{vmatrix}
x'(s) & x''(s)\\
y'(s) & y''(s)
\end{vmatrix}
= x'(s)y''(s)-x''(s)y'(s).
\end{equation}
Substituiert man beide Faktoren, ergibt sich stattdessen
\begin{equation}
(Ry''(s))^2+(Rx''(s))^2 = 1
\implies K = \frac{1}{R} = \sqrt{x''(s)^2+y''(s)^2}.
\end{equation}
Demnach ist die Krümmung tatsächlich die Änderung des
Tangentialvektors:
\begin{equation}\label{eq:Kruemmung-ohne-Vorzeichen}
K = |\gamma''(s)| = |T'(s)|.
\end{equation}
Genauer genommen ist die Krümmung \eqref{eq:Kruemmung-mit-Vorzeichen}
vorzeichenbehaftet, wodurch Information darüber enthalten ist, ob
die Kurve nach links oder rechts gekrümmt ist. Die Formel
\eqref{eq:Kruemmung-ohne-Vorzeichen} ergibt jedoch nur den Betrag
der Krümmung.

Die Krümmung lässt sich auch von solchen Parameterkurven angeben,
die nicht nach der Bogenlänge parametrisiert sind. Dazu substituiert
man $s=L(t)$, womit sich $\gamma'(s) = (\gamma'\circ L)(t)$ und
$\gamma''(s)=(\gamma''\circ L)(t)$ ergibt. Die Aufgabe ist es nun, diese
Ableitungen bezüglich $t$ darzustellen. Sei $\tilde\gamma = \gamma\circ L$.
Nach der Kettenregel gilt%
\begin{equation}
\tilde\gamma'(t) = (\gamma'\circ L)\,L'(t) = \gamma'(s)\,|\tilde\gamma'(t)|
\implies \gamma'(s) = \frac{\tilde\gamma'(t)}{|\tilde\gamma'(t)|}.
\end{equation}
Bei $\gamma''(s)$ ist es etwas komplizierter:
\begin{align}
\tilde\gamma''(t)
&= (\gamma''{\circ}L)(t)\,L'(t)^2+(\gamma'{\circ}L)(t)\,L''(t)\\
&= \gamma''(s)\,L'(t)^2+\gamma'(s)\,L''(t).
\end{align}
Mit $L''(t)=\langle\tilde\gamma',\tilde\gamma''\rangle/|\tilde\gamma'|$
ergibt sich
\begin{equation}
\gamma''(s) = \frac{\tilde\gamma''(t)-\gamma'(s)L''(t)}{L'(t)^2}
= \frac{\tilde\gamma''(t)-\gamma'(s)L''(t)}{|\gamma'(t)|^2}
= \frac{
|\tilde\gamma'|^2\tilde\gamma''
-\tilde\gamma'\langle\tilde\gamma',\tilde\gamma''\rangle
}{|\tilde\gamma'|^4}.
\end{equation}
Daraus erhält man via $|v-w|^2 = |v|^2-2\langle v,w\rangle+|w|^2$ nach
dem Kürzen
\begin{equation}
|\gamma''(s)|^2 = \frac{
|\tilde\gamma'|^2|\tilde\gamma''|^2
-\langle\tilde\gamma',\tilde\gamma''\rangle^2
}{|\tilde\gamma'|^6}.
\end{equation}
Entsprechend dem Formalismus in \eqref{eq:vol} und \eqref{eq:vol-gram}
ergibt sich schließlich
\begin{equation}\label{eq:Kruemmung-t}
K = |\gamma''(s)| = \frac{|\tilde\gamma'(t)\wedge\tilde\gamma''(t)|}{|\tilde\gamma'(t)|^3}
= \frac{\sqrt{\det(A^T A)}}{|\tilde\gamma'(t)|^3},
\end{equation}
wobei $A=(\tilde\gamma'(t),\tilde\gamma''(t))$ ist.

Speziell in der Ebene ist die
Matrix $A$ quadratisch, womit sich $\sqrt{\det(A^T A)}$
zu $|\det(A)|$ vereinfacht. Die Betragsstriche können fallengelassen
werden um auch zu kodieren, ob die Kurve nach links oder rechts
gekrümmt ist. Für $\tilde\gamma(t)=(x(t),y(t))$ ergibt
sich die Formel%
\begin{equation}\label{eq:Kruemmung-t-Ebene}
K = \frac{\det(A)}{|\tilde\gamma'(t)|^3}
= \frac{|x'(t)y''(t)-y'(t)x''(t)|}{(x'(t)^2+y'(t)^2)^{3/2}}.
\end{equation}
Der Graph einer differenzierbare Funktion $f\colon I\to \R$ lässt sich
als Parameterkurve $(x(t),y(t))=(t,f(t))$ darstellen. Für die
Krümmung ergibt sich entsprechend
\begin{equation}
K = \frac{f''(x)}{(1+f'(x)^2)^{3/2}}.
\end{equation}

\subsection{Krümmung beliebiger Kurven}

\begin{definition}[Krümmung einer Parameterkurve]\mbox{}\\*
Sei $\gamma\colon I\to \R^n$ eine reguläre zweimal stetig
differenzierbare Parameterkurve, welche nach der Bogenlänge
parametrisiert ist. Die \emdef{Krümmung} von $\gamma$ an der Stelle
$s$ ist definiert durch%
\begin{equation}
K = |\gamma''(s)|.
\end{equation}
\end{definition}

\begin{theorem}[Berechnungsformel für die Krümmung]\mbox{}\\*
Sei $\gamma\colon I\to\R^n$ eine reguläre zweimal stetig
differenzierbare Parameterkurve. Die Krümmung von $\gamma$ an der
Stelle $t$ ist%
\begin{equation}
K = \frac{|\gamma'(t)\wedge\gamma''(t)|}{|\gamma'(t)|^3}
= \frac{\sqrt{\det(A^T A)}}{|\gamma'(t)|^3},
\end{equation}
wobei $A=(\gamma'(t),\gamma''(t))$.
\end{theorem}
\strong{Beweis.} Die Rechnung zum Resultat \eqref{eq:Kruemmung-t}
lässt sich unverändert übernehmen.\;\qedsymbol

Im $\R^3$ gilt speziell $|v\wedge w| = |v\times w|$,
weshalb in der älteren Literatur oft die Formel%
\begin{equation}
K = \frac{|\gamma'(t)\times\gamma''(t)|}{|\gamma'(t)|^3}
\end{equation}
zu finden ist.

\begin{theorem}[Lagrange-Identität]
Für zwei Vektoren $a,b\in\R^n$ gilt
\begin{equation}
|a\wedge b|^2 = |a|^2 |b|^2 - \langle a,b\rangle^2
= \sum_{i<j} (a_i b_j - a_j b_i)^2.
\end{equation}
\end{theorem}
\strong{Beweis.}
Es gilt
\begin{align}
&\sum_{i<j} (a_i b_j - a_j b_i)^2
= \frac{1}{2}\sum_{i,j} (a_i b_j - a_j b_i)^2\\
&= \frac{1}{2}\sum_{i,j} (a_i^2 b_j^2 - 2a_i b_i a_j b_j + a_j^2 b_i^2)
= \frac{1}{2}\bigg(2\sum_{i,j} a_i^2 b_j^2 - 2\sum_{i,j}a_i b_i a_j b_j\bigg)\\
&= \bigg(\sum_{i=1}^n a_i^2\bigg)\bigg(\sum_{j=1}^n b_j^2\bigg)
- \bigg(\sum_{i=1}^n a_i b_i\bigg)\bigg(\sum_{j=1}^n a_j b_j\bigg)\\
&= |a|^2 |b|^2 - \langle a,b\rangle^2.\;\qedsymbol
\end{align}
Die Lagrange"=Identität gilt sogar für $a,b\in\C^n$, der Beweis
fällt dann aber ein wenig komplizierter aus.

\section{Die frenetschen Formeln}

Sei $\gamma\colon\R\to\R^n$ eine nach Bogenlänge parametrisierte
zweimal stetig differenzierbare Kurve. Aus $|\gamma'(s)|=1$ folgt
$|\gamma'(s)|^2=1$ und somit%
\begin{equation}\label{eq:orthogonale-Ableitungen}
0 = \frac{\mathrm d}{\mathrm ds} |\gamma'(s)|^2
= \langle \gamma'(s),\gamma''(s)\rangle + \langle\gamma''(s),\gamma'(s)\rangle
= 2\langle\gamma'(s),\gamma''(s)\rangle.
\end{equation}
Es ist also $\langle\gamma'(s),\gamma''(s)\rangle=0$
bzw. $\gamma'(s)\perp\gamma''(s)$. Diese Beziehung ist aber nur dann
allgemeingültig, wenn eine nach Bogenlänge parametrisierte Kurve
vorliegt. Dieser Zusammenhang lässt sich auch noch physikalisch
interpretieren. Die Bewegung $\gamma(t)$ eines Teilchens besitzt den
Geschwindigkeitsvektor $v=\gamma'(t)$. Außerdem erfährt das Teilchen
der Masse $m$ die Beschleunigung $a=\gamma''(t)$ durch den
Kraftvektor $F=ma$. Aus rein mathematischen Gründen gilt%
\begin{equation}
\frac{\mathrm d}{\mathrm dt}(\tfrac{1}{2}m |v|^2)
= \langle m\gamma''(t),\gamma'(t)\rangle = \langle ma,v\rangle
= \langle F,v\rangle.
\end{equation}
Der Kraftvektor $F$ lässt sich nun zerlegen in eine Komponente in
Richtung von $v$ und eine orthogonal dazu. Bei Parametrisierung
nach der Bogenlänge durchläuft das Teilchen seine Bahn mit konstanter
Geschwindigkeit $|v|$. Demnach bleibt die kinetische Energie
$\tfrac{1}{2}m|v|^2$ erhalten,
was zur Folge hat, dass $F$ orthogonal auf $v$ steht. Die physikalische
Deutung ist also, dass das Teilchen keine Beschleunigung in
tangentialer Richtung erfährt.

\begin{definition}[Begleitendes Zweibein]\mbox{}\\*
Sei $\gamma\colon I\to\R^2$ eine zweiter Ordnung reguläre
nach Bogenlänge parametrisierte Kurve. Dann ist das
\emdef{begleitende Zweibein} gegeben gemäß
\begin{align}
T(s) &:= \gamma'(s), && \text{(\emph{Tangenteneinheitsvektor})}\\
N(s) &:= \frac{T'(s)}{|T'(s)|}. && \text{(\emph{Normaleneinheitsvektor})}
\end{align}
\end{definition}

\begin{theorem}[Frenetsche Formeln für ebene Kurven]\mbox{}\\*
Sei $\gamma\colon I\to\R^2$ eine zweiter Ordnung reguläre nach
Bogenlänge parametrisierte Kurve. Sei $K$ die betragsmäßige
Krümmung und $T,N$ das begleitende Zweibein. Dann gilt%
\begin{equation}
\begin{bmatrix}
T'(s)\\ N'(s)
\end{bmatrix}
= \begin{bmatrix}
K(s)  & 0\\
0 & -K(s)
\end{bmatrix}\begin{bmatrix}
N(s)\\ T(s)
\end{bmatrix}.
\end{equation}
\end{theorem}

\noindent\strong{Beweis.}
Die Formel $T'(s) = K(s)N(s)$ bekommt man wegen $K(s)=|T'(s)|$
geschenkt. Gemäß \eqref{eq:orthogonale-Ableitungen} gilt
$\langle T(s),N(s)\rangle=0$. Demnach muss
\begin{equation}
N(s) = (\pm 1)\begin{bmatrix}0 & -1\\ 1 & 0\end{bmatrix}T(s)
\end{equation}
sein. Daraus ergibt sich
\begin{align}
N'(s) &= (\pm 1)\begin{bmatrix}0 & -1\\ 1 & 0\end{bmatrix}T'(s)
= (\pm 1)\begin{bmatrix}0 & -1\\ 1 & 0\end{bmatrix}K(s)N(s)\\
&= K(s)(\pm 1)(\pm 1)\begin{bmatrix}0 & -1\\ 1 & 0\end{bmatrix}
  \begin{bmatrix}0 & -1\\ 1 & 0\end{bmatrix}T(s)
= K(s)\begin{bmatrix}-1 & 0\\ 0 & -1\end{bmatrix}T(s)\\
&= -K(s)T(s).\;\qedsymbol
\end{align}

\begin{definition}[Begleitendes Dreibein]%
\index{Dreibein}\index{begleitendes Dreibein}%
\index{Tangenteneinheitsvektor}\index{Hauptnormaleneinheitsvektor}%
\index{Binormaleneinheitsvektor}\mbox{}\\*
Sei $\gamma\colon I\to\R^3$ eine zweiter Ordnung reguläre
nach Bogenlänge parametrisierte Kurve. Dann ist das \emdef{begleitende
Dreibein} gegeben gemäß
\begin{align}
T(s) &:= \gamma'(s), && \text{(\emph{Tangenteneinheitsvektor})}\\
N(s) &:= \frac{T'(s)}{|T'(s)|}, && \text{(\emph{Hauptnormaleneinheitsvektor})}\\
B(s) &:= T(s)\times N(s). && \text{(\emph{Binormaleneinheitsvektor})}
\end{align}
\end{definition}

\begin{theorem}\label{Bstrich-kollinear-zu-N}
Die Vektoren $B'(s)$ und $N(s)$ sind kollinear.
\end{theorem}

\noindent\strong{Beweis.}
Gezeigt wird $B'\times N=0$. Mittels Produktregel und $N\times N=0$
erhält man zunächst%
\begin{equation}
B' = (T\times N)' = T'\times N+T\times N'
= K_1 N\times N + T\times N' = T\times N'.
\end{equation}
Unter Heranziehung der allgemeingültigen Beziehung
\begin{equation}\label{eq:Grassmann-Identitaet}
(a\times b)\times c = \langle a,c\rangle b - \langle b,c\rangle a
\end{equation}
ergibt sich nun
\begin{equation}
B'\times N = (T\times N')\times N
= \langle T,N\rangle N' - \langle N',N\rangle T.
\end{equation}
Der Vektor $N$ steht senkrecht auf $T$, d.h.
$\langle T,N\rangle=0$. Außerdem ist $T(s)\ne 0$ für alle $s$,
da die Kurve als regulär vorausgesetzt ist. Demnach bleibt lediglich
$\langle N',N\rangle=0$ zu zeigen. Nach $|N|=1$ und der Produktregel
gilt aber%
\begin{equation}
0 = \frac{\mathrm d}{\mathrm ds} 1 = \frac{\mathrm d}{\mathrm ds} |N|^2
= \frac{\mathrm d}{\mathrm ds} \langle N,N\rangle
= \langle N',N\rangle + \langle N,N'\rangle
= 2\langle N',N\rangle.\;\qedsymbol
\end{equation}


\begin{definition}[Torsion]%
\index{Torsion}\label{Torsion}\mbox{}\\*
Sei $\gamma\colon I\to\R^3$ eine dreimal differenzierbare und
zweiter Ordnung reguläre nach Bogenlänge parametrisierte Kurve.
Die \emdef{Torsion} ist definiert als der Faktor $K_2$ in der Gleichung%
\begin{equation}
B'(s) = -K_2(s)N(s).
\end{equation}
\end{definition}

\begin{theorem}[Frenetsche Formeln für räumliche Kurven]%
\index{frenetsche Formeln}\mbox{}\\*
Sei $\gamma\colon I\to\R^3$ eine dreimal differenzierbare und zweiter
Ordnung reguläre nach Bogenlänge parametrisierte Kurve.
Sei $K_1$ die Krümmung, $K_2$ die Torsion und $T,N,B$ das begleitende
Dreibein. Dann gilt%
\begin{equation}
\begin{bmatrix}
T'(s)\\ N'(s)\\ B'(s)
\end{bmatrix}
= \begin{bmatrix}
0 & K_1(s) & 0\\
-K_1(s) & 0 & K_2(s)\\
0 & -K_2(s) & 0
\end{bmatrix}
\begin{bmatrix}
T(s)\\ N(s)\\ B(s)
\end{bmatrix}.
\end{equation}
\end{theorem}

\noindent\strong{Beweis.}
Die Formel $T'(s)=K_1(s)N(s)$ bekommt man wegen $K_1(s)=|T'(s)|$
wieder geschenkt. Die Formel $B'(s)=-K_2(s)N(s)$ gilt gemäß
Definition \ref{Torsion}, die auf Satz \ref{Bstrich-kollinear-zu-N}
beruht. Die Gleichung $B=T\times N$ lässt sich nach $N$ umstellen,
da das System orthonormal ist. Aus \eqref{eq:Grassmann-Identitaet}
und $\langle N,T\rangle=0$ ergibt sich
\begin{equation}
B\times T = (T\times N)\times T
= \langle T,T\rangle N - \langle N,T\rangle T
= |T|^2 N = N.
\end{equation}
Analog ergibt sich
\begin{equation}
B\times N = (T\times N)\times N
= \langle T,N\rangle N - \langle N,N\rangle T
= -|N|^2 T = -T
\end{equation}
Man erhält nun
\begin{align}
N' &= (B\times T)' = B'\times T+B\times T'
= -K_2 N\times T+K_1 B\times N\\
&= K_1 B\times N+K_2 T\times N
= -K_1 T + K_2 B.\;\qedsymbol
\end{align}

\section{Die Totalkrümmung}

Liegt eine ebene Kurve $\gamma\colon[a,b]\to\R^2$ vor, kann man sich
überlegen, ob sich der Kurve eine globale Krümmung zuordnen lässt,
die die lokale Krümmung an jeder Stelle einbezieht. Diese Krümmung
wollen wir \emph{Totalkrümmung} nennen. Die Überlegung
ist einfach: Man integriert über die lokale Krümmung. Windet sich
die Kurve in eine Richtung, trägt das zur Totalkrümmung bei, windet sie
sich anschließend in die andere Richtung, wird durch das veränderte
Vorzeichen von dieser Totalkrümmung abgezogen.

Da wir nicht wissen, inwiefern dieser Wert von der Parametrisierung
abhängig ist, definieren wir ihn zunächst am besten für nach Bogenlänge
parametrisierte Kurven.

\begin{definition}[Totalkrümmung]%
\index{Totalkrümmung}\mbox{}\\*
Ist $\gamma\colon [a,b]\to\R^2$ nach Bogenlänge parametrisiert, dann ist
\begin{equation}
K(\gamma) := \int_a^b K(\gamma,s)\,\mathrm ds
= \int_a^b \gamma''(s)\,\mathrm ds.
\end{equation}
\end{definition}
Die Totalkrümmung ist eine geometrische Eigenschaft der Kurve
und nicht der Parametrisierung. Dies ergibt sich automatisch
daraus, dass Parametrisierung nach der Bogenlänge gewählt wurde.
Für eine nicht nach Bogenlänge parametrisierte Kurve
$\tilde \gamma(t) := (x(t),y(t))$ ergibt sich gemäß Formel
\eqref{eq:Kruemmung-t-Ebene} und
$\mathrm ds = |\tilde \gamma'(t)|\,\mathrm dt$
die Formel
\begin{equation}
K(\tilde\gamma) = \int_{t_0}^{t_1}
\frac{x'(t)y''(t)-x''(t)y'(t)}{x'(t)^2+y'(t)^2}\,\mathrm dt.
\end{equation}
Speziell für eine Funktion $f\colon [a,b]\to\R$, als Kurve
$(t,f(t))$ betrachtet, gewinnt man daraus das Resultat
\begin{equation}
K(f) = \int_a^b \frac{f''(x)}{1+f'(x)^2}\,\mathrm dx
= \int_a^b \tfrac{\mathrm d}{\mathrm dx}\arctan f'(x)\,\mathrm dx
= [\arctan f'(x)]_a^b.
\end{equation}
Da der Anstieg der Tangens des Steigungswinkels ist, ist die
Totalkrümmung einfach die Differenz der Steigungswinkel.
Das ist natürlich klar, wenn man weiß, dass die Krümmung
die Winkeländerung des Tangentialvektors ist.

Aus der Totalkrümmung ergibt sich die
\emph{Tangentenumlaufzahl}\index{Tangentenumlaufzahl}
$K(\gamma)/(2\pi)$, die, wie sich herausstellen wird, eine topologische
Größe ist. Man kann dazu mal eine Strecke mit einer Delle
betrachten. Beim Durchlaufen der Delle krümmt sich die Kurve
in unterschiedliche Richtungen, insgesamt hebt sich die Totalkrümmung
tatsächlich wieder zu null auf. Enthält die Strecke anstelle
einer Delle jedoch eine Schlaufe, dann ist dies nicht der
Fall, die Totalkrümmung wird aber exakt $2\pi$ sein, wie sich
herausstellt. Das ist unabhängig von der genauen Gestalt der Kurve.

Welcher praktische Nutzen ergibt sich da nun, etwa für Physiker und
Ingenieure?  Nun, die Tangentenumlaufzahl ist eine globale Größe
über ein Objekt, die eine topologische Aussage über das Objekt
trifft und über ein Integral berechnet wird. Es gibt in der Mathematik
noch andere ähnliche Zusammenhänge dieser Art. In der
Funktionentheorie gibt es da die Windungszahl, diese hat mit
der Unabhängigkeit des komplexen Wegintegrals für
holomorphe\index{holomorph} Funktionen zu tun.
In der Physik ist die Arbeit in einem konservativen
Feld unabhängig vom Weg -- unter genauerer Untersuchung führt das zum
Satz von Stokes. Darauf aufbauend gelangt man zu für uns äußerst
wichtigen Zusammenhängen zwischen geschlossenen und exakten
Differentialformen, die sich im Poincaré"=Lemma manifestieren.
Verbunden damit wiederum ist die De"=Rham"=Kohomologie und der
De"=Rham"=Komplex, was schließlich zur Kohomologie"=Theorie führt.

Zum besseren Verständnis ist es förderlich, wenn wir die
Gemeinsamkeiten und Unterschiede dieser Begriffe herausarbeiten
und jeweils den topologischen Kern herausschälen.
Diese Begriffe sind für die Differentialgeometrie auch nicht
rudimentär, sondern so wesentlich, dass sich daraus eine eigene
auf der Differentialgeometrie aufbauende Theorie ergeben hat,
die sogenannte \emph{Differentialtopologie}. Die Invarianz der
Tangentenumlaufzahl ist eine der einfachsten Einsichten aus
dieser Theorie.



