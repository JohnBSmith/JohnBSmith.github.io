\documentclass[a4paper,11pt,fleqn,BCOR=20mm,%
twoside,twocolumn,dvipdfmx]{scrartcl}
\usepackage[utf8]{inputenc}
\usepackage[T1]{fontenc}
\usepackage[ngerman]{babel}
\usepackage{microtype}

\usepackage{libertine}
\usepackage[libertine]{newtxmath}
\addtokomafont{disposition}{\rmfamily}

\usepackage{amsmath}
\usepackage{amssymb}
\usepackage[all]{xy}
\usepackage{color}
\definecolor{c1}{RGB}{00,40,80}
\usepackage[colorlinks=true,linkcolor=c1]{hyperref}
\usepackage{geometry}
\geometry{a4paper,left=25mm,right=14mm,top=24mm,bottom=32mm}
\setlength{\columnsep}{6mm}

\begin{document}

\section*{Kommutative Diagramme}

Eine Funktion wird für gewöhnlich in der Notation
$f{:}\;A\rightarrow B$ notiert. Das sagt uns erst einmal nur,
dass $f$ irgendeine Funktion mit Definitionsbereich
$D(f)=A$ und Zielmenge $Z(f)=B$ ist. Äquivalent hierzu ist
die Notation $f{\in}B^A$ wobei mit
$B^A=\mathrm{Abb}(A,B)$ die Menge der Abbildungen mit
Definitionsbereich $A$ und Zielmenge $B$ ist.

Diese Notation schreibt man nun in der Form%
\[\xymatrix{A \ar[r]^f & B}.\]
Schreibt man nun also
\[\xymatrix{A \ar[r]^f & B \ar[r]^g & C}\]
so lassen sich folgende Informationen daraus lesen:%
\begin{gather*}
D(f)=A,\quad Z(f)=B,\quad D(g)=B,\quad Z(g)=C.
\end{gather*}
Daher ist $Z(f)=D(g)$. Das sagt uns aber nicht, dass etwa die
Bildmenge $\mathrm{Bild}(f)$ mit $D(g)$ übereinstimmt.
Über die Bildmenge von $f$ können wir nur die Aussage
$\mathrm{Bild}(f)\subseteq D(g)$ treffen.

Nun betrachten wir das folgende Diagramm:%
\[\xymatrix{A \ar[r]^f \ar[dr]_h & B\ar[d]^g\\
& C}
\]
Aus diesem Diagramm lassen sich erst einmal nur folgende
Informationen herauslesen:%
\begin{gather*}
D(f)=A,\quad Z(f)=B,\\
D(g)=B,\quad Z(g)=C,\\
D(h)=A,\quad Z(h)=C.
\end{gather*}
Die Komposition zweier Funktionen ist folgendermaßen definiert:%
\[(g\circ f)(x):=g(f(x)).\]
Man sagt nun, \textit{das Diagramm kommutiert}, wenn die Komposition
$g\circ f$ mit $h$ übereinstimmt. D.h. es ist $h=g\circ f$.
Funktionen vergleicht man \textit{extensional}. Damit meint man%
\[h=g\circ f \quad\Longleftrightarrow\quad \forall x{\in}A{:}\;\;
h(x)=g(f(x)).\]
Wenn man schreibt $f{:}\; A\rightarrow A$, so meint man damit
schlicht, dass $A$ der Definitionsbereich von $f$ ist und
mit der Zielmenge von $f$ übereinstimmt, kurz $D(f)=Z(f)=A$.
Als Diagramm lässt sich dies ausdrücken durch:%
\[\xymatrix{A\ar[r]^f & A}\]
oder auch durch:
\[\xymatrix{\ar@(ul,dl)_f}A\]
Schreibt man
\[\xymatrix{\ar@(ul,dl)_{\mathrm{id}}}A\]
so ist mit dem Pfeil die identische Abbildung $\mathrm{id}_A$
gemeint. Das Diagramm%
\[\xymatrix{A \ar@/^/[r]^f & \ar@/^/[l]^g B}\]
sagt uns nur
\begin{gather*}
D(f)=A,\quad Z(f)=B,\quad D(g)=B,\quad Z(g)=A.
\end{gather*}
Es kann nicht kommutieren, da es keine zwei Wege gibt.
Wir können zwar die Verkettung $h=g\circ f$ bilden. Aber das
bedeutet nur, dass $D(h)=A$ und $Z(h)=A$ ist.

Betrachten wir nun folgendes Diagramm:%
\[\xymatrix{\ar@(ul,dl)_{\mathrm{id}}}
\xymatrix{A \ar@/^/[r]^f & \ar@/^/[l]^g B}\]
Wenn dieses Diagramm kommutiert, dann ist $\mathrm{id}_A=g\circ f$.

Was sagt uns das jetzt? Nun, weil $\mathrm{id}_A$ eine Bijektion
ist, muss schon mal $g$ surjektiv und $f$ injektiv sein.
Es kann aber Elemente in $B$ geben, die nicht in
$\mathrm{Bild}(f)$ enthalten sind. Da $B$ dann
aber mehr Elemente hat als $A$, kann $g$ nicht injektiv sein.

Betrachten wir nun folgendes Diagramm:%
\[\xymatrix{\ar@(ul,dl)_{\mathrm{id}}}
\xymatrix{A \ar@/^/[r]^f & \ar@/^/[l]^g B}
\xymatrix{\ar@(ur,dr)^{\mathrm{id}}}\]
Wenn dieses Diagramm kommutiert, dann ist $\mathrm{id}_A=g\circ f$
und $\mathrm{id}_B=f\circ g$. Das heißt $f$ und $g$ sind
bijektiv und invers zueinander. Man sagt, $g$ ist die
Umkehrfunktion zu $f$ und schreibt $f^{-1}:=g$.

Beispiel:
\[\xymatrix{\ar@(ul,dl)_{\mathrm{id}}}
\xymatrix{\mathbb R \ar@/^/[r]^{\mathrm{exp}} & \ar@/^/[l]^{\mathrm{ln}} \mathbb R_+}
\xymatrix{\ar@(ur,dr)^{\mathrm{id}}}\]
Mit $\mathbb R_+$ ist die Menge $\{y\in\mathbb R|\; y>0\}$
gemeint.

Betrachten wir folgendes Diagramm:%
\[\xymatrix{A \ar[r]^f \ar[d]^p & \ar[d]^q B\\
C \ar[r]^g & D}
\]
Falls dieses Diagramm kommutiert, gilt%
\[q\circ f = g\circ p.\]
Nehmen wir mal ein großen Diagramm:%
\[\xymatrix{A\ar[r]^f\ar[d]^{p_1} & B \ar[r]^g
\ar[d]^{p_2} & C \ar[r]^h\ar[d]^{p_3} & \ar[d]^{p_4} D\\
A' \ar[r]^{f'} & B' \ar[r]^{g'}& C' \ar[r]^{h'} & D'}
\]
Falls dieses Diagramm kommutiert, gilt%
\begin{gather*}
p_2\circ f = f'\circ p_1,\\
p_3\circ g = g'\circ p_2,\\
p_4\circ h = h'\circ p_3,\\
p_3\circ g\circ f = g'\circ f'\circ p_1
= g'\circ p_2\circ f,\\
p_4\circ h\circ g = h'\circ g'\circ p_2
= h'\circ p_3\circ g\\
\end{gather*}
und
\begin{gather*}
p_4\circ h\circ g\circ f\\
= h'\circ g'\circ f'\circ p_1\\
= h'\circ g'\circ p_2\circ f\\
= h'\circ p_3\circ g\circ f.
\end{gather*}
In diesem Fall ist das Diagramm wesentlich
übersichtlicher.

Anmerkung: Die letzte Gleichungskette hat so viele Terme,
wie es bei einer Manhattan"=Metrik kürzeste
Wege von $A$ nach $D'$ gibt. Die Pfeile
geben dabei genau an, in welche Richtungen
man laufen darf, um auf dem kürzesten Weg
zu bleiben.

\newpage
Man kann aber auch einfache Rechenregeln umständlich
mit einem solchen Diagramm formulieren. Nehmen wir mal
das Kommutativgesetz $ab=ba$. Dazu definiert man erst einmal
die Funktion $f{:}\;\mathbb R^2\rightarrow\mathbb R$ mit
$f((a,b)):=ab$. Sei außerdem
$\sigma{:}\;\mathbb R^2\rightarrow\mathbb R^2$
mit $\sigma((a,b)):=(b,a)$. D.h. $\sigma$ vertauscht die
Komponenten eines geordneten Paares. Das Kommutativgesetz lässt sich
nun durch das kommutierende Diagramm%
\[\xymatrix{\mathbb R^2 \ar[r]^\sigma \ar[dr]_f &\ar[d]^f \mathbb R^2\\
& \mathbb R}\]
ausdrücken.

Das bedeutet es ist $f=f\circ\sigma$. Daher ergibt sich%
\[ab = f((a,b))=f(\sigma((a,b))) = f((b,a)) = ba.\]

\end{document}



