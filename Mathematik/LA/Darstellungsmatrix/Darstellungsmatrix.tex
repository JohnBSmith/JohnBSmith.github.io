\documentclass[9pt]{beamer}
\usetheme{Antibes}
\useinnertheme{rectangles}
\useoutertheme{infolines}
\usepackage[utf8]{inputenc}
\usepackage[T1]{fontenc}
\usepackage[ngerman]{babel}

% Patch the look of +, = in arev
\usefonttheme{serif} 

\usepackage{arev}
\usepackage{amsmath}
\usepackage{amssymb}

% Patch punctuation to be upright
\DeclareMathSymbol{.}{\mathpunct}{operators}{`.}
\DeclareMathSymbol{,}{\mathpunct}{operators}{`,}

\setbeamertemplate{footline}{%
\begin{beamercolorbox}[ht=3.0ex,dp=1ex]{title in head/foot}
\hfill\footnotesize\insertpagenumber\enspace\enspace\end{beamercolorbox}}

\definecolor{bluegreen1}{rgb}{0.0,0.20,0.28}
\definecolor{bluegreen2}{rgb}{0.0,0.20,0.28}
\setbeamercolor*{palette primary}{fg=white,bg=bluegreen1}
\setbeamercolor*{palette secondary}{fg=white,bg=bluegreen2}
\setbeamercolor*{palette tertiary}{fg=white,bg=bluegreen2}
\setbeamercolor{itemize item}{fg=black}
\setbeamercolor{block title}{bg=bluegreen2}
\newcommand{\modest}[1]{{\small\color{gray}#1}}

\newcommand{\ee}{\mathrm e}
\newcommand{\ui}{\mathrm i}
\newcommand{\real}{\operatorname{Re}}
\newcommand{\imag}{\operatorname{Im}}
\newcommand{\uv}[1]{\underline{#1}}
\newcommand{\bv}[1]{\mathbf{#1}}

\newcommand{\N}{\mathbb N}
\newcommand{\Z}{\mathbb Z}
\newcommand{\Q}{\mathbb Q}
\newcommand{\R}{\mathbb R}
\newcommand{\C}{\mathbb C}

\newcommand{\id}{\operatorname{id}}
\newcommand{\sgn}{\operatorname{sgn}}
\newcommand{\Abb}{\operatorname{Abb}}
\newcommand{\unit}[1]{\mathrm{#1}}
\newcommand{\chem}[1]{\mathrm{#1}}
\newcommand{\strong}[1]{\textsf{\textbf{#1}}}
\newcommand{\defiff}{\quad:\Longleftrightarrow\quad}
\renewcommand{\qedsymbol}{\ensuremath{\Box}}

\newcommand{\icol}[1]{
  \big(\!\begin{smallmatrix}#1\end{smallmatrix}\!\big)%
}
\newcommand{\parspace}{\vspace{0.8em}}

\title{Was ist eine Darstellungsmatrix?}
\date{}

\begin{document}

\begin{frame}
\titlepage

\end{frame}

\begin{frame}
Gegeben sei eine lineare Abbildung
\[f\colon\R^2\to\R^2,\quad
f(\bv v) = M\bv v,\]
beispielsweise durch die Matrix
\[M := \begin{pmatrix}0 & -1\\ 1 & 0\end{pmatrix}.\]\pause
Außerdem seien zwei Basen gegeben. Diese seien wieder
\[A = (\bv a_1,\bv a_2),\quad \bv a_1 := \icol{4\\ 1},\quad \bv a_2 := \icol{1\\ 3}\phantom{,}\]
und
\[B = (\bv b_1,\bv b_2),\quad \bv b_1 := \icol{7\\ 1},\quad \bv b_2 := \icol{1\\ 2}.\]
\end{frame}

\begin{frame}
Wir wollen eine Darstellung der linearen Abbildung bezüglich Basis
$A$ für das Argument und Basis $B$ für das Bild bestimmen.
Betrachten wir dazu die Gleichung $\bv w = f(\bv v)$.\pause

\parspace
Eingesetzt wird $\bv v = A\bv v_A$ und $\bv w = B\bv w_B$. Die
Gleichung nimmt die Form
\[B\bv w_B = f(A\bv v_A) = MA\bv v_A\]
an. Umformung führt zu
\[\bv w_B = B^{-1}MA\bv v_A.\]\pause
Die Matrix
\[M_B^A(f) := B^{-1}MA\]
nennt man die \emph{Darstellungsmatrix} der linearen Abbildung $f$
bezüglich $A$ für das Argument und $B$ für das Bild.
\end{frame}

\begin{frame}
Man bekommt
\[M_B^A(f) = \tfrac{1}{13}\begin{pmatrix}
-6 & -7\\
29 & 10
\end{pmatrix}.\]
\end{frame}

\begin{frame}
\strong{Bemerkung.} Betrachten wir die Standardbasis $E=(\bv e_1,\bv e_2)$, so dass
\[E = \begin{pmatrix}1 & 0\\ 0 & 1\end{pmatrix}\]
die Einheitsmatrix ist. Für diese gilt $E^{-1}=E$. Dementsprechend ist
\[M_E^E(f) = E^{-1}ME = M.\]
Das heißt, eine als Matrix betrachtete lineare Abbildung zwischen
Koordinatenräumen ist ihre eigene Darstellungsmatrix bezüglich der
Standardbasis.
\end{frame}

\begin{frame}
Für abstrakte Vektorräume gilt eine analoge Überlegung.

\parspace
Sei $f\colon V\to W$ eine lineare Abbildung zwischen abstrakten
Vektorräumen.\pause

\parspace
Um die Vektorräume zugänglich zu machen, benötigen wir
\begin{itemize}
\item eine Basis $A$ von $V$,
\item eine Basis $B$ von $W$.
\end{itemize}

\parspace
Damit erhalten wir
\begin{itemize}
\item ein Koordinatensystem $\Phi_A$, so dass $\bv v = \Phi_A(\bv v_A)$,
\item ein Koordinatensystem $\Phi_B$, so dass $\bv w = \Phi_B(\bv w_B)$.
\end{itemize}
\end{frame}

\begin{frame}
Die Gleichung $\bv w = f(\bv v)$ nimmt damit die Gestalt
\[\Phi_B(\bv w_B) = f(\Phi_A\bv v_A)\]
an. Umformung führt zu
\[\bv w_B = \Phi_B^{-1}(f(\Phi_A(\bv v))) = (\Phi_B^{-1}\circ f\circ\Phi_A)(\bv v_A).\]\pause
Weil
\[M_B^A(f) := \Phi_B^{-1}\circ f\circ\Phi_A\]
ein lineare Abbildung zwischen Koordinatenräumen ist, darf man sie
als Matrix betrachten. Wie zuvor sprechen wir von der Darstellungsmatrix.
\end{frame}

\begin{frame}
\strong{Bemerkung.} Aus der Definition folgt
\[M_B^A(\id) = \Phi_B^{-1}\circ\id\circ\Phi_A = \Phi_B^{-1}\circ\Phi_A = T_B^A.\]
Für $\id(\bv v)=E\bv v$ ist das die Matrizenrechnung
\[M_B^A(\id) = B^{-1}EA = B^{-1}A = T_B^A.\]
Das heißt, die Darstellungsmatrix der identischen Abbildung ist die
Transformationsmatrix für den Basiswechsel von $A$ zu $B$.
\end{frame}

\begin{frame}
Ende.
\vfill\hfill\modest{Juni 2021}\\
\hfill\modest{Creative Commons CC0 1.0}
\end{frame}

\end{document}


