\documentclass[a4paper,10pt,fleqn,twoside,twocolumn,dvipdfmx]{scrartcl}
\usepackage[utf8]{inputenc}
\usepackage[T1]{fontenc}
\usepackage[ngerman]{babel}
\usepackage{amsmath}
\usepackage{amssymb}
\usepackage{lipsum}
\usepackage{booktabs}
\usepackage{makecell}

\usepackage{libertine}
\usepackage[libertine]{newtxmath}
\usepackage[scaled=0.78]{DejaVu Sans}

\usepackage{geometry}
\geometry{a4paper,left=20mm,right=10mm,top=20mm,bottom=10mm}
\setlength{\columnsep}{4mm}

\newcommand{\ee}{\mathrm e}
\newcommand{\ui}{\mathrm i}
\newcommand{\real}{\operatorname{Re}}
\newcommand{\imag}{\operatorname{Im}}
\newcommand{\uv}[1]{\underline{#1}}
\newcommand{\bv}[1]{\mathbf{#1}}

\newcommand{\N}{\mathbb N}
\newcommand{\Z}{\mathbb Z}
\newcommand{\Q}{\mathbb Q}
\newcommand{\R}{\mathbb R}
\newcommand{\C}{\mathbb C}

\newcommand{\id}{\operatorname{id}}
\newcommand{\sgn}{\operatorname{sgn}}
\newcommand{\Abb}{\operatorname{Abb}}
\newcommand{\erf}{\operatorname{erf}}
\newcommand{\unit}[1]{\mathrm{#1}}
\newcommand{\chem}[1]{\mathrm{#1}}
\newcommand{\strong}[1]{\textsf{\textbf{#1}}}
\newcommand{\ds}{\displaystyle}
\newcommand{\bvec}[1]{\mathbf{#1}}
\newcommand{\bvecgreek}[1]{\boldsymbol{#1}}
\newcommand{\tsbrace}[2]{%
  \big\{\!\begin{smallmatrix}#1\\ #2\end{smallmatrix}\!\big\}}
\newcommand{\tsbracket}[2]{%
  \big[\!\begin{smallmatrix}#1\\ #2\end{smallmatrix}\!\big]}

\newcommand{\arccot}{\operatorname{arccot}}
\newcommand{\arsinh}{\operatorname{arsinh}}
\newcommand{\arcosh}{\operatorname{arcosh}}
\newcommand{\artanh}{\operatorname{artanh}}
\newcommand{\arcoth}{\operatorname{arcoth}}

\begin{document}
\pagestyle{empty}

\noindent
\strong{Winkelfunktionen}\\[2pt]
\begin{tabular}{@{}l|l}
\makecell[lt]{
$\sin(x+y) = \sin x\cos y + \cos x\sin y$\\
$\sin(x-y) = \sin x\cos y - \cos x\sin y$\\
$\cos(x+y) = \cos x\cos y - \sin x\sin y$\\
$\cos(x-y) = \cos x\cos y + \sin x\sin y$
} & \makecell[lt]{
$\sin(z\pm\pi) = -\sin z$\\
$\cos(z\pm\pi) = -\cos z$\\
$\sin(\pi/2-x) = \cos x$\\
$\cos(\pi/2-x) = \sin x$
}
\end{tabular}\\
\begin{tabular}{@{}l@{\;}|@{\;}l}
$\sin(nx) \!=\! 2\cos x\sin((n{-}1)x)-\sin((n{-}2)x)$ & $\sin(-z) = -\!\sin z$\\
$\cos(nx) \!=\! 2\cos x\cos((n{-}1)x)-\cos((n{-}2)x)$ & $\cos(-z) = \cos z$
\end{tabular}\\
\begin{tabular}{@{}l@{\;\,}|@{\;\,}l@{\;}|@{\;\,}l}
$\cos^2 z+\sin^2 z=1$
& $\cosh^2 z-\sinh^2 z=1$
& $\cosh(\ui z)=\cos z$\\
$\ee^{\ui z}=\cos z+\ui\sin z$
& $\ee^z=\cosh z+\sinh z$
& $\sinh(\ui z)=\ui\sin z$\\
$2\cos z = \ee^{\ui z}+\ee^{-\ui z}$
& $2\cosh z = \ee^z+\ee^{-z}$
& $\cos(\ui z) = \cosh z$\\
$2\ui\sin z = \ee^{\ui z}-\ee^{-\ui z}$
& $2\sinh z = \ee^z-\ee^{-z}$
& $\sin(\ui z) = \ui\sinh z$
\end{tabular}\\[4pt]
\strong{Reihen}\\
\begin{tabular}{@{}l|l|l}
$\ds\ee^z = {\textstyle\sum\limits_{k=0}^\infty} \frac{z^k}{k!}$
& $\ds\ee^z = \lim\limits_{n\to\infty} \Big(1+\frac{z}{n}\Big)^n$
& $\ds\ln z = \lim\limits_{h\to 0}\frac{z^h-1}{h}$
\end{tabular}\\[2pt]
\begin{tabular}{@{}l|l}
$\sin z = \sum_{k=0}^{\infty} (-1)^k \frac{z^{2k+1}}{(2k+1)!}$
& $\sinh z = \sum_{k=0}^{\infty}\frac{z^{2k+1}}{(2k+1)!}$\\
$\cos z = \sum_{k=0}^{\infty} (-1)^k \frac{z^{2k}}{(2k)!}$
& $\cosh z = \sum_{k=0}^{\infty}\frac{z^{2k}}{(2k)!}$
\end{tabular}\\[4pt]
\begin{tabular}{@{}l|l}
$\ds\frac{z}{\ee^z-1} = {\textstyle\sum\limits_{k=0}^\infty} \overline B_k \frac{z^k}{k!}$
& $\ds \ln(1-x) = (-1){\textstyle\sum\limits_{k=1}^{\infty}} \frac{x^k}{k}
\quad (-1{\le}x{<}1)$\\
$\ds\frac{z}{1-\ee^{-z}} = {\textstyle\sum\limits_{k=0}^\infty} B_k\frac{z^k}{k!}$
& $\ds (z+1)^a = {\textstyle\sum\limits_{k=0}^{\infty}} \binom{a}{k}\,z^k\quad (a\in\C,|z|{<}1)$\\
$\ds\frac{1}{1-z} = {\textstyle\sum\limits_{k=0}^\infty} z^k$
& $\ds\frac{1}{(1-z)^n} = {\textstyle\sum\limits_{k=0}^{\infty}} \binom{n{+}k{-}1}{k}\,z^k
\quad (|z|{<}1)$
\end{tabular}\\
$\ds f[a](z) := \ee^{(z-a)D}f(a)
= {\textstyle\sum\limits_{k=0}^\infty} \frac{f^{(k)}(a)}{k!}(z-a)^k$\\[4pt]
\strong{Differentialrechnung}\;\,|\;\,$x_{n+1} = x_n-f(x_n)/f'(x_n)$\\[2pt]
\begin{tabular}{@{}l|l}
\begin{tabular}{@{}l@{}}
$f'(x) := \lim\limits_{h\to 0}\frac{f(x+h)-f(x)}{h}$
\end{tabular}
&
\begin{tabular}{@{}l@{}}
$T(x) = f(x_0)+f'(x_0)(x-x_0)$\\
$N(x) = f(x_0)-\tfrac{1}{f'(x_0)}(x-x_0)$
\end{tabular}
\end{tabular}\\
\begin{tabular}{@{}l|l}
$(\ee^x)' = \ee^x$ & $(fg)'=f'g+g'f$\\
$\ln' x = 1/x$ & $(f/g)'=(f'g-g'f)/g^2$\\
$(a^x)' = a^x\ln a$ & $(g\circ f)' = (g'\circ f)f'$\\
$(x^n)' = nx^{n-1}$ & $(f^{-1})' = 1/(f'\circ f^{-1})$
\end{tabular}\\
\begin{tabular}{@{}l|l}
$\sin'x = \cos x$ & $\tan'x = 1+\tan^2 x = 1/\cos^2 x$\\
$\cos'x = -\sin x$ & $\cot'x = -1-\cot^2 x = -1/\sin^2 x$\\
$\sinh'x = \cosh x$ & $\tanh'x = 1-\tanh^2 x = 1/\cosh^2 x$\\
$\cosh'x = \sinh x$ & $\coth'x = 1-\coth^2 x = -1/\sinh^2 x$
\end{tabular}\\
\begin{tabular}{@{}l|l}
$\arcsin'x = 1/\sqrt{1-x^2}$ & $\arctan'x = 1/(1+x^2)$\\
$\arccos'x = -1/\sqrt{1-x^2}$ & $\arccot'x = -1/(1+x^2)$\\ 
$\arsinh'x = 1/\sqrt{x^2+1}$ & $\artanh'x = 1/(1-x^2)$\\
$\arcosh'x = 1/\sqrt{x^2-1}$ & $\arcoth'x = 1/(1-x^2)$
\end{tabular}\\[4pt]
\strong{Integralrechnung}\\
$\int_a^b f(x)\,\mathrm dx
:= \lim\limits_{n\to\infty}\sum_{k=1}^n
f\!\big(a+k\frac{b-a}{n}\big)\frac{b-a}{n}\quad (f\in C[a,b])$\\
$\int_a^b f'(x)\,\mathrm dx = [f(x)]_a^b := f(a)-f(b)\qquad (f\in C^1[a,b])$\\
$\int_a^b f(g(x))\,g'(x)\,\mathrm dx = \int_{g(a)}^{g(b)} f(u)\,\mathrm du\quad
\Big(\begin{smallmatrix}f\in C(I,\R),\\ g\in C^1([a,b],I)\end{smallmatrix}\Big)$\\
$\int_a^b\! f'(x)g(x)\,\mathrm dx = [f(x)g(x)]_a^b-\int_a^b\! f(x)g'(x)\,\mathrm dx$%
\;\;{\footnotesize $(f,g\in C^1)$}\\
\begin{tabular}{@{}l}
$t=\tan(\frac{x}{2})$,\; $\sin x = \frac{2t}{1+t^2}$,\;
$\cos x = \frac{1-t^2}{1+t^2}$,\; $\mathrm dx = \frac{2\mathrm dt}{1+t^2}$
\end{tabular}\\
$L\{f(t)\} := \int_0^{\infty}\! f(t)\ee^{-pt}\mathrm dt$,\;
$F\{f(t)\} := \frac{1}{\sqrt{2\pi}}\int_{-\infty}^{\infty}
f(t)\ee^{-\ui\omega t}\mathrm dt$\\
$\ds\frac{\mathrm d}{\mathrm dx}\int_{a(x)}^{b(x)} f(x,t)\,\mathrm dt
= \int_{a(x)}^{b(x)} \frac{\partial}{\partial x}f(x,t)\,\mathrm dt+g(x)$\\
$g(x) = f(x,b(x))b'(x) - f(x,a(x))a'(x)$

\newpage
\noindent
\strong{Komplexe Zahlen}\\
\begin{tabular}{@{}l|l}
\makecell[lt]{
$z = r\ee^{\ui\varphi} = a+b\ui$\\
$\overline z = r\ee^{-\ui\varphi} = a-b\ui$\\
$\real z = a = r\cos\varphi$\\
$\imag z = b = r\sin\varphi$
} & \makecell[lt]{
$|z| = r = \sqrt{a^2+b^2}$\\
$\ds\arg(z) =\varphi = \sgn(b)\arccos(a/r)$\\
$z_1+z_2 = (a_1+a_2)+(b_1+b_2)\ui$\\
$z_2-z_2 = (a_1-a_2)+(b_1-b_2)\ui$\\
}
\end{tabular}\\[2pt]
$z_1 z_2 = r_1 r_2 \ee^{\ui(\varphi_1+\varphi_2)}
= (a_1 a_2 - b_1 b_2)+(a_1 b_2+a_2 b_1)\ui$\\
$\ds\frac{z_1}{z_2}
=\frac{r_1}{r_2}\ee^{\ui(\varphi_1-\varphi_2)}
=\frac{a_1 a_2 + b_1 b_2}{a_2^2+b_2^2}
+ \frac{a_2 b_1 - a_1 b_2}{a_2^2+b_2^2}\ui$\\
$\ds\frac{1}{z} =\frac{1}{r}\ee^{-\ui\varphi}
=\frac{a}{a^2+b^2}-\frac{b}{a^2+b^2}\ui$\\[4pt]
\strong{Algebra}\\
$x^2+px+q=0\colon\; x = -\frac{p}{2}\pm\frac{1}{2}\sqrt{p^2-4q}$\\[2pt]
\begin{tabular}{@{}ll}
$f(x)=f(2a-x)$ & (Achsensymmetrie)\\
$f(x)=2b-f(2a-x)$ & (Punktsymmetrie)
\end{tabular}\\[4pt]
\strong{Lineare Algebra} | $\det(\lambda A) = \lambda^n\det(A)$,\; $\det(A^{-1}) = \frac{1}{\det A}$\\
\begin{tabular}{@{}l|l}
$\langle Av,w\rangle = \langle v,A^H w\rangle$
& $(AB)^H = B^H A^H$\\
$\langle v,w\rangle = |v||w|\cos\varphi$
& $(AB)^{-1} = B^{-1} A^{-1}$\\
$|v\times w| = |v||w|\sin\varphi$
& $\det(AB) = \det(A)\det(B)$
\end{tabular}\\
$\operatorname{proj}[w](v) = \frac{\langle v,w\rangle}{\langle w,w\rangle} w$,\quad
$w_k := v_k - \sum_{i=1}^{k-1}\operatorname{proj}[w_i](v_k)$\\
$\bigg(\begin{smallmatrix}
a_{\scriptstyle 1}\\
a_{\scriptstyle 2}\\
a_{\scriptstyle 3}\end{smallmatrix}\bigg)\times
\bigg(\begin{smallmatrix}
b_{\scriptstyle 1}\\
b_{\scriptstyle 2}\\
b_{\scriptstyle 3}\end{smallmatrix}\bigg)
= \bigg(\begin{smallmatrix}
a_{\scriptstyle 2} b_{\scriptstyle 3} - a_{\scriptstyle 3} b_{\scriptstyle 2}\\
a_{\scriptstyle 3} b_{\scriptstyle 1} - a_{\scriptstyle 1} b_{\scriptstyle 3}\\
a_{\scriptstyle 1} b_{\scriptstyle 2} - a_{\scriptstyle 2} b_{\scriptstyle 1}
\end{smallmatrix}\bigg)$\\[4pt]
$\ds\big(\begin{smallmatrix}a & b\\ c & d\end{smallmatrix}\big)^{-1}
= \tfrac{1}{ad-bc}\big(\begin{smallmatrix}d & -b\\ -c & a\end{smallmatrix}\big)$,\;\;
$R(\varphi) = \big(\begin{smallmatrix}
\cos\varphi & \;-\sin\varphi\\
\sin\varphi & \cos\varphi
\end{smallmatrix}\big)$\\[4pt]
\begin{tabular}{@{}l|l}
\makecell[lt]{
\strong{Polarkoordinaten}\\
$x = r\cos\varphi$\\
$y = r\sin\varphi$\\
$\varphi\in(-\pi,\pi]$\\
$\det J = r$\\[4pt]
\strong{Zylinderkoordinaten}\\
$x = r_{xy}\cos\varphi$\\
$y = r_{xy}\sin\varphi$\\
$z = z$\\
$\det J = r_{xy}$
} & \makecell[lt]{
\strong{Kugelkoordinaten}\\
$x = r\sin\theta\,\cos\varphi$\\
$y = r\sin\theta\,\sin\varphi$\\
$z = r\cos\theta$\\
$\varphi\in(-\pi,\pi],\;\theta\in[0,\pi]$\\
$\det J = r^2\sin\theta$\\[4pt]
$\theta = \beta-\pi/2$\\
$\beta\in[-\pi/2,\pi/2]$\\
$\cos\theta = \sin\beta$\\
$\sin\theta = \cos\beta$
}
\end{tabular}\\[4pt]
\strong{Vektoranalysis}\\[2pt]
\begin{tabular}{@{}l|l}
$\nabla(|\mathbf x|^2) = 2\mathbf x$ & $\nabla (fg) = g\nabla f+f\nabla g$\\
$\nabla |\mathbf x| = \mathbf x/|\mathbf x|$ & $\nabla\langle f,g\rangle = (Df)^T g+(Dg)^T f$\\
$\nabla (\tfrac{1}{g}) = -\frac{\nabla g}{g^2}$
& $\nabla(f/g) = (g\nabla f-f\nabla g)/g^2$\\
$\nabla\times\nabla f = 0$
& $\langle\nabla,f\bvec v\rangle
= \langle\nabla f,\bvec v\rangle + f\langle\nabla,\bvec v\rangle$\\
$\langle\nabla,\nabla\times\bvec v\rangle = 0$
& $\nabla\times(f\bvec v)
= f(\nabla\times\bvec v) - \bvec v\times \nabla f$
\end{tabular}\\
$\nabla\times\nabla\times\bvec v
= \nabla\langle\nabla,\bvec v\rangle-\Delta\bvec v$\\
$\langle\nabla,v\times w\rangle
= \langle\bvec w,\nabla\times\bvec v\rangle
- \langle\bvec v,\nabla\times\bvec w\rangle$\\
$\int_\gamma f\mathrm ds := \!\int_a^b\! f(t)\,|\gamma'(t)|\,\mathrm dt$,\;
$\int_\gamma\!\langle\bvec F,\mathrm d\bvec x\rangle
:= \!\int_a^b\! \langle\bvec F(\bvec x(t)),\bvec x'(t)\rangle\,\mathrm dt$\\
$\int_{\varphi(U)} f(\bvec x)\,\mathrm d\bvec x
= \int_U f(\varphi(\bvec u))\,|{\det D\varphi(\bvec u)}|\,
\mathrm d\bvec u$\\[4pt]
\strong{Extremwerte}\\[2pt]
$f(x)=\text{extrem} \Rightarrow f'(x)=0$,\; $f(p)=\text{extrem} \Rightarrow \mathrm df_p=0$\\
$f(x,y)=\text{extrem unter}\;g(x,y)=0 \Rightarrow \mathrm df = \lambda\mathrm dg$\\
$\ds J[\bvec x] := {\textstyle\int_a^b} L(t,\bvec x(t),\bvec x'(t))\,\mathrm dt
= \text{extrem}
\Rightarrow\frac{\partial L}{\partial x_k}
= \frac{\mathrm d}{\mathrm dt}\frac{\partial L}{\partial x_k'}$\\
\strong{Interpolation}\\
Linear: $\ds p(x) = y_1+\frac{y_2-y_1}{x_2-x_1}(x-x_1)$\\
Quadratisch: $p(x) = y_0+a_1(x-x_0)+a_2(x-x_0)(x-x_1)$\\
$\ds a_1 = \frac{y_1-y_0}{x_1-x_0},\quad
a_2 = \frac{1}{x_2-x_1}
\Big(\frac{y_2-y_0}{x_2-x_0}-a_1\Big)$\\[4pt]
\strong{Regression}\\[2pt]
$\ds y = \overline y + \frac{s_{xy}}{s_x}(x{-}\overline x)$,\;
$s_x = \sum\limits_{k=1}^n (x_k{-}\overline x)^2$,\;
$s_{xy} = \sum\limits_{k=1}^n (x_k{-}\overline x)(y_k{-}\overline y)$

\clearpage

\noindent\strong{Logik}\\[2pt]
\begin{tabular}{@{}cccccccc@{}}
\toprule
$A$ & $B$ & $A\land B$ & $A\lor B$
& $A\rightarrow B$ & $A\leftrightarrow B$ & $A\oplus B$ & $A\uparrow B$\\
0 & 0 & 0 & 0 & 1 & 1 & 0 & 1\\
0 & 1 & 0 & 1 & 1 & 0 & 1 & 1\\
1 & 0 & 0 & 1 & 0 & 0 & 1 & 1\\
1 & 1 & 1 & 1 & 1 & 1 & 0 & 0
\end{tabular}

\begingroup\footnotesize
\noindent
\begin{tabular}{@{}c@{\;\;}|@{\;\;}c@{\;\;}|@{\;\;}l@{}}
\toprule
\strong{Disjunktion} & \strong{Konjunktion} & \strong{Bezeichnung}\\
\midrule
  $A\lor A \equiv A$
& $A\land A \equiv A$
& Idempotenzgesetze\\
  $A\lor 0 \equiv A$
& $A\land 1 \equiv A$
& Neutralitätsgesetze\\
  $A\lor 1 \equiv 1$
& $A\land 0 \equiv 0$
& Extremalgesetze\\
  $A\lor \overline A \equiv 1$
& $A\land \overline A \equiv 0$
& Komplementärgesetze\\
\midrule
  $A\lor B \equiv B\lor A$
& $A\land B \equiv B\land A$
& Kommutativgesetze\\
  $(A{\lor}B){\lor}C \equiv A{\lor}(B{\lor}C)$
& $(A{\land}B){\land}C \equiv A{\land}(B{\land}C)$
& Assoziativgesetze\\
  $\overline{A\lor B} \equiv \overline A\land\overline B$
& $\overline{A\land B} \equiv \overline A\lor\overline B$
& De Morgansche Regeln\\
  $A\lor (A\land B) \equiv A$
& $A\land (A\lor B) \equiv A$
& Absorptionsgesetze\\
\bottomrule
\end{tabular}
\endgroup

\vspace{1pt}
\noindent
\begin{tabular}{@{}l|l}
\makecell[lt]{
$(A\rightarrow B) \equiv \overline A\lor B$\\
$(A\rightarrow B) \equiv (\overline B\rightarrow\overline A)$
}
&
\makecell[lt]{
$(A\leftrightarrow B) \equiv (A\rightarrow B)\land (B\rightarrow A)$\\
$(A\leftrightarrow B) \equiv (\overline A\lor B)\land (\overline B\lor A)$
}
\end{tabular}

\noindent
\begin{tabular}{@{}l|l}
$A\lor\forall_{\!x} P_x \equiv \forall_{\!x}(A\lor P_x)$
&$\forall_{\!x}(P_x\land Q_x) \equiv \forall_{\!x}P_x\land\forall_{\!x}Q_x$\\
$A\land\exists_x P_x \equiv \exists_x(A\land P_x)$
&$\exists_{x}(P_x\lor Q_x) \equiv \exists_x P_x\lor\exists_x Q_x$
\end{tabular}\\
$(I\models M) :\Leftrightarrow \forall\varphi{\in}M\colon I(\varphi)$\\
\begin{tabular}{@{}l|l}
$(\models\varphi) :\Leftrightarrow \forall I\colon I(\varphi)$
& $(M\models\varphi) :\Leftrightarrow \forall I\colon ((I\models M)\Rightarrow I(\varphi))$\\
$\erf(\varphi) :\Leftrightarrow \exists I\colon I(\varphi)$
& $\erf(M) :\Leftrightarrow \exists I\colon (I\models M)$
\end{tabular}\\
$\erf(\{\varphi_1,\ldots,\varphi_n\}) \Leftrightarrow
\erf(\varphi_1\land\ldots\land\varphi_n)$\\
$\erf(\varphi_1\lor\ldots\lor\varphi_n) \Leftrightarrow
\erf(\varphi_1)\lor\ldots\lor\erf(\varphi_n)$\\
\begin{tabular}{@{}ll}
$(M\vdash\varphi)\Rightarrow (M\models\varphi)$ & (Korrektheit)\\
$(M\models\varphi)\Rightarrow (M\vdash\varphi)$ & (Vollständigkeit)
\end{tabular}\\
$(M\cup\{\varphi\}\vdash\psi)\Leftrightarrow (M\vdash \varphi\rightarrow\psi)$\\
$(M\cup\{\varphi\}\models\psi)\Leftrightarrow (M\models \varphi\rightarrow\psi)$\\[-4pt]
\rule{\columnwidth}{\heavyrulewidth}\\
\strong{Mengenlehre}\\[2pt]
\begin{tabular}{@{}l@{\;\,}|@{\;\;}l}
\makecell[lt]{
$A\cap B := \{x\mid x\in A\land x\in B\}$\\
$A\cup B := \{x\mid x\in A\lor x\in B\}$\\
$A\setminus B := \{x\mid x\in A\land x\notin B\}$\\
$\bigcap_{i\in I} A_i := \{x\mid \forall i{\in}I\colon x{\in}A_i\}$\\
$\bigcup_{i\in I} A_i := \{x\mid \exists i{\in}I\colon x{\in}A_i\}$
}
&\makecell[lt]{
$A\subseteq B :\Leftrightarrow \forall_{\!x}(x\in A\Rightarrow x\in B)$\\
$A=B :\Leftrightarrow \forall_{\!x}(x\in A\Leftrightarrow x\in B)$\\
$A=B :\Leftrightarrow A\subseteq B\land B\subseteq A$\\
$f(M) := \{y\mid\exists x{\in}M\colon y{=}f(x)\}$\\
$f^{-1}(N) := \{x\mid f(x)\in N\}$
}
\end{tabular}\\
$A\times B := \{t\mid \exists x{\in}A\colon \exists y{\in}B\colon t=(x,y)\}$\\
$A\subseteq B\Leftrightarrow A\cup B=B\Leftrightarrow A\cap B=A$\\[2pt]
\begin{tabular}{@{}l@{\;\,}|@{\;\,}l}
$f(M\cup N) = f(M)\cup f(N)$ & $f^{-1}(M\cup N) = f^{-1}(M)\cup f^{-1}(N)$\\
$f(M\cap N) \subseteq f(M)\cap f(N)$ & $f^{-1}(M\cap N) = f^{-1}(M)\cap f^{-1}(N)$\\
$M\subseteq N\Rightarrow f(M)\subseteq f(N)$ & $M\subseteq N\Rightarrow f^{-1}(M)\subseteq f^{-1}(N)$\\
$(g\circ f)(M) = g(f(M))$ & $(g\circ f)^{-1}(M) = f^{-1}(g^{-1}(M))$
\end{tabular}

\newpage

\noindent
\strong{Kombinatorik}\\[2pt]
\begin{tabular}{@{}ll}
$\ds{\textstyle\sum\limits_{k=m}^{n-1}} q^k = \frac{q^m-q^n}{q-1}$,
& $\ds{\textstyle\sum\limits_{k=m}^{n-1}} k^p q^k
= \Big(q\frac{\mathrm d}{\mathrm dq}\Big)^p\;\frac{q^m-q^n}{q-1}$
\end{tabular}\\[2pt]
\begin{tabular}{@{}l|l}
$\sum_{k=1}^n k = (n/2)(n+1)$ & $\sum_{k=m}^{n-1}(\Delta a)_k = a_n-a_m$\\
$\sum_{k=1}^n k^2 = (n/6)(n+1)(2n+1)$ & $(\Delta a)_k := a_{k+1}-a_k$\\
$\sum_{k=1}^n k^3 = (n/2)^2(n+1)^2$ & $n! = n\cdot (n-1)!$
\end{tabular}\\
$\ds (a+b)^n = {\textstyle\sum\limits_{k=0}^n} \binom{n}{k}a^{n-k} b^k$,\quad
$\ds\binom{n}{k} := \frac{1}{k!}n^{\underline k} = \frac{n!}{k!(n-k)!}$\\[4pt]
$\quad n!=\Gamma(n+1),\;\; \Gamma(z+1) = z\Gamma(z),\;\;
\Gamma(z)\Gamma(1-z) = \frac{\pi}{\sin(\pi z)}$\\[2pt]
$\binom{n+1}{k+1} = \binom{n}{k}+\binom{n+1}{k}$,\quad
$\binom{n}{k} = \binom{n}{n-k}$,\quad $\binom{n}{0} = \binom{n}{n} = 1$\\[4pt]
$n! \approx\sqrt{2\pi n}\,n^n \exp\big(\frac{1}{12n}-n\big)
\approx\sqrt{2\pi n}\,(n/e)^n$\\[4pt]
\strong{Twelvefold way.} Fächer: $n:=|N|$, Karten: $k:=|K|$\\
\begin{tabular}{@{}c|ccc@{}}
\toprule
& $f\colon K{\to}N$
& $f\in\mathrm{Inj}(K,N)$
& $f\in\mathrm{Sur}(K,N)$\\
\midrule
$f$
& $n^k$
& $n^{\underline k}$
& $n!\tsbrace{k}{n}$\\[2pt]
$f\circ S_k$
& $\binom{n+k-1}{k}$
& $\binom{n}{k}$
& $\binom{k-1}{k-n}$\\[2pt]
$S_n\circ f$
& $\sum_{i=0}^n\tsbrace{k}{i}$
& $[k\le n]$
& $\tsbrace{k}{n}$\\[2pt]
$S_n\circ f\circ S_k$
& $p_n(n+k)$
& $[k\le n]$
& $p_n(k)$\\
\bottomrule
\end{tabular}
{\small\begin{tabular}{@{}l|l}
$S_k$: Karten nicht unterscheidbar & $\mathrm{Inj}$: max. 1 Karte pro Fach\\
$S_n$: Fächer nicht unterscheidbar & $\mathrm{Sur}$: mind. 1 Karte pro Fach
\end{tabular}}\\[4pt]
\begin{tabular}{@{}l|l}
$\tsbrace{n}{k} = k\tsbrace{n-1}{k}+\tsbrace{n-1}{k-1}$
& $\tsbrace{n}{0} = \tsbracket{n}{0} = [n{=}0]$\\[3pt]
$\tsbracket{n}{k} = (n-1)\tsbracket{n-1}{k}+\tsbracket{n-1}{k-1}$
& $\tsbrace{n}{1}=[n{>}0]$\\[3pt]
$x^n = \sum_{k=0}^n \tsbrace{n}{k} x^{\underline k}$
& $\tsbrace{n}{2} = (2^{n-1}-1)[n{>}0]$
\end{tabular}\\[4pt]
\strong{Wahrscheinlichkeitsrechnung}\\
$P(A\cup B) = P(A)+P(B)-P(A\cap B)$\\
$A\;\text{unabhängig zu}\;B :\Leftrightarrow P(A\cap B) = P(A)P(B)$\\
$P(A\cap B) = P(A)P(B|A)$


\clearpage

\noindent\strong{Mechanik}\\
$\begin{array}{@{}l|l}
\bvec v = \bvec x'(t) & \omega = \varphi'(t)\\
\bvec a = \bvec v'(t) & \alpha = \omega'(t)\\
\bvec F = \bvec p'(t) & \bvec M = \bvec L'(t)\\
\bvec p = m\bvec v & L = J\omega\\
\bvec F = m\bvec a & M = J\alpha\\
P = \langle\bvec F,\bvec v\rangle & P = \langle\bvec M,\bvecgreek\omega\rangle\\
E_\mathrm{kin} = \frac{1}{2}m|\bvec v|^2
& E_\mathrm{rot} = \frac{1}{2}J\omega^2
\end{array}$\\[2pt]
$\begin{array}{@{}l|l|l}
s =\varphi r
& \bvec M = \bvec r\times\bvec F & E_\mathrm{pot} = mgh\\
v = \omega r
& \bvec L = \bvec r\times\bvec p
& E_\mathrm{kin}+E_\mathrm{pot} = \mathrm{const.}\\
a =\alpha r & \bvec v = \bvecgreek\omega\times\bvec r
& F = Ds\quad \mathrm{(Feder)}
\end{array}$\\[4pt]
\strong{Gleichstrom}\\
$\begin{array}{@{}l|l|l}
U = RI & Q = It & GR = 1\\
I = GU & W = P\;\!t\\
P = UI & W = QU\\
\end{array}$\\[4pt]
\strong{Wechselstrom}\\[2pt]
$\begin{array}{@{}l|l|l}
\underline U = \underline Z\underline I
& \underline Z = R+jX
& Z^2 = R^2+X^2\\
\underline I=\underline Y\underline U
& \underline Y = G+jQ
& R = Z\cos\varphi\\
\underline S = \underline U\underline I
& \underline S = P+jB
& X = Z\sin\varphi
\end{array}$\\[4pt]
$\begin{array}{@{}l|l|l}
\underline Z = R & \mathrm{Widerstand}
& \omega = 2\pi f\\
\underline Z = jX_C & \mathrm{Kondensator}
& X_C = -1/(\omega C)\\
\underline Z = jX_L & \mathrm{Spule}
& X_L = \omega L
\end{array}$\\[4pt]
$\begin{array}{@{}l|l}
u_s = \sqrt{2}\,U_\mathrm{eff}
& u = u_s\sin(\omega t+\varphi_0)\\
i_s = \sqrt{2}\,I_\mathrm{eff}
& i = i_s\sin(\omega t+\varphi_0)
\end{array}$\\[4pt]
\strong{Allgemeine Gleichungen}\\
$\begin{array}{@{}l|l}
u = Ri & p=ui\\
i = Cu'(t)\\
u = Li'(t)
\end{array}$\\[4pt]
\strong{Elektrostatisches Feld}\\[2pt]
$\begin{array}{@{}l|l}
\ds F=\frac{1}{4\pi\varepsilon}\frac{Q_1Q_2}{r^2}
&\ds\bvec F_1 = \frac{1}{4\pi\varepsilon}
Q_1Q_2\frac{\bvec x_1-\bvec x_2}
{|\bvec x_1-\bvec x_2|^3}
\end{array}$\\
$\begin{array}{@{}l|l|l}
\bvec F = q\bvec E & Q=CU
&U = \varphi(B)-\varphi(A)\\
\bvec D = \varepsilon \bvec E
&\varepsilon = \varepsilon_0\varepsilon_r
& W=QU
\end{array}$\\
$\bvec E = -\nabla\varphi$\\
$\varepsilon_0 E^2 = 2w_e$\\[4pt]
\strong{Plattenkondensator}\\
$\begin{array}{@{}l|l}
U=Ed & C=\varepsilon A/d
\end{array}$\\[4pt]
\strong{Homogenes Feld in der Spule}\\
$\begin{array}{@{}l|l|l}
H\;\!l = NI & B\;\!l = \mu NI & \Theta = NI
\end{array}$\\[4pt]
\strong{Magnetostatisches Feld}\\
$\begin{array}{@{}l|l}
\bvec F = q\bvec v\times\bvec B
& \Phi = BA\\
F = qvB & \bvec B = \mu\bvec H\\
F = BIl & \mu = \mu_0\mu_r
\end{array}$\\
$H = I/(2\pi r)\quad (\text{Feld um einen geraden Leiter})$\\
$B^2 = 2\mu_0 w_m$\\[4pt]
\strong{Elektrodynamik}\\[2pt]
$\bvec E = -\nabla\varphi$\\
$\varepsilon\Delta\varphi = -\rho(x)$\\
$\varepsilon_0 E^2 = 2w_e$\\
$B^2 = 2\mu_0 w_m$

\newpage

\noindent
\strong{Maxwell-Gleichungen}\\
$\begin{array}{@{}l|l}
\langle\nabla,\bvec D\rangle = \rho_f(x)
& \langle\nabla,\varepsilon_0\bvec E\rangle = \rho(x)
\\
\langle\nabla,\bvec B\rangle = 0
& \langle\nabla,\bvec B\rangle = 0\\
\nabla\times\bvec E = -\partial_t\bvec B
& \nabla\times\bvec E = -\partial_t\bvec B\\
\nabla\times\bvec H = \bvec J_f+\partial_t\bvec D
& \nabla\times\bvec B = \mu_0 (\bvec J+\varepsilon_0\partial_t \bvec E)
\end{array}$\\[4pt]
\strong{Spezielle Relativitätstheorie}\\
$\ds \gamma = \frac{1}{\sqrt{1-\beta^2}},\quad\beta=v/c$\\
$\gamma = \cosh\varphi,\quad
\beta\gamma = \sinh\varphi,\quad
\beta = \tanh\varphi$\\
$ct'=\gamma(ct-\beta x),\;\; x'=\gamma(x-vt),
\;\; (y,z)'=(y,z)$\\
$\begin{array}{@{}l|l}
t = \gamma\tau & E_\mathrm{kin} = E-E_0\\
p = \gamma mv & E_\mathrm{kin} = \gamma mc^2-mc^2\\
E = \gamma mc^2 & E^2 = (pc)^2 + (mc^2)^2\\
\end{array}$\\
$\Lambda_v = \begin{bmatrix}
\gamma & -\beta\gamma & 0 & 0\\
-\beta\gamma & \gamma & 0 & 0\\
0 & 0 & 1 & 0\\
0 & 0 & 0 & 1
\end{bmatrix}$\\
$g=\mathrm{diag}(1,-1,-1,-1)$\\
$(\partial_\mu) = (\partial_{ct},\partial_x,\partial_y,\partial_z)$\\
$(\partial^\mu) = (\partial_{ct},-\partial_x,-\partial_y,-\partial_z)$\\[4pt]
\strong{Optik}\\
$\ds\frac{1}{f}=\frac{1}{g}+\frac{1}{b},\quad
A = \frac{B}{G} = \frac{b}{g}$\\
$n_1\sin(\varphi_1) = n_2\sin(\varphi_2)$\\
$c_0 = nc$\\[4pt]
\strong{Thermodynamik}\\
$\begin{array}{@{}l|l|l}
R = N_{\!A} k_B & m = nM & V = nV_m\\
R = R_s M & m = Nm_T & N=nN_{\!A}
\end{array}$\\
$pV = nRT$\\
$\ds \frac{p_1 V_1}{T_1} = \frac{p_2 V_2}{T_2}$\\
$Q =mc\Delta T$\\[4pt]
\strong{Konstanten}\\
$\varepsilon_0 = 8.8542\times 10^{-12}\,\mathrm{C/(V\,m)}$\\
$\mu_0 = 4\pi\times 10^{-7}\,\mathrm{H/m}$\\
$c_0 = 2.9979\times 10^{8}\,\mathrm{m/s}$\\
$e = 1.6022\times 10^{-19}\,\mathrm{C}$\\
$G = 6.674\times 10^{-11}\,\mathrm{m^3/(kg\,s^2)}$\\[4pt]
%
$N_{\!A} = 6.0221\times 10^{23}\,\mathrm{mol}^{-1}$\\
$k_B = 1.3806\times 10^{-23}\,\mathrm{J/K}$\\
$R = 8.3145\,\mathrm{J/(mol\,K)}$\\[4pt]
%
$0\,\mathrm K = -273.15\,\mathrm{{}^\circ C}$\\
$u = 1.6605\times 10^{-27}\,\mathrm{kg}$\\
$h = 6.6261\times 10^{-34}\,\mathrm{Js}$\\
$\hbar = 1.0546\times 10^{-34}\,\mathrm{Js}$\\
$\sigma = 5.6704\times 10^{-8}\,\mathrm{W/(m^2 K^4)}$\\[4pt]
%
$m_e = 9.1094\times 10^{-31}\,\mathrm{kg}$\\
$m_p = 1.6726\times 10^{-27}\,\mathrm{kg}$\\
$m_n = 1.6749\times 10^{-27}\,\mathrm{kg}$\\
$m_\alpha = 6.6447\times 10^{-27}\,\mathrm{kg}$

% \lipsum[1-20]

\end{document}
