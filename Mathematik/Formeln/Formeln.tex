\documentclass[a4paper,10pt,fleqn,twoside,twocolumn]{scrartcl}
\usepackage[utf8]{inputenc}
\usepackage[T1]{fontenc}
\usepackage[ngerman]{babel}
\usepackage{amsmath}
\usepackage{amssymb}
\usepackage{lipsum}
\usepackage{booktabs}
\usepackage{makecell}

\usepackage{libertine}
\usepackage[libertine]{newtxmath}
\usepackage[scaled=.80]{DejaVu Sans}

\usepackage{geometry}
\geometry{a4paper,left=20mm,right=10mm,top=20mm,bottom=10mm}
\setlength{\columnsep}{4mm}

\newcommand{\ee}{\mathrm e}
\newcommand{\ui}{\mathrm i}
\newcommand{\real}{\operatorname{Re}}
\newcommand{\imag}{\operatorname{Im}}
\newcommand{\uv}[1]{\underline{#1}}
\newcommand{\bv}[1]{\mathbf{#1}}

\newcommand{\N}{\mathbb N}
\newcommand{\Z}{\mathbb Z}
\newcommand{\Q}{\mathbb Q}
\newcommand{\R}{\mathbb R}
\newcommand{\C}{\mathbb C}

\newcommand{\id}{\operatorname{id}}
\newcommand{\sgn}{\operatorname{sgn}}
\newcommand{\Abb}{\operatorname{Abb}}
\newcommand{\unit}[1]{\mathrm{#1}}
\newcommand{\chem}[1]{\mathrm{#1}}
\newcommand{\strong}[1]{\textsf{\textbf{#1}}}
\newcommand{\ds}{\displaystyle}

\begin{document}
\pagestyle{empty}

\noindent
\begin{tabular}{@{}l|l}
\makecell[lt]{
$\sin(x+y) = \sin x\cos y + \cos x\sin y$\\
$\sin(x-y) = \sin x\cos y - \cos x\sin y$\\
$\cos(x+y) = \cos x\cos y - \sin x\sin y$\\
$\cos(x-y) = \cos x\cos y + \sin x\sin y$
} & \makecell[lt]{
$\sin(z\pm\pi) = -\sin z$\\
$\cos(z\pm\pi) = -\cos z$\\
$\cos z = \cos(-z)$\\
$\sin(-z) = -\sin z$
}
\end{tabular}\\[2pt]
\begin{tabular}{@{}l@{\;\,}|@{\;\,}l@{\;}|@{\;\,}l}
$\cos^2 z+\sin^2 z=1$
& $\cosh^2 z-\sinh^2 z=1$
& $\cosh(\ui z)=\cos z$\\
$\ee^{\ui z}=\cos z+\ui\sin z$
& $\ee^z=\cosh z+\sinh z$
& $\sinh(\ui z)=\ui\sin z$\\
$2\cos z = \ee^{\ui z}+\ee^{-\ui z}$
& $2\cosh z = \ee^z+\ee^{-z}$
& $\cos(\ui z) = \cosh z$\\
$2\ui\sin z = \ee^{\ui z}-\ee^{-\ui z}$
& $2\sinh z = \ee^z-\ee^{-z}$
& $\sin(\ui z) = \ui\sinh z$
\end{tabular}\\[2pt]
\begin{tabular}{@{}l|l|l}
$\ds\ee^z = {\textstyle\sum\limits_{k=0}^\infty} \frac{z^k}{k!}$
& $\ds\ee^z = \lim\limits_{n\to\infty} \Big(1+\frac{z}{n}\Big)^n$
& $\ds\ln z = \lim\limits_{h\to 0}\frac{z^h-1}{h}$
\end{tabular}\\[2pt]
\begin{tabular}{@{}l|l}
$\sin z = \sum_{k=0}^{\infty} (-1)^k \frac{z^{2k+1}}{(2k+1)!}$
& $\sinh z = \sum_{k=0}^{\infty}\frac{z^{2k+1}}{(2k+1)!}$\\
$\cos z = \sum_{k=0}^{\infty} (-1)^k \frac{z^{2k}}{(2k)!}$
& $\cosh z = \sum_{k=0}^{\infty}\frac{z^{2k}}{(2k)!}$
\end{tabular}\\[4pt]
\begin{tabular}{@{}l|l}
\makecell[lt]{
$z = r\ee^{\ui\varphi} = a+b\ui$\\
$\overline z = r\ee^{-\ui\varphi} = a-b\ui$\\
$\real z = a = r\cos\varphi$\\
$\imag z = b = r\sin\varphi$
} & \makecell[lt]{
$|z| = r = \sqrt{a^2+b^2}$\\
$\ds\arg(z) =\varphi = \sgn(b)\arccos(a/r)$\\
$z_1+z_2 = (a_1+a_2)+(b_1+b_2)\ui$\\
$z_2-z_2 = (a_1-a_2)+(b_1-b_2)\ui$\\
}
\end{tabular}\\[2pt]
$z_1 z_2 = r_1 r_2 \ee^{\ui(\varphi_1+\varphi_2)}
= (a_1 a_2 - b_1 b_2)+(a_1 b_2+a_2 b_1)\ui$\\
$\ds\frac{z_1}{z_2}
=\frac{r_1}{r_2}\ee^{\ui(\varphi_1-\varphi_2)}
=\frac{a_1 a_2 + b_1 b_2}{a_2^2+b_2^2}
+ \frac{a_2 b_1 - a_1 b_2}{a_2^2+b_2^2}\ui$\\
$\ds\frac{1}{z} =\frac{1}{r}\ee^{-\ui\varphi}
=\frac{a}{a^2+b^2}-\frac{b}{a^2+b^2}\ui$

\newpage

\noindent
$\ds\begin{bmatrix}a & b\\ c & d\end{bmatrix}^{-1}
= \frac{1}{ad-bc}\begin{bmatrix}d & -b\\ -c & a\end{bmatrix}$
\\[4pt]
\begin{tabular}{@{}l|l}
\makecell[lt]{
\strong{Polarkoordinaten}\\
$x=r\cos\varphi$\\
$y=r\sin\varphi$\\
$\varphi\in(-\pi,\pi]$\\
$\det J=r$\\[4pt]
\strong{Zylinderkoordinaten}\\
$x=r_{xy}\cos\varphi$\\
$y=r_{xy}\sin\varphi$\\
$z=z$\\
$\det J=r_{xy}$
} & \makecell[lt]{
\strong{Kugelkoordinaten}\\
$x=r\sin\theta\,\cos\varphi$\\
$y=r\sin\theta\,\sin\varphi$\\
$z=r\cos\theta$\\
$\varphi\in(-\pi,\pi],\;\theta\in[0,\pi]$\\
$\det J=r^2\sin\theta$\\[4pt]
$\theta=\beta-\pi/2$\\
$\beta\in[-\pi/2,\pi/2]$\\
$\cos\theta=\sin\beta$\\
$\sin\theta=\cos\beta$
}
\end{tabular}

\clearpage

\noindent
\begin{tabular}{@{}cccccccc@{}}
\toprule
$A$ & $B$ & $A\land B$ & $A\lor B$
& $A\rightarrow B$ & $A\leftrightarrow B$ & $A\oplus B$ & $A\uparrow B$\\
0 & 0 & 0 & 0 & 1 & 1 & 0 & 1\\
0 & 1 & 0 & 1 & 1 & 0 & 1 & 1\\
1 & 0 & 0 & 1 & 0 & 0 & 1 & 1\\
1 & 1 & 1 & 1 & 1 & 1 & 0 & 0
\end{tabular}

\begingroup\footnotesize
\noindent
\begin{tabular}{@{}c@{\;\;}|@{\;\;}c@{\;\;}|@{\;\;}l@{}}
\toprule
\strong{Disjunktion} & \strong{Konjunktion} & \strong{Bezeichnung}\\
\midrule
  $A\lor A \equiv A$
& $A\land A \equiv A$
& Idempotenzgesetze\\
  $A\lor 0 \equiv A$
& $A\land 1 \equiv A$
& Neutralitätsgesetze\\
  $A\lor 1 \equiv 1$
& $A\land 0 \equiv 0$
& Extremalgesetze\\
  $A\lor \overline A \equiv 1$
& $A\land \overline A \equiv 0$
& Komplementärgesetze\\
\midrule
  $A\lor B \equiv B\lor A$
& $A\land B \equiv B\land A$
& Kommutativgesetze\\
  $(A{\lor}B){\lor}C \equiv A{\lor}(B{\lor}C)$
& $(A{\land}B){\land}C \equiv A{\land}(B{\land}C)$
& Assoziativgesetze\\
  $\overline{A\lor B} \equiv \overline A\land\overline B$
& $\overline{A\land B} \equiv \overline A\lor\overline B$
& De Morgansche Regeln\\
  $A\lor (A\land B) \equiv A$
& $A\land (A\lor B) \equiv A$
& Absorptionsgesetze\\
\bottomrule
\end{tabular}
\endgroup

\vspace{1pt}
\noindent
\begin{tabular}{@{}l|l}
\makecell[lt]{
$(A\rightarrow B) \equiv \overline A\lor B$\\
$(A\rightarrow B) \equiv (\overline B\rightarrow\overline A)$
}
&
\makecell[lt]{
$(A\leftrightarrow B) \equiv (A\rightarrow B)\land (B\rightarrow A)$\\
$(A\leftrightarrow B) \equiv (\overline A\lor B)\land (\overline B\lor A)$
}
\end{tabular}

\noindent
\begin{tabular}{@{}l|l}
$A\lor\forall_{\!x} P_x \equiv \forall_{\!x}(A\lor P_x)$
&$\forall_{\!x}(P_x\land Q_x) \equiv \forall_{\!x}P_x\land\forall_{\!x}Q_x$\\
$A\land\exists_x P_x \equiv \exists_x(A\land P_x)$
&$\exists_{x}(P_x\lor Q_x) \equiv \exists_x P_x\lor\exists_x Q_x$
\end{tabular}

\noindent
\begin{tabular}{@{}l@{\;\,}|@{\;\;}l}
\makecell[lt]{
$A\cap B := \{x\mid x\in A\land x\in B\}$\\
$A\cup B := \{x\mid x\in A\lor x\in B\}$\\
$A\setminus B := \{x\mid x\in A\land x\notin B\}$
}
&\makecell[lt]{
$A\subseteq B :\Leftrightarrow \forall_{\!x}(x\in A\Rightarrow x\in B)$\\
$A=B :\Leftrightarrow \forall_{\!x}(x\in A\Leftrightarrow x\in B)$\\
$A=B :\Leftrightarrow A\subseteq B\land B\subseteq A$
}
\end{tabular}

\noindent
$A\subseteq B\Leftrightarrow A\cup B=B\Leftrightarrow A\cap B=A$

\noindent
\begin{tabular}{@{}l}
$\bigcap_{i\in I} A_i := \{x\mid \forall i{\in}I\colon x\in A_i\}$\\
$\bigcup_{i\in I} A_i := \{x\mid \exists i{\in}I\colon x\in A_i\}$\\
\end{tabular}

% \lipsum[1-20]

\end{document}
