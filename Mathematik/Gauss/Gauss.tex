\documentclass[a4paper,10pt,fleqn,twocolumn,twoside]{article}
\usepackage[utf8]{inputenc}
%
\usepackage[T1]{fontenc}
\usepackage{lmodern}
%
\usepackage{ngerman}
\usepackage{amsmath}
\usepackage{amssymb}
\usepackage{color}
\definecolor{c1}{RGB}{60,00,40}
\usepackage[colorlinks=true,linkcolor=c1]{hyperref}
\usepackage{geometry}
\geometry{a4paper,left=25mm,right=15mm,top=20mm,bottom=28mm}
\setlength{\columnsep}{6mm}
\begin{document}
\thispagestyle{empty}

\begin{huge}
\noindent
\textbf{Die gaußsche\\
Summenformel}
\par
\end{huge}

% \tableofcontents
% \newpage

\section{Klassische Methode}

Man schreibt die Summe einmal in Reihenfolge und einmal in
umgekehrter Reihenfolge auf. Beides addiert man summandenweise.
\begin{gather*}
s = 1+2+\ldots+(n-1)+n\\
s = n+(n-1)+\ldots+2+1\\
2s = (n+1)+(n+1)+\ldots+(n+1)+(n+1)
\end{gather*}
Man hat also $2s=n(n+1)$. Damit erhält man
\[s = \frac{n}{2}(n+1).\]


\section{Flächeninhalt}

Man kann die Summe als den Flächeninhalt eines Treppenranddreiecks
interpretieren. Zuerst ein Quadrat, daneben zwei übereinander,
daneben dann drei übereinander, usw.
Daher muss die Summe ungefähr $n^2/2$ sein. Da man Treppenstufen
hat und keine gerade Linie, kommen noch $n$ kleine Dreiecke mit
dem Flächeninhalt $1/2$ hinzu. Man erhält
\[A = \frac{n^2}{2}+\frac{n}{2} = \frac{n}{2}(n+1).\]


\section{Vervollständigung}

Man stellt sich wieder ein Treppenranddreieck vor.
Zuerst ein Quadrat, daneben zwei übereinander, daneben dann drei übereinander, usw.
Nun macht man die Beobachtung, dass man das Treppenranddreieck mit
einem zweiten Treppenranddreieck, das eine Säule weniger hat,
zu einem großen Quadrat vervollständigen kann. Es ist also
\[s_n+s_{n-1} = n^2.\]
Nimmt man nun noch die Rekursionsgleichung $s_{n}=s_{n-1}+n$,
so kann man $s_{n-1}$ eliminieren und erhält $2s_n-n = n^2$
und nach Umformung schließlich
\[s_n = \frac{n}{2}(n+1).\]


\section{Ungerade Zahlen}

Man beobachtet, dass die Summe der ersten ungeraden Zahlen immer
eine Quadratzahl ist. Macht man eine Wertetabelle, so findet man leicht
die Formel
\[\sum_{k=1}^n (2k-1) = n^2.\]
Nun ist aber
\begin{gather*}
\sum_{k=1}^n (2k-1) = 2\sum_{k=1}^n k - \sum_{k=1}^n 1
= -n+2\sum_{k=1}^n k
\end{gather*}
Damit ergibt sich
\[2\sum_{k=1}^n k = n^2+n.\]
Umformen bringt dann
\[\sum_{k=1}^n k = \frac{n}{2}(n+1).\]
Hierbei ist noch zu bemerken, dass es sich im Term $\sum(2k-1)$ bei
$\sum 2k$ gerade um den Flächeninhalt von zwei Treppenranddreiecken
handelt. Beide Treppenranddreiecke kann man zu einem Quadrat
zusammenlegen, wobei die Diagonale dann doppelt vorkommt. Bei $\sum 1$
handelt es sich gerade um den Flächeninhalt der Diagonale,
der abgezogen werden muss, um zum richtigen Ergebnis zu kommen.


\section{Pascalsches Dreieck}

Man betrachtet das pascalsche Dreieck und sieht, dass
die Summen auf der dritten Diagonalen liegen. Es ergibt sich
\begin{gather*}
s = \binom{n+1}{2} = \frac{(n+1)!}{2!(n+1-2)!}\\
= \frac{(n+1)n(n-1)!}{2(n-1)!} = \frac{n}{2}(n+1).
\end{gather*}


\section{Interpolation}

Sei $s_p(n+1)=s_p(n)+a_n$ mit $a_k=k^p$ und $s_p(0)=0$. Es ist nun so, dass sich $s_p(n)$ als Polynom vom Grad $p+1$ schreiben lässt, welches ja zwingend durch den Nullpunkt gehen muss. Mit diesem Wissen können wir das Polynom mit weiteren $p+1$ Punkten interpolieren.

Wir haben $s(1)=1$ und $s(2)=3$. Mit dem Ansatz
\[s(n)=c_2n^2+c_1n.\]
erhält man $1=c_2+c_1$ und $3=4c_2+2c_1$. Das lineare
Gleichungssystem hat die Lösungen $c_1=1/2$ und $c_2=1/2$.
Damit ergibt sich
\[s=\frac{1}{2}n^2+\frac{1}{2}n = \frac{n}{2}(n+1).\]

\newpage
\section{Interpolationsformel}

Zum Interpolieren wird man besser die Lagrange-Interpolation oder
die Newton-Interpolation verwenden. Da man aber immer äquidistante
Abstände der Stellen wählen kann, lässt sich die hübsche Formel
\[f(n) = \sum_{k=0} \frac{(\Delta^k f)(0)}{k!}n^{\underline k}\]
mit $(\Delta f)(n):=f(n+1)-f(n)$ verwenden.
Mit $n^{\underline k}$ ist die absteigende Faktorielle gemeint.

Man Berechnet nun
folgende Wertetabelle.

\begin{tabular}{r|r|r|r|r|r}
$n$    & 0 & 1 & 2 & 3 & 4\\
$f(n)$ & 0 & 1 & 3 & 6 & 10\\
$\Delta f(n)$ & 1 & 2 & 3 & 4\\
$\Delta^2 f(n)$ & 1 & 1 & 1
\end{tabular}

Damit ergibt sich
\begin{gather*}
f(n) = \frac{0}{1!} + \frac{1}{1!}n + \frac{1}{2!}n(n-1)
= \frac{n}{2}(n+1).
\end{gather*}

\section{Differenzenrechnung}

Der Hauptsatz lautet
\[\sum_{k=a}^{b-1} \Delta f(k) = [f(x)]_a^b = f(b)-f(a).\]
Außerdem gilt die Formel $\Delta x^{\underline n} = nx^{\underline n-1}$.
Mit $x^{\underline n}$ ist die absteigende Faktorielle gemeint. Man rechnet nun
\begin{gather*}\sum_{k=0}^{n} k = \sum_{k=0}^{n} k^{\underline 1}
= \frac{1}{2}[x^{\underline{2}}]_0^{n+1}\\
= \frac{1}{2}[x(x-1)]_0^{n+1}
= \frac{n}{2}(n+1).
\end{gather*}

\section{Perturbation}

Auf der einen Seite ist
\begin{gather*}
\sum_{k=1}^{n+1} k^2 = \sum_{k=0}^n (k+1)^2
= \sum_{k=0}^n (k^2+2k+1)\\
= \sum_{k=1}^n k^2 + 2\sum_{k=1}^n k + n+1.
\end{gather*}
Auf der anderern Seite ist
\[\sum_{k=1}^{n+1} k^2 = \sum_{k=1}^n k^2 + (n+1)^2.\]
Ein Vergleich bringt
\[2\sum_{k=1}^n k +n+1 = (n+1)^2 = n^2+2n+1.\]
Man erhält
\[\sum_{k=1}^n k = \frac{1}{2}(n^2+n) = \frac{n}{2}(n+1).\]


\section{Geometrische Methode}

Die Summe lässt sich als Skalarprodukt von zwei Vektoren im Koordinatenraum interpretieren. Es ist
\[\sum_{k=1}^n k = \langle \underline a,\underline b\rangle
= \langle (1,2,\ldots,n),(1,1,\ldots,1)\rangle.\]
Man benutzt nun die erste binomische Formel
\[|\underline a+\underline b|^2
= |\underline a|^2+2\langle\underline a,\underline b\rangle
+|\underline b|^2.\]
Mit dieser Formel ist
\begin{gather*}
2\langle\underline a,\underline b\rangle
= |\underline a+\underline b|^2-|\underline a|^2-|\underline b|^2\\
= \sum_{k=1}^n (k+1)^2-\sum_{k=1}^n k^2-n\\
= \sum_{k=1}^n k^2+(n+1)^2-1-\sum_{k=1}^n k^2-n\\
= (n+1)^2-1-n = n^2+2n+1-1-n\\
= n^2+n.
\end{gather*}
Man erhält also
\[\sum_{k=1}^n k = \frac{n}{2}(n+1).\]

\section{Ableitungsmethode}

Mit der Summe der geometrischen Folge erhält man
\[s=q\frac{\mathrm d}{\mathrm dq} \frac{q^{n+1}-1}{q-1}
= q\frac{\mathrm d}{\mathrm dq}
\sum_{k=0}^n q^k = \sum_{k=0}^n kq^k.\]
Auf der linken Seite ist
\begin{gather*}
s=\frac{q(n+1)q^n(q-1)-q(q^{n+1}-1)}{(q-1)^2}\\
= \frac{(n+1)(q^{n+2}-q^{n+1})-(q^{n+2}-q)}{(q-1)^2}\\
= \frac{nq^{n+2}-(n+1)q^{n+1}+q}{(q-1)^2}.
\end{gather*}
Die zweimalige Anwendung der Regel von l'Hospital bringt
\begin{gather*}
\lim_{q\rightarrow 1}s = \lim_{q\rightarrow 1}
\frac{n(n+2)(n+1)q^n-n(n+1)^2q^{n-1}}{2}\\
= \frac{1}{2}[n(n+2)(n+1)-n(n+1)^2]\\
= \frac{1}{2}(n+1)[n(n+2)-n(n+1)]\\
= \frac{1}{2}(n+1)[n^2+2n-n^2-n]
= \frac{n}{2}(n+1).
\end{gather*}



%\newpage
\section{Stieltjes-Integral}

Man wandelt die Summe zunächst in ein Stieltjes"=Integral um. Es ist
\[\sum_{k=1}^n f(k) = \int_0^n f(x)\,\mathrm d\lfloor x\rfloor.\]
Nun schaut man sich an, wie dieses Integral von einem gewöhnlichen
Integral abweicht. Man rechnet
\begin{gather*}
\int_0^x f(x)\,\mathrm d\lfloor x\rfloor - \int_0^n f(x)\,\mathrm dx\\
= \int_0^n f(x)\,\mathrm d(\lfloor x\rfloor-x)\\
= [f(x)(\lfloor x\rfloor-x)]_0^n
- \int_0^n (\lfloor x\rfloor -x)\,\mathrm df(x)\\
= \int_0^n (x-\lfloor x\rfloor)f'(x)\,\mathrm dx.
\end{gather*}
Damit erhält man die Formel
\[\sum_{k=1}^n f(k) = \int_0^n f(x)\,\mathrm dx
+\int_0^n (x-\lfloor x\rfloor)f'(x)\,\mathrm dx.\]
Für $f(k)=k$ wird der Integrand im zweiten Integral zu
$g(x)=x-\lfloor x\rfloor$. Die Fläche unter dem Graphen von $g(x)$
besteht jedoch aus Dreiecken. Der gesamte Flächeninhalt unter $g(x)$
ist also der Flächeninhalt eines Dreiecks (1/2) multipliziert
mit der Anzahl der Dreiecke ($n$ Stück). Es ergibt sich
\begin{gather*}
\sum_{k=1}^n k = \int_0^n x\,\mathrm dx
+\int_0^n (x-\lfloor x\rfloor)\,\mathrm dx\\
= \frac{1}{2}n^2 + \frac{1}{2}n = \frac{n}{2}(n+1).
\end{gather*}

\section{Erzeugende Funktionen}

Als Hilfsmittel benötigt man die Formel
\[\frac{1}{(1-x)^n} = \sum_{k=0}^{\infty} \binom{n+k-1}{k} x^k.\]
Mit dieser Formel erhält man
\begin{gather*}
\frac{x}{(1-x)^2} = x\sum_{k=0}^{\infty} (k+1)x^k
= \sum_{k=1}^{\infty} kx^k,\\
\frac{x}{(1-x)^3} = x\sum_{k=0}^{\infty} \binom{k+2}{k} x^k
= \sum_{k=1}^{\infty} \binom{k+1}{k-1}x^k.
\end{gather*}
Die Dreieckszahlen erhält man mit der Rekursionsformel
\begin{gather*}
s_k=s_{k-1}+k,\\
s_0=0.
\end{gather*}
Man multipliziert die Formel auf beiden Seiten mit $x^k$
und summiert ab $k=1$ auf. Die Gleichung lautet dann
\[\sum_{k=1}^{\infty} s_k x^k
= \sum_{k=1}^{\infty} s_{k-1}x^k + \sum_{k=1}^{\infty} kx^k.\]
Die Summe auf der linken Seite kürzt man mit $A(x)$ ab.
Die Gleichung wird umgeschrieben zu
\[A(x) = xA(x) + \frac{x}{(1-x)^2}.\]
Umformung nach $A(x)$ bringt die erzeugende Funktion der
Dreieckszahlen in expliziter Form. Es ist
\[A(x) = \frac{x}{(1-x)^3}.\]
Damit ergibt sich
\begin{gather*}
s_n = \binom{n+1}{n-1} = \frac{(n+1)!}{2(n-1)!}.
\end{gather*}
Mit der Rekursionsformel der Fakultät erhält man
\begin{gather*}
s_n = \frac{1}{2}(n+1)n\frac{(n-1)!}{(n-1)!}
= \frac{n}{2}(n+1).
\end{gather*}

\section{Z-Transformation}

Diese Methode ist zur Methode mit den erzeugenden
Funktionen äquivalent.

\section{Operatormethode}

Die folgende Rechnung ist nur formal dargestellt. Eine
mathematisch strenge Begründung fehlt.

Mit dem Differenzenoperator
\[[\Delta f](n) := f(n+1)-f(n)\]
gilt die Formel
\[[If](n) = f(n) = \Delta\sum_{k=0}^{n-1}f(k)
= [\Delta\Delta^{-1}f](n).\]
Der Differenzenoperator kann aufgespalten werden in
Verschiebeoperator und Identitätsoperator, es gilt
\[\Delta = e^D-I.\]
Bildet man nun formal auf beiden Seiten den Kehrwert, so erhält
man
\[\Delta^{-1} = \frac{I}{e^D-I} = \frac{D}{e^D-I}D^{-1}.\]
Nun gilt die Formel
\[\frac{x}{e^x-1} = \sum_{k=0}^\infty \overline{B}_k \frac{x^k}{k!}.\]
Mit $\overline{B}_k$ sind die Bernoulli"=Zahlen mit
$\overline{B}_1=-1/2$ gemeint.
Der inverse Operator zur Ableitung ist das Integral, es gilt
\[[D^{-1}f](n) = \int_0^n f(x)\,\mathrm dx.\]
Damit erhält man die Formel
\begin{gather*}
[\Delta^{-1} f](n) = \sum_{k=0}^{n-1} f(k)\\
= \int_0^n f(x)\,\mathrm dx
+ \sum_{k=1}^\infty \overline{B}_k
\frac{f^{(k-1)}(n)-f^{(k-1)}(0)}{k!}.
\end{gather*}
Für $f(x)=x$ ergibt sich damit
\begin{gather*}
\sum_{k=0}^{n-1}k
= \frac{1}{2}n^2+\overline{B}_1 \frac{n-0}{0!}
+\overline{B}_2 \frac{1-1}{1!}\\
= \frac{1}{2}n^2-\frac{1}{2}n
= \frac{1}{2}(n-1)n.
\end{gather*}
Inkrementiert man $n$ auf beiden Seiten, so ergibt sich das
gesuchte Resultat
\[\sum_{k=0}^n k = \frac{n}{2}(n+1).\]

\subsection*{Literatur}

Daniel Grieser: »\emph{Mathematisches Problemlösen und Beweisen}«.
Springer-Verlag, Wiesbaden (2013),\\
2.~Auflage 2017.
Abschnitt 1.3: »\emph{Ein Problem über Geraden in der Ebene}«.

Harro Heuser: »\emph{Lehrbuch der Analysis, Teil 1}«.\\
Teubner, Stuttgart/Leipzig/Wiesbaden (1980),\\
15. Auflage 2003. Abschnitt 95: »\emph{Die Eulersche Summenformel}«.

\end{document}





