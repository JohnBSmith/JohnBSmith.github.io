
\chapter{Differentialgleichungen}

\section{Das Euler-Verfahren}

Das Auffinden symbolischer Lösungen von Anfangswertproblemen kann
schwierig sein. Manchmal muss man dafür neue Funktionen definieren,
diese werden zu den speziellen Funktionen gezählt. Bei manchen
Gleichungen stellt sich die Frage, ob es denn überhaupt möglich ist,
eine symbolische Lösung zu gewinnen.

Allerdings gibt es allgemeine Verfahren zur numerischen Lösung von
Anfangswertproblemen am Computer. Hiermit lässt sich ggf. auch eine
numerische Probe bereits gefundener symbolischer Lösungen vornehmen.
Das einfachste wie auch prototypische dieser Verfahren ist das
Euler"=Verfahren, welches im Folgenden erläutert werden soll.

Betrachten wir
\begin{equation}\label{eq:AWP-O1}
x' = f(t,x),\quad x'(t_0) = x_0,
\end{equation}
das ist das allgemeine Anfangswertproblem erster Ordnung. Den
Differentialquotient approximieren wir nun durch einen
Differenzenquotient:%
\begin{equation}
\frac{x(t+h)-x(h)}{h} \approx f(t,x).
\end{equation}
Umformung dieser Gleichung bringt
\begin{equation}
x(t+h)\approx x(h)+hf(t,x).
\end{equation}
Demnach ergibt sich eine näherungsweise Lösung des Anfangswertproblems
als iterative Lösung der Differenzengleichung%
\begin{align}
t_{k} &= t_0+hk,\\
x_{k+1} &= x_k+hf(t_k,x_k).
\end{align}
Diese Iteration wird als \emph{Euler"=Verfahren} bezeichnet.

Nun interessieren uns auch Dgln. höherer Ordnung, und ganz besonders
die physikalisch wichtigen Dgln. zweiter Ordnung. Jede Dgl. höherer
Ordnung ist aber als System erster Ordnung formulierbar. Vielen
wichtige Modellen liegt außerdem ein System erster Ordnung zugrunde.
Es ist nun so, dass das Euler"=Verfahren ohne weitere Umstände auch
auf ein System erster Ordnung übertragbar ist.

Das Anfangswertproblem zu einem System von zwei Dgln. hat
z.\,B. die allgemeine Form
\begin{equation}
\begin{split}
x' &= f_1(t,x,y),\quad x'(t_0)=x_0,\\
y' &= f_2(t,x,y),\quad y'(t_0)=y_0.
\end{split}
\end{equation}
Fasst man die beiden Variablen $x,y$ zu einem Vektor
$\bvec x = (x,y)$ zusammen und definiert
\begin{equation}
f(t,\bvec x) := \begin{pmatrix}
f_1(t,\bvec x)\\
f_2(t,\bvec x)
\end{pmatrix},
\end{equation}
dann bekommt das Anfangswertproblem die kompakte
vektorielle Form
\begin{equation}
\begin{split}
\bvec x' &= f(t,\bvec x),\quad \bvec x'(t_0)=\bvec x_0.
\end{split}
\end{equation}
Diese Gleichung stimmt strukturell mit \eqref{eq:AWP-O1} überein
und die Vorgehensweise ist identisch übertragbar. Entsprechend ergibt
sich
\begin{align}
t_{k} &= t_0+hk,\\
\bvec x_{k+1} &= \bvec x_k+hf(t_k,\bvec x_k).
\end{align}
So unscheinbar diese Iteration auch ausschaut, ermöglicht sie die
numerische Lösung einer unheimlichen Vielfalt an Gleichungen.


