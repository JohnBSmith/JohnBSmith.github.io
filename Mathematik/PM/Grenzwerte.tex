
\chapter{Grenzwerte}

\section{Asymptoten}

Es sei eine reelle Funktion $f$ gegeben. Eine weitere Funktion $g$
wollen wir eine \emph{Asymptote}\index{Asymptote}
von $f$ nennen, wenn die Konvergenz
\[f(x)-g(x)\to 0\]
für $x\to\infty$ bzw. $x\to -\infty$ vorliegt. Untersuchen wir,
welche Funktionen geradlinige Asymptoten haben, das sind solche
der Form
\[g(x) = ax + b.\]
Man betrachte die Faktorisierung
\[f(x) - (ax + b) = \left(\frac{f(x)}{x}-a-\frac{b}{x}\right)x.\]
Damit der Term gegen null konvergiert, muss notwendigerweise der erste
Faktor gegen null konvergieren, da ja $x$ bei $x\to\infty$ immer größer
wird. Wir haben nun $a\to a$ und $b/x\to 0$ für $x\to\infty$.
Ergo muss $f(x)/x$ gegen $a$ konvergieren, denn gemäß den
Grenzwertsätzen gilt
\begin{align*}
0 &= \lim_{x\to\infty}\left(\frac{f(x)}{x}-a-\frac{b}{x}\right)\\
&= \lim_{x\to\infty}\frac{f(x)}{x}-\lim_{x\to\infty}a-\lim_{x\to\infty}\frac{n}{x}
= \lim_{x\to\infty}\frac{f(x)}{x} - a.
\end{align*}
Existiert der Grenzwert $\lim_{x\to\infty}\frac{f(x)}{x}$ nicht,
folgt in Kontraposition die Nichtexistenz einer nichtvertikalen
geradlinigen Asymptote.

Wegen $b\to b$ für $x\to\infty$ gilt nach den Grenzwertsätzen außerdem
\begin{align*}
b &= 0 + b = \lim_{x\to\infty}(f(x) - (ax + b)) + \lim_{x\to\infty}b\\
&= \lim_{x\to\infty} (f(x) - (ax + b) + b) = \lim_{x\to\infty} (f(x) - ax).
\end{align*}
Abschließend stellen wir fest, dass die Existenz der beiden Grenzwerte
für die Existenz der Asymptote hinreichend ist, denn unter
abermaliger Anwendung der Grenzwertsätze erhält man
\begin{align*}
0 &= b-b = \lim_{x\to\infty}(f(x)-ax) - \lim_{x\to\infty} b\\
&= \lim_{x\to\infty} (f(x)-(ax+b)),
\end{align*}
was ja die definierende Bedingung der Asymptote darstellt.

\strong{Beispiel 1.} Es sei beispielsweise die Funktion
\[f(x) := \frac{x^2}{x+1}\]
gegeben. Wir haben nun
\[a = \lim_{x\to\infty}\frac{f(x)}{x} = \lim_{x\to\infty}\frac{x}{x+1}
= \lim_{x\to\infty}\frac{1}{1+\tfrac{1}{x}} = \frac{1}{1+0} = 1.\]
Weiterhin erhält man
\[\frac{x^2}{x+1} = x-\frac{x}{x+1}\]
via einem Schritt Polynomdivision. Hiermit findet sich
\[b = \lim_{x\to\infty} (f(x) - ax) = \lim_{x\to\infty} \frac{-x}{x+1}
= \lim_{x\to\infty}\frac{-1}{1+\tfrac{1}{x}} = -1.\]
Wir haben für $x\to\infty$ die schiefe Asymptote $y = x-1$
ermittelt. Führt man die analoge Überlegung für $x\to -\infty$,
findet sich dieselbe Asymptote.

\strong{Beispiel 2.} Betrachten wir $x^2-y^2=1$. Beschränken wir uns
zunächst $y\ge 0$ auf, da Symmetrie besteht. Formt man die Gleichung
nun nach $y$ um, erhält man die Funktion
\[f(x) := y = \sqrt{x^2-1}.\]
Für $x>0$ findet sich
\[\frac{f(x)}{x} = \frac{\sqrt{x^2-1}}{\sqrt{x^2}}
= \sqrt{\frac{x^2-1}{x^2}} = \sqrt{1-\frac{1}{x^2}}.\]
Man erhält somit $a = \lim_{x\to\infty} f(x)/x = 1$. Im Fortgang nutzen
wir nun den Trick
\[(\sqrt{x^2-1}-x)(\sqrt{x^2-1}+x) = x^2-1-x^2 = 1.\]
Hiermit erhält man
\[f(x) - ax = \sqrt{x^2-1}-x = \frac{1}{x+\sqrt{x^2-1}}.\]
Nun ist $\sqrt{x^2-1}>0$ als Wurzel, womit $x+\sqrt{x^2-1}\to\infty$
für $x\to\infty$. Man erhält schließlich
\[b = \lim_{x\to\infty} (f(x)-ax) = 0.\]
Man erhält also die Asymptote $y=x$. Analog erhält man $y=-x$ für
$x\to -\infty$.

\strong{Beispiel 3.}
Es sei $f(x):=x+\tfrac{\sin(\ee^x)}{\ee^x}$, definiert auf ganz $\R$.
Man erhält
\[a = \lim_{x\to\infty}\frac{f(x)}{x} = \lim_{x\to\infty} (1+\tfrac{\sin(\ee^x)}{x\ee^x})
= 1+0 = 1\]
und
\[b = \lim_{x\to\infty}(f(x)-ax) = \lim_{x\to\infty}\frac{\sin(\ee^x)}{\ee^x} = 0.\]
Somit ist $y = x$ eine schiefe Asymptote von $f$.

Den Grenzwert $\lim_{h\to 0}\frac{\sin(h)}{h}=1$ nutzen wir nun
als Hilfsmittel. Man erhält hiermit
\begin{align*}
a' &= \lim_{x\to -\infty}\frac{f(x)}{x} =
\lim_{x\to -\infty}(1+\tfrac{\sin(\ee^x)}{x\ee^x})\\
&= 1+(\lim_{x\to -\infty}\tfrac{1}{x})(\lim_{x\to -\infty}\tfrac{\sin(\ee^x)}{\ee^x})
= 1+0\cdot 1 = 1.
\end{align*}
Im Fortgang findet sich
\[b' = \lim_{x\to -\infty} (f(x)-ax) = \lim_{x\to -\infty}\tfrac{\sin(\ee^x)}{\ee^x}
= 1.\]
Somit ist $y = x+1$ ebenfalls eine schiefe Asymptote von $f$.
