
\chapter{Komplexe Zahlen}

\section{Einführung}

Die Gleichung $x^2=4$ besitzt zwei Lösungen in den reellen Zahlen,
das sind $x=-2$ und $x=2$. Die Gleichung $x^2=-4$ besitzt jedoch
keine Lösung, denn für $x>0$ ist $x^2>0$ und für $x<0$ ist auch
$x^2>0$. Es gibt jedoch eine natürliche Zahlenbereichserweiterung
der reellen Zahlen, so dass jede solche Gleichung Lösungen besitzt.

Dazu betrachten wir den transformativen Charakter von Zahlen.
Einer reellen Zahl $r$ entspricht nämlich eine Abbildung
$x\mapsto rx$. Die Zahl $r$ skaliert $x$. Man kann sich $x$ hierbei
als einen Pfeil vom Ursprung zum Punkt $x$ auf der reellen
Zahlengeraden vorstellen. Dann verdoppelt $2x$ die Länge des
Pfeils, $\frac{1}{2}x$ halbiert die Länge usw. Es genügt hierbei
zu betrachten, was $r$ mit $x=1$ tut. Nun entspricht $r=-1$ einer
Spiegelung am Ursprung. Das Quadrat $r^2$ entspricht der zweifachen
Anwendung dieser Transformation. Es gibt also keine Transformation,
deren zweifache Anwendung eine Punktspiegelung ist.

Wir nehmen nun an, es gibt eine Zahl $\ui$ mit $\ui^2=-1$, welche
wir als \emph{imaginäre Einheit} bezeichnen. Als Transformation
betrachtet muss $\ui$ einer Drehung um $90^\circ$ entsprechen.
Wir legen uns außerdem darauf fest, dass diese Drehung gegen
den Uhrzeigersinn stattfinden soll.

Der Pfeil $\ui\cdot 1$ steht nun rechtwinklig zu $1$. Wir wollen
außerdem wie bei den reellen Zahlen die Zahlen mit ihren
Transformationen verschmelzen, d.\,h. es soll
$\ui\cdot 1=\ui$ sein. Die beiden Pfeile $1$  und $\ui$ lassen sich
nun als Basisvektoren $\bvec e_1=1$ und $\bvec e_2=\ui$ interpretieren,
die komplexen Zahlen haben dann die Struktur des Vektorraums $\R^2$.
Jeder weitere Pfeil ist daher als
\begin{equation}
z = a\cdot 1+b\cdot\ui = a+b\ui
\end{equation}
darstellbar. Die Menge aller solcher Pfeile nennen wir die
\emph{komplexen Zahlen} bzw. die \emph{komplexe Zahlenebene}.

Nach den bisherigen Ausführungen werden Addition und skalare
Multiplikation wie bei Vektoren ausgeführt:
\begin{gather*}
(a_1+b_1\ui) + (a_2+b_2\ui) = (a_1+a_2)+(b_1+b_2)\ui,\\
(a_1+b_1\ui) - (a_2+b_2\ui) = (a_1-a_2)+(b_1-b_2)\ui,\\
r(a+b\ui) = ra+rb\ui.
\end{gather*}
Die Verschmelzung von Transformation und Zahl manifestiert sich nun
in der Forderung, dass das Kommutativgesetz gültig bleiben soll.
Wenn man also wissen möchte, wie die Transformation
$a+b\ui$ auf einen Pfeil $v=x+y\ui$ wirkt, d.\,h. was das Resultat
von
\[(a+b\ui)(x+y\ui)\]
ist, kann man auch
\[(x+y\ui)(a+b\ui)\]
betrachten, womit die Bedeutung von Transformation und Pfeil
vertauscht wird. Punktspiegelungen und Drehungen sind nun
allerdings lineare Abbildungen, es erscheint demnach nur richtig,
wenn auch $a+b\ui$ eine lineare Abbildung ist. Einer linearen
Abbildung entspricht wiederum genau eine Matrix aus $\R^{2\times 2}$.
Diese Matrix ist eindeutig bestimmt durch ihre Wirkung auf die
Basisvektoren. Aufgrund von Kommutativität und Linearität bekommt
man nun
\begin{gather}
(a+b\ui)1 = 1(a+b\ui) = a+b\ui,\\
(a+b\ui)\ui = \ui(a+b\ui) = a\ui-b.
\end{gather}
Der Zahl $a+b\ui$ entspricht daher die Matrix
\begin{equation}
\Phi(a+b\ui) = \begin{pmatrix}a & -b\\ b & a\end{pmatrix}.
\end{equation}
Bezüglich Polarkoordinaten $(r,\varphi)$ hat $a+b\ui$
die Komponenten $a=r\cos\varphi$ und $b=r\sin\varphi$.
Für die Matrix gilt daher%
\begin{align}
\begin{pmatrix}a & -b\\ b & a\end{pmatrix}
&= \begin{pmatrix}
r\cos\varphi & -r\sin\varphi\\
r\sin\varphi & r\cos\varphi
\end{pmatrix}\\
&= r\begin{pmatrix}
\cos\varphi & -\sin\varphi\\
\sin\varphi & \cos\varphi
\end{pmatrix}.
\end{align}
Demzufolge entspricht der komplexen Zahl nichts anderes als eine
Drehung, gefolgt von einer Skalierung, kurz \emph{Drehskalierung}.
Zusammen mit der vektoriellen Addition ist die Arithmetik mit komplexen
Zahlen damit vollständig charakterisiert.



