
\chapter{Analytische Geometrie}

Die analytische Geometrie stellt eine Weiterentwicklung der
klassischen euklidischen Geometrie dar. Diese Entwicklung erfolgte
in zwei Schritten. Im ersten Schritt wurden Koordinatensysteme
eingeführt, um eine Synthese von rechnerischen und geometrischen
Methoden zu ermöglichen. Geometrische Aufgabenstellungen ließen
sich hiermit in Gleichungen und Gleichungssysteme übersetzen.
In einem zweiten Schritt erfolgte die Einführung der Vektorrechnung,
welche eine Übersetzung geometrischer Sachverhalte in rechnerische
Ausdrücke erlaubt, die anschaulicher und prägnanter als Ausdrücke
mit Koordinaten sind. Mit der Vektorrechnung verbunden sind neue
mathematische Objekte, die Vektoren, das sind Verschiebungspfeile
die sich addieren und skalieren lassen. Wesentliches Werkzeug sind
außerdem neuartige Rechenoperationen mit geometrischer Deutung: das
Skalarprodukt, das äußere Produkt, das Vektorprodukt
und das Clifford"=Produkt. Als weiteres maßgebliches Werkzeug kam
später die Matrizenrechnung hinzu. Zwischen all diesen Operationen
gibt es vielfältige Beziehungen.

Die Vektorrechnung wurde später selbst weiterentwickelt zur linearen
Algebra, wo die Matrizen als Darstellungen linearer Abbildungen
zwischen Vektorräumen gedeutet werden konnten. Die Vektorrechnung
ist in der linearen Algebra als Spezialfall enthalten, bei dem die
Vektoren aus dem reellen euklidischen Vektorraum entstammen.
Neben diesem kommen in der linearen Algebra noch viele andere
Vektorräume vor. Um Übersicht zu behalten, ermittelt man in der
linearen Algebra abstrakte Regeln und Gesetzmäßigkeiten, die in allen
Vektorräumen gültig sind.

Abgerundet wird die analytische Geometrie durch Isomorphien,
das sind eins"=zu"=eins-Beziehungen zwischen unterschiedlichen
Rechenformalismen. Z.\,B. lassen sich Vektoren in der Ebene auch als
komplexe Zahlen betrachten. Komplexe Zahlen sind wiederum als
spezielle Matrizen darstellbar.

Zu Bemerken ist noch, dass die analytische Geometrie nicht
als Ersatz für die klassische euklidische Geometrie gedacht ist,
sondern als \emph{Vervollständigung}. Die Sätze, Methoden und Beweise
der euklidischen Geometrie behalten ihre Gültigkeit, allerdings
kommen neue Methoden hinzu. Einige Sachverhalte sind etwas leichter mit
klassischer Geometrie verständlich, andere sind besonders
elegant mit Vektorrechnung formulierbar.

\section{Rechnen mit Koordinaten}
\subsection{Die Koordinatenebene}

Die Koordinatenebene ist das kartesische Produkt der reellen
Zahlen mit sich selbst, besteht also aus allen geordneten Paaren,
deren Komponenten reelle Zahlen sind, kurz
\begin{equation}
\R\times\R := \{(x,y)\mid x\in\R\;\text{und}\;y\in\R\}.
\end{equation}
Man kann nun jedem Punkt der euklidischen Ebene ein Koordinatenpaar
zuordnen, indem man ein Koordinatensystem in die Ebene einzeichnet.
Aus Gründen der Einfachheit sollte dieses Koordinatensystem
\emph{kartesisch} sein, das heißt die Koordinatenachsen sollten
rechtwinklig aufeinander stehen und die Skaleneinheit sollte genau
einer Längeneinheit entsprechen.

Die Beschreibung von waagerechten und senkrechten Geraden
erfolgt gemäß Einschränkung der Koordinatenebene auf Teilmengen.
Eine waagerechte Gerade wird durch eine feste Koordinate $y_0$
beschrieben, während die Koordinate $x$ frei variieren darf:
\begin{equation}
\R\times\{y_0\} = \{(x,y_0)\mid x\in\R\}.
\end{equation}
Entsprechend ist bei einer senkrechten Gerade die Koordinate
$x_0$ fest, während $y$ frei variieren darf:
\begin{equation}
\{x_0\}\times\R = \{(x_0,y)\mid y\in\R\}.
\end{equation}
Man bezeichnet mit $\R^+ = \R_{>0}$ die positiven und mit
$\R^- = \R_{<0}$ die negativen reellen Zahlen. Hiermit lassen
sich vier Halbebenen beschreiben:
\begin{equation}
\R^+\times\R,\;\; \R^-\times\R,\;\; \R\times\R^+,\;\; \R\times\R^-.
\end{equation}
Außerdem gibt es vier Quadranten:
\begin{equation}
\R^+\times\R^+,\;\; \R^-\times\R^+,\;\; \R^-\times\R^-,\;\; \R^+\times\R^-.
\end{equation}
Man kann die Betrachtung auch auf Rechtecke einschränken:
\begin{equation}
\begin{split}
&[a,b]\times [c,d]\\
&= \{(x,y)\mid a\le x\le b\;\mathrm{und}\;c\le y\le d\}.
\end{split}
\end{equation}
Entsprechend gibt es offene Rechtecke:
\begin{equation}
\begin{split}
&(a,b)\times (c,d)\\
&= \{(x,y)\mid a<x<b\;\mathrm{und}\;c<y<d\}.
\end{split}
\end{equation}

\subsection{Geraden}

Durch je zwei unterschiedliche Punkte verläuft genau eine Gerade $g$.
Gegeben seien daher zwei Punkte $P_1=(x_1,y_1)$ und $P_2=(x_2,y_2)$
mit $P_1\ne P_2$. Es gilt
\begin{equation}
P_1\ne P_2 \iff x_1\ne x_2\;\text{oder}\; y_1\ne y_2.
\end{equation}
Gibt es nun einen weiteren Punkt $P_0=(x_0,y_0)$, möchte man wissen, ob
$P_0$ auf der Geraden $g$ liegt, kurz $P_0\in g$ gilt. Ein solche
Situation wird klassisch als \emph{Inzidenz} bezeichnet. Zur Lösung
dieser Aufgabe ist die Bestimmung einer beschreibenden Gleichung
der Geraden notwendig. Wie findet man diese Gleichung?

Betrachten wir den Punkt $P_1$. Eine Entfernung $\Delta x$ von $x_1$
führt dann zu einer Entfernung $\Delta y$ von $y_1$. Nach den
Strahlensätzen muss aber das Verhältnis $m=\frac{\Delta y}{\Delta x}$
eine Konstante sein. Dieses feste $m$ wird als \emph{Anstieg}
bezeichnet. D.\,h., für einen beliebigen weiteren
Punkt $P=(x,y)$ auf der Geraden muss die Beziehung
\begin{equation}\label{eq:Anstieg}
m = \frac{\Delta y}{\Delta x} = \frac{y_2-y_1}{x_2-x_1}
= \frac{y-y_1}{x-x_1}
\end{equation}
erfüllt sein. Zum Zeichnen der Gerade ist es ggf. günstig,
diese Gleichung nach $y$ umzuformen, dann ergibt sich eine
Funktion $f$ die jedem $x$ ein $y=f(x)$ zuordnet. Man bekommt
\begin{equation}
f(x) = y_1+\frac{y_2-y_1}{x_2-x_1}(x-x_1),
\end{equation}
bzw. kurz
\begin{equation}
f(x) = y_1+m\cdot (x-x_1).
\end{equation}
Hier besteht allerdings die Einschränkung $x_1\ne x_2$, d.\,h. die
Punkte $P_1$ und $P_2$ dürfen nicht senkrecht aufeinander liegen,
sonst bekommt man eine unerlaubte Division durch null. Entgehen dieser
Einschränkung ist möglich, indem die Gleichung \eqref{eq:Anstieg}
durch Umformung von allen Divisionen befreit wird, das ergibt
\begin{equation}
(y_2-y_1)(x-x_1) = (y-y_1)(x_2-x_1).
\end{equation}
Die Inzidenz lässt sich nun leicht prüfen, indem einfach
$x:=x_0$ und $y:=y_0$ eingesetzt wird. Die Gleichung ist nur
dann erfüllt, wenn $P_0$ auf der Geraden liegt.

\subsection{Kreise}

Viele Kurven sind mit Gleichungen beschreibbar, darunter fallen
auch Kreise. Ein Kreis ist eine Menge von Punkten, die alle
den gleichen Abstand $r$ vom Mittelpunkt haben. Der Mittelpunkt
falle zunächst mit dem Koordinatenursprung zusammen. Ist nun
$P=(x,y)$ ein beliebiger Punkt auf dem Kreis, dann ergibt sich
ein rechtwinkliges Dreieck mit den Kathetenlängen $|x|$ und $|y|$
und der Hypotenusenlänge $r$. Gemäß des Satzes von Pythagoras
muss also $|x|^2+|y|^2=r^2$ sein. Da negative Vorzeichen beim
Quadrieren verschwinden, können die Betragsstriche entfallen.
Der Kreis ist demnach die Punktmenge
\begin{equation}
K(r) := \{(x,y)\mid x^2+y^2=r^2\}.
\end{equation}
Liegt der Mittelpunkt nicht im Koordinatenursprung, sondern im
Punkt $P_0=(x_0,y_0)$, kann man die Differenzen $\Delta x := x-x_0$
und $\Delta y := x-y_0$ bilden, dann sind $|\Delta x|$ und $|\Delta y|$
wieder die Längen der Katheten. Somit ergibt sich
\begin{equation}
(x-x_0)^2+(y-y_0)^2=r^2
\end{equation}
als allgemeine Gleichung.

\subsection{Schnittmengen}

Wir haben gesehen, dass sich bestimmte geometrische Objekte durch
Gleichungen beschreiben lassen. Eine Gleichung ist eine Relation
$R(x,y)$. Jede Relation beschreibt eine Punktmenge
\begin{equation}
R := \{(x,y)\mid R(x,y)\}.
\end{equation}
Eine wichtige Aufgabe in der Geometrie besteht nun darin, für
zwei solcher Punktmengen $R_1$ und $R_2$ die Schnittmenge
$R_1\cap R_2$ zu bestimmen. Der Schnittmenge entspricht
eine und"=Verknüpfung der beiden Relationen, d.\,h.
\begin{equation}
R_1\cap R_2 = \{(x,y)\mid R_1(x,y)\;\text{und}\; R_2(x,y)\},
\end{equation}
bzw.
\begin{equation}
(x,y)\in R_1\cap R_2 \Leftrightarrow R_1(x,y)\;\text{und}\; R_2(x,y).
\end{equation}
Eine solche und"=Verknüpfung wird als \emph{System} von
Relationen bezeichnet. Handelt es sich bei den Relationen um
Gleichungen, spricht man von einem \emph{Gleichungssystem}.

Zunächst ein ganz einfaches Beispiel. Wir wollen eine waagerechte
Gerade $\R\times\{y_0\}$ und eine senkrechte Gerade $\{x_0\}\times\R$
schneiden. Dass sich als Schnittmenge nur ein einziger Punkt ergibt,
und zwar $\{(x_0,y_0)\}$, sollte klar sein, im Zweifelsfall fertige
man eine Skizze an. Das kann man nun auch
genau nachrechnen:
\begin{gather*}
(x,y)\in\R\times\{y_0\}\cap \{x_0\}\times\R\\
\Leftrightarrow x\in\R\;\text{und}\; y=y_0
\;\text{und}\;y\in\R\;\text{und}\;x=x_0\\
\Leftrightarrow x=x_0\;\text{und}\;y=y_0\\
\Leftrightarrow (x,y)=(x_0,y_0).
\end{gather*}
