
\chapter{Algebra}
\section{Gruppentheorie}
\subsection{Grundbegriffe}
\begin{Definition}
Sind $(G,*)$ und $(H,\bullet)$ zwei Gruppen, so
heißt $\varphi\colon G\to H$ \emdef{Gruppenhomomorphismus}
\index{Gruppenhomomorphismus}, wenn
\begin{gather}
\forall g_1,g_2\in G\colon
  \varphi(g_1*g_2) = \varphi(g_1)\bullet\varphi(g_2)
\end{gather}
gilt.
\end{Definition}
\begin{Definition}
\emdef{Direktes Produkt}\index{direktes Produkt}:
\begin{gather}
G\times H := \{(g,h)\mid g\in G, h\in H\},\\
(g_1,h_1)*(g_2,h_2) := (g_1*g_2, h_1*h_2).
\end{gather}
\end{Definition}
\noindent
\strong{Satz von Lagrange:} Für Gruppen $G,H$ gilt:
\begin{equation}
H\le G\implies |G| = |G/H|\cdot |H|.
\end{equation}

\subsection{Gruppenaktionen}
\begin{Definition}
Eine Funktion $f\colon G\times X\to X$ heißt
\emdef{Gruppenaktion}\index{Gruppenaktion}, wenn
\begin{gather}
\hspace{-1em}\forall g_1,g_2{\in}G, x{\in}X\colon f(g_1,f(g_2,x)) = f(g_1 g_2,x),\\
\hspace{-1em}\forall x\in X\colon f(e,x) = x
\end{gather}
gilt, wobei mit $e$ das neutrale Element von $G$ gemeint ist.
Anstelle von $f(g,x)$ wird üblicherweise kurz $gx$ (oder
$g+x$ bei einer Gruppe $(G,+)$) geschrieben.
\end{Definition}
\begin{Definition}
Für ein $x\in X$ wird
\begin{equation}
Gx := \{gx\mid g\in G\}
\end{equation}
\emdef{Bahn}\index{Bahn} oder \emdef{Orbit}\index{Orbit} genannt.
Die Menge
\begin{equation}
G_x := \{g\in G\mid gx=x\}
\end{equation}
wird \emdef{Fixgruppe}\index{Fixgruppe}
oder \emdef{Stabilisator}\index{Stabilisator} genannt.
Die Menge
\begin{equation}
X^g := \{x\in X\mid gx=x\}
\end{equation}
heißt \emdef{Fixpunktmenge}.
\end{Definition}

Fixgruppen sind immer Untergruppen:
\begin{equation}
\forall x\colon G_x\le G.
\end{equation}
Bahnen sind Äquivalenzklassen, die Quotientenmenge
\begin{equation}
X/G := \{Gx\mid x\in X\}
\end{equation}
wird \emdef{Bahnenraum}\index{Bahnenraum} genannt.

\noindent
\strong{Bahnformel}\index{Bahnformel}:
Ist $G$ eine endliche Gruppe, so gilt:
\begin{equation}
|G| = |Gx|\cdot |G_x|.
\end{equation}
\strong{Lemma von Burnside:}\index{Lemma von Burnside}
Ist $G$ eine endliche Gruppe, so gilt:
\begin{equation}
|X/G| = \frac{1}{|G|}\sum_{g\in G}|X^g|.
\end{equation}

\section{Ringe}
\subsection{Polynome}
Für zwei Polynome $f,g\in R[X_1,\ldots,X_n]$ gilt:
\begin{align}
\deg(f+g)&\le \max(\deg f,\deg g),\\
\deg(fg)&\le (\deg f)(\deg g).
\end{align}
Für zwei Polynome $f,g$ mit $\deg f\ne\deg g$ gilt:
\begin{equation}
\deg(f+g) = \max(\deg f,\deg g).
\end{equation}
Ist $R$ ein Integritätsring, so gilt für $f,g\in R[X_1,\ldots,X_n]$:%
\begin{equation}
\deg(fg) = (\deg f)(\deg g).
\end{equation}
Seien $R,S$ kommutative unitäre Ringe, sei $R\subseteq S$
und sei $r\in S$.
Die Funktion $\varphi_r\colon R[X]\to S$ mit
\begin{equation}\textstyle
\varphi_r(\sum_{k=0}^n a_k X^k):=\sum_{k=0}^n a_k r^k 
\end{equation}
ist ein Ringhomomorphismus und wird
\emdef{Einsetzungshomomorphismus}\index{Einsetzungshomomorphimus}
genannt.
\begin{equation}
\end{equation}

