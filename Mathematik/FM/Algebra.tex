
\chapter{Algebra}
\section{Gruppentheorie}
\subsection{Grundbegriffe}
\begin{definition}[Gruppenhomomorphismus]
Sind $(G,*)$ und $(H,\bullet)$ zwei Gruppen, so
heißt $\varphi\colon G\to H$ \emdef{Gruppenhomomorphismus}%
\index{Gruppenhomomorphismus}, wenn
\begin{gather}
\forall g_1,g_2\in G\colon
  \varphi(g_1*g_2) = \varphi(g_1)\bullet\varphi(g_2)
\end{gather}
gilt. Ein \emdef{Gruppenisomorphismus}\index{Isomorphismus!zwischen Gruppen}
ist ein bijektiver Gruppenhomomorphismus, da die Umkehrabbildung
auch wieder ein Gruppenhomomorphismus ist.
\end{definition}
\begin{definition}[Direktes Produkt]
\emdef{Direktes Produkt}\index{direktes Produkt}:
\begin{gather}
G\times H := \{(g,h)\mid g\in G, h\in H\},\\
(g_1,h_1)*(g_2,h_2) := (g_1*g_2, h_1*h_2).
\end{gather}
\end{definition}
\noindent
\strong{Satz von Lagrange.} Für Gruppen $G,H$ gilt:
\begin{equation}
H\le G\implies |G| = |G/H|\cdot |H|.
\end{equation}

\subsection{Gruppenaktionen}\label{Gruppenaktion}
\begin{definition}[Gruppenaktion]
Eine Funktion $f\colon G\times X\to X$ heißt
\emdef{Gruppenaktion}\index{Gruppenaktion}, wenn
\begin{gather}
\hspace{-1em}\forall g_1,g_2{\in}G, x{\in}X\colon f(g_1,f(g_2,x)) = f(g_1 g_2,x),\\
\hspace{-1em}\forall x\in X\colon f(e,x) = x
\end{gather}
gilt, wobei mit $e$ das neutrale Element von $G$ gemeint ist.
Anstelle von $f(g,x)$ wird üblicherweise kurz $gx$ (oder
$g+x$ bei einer Gruppe $(G,+)$) geschrieben.

Anstelle von \emdef{Linksaktionen} kommen auch \emdef{Rechtsaktionen}
vor, die sich von Linksaktionen in der Reihenfolge unterscheiden.
Eine Rechtsaktion $f\colon X\times G\to X$ genügt den Regeln
\begin{gather}
\hspace{-1em}\forall g_1,g_2{\in}G, x{\in}X\colon f(f(x,g_1),g_2) = f(x,g_1 g_2),\\
\hspace{-1em}\forall x\in X\colon f(x,e) = x.
\end{gather}
\end{definition}

\begin{definition}[Orbit, Stabilisator]
Für ein $x\in X$ wird
\begin{equation}\label{eq:Orbit}
Gx := \{gx\mid g\in G\}
\end{equation}
\emdef{Bahn}\index{Bahn} oder
\emdef{Orbit}\index{Orbit!unter einer Gruppenaktion} genannt.
Die Menge
\begin{equation}
G_x := \{g\in G\mid gx=x\}
\end{equation}
wird \emdef{Fixgruppe}\index{Fixgruppe}
oder \emdef{Stabilisator}\index{Stabilisator} genannt.
Die Menge
\begin{equation}
X^g := \{x\in X\mid gx=x\}
\end{equation}
heißt \emdef{Fixpunktmenge}.
\end{definition}

\noindent
Fixgruppen sind immer Untergruppen:
\begin{equation}
\forall x\colon G_x\le G.
\end{equation}
Bahnen sind Äquivalenzklassen, die Quotientenmenge
\begin{equation}
X/G := \{Gx\mid x\in X\}
\end{equation}
wird \emdef{Bahnenraum}\index{Bahnenraum} genannt.

\strong{Bahnformel.}\index{Bahnformel}
Ist $G$ eine endliche Gruppe, so gilt:
\begin{equation}
|G| = |Gx|\cdot |G_x|.
\end{equation}
\strong{Lemma von Burnside.}\index{Lemma von Burnside}
Für eine endliche Gruppe $G$ gilt:%
\begin{equation}
|X/G| = \frac{1}{|G|}\sum_{g\in G}|X^g|.
\end{equation}

\section{Ringe}\index{Ring}
Ist $(R,+,*)$ ein Ring, so gilt für alle $a,b\in R$:
\begin{align}
0*a &= a*0 = 0,\\
(-a)*b &= a*(-b) = -(a*b),\\
(-a)*(-b) &= -(a*b).
\end{align}
\begin{definition}[Ringhomomorphismus]
Sind $(R,+,*)$ und $(R',+',*')$ Ringe, so wird
$\varphi\colon R\to R'$ als \emdef{Ringhomomorphismus}
bezeichnet, wenn
\begin{align}
\varphi(a+b) &= \varphi(a)+'\varphi(b),\\
\varphi(a*b) &= \varphi(a)*'\varphi(b),
\end{align}
für alle $a,b\in R$ gilt und $\varphi(1)=1$ ist.
\end{definition}

\subsection{Polynome}\index{Polynom}
\begin{definition}[Polynom, Polynomring, Koeffizienten]
Sei $R$ ein kommutativer unitärer Ring.
Mit $R[X]$ bezeichnen wir die Menge der unendlichen Folgen
\begin{equation}
(a_k) = (a_0,a_1,\ldots,a_n,0,0,0,\ldots)
\end{equation}
mit $a_k\in R$, bei denen ab einem Index alle Folgenglieder null sind.

Für zwei Folgen aus $R[X]$ wird nun die Addition
\begin{equation}
(a_k) + (b_k) := (a_k+b_k)
\end{equation}
und die Multiplikation
\begin{equation}\label{eq:Faltung}
(a_i)*(b_j) = \bigg(\sum_{i=0}^k a_i b_{k-i}\bigg)
\end{equation}
erklärt. In der Form \eqref{eq:Faltung} wird die Operation auch
\emdef{Faltung}\index{Faltung!von zwei Folgen}
der Folgen $(a_i)$ und $(b_j)$ genannt.

Die Menge $R[X]$ bildet mit der Addition und Multiplikation
einen kommutativen unitären Ring, den \emdef{Polynomring}
mit Koeffizienten in $R$. Ein Element von $R[X]$ wird
\emdef{Polynom} genannt.

Man definiert nun
\begin{equation}
X:=(0,1,0,0,0,\ldots),
\end{equation}
womit sich jedes Polynom in der Form
\begin{equation}\textstyle
(a_k) = \sum_{k=0}^n a_k X^k
\end{equation}
schreiben lässt. Die $a_k$ nennt man \emdef{Koeffizienten}
des Polynoms.
\end{definition}

\noindent
Die Addition bekommt nun die Form
\begin{equation}
\sum_{k=0}^m a_k X^k + \sum_{k=0}^n b_k X^k
:= \sum_{k=0}^p (a_k+b_k)X^k.
\end{equation}
mit $p=\max(m,n)$. Die Multiplikation lässt sich nun in der Form
\begin{equation}
\bigg(\sum_{i=0}^m a_i X^i\bigg)\bigg(\sum_{j=0}^n b_j X^j\bigg)
:= \sum_{k=0}^{m+n}\bigg(\sum_{i=0}^k a_i b_{k-i}\bigg) X^k.
\end{equation}
schreiben. Die Multiplikation von Polynomen ist das gewöhnlichen
Ausmultiplizieren der Polynome, wobei $X^i X^j=X^{i+j}$ gilt.

Die $X^k$ können als Vektorraumbasis betrachtet
werden und dienen dabei dazu, die $a_k$ auseinanderzuhalten.
Zwei Polynome $\sum_{k=0}^m a_k X^k$ und $\sum_{k=0}^n b_k X^k$
sind genau dann gleich, wenn $a_k=b_k$ für alle $k\le\max(m,n)$ gilt.

Da $R[X]$ wieder ein kommutativer unitärer Ring ist,
ist auch $R[X][Y]$ ein Polynomring. Man definiert
\begin{equation}
R[X,Y] := R[X][Y].
\end{equation}
Polynome aus $R[X,Y]$ lassen sich in der Form
\begin{equation}
\sum_{j=0}^n \bigg(\sum_{i=0}^m a_{ij}X^i\bigg)Y^j
= \sum_{i=0}^m\sum_{j=0}^n a_{ij} X^i Y^j
\end{equation}
mit $a_{ij}\in R$ schreiben.

Allgemein ist die Menge
\begin{equation}
R[X_1,\ldots,X_q] := X[X_1,\ldots,X_{q-1}][X_q]
\end{equation}
ein kommutativer unitärer Ring. Die Polynome lassen sich in der Form
\begin{equation}
\sum_{k\in\N_0^q} a_k X^k\quad (a_k\in R)
\end{equation}
mit
\[k=(k_1,\ldots,k_q)\quad\text{und}\quad X^k:=\prod_{i=1}^q X_i^{k_i}\]
schreiben.

\begin{definition}[Grad]
Für ein Polynom $f=\sum_{k=0}^n a_k X^k$ mit $a_n\ne 0$ wird
$n$ als \emdef{Grad} von $f$ bezeichnet. Man schreibt $n=\deg f$.

Für ein Monom $a_{ij} X^i Y^j$ mit $a_{ij}\ne 0$ heißt $i+j$
\emdef{Totalgrad}. Der \emdef {Grad} eines Polynoms
\begin{equation}
\textstyle\sum_{i=1}^m\sum_{j=1}^n a_{ij} X^i Y^j
\end{equation}
ist der maximale Totalgrad aller Monome mit $a_{ij}\ne 0$.
Für Polynome in mehr als zwei Variablen ist die Definition analog.
\end{definition}

\strong{Regeln.}\\
Für zwei Polynome $f,g\in R[X_1,\ldots,X_q]$ gilt:
\begin{align}
\deg(f+g)&\le \max(\deg f,\deg g),\\
\deg(fg)&\le (\deg f)(\deg g).
\end{align}
Für zwei Polynome $f,g$ mit $\deg f\ne\deg g$ gilt:
\begin{equation}
\deg(f+g) = \max(\deg f,\deg g).
\end{equation}
Ist $R$ ein Integritätsring, so gilt für $f,g\in R[X_1,\ldots,X_q]$:%
\begin{equation}
\deg(fg) = (\deg f)(\deg g).
\end{equation}
Jeder Körper, z.\,B. $\R$ oder $\C$ ist ein Integritätsring.
Auch die ganzen Zahlen $\Z$ bilden einen Integritätsring.
Ein Polynomring ist genau dann ein Integritätsring, wenn die
Koeffizienten aus einem Integritätsring entstammen.

\begin{definition}[Einsetzungshomomorphismus]
Seien $R,R'$ kommutative unitäre Ringe. Sei $R'$ eine Ringerweiterung
von $R$ und sei $r\in R'$. Die Abbildung $\varphi_r\colon R[X]\to R'$
mit
\begin{equation}\textstyle
\varphi_r(p) = p(r) := \sum_{k=0}^n a_k r^k 
\end{equation}
für jedes Polynom
\[\textstyle p = \sum_{k=0}^n a_k X^k\]
ist ein Ringhomomorphismus. Man bezeichnet $p(r)$ als \emdef{Einsetzung}
von $r$ in $p$ und $\varphi_r$ als
\emdef{Einsetzungshomomorphismus}\index{Einsetzungshomomorphimus}.
\end{definition}

Man kann auch $R'=R$ und $r=X$ setzen, dann gilt $p=p(X)$.
Ein Polynom stimmt also mit der Einsetzung seiner eigenen formalen
Variablen überein.

\begin{definition}[Polynomfunktion]
Für ein festes $p\in R[X]$ wird die Funktion
\begin{equation}
f\colon R'\to R',\quad f(x):=p(x)
\end{equation}
als \emdef{Polynomfunktion} bezeichnet.
\end{definition}

\noindent
In einigen Ringen können unterschiedliche Polynome zur selben
Polynomfunktion führen. Handelt es sich bei $R$ jedoch um einen
unendlichen Körper, z.\,B. $R=\R$ oder $R=\C$, dann gibt es zu jeder
Polynomfunktion nur ein einziges Polynom.

\section{Körper}
\begin{definition}[Körperhomomorphismus]
Sind $(K,+,\bullet)$ und $(K',+',\bullet')$ Körper, so
wird $\varphi\colon K\to K'$ als \emph{Körperhomomorphismus}
bezeichnet, wenn
\begin{align}
\varphi(a+b) &= \varphi(a)+'\varphi(b),\\
\varphi(a\bullet b) &= \varphi(a)\bullet'\varphi(b)
\end{align}
für alle $a,b\in K$ gilt und $\varphi(1)=1$ ist.
\end{definition}


