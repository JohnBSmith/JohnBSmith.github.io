
\chapter{Funktionen}
\section{Elementare Funktionen}
\subsection{Exponentialfunktion}
\begin{Definition}
$\exp\colon\C\to\C$ mit
\begin{equation}
\exp(x) := \sum_{k=0}^{\infty} \frac{x^k}{k!}
= 1+x+\frac{x^2}{2!}+\frac{x^3}{3!}+\ldots
\end{equation}
\end{Definition}
\noindent
Die Einschränkung von $\exp$ auf $\R$ ist injektiv und
hat die Bildmenge $\{x{\in}\R\mid x>0\}$.

Für $x,y\in\C$ gilt:
\begin{gather}
\exp(x+y) = \exp(x)\exp(y),\\
\exp(x-y) = \frac{\exp(x)}{\exp(y)},\\
\exp(-x) = \frac{1}{\exp(x)}.
\end{gather}
\strong{Eulersche Formel.} Für alle $x\in\C$ gilt:
\begin{equation}
\ee^{\ui x} = \cos x+\ui\sin x.
\end{equation}

\subsection{Winkelfunktionen}\index{Winkelfunktion}
\begin{Definition}
\emdef{Kosinus}\index{Kosinus}\index{Cosinus}: $\C\to\C$,
\begin{equation}
\cos(x) := \sum_{k=0}^\infty \frac{x^{2k}}{(2k)!}
= 1+\frac{x^2}{2!}+\frac{x^4}{4!}+\ldots
\end{equation}
\emdef{Sinus}\index{Sinus}: $\C\to\C$,
\begin{equation}
\sin(x) := \sum_{k=0}^\infty \frac{x^{2k+1}}{(2k+1)!}
= x+\frac{x^3}{3!}+\frac{x^5}{5!}+\ldots
\end{equation}
\emdef{Tangens}\index{Tangens}: $\C\setminus\{k\pi+\pi/2\mid k\in\Z\}\to\C$,
\begin{equation}
\tan(x) := \frac{\sin(x)}{\cos(x)}.
\end{equation}
\emdef{Kotangens}\index{Kotangens}: $\C\setminus\{k\pi\mid k\in\Z\}\to\C$,
\begin{equation}
\cot(x) := \frac{\cos(x)}{\sin(x)}.
\end{equation}
\emdef{Sekans}\index{Sekans}: $\C\setminus\{k\pi+\pi/2\mid k\in\Z\}\to\C$,
\begin{equation}
\sec(x) := \frac{1}{\cos(x)}.
\end{equation}
\emdef{Kosekans}\index{Kosekans}: $\C\setminus\{k\pi\mid k\in\Z\}\to\C$,
\begin{equation}
\csc(x) := \frac{1}{\sin(x)}.
\end{equation}
\end{Definition}
\noindent
Darstellung durch die Exponentialfunktion:\\
Für $x\in\C$ gilt:
\begin{align}
\cos x &= \real(\ee^{\ui x}) = \frac{\ee^{\ui x}+\ee^{-\ui x}}{2},\\
\sin x &= \imag(\ee^{\ui x}) = \frac{\ee^{\ui x}-\ee^{-\ui x}}{2\ui}.
\end{align}

\subsubsection{Symmetrie und Periodizität}
Für $x\in\C$ gilt:
\begin{align}
\sin(-x) &= -\sin x,\enspace(\text{Punktsymmetrie})\\
\cos(-x) &= \cos x,\quad\;(\text{Achsensymmetrie})\\
\sin(x+2\pi) &= \sin x,\\
\cos(x+2\pi) &= \cos x,\\
\sin(x+\pi)  &=-\sin x,\\
\cos(x+\pi)  &=-\cos x,\\
\sin\Big(x+\frac{\pi}{2}\Big) &= \cos x = -\sin\Big(x-\frac{\pi}{2}\Big),\\
\cos\Big(x+\frac{\pi}{2}\Big) &= -\sin x = -\cos\Big(x-\frac{\pi}{2}\Big).
\end{align}

\subsubsection{Additionstheoreme}
\index{Additionstheoreme}

Für $x,y\in\C$ gilt:
\begin{align}
\sin(x+y) &= \sin x\cos y+\cos x\sin y,\\
\sin(x-y) &= \sin x\cos y-\cos x\sin y,\\
\cos(x+y) &= \cos x\cos y-\sin x\sin y,\\
\cos(x-y) &= \cos x\cos y+\sin x\sin y.
\end{align}

\subsubsection{Trigonometrischer Pythagoras}
Für $x\in\C$ gilt:
\begin{equation}
\sin^2 x+\cos^2 x=1.
\end{equation}

\subsubsection{Produkte}
Für $x,y\in\C$ gilt:
\begin{align}
2\sin x\sin y &= \cos(x-y)-\cos(x+y),\\
2\cos x\cos y &= \cos(x-y)+\cos(x+y),\\
2\sin x\cos y &= \sin(x-y)+\sin(x+y).
\end{align}

\subsubsection{Summen und Differenzen}
Für $x,y\in\C$ gilt:
\begin{align}
\sin x+\sin y &= 2\sin\frac{x+y}{2}\cos\frac{x-y}{2},\\
\sin x-\sin y &= 2\cos\frac{x+y}{2}\sin\frac{x-y}{2},\\
\cos x+\cos y &= 2\cos\frac{x+y}{2}\cos\frac{x-y}{2},\\
\cos x-\cos y &= 2\sin\frac{x+y}{2}\sin\frac{y-x}{2}.
\end{align}

\subsubsection{Winkelvielfache}
Für $x\in\C$ gilt:
\begin{align}
\sin(2x) &= 2\sin x\cos x,\\
\cos(2x) &= \cos^2 x-\sin^2 x,\\
\sin(3x) &= 3\sin x-4\sin^3 x,\\
\cos(3x) &= 4\cos^3 x-3\cos x.
\end{align}

\section{Zahlentheoretische Funktionen}
\subsection{Eulersche Phi-Funktion}

\begin{Definition}
\emdef{Eulersche Phi-Funktion}:
\begin{equation}
\varphi(n) := |\{a\in N\mid 1\le a\le n\wedge\operatorname{ggT}(a,n)=1\}|.
\end{equation}
\end{Definition}
\noindent
Für zwei teilerfremde Zahlen $m,n$ gilt:
\begin{equation}
\varphi(mn) = \varphi(m)\,\varphi(n).
\end{equation}
Für jede Primzahlpotenz $p^k$ mit $k\in\Z$ und $k\ge 1$ gilt:
\begin{equation}
\varphi(p^k) = p^k-p^{k-1}.
\end{equation}
Besitzt die Zahl $n$ die Primfaktorzerlegung
\begin{equation}
n=\prod_{p|n} p^{k_p},
\end{equation}
so gilt:
\begin{equation}
\varphi(n) = \prod_{p|n} (p^{k_p}-p^{k_p-1})
= n\prod_{p|n} \Big(1-\frac{1}{p}\Big).
\end{equation}

\subsection{Carmichael-Funktion}
\begin{Definition} \emdef{Carmichael-Funktion}:
\begin{equation}
\begin{split}
\lambda(n) &:= \min\{m\mid \forall a\,[\operatorname{ggT}(a,n)=1\\
&\implies a^m\equiv 1\mod n]\}.
\end{split}
\end{equation}
\end{Definition}
