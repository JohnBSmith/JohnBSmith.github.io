
\chapter{Wahrscheinlichkeitsrechnung}

\section{Diskrete Verteilungen}

\subsection{Diskreter Wahrscheinlichkeitsraum}

\begin{definition}[Ergebnis, Ereignis, Ergebnismenge, Ereignisraum,
unmögliches Ereignis, sicheres Ereignis]\mbox{}\newline%
\index{Ergebnismenge}\index{Ereignisraum}%
\index{sicheres Ereignis}\index{unmögliches Ereignis}
Eine abzählbare \emdef{Ergebnismenge} $\Omega$ ist eine endliche
(oder abzählbar unendliche) Menge, die als Grundmenge verwendet wird.
Ein Element von $\Omega$ heißt \emdef{Ergebnis} oder
\emdef{Elementarereignis}.

Die Potenzmenge $2^\Omega$ heißt \emdef{Ereignisraum}, die
Elemente heißen \emdef{Ereignisse}.
Man nennt die leere Menge $\emptyset$ das \emdef{unmögliche} und $\Omega$
das \emdef{sichere} Ereignis.
\end{definition}

\begin{definition}[Diskreter Wahrscheinlichkeitsraum,\\
Wahrscheinlichkeitsmaß]\mbox{}\newline%
\index{Wahrscheinlichkeitsraum!diskreter}\index{Wahrscheinlichkeitsmaß!diskretes}%
\index{Verteilung!diskrete Wahrscheinlichkeitsverteilung}
Ein Paar $(\Omega,P)$ heißt \emdef{diskreter Wahrscheinlichkeitsraum}, wenn
$\Omega$ eine abzählbare Ergebnismenge ist und%
\begin{equation}
P(A):=\sum_{\omega\in A} P(\{\omega\}),\quad P\colon 2^\Omega\to [0,1]
\end{equation}
die Eigenschaft
\begin{equation}
\sum_{\omega\in\Omega} P(\{\omega\})=1
\end{equation}
besitzt. Die Abbildung $P$ heißt (das von den
Einzelwahrscheinlichkeiten induzierte) \emdef{Wahrscheinlichkeitsmaß}.
Man spricht auch von einer \emdef{Verteilung} auf $\Omega$.
\end{definition}


\subsection{Axiome von Kolmogorow}
\begin{definition}[Wahrscheinlichkeitsmaß\\
(Axiome von Kolmogorow)]\mbox{}\newline%
\index{Wahrscheinlichkeitsmaß!Axiome von Kolmogorow}%
\index{Axiome von Kolmogorow}
Gegeben ist ein Messraum $(\Omega,\Sigma)$. Man nennt $P$ ein
\emdef{Wahrscheinlichkeitsmaß}, wenn gilt:

1. $P$ ist eine Funktion $P\colon\Sigma\to [0,1]$.

2. $P(\Omega)=1$.

3. Ist $I$ eine abzählbare Indexmenge und sind die $A_i$
für $i\in I$ paarweise disjunkte Ereignisse, so gilt
\begin{equation}
P\Big(\bigcup_{i\in I} A_i\Big) = \sum_{i\in I}P(A_i).
\end{equation}
\end{definition}

\noindent
Bei einem diskreten Wahrscheinlichkeitsraum $(\Omega,P)$ mit
$\Sigma=2^\Omega$ sind die Axiome erfüllt.

\subsection{Rechenregeln}
Aus den Axiomen von Kolmogorow folgen folgende
Rechenregeln für ein Wahrscheinlichkeitsmaß $P$:
\begin{gather}
P(\emptyset) = 0,\\
P(\Omega) = 1,\\
P(A\cup B) = P(A)+P(B)-P(A\cap B).
\end{gather}
Man nennt $A^\comp:=\Omega\setminus A$ das
\emdef{komplementäre Ereignis}\index{komplementäres Ereignis}
zu $A$. Es gilt:
\begin{gather}
A\cup A^\comp = \Omega,\\
A\cap A^\comp = \emptyset,\\
P(A\cup A^\comp) = P(A)+P(A^\comp) = 1.
\end{gather}
\strong{Mehrstufige Experimente.}
Ein zweistufiges Zufallsexperiment mit einem ersten Ergebnis aus
$\Omega_1$ und einem zweiten aus $\Omega_2$ lässt sich als
Zufallsexperiment modellieren, bei dem die Ergebnismenge das
kartesische Produkt $\Omega=\Omega_1\times\Omega_2$ ist. Bei einem
$n$-stufigen Experiment gilt
\begin{equation}
\Omega = \Omega_1\times\ldots\times\Omega_n.
\end{equation}

\strong{Erste Pfadregel.}
Sei $a\in\Omega_1$, $b\in\Omega_2$, $A=\{a\}\times\Omega_2$
und $B=\Omega_1\times\{b\}$. Es gilt
\begin{equation}
P(\{(a,b)\}) = P(A\cap B) = P(A)\,P(B|A).
\end{equation}
Das Ereignis $\{(a,b)\}$ tritt ein, wenn zuerst
der Pfad $A$ eingetreten ist, und dann auch der Pfad $B$.
Die Wahrscheinlichkeit ist das Produkt der Pfadwahrscheinlichkeiten
$P(A)$ und $P(B|A)$.

\strong{Zweite Pfadregel.}
Sind $a,b\in\Omega$ zwei unterschiedliche
Ergebnisse, dann gilt
\begin{equation}
P(\{a\}\cup\{b\}) = P(\{a\})+P(\{b\}).
\end{equation}
Wenn die Teilexperimente eines mehrstufigen Experiments
stochastisch unabhängig sind, dann gilt nach der ersten Pfadregel
die Formel
\begin{equation}
P(\{(a_1,\ldots,a_n)\}) = \prod_{k=1}^n P(A_k),
\end{equation}
wobei $A_k$ der Pfad zu $a_k$ ist.
Für den Fall, dass die einzelnen
Experimente alle Laplace-Experimente sind, gilt speziell
\begin{equation}
P(\{(a_1,\ldots,a_n)\}) = \frac{1}{|\Omega|} = \prod_{k=1}^n \frac{1}{|\Omega_k|}
\end{equation}
mit $\Omega=\Omega_1\times\ldots\times\Omega_n$ und $(a_1,\ldots,a_n)\in\Omega$.

Führt man immer wieder dasselbe Laplace-Experiment aus, gilt
mit $t\in\Omega$ und $\Omega=\Omega_1^n$ die Regel
\begin{equation}
P(t) = \frac{1}{|\Omega|} = \frac{1}{|\Omega_1|^n}.
\end{equation}
Würfelt man z.\,B. $n$-mal hintereinander, dann gibt es $6^n$ Pfade
und für jeden Pfad ergibt sich eine Wahrscheinlichkeit von $(1/6)^n$.

\subsection{Bedingte Wahrscheinlichkeit}
\begin{definition}[Bedingte Wahrscheinlichkeit]\mbox{}\newline%
\index{bedingte Wahrscheinlichkeit}
Für zwei Ereignisse $A,B$ mit $P(B)>0$ nennt man
\begin{equation}
P(A|B) := \frac{P(A\cap B)}{P(B)}
\end{equation}
die \emdef{bedingte Wahrscheinlichkeit} von $A$, vorausgesetzt $B$.
\end{definition}
Bei
\begin{equation}
P'(A) := P(A|B),\quad P'\colon 2^B\to [0,1]
\end{equation}
handelt es sich wieder um ein Wahrscheinlichkeitsmaß.

\strong{Satz von Bayes.} Für $P(A)>0$ und $P(B)>0$ gilt
\begin{equation}
P(A|B) = \frac{P(B|A)\, P(A)}{P(B)}.
\end{equation}

\newpage
\subsection{Unabhängige Ereignisse}
\begin{definition}[Stochastische Unabhängigkeit]\mbox{}\newline%
\index{stochastisch unabhängig}
Zwei Ereignisse $A,B$ heißen \emdef{stochastisch unabhängig}, wenn
\begin{equation}
P(A\cap B) = P(A)\, P(B)
\end{equation}
gilt.
\end{definition}

\subsection{Gleichverteilung}
\begin{definition}[Gleichverteilung (Laplace-Verteilung)]%
\mbox{}\newline\index{Gleichverteilung}\index{Laplace-Verteilung}
Sei $\Omega$ eine endliche Ergebnismenge. Mann nennt $P$ eine
\emdef{Gleichverteilung} oder \emdef{Laplace-Verteilung}, wenn
\begin{equation}
P(\{\omega\}) = \frac{1}{|\Omega|}
\end{equation}
für alle Ergebnisse $\omega\in\Omega$ gilt.
\end{definition}

\noindent
Für eine Gleichverteilung gilt
\begin{equation}
P(A) = \frac{|A|}{|\Omega|}.
\end{equation}

\subsection{Zufallsvariablen}
\begin{definition}[Zufallsvariable]\mbox{}\newline
Sei $(\Omega,P)$ ein diskreter Wahrscheinlichkeitsraum. Jede Funktion
\begin{equation}
X\colon\Omega\to\R
\end{equation}
heißt \emdef{Zufallsvariable}. Die Funktionswerte $x=X(\omega)$ heißen
\emdef{Realisationen} der Zufallsvariable.
\end{definition}

\noindent
Eine Zufallsvariable $X$ ordent dem Raum $(\Omega,P)$
einen neuen Wahrscheinlichkeitsraum $(\R,P_X)$ zu, wobei
\begin{equation}
P_X\colon 2^{X(\Omega)}\to [0,1],\; P_X(A):=P(X^{-1}(A))
\end{equation}
definiert wird. Mit
\begin{equation}
X^{-1}(A) := \{\omega\in\Omega\mid X(\omega)\in A\}
\end{equation}
ist das Urbild von $A$ gemeint.
Die folgenden Kurzschreibweisen haben sich
eingebürgert:
\begin{align}
P(X\in A) &:= P(\{\omega\mid X(\omega)\in A\}),\\
P(X=x) &:= P(\{\omega\mid X(\omega)=x\}),\\
P(X\le x) &:= P(\{\omega\mid (X\omega)\le x\}).
\end{align}

\begin{definition}[Verteilungsfunktion]\mbox{}\newline
Für eine Zufallsvariable $X$ wird
\begin{equation}
F(x):=P(X\le x),\quad F\colon\R\to [0,1]
\end{equation}
\emdef{Verteilungsfunktion} von $X$ genannt.
\end{definition}

\noindent
\strong{Eigenschaften von Verteilungsfunktionen.}\\
Für eine Verteilungsfunktion $F$ gilt:
\begin{gather}
\bulletbs F\text{ ist monoton wachsend},\\
\bulletbs F\text{ ist rechtsseitig stetig},\\
\bulletbs \lim\limits_{x\to -\infty} F(x) = 0,\\
\bulletbs \lim\limits_{x\to\infty} F(x)=1,\\
\bulletbs P(a<X\le b) = F(b)-F(a).
\end{gather}

\newpage
\begin{definition}[Zufallszahlengenerator]\mbox{}\\*
Sei $(X_k)$ eine Folge von unabhängigen und
identisch verteilten Zufallsgrößen. Eine Folge
$(x_k)$ von Realisierungen $x_k=X_k(\omega_k)$ wird
Zufallszahlengenerator (kurz RNG, engl. random number generator)
genannt.
\end{definition}

\noindent
Bemerkung: Die $x_k$ werden durch Auswürfeln oder algorithmisch
ermittelt, wobei die $\omega_k$ unbekannt bleiben und auch
mathematisch keine Rolle spielen.

\minisection\strong{Inversionsmethode.}
Die uniforme Verteilung ist definiert durch die Verteilungsfunktion
\begin{equation}
U\colon\R\to [0,1],\quad U(x):=\begin{cases}
0\;\text{wenn}\;x<0,\\
x\;\text{wenn}\;x\in [0,1],\\
1\;\text{wenn}\;x>1.
\end{cases}
\end{equation}
Hat man nur einen Generator $(u_k)$ zur Verfügung, der uniform
verteilte Zufallszahlen erzeugt, möchte aber Zufallszahlen $x_k$
mit Verteilungsfunktion $F$ erzeugen, dann lassen sich diese
gemäß
\begin{equation}
x_k = F^{-1}(u_k)
\end{equation}
ermitteln. Ist $F$ stetig und streng monoton steigend, dann
ist $F^{-1}$ die Umkehrfunktion von $F$, andernfalls setzt man
\begin{equation}
F^{-1}(u) := \inf\{x\in\R\mid F(x)\ge u\}.
\end{equation}


