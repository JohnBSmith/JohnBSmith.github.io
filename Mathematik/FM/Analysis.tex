
\chapter{Analysis}
\section{Ungleichungen}
\subsection{Dreiecksungleichung}
In einem metrischen Raum $(X,d)$ gilt axiomatisch für $x,y,z\in X$
die allgemeine Dreiecksungleichung:
\begin{equation}
d(x,z) \le d(x,y)+d(y,z).
\end{equation}
Infolge gilt auch die umgekehrte Dreiecksungleichung:
\begin{equation}
|d(x,y)-d(y,z)|\le d(x,z).
\end{equation}
Ist $X$ ein normierter Raum, so wird durch $d(x,y) := \|x-y\|$
eine Metrik induziert. Somit gilt
\begin{gather}
\|x-z\| \le \|x-y\|+\|y-z\|,\\
|\|x-y\|-\|y-z\|| \le \|x-z\|.
\end{gather}
Wird nun $x:=x_1$, $z:=-x_2$ und $y:=0$ gesetzt, so ergibt sich
die Dreiecksungleichung für normierte Räume:
\begin{equation}
\|x_1+x_2\| \le \|x_1\|+\|x_2\|,
\end{equation}
und die umgekehrte Dreiecksungleichung:
\begin{equation}
|\|x_1\|-\|x_2\||\le \|x_1-x_2\|.
\end{equation}
Normen sind z.\,B. $\|x\|=|x|$ für $x\in\R$ und $\|z\|=|z|$ für
$z\in\C$. Allgemeiner
\begin{equation}
\|v\|^2 = \sum_{k=1}^n v_k^2
\end{equation}
für einen Koordinatenvektor $v\in\R^n$, $v=(v_k)_{k=1}^n$.

Ist $\langle v,w\rangle$ ein Skalarprodukt, so wird durch
\begin{equation}
\|v\| := \sqrt{\langle v,v\rangle}
\end{equation}
eine Norm induziert.

\subsection{Bernoullische Ungleichung}
Für $x\in\R$, $x\ge -1$ und $n\in\Z$, $n\ge 1$ gilt
\begin{equation}
(1+x)^n \ge 1+nx.
\end{equation}
Die Ungleichung wird nur für $n=1$ oder $x=0$ zu einer Gleichung.

\section{Konvergenz}
\subsection{Beschränkte Folgen}
\begin{definition}[Beschränkte Folge]\mbox{}\newline
Eine Teilmenge $M\subseteq\R$ heißt \emdef{nach oben beschränkt},
wenn
\begin{equation}\label{eq:Schranke-oben}
\exists S_o\,\forall x{\in}M\,(x\le S_o).
\end{equation}
und \emdef{nach unten beschränkt}, wenn
\begin{equation}
\exists S_u\forall x{\in M}\,(x\ge S_u).
\end{equation}
Die Zahl $S_o$ heißt \emdef{obere Schranke}
und $S_u$ heißt \emdef{untere Schranke}. Eine Folge heißt
\emdef{beschränkt}, wenn sowohl eine untere als auch
eine obere Schranke existiert.
\end{definition}

\begin{definition}[Supremum, Infimum]\mbox{}\newline
\emdef{Supremum}:
\begin{equation}
\sup(M) := \min\{S_o\mid \forall x{\in}M\,(x\le S_o)\}.
\end{equation}

\emdef{Infimum}:
\begin{equation}\label{eq:Infimum}
\inf(M) := \max\{S_u\mid \forall x{\in}M\,(x\ge S_u)\}.
\end{equation}
\end{definition}

\begin{definition}[Supremum, Infimum einer Folge]\mbox{}\newline
Bei einer Folge $(a_n)\colon\N\to\R$ sind die
Begriffe \eqref{eq:Schranke-oben}
bis \eqref{eq:Infimum} bezüglich der Bildmenge
von $(a_n)$ definiert.
\end{definition}

\noindent
Jede nach oben beschränkte nichtleere Teilmenge $M\subseteq\R$
besitzt ein Supremum.
Jede nach unten beschränkte nichtleere Teilmenge $M\subseteq\R$
besitzt ein Infimum.
Jede beschränkte nichtleere Teilmenge $M\subseteq\R$
besitzt ein Infimum und ein Supremum.

\subsection{Umgebungen}\index{Umgebung}

Sei $(X,d)$ ein metrischer Raum und $p\in X$.

\begin{definition}[Offene $r$-Umgebung]\mbox{}\newline
\emdef{Offene $r$-Umgebung von $p$}:
\begin{equation}
U_r(p) := \{q\mid d(p,q)<r\}.\qquad (r>0)
\end{equation}
\end{definition}
Standardmetrik:
\begin{align}
d(p,q) &:= |p-q|,\quad\; (\text{$X=\R$,\;\,$X=\C$})\\
d(p,q) &:=\|p-q\|.\quad (\text{normierte Räume})
\end{align}

\subsection{Konvergente Folgen}\index{konvergente Folge}
\begin{definition}[Konvergente Folge]\mbox{}\newline
Eine Folge $(a_n)\colon \N\to X$ heißt \emdef{konvergent} gegen $g$,
wenn%
\begin{equation}\label{eq:konvergent}
\forall r{>}0\enspace\exists n_0\enspace\forall n{>}n_0\;
(a_n\in U_r(g)).
\end{equation}
Man schreibt dann $\lim\limits_{n\to\infty} a_n=g$ bzw. $a_n\to g$
und bezeichnet $g$ als den \emdef{Grenzwert}\index{Grenzwert}
von $(a_n)$.
Hierbei gilt
\begin{equation}
a_n\in U_r(g)\iff d(a_n,g)<r.
\end{equation}
\end{definition}
\strong{Einschnürungssatz.} Seien $(a_n)$ und $(b_n)$ reelle Folgen
mit $a_n\to g$ und $b_n\to g$. Gilt $a_n\le c_n\le b_n$ für fast alle
$n$, so konvergiert $(c_n)$ auch gegen $g$.

\minisection\strong{Vergleichssatz.}
Seien $(a_n)$ und $(b_n)$ konvergente Folgen mit $a_n\to a$ und
$b_n\to b$. Ist $a_n\le b_n$ für alle $n\ge n_0$, dann gilt auch
$a\le b$.

Folgerung: Sei $(a_n)$ eine konvergente Folge mit $a_n\to a$ und
seien $A,B$ reelle Zahlen mit $A\le B$. Ist $A\le a_n\le B$ für alle
$n\ge n_0$, dann gilt auch $A\le a\le B$.

\minisection
\strong{Grenzwertsätze.} Sind $(a_n)$ und $(b_n)$ konvergente
Folgen mit $a_n\to a$ und $b_n\to b$, dann gilt:
\begin{align}
\lim_{n\to\infty} (a_n+b_n) &= a+b,\\
\lim_{n\to\infty} (a_n-b_n) &= a-b,\\
\lim_{n\to\infty} (a_n b_n) &= ab.
\end{align}
Ist zusätzlich für $n\ge n_0$ immer $b_n\ne 0$, dann gilt auch
\begin{equation}
\lim_{n\to\infty} \frac{a_n}{b_n} = \frac{a}{b}.
\end{equation}
Bemerkung: Die Regeln gelten speziell, wenn $(a_n)$ oder $(b_n)$
eine konstante Folge ist, denn eine konstante Folge ist auch
konvergent.

\minisection\strong{Allgemeine Grenzwertsätze.}\\
Sei $f\colon U\to\R$ mit $U\subseteq\R$ und $(a_n)$ eine konvergente
Folge von Werten in $U$ mit $a_n\to a\in U$.
Die Funktion $f$ ist genau dann stetig, wenn für jede solche
Folge gilt:
\begin{equation}
\lim_{n\to\infty} f(a_n) = f(\lim_{n\to\infty} a_n) = f(a).
\end{equation}

\noindent
Sei $f\colon U\to\R$ mit $U\subseteq\R^2$.
Seien $(a_n),(b_n)$ konvergente Folgen von Werten in $U$ mit
$a_n\to a\in U$ und $b_n\to b\in U$. Die Funktion $f$ ist genau dann
stetig, wenn für alle solche Folgen gilt:
\begin{equation}
\lim_{n\to\infty} f(a_n,b_n)
= f(\lim_{n\to\infty}a_n,\lim_{n\to\infty} b_n)
= f(a,b).
\end{equation}

\noindent
\strong{Beschränkte Folgen.}
Ist $(a_n)$ konvergent und gilt für alle $n\ge n_0$ immer $a_n\le s$,
dann ist auch $\lim\limits_{n\to\infty} a_n\le s$.

\minisection\strong{Monotoniekriterium.}
Jede streng monoton wachsende nach oben beschränkte Folge ist
konvergent. Jede streng monoton fallende nach unten beschränkte
Folge ist konvergent.

\minisection\strong{Cauchy-Kriterium.}
Jede reelle Cauchy-Folge ist konvergent.


\subsection{Häufungspunkte}\index{Häufungspunkt}
\begin{definition}[Häufungspunkt]\mbox{}\newline
Eine Punkt $h$ heißt \emdef{Häufungspunkt} einer Folge $(a_n$), wenn
\begin{equation}
\forall r{>}0\enspace\forall n_0\enspace\exists n{>}n_0\;
(a_n\in U_r(h)).
\end{equation}
In Worten: Ein Punkt $h$ heißt Häufungspunkt, wenn in jeder Umgebung
von $h$ unendlich viele Werte der Folge liegen.
\end{definition}

\noindent
Besitzt eine Folge $(a_n)$ einen Grenzwert $g$, so ist $g$ auch ein
Häufungspunkt von $(a_n)$.

\subsection{Cauchy-Folge}\index{Cauchy-Folge}
\begin{definition}[Cauchy-Folge, vollständiger Raum]\mbox{}\newline
Sei $(X,d)$ ein metrischer Raum.
Eine Folge $(a_n)$ heißt \emdef{Cauchy-Folge}, wenn
\begin{equation}
\forall r{>}0\;\exists N{\in}\N\;\forall m{>}N,n{>}N\;
(d(a_m,a_n)<r).
\end{equation}
\noindent
Ein metrischer Raum $(X,d)$ heißt \emdef{vollständig}\index{vollständig},
wenn jede Cauchy-Folge von Punkten aus $X$ einen Grenzwert $g$
mit $g\in X$ besitzt. Ein vollständiger normierter Raum heißt
\emdef{Banachraum}\index{Banachraum}.
\end{definition}

\clearpage
\section{Reihen}\index{Reihe}
\begin{definition}[Reihe]\mbox{}\newline
Sei $(a_n)$ eine Folge. Die Folge $(s_n)$ von
Partialsummen\index{Partialsumme}%
\begin{equation}
s_n = \sum_{k=0}^n a_k
\end{equation}
wird \emdef{Reihe} genannt. Der Grenzwert
\begin{equation}
\sum_{k=0}^\infty a_k := \lim_{n\to\infty}\sum_{k=0}^n a_k
\end{equation}
wird als \emdef{Summe} der Reihe bezeichnet.
\end{definition}
Jede beliebige Folge $(a_n)$ lässt sich durch
\begin{equation}
b_0:=a_0,\quad b_k:=a_k-a_{k-1}
\end{equation}
als Reihe
\begin{equation}\label{eq:Teleskopsumme}
a_n = \sum_{k=0}^n b_k = a_0+\sum_{k=1}^n (a_k-a_{k-1})
\end{equation}
darstellen. Die Summe auf der rechten Seite von \eqref{eq:Teleskopsumme}
wird als \emdef{Teleskopsumme}\index{Teleskopsumme} bezeichnet.

\subsection{Absolute Konvergenz}\index{absolut konvergent}
\begin{definition}[Absolute Konvergenz]\mbox{}\newline
Sei $X$ ein normierter Raum.
Eine Reihe $s_n=\sum_{k=0}^n a_k$ mit $a_k\in X$
heißt \emdef{absolut konvergent}, wenn
\begin{equation}\label{eq:absolut-convergence}\textstyle
\sum_{k=0}^\infty \|a_k\| < \infty.
\end{equation}
\end{definition}
\noindent
Es gilt: $X$ ist ein Banachraum gdw. jede absolut konvergente
Reihe konvergent ist.

Ist $X$ ein Banachraum und $s_n=\sum_{k=0}^n a_k$ eine
absolut konvergente Reihe mit $a_k\in X$, so gilt:
\begin{equation}\label{eq:unconditional-convergence}
\sum_{k=0}^\infty a_k = \sum_{k=0}^\infty a_{\sigma(k)},\quad
\sigma\in\operatorname{Sym}(\N_0).
\end{equation}
Eine konvergente Reihe, für die \eqref{eq:unconditional-convergence}
gilt, heißt \emdef{unbedingt konvergent}\index{unbedingt konvergent}.

\subsection{Konvergenzkriterien}\index{Konvergenzkriterium}
\subsubsection{Nullfolgenkriterium}
Wenn $\lim\limits_{n\to\infty}a_n\ne 0$, dann divergiert $s_n=\sum\limits_{k=0}^n a_k$.

\subsubsection{Quotientenkriterium}\index{Quotientenkriterium}
Gegeben ist eine unendliche Reihe $s_n=\sum_{k=0}^n a_k$, wobei
die $a_k$ reelle oder komplexe Zahlen sind und $a_k\ne 0$ ab einem
gewissen $k$ ist. Gilt
\begin{equation}
\exists q{<}1\;\exists k_0\;\forall k{>}k_0\;(\Big|\frac{a_{k+1}}{a_k}\Big|\le q),
\end{equation}
so ist $(s_n)$ absolut konvergent. S.\,\eqref{eq:absolut-convergence}.
Gilt jedoch
\begin{equation}
\exists k_0\;\forall k{>}k_0\;(\Big|\frac{a_{k+1}}{a_k}\Big|\ge 1),
\end{equation}
so ist $(s_n)$ divergent.

Existiert der Grenzwert
\begin{equation}
g = \lim_{k\to\infty}\Big|\frac{a_{k+1}}{a_k}\Big|,
\end{equation}
so gilt:
\begin{gather}
g<1\implies(s_n)\;\text{ist absolut konvergent},\\
g>1\implies(s_n)\;\text{ist divergent},\\
g=1\implies\;\text{keine Aussage}.
\end{gather}

\subsection{Cauchy-Produkt}\index{Cauchy-Produkt}
Sei
\begin{gather}
\textstyle A_m:=\sum_{n=0}^m a_n,\quad A:=\lim_{m\to\infty} A_m,\\
\textstyle B_m:=\sum_{n=0}^m b_n,\quad B:=\lim_{m\to\infty} B_m,\\
\textstyle C_m:=\sum_{n=0}^m c_n,\quad C:=\lim_{m\to\infty} C_m.
\end{gather}
\pagebreak[2]
\begin{definition}[Cauchy-Produkt]\mbox{}\newline
Das \emdef{Cauchy-Produkt} von zwei Reihen $(A_m)$
und $(B_m)$ ist definiert durch
\begin{equation}
C_m:=\sum_{n=0}^m c_n
\quad\text{mit}\enspace
c_n := \sum_{k=0}^n a_k b_{n-k}.
\end{equation}
\end{definition}
\noindent
Das Cauchy-Produkt von zwei reellen oder komplexen
absolut konvergenten Reihen ist absolut konvergent und es gilt%
\begin{equation}\label{eq:cauchy-product-limits}
C = AB.
\end{equation}
\strong{Satz von Mertens.}
Das Cauchy-Produkt von reellen oder komplexen konvergenten Reihen,
eine davon absolut konvergent, ist konvergent und es gilt
\eqref{eq:cauchy-product-limits}.

\section{Reelle Funktionen}
\begin{definition}[Reelle Funktion]\index{reelle Funktion}%
\mbox{}\newline
Eine Funktion $f\colon D\to\R$ mit $D\subseteq\R$
heißt \emdef{reelle} Funktion.
\end{definition}

\subsection{Monotone Funktionen}
Jede streng monotone reelle Funktion
ist injektiv.

\subsection{Grenzwert einer Funktion}
Ist $f\colon I\to\R$ eine reelle Funktion, $I$ eine offenes Intervall
und $x_0\in I$, so gilt:
\begin{equation}
\begin{split}
&g=\lim_{x\to x_0} f(x)\\
&\iff g=\lim_{x\uparrow x_0} f(x)\;\land\; g=\lim_{x\downarrow x_0} f(x).
\end{split}
\end{equation}

\subsection{Stetige Funktionen}
Sei $f\colon I\to\R$ eine reelle Funktion und $I$ ein offenes
Intervall. Die Funktion $f$ ist stetig bei $x_0\in I$ gdw.
\begin{equation}
\lim_{x\to x_0} f(x)=f(x_0).
\end{equation}
Sind $f,g$ stetige Funktion, so ist auch $g\circ f$ stetig.

\noindent
\strong{Zwischenwertsatz.}\index{Zwischenwertsatz}
Sei $f\colon [a,b]\to\R$ eine stetige Funktion und sei
$a<b$. Bei $f(a)<f(b)$ gilt:
\begin{equation}
\forall y{\in}[f(a),f(b)]\;\exists x{\in}[a,b]\;(y=f(x)).
\end{equation}
Bei $f(a)>f(b)$ gilt:
\begin{equation}
\forall y{\in}[f(b),f(a)]\;\exists x{\in}[a,b]\;(y=f(x)).
\end{equation}

\newpage
\section{Differentialrechnung}\index{Differentialrechnung}
\subsection{Differentialquotient}\index{Differentialquotient}\index{Ableitung}
\begin{definition}[Differentialquotient]\mbox{}\newline
Sei $U\subseteq\R$ ein offenes Intervall
und sei $f\colon U\to\R$. Die Funktion $f$ heißt
differenzierbar\index{differenzierbar}
an der Stelle $x_0\in U$, falls der Grenzwert
\begin{equation}
\begin{split}
&\lim_{x\to x_0} \frac{f(x)-f(x_0)}{x-x_0}
= \lim_{h\to 0}\frac{f(x_0+h)-f(x_0)}{h}
\end{split}
\end{equation}
existiert. Dieser Grenzwert heißt
\emdef{Differentialquotient} oder \emdef{Ableitung}
von $f$ an der Stelle $x_0$. Notation:
\begin{equation}
f'(x_0),\,\qquad (Df)(x_0),\qquad \frac{\mathrm df(x)}{\mathrm dx}\Big|_{x=x_0}.
\end{equation}
\end{definition}

\subsection{Ableitungsregeln}
Sind $f,g,h$ an der Stelle $x$ differenzierbare Funktionen,
ist $h(x)\ne 0$ und ist $a$ eine reelle Zahl, so gilt
\begin{gather}
(af)'(x) = af'(x),\\
(f+g)'(x) = f'(x)+g'(x),\\
(f-g)'(x) = f'(x)-g'(x),\\
(fg)'(x) = f'(x)g(x)+g'(x)f(x),\\
\Big(\frac{f}{h}\Big)'(x) = \frac{f'(x)h(x)-h'(x)f(x)}{h(x)^2}.
\end{gather}
\strong{Kettenregel}.
Ist $g$ differenzierbar an der Stelle $x_0$ und
$f$ differenzierbar an der Stelle $g(x_0)$, dann ist $f\circ g$
differenzierbar an der Stelle $x_0$ und es gilt
\begin{equation}
(f\circ g)'(x_0) = (f'\circ g)(x_0)\cdot g'(x_0).
\end{equation}

\subsection{Ableitung elementarer Funktionen}
\begin{tabular}{l|l|l|l}
$f(x)$ & $f'(x)$ & $D(f)$ & $D(f')$\\
\hline\pstrut{2pt}%
$x^n, n{\in}\N$ & $nx^{n-1}$ & $\R$ & $D(f)$\\
$x^r, r{\in}\R$ & $rx^{r-1}$ & $(0,\infty)$ & $D(f)$\\
$\frac{1}{x}$ & $-\frac{1}{x^2}$ & $\R{\setminus}\{0\}$ & $D(f)$\\
\pstrut{2pt}%
$\sqrt{x}$ & $\frac{1}{2\sqrt{x}}$ & $[0,\infty)$ & $(0,\infty)$\\
$\sqrt[n]{x}$ & $\frac{1}{n\sqrt[n]{x^{n-1}}}$ & $[0,\infty)$ & $(0,\infty)$\\
\pstrut{2pt}%
$\ee^x$ & $\ee^x$ & $\R$ & $D(f)$\\
$a^x, a{>}0$ & $a^x\ln a$ & $\R$ & $D(f)$\\
$\ln x$ & $\frac{1}{x}$ & $(0,\infty)$ & $D(f)$\\
$\log_a x$ & $\frac{1}{x\ln a}$ & $(0,\infty)$ & $D(f)$\\
$\sin x$ & $\cos x$ & $\R$ & $D(f)$\\
$\cos x$ & $\sin x$ & $\R$ & $D(f)$\\
$\tan x$ & $1+\tan^2 x$ & $\{x\mid x{\ne}k\pi{+}\frac{\pi}{2}\}$ & $D(f)$\\
$\cot x$ & $-1-\cot^2 x$ & $\{x\mid x{\ne}k\pi\}$ & $D(f)$\\
$\arcsin x$ & $\frac{1}{\sqrt{1-x^2}}$ & $[-1,1]$ & $(-1,1)$\\
$\arccos x$ & $-\frac{1}{\sqrt{1-x^2}}$ & $[-1,1]$ & $(-1,1)$\\
$\arctan x$ & $\frac{1}{1+x^2}$ & $\R$ & $D(f)$\\
$\arccot x$ & $-\frac{1}{1+x^2}$ & $\R$ & $D(f)$\\
$\sinh x$ & $\cosh x$ & $\R$ & $D(f)$\\
$\cosh x$ & $\sinh x$ & $\R$ & $D(f)$\\
$\tanh x$ & $1-\tanh^2 x$ & $\R$ & $D(f)$\\
$\coth x$ & $1-\coth^2 x$ & $\R{\setminus}\{0\}$ & $D(f)$\\
$\arsinh x$ & $\frac{1}{\sqrt{1+x^2}}$ & $\R$ & $D(f)$\\
$\arcosh x$ & $\frac{1}{\sqrt{x^2-1}}$ & $[1,\infty)$ & $(1,\infty)$\\
$\artanh x$ & $\frac{1}{1-x^2}$ & $(-1,1)$ & $D(f)$\\
$\arcoth x$ & $\frac{1}{1-x^2}$ & $(-\infty,-1){\cup}(1,\infty)$ & $D(f)$
\end{tabular}


\subsection{Tangente und Normale}
Funktionsgleichung der Tangente an den Graphen von
$f$ an der Stelle $x_0$:
\begin{equation}
T(x) = f(x_0)+f'(x_0)(x-x_0).
\end{equation}
Funktionsgleichung der Normale an den Graphen von $f$
an der Stelle $x_0$:
\begin{equation}
N(x) = f(x_0)+\frac{1}{f'(x_0)}(x-x_0).
\end{equation}

\subsection{Taylorreihe}
Sei $f$ eine an der Stelle $a$ unendlich oft differenzierbare
reelle Funktion.

\begin{definition}[Taylorreihe]\mbox{}\newline
\emdef{Taylorreihe} von $f$ an der Stelle $a$:
\begin{equation}
\begin{split}
&f[a](x) :=  (\exp((x-a)D)f)(a)\\
&= \sum_{k=0}^\infty\frac{(D^k f)(a)}{k!}\cdot (x-a)^k\\
&=f(a)+f'(a)\cdot (x-a)+\frac{f''(a)}{2}\cdot (x-a)^2+\ldots
\end{split}
\end{equation}
mit $f^{(k)}(a)=(D^k f)(a)$.
\end{definition}

\noindent
Für Polynomfunktionen und für $\exp$, $\sin$, $\cos$ gilt
\begin{equation}
\forall x{\in}\R\colon\; f[a](x)=f(x).
\end{equation}

\subsection{Kurvendiskussion}
\subsubsection{Extrempunkte}

\begin{definition}[Lokaler Extremwert]\mbox{}\newline
Sei $D$ eine offene Menge und $f\colon D\to\R$. Ein Wert $f(x_0)$
heißt \emdef{lokales Maximum}, wenn
\begin{equation}
\exists r{>}0\;\forall x(x\in U_r(x_0)\implies f(x)\le f(x_0)).
\end{equation}
Ein Wert $f(x_0)$ heißt \emdef{lokales Minimum}, wenn
\begin{equation}
\exists r{>}0\;\forall x(x\in U_r(x_0)\implies f(x)\ge f(x_0)).
\end{equation}
Ist $f(x)=f(x_0)$ nur bei $x=x_0$, dann spricht man von einem
\emdef{strengen} lokalen Minimum bzw. Maximum.
\end{definition}

\section{Integralrechnung}
\subsection{Regelfunktionen}
Ist $T$ eine Treppenfunktion\index{Treppenfunktion}
mit $T(x):=t_k$ für $x\in(x_k,x_{k+1})$,
so gilt:
\begin{equation}
\int_a^b T(x)\,\mathrm dx = \sum_{k=0}^{n-1} (x_{k+1}-x_k)\,t_k.
\end{equation}
\begin{definition}[Regelfunktion]\mbox{}\newline
Eine Funktion $f\colon[a,b]\to\R$ heißt
\emdef{Regelfunktion}\index{Regelfunktion}, wenn es eine
Folge von Treppenfunktionen gibt, die gleichmäßig gegen $f$
konvergiert.
\end{definition}

Ist $(T_n)$ eine gleichmäßig gegen die Regelfunktion $f$ konvergente
Folge von Treppenfunktionen, so gilt:
\begin{equation}
\int_a^b f(x)\,\mathrm dx = \lim_{n\to\infty} \int_a^b T_n(x)\,\mathrm dx.
\end{equation}
Jede stückweise stetige Funktion ist eine Regelfunktion.

\subsection{Stetige Funktionen}
Sei $f\colon [a,b]\to\R$ eine stetige, monoton steigende
Funktion mit $f(x)\ge 0$ auf dem gesamten Definitionsbereich.

Untersumme:
\begin{equation}
\underline{A}_n = \sum_{k=0}^{n-1}
f\Big(a+k\,\frac{b-a}{n}\Big)\,\frac{b-a}{n}.
\end{equation}

Obersumme:
\begin{equation}
\overline{A}_n = \sum_{k=1}^{n}
f\Big(a+k\,\frac{b-a}{n}\Big)\,\frac{b-a}{n}.
\end{equation}
Es gilt:
\begin{equation}
\int_a^b f(x)\,\mathrm dx
= \lim_{n\to\infty}\underline A_n
= \lim_{n\to\infty}\overline A_n.
\end{equation}

\subsection{Hauptsatz}\index{Hauptsatz der Analysis}
Sei $I$ ein Intervall, offen, halboffen, geschlossen oder unendlich.
Sei $f\colon I\to\R$ stetig.

\begin{definition}[Integralfunktion]\mbox{}\newline
\emdef{Integralfunktion}:
\begin{equation}
F(x):=\int_a^x f(x)\,\mathrm dx,\quad F\colon I\to\R.
\end{equation}
\end{definition}

\begin{definition}[Stammfunktion]\mbox{}\newline
Gilt $F'=f$, so wird $F$ \emdef{Stammfunktion} von $f$ genannt.
\end{definition}

\theorem{Hauptsatz}
Die Integralfunktion ist differenzierbar und es gilt $F'=f$.
Ist $f\colon I\to\R$ stetig und $F$ eine Stammfunktion von $f$,
so gilt
\begin{equation}\label{eq:Hauptsatz}
\int_a^b f(x)\,\mathrm dx = [F(x)]_{a}^{b} = F(b)-F(a)
\end{equation}
für $a,b\in I$.

\subsection{Integrationsregeln}
\subsubsection{Linearität}
Für integrierbare Funktionen $f,g\colon [a,b]\to\R$ und eine
Konstante $c\in\R$ gilt die Additivität:
\begin{equation}
\int_a^b f(x)+g(x)\,\mathrm dx
= \int_a^b f(x)\,\mathrm dx+\int_a^b g(x)\,\mathrm dx\\
\end{equation}
und die Homogenität:
\begin{equation}
\int_a^b c f(x)\,\mathrm dx
= c\int_a^b f(x)\,\mathrm dx.
\end{equation}

\subsubsection{Substitutionsregel}
Für $f\in C(I\to\R)$ und
$\varphi\in C^1([a,b]\to\R)$\\
mit $\varphi([a,b])\subseteq I$ gilt%
\begin{equation}
\int_a^b f(\varphi(t))\,\varphi'(t)\,\mathrm dt
= \int_{\varphi(a)}^{\varphi(b)} f(x)\,\mathrm dx.
\end{equation}
\subsubsection{Partielle Integration}
Für $f,g\in C^1([a,b]\to\R)$ gilt
\begin{equation}
\int_a^b f(x)\,g'(x)\,\mathrm dx = [f(x)g(x)]_a^b
- \int_a^b g(x)f'(x)\,\mathrm dx.
\end{equation}

\subsection{Integral bei Polstellen}
Bei Polstellen im Integrationsintervall ist Vorsicht geboten.
Man könnte z.\,B. auf die Idee kommen, dass einfach%
\begin{equation}
\int_{-1}^1 \frac{1}{x^3}\,\mathrm dx
= \Big[-\frac{1}{2x^2}\Big]_{-1}^1 = 0
\end{equation}
gerechnet werden kann. Die Funktion $f(x):=x^{-3}$ besitzt jedoch eine
Polstelle bei $x=0$, ist dort somit nicht definiert und die Lücke
ist auch nicht stetig behebbar. Der Hauptsatz \eqref{eq:Hauptsatz}
setzt aber einen stetigen Integranden voraus. 

Um solche Situationen angehen zu können, ist eine Erweiterung
des Integralbegriffs notwendig.

\begin{definition}[Cauchy-Hauptwert]\mbox{}\newline
\emdef{Cauchy-Hauptwert}\index{Cauchy Hauptwert}
(kurz CH, engl. PV für \emdef{principial value}\index{principial value})
bei einer Definitionslücke $x=c$\,:
\begin{equation}
\begin{split}
&\mathrm{PV}\int_a^b f(x)\,\mathrm dx :=\\
&\quad\lim_{\varepsilon\to 0}\bigg(\int_a^{c-\varepsilon} f(x)\,\mathrm dx
+\int_{c+\varepsilon}^b f(x)\,\mathrm dx\bigg).
\end{split}
\end{equation}
\end{definition}
Nun gilt:
\begin{equation}
\mathrm{PV}\int_{-1}^1 \frac{1}{x^3}\,\mathrm dx = 0.
\end{equation}
Die Flächeninhalte auf beiden Seiten der Polstelle sind
von unterschiedlichem Vorzeichen und heben sich gegenseitig auf.

Eine alternative Erweiterung
ist die Erweiterung des Integranden auf einen komplexen
Definitionsbereich. Da die Funktion $f(z):=z^{-3}$ meromorph
ist, lässt sich der Integrationsweg um die Polstelle herumführen
und es gilt%
\begin{equation}
\int_{-1}^1 \frac{1}{z^3}\,\mathrm dz = 0.
\end{equation}
Zu beachten ist aber, dass z.\,B.
\begin{equation}
\int_{-1}^1 \frac{1}{z^2}\,\mathrm dz = -2
\end{equation}
ist, obwohl
\begin{equation}
\mathrm{PV}\int_{-1}^1 \frac{1}{x^2}\,\mathrm dx
\end{equation}
nicht existiert.

Man beachte auch, dass in der komplexen Ebene der Umlaufsinn um
die Polstelle unter Umständen eine Rolle spielt, denn die
Wegunabhängigkeit des Integrals für einen holomorphen Integranden
ist nur für einfach zusammenhängende Gebiete
sichergestellt. Z.\,B. ist
\begin{equation}
\int_{-1}^1 \frac{1}{z}\,\mathrm dz = -\pi\ui
\end{equation}
für den Integrationsweg oberhalb der Polstelle,
\begin{equation}
\int_{-1}^1 \frac{1}{z}\,\mathrm dz = +\pi\ui
\end{equation}
für den Integrationsweg unterhalb der Polstelle und
\begin{equation}
\mathrm{PV}\int_{-1}^1 \frac{1}{x}\,\mathrm dx = 0.
\end{equation}

% \newpage
\section{Skalarfelder}\index{Skalarfeld!auf dem Koordinatenraum}
Sei $x:=(x_k)_{k=1}^n$ und $a:=(a_k)_{k=1}^n$. Sei $f\colon G\to\R$
wobei $G\subseteq\R^n$ eine offene Menge ist.
\subsection{Partielle Ableitungen}
\begin{definition}[Partielle Ableitung]\mbox{}\newline
Die \emdef{partiellen Ableitungen}\index{partielle Ableitung}
von $f$ an der Stelle $a\in G$ sind definiert durch
\begin{equation}
\begin{split}
&\frac{\partial f(x)}{\partial x_k}\bigg|_{x=a}
:= \frac{\mathrm df(a_1,\ldots,t,\ldots,a_n)}{\mathrm dt}\bigg|_{t=a_k}\\
&= \lim_{h\to 0}\frac{f(a_1,\ldots,a_k+h,\ldots,a_n)-f(a)}{h}.
\end{split}
\end{equation}
Kurzschreibweisen:
\begin{equation}
(D_k f)(a),\quad (\partial_k f)(a).
\end{equation}
\end{definition}
\subsection{Gradient}\index{Gradient}
Sei $(\ue_k)_{k=1}^n$ die kanonische Basis des $\R^n$.
\begin{definition}[Gradient]\mbox{}\newline
An der Stelle $a$:
\begin{equation}
\begin{split}
&(\nabla f)(a) := \textstyle\sum_{k=1}^n \ue_k (D_k f)(a)\\
&= ((D_1 f)(a),\ldots,(D_n f)(a)).
\end{split}
\end{equation}
Formale Schreibweise:
\begin{equation}
\nabla := \textstyle\sum_{k=1}^n \ue_k D_k.
\end{equation}
\end{definition}
\noindent
Ist $(\nabla f)(x)$ stetig bei $x=a$, so
ist $f$ bei $a$ differenzierbar.

\subsubsection{Tangentialraum}
Ist $f\colon G\to\R$ in einer Umgebung von $x_0\in G$
differenzierbar, so existiert bei $x_0$ auf eindeutige Art
ein Tangentialraum, der durch
\begin{equation}
T(x) = f(x_0)+\langle(\nabla f)(x_0),\,x-x_0\rangle
\end{equation}
beschrieben wird.

\subsection{Richtungsableitung}
\begin{definition}[Richtungsableitung]\mbox{}\newline
An der Stelle $a$ in Richtung $v$:
\begin{equation}
\begin{split}
& (D_v f)(a) := \frac{\mathrm d}{\mathrm dt} f(a+tv)\Big|_{t=0}\\
& = \lim_{h\to 0} \frac{f(a+hv)-f(a)}{h}.
\end{split}
\end{equation}
\end{definition}
\noindent
Die partiellen Ableitungen sind die Richtungsableitungen
bezüglich der Standardbasis $(e_k)$:
\begin{equation}
(D_{\displaystyle e_k}f)(a) = (D_k f)(a).
\end{equation}
Ist $f$ an der Stelle $a$ differenzierbar, so gilt:
\begin{equation}
(D_v f)(a) = \langle v,(\nabla f)(a)\rangle
= \sum_{k=1}^n v_k (D_k f)(a).
\end{equation}
Sind $f,g$ an der Stelle $a$ differenzierbar, so gilt dort:
\begin{align}
D_v (f+g) &= D_v f+D_v g,\\
\forall r\in\R\colon D_v (rf) &= rD_v f,\\
D_v (fg) &= gD_v f+fD_v g,\\
D_{v+w} f &= D_v f+D_w f.
\end{align}

\section{Vektorfelder}\index{Vektorfeld!auf dem Koordinatenraum}
Sei $f\colon G\to\R^m$ wobei $G\subseteq\R^n$ eine offene Menge ist.
\begin{definition}[Jacobi-Matrix]\index{Jacobi-Matrix}\mbox{}\newline
An der Stelle $a$:
\begin{equation}
(J[f](a))_{ij} := (D_j f_i)(a).
\end{equation}
Schreibweisen:
\begin{equation}
J[f](a) = (Df)(a) = (\nabla\otimes f)^T(a)
\end{equation}
und
\begin{equation}
J[f](x) = \frac{\partial f(x)}{\partial x}
= \frac{\partial(f_1,\ldots,f_m)}{\partial(x_1,\ldots,x_n)}.
\end{equation}
\end{definition}

\subsection{Tangentialraum}
Ist $f\colon (G\subseteq\R^n)\to\R^m$ bei $x_0\in G$ differenzierbar,
so gibt es dort einen Tangentialraum, der durch
\begin{equation}
T(x) = f(x_0)+(Df)(x_0)\,(x-x_0)
\end{equation}
beschrieben wird.

\subsection{Richtungsableitung}
\begin{definition}[Richtungsableitung]\mbox{}\newline
An der Stelle $a$ in Richtung $v$:
\begin{equation}
(D_v f)(a) := \frac{\mathrm d}{\mathrm dt}f(a+tv)\Big|_{t=0}.
\end{equation}
\end{definition}
\noindent
Ist $f\colon (G\subseteq\R^n)\to\R^m$ bei $a\in G$ differenzierbar,
so gilt:%
\begin{equation}
(D_v f)(a) = (\langle v,\nabla\rangle f)(a) = J[f](a)\,v,
\end{equation}
kurz $D_v = \langle v,\nabla\rangle$.

\section{Variationsrechnung}\index{Variationsrechnung}

\subsection{Fundamentallemma}\index{Fundamentallemma}

Sei $I:=[a,b]$ kompakt und sei $g\colon I\to\R$
stetig. Wenn 
\begin{equation}
\int_a^b g(x)h(x)\,\mathrm dx=0
\end{equation}
für jede unendlich oft differenzierbare Funktion $h\colon I\to\R$
mit $h(a)=h(b)=0$ gilt, so ist $g(x)=0$ für alle $x$.

\subsection{Euler-Lagrange-Gleichung}

Sei $I:=[a,b]$ kompakt. Sei
\begin{equation}
F\colon I\times\R\times\R\to\R
\end{equation}
zweimal stetig differenzierbar. Gesucht ist eine zweimal
stetig differenzierbare Funktion $f\colon I\to\R$ mit fixen
Randwerten $f(a)=A$ und $f(b)=B$, für die
\begin{equation}
J(f) := \int_a^b F(x,f(x),f'(x))\,\mathrm dx
\end{equation}
einen Extremwert annimmt.

Die Euler-Lagrange-Gleichung\index{Euler-Lagrange-Gleichung}
\begin{equation}
\frac{\partial F(x,y,y')}{\partial y}
-\frac{\mathrm d}{\mathrm dx}\frac{\partial F(x,y,y')}{\partial y'}
=0
\end{equation}
mit $y=f(x)$ und $y'=f'(x)$ ist eine notwendige Bedingung dafür.

\newpage
\section{Fourier-Analysis}
\subsection{Fourierreihen}\index{Fourierreihe}
\subsubsection{Fourier-Skalarprodukt}%
\index{Fourier-Skalarprodukt}\index{Skalarprodukt!Fourier-Analysis}
\begin{definition}[Fourier-Skalarprodukt]\mbox{}\newline
Für periodische Funktionen $f,g\colon\R\to\C$ mit\\
Periodendauer $T$:
\begin{equation}\label{eq:Fourier-Skalarprodukt}
\langle f,g\rangle := \frac{1}{T}\int_{t_0}^{t_0+T} \overline{f(t)}g(t)\,\mathrm dt.
\end{equation}
Speziell für $T=2\pi$ und $t_0=-\pi$:
\begin{equation}
\langle f,g\rangle = \frac{1}{2\pi}\int_{-\pi}^{\pi} \overline{f(t)}g(t)\,\mathrm dt.
\end{equation}
Für rein reelle Funktionen ist die Konjugation wirkungslos
und entfällt daher.
\end{definition}
Das Fourier-Skalarprodukt ist ein Skalarprodukt auf dem Hilbertraum
$L^2([t_0,t_0+T])$. Wie jedes Skalarprodukt induziert es eine Norm:%
\begin{equation}\label{eq:Fourier-Norm}
\|f\|^2 = \langle f,f\rangle
= \frac{1}{T}\int_{t_0}^{t_0+T} |f(t)|^2\,\mathrm dt.
\end{equation}
Man nennt $\|f\|$ auch das \emph{quadratische Mittel} von $f$.
Die Definition des Skalarproduktes ist so gewählt, dass es sich
bei $\|f\|$ um den Effektivwert des Signals $f$ handelt.

\subsubsection{Fourier-Koeffizienten}\index{Fourier-Koeffizient}
\strong{Komplexe Fourier-Koeffizienten:}
\begin{equation}
c_k[f] := \frac{1}{T}\int_{t_0}^{t_0+T} \ee^{-k\ui\omega t}f(t)\,\mathrm dt.
\end{equation}
Für $T=2\pi$ und $t_0=-\pi$ gilt speziell:
\begin{equation}
c_k[f] = \frac{1}{2\pi}\int_{-\pi}^{\pi} \ee^{-k\ui t}f(t)\,\mathrm dt.
\end{equation}
Es gilt ($\lambda$: eine Konstante):
\begin{gather}
c_k[f+g] = c_k[f]+c_k[g],\\
c_k[\lambda f] = \lambda c_k[f].
\end{gather}
\strong{Reelle Fourier-Koeffizienten:}
\begin{align}
a_k[f] &:= \frac{2}{T}\int_{t_0}^{t_0+T} \cos(k\omega t)\,f(t)\,\mathrm dt,\\
b_k[f] &:= \frac{2}{T}\int_{t_0}^{t_0+T} \sin(k\omega t)\,f(t)\,\mathrm dt.
\end{align}
Für $T=2\pi$ und $t_0=-\pi$ gilt speziell:
\begin{align}
a_k[f] &= \frac{1}{\pi}\int_{-\pi}^{\pi} \cos(kt)\,f(t)\,\mathrm dt,\\
b_k[f] &= \frac{1}{\pi}\int_{-\pi}^{\pi} \sin(kt)\,f(t)\,\mathrm dt.
\end{align}
Es gilt ($\lambda$: eine Konstante):
\begin{gather}
a_k[f+g] = a_k[f]+a_k[g],\\
b_k[f+g] = b_k[f]+b_k[g],\\
a_k[\lambda f] = \lambda a_k[f],\\
b_k[\lambda f] = \lambda b_k[f].
\end{gather}

\newpage\noindent
Umrechnung zwischen den reellen und den komplexen
Koeffizienten für $k>0$:
\begin{align}
c_0 &= \frac{1}{2}a_0, & a_0&=2c_0,\\
c_k &= \frac{1}{2}(a_k-b_k\ui), & a_k &= c_k+c_{-k},\\
c_{-k} &= \frac{1}{2}(a_k+b_k\ui), & b_k &= (c_k-c_{-k})\ui.
\end{align}

\subsubsection{Fourierreihe}

\strong{Reelles Fourier-Polynom:}
\begin{equation}
p_n(t) := \frac{a_0}{2}+\sum_{k=1}^n [a_k\cos(k\omega t)+b_k\sin(k\omega t)].
\end{equation}
\strong{Komplexes Fourier-Polynom:}
\begin{equation}
p_n(t) := \sum_{k=-n}^n c_k\ee^{\ui k\omega t}.
\end{equation}
\begin{definition}[Fourierreihe]\mbox{}\newline
Sind $c_k[f]$ die Fourierkoeffizienten zu $f$ und ist $p_n$
das daraus gebildete Fourierpolynom, dann bezeichnet man die
Folge $(p_n)$ als \emdef{Fourierreihe} von $f$.
\end{definition}
Ist $f$ stetig differenzierbar, dann gilt
\begin{equation}
f(t) = \lim_{n\to\infty} p_n(t)
\end{equation}
für alle $t$. Ist $f\in L^2([t_0,t_0+T])$, dann gilt
\begin{equation}\label{eq:Konvergenz-RMS}
f = \lim_{n\to\infty} p_n,
\end{equation}
bezüglich \eqref{eq:RMS-eq}. Das heißt, es liegt Konvergenz im
quadratischen Mittel vor:%
\begin{equation}
\lim_{n\to\infty}\|f-p_n\| = 0.
\end{equation}

\subsubsection{Raum der quadratintegrablen\\
Funktionen}
Sei $I=[t_0,t_0+T]$. Man definiert zunächst
\begin{equation}
\mathcal L^2(I) := \{f\colon I\to\C\mid \|f\|<\infty\},
\end{equation}
wobei $\|f\|$ gemäß \eqref{eq:Fourier-Norm} definiert ist.
Für zwei Funktionen $f,g\in \mathcal L^2(I)$ ist wie folgt
eine Äquivalenzrelation gegeben:%
\begin{equation}\label{eq:RMS-eq}
f\sim g \defiff \|f-g\|=0.
\end{equation}
Man sagt, die Funktionen stimmen \emph{im quadratischen Mittel}
überein. Der Quotientenraum
\begin{equation}
L^2(I) := \mathcal L^2(I)/{\sim}
\end{equation}
bildet bezüglich \eqref{eq:Fourier-Skalarprodukt} einen Hilbertraum,
welcher als \emph{Raum der quadratintegrablen Funktionen} bezeichnet
wird.

Die Fourier-Basis
\begin{equation}
B := \{e_k\mid k\in\Z\},\quad e_k(t):=\ee^{\ui k\omega t}
\end{equation}
ist eine Orthonormalbasis von $L^2(I)$. Die gegen $f$ konvergente
Fourier-Reihe \eqref{eq:Konvergenz-RMS} bekommt nun die Form
\begin{equation}
f = \sum_{k\in\Z} \langle e_k,f\rangle\, e_k. 
\end{equation}
Hierbei gilt $c_k[f]=\langle e_k,f\rangle$.
