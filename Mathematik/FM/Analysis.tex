
\chapter{Analysis}
\section{Konvergenz}
\subsection{Umgebungen}\index{Umgebung}
Sei $(X,T)$ ein topologischer Raum und $x\in X$.
\begin{Definition}
\emdef{Umgebungsfilter}:\index{Umgebungsfilter}
\begin{equation}
\mathfrak U(x) := \{U\subseteq X\mid x\in O\land O\in T
\land O\subseteq U\}.
\end{equation}
Ein $U\in\mathfrak U(x)$ wird Umgebung von $x$ genannt.
\end{Definition}
\begin{Definition}
Eine Menge $\mathfrak B(x)\subseteq \mathfrak U(x)$
heißt \emdef{Umgebungsbasis} gdw.
\begin{equation}
\forall U{\in}\mathfrak U(x)\,\exists B{\in}\mathfrak B(x)\colon
B\subseteq U.
\end{equation}
\end{Definition}

\noindent
Sei $(X,d)$ ein metrischer Raum und $x\in X$.
\begin{Definition}
$\varepsilon$-\emdef{Umgebung}:
\begin{equation}\label{eq:epsilon-Umgebung}
U_\varepsilon(x) := \{y\in X\mid d(x,y)<\varepsilon\}.
\end{equation}
\emdef{Punktierte $\varepsilon$-Umgebung}:
\begin{equation}
\dot U_\varepsilon(x) := U_\varepsilon(x)\setminus\{x\}.
\end{equation}
\end{Definition}
\noindent
Bei
\begin{equation}
\mathfrak B(x) = \{U_\varepsilon(x)\mid\varepsilon>0\}
\end{equation}
handelt es sich um eine Umgebungsbasis.

Für einen normierten Raum ist durch $d(x,y):=\|x-y\|$ eine
Metrik gegeben. Speziell für $X=\R$ oder $X=\C$ wird fast immer
$d(x,y):=|x-y|$ verwendet.

\subsection{Konvergente Folgen}\index{konvergente Folge}
\begin{Definition}
Eine Folge $(a_n)\colon \N\to X$ heißt \emdef{konvergent} gegen $g$, wenn%
\begin{equation}\label{eq:konvergent}
\forall U{\in}\mathfrak B(g)\,\exists n_0\,\forall n{>}n_0\colon\;
a_n\in U.
\end{equation}
Man schreibt dann $\lim\limits_{n\to\infty} a_n=g$ und bezeichnet
$g$ als \emdef{Grenzwert}\index{Grenzwert}.
\end{Definition}

Für eine Folge $(a_n)\colon\N\to\R$ wird \eqref{eq:konvergent} zu:
\begin{equation}
\forall\varepsilon{>}0\;\exists n_0\;\forall n{>}n_0\colon\;
|a_n-g|<\varepsilon.
\end{equation}
\strong{Sandwichsatz:} Seien $(a_n)$ und $(b_n)$ reelle Folgen
mit $a_n\to g$ und $b_n\to g$. Gilt $a_n\le c_n\le b_n$ für fast alle
$n$, so konvergiert $(c_n)$ auch gegen $g$.

\subsection{Häufungspunkte}\index{Häufungspunkt}
\begin{Definition}
Eine Punkt $h$ heißt \emdef{Häufungspunkt} einer Folge $(a_n$), wenn
\begin{equation}
\forall U{\in}\mathfrak B(h)\;\forall n_0\;\exists n{>}n_0\colon\;
a_n\in U.
\end{equation}
\end{Definition}
\noindent
Besitzt eine Folge $(a_n)$ einen Grenzwert $g$, so ist $g$ auch ein
Häufungspunkt von $(a_n)$.

\subsection{Cauchy-Folge}\index{Cauchy-Folge}
Sei $(X,d)$ ein metrischer Raum.
\begin{Definition}
Eine Folge $(a_n)$ heißt \emdef{Cauchy-Folge}
gdw.
\begin{equation}
\forall\varepsilon{>}0\;\exists N{\in}\N\;\forall m,n>N\colon\;
d(a_m,a_n)<\varepsilon.
\end{equation}
\end{Definition}
\noindent
Ein metrischer Raum $(X,d)$ heißt \emdef{vollständig}\index{vollständig},
wenn jede Cauchy-Folge von Punkten aus $X$ einen Grenzwert $g$
mit $g\in X$ besitzt. Ein vollständiger normierter Raum heißt
\emdef{Banachraum}\index{Banachraum}.

\section{Reihen}\index{Reihe}
\begin{Definition}
Sei $(a_n)$ eine Folge. Die Folge $(s_n)$ von
Partialsummen\index{Partialsumme}
\begin{equation}
s_n = \sum_{k=0}^n a_k
\end{equation}
wird \emdef{Reihe} genannt. Der Grenzwert
\begin{equation}
\sum_{k=0}^\infty a_k := \lim_{n\to\infty}\sum_{k=0}^n a_k
\end{equation}
wird als \emdef{Summe} der Reihe bezeichnet.
\end{Definition}
Jede beliebige Folge $(a_n)$ lässt sich durch
\begin{equation}
b_0:=a_0,\quad b_k:=a_k-a_{k-1}
\end{equation}
als Reihe
\begin{equation}\label{eq:Teleskopsumme}
a_n = \sum_{k=0}^n b_k = a_0+\sum_{k=1}^n (a_k-a_{k-1})
\end{equation}
darstellen. Die Summe auf der rechten Seite von \eqref{eq:Teleskopsumme}
wird als \emdef{Teleskopsumme}\index{Teleskopsumme} bezeichnet.

\subsection{Absolute Konvergenz}\index{absolut konvergent}
Sei $X$ ein normierter Raum.
\begin{Definition}
Eine Reihe $s_n=\sum_{k=0}^n a_k$ mit $a_k\in X$
heißt \emdef{absolut konvergent}, wenn
\begin{equation}\label{eq:absolut-convergence}\textstyle
\sum_{k=0}^\infty \|a_k\| < \infty.
\end{equation}
\end{Definition}
\noindent
Es gilt: $X$ ist ein Banachraum gdw. jede absolut konvergente
Reihe konvergent ist.

Ist $X$ ein Banachraum und $s_n=\sum_{k=0}^n a_k$ eine
absolut konvergente Reihe mit $a_k\in X$, so gilt:
\begin{equation}\label{eq:unconditional-convergence}
\sum_{k=0}^\infty a_k = \sum_{k=0}^\infty a_{\sigma(k)},\quad
\sigma\in\operatorname{Sym}(\N_0).
\end{equation}
Eine konvergente Reihe, für die \eqref{eq:unconditional-convergence}
gilt, heißt \emdef{unbedingt konvergent}\index{unbedingt konvergent}.

\subsection{Konvergenzkriterien}\index{Konvergenzkriterium}
\subsubsection{Quotientenkriterium}\index{Quotientenkriterium}
Gegeben ist eine unendliche Reihe $s_n=\sum_{k=0}^n a_k$, wobei
die $a_k$ reelle oder komplexe Zahlen sind und $a_k\ne 0$ ab einem
gewissen $k$ ist. Gilt
\begin{equation}
\exists q{<}1\;\exists k_0\;\forall k{>}k_0\colon \bigg|\frac{a_{k+1}}{a_k}\bigg|\le q,
\end{equation}
so ist $(s_n)$ absolut konvergent. S.\,\eqref{eq:absolut-convergence}.
Gilt jedoch
\begin{equation}
\exists k_0\;\forall k{>}k_0\colon\bigg|\frac{a_{k+1}}{a_k}\bigg|\ge 1,
\end{equation}
so ist $(s_n)$ divergent.

Existiert der Grenzwert
\begin{equation}
g = \lim_{k\to\infty}\bigg|\frac{a_{k+1}}{a_k}\bigg|,
\end{equation}
so gilt:
\begin{gather}
g<1\implies(s_n)\;\text{ist absolut konvergent},\\
g>1\implies(s_n)\;\text{ist divergent},\\
g=1\implies\;\text{keine Aussage}.
\end{gather}

\subsection{Cauchy-Produkt}\index{Cauchy-Produkt}
Sei
\begin{gather}
\textstyle A_m:=\sum_{n=0}^m a_n,\quad A:=\lim_{m\to\infty} A_m,\\
\textstyle B_m:=\sum_{n=0}^m b_n,\quad B:=\lim_{m\to\infty} B_m,\\
\textstyle C_m:=\sum_{n=0}^m c_n,\quad C:=\lim_{m\to\infty} C_m.
\end{gather}
\begin{Definition}
Das \emdef{Cauchy-Produkt} von zwei Reihen $(A_m)$
und $(B_m)$ ist definiert durch
\begin{equation}
C_m:=\sum_{n=0}^m c_n
\quad\text{mit}\enspace
c_n := \sum_{k=0}^n a_k b_{n-k}.
\end{equation}
\end{Definition}
\noindent
Das Cauchy-Produkt von zwei reellen oder komplexen
absolut konvergenten Reihen ist absolut konvergent und es gilt
\begin{equation}\label{eq:cauchy-product-limits}
C = AB.
\end{equation}
\strong{Satz von Mertens:}
Das Cauchy-Produkt von reellen oder komplexen konvergenten Reihen,
eine davon absolut konvergent, ist konvergent und es gilt
\eqref{eq:cauchy-product-limits}.

\section{Reelle Funktionen}
\begin{Definition}\index{reelle Funktion}
Eine Funktion $f\colon D\to\R$ mit $D\subseteq\R$
heißt \emdef{reelle} Funktion.
\end{Definition}

\subsection{Monotone Funktionen}
Jede streng monotone reelle Funktion
ist injektiv.

\subsection{Grenzwert einer Funktion}
Ist $f\colon I\to\R$ eine reelle Funktion, $I$ eine offenes Intervall
und $x_0\in I$, so gilt:
\begin{equation}
\begin{split}
&g=\lim_{x\to x_0} f(x)\\
&\iff g=\lim_{x\uparrow x_0} f(x)\;\land\; g=\lim_{x\downarrow x_0} f(x).
\end{split}
\end{equation}

\subsection{Stetige Funktionen}
Sei $f\colon I\to\R$ eine reelle Funktion und $I$ ein offenes
Intervall. Die Funktion $f$ ist stetig bei $x_0\in I$ gdw.
\begin{equation}
\lim_{x\to x_0} f(x)=f(x_0).
\end{equation}
Sind $f,g$ stetige Funktion, so ist auch $g\circ f$ stetig.

\noindent
\strong{Zwischenwertsatz:}\index{Zwischenwertsatz}
Sei $f\colon [a,b]\to\R$ eine stetige Funktion und sei
$a<b$. Bei $f(a)<f(b)$ gilt:
\begin{equation}
\forall y\in [f(a),f(b)]\enspace\exists x\in [a,b]\colon y=f(x).
\end{equation}
Bei $f(a)>f(b)$ gilt:
\begin{equation}
\forall y\in [f(b),f(a)]\enspace\exists x\in [a,b]\colon y=f(x).
\end{equation}

\newpage
\section{Differentialrechnung}\index{Differentialrechnung}
\subsection{Differentialquotient}\index{Differentialquotient}\index{Ableitung}
\begin{Definition}
Sei $U\subseteq\R$ ein offenes Intervall
und sei $f\colon U\to\R$. Die Funktion $f$ heißt
differenzierbar\index{differenzierbar}
an der Stelle $x_0\in U$, falls der Grenzwert
\begin{equation}
\begin{split}
&\lim_{x\to x_0} \frac{f(x)-f(x_0)}{x-x_0}
= \lim_{h\to 0}\frac{f(x_0+h)-f(x_0)}{h}
\end{split}
\end{equation}
existiert. Dieser Grenzwert heißt
\emdef{Differentialquotient} oder \emdef{Ableitung}
von $f$ an der Stelle $x_0$. Notation:
\begin{equation}
f'(x_0),\,\qquad (Df)(x_0),\qquad \frac{\mathrm df(x)}{\mathrm dx}\Big|_{x=x_0}.
\end{equation}
\end{Definition}

\subsection{Ableitungsregeln}
Sind $f,g$ differenzierbare Funktionen und ist $a$ eine reelle Zahl,
so gilt
\begin{gather}
(af)'(x) = af'(x),\\
(f+g)'(x) = f'(x)+g'(x),\\
(f-g)'(x) = f'(x)-g'(x),\\
(fg)'(x) = f'(x)g(x)+g'(x)f(x),\\
\Big(\frac{f}{g}\Big)'(x) = \frac{f'(x)g(x)-g'(x)f(x)}{g(x)^2}.
\end{gather}
\subsubsection{Kettenregel}
Ist $g$ differenzierbar an der Stelle $x_0$ und
$f$ differenzierbar an der Stelle $g(x_0)$, so ist $f\circ g$
differenzierbar an der Stelle $x_0$ und es gilt
\begin{equation}
(f\circ g)'(x_0) = (f'\circ g)(x_0)\, g'(x_0).
\end{equation}
\subsection{Tangente und Normale}
Funktionsgleichung der Tangente an den Graphen von
$f$ an der Stelle $x_0$:
\begin{equation}
T(x) = f(x_0)+f'(x_0)(x-x_0).
\end{equation}
Funktionsgleichung der Normale an den Graphen von $f$
an der Stelle $x_0$:
\begin{equation}
N(x) = f(x_0)+\frac{1}{f'(x_0)}(x-x_0).
\end{equation}

\subsubsection{Taylorreihe}
Sei $f$ eine an der Stelle $a$ unendlich oft differenzierbare
reelle Funktion.
\begin{Definition}
\emdef{Taylorreihe} von $f$ an der Stelle $a$:
\begin{equation}
\begin{split}
&f[a](x) :=  (\exp((x-a)D)f)(a)\\
&= \sum_{k=0}^\infty\frac{(D^k f)(a)}{k!}\cdot (x-a)^k\\
&=f(a)+f'(a)\cdot (x-a)+\frac{f''(a)}{2}\cdot (x-a)^2+\ldots
\end{split}
\end{equation}
mit $f^{(k)}(a)=(D^k f)(a)$.
\end{Definition}

Für Polynomfunktionen und für $\exp$, $\sin$, $\cos$ gilt
\begin{equation}
\forall x\colon\; f[a](x)=f(x).
\end{equation}

\section{Integralrechnung}
\subsection{Regelfunktionen}
Ist $T$ eine Treppenfunktion\index{Treppenfunktion}
mit $T(x):=t_k$ für $x\in(x_k,x_{k+1})$,
so gilt:
\begin{equation}
\int_a^b T(x)\,\mathrm dx = \sum_{k=0}^{n-1} (x_{k+1}-x_k)\,t_k.
\end{equation}
\begin{Definition}
Eine Funktion $f\colon[a,b]\to\R$ heißt
\emdef{Regelfunktion}\index{Regelfunktion}, wenn es eine
Folge von Treppenfunktionen gibt, die gleichmäßig gegen $f$
konvergiert.
\end{Definition}

Ist $(T_n)$ eine gleichmäßig gegen die Regelfunktion $f$ konvergente
Folge von Treppenfunktionen, so gilt:
\begin{equation}
\int_a^b f(x)\,\mathrm dx = \lim_{n\to\infty} \int_a^b T_n(x)\,\mathrm dx.
\end{equation}
Jede stückweise stetige Funktion ist eine Regelfunktion.

\subsection{Stetige Funktionen}
Sei $f\colon [a,b]\to\R$ eine stetige, monoton steigende
Funktion mit $f(x)\ge 0$ auf dem gesamten Definitionsbereich.

Untersumme:
\begin{equation}
\underline{A}_n = \sum_{k=0}^{n-1}
f\Big(a+k\,\frac{b-a}{n}\Big)\,\frac{b-a}{n}
\end{equation}

Obersumme:
\begin{equation}
\overline{A}_n = \sum_{k=1}^{n}
f\Big(a+k\,\frac{b-a}{n}\Big)\,\frac{b-a}{n}
\end{equation}
Es gilt:
\begin{equation}
\int_a^b f(x)\,\mathrm dx
= \lim_{n\to\infty}\underline A_n
= \lim_{n\to\infty}\overline A_n.
\end{equation}

\subsection{Hauptsatz}\index{Hauptsatz der Analysis}
\begin{Definition} \emdef{Integralfunktion}:
\begin{equation}
F(x):=\int_a^x f(x)\,\mathrm dx.
\end{equation}
\end{Definition}

\newpage
\section{Skalarfelder}
Sei $x:=(x_k)_{k=1}^n$ und $a:=(a_k)_{k=1}^n$. Sei $f\colon G\to\R$
wobei $G\subseteq\R^n$ eine offene Menge ist.
\subsection{Partielle Ableitungen}
\begin{Definition}
Die \emdef{partiellen Ableitungen}\index{partielle Ableitung}
von $f$ an der Stelle $a\in G$ sind definiert durch
\begin{equation}
\begin{split}
&\frac{\partial f(x)}{\partial x_k}\bigg|_{x=a}
:= \frac{\mathrm df(a_1,\ldots,t,\ldots,a_n)}{\mathrm dt}\bigg|_{t=a_k}\\
&= \lim_{h\to 0}\frac{f(a_1,\ldots,a_k+h,\ldots,a_n)-f(a)}{h}.
\end{split}
\end{equation}
Kurzschreibweisen:
\begin{equation}
(D_k f)(a),\quad (\partial_k f)(a).
\end{equation}
\end{Definition}
\subsection{Gradient}
Sei $(\ue_k)_{k=1}^n$ die kanonische Basis des $\R^n$.
\begin{Definition} \emdef{Gradient} an der Stelle $a$:
\begin{equation}
\begin{split}
&(\nabla f)(a) := \textstyle\sum_{k=1}^n \ue_k (D_k f)(a)\\
&= ((D_1 f)(a),\ldots,(D_n f)(a)).
\end{split}
\end{equation}
Formale Schreibweise:
\begin{equation}
\nabla := \textstyle\sum_{k=1}^n \ue_k D_k.
\end{equation}
\end{Definition}
\noindent
Ist $(\nabla f)(x)$ stetig bei $x=a$, so
ist $f$ bei $a$ differenzierbar.

\subsubsection{Tangentialraum}
Ist $f\colon G\to\R$ in einer Umgebung von $x_0\in G$
differenzierbar, so existiert bei $x_0$ auf eindeutige Art
ein Tangentialraum, der durch
\begin{equation}
T(x) = f(x_0)+\langle(\nabla f)(x_0),\,x-x_0\rangle
\end{equation}
beschrieben wird.

\subsection{Richtungsableitung}
\begin{Definition} \emdef{Richtungsableitung} an der Stelle $a$
in Richtung $v$:
\begin{equation}
\begin{split}
& (D_v f)(a) := \frac{\mathrm d}{\mathrm dt} f(a+tv)\Big|_{t=0}\\
& = \lim_{h\to 0} \frac{f(a+hv)-f(a)}{h}.
\end{split}
\end{equation}
\end{Definition}
\noindent
Die partiellen Ableitungen sind die Richtungsableitungen
bezüglich der Standardbasis $(e_k)$:
\begin{equation}
(D_{\displaystyle e_k}f)(a) = (D_k f)(a).
\end{equation}
Ist $f$ an der Stelle $a$ differenzierbar, so gilt:
\begin{equation}
(D_v f)(a) = \langle v,(\nabla f)(a)\rangle
= \sum_{k=1}^n v_k (D_k f)(a).
\end{equation}
Sind $f,g$ an der Stelle $a$ differenzierbar, so gilt dort:
\begin{align}
D_v (f+g) &= D_v f+D_v g,\\
\forall r\in\R\colon D_v (rf) &= rD_v f,\\
D_v (fg) &= gD_v f+fD_v g,\\
D_{v+w} f &= D_v f+D_w f.
\end{align}

\section{Vektorfelder}
Sei $f\colon G\to\R^m$ wobei $G\subseteq\R^n$ eine offene Menge ist.
\begin{Definition} \emdef{Jacobi-Matrix} an der Stelle $a$:
\begin{equation}
(J[f](a))_{ij} := (D_j f_i)(a).
\end{equation}
Schreibweisen:
\begin{equation}
J[f](a) = (Df)(a) = (\nabla\otimes f)^T(a)
\end{equation}
und
\begin{equation}
J[f](x) = \frac{\partial f(x)}{\partial x}
= \frac{\partial(f_1,\ldots,f_m)}{\partial(x_1,\ldots,x_n)}.
\end{equation}
\end{Definition}

\subsection{Tangentialraum}
Ist $f\colon (G\subseteq\R^n)\to\R^m$ bei $x_0\in G$ differenzierbar,
so gibt es dort einen Tangentialraum, der durch
\begin{equation}
T(x) = f(x_0)+(Df)(x_0)\,(x-x_0)
\end{equation}
beschrieben wird.

\subsection{Richtungsableitung}
\begin{Definition} \emdef{Richtungsableitung} von $f$ an der
Stelle $a$:
\begin{equation}
(D_v f)(a) := \frac{\mathrm d}{\mathrm dt}f(a+tv)\Big|_{t=0}.
\end{equation}
\end{Definition}
\noindent
Ist $f\colon (G\subseteq\R^n)\to\R^m$ bei $a\in G$ differenzierbar,
so gilt:%
\begin{equation}
(D_v f)(a) = (\langle v,\nabla\rangle f)(a) = J[f](a)\,v,
\end{equation}
kurz $D_v = \langle v,\nabla\rangle$.

\section{Variationsrechnung}\index{Variationsrechnung}

\subsection{Fundamentallemma}\index{Fundamentallemma}

Sei $I:=[a,b]$ kompakt und sei $g\colon I\to\R$
stetig. Wenn 
\begin{equation}
\int_a^b g(x)h(x)\,\mathrm dx=0
\end{equation}
für jede unendlich oft differenzierbare Funktion $h\colon I\to\R$
mit $h(a)=h(b)=0$ gilt, so ist $g(x)=0$ für alle $x$.

\subsection{Euler-Lagrange-Gleichung}

Sei $I:=[a,b]$ kompakt. Sei
\begin{equation}
F\colon I\times\R\times\R\to\R
\end{equation}
zweimal stetig differenzierbar. Gesucht ist eine zweimal
stetig differenzierbare Funktion $f\colon I\to\R$ mit fixen
Randwerten $f(a)=A$ und $f(b)=B$, für die
\begin{equation}
J(f) := \int_a^b F(x,f(x),f'(x))\,\mathrm dx
\end{equation}
einen Extremwert annimmt.

Die Euler-Lagrange-Gleichung\index{Euler-Lagrange-Gleichung}
\begin{equation}
\frac{\partial F(x,y,y')}{\partial y}
-\frac{\mathrm d}{\mathrm dx}\frac{\partial F(x,y,y')}{\partial y'}
=0
\end{equation}
mit $y=f(x)$ und $y'=f'(x)$ ist eine notwendige Bedingung dafür.

\newpage
\section{Fourier-Analysis}
\subsection{Fourierreihen}\index{Fourierreihe}
\subsubsection{Fourier-Koeffizienten}\index{Fourier-Koeffizient}
\strong{Komplexe Fourier-Koeffizienten:}
\begin{equation}
c_k[s] = \frac{1}{T}\int_{t_0}^{t_0+T} \ee^{-k\ui\omega t}s(t)\,\mathrm dt.
\end{equation}
Nach Normierung $x:=\omega t$, $f(x):=s(x/\omega)$:
\begin{equation}
c_k[f] = \frac{1}{2\pi}\int_{-\pi}^{\pi} \ee^{-k\ui x}f(x)\,\mathrm dx.
\end{equation}
Es gilt ($\lambda$: eine Konstante):
\begin{gather}
c_k[f+g] = c_k[f]+c_k[g],\\
c_k[\lambda f] = \lambda c_k[f].
\end{gather}
\strong{Reelle Fourier-Koeffizienten:}
\begin{align}
a_k[s] &= \frac{2}{T}\int_{t_0}^{t_0+T} \cos(k\omega t)\,s(t)\,\mathrm dt,\\
b_k[s] &= \frac{2}{T}\int_{t_0}^{t_0+T} \sin(k\omega t)\,s(t)\,\mathrm dt.
\end{align}
Nach Normierung $x:=\omega t$, $f(x):=s(x/\omega)$:
\begin{align}
a_k[f] &= \frac{1}{\pi}\int_{-\pi}^{\pi} \cos(kx)\,f(x)\,\mathrm dx,\\
b_k[f] &= \frac{1}{\pi}\int_{-\pi}^{\pi} \sin(kx)\,f(x)\,\mathrm dx.
\end{align}

