
\chapter{Grundlagen}
\section{Arithmetik}
\subsection{Binomischer Lehrsatz}\index{binomischer Lehrsatz}
Sei $R$ ein unitärer Ring. 
Für $a,b\in R$ mit $ab=ba$ gilt:%
\begin{equation}
(a+b)^n = \sum_{k=0}^n \binom{n}{k} a^{n-k} b^k
\end{equation}
und
\begin{equation}
(a-b)^n = \sum_{k=0}^n \binom{n}{k} (-1)^k a^{n-k} b^k.
\end{equation}
Die ersten Formeln sind:\index{binomische Formeln}
\begin{gather}
(a+b)^2 = a^2+2ab+b^2,\\
(a-b)^2 = a^2-2ab+b^2,\\
(a+b)^3 = a^3+3a^2 b+3ab^2+b^3,\\
(a-b)^3 = a^3-3a^2 b+3ab^2-b^3,\\
(a+b)^4 = a^4+4a^3 b+6a^2 b^2+4ab^3+b^4,\\
(a-b)^4 = a^4-4a^3 b+6a^2 b^2-4ab^3+b^4.
\end{gather}
\subsection{Potenzgesetze}
\begin{Definition}
Für $a\in\R, a>0$ und $x\in\C$:
\begin{equation}
a^x := \exp(\ln(a)\,x).
\end{equation}
\end{Definition}
\noindent
Für $a\in\R, a>0$ und $x,y\in\C$ gilt:
\begin{gather}
a^{x+y} = a^x a^y,\quad a^{x-y} = \frac{a^x}{a^y},
\quad a^{-x} = \frac{1}{a^x}.
\end{gather}

\section{Komplexe Zahlen}\index{komplexe Zahl}
\subsection{Rechenoperationen}

\begin{gather}
\frac{z_1}{z_2}
= \frac{z_1\overline z_2}{z_2\overline z_2}
= \frac{z_1\overline z_2}{|z_2|^2},\\
\frac{1}{z} = \frac{\overline z}{z\overline z}
= \frac{\overline z}{|z|^2}.
\end{gather}

\subsection{Betrag}\index{Betrag!einer komplexen Zahl}
Für alle $z_1,z_2\in\C$ gilt:
\begin{gather}
|z_1z_2| = |z_1|\,|z_2|,\\
z_2\ne 0\implies \Big|\frac{z_1}{z_2}\Big|
= \frac{|z_1|}{|z_2|},\\
z\,\overline z = |z|^2.
\end{gather}

\subsection{Konjugation}\index{Konjugation!einer komplexen Zahl}
Für alle $z_1,z_2\in\C$ gilt:
\begin{gather}
\overline{z_1+z_2} = \overline z_1+\overline z_2,\qquad
\overline{z_1-z_2} = \overline z_1-\overline z_2,\\
\overline{z_1 z_2} = \overline z_1\,\overline z_2,\qquad
z_2\ne 0 \implies \overline{\Big(\frac{z_1}{z_2}\Big)}
= \frac{\overline z_1}{\overline z_2},\\
\overline{\overline z}=z,\qquad
|\overline{z}| = |z|,\qquad
z\,\overline z = |z|^2,\\
\Re(z) = \frac{z+\overline z}{2},\qquad
\Im(z) = \frac{z-\overline z}{2\ui},\\
\overline{\cos(z)} = \cos(\overline z),\qquad
\overline{\sin(z)} = \sin(\overline z),\\
\overline{\exp(z)} = \exp(\overline z).
\end{gather}

\begin{table*}[t]
\caption{Rechenoperationen}
\bgroup
\def\arraystretch{1.4}
\begin{tabular}{|l|r|l|l|}
\hline
  \thbf{Name}
& \thbf{Operation}
& \thbf{Polarform}
& \thbf{kartesische Form}\\
\hline
  Identität
& $z$ & $=r\ee^{\ui\varphi}$
& $= a+b\ui$\\
\hline
  Addition
& $z_1+z_2$ &
& $= (a_1+a_2)+(b_1+b_2)\ui$\\
\hline
  Subtraktion
& $z_1-z_2$ &
& $= (a_1-a_2)+(b_1-b_2)\ui$\\
\hline
  Multiplikation
& $z_1 z_2$
& $= r_1 r_2 \ee^{\ui(\varphi_1+\varphi_2)}$
& $= (a_1 a_2 - b_1 b_2)+(a_1 b_2+a_2 b_1)\ui$\\
\hline
  Division
& $\displaystyle\frac{z_1}{z_2}$
& $\displaystyle =\frac{r_1}{r_2}\ee^{\ui(\varphi_1-\varphi_2)}$
& $\displaystyle =\frac{a_1 a_2 + b_1 b_2}{a_2^2+b_2^2}
   + \frac{a_2 b_1 - a_1 b_2}{a_2^2+b_2^2}\ui$\\
\hline
  Kehrwert
& $\displaystyle\frac{1}{z}$
& $\displaystyle =\frac{1}{r}\ee^{-\ui\varphi}$
& $\displaystyle =\frac{a}{a^2+b^2}-\frac{b}{a^2+b^2}\ui$\\
\hline
  Realteil
& $\Re(z)$
& $=\cos\varphi$
& $=a$\\
\hline
  Imaginärteil
& $\Im(z)$
& $=\sin\varphi$
& $=b$\\
\hline
  Konjugation
& $\overline{z}$
& $=r\ee^{-\varphi\ui}$
& $=a-b\ui$\\
\hline
Betrag
& $|z|$
& $=r$
& $=\sqrt{a^2+b^2}$\\
\hline
  Argument
& $\arg(z)$
& $=\varphi$
& $\displaystyle = s(b)\arccos\Big(\frac{a}{r}\Big)$\\
\hline
\end{tabular}
\egroup\\
\\
$s(b):=\begin{cases}
+1 & \text{if}\;b\ge 0,\\
-1 & \text{if}\;b<0
\end{cases}$
\end{table*}



\section{Logik}
\subsection{Aussagenlogik}\index{Aussagenlogik}
\subsubsection{Boolesche Algebra}\index{boolesche Algebra}
\begin{table*}[t]
\caption{Boolesche Algebra}
\begin{tabular}{l|l|l}
\thbf{Disjunktion} & \thbf{Konjunktion} &\\
  $A\lor A \Leftrightarrow A$
& $A\land A \Leftrightarrow A$
& Idempotenzgesetze\\
  $A\lor 0 \Leftrightarrow A$
& $A\land 1 \Leftrightarrow A$
& Neutralitätsgesetze\\
  $A\lor 1 \Leftrightarrow 1$
& $A\land 0 = 0$
& Extremalgesetze\\
  $A\lor \overline A \Leftrightarrow 1$
& $A\land \overline A \Leftrightarrow 0$
& Komplementärgesetze\\
\noalign{\vspace{1em}}
  $A\lor B \Leftrightarrow B\lor A$
& $A\land B \Leftrightarrow B\land A$
& Kommutativgesetze\\
  $(A\lor B)\lor C \Leftrightarrow A\lor (B\lor C)$
& $(A\land B)\land C \Leftrightarrow A\land (B\land C)$
& Assoziativgesetze\\
  $\overline{A\lor B} \Leftrightarrow \overline A\land\overline B$
& $\overline{A\land B} \Leftrightarrow \overline A\lor\overline B$
& De Morgansche Regeln\\
  $A\lor (A\land B) \Leftrightarrow A$
& $A\land (A\lor B) \Leftrightarrow A$
& Absorptionsgesetze\\
\end{tabular}
\end{table*}

\noindent
\strong{Distributivgesetze}:
\begin{gather}
A\lor (B\land C) = (A\lor B)\land (A\lor C),\\
A\land (B\lor C) = (A\land B)\lor (A\land C).
\end{gather}

\subsubsection{Zweistellige Funktionen}
Es gibt 16 zweistellige boolesche\\
Funktionen.

\begin{tabular}{r|l}
\textbf{\texttt{AB}} & \thbf{Wert}\\
\texttt{00} & \texttt{a}\\
\texttt{01} & \texttt{b}\\
\texttt{10} & \texttt{c}\\
\texttt{11} & \texttt{d}
\end{tabular}

\begin{tabular}{r|l|l|l}
\thbf{Nr.}& \textbf{\texttt{dcba}} & \thbf{Fkt.} & \thbf{Name}\\
 0 & \texttt{0000} & 0 & Kontradiktion\\
 1 & \texttt{0001} & $\overline{A\lor B}$ & NOR\\
 2 & \texttt{0010} & $\overline{B\Rightarrow A}$\\
 3 & \texttt{0011} & $\overline A$\\
 4 & \texttt{0100} & $\overline{A\Rightarrow B}$\\
 5 & \texttt{0101} & $\overline{B}$\\
 6 & \texttt{0110} & $A\oplus B$ & Kontravalenz\\
 7 & \texttt{0111} & $\overline{A\land B}$ & NAND\\
 8 & \texttt{1000} & $A\land B$ & Konjunktion\\
 9 & \texttt{1001} & $A\Leftrightarrow B$ & Äquivalenz\\
10 & \texttt{1010} & $B$ & Projektion\\
11 & \texttt{1011} & $A\Rightarrow B$ & Implikation\\
12 & \texttt{1100} & $A$ & Projektion\\
13 & \texttt{1101} & $B\Rightarrow A$ & Implikation\\
14 & \texttt{1110} & $A\lor B$ & Disjunktion\\
15 & \texttt{1111} & $1$ & Tautologie
\end{tabular}

\subsubsection{Darstellung mit Negation, Konjunktion und Disjunktion}
\begin{gather}
A\Rightarrow B \iff \overline A\lor B,\\
(A\Leftrightarrow B) \iff
  (\overline A\land\overline B)\lor(A\land B),\\
A\oplus B \iff (\overline A\land B)\lor(A\land\overline B).
\end{gather}

\subsubsection{Tautologien}
Modus ponens:
\begin{equation}
(A\Rightarrow B)\land A\implies B.
\end{equation}
Modus tollens:
\begin{equation}
(A\Rightarrow B)\land\overline B\implies\overline A.
\end{equation}
Modus tollendo ponens:
\begin{equation}
(A\lor B)\land\overline A \implies B.
\end{equation}
Modus ponendo tollens:
\begin{equation}
\overline{A\land B}\land A\implies\overline B.
\end{equation}
Kontraposition:
\begin{equation}
A\Rightarrow B \iff \overline B\Rightarrow \overline A.
\end{equation}
Beweis durch Widerspruch:
\begin{equation}
(\overline A\Rightarrow B\land\overline B)\implies A.
\end{equation}
Zerlegung einer Äquivalenz:
\begin{equation}
(A\Leftrightarrow B) \iff (A\Rightarrow B)\land(B\Rightarrow A).
\end{equation}
Kettenschluss:
\begin{equation}
(A\Rightarrow B)\land(B\Rightarrow C)\implies (A\Rightarrow C).
\end{equation}
Ringschluss:
\begin{equation}
\begin{split}
&(A\Rightarrow B)\land (B\Rightarrow C)\land(C\Rightarrow A)\\
&\implies (A\Leftrightarrow B)\land(A\Leftrightarrow C)\land(B\Leftrightarrow C).
\end{split}
\end{equation}
Ringschluss, allgemein:
\begin{equation}
\begin{split}
& (A_1{\Rightarrow }A_2)\land\ldots\land(A_{n-1}{\Rightarrow}A_n)
\land(A_n{\Rightarrow}A_1)\\
& \implies \forall i,j\,[A_i\Leftrightarrow A_j].
\end{split}
\end{equation}
Ersetzungsregel:

Für jede Funktion $P\colon\{0,1\}\to\{0,1\}$ gilt:
\begin{equation}
P(A)\land (A\Leftrightarrow B)\implies P(B).
\end{equation}
Regel zur Implikation:
\begin{equation}
A\land B\Rightarrow C \iff A\Rightarrow (B\Rightarrow C).
\end{equation}
Vollständige Fallunterscheidung:
\begin{gather}
(A\Rightarrow C)\land (B\Rightarrow C)\implies (A\oplus B\Rightarrow C),\\
(A\Rightarrow C)\land (B\Rightarrow C)\iff (A\lor B\Rightarrow C).
\end{gather}
Vollständige Fallunterscheidung, allgemein:
\begin{gather}
\textstyle \forall k[A_k\Rightarrow C]
\implies (\bigoplus_{k=1}^n A_k\Rightarrow C),\\
\forall k[A_k\Rightarrow C]
\iff (\exists k[A_k]\Rightarrow C).
\end{gather}

\newpage
\subsection{Prädikatenlogik}
\subsubsection{Rechenregeln}
Verneinung (De Morgansche Regeln):
\begin{gather}
\overline{\forall x[P(x)]}\iff \exists x[\overline{P(x)}],\\
\overline{\exists x[P(x)]}\iff \forall x[\overline{P(x)}].
\end{gather}
Verallgemeinerte Distributivgesetze:
\begin{gather}
P\lor\forall x[Q(x)] \iff \forall x[P\lor Q(x)],\\
P\land\exists x[Q(x)] \iff \exists x[P\land Q(x)].
\end{gather}
Verallgemeinerte Idempotenzgesetze:
\begin{gather}
\begin{split}
\exists x{\in}M\,[P] & \iff
(M\ne\{\})\land P\\
& \iff\begin{cases}
P & \text{wenn}\; M\ne\{\},\\
0 & \text{wenn}\; M=\{\}.
\end{cases}
\end{split}\\
\begin{split}
\forall x{\in}M\,[P]& \iff
(M=\{\})\lor P\\
&\iff\begin{cases}
P & \text{wenn}\; M\ne\{\},\\
1 & \text{wenn}\; M=\{\}.
\end{cases}
\end{split}
\end{gather}
Äquivalenzen:
\begin{gather}
\hspace{-2em}\forall x\forall y[P(x,y)] \iff \forall y\forall x[P(x,y)],\\
\hspace{-2em}\exists x\exists y[P(x,y)] \iff \exists y\exists x[P(x,y)],\\
\hspace{-2em}\forall x[P(x)\land Q(x)] \iff \forall x[P(x)]\land\forall x[Q(x)],\\
\hspace{-2em}\exists x[P(x)\lor Q(x)] \iff \exists x[P(x)]\lor\exists x[Q(x)],\\
\hspace{-2em}\forall x[P(x)\Rightarrow Q] \iff \exists x[P(x)]\Rightarrow Q,\\
\hspace{-2em}\forall x[P\Rightarrow Q(x)] \iff P\Rightarrow\forall x[Q(x)],\\
\hspace{-2em}\exists x[P(x)\Rightarrow Q(x)]
  \iff\forall x[P(x)]\Rightarrow\exists x[Q(x)].
\end{gather}
Implikationen:
\begin{gather}
\hspace{-2em}\exists x\forall y[P(x,y)]\implies \forall y\exists x[P(x,y)],\\
\hspace{-2em}\forall x[P(x)]\lor\forall x[Q(x)]\implies\forall x[P(x)\lor Q(x)],\\
\hspace{-2em}\exists x[P(x)\land Q(x)]\implies
  \exists x[P(x)]\land \exists x[Q(x)],\\
\hspace{-2em}\forall x[P(x)\Rightarrow Q(x)]\implies
  (\forall x[P(x)]\Rightarrow\forall x[Q(x)]),\\
\hspace{-2em}\forall x[P(x)\Leftrightarrow Q(x)]\implies
  (\forall x[P(x)]\Leftrightarrow\forall x[Q(x)]).
\end{gather}

\subsubsection{Endliche Mengen}
Sei $M=\{x_1,\ldots,x_n\}$. Es gilt:
\begin{gather}
\forall x{\in}M\,[P(x)]\iff P(x_1)\land\ldots\land P(x_n),\\
\exists x{\in}M\,[P(x)]\iff P(x_1)\lor\ldots\lor P(x_n).
\end{gather}

\subsubsection{Beschränkte Quantifizierung}
\begin{gather}
\begin{split}
& \forall x{\in}M\,[P(x)]\;:\Longleftrightarrow\;\forall x[x\notin M\lor P(x)]\\
& \quad\iff\forall x[x\in M\Rightarrow P(x)],
\end{split}\\
\exists x{\in}M\,[P(x)]\;:\Longleftrightarrow\;\exists x[x\in M\land P(x)],\\
\forall x{\in}M{\setminus}N\,[P(x)]\iff \forall x[x\notin N\Rightarrow P(x)].
\end{gather}

\subsubsection{Quantifizierung über Produktmengen}
\begin{gather}
\forall(x,y)\,[P(x,y)]\iff \forall x\forall y[P(x,y)],\\
\exists(x,y)\,[P(x,y)]\iff \exists x\exists y[P(x,y)].
\end{gather}
Analog gilt
\begin{gather}
\forall(x,y,z)\,\iff \forall x\forall y\forall z,\\
\exists(x,y,z)\,\iff \exists x\exists y\exists z
\end{gather}
usw.

\subsubsection{Alternative Darstellung}
Sei $P\colon G\to\{0,1\}$ und $M\subseteq G$.
Mit $P(M)$ ist die Bildmenge von $P$ bezüglich $M$ gemeint.
Es gilt
\begin{equation}
\begin{split}
&\forall x{\in}M\,[P(x)] \iff P(M)=\{1\}\\
& \iff M\subseteq\{x{\in}G\mid P(x)\}
\end{split}
\end{equation}
und
\begin{equation}
\begin{split}
& \exists x{\in}M\,[P(x)] \iff \{1\}\subseteq P(M)\\
& \iff M\cap\{x{\in}G\mid P(x)\}\ne\{\}.
\end{split}
\end{equation}

\subsubsection{Eindeutigkeit}
Quantor für eindeutige Existenz:
\begin{equation}
\begin{split}
&\exists!x\,[P(x)]\\
&:\Longleftrightarrow\; \exists x\,[P(x)\land \forall y\,[P(y)\Rightarrow x=y]]\\
&\iff \exists x\,[P(x)]\land \forall x\forall y[P(x)\land P(y)\Rightarrow x=y].
\end{split}
\end{equation}



\section{Mengenlehre}
\subsection{Definitionen}
Teilmengenrelation:
\begin{equation}
A\subseteq B\;:\Longleftrightarrow\; \forall x\,[x\in A\implies x\in B].
\end{equation}
Gleichheit:
\begin{equation}
A=B\;:\Longleftrightarrow\; \forall x\,[x\in A\iff x\in B].
\end{equation}
Vereinigungsmenge:
\begin{equation}
A\cup B:=\{x\mid x\in A\lor x\in B\}.
\end{equation}
Schnittmenge:
\begin{equation}
A\cap B:=\{x\mid x\in A\land x\in B\}.
\end{equation}
Differenzmenge:
\begin{equation}
A\setminus B:=\{x\mid x\in A\land x\not\in B\}.
\end{equation}
Symmetrische Differenz:
\begin{equation}
A\triangle B:=\{x\mid x\in A\oplus x\in B\}.
\end{equation}

\subsection{Boolesche Algebra}
\begin{table*}[t]
\caption{Boolesche Algebra}
\begin{tabular}{l|l|l}
\thbf{Vereinigung} & \thbf{Schnitt} &\\
  $A\cup A = A$
& $A\cap A = A$
& Idempotenzgesetze\\
  $A\cup \{\} = A$
& $A\cap G = A$
& Neutralitätsgesetze\\
  $A\cup G = G$
& $A\cap \{\} = \{\}$
& Extremalgesetze\\
  $A\cup \overline A = G$
& $A\cap \overline A = \{\}$
& Komplementärgesetze\\
\noalign{\vspace{1em}}
  $A\cup B = B\cup A$
& $A\cap B = B\cap A$
& Kommutativgesetze\\
  $(A\cup B)\cup C = A\cup (B\cup C)$
& $(A\cap B)\cap C = A\cap (B\cap C)$
& Assoziativgesetze\\
  $\overline{A\cup B} = \overline A\cap\overline B$
& $\overline{A\cap B} = \overline A\cup\overline B$
& De Morgansche Regeln\\
  $A\cup (A\cap B) = A$
& $A\cap (A\cup B) = A$
& Absorptionsgesetze\\
\end{tabular}\\
\\
$G$: Grundmenge
\end{table*}

\noindent
\strong{Distributivgesetze}:
\begin{gather}
M\cup (A\cap B) = (M\cup A)\cap (M\cup B),\\
M\cap (A\cup B) = (M\cap A)\cup (M\cap B).
\end{gather}

\subsection{Teilmengenrelation}
Zerlegung der Gleichheit:
\begin{equation}
A=B \iff A\subseteq B \land B\subseteq A.
\end{equation}
Umschreibung der Teilmengenrelation:
\begin{equation}
\begin{split}
A\subseteq B &\iff A\cap B=A\\
& \iff A\cup B=B\\
& \iff A\setminus B=\{\}.
\end{split}
\end{equation}
Kontraposition:
\begin{equation}
A\subseteq B = \overline B\subseteq \overline A.
\end{equation}

\subsection{Induktive Mengen}
Mengentheoretisches Modell der natürlichen Zahlen:
\begin{equation}
\begin{split}
& 0:=\{\},\quad 1:=\{0\},\quad 2:=\{0,1\},\\
& 3:=\{0,1,2\},\quad \text{usw.}
\end{split}
\end{equation}
Nachfolgerfunktion:
\begin{equation}
x' := x\cup\{x\}.
\end{equation}
Vollständige Induktion: Ist $A(n)$ mit $n\in\N$
eine Aussageform, so gilt:
\begin{equation}
\begin{split}
& A(n_0)\land \forall n\ge n_0\,[A(n)\Rightarrow A(n+1)]\\
& \implies \forall n\ge n_0\,[A(n)].
\end{split}
\end{equation}

\section{Funktionen}
\subsection{Surjektionen}\index{surjektiv}
\begin{Definition}
Eine Funktion $f\colon A\to B$ heißt \emdef{surjektiv},\\
wenn $f(A)=B$ ist. Damit ist gemeint, dass jedes Element
der Zielmenge wenigstens einmal der Funktionswert von einem
Element der Definitionsmenge ist.
\end{Definition}

\subsection{Injektionen}\index{injektiv}
\begin{Definition}
Eine Funktion $f\colon A\to B$ heißt \emdef{injektiv},\\
wenn
\begin{equation}
\forall x_1,x_2\in A\,[f(x_1)=f(x_2)\implies x_1=x_2]
\end{equation}
gilt.
\end{Definition}

\subsection{Bijektionen}\index{bijektiv}
\begin{Definition}
Eine Funktion $f\colon A\to B$ heißt \emdef{bijektiv},
wenn sie injektiv und surjektiv ist.

Eine Funktion $f\colon A\to B$ ist genau dann bijektiv, wenn es
ein $g$ mit
\begin{equation}
g\circ f = \id_A\quad\text{und}\quad f\circ g = \id_B
\end{equation}
gibt. Wenn $f$ bijektiv ist, so gibt es $g$ genau einmal und
$g$ wird die \emph{Umkehrfunktion}\index{Umkehrfunktion}
oder \emph{Inverse}
von $f$ genannt und als $f^{-1}$ notiert.
\end{Definition}

\subsection{Komposition}\index{Komposition}
\begin{Definition} Für zwei Funktionen $f\colon A\to B$
und $g\colon B\to C$ ist die \emdef{Komposition}
($g$ nach $f$)
durch
\begin{equation}
g\circ f\colon A\to C,\quad (g\circ f)(x) := g(f(x))
\end{equation}
definiert.
\end{Definition}
Für die Komposition gilt das Assozativgesetz:
\begin{equation}
(f\circ g)\circ h = f\circ(g\circ h).
\end{equation}

Die Komposition von Injektionen ist eine Injektion.

Die Komposition von Surjektionen ist eine Surjektion.

Die Komposition von Bijektionen ist eine Bijektion.

Sind $f,g$ Bijektionen, so gilt
\begin{equation}
(g\circ f)^{-1} = f^{-1}\circ g^{-1}.
\end{equation}

Ist $g\circ f$ injektiv, so ist $f$ injektiv.

Ist $g\circ f$ surjektiv, so ist $g$ surjektiv.

Ist $g\circ f$ bijektiv, so ist $f$ injektiv und $g$ surjektiv.

\begin{Definition}
Für eine Funktion $\varphi\colon A\to A$ wird
\begin{equation}
\varphi^0:=\operatorname{id}_A,\quad \varphi^{n+1}:=\varphi^n\circ\varphi
\end{equation}
\emdef{Iteration}\index{Iteration} von $\varphi$ genannt.
\end{Definition}
\begin{Definition}
Für eine Funktion $\varphi\colon A\to A$ wird der Operator
\begin{equation}
C_\varphi (g) := g\circ\varphi,\quad C_\varphi\colon B^A\to B^A
\end{equation}
\emdef{Kompositionsoperator}\index{Kompositionsoperator} genannt
\end{Definition}
Ist $B^A$ ein Funktionenraum, so ist der Kompositionsoperator
ein linearer Operator.

\subsection{Einschränkung}\index{Einschränkung}
\begin{Definition} Sei $f\colon A\to B$ und $M\subseteq A$.
Die Funktion $g(x)=f(x)$ mit $g\colon M\to B$ wird \emdef{Einschränkung}
von $f$ genannt und mit $f|_M$ notiert.
\end{Definition}
Sei $f\colon A\to B$ und $M\subseteq A$.
Mit der Inklusionsabbildung $i(x):=x$ mit $i\colon M\to A$ gilt:
\begin{equation}
f|_M = f\circ i.
\end{equation}
Es gilt
\begin{equation}
g\circ (f|_M) = (g\circ f)|_M.
\end{equation}

\subsection{Bild}\index{Bild}
\begin{Definition} Ist $f\colon A\to B$ und $M\subseteq A$, so wird
\begin{equation}
f(M) := \{f(x)\mid x\in M\}
\end{equation}
das \emdef{Bild} von $M$ unter $f$ genannt.
\end{Definition}
Es gilt
\begin{align}
&f(M\cup N) = f(M)\cup f(N),\\
&f(M\cap N) = f(M)\cap f(N),\\
&f\Big(\bigcup_{i\in I}M_i\Big) = \bigcup_{i\in I} f(M_i),\\
&I\ne\emptyset\implies f\Big(\bigcap_{i\in I} M_i\Big) = \bigcap_{i\in I} f(M_i),\\
&M\subseteq N\implies f(M)\subseteq f(N),\\
&f(\emptyset) = \emptyset,\\
&(g\circ f)(M) = g(f(M)).
\end{align}

\subsection{Urbild}\index{Urbild}
\begin{Definition}
Ist $f\colon A\to B$, so wird
\begin{equation}
f^{-1}(M) := \{x\in A\mid f(x)\in M\}.
\end{equation}
das \emdef{Urbild} von $M$ unter $f$ genannt.
\end{Definition}
Es gilt
\begin{align}
& f^{-1}(M\cup N) = f^{-1}(M)\cup f^{-1}(N),\\
& f^{-1}(M\cap N) = f^{-1}(M)\cap f^{-1}(N),\\
& f^{-1}\Big(\bigcup_{i\in I}M_i\Big) = \bigcup_{i\in I} f^{-1}(M_i),\\
& I\ne\emptyset\implies f^{-1}\Big(\bigcap_{i\in I} M_i\Big) = \bigcap_{i\in I}f^{-1}(M_i),\\
& M\subseteq N\implies f^{-1}(M)\subseteq f^{-1}(N),\\
& f^{-1}(\emptyset) = \emptyset,\\
& f^{-1}(B) = A,\\
& f^{-1}(M\setminus N) = f^{-1}(M)\setminus f^{-1}(N),\\
& f^{-1}(B\setminus M) = B\setminus f^{-1}(M),\\
& (g\circ f)^{-1}(M) = f^{-1}(g^{-1}(M)),\\
& (f|_M)^{-1}(N) = M\cap f^{-1}(N).
\end{align}

\section{Mathematische Strukturen}
\strong{Axiome:}

\bsf{E}: Abgeschlossenheit.

\bsf{A}: Assoziativgesetz.

\bsf{N}: Existenz des neutralen Elements.

\bsf{I}: Zu jedem Element gibt es ein Inverses.

\bsf{K}: Kommutativgesetz.

\bsf{I*}: Zu jedem Element außer dem additiven neutralen Element
gibt es ein Inverses.

\bsf{Dl}: Linksdistributivgestz.

\bsf{Dr}: Rechtsdistributivgesetz.

\bsf{D}: Dl und Dr.

\bsf{T}: Nullteilerfreiheit

\bsf{U}: Die neutralen Elemente bezüglich Addition und
  Multiplikation sind unterschiedlich.\\

\noindent
Strukturen mit einer inneren Verknüpfung:\\
\begin{tabular}{l|l}
\bsf{EA} & Halbgruppe\\
\bsf{EAN} & Monoid\\
\bsf{EANI} & Gruppe\\
\bsf{EANIK} & abelsche Gruppe
\end{tabular}

\noindent
Strukturen mit zwei inneren Verknüpfungen:\\
\begin{tabular}{l|l}
\bsf{EANIK, EA, D}\dotfill & Ring\\
\bsf{EANIK, EAK, D}\dotfill & kommutativer Ring\\
\bsf{EANIK, EAN, D}\dotfill & unitärer Ring\\
\bsf{EANIK, EANK, DTU} & Integritätsring\\
\bsf{EANIK, EANI*K, DTU} & Körper
\end{tabular}

