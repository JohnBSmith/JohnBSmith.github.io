
\chapter{Grundlagen}
\section{Arithmetik}
\subsection{Zahlenbereiche}
Natürliche Zahlen ab null:
\begin{equation}
\N_0 := \{0,1,2,3,4,\ldots\}.
\end{equation}
Natürliche Zahlen ab eins:
\begin{equation}
\N := \{1,2,3,4,5,\ldots\}.
\end{equation}
Ganze Zahlen:
\begin{equation}
\Z := \{\ldots,-3,-2,-1,0,1,2,3,\ldots\}.
\end{equation}
Rationale Zahlen:
\begin{equation}
\Q := \{\tfrac{z}{n}\mid z\in\Z,n\in\N\}.
\end{equation}
Reelle Zahlen:
\begin{equation}
\begin{split}
&\R := \text{Vervollständigung von }\Q.
\end{split}
\end{equation}
Positive reelle Zahlen:
\begin{equation}
\R^+ := \{x\in\R\mid x>0\}.
\end{equation}
Nichtnegative reelle Zahlen:
\begin{equation}
\R_0^+ := \{x\in\R\mid x\ge 0\}.
\end{equation}
Negative reelle Zahlen:
\begin{equation}
\R^- := \{x\in\R\mid x<0\}.
\end{equation}
Nichtpositive reelle Zahlen:
\begin{equation}
\R_0^- := \{x\in\R\mid x\le 0\}.
\end{equation}
Komplexe Zahlen:
\begin{equation}
\C := \{a+b\ui\mid a,b\in\R\}.
\end{equation}
Quaternionen:
\begin{equation}
\mathbb H := \{a+b\ui+c\uj+d\uk\mid a,b,c,d\in\R\}.
\end{equation}
Algebraische Zahlen:
\begin{equation}
\mathbb A := \{a\in\C\mid \exists P{\in}\Q[X]\colon P(a)=0\}.
\end{equation}
Irrationale Zahlen:
\begin{equation}
\R\setminus\Q = \{\sqrt{2},\sqrt{3},\pi,\ee,\ldots\}.
\end{equation}
Transzendente Zahlen:
\begin{equation}
\R\setminus\mathbb A = \{\pi,\ee,\ldots\}.
\end{equation}
Es gelten die folgenden Teilmengenbeziehungen:
\begin{equation}
\N\subset\Z\subset\Q\subset\R\subset\C\subset\mathbb H.
\end{equation}
Es gilt die folgende Abstufung der Mächtigkeit:
\begin{equation}
|\N| = |\Z| = |\Q| = |\mathbb A| < |\R| = |\C|.
\end{equation}

\newpage
\subsection{Intervalle}
Abgeschlossene Intervalle:
\begin{equation}
[a,b] := \{x\in\R\mid a\le x\le b\}.
\end{equation}
Offene Intervalle:
\begin{equation}
(a,b) := \{x\in\R\mid a<x<b\}.
\end{equation}
Halboffene Intervalle:
\begin{align}
(a,b] &:= \{x\in\R\mid a<x\le b\},\\
[a,b) &:= \{x\in\R\mid a\le x<b\}.
\end{align}
Unbeschränkte Intervalle:
\begin{align}
[a,\infty) &:= \{x\in\R\mid a\le x\},\\
(a,\infty) &:= \{x\in\R\mid a<x\},\\
(-\infty,b] &:= \{x\in\R\mid x\le b\},\\
(-\infty,b) &:= \{x\in\R\mid x<b\}.
\end{align}

\subsection{Summen}
\begin{definition}[Summe]\mbox{}\newline
Für eine Folge $(a_n)$:
\begin{align}
\sum_{k=m}^{m-1} a_k &:= 0,\qquad(\text{leere Summe})\\
\sum_{k=m}^n a_k &:= a_n+\sum_{k=m}^{n-1} a_k.\qquad(n\ge m)
\end{align}
\end{definition}
\noindent
Für eine Konstante $c$ gilt:
\begin{equation}
\sum_{k=m}^n c = (n-m+1)\,c.
\end{equation}
Der Summierungsoperator ist linear:
\begin{align}
\sum_{k=m}^n (a_k+b_k) &= \sum_{k=m}^n a_k + \sum_{k=m}^n b_k,\\
\sum_{k=m}^n ca_k &= c\sum_{k=m}^n a_k.
\end{align}
Indexverschiebung ist möglich:
\begin{equation}
\sum_{k=m}^n a_k = \sum_{k=m-j}^{n-j} a_{k+j} = \sum_{k=m+j}^{n+j} a_{k-j}.
\end{equation}
Aufspaltung ist möglich:
\begin{equation}
\sum_{k=m}^n a_k = \sum_{k=m}^p a_k + \sum_{k=p+1}^n a_k.
\end{equation}
Vertauschung der Reihenfolge bei Doppelsummen:
\begin{equation}
\sum_{i=p}^m \sum_{j=q}^n a_{ij} = \sum_{j=q}^n \sum_{i=p}^m a_{ij}.
\end{equation}

\subsection{Produkte}
\begin{definition}[Produkt]\mbox{}\newline
Für eine Folge $(a_n)$:
\begin{align}
\prod_{k=m}^{m-1} a_k &:= 1,\qquad(\text{leeres Produkt})\\
\prod_{k=m}^n a_k &:= a_n\prod_{k=m}^{n-1} a_k.\qquad(n\ge m)
\end{align}
\end{definition}

\noindent
Für eine Konstante $c$ gilt:
\begin{equation}
\prod_{k=m}^n c = c^{n-m+1}.
\end{equation}
Unter Voraussetzung des Kommutativgesetzes gilt
\begin{align}
\label{eq:product-product}
\prod_{k=m}^n (a_k b_k) &= \bigg(\prod_{k=m}^n a_k\bigg)\bigg(\prod_{k=m}^n b_k\bigg),\\
\label{eq:product-power}
\prod_{k=m}^n a_k^c &= \bigg(\prod_{k=m}^n a_k\bigg)\Big.^c.\qquad (c\in\N_0)
\end{align}

Formel \eqref{eq:product-power} gilt auch für $a_k\in\R^+$ und $c\in\C$.

Formel \eqref{eq:product-product} ist ein Spezialfall von
\begin{equation}
\prod_{i=p}^m \prod_{j=q}^n a_{ij} = \prod_{j=q}^n \prod_{i=p}^m a_{ij}.
\end{equation}
Indexverschiebung ist möglich:
\begin{equation}
\prod_{k=m}^n a_k = \prod_{k=m-j}^{n-j} a_{k+j} = \prod_{k=m+j}^{n+j} a_{k-j}.
\end{equation}
Aufspaltung ist möglich:
\begin{equation}
\prod_{k=m}^n a_k = \bigg(\prod_{k=m}^p a_k\bigg)\bigg(\prod_{k=p+1}^n a_k\bigg).
\end{equation}
Für $a_k\in\R^+$ gilt
\begin{equation}
\prod_{k=m}^n a_k = \exp\bigg(\sum_{k=m}^n \ln(a_k)\bigg).
\end{equation}

\subsection{Binomischer Lehrsatz}\index{binomischer Lehrsatz}
Sei $R$ ein unitärer Ring, z.\,B. $R=\R$ oder $R=\C$.\\
Für $a,b\in R$ mit $ab=ba$ gilt:%
\begin{equation}
(a+b)^n = \sum_{k=0}^n \binom{n}{k} a^{n-k} b^k
\end{equation}
und
\begin{equation}
(a-b)^n = \sum_{k=0}^n \binom{n}{k} (-1)^k a^{n-k} b^k.
\end{equation}
Die ersten Formeln sind:\index{binomische Formeln}
\begin{gather}
(a+b)^2 = a^2+2ab+b^2,\\
(a-b)^2 = a^2-2ab+b^2,\\
(a+b)^3 = a^3+3a^2 b+3ab^2+b^3,\\
(a-b)^3 = a^3-3a^2 b+3ab^2-b^3,\\
(a+b)^4 = a^4+4a^3 b+6a^2 b^2+4ab^3+b^4,\\
(a-b)^4 = a^4-4a^3 b+6a^2 b^2-4ab^3+b^4.
\end{gather}

\newpage
\subsection{Potenzgesetze}
\begin{definition}[Potenz]\mbox{}\newline
Für $a$ aus einem Monoid:
\begin{align}
a^0 &:= 1,\\
a^n &:= a^{n-1}\cdot a.\quad (n\in\N)
\end{align}
Für $a\in\R$, $a>0$ und $x\in\C$:
\begin{equation}
a^x := \exp(\ln(a)\,x).
\end{equation}
\end{definition}
\noindent
Für $a\in\R$, $a>0$ und $x,y\in\C$ gilt:
\begin{gather}
a^{x+y} = a^x a^y,\quad a^{x-y} = \frac{a^x}{a^y},
\quad a^{-x} = \frac{1}{a^x}.
\end{gather}

\section{Gleichungen}
\begin{definition}[Bestimmungsgleichung]\mbox{}\newline
Sind $f,g$ auf der Grundmenge $G$ definierte Funktionen, so nennt
man
\begin{equation}
f(x) = g(x)\\
\end{equation}
eine \emdef{Bestimmungsgleichung}\index{Bestimmungsgleichung},
wenn die Lösungemenge
\begin{equation}
L = \{x\in G\mid f(x)=g(x)\}
\end{equation}
gesucht ist.
\end{definition}
Bei den $x\in G$ kann es sich auch um Tupel $x=(x_1,x_2)$ oder
$x=(x_1,x_2,x_3)$ usw. handeln. Man spricht in diesem Fall
von einer Gleichung \emdef{in mehreren Variablen}.

Handelt es sich bei den Funktionswerten von $f,g$ um Tupel,
dann spricht man von einem
\emdef{Gleichungssystem}\index{Gleichungssystem}.

\subsection{Äquivalenzumformungen}

Äquivalenzumformungen lassen die Lösungsmenge einer Gleichung
unverändert. Seien $A(x),B(x)$ zwei Aussageformen bzw. zwei
Gleichungen. Aus
\begin{equation}
\forall x{\in}G\,[A(x)\Longleftrightarrow B(x)]
\end{equation}
folgt
\begin{equation}
\{x\in G\mid A(x)\} = \{x\in G\mid B(x)\}.
\end{equation}
Aus
\begin{equation}
\forall x{\in} G\,[A(x)\Longrightarrow B(x)]
\end{equation}
folgt jedoch nur noch
\begin{equation}
\{x\in G\mid A(x)\}\subseteq\{x\in G\mid B(x)\}.
\end{equation}
Seien $f,g,h$ Funktionen mit Definitionsmenge $G$ und
Zielmenge $Z=\R$ oder $Z=\C$.

Für alle $x$ gilt:
\begin{align}
f(x)=g(x) &\Longleftrightarrow f(x){+}h(x)=g(x){+}h(x),\\
f(x)=g(x) &\Longleftrightarrow f(x){-}h(x)=g(x){-}h(x).
\end{align}
Besitzt $h(x)$ keine Nullstellen, dann gilt für alle $x$:
\begin{align}
f(x)=g(x) &\iff f(x)h(x)=g(x)h(x),\\
f(x)=g(x) &\iff \frac{f(x)}{h(x)}=\frac{f(x)}{h(x)}.
\end{align}
Besitzt $h(x)$ aber Nullstellen, dann gilt immerhin noch für alle $x$:
\begin{equation}
f(x)=g(x) \implies f(x)h(x)=g(x)h(x).
\end{equation}

\newpage\noindent
Sei $f,g\colon G\to Z$. Sei $\varphi_x\colon Z\to Z'$ eine Injektion
für jedes $x\in G$. Es gilt
\begin{equation}
f(x)=g(x) \iff \varphi_x(f(x))=\varphi_x(g(x))
\end{equation}
für alle $x\in G$.

Bei einer Kette von Äquivalenzumformungen wird links das
Äquivalenzzeichen geschrieben, in der Mitte die Gleichung
und rechts hinter einem senkrechten Strich die Operation
$\varphi_x(t)$, welche als nächstes auf beide Seiten der Gleichung
angwendet werden soll.

Beispiel:
\begin{equation*}\setlength{\arraycolsep}{2pt}
\begin{array}{rrl@{\qquad}l}
& 2x+4 &= 2x^2-8x+2 &\mid t/2\\[2pt]
\Longleftrightarrow& x+2 &= x^2-4x+1 &\mid t-2\\[2pt]
\Longleftrightarrow& x &= x^2-4x-1 &\mid t-x\\[2pt]
\Longleftrightarrow& 0 &= x^2-7x-1.
\end{array}
\end{equation*}
Am Anfang befinden sich eventuell Bedingungen für $x$.
Bei Fallunterscheidungen wird eine Verschärfung der Bedingungen
vorgenommen, so dass es zur Verkleinerung der Grundmenge kommt.
Nach einer Fallunterscheidung ergeben sich unter Umständen neue
Injektionen.

Eine Gleichung impliziert immer auch Gleichheit nach
Anwendung einer beliebigen Abbildung $\varphi$ auf beide
Seiten, d.\,h.
\begin{equation}
f(x)=g(x)\implies \varphi(f(x))=\varphi(g(x)).
\end{equation}
Bei einem nichtinjektiven $\varphi$ handelt es sich jedoch
nicht mehr um eine Äquivalenzumformung, wodurch es zur Vergrößerung
der Lösungsmenge kommt. 

Die tatsächliche Lösungsmenge lässt sich finden, indem für alle
Lösungen die Probe durch Einsetzen in die ursprüngliche Gleichung
gemacht wird, wodurch sich die Scheinlösungen abscheiden lassen.


\subsection{Quadratische Gleichungen}
\begin{definition}[Quadratische Gleichung]\mbox{}\newline
Eine Gleichung der Form $ax^2+bx+c=0$ mit $a\ne 0$ heißt
\emdef{quadratische Gleichung}.
\end{definition}

\noindent
Wegen $a\ne 0$ lässt sich die Gleichung durch $a$ dividieren
und es ensteht die äquivalente Normalform $x^2+px+q=0$
mit $p:=b/a$ und $q:=c/a$.

\minisection\strong{Lösung.}
Seien nun die $a,b,c$ reelle Zahlen. Die Zahl
\begin{equation}
D = p^2-4q
\end{equation}
heißt \emdef{Diskriminante}. Für $D>0$ gibt es zwei reelle Lösungen:
\begin{align}
x_1 &= \frac{-p-\sqrt{D}}{2} = \frac{-b-\sqrt{b^2-4ac}}{2a},\\
x_2 &= \frac{-p+\sqrt{D}}{2} = \frac{-b+\sqrt{b^2-4ac}}{2a}.
\end{align}
Für $D=0$ fallen beiden Lösungen zu einer \emdef{doppelten Lösung}
zusammen:
\begin{equation}
x_1 = x_2 = -\frac{p}{2} = -\frac{b}{2a}.
\end{equation}
Für $D<0$ gibt es keine reelle Lösung. Aber es gibt zwei komplexe
Lösungen, die zueinander konjugiert sind:
\begin{equation}
x_1 = \frac{-p-\ui\sqrt{|D|}}{2},\quad
x_2 = \frac{-p+\ui\sqrt{|D|}}{2}.
\end{equation}
In jedem Fall gelten die Formeln von Vieta:
\begin{equation}
p = -(x_1+x_2),\qquad q = x_1 x_2.
\end{equation}

\section{Komplexe Zahlen}\index{komplexe Zahl}

\subsection{Rechenoperationen}
\begin{table*}[t]
\caption{Rechnen mit komplexen Zahlen}%
\label{tab:komplex}
\bgroup
% \def\arraystretch{1.4}
\begin{tabular}{lrll}
\toprule
  \thbf{Name}
& \thbf{Operation}
& \thbf{Polarform}
& \thbf{kartesische Form}\\
\midrule
  Identität
& $z$ & $=r\ee^{\ui\varphi}$
& $= a+b\ui$\\
\midrule
  Realteil
& $\real(z)$
& $=r\cos\varphi$
& $=a$\\
\midrule
  Imaginärteil
& $\imag(z)$
& $=r\sin\varphi$
& $=b$\\
\midrule
  Addition
& $z_1+z_2$ &
& $= (a_1+a_2)+(b_1+b_2)\ui$\\
\midrule
  Subtraktion
& $z_1-z_2$ &
& $= (a_1-a_2)+(b_1-b_2)\ui$\\
\midrule
  Multiplikation
& $z_1 z_2$
& $= r_1 r_2 \ee^{\ui(\varphi_1+\varphi_2)}$
& $= (a_1 a_2 - b_1 b_2)+(a_1 b_2+a_2 b_1)\ui$\\
\midrule
  Division
& $\displaystyle\frac{z_1}{z_2}$
& $\displaystyle =\frac{r_1}{r_2}\ee^{\ui(\varphi_1-\varphi_2)}$
& $\displaystyle =\frac{a_1 a_2 + b_1 b_2}{a_2^2+b_2^2}
   + \frac{a_2 b_1 - a_1 b_2}{a_2^2+b_2^2}\ui$\\
\midrule
  Kehrwert
& $\displaystyle\frac{1}{z}$
& $\displaystyle =\frac{1}{r}\ee^{-\ui\varphi}$
& $\displaystyle =\frac{a}{a^2+b^2}-\frac{b}{a^2+b^2}\ui$\\
\midrule
  Konjugation
& $\overline{z}$
& $=r\ee^{-\varphi\ui}$
& $=a-b\ui$\\
\midrule
Betrag
& $|z|$
& $=r$
& $=\sqrt{a^2+b^2}$\\
\midrule
  Argument
& $\arg(z)$
& $=\varphi$
& $\displaystyle = s(b)\arccos\Big(\frac{a}{r}\Big)$\\
\bottomrule
\end{tabular}
\egroup\\
\\
$s(b):=\begin{cases}
+1 & \text{wenn}\;b\ge 0,\\
-1 & \text{wenn}\;b<0
\end{cases}$
\end{table*}

Siehe Tabelle \ref{tab:komplex}. Für die Division gilt
\begin{gather}
\frac{z_1}{z_2}
= \frac{z_1\overline z_2}{z_2\overline z_2}
= \frac{z_1\overline z_2}{|z_2|^2},\\
\frac{1}{z} = \frac{\overline z}{z\overline z}
= \frac{\overline z}{|z|^2}.
\end{gather}

\subsection{Betrag}\index{Betrag!einer komplexen Zahl}
Für alle $z_1,z_2\in\C$ gilt:
\begin{gather}
|z_1z_2| = |z_1|\,|z_2|,\\
z_2\ne 0\implies \Big|\frac{z_1}{z_2}\Big|
= \frac{|z_1|}{|z_2|},\\
z\,\overline z = |z|^2.
\end{gather}

\subsection{Konjugation}\index{Konjugation!einer komplexen Zahl}
Für alle $z_1,z_2\in\C$ gilt:
\begin{gather}
\overline{z_1+z_2} = \overline z_1+\overline z_2,\qquad
\overline{z_1-z_2} = \overline z_1-\overline z_2,\\
\overline{z_1 z_2} = \overline z_1\,\overline z_2,\qquad
z_2\ne 0 \implies \overline{\Big(\frac{z_1}{z_2}\Big)}
= \frac{\overline z_1}{\overline z_2},\\
\overline{\overline z}=z,\qquad
|\overline{z}| = |z|,\qquad
z\,\overline z = |z|^2,\\
\real(z) = \frac{z+\overline z}{2},\qquad
\imag(z) = \frac{z-\overline z}{2\ui},\\
\overline{\cos(z)} = \cos(\overline z),\qquad
\overline{\sin(z)} = \sin(\overline z),\\
\overline{\exp(z)} = \exp(\overline z).
\end{gather}
Ist $f$ holomorph auf ganz $\C$ und $f(\R)\subseteq\R$, dann gilt
\begin{equation}
\overline{f(z)} = f(\,\overline z\,).
\end{equation}

\subsection{Darstellungen}
Die Darstellung der komplexen Zahlen als Matrizen
ermöglicht der Körperisomorphismus
\begin{equation}
\Phi\colon\C\to\Phi(\C),
\quad \Phi(a+b\ui) := \begin{bmatrix}a & -b\\ b & a\end{bmatrix}.
\end{equation}
Hierbei gilt $\Phi(\C\setminus\{0\})\subset\operatorname{GL}(\R,2)$. Gemäß
\begin{equation}
\Phi(r\ee^{\ui\varphi}) = r\begin{bmatrix}
\cos\varphi & -\sin\varphi\\
\sin\varphi & \cos\varphi
\end{bmatrix}
\end{equation}
ist $\C\setminus\{0\}$ isomorph zu $\R^+\operatorname{SO}(2)$,
der Gruppe der Drehskalierungen.

Es gilt
\begin{align}
& \Phi(\,\overline{z}\,) = \Phi(z)^T,
&& \Phi(z^{-1}) = \Phi(z)^{-1},\\
& |z| = \det(\Phi(z)), && \Phi(\ee^z) = \exp(\Phi(z)).
\end{align}

\section{Logik}
\subsection{Aussagenlogik}\index{Aussagenlogik}\label{subsec:Aussagenlogik}
\subsubsection{Boolesche Algebra}\index{boolesche Algebra}

\begin{table*}[t]
\caption{Boolesche Algebra}
\begin{tabular}{c@{\qquad}c@{\qquad}l}
\toprule
\thbf{Disjunktion} & \thbf{Konjunktion} & \thbf{Bezeichnung}\\
\midrule
  $A\lor A \equiv A$
& $A\land A \equiv A$
& Idempotenzgesetze\\
  $A\lor 0 \equiv A$
& $A\land 1 \equiv A$
& Neutralitätsgesetze\\
  $A\lor 1 \equiv 1$
& $A\land 0 \equiv 0$
& Extremalgesetze\\
  $A\lor \overline A \equiv 1$
& $A\land \overline A \equiv 0$
& Komplementärgesetze\\
\midrule
  $A\lor B \equiv B\lor A$
& $A\land B \equiv B\land A$
& Kommutativgesetze\\
  $(A{\lor}B){\lor}C \equiv A{\lor}(B{\lor}C)$
& $(A{\land}B){\land}C \equiv A{\land}(B{\land}C)$
& Assoziativgesetze\\
  $\overline{A\lor B} \equiv \overline A\land\overline B$
& $\overline{A\land B} \equiv \overline A\lor\overline B$
& De Morgansche Regeln\\
  $A\lor (A\land B) \equiv A$
& $A\land (A\lor B) \equiv A$
& Absorptionsgesetze\\
\bottomrule
\end{tabular}
\end{table*}

\noindent
\strong{Distributivgesetze}:
\begin{gather}
A\lor (B\land C) \iff (A\lor B)\land (A\lor C),\\
A\land (B\lor C) \iff (A\land B)\lor (A\land C).
\end{gather}

\subsubsection{Zweistellige Funktionen}

Es gibt 16 zweistellige boolesche
Funktionen. Die wichtigsten sind in Tabelle \ref{tab:Wahrheitstafel}
definiert. Tabelle \ref{tab:Wahrheitstafel-16} gibt eine
Übersicht über alle 16.

\begin{table}
\caption{Wahrheitstafel}\label{tab:Wahrheitstafel}
\begin{tabular}{ccccccc}
\toprule
$A$ & $B$ & Wert & $A\land B$ & $A\lor B$
& $A\Rightarrow B$ & $A\Leftrightarrow B$\\
\midrule
0 & 0 & \texttt{a} & 0 & 0 & 1 & 1\\
0 & 1 & \texttt{b} & 0 & 1 & 1 & 0\\
1 & 0 & \texttt{c} & 0 & 1 & 0 & 0\\
1 & 1 & \texttt{d} & 1 & 1 & 1 & 1\\
\bottomrule
\end{tabular}
\end{table}

\begin{table}
\caption{Große Wahrheitstafel}
\label{tab:Wahrheitstafel-16}
\begin{tabular}{rlcl}
\toprule
\thbf{Nr.}& \textbf{\texttt{dcba}} & \thbf{Funktion} & \thbf{Name}\\
\midrule
 0 & \texttt{0000} & 0 & Kontradiktion\\
 1 & \texttt{0001} & $\neg(A\lor B)$ & NOR\\
 2 & \texttt{0010} & $\neg(B\Rightarrow A)$\\
 3 & \texttt{0011} & $\neg A$\\
\midrule
 4 & \texttt{0100} & $\neg(A\Rightarrow B)$\\
 5 & \texttt{0101} & $\neg B$\\
 6 & \texttt{0110} & $A\oplus B$ & Kontravalenz\index{Kontravalenz}\\
 7 & \texttt{0111} & $\neg(A\land B)$ & NAND\\
\midrule
 8 & \texttt{1000} & $A\land B$ & Konjunktion\index{Konjunktion}\\
 9 & \texttt{1001} & $A\Leftrightarrow B$ & Äquivalenz\\
10 & \texttt{1010} & $B$ & Projektion\\
11 & \texttt{1011} & $A\Rightarrow B$ & Implikation\\
\midrule
12 & \texttt{1100} & $A$ & Projektion\\
13 & \texttt{1101} & $B\Rightarrow A$ & Implikation\\
14 & \texttt{1110} & $A\lor B$ & Disjunktion\index{Disjunktion}\\
15 & \texttt{1111} & $1$ & Tautologie\\
\bottomrule
\end{tabular}
\end{table}

\subsubsection[Darstellung mit Negation, Konjunktion und Disjunktion]%
{Darstellung mit Negation,\newline Konjunktion und Disjunktion}
\begin{gather}\label{eq:implication-definition}
(A\Rightarrow B) \equiv \overline A\lor B,\\
(A\Leftrightarrow B) \equiv
  (\overline A\land\overline B)\lor(A\land B),\\
A\oplus B \equiv (\overline A\land B)\lor(A\land\overline B).
\end{gather}

\newpage
\subsubsection{Tautologien}
Modus ponens:
\begin{equation}\label{eq:modus-ponens}
(A\Rightarrow B)\land A\implies B.
\end{equation}
Modus tollens:
\begin{equation}
(A\Rightarrow B)\land\overline B\implies\overline A.
\end{equation}
Modus tollendo ponens:
\begin{equation}
(A\lor B)\land\overline A \implies B.
\end{equation}
Modus ponendo tollens:
\begin{equation}
\overline{A\land B}\land A\implies\overline B.
\end{equation}
Kontraposition:\index{Kontraposition}
\begin{equation}
A\Rightarrow B \iff \overline B\Rightarrow \overline A.
\end{equation}
Beweis durch Widerspruch:\index{Widerspruch}
\begin{equation}
(\overline A\Rightarrow B\land\overline B)\implies A.
\end{equation}
Zerlegung einer Äquivalenz:
\begin{equation}
(A\Leftrightarrow B) \iff (A\Rightarrow B)\land(B\Rightarrow A).
\end{equation}
Kettenschluss:
\begin{equation}
(A\Rightarrow B)\land(B\Rightarrow C)\implies (A\Rightarrow C).
\end{equation}
Ringschluss:
\begin{equation}
\begin{split}
&(A\Rightarrow B)\land (B\Rightarrow C)\land(C\Rightarrow A)\\
&\implies (A\Leftrightarrow B)\land(A\Leftrightarrow C)\land(B\Leftrightarrow C).
\end{split}
\end{equation}
Ringschluss, allgemein:
\begin{equation}
\begin{split}
& (A_1{\Rightarrow }A_2)\land\ldots\land(A_{n-1}{\Rightarrow}A_n)
\land(A_n{\Rightarrow}A_1)\\
& \implies \forall i,j\,[A_i\Leftrightarrow A_j].
\end{split}
\end{equation}
Für jede Funktion $P\colon\{0,1\}\to\{0,1\}$ gilt:
\begin{equation}
P(A)\land (A\Leftrightarrow B)\implies P(B).
\end{equation}
Regel zur Implikation:
\begin{equation}
A\land B\Rightarrow C \iff A\Rightarrow (B\Rightarrow C).
\end{equation}
Vollständige Fallunterscheidung:
\begin{gather}
(A\Rightarrow C)\land (B\Rightarrow C)\implies (A\oplus B\Rightarrow C),\\
(A\Rightarrow C)\land (B\Rightarrow C)\iff (A\lor B\Rightarrow C).
\end{gather}
Vollständige Fallunterscheidung, allgemein:
\begin{gather}
\textstyle \forall k[A_k\Rightarrow C]
\implies (\bigoplus_{k=1}^n A_k\Rightarrow C),\\
\forall k[A_k\Rightarrow C]
\iff (\exists k[A_k]\Rightarrow C).
\end{gather}

\subsubsection{Schlussregeln}
\begin{Satz}[Ersetzungsregel]
Sei $F(\varphi)$ eine aussagenlogische
Formel in expliziter Abhängigkeit von der Formelvariablen $\varphi$.
Ist $\varphi\leftrightarrow\psi$, dann darf $\varphi$ gegen $\psi$
ersetzt werden:
\begin{equation}
\{F(\varphi),\varphi\leftrightarrow\psi\}\vdash F(\psi).
\end{equation}
\end{Satz}

\newpage\noindent
\strong{Beispiel.} Betrachte $\varphi\land A\rightarrow B$ mit
$\varphi:=(A\rightarrow B)$, was expandiert wird zu
\[(A\rightarrow B)\land A\rightarrow B.\qquad\text{(s. \eqref{eq:modus-ponens})}\]
Nun gilt nach \eqref{eq:implication-definition} aber
\[A\rightarrow B\leftrightarrow \overline A\lor B.\]
Daher lässt sich folgern:
\[(\overline A\lor B)\land A\rightarrow B.\]

\subsubsection{Metatheoreme}

\begin{Satz}[Korrektheit der Aussagenlogik]\mbox{}\\*
Für die Aussagenlogik gilt:
\begin{equation}
(\Gamma\vdash\psi)\implies (\Gamma\models\psi).
\end{equation}
\end{Satz}

\begin{Satz}[Vollständigkeit der Aussagenlogik]\mbox{}\\*
Für die Aussagenlogik gilt:
\begin{equation}
(\Gamma\models\psi)\implies (\Gamma\vdash\psi).
\end{equation}
\end{Satz}

\begin{Satz}[Deduktionstheorem (syntaktisch)]\mbox{}\\*
Für die Aussagenlogik gilt:
\begin{equation}
(\Gamma\cup\{\varphi\}\vdash\psi)\iff (\Gamma\vdash\varphi\rightarrow\psi).
\end{equation}
\end{Satz}
Infolge gilt auch:
\begin{equation}
\begin{split}
&(\{\varphi_1,\ldots,\varphi_n\}\vdash\psi)\\
&\iff (\vdash \varphi_1\land\ldots\land\varphi_n\rightarrow\psi).
\end{split}
\end{equation}

\begin{Satz}[Deduktionstheorem (semantisch)]\mbox{}\\*
Für die Aussagenlogik gilt:
\begin{equation}
(\Gamma\cup\{\varphi\}\models\psi)\iff (\Gamma\models\varphi\rightarrow\psi).
\end{equation}
\end{Satz}
Infolge gilt auch:
\begin{equation}
\begin{split}
&(\{\varphi_1,\ldots,\varphi_n\}\models\psi)\\
&\iff (\models \varphi_1\land\ldots\land\varphi_n\rightarrow\psi).
\end{split}
\end{equation}

\begin{Satz}[Einsetzungsregel]
Sei $v$ eine metasprachliche Variable, die für eine beliebige
objektsprachliche Variable steht. Dann gilt:
\begin{equation}
(\models\varphi) \implies (\models\varphi[v:=\psi]).
\end{equation}
D.\,h. wenn in der tautologischen Formel $\varphi$ jedes auftreten
der Variable $v$ gegen die Formel $\psi$ ersetzt wird, ergibt
sich wieder eine tautologische Formel.
\end{Satz}

\subsubsection{Regeln zum Tableaukalkül}

\begin{tabular}{llll}
\begin{tabular}{c}
$\varphi\land\psi$\\
\hline
$\varphi$\\
$\psi$
\end{tabular}
&
\begin{tabular}{c}
$\neg(\varphi\land\psi)$\\
\hline
$\neg\varphi$ | $\neg\psi$
\end{tabular}
&
\begin{tabular}{c}
$\varphi\lor\psi$\\
\hline
$\varphi$ | $\psi$
\end{tabular}
&
\begin{tabular}{c}
$\hspace{-4pt}\neg(\varphi\lor\psi)\hspace{-4pt}$\\
\hline
$\neg\varphi$\\
$\neg\psi$
\end{tabular}
\end{tabular}

\vspace{1em}
\noindent
\begin{tabular}{llll}
\begin{tabular}{c}
$\varphi\rightarrow\psi$\\
\hline
$\neg\varphi$ | $\psi$
\end{tabular}
&
\begin{tabular}{c}
$\hspace{-4pt}\neg(\varphi\rightarrow\psi)\hspace{-4pt}$\\
\hline
$\varphi$\\
$\neg\psi$
\end{tabular}
&
\begin{tabular}{c}
$\varphi\leftrightarrow\psi$\\
\hline
\hspace{-10pt}\begin{tabular}{c|c}
$\varphi$ & $\neg\varphi$\\
$\psi$    & $\neg\psi$
\end{tabular}\hspace{-10pt}
\end{tabular}
&
\begin{tabular}{c}
$\hspace{-4pt}\neg(\varphi\leftrightarrow\psi)\hspace{-4pt}$\\
\hline
\hspace{-10pt}\begin{tabular}{c|c}
$\varphi$ & $\psi$\\
$\neg\psi$    & $\neg\varphi$
\end{tabular}\hspace{-10pt}
\end{tabular}
\end{tabular}

% \vspace{1em}

\newpage
\subsection{Prädikatenlogik}\label{subsec:Praedikatenlogik}
\subsubsection{Rechenregeln}
Verneinung (De Morgansche Regeln):
\begin{gather}
\overline{\forall x[P(x)]}\equiv \exists x[\overline{P(x)}],\\
\overline{\exists x[P(x)]}\equiv \forall x[\overline{P(x)}].
\end{gather}
Verallgemeinerte Distributivgesetze:
\begin{gather}
P\lor\forall x[Q(x)] \equiv \forall x[P\lor Q(x)],\\
P\land\exists x[Q(x)] \equiv \exists x[P\land Q(x)],\\
P\land\forall x[Q(x)] \equiv \forall x[P\land Q(x)],\quad (U\ne\emptyset)\\
P\lor\exists x[Q(x)] \equiv \exists x[P\lor Q(x)].\quad (U\ne\emptyset)
\end{gather}
Verallgemeinerte Idempotenzgesetze:
\begin{gather}
\begin{split}
\exists x{\in}M\,[P] & \equiv
(M\ne\{\})\land P\\
& \equiv\begin{cases}
P & \text{wenn}\; M\ne\emptyset,\\
0 & \text{wenn}\; M=\emptyset.
\end{cases}
\end{split}\\
\begin{split}
\forall x{\in}M\,[P]& \equiv
(M=\{\})\lor P\\
&\equiv\begin{cases}
P & \text{wenn}\; M\ne\emptyset,\\
1 & \text{wenn}\; M=\emptyset.
\end{cases}
\end{split}
\end{gather}
%\newpage\noindent
Äquivalenzen:
\begin{gather}
\forall x\forall y[P(x,y)] \equiv \forall y\forall x[P(x,y)],\\
\exists x\exists y[P(x,y)] \equiv \exists y\exists x[P(x,y)],\\
\forall x[P(x)\land Q(x)] \equiv \forall x[P(x)]\land\forall x[Q(x)],\\
\exists x[P(x)\lor Q(x)] \equiv \exists x[P(x)]\lor\exists x[Q(x)],\\
\forall x[P(x)\Rightarrow Q] \equiv \exists x[P(x)]\Rightarrow Q,\\
\forall x[P\Rightarrow Q(x)] \equiv P\Rightarrow\forall x[Q(x)],\\
\exists x[P(x)\Rightarrow Q(x)]
  \equiv\forall x[P(x)]\Rightarrow\exists x[Q(x)].
\end{gather}
% \newpage\noindent
Implikationen:
\begin{gather}
\hspace{-1em}\exists x\forall y[P(x,y)]\Rightarrow \forall y\exists x[P(x,y)],\\
\hspace{-1em}\forall x[P(x)]\lor\forall x[Q(x)]\Rightarrow\forall x[P(x)\lor Q(x)],\\
\hspace{-1em}\exists x[P(x)\land Q(x)]\Rightarrow
  \exists x[P(x)]\land \exists x[Q(x)],\\
\hspace{-1em}\forall x[P(x)\Rightarrow Q(x)]\Rightarrow
  (\forall x[P(x)]\Rightarrow\forall x[Q(x)]),\\
\hspace{-1em}\forall x[P(x)\Leftrightarrow Q(x)]\Rightarrow
  (\forall x[P(x)]\Leftrightarrow\forall x[Q(x)]).
\end{gather}

\subsubsection{Endliche Mengen}
Sei $M=\{x_1,\ldots,x_n\}$. Es gilt:
\begin{gather}
\forall x{\in}M\,[P(x)]\equiv P(x_1)\land\ldots\land P(x_n),\\
\exists x{\in}M\,[P(x)]\equiv P(x_1)\lor\ldots\lor P(x_n).
\end{gather}

\subsubsection{Beschränkte Quantifizierung}
\begin{gather}
\begin{split}
& \forall x{\in}M\,[P(x)]:\equiv\forall x[x\notin M\lor P(x)]\\
& \quad\equiv\forall x[x\in M\Rightarrow P(x)],
\end{split}\\
\exists x{\in}M\,[P(x)]:\equiv\exists x[x\in M\land P(x)].
\end{gather}

\subsubsection[Quantifizierung über Produktmengen]%
{Quantifizierung über\newline Produktmengen}
\begin{gather}
\forall(x,y)\,[P(x,y)]\equiv \forall x\forall y[P(x,y)],\\
\exists(x,y)\,[P(x,y)]\equiv \exists x\exists y[P(x,y)].
\end{gather}
Analog gilt
\begin{gather}
\forall(x,y,z)\,\equiv \forall x\forall y\forall z,\\
\exists(x,y,z)\,\equiv \exists x\exists y\exists z
\end{gather}
usw.

\subsubsection{Alternative Darstellung}
Sei $P\colon G\to\{0,1\}$ und $M\subseteq G$.
Mit $P(M)$ ist die Bildmenge von $P$ bezüglich $M$ gemeint.
Es gilt
\begin{equation}
\begin{split}
&\forall x{\in}M\,[P(x)] \iff P(M)=\{1\}\\
& \iff M\subseteq\{x{\in}G\mid P(x)\}
\end{split}
\end{equation}
und
\begin{equation}
\begin{split}
& \exists x{\in}M\,[P(x)] \iff \{1\}\subseteq P(M)\\
& \iff M\cap\{x{\in}G\mid P(x)\}\ne\emptyset.
\end{split}
\end{equation}

\subsubsection{Eindeutige Existenz}
\begin{definition}[Quantor für eindeutige Existenz]
\begin{equation}
\begin{split}
&\exists!x\,[P(x)]:\equiv
\exists x\,[P(x)\land \forall y\,[P(y)\Rightarrow x=y]]\\
&{\equiv}\;\exists x\,[P(x)]\land \forall x\forall y[P(x)\land P(y)\Rightarrow x=y].
\end{split}
\end{equation}
\end{definition}
Es gilt:
\begin{gather}
\exists!x[P\land Q(x)] \equiv P\land\exists!x[Q(x)].
\end{gather}

\newpage
\section{Mengenlehre}
\subsection{Definitionen}
Aufzählende Angabe einer Menge:
\begin{equation}
\hspace{-1em} a\in\{x_1,\ldots,x_n\} :\Leftrightarrow a=x_1\lor\ldots\lor a=x_n.
\end{equation}
Beschreibende Angabe einer Menge:
\begin{gather}
a\in\{x\mid P(x)\}\defiff P(a),\\
\{x\in M\mid P(x)\} := \{x\mid x\in M\land P(x)\},\\
\hspace{-1em}\{f(x)\mid P(x)\} := \{y\mid \exists x(y=f(x)\land P(x))\}.
\end{gather}
Teilmengenrelation:
\begin{equation}
A\subseteq B\defiff \forall x\,(x\in A\implies x\in B).
\end{equation}
Gleichheit:
\begin{equation}
A=B\defiff \forall x\,(x\in A\iff x\in B).
\end{equation}
Vereinigungsmenge:
\begin{equation}
A\cup B:=\{x\mid x\in A\lor x\in B\}.
\end{equation}
Schnittmenge:
\begin{equation}
A\cap B:=\{x\mid x\in A\land x\in B\}.
\end{equation}
Differenzmenge:
\begin{equation}
A\setminus B:=\{x\mid x\in A\land x\not\in B\}.
\end{equation}
Symmetrische Differenz:
\begin{equation}
A\triangle B:=\{x\mid x\in A\oplus x\in B\}.
\end{equation}
Komplementärmenge:
\begin{equation}
A^\comp := G\setminus A.\qquad (\text{$G$: Grundmenge})
\end{equation}
Vereinigung über indizierte Mengen:
\begin{equation}
\bigcup_{i\in I} A_i := \{x\mid\exists i{\in}I\,(x\in A_i)\}.
\end{equation}
Schnitt über indizierte Mengen:
\begin{equation}
\bigcap_{i\in I} A_i := \{x\mid\forall i{\in}I\,(x\in A_i)\}.
\end{equation}


\subsection{Boolesche Algebra}
\begin{table*}[t]
\caption{Boolesche Algebra}
\begin{tabular}{c@{\qquad}c@{\qquad}l}
\toprule
\thbf{Vereinigung} & \thbf{Schnitt} & \thbf{Bezeichnung}\\
\midrule
  $A\cup A = A$
& $A\cap A = A$
& Idempotenzgesetze\\
  $A\cup \emptyset = A$
& $A\cap G = A$
& Neutralitätsgesetze\\
  $A\cup G = G$
& $A\cap \emptyset = \emptyset$
& Extremalgesetze\\
  $A\cup A^\comp = G$
& $A\cap A^\comp = \emptyset$
& Komplementärgesetze\\
\midrule
  $A\cup B = B\cup A$
& $A\cap B = B\cap A$
& Kommutativgesetze\\
  $(A{\cup}B){\cup}C = A{\cup}(B{\cup}C)$
& $(A{\cap}B){\cap}C = A{\cap}(B{\cap}C)$
& Assoziativgesetze\\
  $(A\cup B)^\comp = A^\comp\cap B^\comp$
& $(A\cap B)^\comp = A^\comp\cup B^\comp$
& De Morgansche Regeln\\
  $A\cup (A\cap B) = A$
& $A\cap (A\cup B) = A$
& Absorptionsgesetze\\
\bottomrule
\end{tabular}\\
\\
$G$: Grundmenge
\end{table*}

\noindent
\strong{Distributivgesetze}:
\begin{gather}
M\cup (A\cap B) = (M\cup A)\cap (M\cup B),\\
M\cap (A\cup B) = (M\cap A)\cup (M\cap B).
\end{gather}

\subsection{Teilmengenrelation}
Zerlegung der Gleichheit:
\begin{equation}
A=B \iff A\subseteq B \land B\subseteq A.
\end{equation}
Umschreibung der Teilmengenrelation:
\begin{equation}
\begin{split}
A\subseteq B &\iff A\cap B=A\\
& \iff A\cup B=B\\
& \iff A\setminus B=\{\}.
\end{split}
\end{equation}
Kontraposition:
\begin{equation}
A\subseteq B \iff B^\comp\subseteq A^\comp.
\end{equation}

\subsection{Natürliche Zahlen}
\subsubsection{Von-Neumann-Modell}
Mengentheoretisches Modell der natürlichen Zahlen:
\begin{equation}
\begin{split}
& 0:=\emptyset,\quad 1:=\{0\},\quad 2:=\{0,1\},\\
& 3:=\{0,1,2\},\quad \text{usw.}
\end{split}
\end{equation}
Nachfolgerfunktion:
\begin{equation}
x' := x\cup\{x\}.
\end{equation}
\subsubsection{Vollständige Induktion}
Ist $A(n)$ mit $n\in\N$
eine Aussageform, so gilt:
\begin{equation}
\begin{split}
& A(n_0)\land \forall n\ge n_0\,[A(n)\Rightarrow A(n+1)]\\
& \implies \forall n\ge n_0\,[A(n)].
\end{split}
\end{equation}
Die Aussage $A(n_0)$ ist der \emph{Induktionsanfang}.

Die Implikation
\begin{equation}
A(n)\Rightarrow A(n+1)
\end{equation}
heißt \emph{Induktionsschritt}. Beim Induktionsschritt muss
$A(n+1)$ gezeigt werden, wobei $A(n)$ als gültig vorausgesetzt werden
darf.

% \newpage
\subsection{ZFC-Axiome}

Axiom der Bestimmtheit:
\begin{equation}
\forall A\forall B\,[A=B\iff\forall x\,[x\in A\Leftrightarrow x\in B]].
\end{equation}
Axiom der leeren Menge:
\begin{equation}
\exists M\forall x\,[x\notin M].
\end{equation}
Axiom der Paarung:
\begin{equation}
\forall x\forall y\exists M\forall a\,[a\in M\iff x=a\lor y=a].
\end{equation}
Axiom der Vereinigung:
\begin{equation}
\forall S\exists M\forall x\,[x\in M\iff\exists A{\in}S\,[x\in A]].
\end{equation}
Axiom der Aussonderung:
\begin{equation}
\forall A\exists M\forall x\,[x\in M\iff x\in A\land\varphi(x)].
\end{equation}
Axiom des Unendlichen:
\begin{equation}
\exists M\,[\emptyset\in M\land\forall x{\in}M\,[x\cup\{x\}\in M]].
\end{equation}
Axiom der Potenzmenge:
\begin{equation}
\forall A\exists M\forall T\,[T\in M\iff T\subseteq A].
\end{equation}
Axiom der Ersetzung:
\begin{equation}
\begin{split}
&\forall a{\in}A\;\exists^{=1} b\,[\varphi(a,b)]\\
&\implies\exists B\,\forall b\,[b\in B\iff\exists a{\in}A\,[\varphi(a,b)]].
\end{split}
\end{equation}
Axiom der Fundierung:
\begin{equation}
\forall A\,[A\ne\emptyset\implies\exists x{\in}A\,[x\cap A=\emptyset]].
\end{equation}
Auswahlaxiom:
\begin{equation}
\begin{split}
&\forall x,y{\in}A\,[x\ne y\implies x\cap y=\emptyset]\\
&\quad\land\forall x{\in}A\,[x\ne\emptyset]\\
&\implies\exists M\;\forall x{\in}A\;\exists^{=1}u{\in}x\,[u\in M].
\end{split}
\end{equation}

\newpage
\section{Funktionen}
\subsection{Injektionen}\index{injektiv}
\begin{definition}[Injektion]\mbox{}\newline
Eine Funktion $f\colon A\to B$ heißt \emdef{injektiv},
wenn
\begin{equation}
\forall x_1,x_2\in A\,[f(x_1)=f(x_2)\implies x_1=x_2]
\end{equation}
gilt.
\end{definition}

\begin{definition}[Linksinverse]\mbox{}\newline
Sei $f\colon A\to B$. Eine Funktion $g\colon B\to A$ mit
\begin{equation}
g\circ f = \id_A
\end{equation}
heißt \emdef{Linksinverse} von $f$.
\end{definition}
Eine Funktion ist genau dann injektiv, wenn sie eine Linksinverse
besitzt. Zu einer Injektion kann es aber mehrere unterschiedliche
Linksinverse geben.

\subsection{Surjektionen}\index{surjektiv}
\begin{definition}[Surjektion]\mbox{}\newline
Eine Funktion $f\colon A\to B$ heißt \emdef{surjektiv},\\
wenn $f(A)=B$ ist. Damit ist gemeint, dass jedes Element
der Zielmenge wenigstens einmal der Funktionswert von einem
Element der Definitionsmenge ist.
\end{definition}

\begin{definition}[Rechtsinverse]\mbox{}\newline
Sei $f\colon A\to B$. Eine Funktion $g\colon B\to A$ mit
\begin{equation}
f\circ g = \id_B
\end{equation}
heißt \emdef{Rechtsinverse} von $f$.
\end{definition}
Eine Funktion ist genau dann surjektiv, wenn sie eine Rechtsinverse
besitzt. Zu einer Surjektion kann es aber mehrere unterschiedliche
Rechtsinverse geben.

\subsection{Bijektionen}\index{bijektiv}
\begin{definition}[Bijektion]\mbox{}\newline
Eine Funktion $f\colon A\to B$ heißt \emdef{bijektiv},
wenn sie injektiv und surjektiv ist.

Eine Funktion $f\colon A\to B$ ist genau dann bijektiv, wenn es
ein $g$ mit
\begin{equation}
g\circ f = \id_A\quad\text{und}\quad f\circ g = \id_B
\end{equation}
gibt. Wenn $f$ bijektiv ist, so gibt es $g$ genau einmal und
$g$ wird die \emph{Umkehrfunktion}\index{Umkehrfunktion}
oder \emph{Inverse}
von $f$ genannt und als $f^{-1}$ notiert.
\end{definition}

\newpage
\subsection{Komposition}\index{Komposition}
\begin{definition}[Komposition]\mbox{}\newline
Für zwei Funktionen $f\colon A\to B$
und $g\colon B\to C$ ist die \emdef{Komposition}
($g$ nach $f$)
durch
\begin{equation}\label{eq:composition}
g\circ f\colon A\to C,\quad (g\circ f)(x) := g(f(x))
\end{equation}
definiert.
\end{definition}
Für die Komposition gilt das Assozativgesetz:
\begin{equation}
(f\circ g)\circ h = f\circ(g\circ h).
\end{equation}

Die Komposition von Injektionen ist eine Injektion.

Die Komposition von Surjektionen ist eine Surjektion.

Die Komposition von Bijektionen ist eine Bijektion.

Sind $f,g$ Bijektionen, so gilt
\begin{equation}
(g\circ f)^{-1} = f^{-1}\circ g^{-1}.
\end{equation}

Ist $g\circ f$ injektiv, so ist $f$ injektiv.

Ist $g\circ f$ surjektiv, so ist $g$ surjektiv.

Ist $g\circ f$ bijektiv, so ist $f$ injektiv und $g$ surjektiv.

\begin{definition}[Iteration]\mbox{}\newline
Für eine Funktion $\varphi\colon A\to A$ wird
\begin{equation}
\varphi^0:=\operatorname{id}_A,\quad \varphi^{n+1}:=\varphi^n\circ\varphi
\end{equation}
\emdef{Iteration}\index{Iteration} von $\varphi$ genannt.
\end{definition}

\subsection{Einschränkung}\index{Einschränkung}
\begin{definition}[Einschränkung]\mbox{}\newline
Sei $f\colon A\to B$ und $M\subseteq A$.
Die Funktion $g(x)=f(x)$ mit $g\colon M\to B$ wird \emdef{Einschränkung}
von $f$ genannt und mit $f|_M$ notiert.
\end{definition}
Sei $f\colon A\to B$ und $M\subseteq A$.
Mit der Inklusionsabbildung $i(x):=x$ mit $i\colon M\to A$ gilt:
\begin{equation}
f|_M = f\circ i.
\end{equation}
Es gilt
\begin{equation}
g\circ (f|_M) = (g\circ f)|_M.
\end{equation}

\newpage
\subsection{Bild}\index{Bild}
\begin{definition}[Bild]\mbox{}\newline
Für $f\colon A\to B$ und $M\subseteq A$ wird
\begin{equation}
\begin{split}
&f(M) := \{f(x)\mid x\in M\}\\
& = \{y\mid \exists x(x\in M\land y=f(x))\}
\end{split}
\end{equation}
das \emdef{Bild} von $M$ unter $f$ genannt. Genauer:
\end{definition}
Es gilt
\begin{align}
&f(M\cup N) = f(M)\cup f(N),\\
&f(M\cap N) \subseteq f(M)\cap f(N),\\
&f\Big(\bigcup_{i\in I}M_i\Big) = \bigcup_{i\in I} f(M_i),\\
&I\ne\emptyset\implies f\Big(\bigcap_{i\in I} M_i\Big) \subseteq \bigcap_{i\in I} f(M_i),\\
&M\subseteq N\implies f(M)\subseteq f(N),\\
&f(\emptyset) = \emptyset,\\
&(g\circ f)(M) = g(f(M)),\\
&f(M) = \bigcup_{x\in M} \{f(x)\}.
\end{align}
Ist $f$ injektiv, dann gilt auch
\begin{gather}
f(M\cap N) = f(M)\cap f(N),\\
f(M\setminus N) = f(M)\setminus f(N).
\end{gather}

\subsection{Urbild}\index{Urbild}
\begin{definition}[Urbild]\mbox{}\newline
Für $f\colon A\to B$ wird
\begin{equation}
f^{-1}(M) := \{x\in A\mid f(x)\in M\}
\end{equation}
das \emdef{Urbild} von $M$ unter $f$ genannt.
\end{definition}
Es gilt
\begin{align}
& f^{-1}(M\cup N) = f^{-1}(M)\cup f^{-1}(N),\\
& f^{-1}(M\cap N) = f^{-1}(M)\cap f^{-1}(N),\\
& f^{-1}\Big(\bigcup_{i\in I}M_i\Big) = \bigcup_{i\in I} f^{-1}(M_i),\\
& I\ne\emptyset\implies f^{-1}\Big(\bigcap_{i\in I} M_i\Big) = \bigcap_{i\in I}f^{-1}(M_i),\\
& M\subseteq N\implies f^{-1}(M)\subseteq f^{-1}(N),\\
& f^{-1}(\emptyset) = \emptyset,\\
& f^{-1}(B) = A,\\
& f^{-1}(M\setminus N) = f^{-1}(M)\setminus f^{-1}(N),\\
& f^{-1}(B\setminus M) = B\setminus f^{-1}(M),\\
& (g\circ f)^{-1}(M) = f^{-1}(g^{-1}(M)),\\
& (f|_M)^{-1}(N) = M\cap f^{-1}(N),\\
& f(f^{-1}(N)\subseteq N.
\end{align}
Ist $N\subseteq f(A)$ und $f\colon A\to B$, dann gilt
\begin{equation}
f(f^{-1}(N)) = N.
\end{equation}


\newpage
\section{Kardinalzahlen}
\subsection{Definitionen zur Mächtigkeit}

\begin{definition}[Gleichmächtigkeit]\mbox{}\newline
Zwei Mengen $A,B$ heißen \emdef{gleichmächtig}, notiert als
$|A|=|B|$, wenn es eine bijektive Abbildung $f\colon A\to B$ gibt.
\end{definition}
Gleichmächtigkeit ist eine Äquivalenzrelation.
\begin{definition}[Kardinalzahl]
Die Äquivalenzklassen
\begin{equation}
|M| := \{A\mid A\text{ ist gleichmächtig zu }M\}
\end{equation}
heißen \emdef{Kardinalzahlen}.
\end{definition}

\begin{definition}[Höchstens gleichmächtig]\mbox{}\newline
Eine Menge $A$ heißt \emdef{höchstens gleichmächtig} zu $B$,
notiert als $|A|\le|B|$, wenn es eine injektive Abbildung
$f\colon A\to B$ gibt.
\end{definition}

\begin{definition}[Weniger mächtig]\mbox{}\newline
Eine Menge $A$ heißt \emdef{weniger mächtig} als $B$,
notiert als $|A|<|B|$, wenn es eine injektive Abbildung
$f\colon A\to B$ gibt, aber keine bijektive Abbildung
$g\colon A\to B$ existiert.
\end{definition}

\begin{definition}[Abzählbar unendlich]\mbox{}\newline
Eine Menge heißt \emdef{abzählbar unendlich}, wenn sie gleichmächtig
zu den natürlichen Zahlen ist.
\end{definition}

\begin{definition}[Höchstens abzählbar]\mbox{}\newline
Eine Menge heißt \emdef{höchstens abzählbar}, wenn sie höchstens
gleichmächtig zu den natürlichen Zahlen ist.
\end{definition}

\begin{definition}[Überabzählbar]\mbox{}\newline
Eine Menge heißt \emdef{überabzählbar}, wenn die Menge der
natürlichen Zahlen weniger mächtig als diese Menge ist.
\end{definition}

\begin{definition}[Endliche Menge]\mbox{}\newline
Eine Menge heißt \emdef{endlich}, wenn sie weniger mächtig
als die Menge der natürlichen Zahlen ist.
\end{definition}

\subsection{Sätze zur Mächtigkeit}
\begin{Satz}[Satz von Cantor]\mbox{}\\*
Jede Menge ist weniger mächtig als ihre Potenzmenge:
\begin{equation}
|M|<|2^M|.
\end{equation}
Ist $M$ endlich, dann gilt $|M|=2^{|M|}$.
\end{Satz}

\begin{Satz}[Satz von Cantor-Bernstein]\mbox{}\\*
Aus $|A|\le |B|$ und $|B|\le |A|$ folgt $|A|=|B|$.
\end{Satz}

\noindent
\strong{Totalordnung der Kardinalzahlen.}
Die Kardinalzahlen sind total geordnet, da die folgenden Axiome
erfüllt sind.

\strong{Reflexivität.} Es gilt:
\begin{equation}
|A|\le |A|.
\end{equation}

\strong{Antisymmetrie (Satz von Cantor-Bernstein).}\\
\indent Es gilt:
\begin{equation}
|A|\le |B| \land |B|\le |A|\implies |A|=|B|.
\end{equation}

\strong{Transitivität.} Es gilt:
\begin{equation}
|A|\le |B| \land |B|\le |C|\implies |A|\le |C|.
\end{equation}

\strong{Totalität (Vergleichbarkeitssatz).} Es gilt:
\begin{equation}
|A|\le |B| \lor |B|\le |A|.
\end{equation}

\newpage
\noindent
\strong{Weitere Regeln.}

\noindent
Es gilt:
\begin{equation}
A\subseteq B \implies |A|\le |B|.
\end{equation}
Wenn es surjektive Abbildung $g\colon A\to B$ gibt,
dann ist $B$ höchstens gleichmächtig zu $A$:
\begin{equation}
\exists g{\in}B^A (B\subseteq g(A))\implies |B|\le |A|.
\end{equation}
Aus $|A|=|B|$ folgt immer $|A|\le |B|$, denn jede Bijektion
ist auch injektiv.

Nach Definition gilt:
\begin{align}
|A|<|B| &\iff |A|\le |B|\land |A|\ne |B|,\\
|A|\le |B| &\iff |A|<|B|\lor |A|=|B|.
\end{align}
Nach den Axiomen gilt:
\begin{align}
\neg (|A|\le |B|) &\iff |B|<|A|,\\
\neg (|A|<|B|) &\iff |B|\le |A|.
\end{align}
Die Relation $|A|<|B|$ erfüllt die Axiome einer
strengen Totalordnung.

\subsection{Kardinalzahlarithmetik}
\begin{definition}[Summe von Kardinalzahlen]\mbox{}\newline
Die Summe von zwei Kardinalzahlen ist die Mächtigkeit
der disjunkten Vereinigung der Repräsentanten:%
\begin{equation}
|A|+|B| := |A\sqcup B|.
\end{equation}
\end{definition}

\begin{definition}[Produkt von Kardinalzahlen]\mbox{}\newline
Das Produkt von zwei Kardinalzahlen ist die Mächtigkeit
des kartesischen Produktes der Repräsentanten:%
\begin{equation}
|A|\cdot |B| := |A\times B|.
\end{equation}
\end{definition}

\begin{definition}[Potenz von Kardinalzahlen]\mbox{}\newline
Die Potenz von zwei Kardinalzahlen ist die Mächtigkeit
der Menge der Abbildungen von einem Exponent-Repräsentant
zu einem Basis-Repräsentant:%
\begin{equation}
|B|^{|A|} := |B^A|.
\end{equation}
\end{definition}
Für Kardinalzahlen $a,b,c$ gilt:
\begin{align}
a+b &= b+a,\\
ab &= ba,\\
(a+b)+c &= a+(b+c),\\
(ab)c &= a(bc),\\
a(b+c) &= ab+ac,\\
(a+b)c &= ac+bc.
\end{align}
Sind $a,b,c$ Kardinalzahlen und ist $a\le b$, dann gilt:
\begin{align}
a+c &\le b+c,\\
ac &\le bc,\\
a^c &\le b^c.
\end{align}
Wenn zusätzlich $b\ne 0$ oder $a\ne 0$ ist, dann gilt auch
\begin{equation}
c^a \le c^b.
\end{equation}
Für Kardinalzahlen $a,b,c$ gilt:
\begin{align}
a^{b+c} &= a^b a^c,\\
(ab)^c &= a^c b^c,\\
(a^b)^c &= a^{bc}.
\end{align}
Für eine unendliche Kardinalzahl $a$ gilt:
\begin{gather}
a+a = aa = a,\\
|\N|\cdot a = a.
\end{gather}
Für Kardinalzahlen $a,b$ mit $|\N|\le\max(a,b)$ gilt:
\begin{equation}
a+b = \max(a,b).
\end{equation}
Ist zusätzlich $a\ne 0$ und $b\ne 0$, dann gilt auch:
\begin{equation}
ab = \max(a,b).
\end{equation}
Für eine unendliche Menge $A$ und $U\subseteq A$ mit $|U|<|A|$ gilt:
\begin{equation}
|A\setminus U| = |A|.
\end{equation}
Sind $a,b$ Kardinalzahlen mit $2\le b\le a$ und $|\N|\le a$,\\
dann gilt:
\begin{equation}
b^a = 2^a.
\end{equation}
\strong{Spezielle Kardinalitäten.}
Es gilt
\begin{equation}
|\mathbb P| = |\N| = |\Z| = |\Q| = |\mathbb A| < |\R| = |\C|,
\end{equation}
wobei mit $\mathbb P$ die Menge der Primzahlen und mit $\mathbb A$ die
Menge der algebraischen Zahlen gemeint ist. Es gilt
\begin{equation}
|\R^\R| = n^{|\R|} = |2^\R|.\quad (n\in\N, n\ge 2)
\end{equation}
Es gilt
\begin{equation}
|\R^\N| = |\R^n| = |\R|.\quad (n\in\N, n\ge 1)
\end{equation}

\newpage
\section{Formale Systeme}
\subsection{Formale Sprachen}
\begin{definition}[Formale Sprache]\mbox{}\newline
Eine \emdef{formale Sprache} $L$ ist eine Teilmenge der kleenschen
Hülle über einer Menge $\Sigma$, kurz $L\subseteq\Sigma^*$.
Die Menge $\Sigma$ wird \emdef{Alphabet} genannt,
ihre Elemente heißen \emdef{Symbole}.

Die kleensche Hülle $\Sigma^*$ besteht aus allen möglichen
Konkatenationen von Symbolen aus $\Sigma$. Die Konkatenationen
von $\Sigma^*$ heißen \emdef{Wörter}. Die leere Konkatenation ist
zulässig und wird mit $\varepsilon$ notiert. Die Elemente von $L$ heißen
\emdef{wohlgeformte Wörter} oder \emdef{wohlgeformte Formeln},
engl. \emdef{well formed formulas}, kurz \emdef{wff}.
\end{definition}

\noindent
Ein Wort $a$ ist ein Tupel
\begin{equation}
a = (a_1,\ldots, a_m).\qquad (a_k\in\Sigma)
\end{equation}
Sind $a,b$ zwei Wörter, dann ist mit $ab$ deren Konkatenation
gemeint:
\begin{equation}
ab := (a_1,\ldots,a_m,b_1,\ldots b_n).
\end{equation}
Es gilt $\varepsilon a=a$ und $a\varepsilon=a$.
Bei $\varepsilon$ handelt es sich um das leere Tupel.

\begin{definition}[Konkatenation von Sprachen]\mbox{}\newline
\emdef{Konkatenation} von $L_1$ und $L_2$:
\begin{equation}
L_1\circ L_2 := \{ab\mid a\in L_1, b\in L_2\}.
\end{equation}
\end{definition}

\begin{definition}[Potenz einer Sprache]\mbox{}\newline
\emdef{Potenzen} von $L$:
\begin{align}
L^0 &:= \{\varepsilon\},\\
L^n &:= L^{n-1}\circ L.
\end{align}
\end{definition}

\begin{definition}[Kleensche Hülle einer Sprache]\mbox{}\newline
\emdef{Kleensche Hülle} von $L$:
\begin{equation}
L^* := \bigcup_{k\in\N_0} L^k.
\end{equation}

\emdef{Positive Hülle} von $L$:
\begin{equation}
L^+ := \bigcup_{k\in\N_1} L^k.
\end{equation}
\end{definition}

% \newpage
\subsection{Formale Grammatiken}
\begin{definition}[Formale Grammatik]\mbox{}\newline
Eine \emdef{formale Grammatik} ist ein Tupel $(N,\Sigma,P,S)$,
wobei $N$ die \emdef{Nonterminalsymbolen}\index{Nonterminalsymbol},
$\Sigma$ die \emdef{Terminalsymbolen}\index{Terminalsymbol},
$P$ die \emdef{Produktionsregeln}\index{Produktionsregel} sind
und $S$ ein \emdef{Startsymbol}\index{Startsymbol} ist.
Die Mengen $N,\Sigma,P$ müssen endlich sein. Die Mengen $N$ und
$\Sigma$ müssen disjunkt sein. Bei $\Sigma$ handelt es sich um
ein Alphabet. Das Startsymbol ist ein Element $S\in N$.

Bei $P$ handelt es sich um eine Relation
\begin{equation}\label{eq:einfache-Produktionsregeln}
P\subseteq N\times (N\cup\Sigma)^*
\end{equation}
oder allgemeiner
\begin{equation}
P\subseteq (N\cup\Sigma)^*\setminus\Sigma^*\times (N\cup\Sigma)^*.
\end{equation}
Produktionsregeln werden in der Form $n\to w$ notiert und drücken aus,
dass in jedem Wort das Nonterminalsymbol $n$ durch das Wort $w$ ersetzt
werden darf. Allgemeiner bedeutet $t\to w$, dass ein Teilwort $t$
durch $w$ ersetzt werden darf.

Die Produktionsregeln werden ausgehend vom Startsymbol immer weiter
angewendet bis keine Nonterminalsymbole mehr vorhanden sind.
Die Menge aller möglichen Produktionen bildet
eine formale Sprache $L\subseteq\Sigma^*$.
\end{definition}

\noindent
Für Produktionsregeln der Form \eqref{eq:einfache-Produktionsregeln}
wurde eine Kurznotation geschaffen, die EBNF:

\begin{tabular}{l|l}
\verb|Symbol| & Nonterminalsymbol\\
\verb|"Symbol"| & Terminalsymbol\\
\verb|w1, w2| & $w_1w_2$ (Konkatenation)\\
\verb/n = w1 | w2./ & $n\to w_1,\; n\to w_2$\\
\verb|n = {w}.| & $n\to \varepsilon,\; n\to wn$\\
\verb|n = [w].| & $n\to w,\; n\to wn$
\end{tabular}

\subsection{Formale Systeme}
\begin{definition}[Formales System]\mbox{}\newline
Ein \emdef{formales System} ist ein Tupel $(\Sigma,L,A,R)$, wobei
$\Sigma$ ein Alphabet, $L$ eine formale Sprache über
dem Alphabet, $A$ eine Menge von Axiomen und $R$ eine Menge von
Ableitungsrelationen ist. Die Menge der \emdef{Axiome} ist eine
beliebige Teilmenge von $L$. 
Eine \emdef{Ableitungsrelation} ist eine zwei oder mehrstellige
Relation über $L$, die
\begin{equation}\label{eq:Ableitungsrelation}
a_1,\ldots,a_n\vdash b
\end{equation}
geschrieben wird. Eine wohlgeformte Formel wird $\emdef{Satz}$
genannt, wenn sie ein Axiom ist oder über eine Kette von
Ableitungen aus den Axiomen folgt.
\end{definition}

\subsection{Semantik}
\begin{definition}[Interpretation (Aussagenlogik)]%
\label{def:Interpretation}\mbox{}\newline
Eine \emdef{Interpretation}\index{Interpretation}
$I\colon V\to\{0,1\}$ ist eine Abbildung,
welche jeder logischen Variablen einen Wahrheitswert zuordnet.

Eine \emdef{Interpretation} $I\colon F\to\{0,1\}$ erweitert den
Definitionsbereich einer Interpretation wie folgt auf die
Menge aller wohlgeformten Formeln:
\begin{align}
I(0) &= 0,\\
I(1) &= 1,\\
I(\neg\varphi) &= (\neg I(\varphi)),\\
I(\varphi\land\psi) &= (I(\varphi)\land I(\psi)),\\
I(\varphi\lor\psi) &= (I(\varphi)\lor I(\psi)),\\
I(\varphi\rightarrow\psi) &= (I(\varphi)\rightarrow I(\psi)),\\
I(\varphi\leftrightarrow\psi) &= (I(\varphi)\leftrightarrow I(\psi)).
\end{align}
Die rechten Seite der jeweiligen Zeile wird hierbei entsprechend
ihrer Wahrheitstabelle ausgewertet.
\end{definition}

\begin{definition}[Modell]\mbox{}\newline
Eine Interpretation $I$ wird \emdef{Modell}\index{Modell}
von $\varphi$ genannt, wenn $I(\varphi)=1$ ist. Man schreibt
dafür auch $I\models\varphi$.

Eine Interpretation $I$ wird \emdef{Modell} der Formelmenge
$M=\{\varphi_1,\ldots,\varphi_n\}$ genannt, wenn sie für jede
Formel der Menge ein Modell ist:
\begin{equation}
(I\models M) \defiff \forall\varphi{\in}M\;(I\models\varphi).
\end{equation}
\end{definition}

\begin{definition}[Modellrelation]\mbox{}\newline
Sei $M=\{\varphi_1,\ldots,\varphi_n\}$ eine endliche Menge
von Formeln und sei $\psi$ eine Formel. Die Formelmenge $M$
\emph{modelliert}\index{Modellrelation} $\psi$, wenn jedes Modell
von $M$ auch auch ein Modell von $\psi$ ist. Kurz:
\begin{equation}\label{eq:semantische-Implikation}
(M\models\psi) \defiff \forall I[(I\models M)\Rightarrow (I\models\psi)].
\end{equation}
Die Modellrelation wird auch als \emdef{metasprachliche
semantische Implikation} bezeichnet.
\end{definition}


\begin{definition}[Tautologie]\mbox{}\newline
Eine Formel $\varphi$ heißt \emdef{tautologisch}\index{Tautologie},
wenn jede Interpretation auch ein Modell von $\varphi$ ist:
\begin{equation}\label{eq:tautologisch}
(\models\varphi) \defiff \forall I(I(\varphi)=1).
\end{equation}
\end{definition}

\begin{definition}[Äquivalente Formeln]\mbox{}\\*
Zwei Formeln $\varphi,\psi$ heißen äquivalent, wenn
$\varphi\Leftrightarrow\psi$ tautologisch ist, kurz
\begin{equation}\label{eq:aequivalente-Formeln}
(\varphi\equiv\psi)\defiff (\models\varphi\Leftrightarrow\psi).
\end{equation}
\end{definition}

\noindent
Äquivalenz von Formeln ist eine Äquivalenzrelation.

\section{Mathematische Strukturen}\label{sec:Strukturen}
\subsection{Algebraische Strukturen}
\subsubsection*{Axiome}

\noindent\bsf{E:} Abgeschlossenheit.
\ibox{Die Verknüpfung führt nicht aus der Menge heraus.}

\noindent\bsf{A:} Assoziativgesetz.
\ibox{$\forall a,b,c\bright (a*b)*c = a*(b*c)\bleft$.}

\noindent\bsf{N:} Existenz des neutralen Elements.
\ibox{$\exists e\forall a\bright e*a=a*e=a\bleft$.}

\noindent\bsf{I:} Existenz der inversen Elemente.
\ibox{$\forall a\exists b\bright a*b=b*a=e\bleft$.}

\noindent\bsf{K:} Kommutativgesetz.
\ibox{$\forall a,b\bright a*b=b*a\bleft.$}

\noindent
\bsf{I*:} Existenz der multiplikativ inversen Elemente.
\ibox{$\forall a{\ne}0\;\exists b\bright a*b=b*a=1\bleft$.}

\noindent\bsf{Dl:} Linksdistributivgestz.
\ibox{$\forall a,x,y\bright a*(x+y) = a*x+a*y\bleft$.}

\noindent\bsf{Dr:} Rechtsdistributivgesetz.
\ibox{$\forall a,x,y\bright (x+y)*a = x*a+y*a\bleft$.}

\noindent\bsf{D:} Distributivgesetze.
\ibox{Dl und Dr.}

\noindent\bsf{T:} Nullteilerfreiheit.
\ibox{$\forall a,b\bright a\ne 0\land b\ne 0\implies a*b\ne 0\bleft$}
\ibox{bzw. die Kontraposition}
\ibox{$\forall a,b\bright a*b=0\implies a=0\lor b=0\bleft$.}

\noindent\bsf{U:} Unterscheibarkeit von Null- und Einselement.
\ibox{Die neutralen Elemente bezüglich Addition und}
\ibox{Multiplikation sind unterschiedlich.}

\subsubsection*{Strukturen}
Strukturen mit einer inneren Verknüpfung:\\
\begin{tabular}{l|l}
\bsf{EA} & Halbgruppe\\
\bsf{EAN} & Monoid\\
\bsf{EANI} & Gruppe\\
\bsf{EANIK} & abelsche Gruppe
\end{tabular}

\noindent
Strukturen mit zwei inneren Verknüpfungen:\\
\begin{tabular}{l|l}
\bsf{EANIK, EA, D}\dotfill & Ring\\
\bsf{EANIK, EAK, D}\dotfill & kommutativer Ring\\
\bsf{EANIK, EAN, D}\dotfill & unitärer Ring\\
\bsf{EANIK, EANK, DTU} & Integritätsring\\
\bsf{EANIK, EANI*K, DTU} & Körper
\end{tabular}

\newpage
\subsection{Relationen, Ordnungsstrukturen}
\subsubsection*{Axiome für Relationen}

\noindent\bsf{R:} Reflexivität.
\ibox{$\forall a\,(a R a)$.}

\noindent\bsf{S:} Symmetrie.
\ibox{$\forall a,b\,(aRb\iff bRa)$.}

\noindent\bsf{T:} Transitivität.
\ibox{$\forall a,b,c\,(aRb\land bRc\implies aRc)$.}

\noindent\bsf{An:} Antisymmetrie.
\ibox{$\forall a,b\,(aRb\land bRa\implies a=b)$.}

\noindent\bsf{L:} Linearität.
\ibox{$\forall a,b\,(aRb\lor bRa)$.}

\noindent\bsf{Ri:} Irrreflexivität.
\ibox{$\forall a\,(\neg aRa)$.}

\noindent\bsf{A:} Asymmetrie.
\ibox{$\forall a,b\,(aRb\implies \neg bRa)$.}

\noindent\bsf{Min:} Existenz der Minimalelemente.
\ibox{$\forall T{\subseteq}M, T{\ne}\emptyset\;\exists x{\in}T\;\forall y{\in}T{\setminus}\{x\}\,(x<y)$.}

\subsubsection*{Relationen}
\begin{tabular}{l|l}
\bsf{RST}\dotfill & Äquivalenzrelation\\
\bsf{RAnT}\dotfill & Halbordnung\\
\bsf{RAnTL}\dotfill & Totalordnung\\
\bsf{RiAT}\dotfill & strenge Halbordnung\\
\bsf{RiATL}\dotfill & strenge Totalordnung\\
\bsf{RiATLMin} & Wohlordnung
\end{tabular}

\newpage
\section{Zahlenbereiche}

\subsection{Natürliche Zahlen}
\begin{definition}[Natürliche Zahlen (Peano-Axiome)]\mbox{}\newline
Unter den natürlichen Zahlen versteht man eine Menge $N$, die
als dynamisches System $(N,s)$ mit $s(n)=n'$ die folgenden Axiome erfüllt:
\begin{align}
& \text{(P1)}\quad 0\in N,\\
& \text{(P2)}\quad \forall n{\in}N\;(n'\in N),\\
& \text{(P3)}\quad \forall n{\in}N\;(n'\ne 0),\\
& \text{(P4)}\quad \forall m,n{\in}N\;(n'=m'\implies m=n)
\end{align}
und
\begin{equation}
\begin{split}
\text{(P5)}\quad \forall M(&0\in M\land\\
& \forall n{\in}N\;(n\in M\implies n'\in M)\\
& \implies N\subseteq M)
\end{split}
\end{equation}
\end{definition}
Axiom (P2) besagt, dass $s$ eine Selbstabbildung $s\colon N\to N$
ist und Axiom (P4), dass $s$ injektiv ist. Die Axiome (P1) bis (P5)
charakterisieren die Struktur der natürlichen Zahlen, so dass
$N$ als $\N_0$ und $s$ als Nachfolgerfunktion $s(n)=n+1$ interpretiert
werden kann.

Die Addition wir rekursiv definiert:
\begin{equation}
a+0 := a,\qquad a+s(b) := s(a+b).
\end{equation}

Die Multiplikation wird ebenfalls rekursiv definiert:
\begin{equation}
a\cdot 0 := 0,\qquad a\cdot s(b):=a+a\cdot b.
\end{equation}

\noindent
\strong{Von-Neumann-Modell der natürlichen Zahlen.}\\
Eine Menge $M$ heißt \emph{induktiv}, wenn gilt:
\begin{equation}
0\in M\land \forall n(n\in M\implies s(n)\in M).
\end{equation}
Sei $0:=\emptyset$ und $s(n):=n\cup\{n\}$. Sei
\begin{equation}
\N := \bigcap \{M\mid M\text{ ist induktiv}\}.
\end{equation}
Bei $(\N,s,0)$ handelt es sich um ein Modell
natürlichen Zahlen, das die Peano-Axiome (P1) bis (P5) erfüllt.

Es ergibt sich
\begin{align*}
0 &= \emptyset,\\
1 &= \{0\} = \{\emptyset\},\\
2 &= \{0,1\} = \{\emptyset,\{\emptyset\}\},\\
3 &= \{0,1,2\},\\
4 &= \{0,1,2,3\}
\end{align*}
usw.

\newpage
\subsection{Rationale Zahlen}
\begin{definition}[Rationale Zahlen]\mbox{}\newline
Die rationalen Zahlen sind die Quotientenmenge
\begin{equation}
\Q := (\Z\times(\Z{\setminus}\{0\}))/{\sim}
\end{equation}
bezüglich der Äquivalenzrelation
\begin{equation}
(a,b)\sim (c,d) \defiff ad=bc.
\end{equation}
\end{definition}
Wohldefiniert (unabhängig von den Repräsentanten) ist
\begin{align}
[(a,b)]+[(c,d)] &:= [(ad+bc, bd)],\\
[(a,b)]\cdot [(c,d)] &:= [(ac,bd)].
\end{align}
Die Einbettung von $\Z$ in $\Q$ ist gegeben durch
\begin{equation}
\varphi\colon \Z\to\Q,\quad \varphi(z):=[(z,1)].
\end{equation}
Bei $\varphi$ handelt es sich um einen Monomorphismus
bezüglich der grundlegenden Rechenoperationen, so
dass $\Z$ und $\varphi(\Z)$ miteinander
identifiziert werden können.

\subsection{Reelle Zahlen}
\begin{definition}[Reelle Zahlen]\mbox{}\newline
Unter den reellen Zahlen versteht man eine Menge $\R$, die folgende
Axiome erfüllt:
\begin{enumerate}[itemsep=0pt]
\item $(\R,+,\cdot)$ ist ein Körper.
\item $\R$ ist total geordnet, wobei die Ordnungsrelation mit
Addition und Multiplikation verträglich ist.
\item Jede nach oben beschränkte nichtleere Teilmenge von $\R$ hat
  ein Supremum in $\R$.
\end{enumerate}
\end{definition}

\noindent
\strong{Konstruktion der reellen Zahlen.}
Sei $C(\Q)$ die Menge der Cauchyfolgen mit Werten in $\Q$.
Für $x_n,y_n\in C(\Q)$ ist
\begin{equation}
(x_n)\sim (y_n) \defiff x_n-y_n\to 0.
\end{equation}
eine Äquivalenzrelation. Man setzt nun
\begin{equation}
\R := C(\Q)/{\sim}.
\end{equation}
Wohldefiniert (unabhängig von den Repräsentanten) ist
\begin{align}
[(x_n)]+[(y_n)] &:= [(x_n+y_n)],\\
[(x_n)]\cdot [(y_n)] &:= [(x_n y_n)]
\end{align}
und
\begin{equation}
[(x_n)]\le [(y_n)] \defiff
(x_n){\sim}(y_n)\lor \exists n_0\;\forall n{>}n_0\;(x_n\le y_n).
\end{equation}

