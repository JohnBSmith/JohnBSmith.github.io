
\chapter{Kombinatorik}
\section{Kombinatorische Funktionen}
\subsection{Faktorielle}\index{Faktorielle}
\subsubsection{Fakultät}\index{Fakultät}
\begin{definition}[Fakultät]\mbox{}\newline
Für $n\in\Z, n\ge 0$:
\begin{equation}
n! := \prod_{k=1}^n k.
\end{equation}
\end{definition}
\noindent
Rekursionsgleichung:
\begin{equation}
(n+1)! = n!\,(n+1)
\end{equation}
Die Gammafunktion ist eine Verallgemeinerung der Fakultät:
\begin{equation}
n! = \Gamma(n+1).
\end{equation}

\subsubsection{Fallende Faktorielle}
\begin{definition}[Fallende Faktorielle]\mbox{}\newline
Für $a\in\C$ und $k\ge 0$:
\begin{equation}\label{eq:FF}
a^{\underline k} := \prod_{j=0}^{k-1} (a-j).
\end{equation}
Für $a,k\in\C$:
\begin{equation}
a^{\underline k} := \lim_{x\to a}\frac{\Gamma(x+1)}{\Gamma(x-k+1)}.
\end{equation}
\end{definition}
\noindent
Für $n\ge k$ und $k\ge 0$ gilt:
\begin{equation}
n^{\underline k} = \frac{n!}{(n-k)!}.
\end{equation}

\subsubsection{Steigende Faktorielle}
\begin{definition}[Steigende Faktorielle]\mbox{}\newline
Für $a\in\C$ und $k\ge 0$:
\begin{equation}
a^{\overline k} := \prod_{j=0}^{k-1} (a+j).
\end{equation}
Für $a,k\in\C$:
\begin{equation}
a^{\overline k} := \lim_{x\to a}\frac{\Gamma(x+k)}{\Gamma(x)}.
\end{equation}
\end{definition}
\noindent
Für $n\ge 1$ und $n+k\ge 1$ gilt:
\begin{equation}
n^{\overline k} = \frac{(n+k-1)!}{(n-1)!}.
\end{equation}

\subsection{Binomialkoeffizienten}\index{Binomialkoeffizient}
\begin{definition}[Binomialkoeffizient]\mbox{}\newline
Für $a\in\C$ und $k\in\Z$:
\begin{equation}
\binom{a}{k} := \begin{cases}
\frac{a^{\underline k}}{k!} & \text{wenn}\;k>0,\\
1 & \text{wenn}\;k=0,\\
0 & \text{wenn}\;k<0.
\end{cases}
\end{equation}
Für $a,b\in\C$:
\begin{equation}\label{eq:bc-allg}
\binom{a}{b} := \lim_{x\to a}\lim_{y\to b}
\frac{\Gamma(x+1)}{\Gamma(y+1)\Gamma(x-y+1)}.
\end{equation}
\end{definition}
\noindent
Für $0\le k\le n$ gilt die Symmetriebeziehung
\begin{equation}
\binom{n}{k} = \binom{n}{n-k}
\end{equation}
und die Rekursionsgleichung
\begin{equation}
\binom{n+1}{k+1} = \binom{n}{k+1}+\binom{n}{k}.
\end{equation}
Für $a\in\C$ und $k\in\Z$ gilt:
\begin{equation}
\binom{-a}{k} = (-1)^k \binom{a+k-1}{k}.
\end{equation}

\section{Differenzenrechnung}
\begin{definition}[Differenzoperator]\mbox{}\newline
\emdef{Vorwärtsdifferenz}:
\begin{align}
(\Delta f)(x) &:= f(x+1)-f(x),\\
(\Delta_h f)(x) &:= f(x+h)-f(x).
\end{align}
\emdef{Rückwärtsdifferenz}:
\begin{equation}
(\nabla_h f)(x) := f(x)-f(x-h).
\end{equation}
\end{definition}
\noindent
Für $n\in\N_0$ und $x\in\C$ gilt:
\begin{equation}
\Delta(x^{\underline{n}}) = nx^{\underline{n-1}}.
\end{equation}
Die Formel gilt auch für $n\in\C$, dann aber\\
$x\in\C\setminus\{k\in\Z\mid k<0\}$, da auf dem Streifen unter
Umständen Polstellen sind.

Für $n\in\Z,n\ge 0$ gilt:
\begin{equation}
\sum_{x=a}^{b-1} x^{\underline n} = \frac{1}{n+1}\big[x^{\underline{n+1}}\big]_{x=a}^{x=b}.
\end{equation}
Die Formel gilt auch für $a,b\ge 0$ und $n\in\C\setminus\{-1\}$.

Für $a>0$ und $x\in\C$ gilt:
\begin{equation}
\Delta(a^x) = (a-1)\,a^x.
\end{equation}

\section{Endliche Summen}
Summe der Dreieckszahlen:
\begin{align}
\sum_{k=1}^n k &= \frac{n}{2}(n+1),\\
\sum_{k=m}^n k &= \frac{1}{2}(n-m+1)(n+m).
\end{align}
Partialsumme der geometrischen Reihe:
\begin{gather}
\sum_{k=m}^{n-1} q^k = \frac{q^n-q^m}{q-1},\qquad (q\ne 1)\\
\sum_{k=m}^{n-1} k^p q^k
= \Big(q\frac{\mathrm d}{\mathrm dq}\Big)^p\,\frac{q^n-q^m}{q-1}.\quad (q\ne 1)
\end{gather}
Verallgemeinerung von $a^2-b^2=(a-b)(a+b)$:
\begin{gather}
a^n-b^n = (a-b)\sum_{k=0}^{n-1}a^{n-1-k}b^k.
\end{gather}

\newpage
\section{Formale Potenzreihen}
\subsection{Ring der formalen Potenzreihen}
\begin{definition}[Formale Potenzreihe]\mbox{}\newline
Ein Ausdruck der Form
\begin{equation}
\sum_{k=0}^\infty a_k X^k := (a_k)_{k=0}^\infty = (a_0,a_1,a_2,\ldots)
\end{equation}
heißt \emdef{formale Potenzreihe}. Mit $R[[X]]$ wird die Menge
der formalen Potenzreihen in der Variablen $X$ mit Koeffizienten
$a_k\in R$ bezeichnet, wobei $R$ ein kommutativer Ring
mit Einselement ist.
\end{definition}
Die Menge $R[[X]]$ bildet bezüglich der Addition
\begin{equation}
\sum_{k=0}^\infty a_k X^k+\sum_{k=0}^\infty b_k X^k
:= \sum_{k=0}^\infty (a_k+b_k)X^k
\end{equation}
und der Multiplikation
\begin{equation}
\bigg(\sum_{i=0}^\infty a_i X^i\bigg)\bigg(\sum_{j=0}^\infty b_j X^j\bigg)
:= \sum_{k=0}^\infty \bigg(\sum_{i=0}^{k} a_ib_{k-i}\bigg)X^k
\end{equation}
einen kommutativen Ring.

\strong{Koeffizientenvergleich.} Weil formale Potenzreihen
Folgen entsprechen, sind sie genau dann gleich, wenn sie
komponentenweise gleich sind:
\begin{equation}
\sum_{k=0}^\infty a_k X^k = \sum_{k=0}^\infty b_k X^k
\iff \forall k\,(a_k=b_k).
\end{equation}

\strong{Division.}
Eine formale Potenzreihe $B$ besitzt höchstens eine Inverse $B^{-1}$,
so dass $BB^{-1}=1$ gilt. Da der Ring kommutativ ist, darf die
Division
\begin{equation}
\frac{A}{B} := AB^{-1} = B^{-1}A
\end{equation}
definiert werden, falls $B$ invertierbar ist.

\subsection{Binomische Reihe}
\begin{definition}[Binomische Reihe]\mbox{}\newline
Für $a\in\C$:
\begin{equation}
(1+X)^a := \sum_{k=0}^\infty \binom{a}{k} X^k
\end{equation}
\end{definition}
\noindent
Es gilt:
\begin{equation}
(1+X)^{a+b} = (1+X)^a (1+X)^b 
\end{equation}
und
\begin{equation}
(1+X)^{ab} = ((1+X)^a)^b.
\end{equation}
