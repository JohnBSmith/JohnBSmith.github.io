
\chapter{Funktionentheorie}
\section{Holomorphe Funktionen}

\begin{definition}[Holomorphe Funktion]
Sei $U\subseteq\C$ eine offene Menge und $f\colon U\to\C$.
Die Funktion $f$ wird \emdef{holomorph}\index{holomorph} an der
Stelle $z_0\in U$ genannt, wenn der Grenzwert
\begin{equation}
f'(z_0) := \lim_{z\to z_0} \frac{f(z)-f(z_0)}{z-z_0}
\end{equation}
existiert.
\end{definition}

\noindent
Das Argument und Bild von $f$ werden nun in Real- und Imaginärteil
zerlegt. Das sind die Zerlegungen $z=x+y\ui$ und $f(z)=u(x,y)+v(x,y)\ui$.
Die Funktion $f(z)$ ist genau dann holomorph an der Stelle
$z_0=x_0+y_0\ui$, wenn bei $(x_0,y_0)$ die partiellen Ableitungen
stetig sind und die \emdef{Cauchy-Riemann-Gleichungen}
\begin{equation}\label{eq:Cauchy-Riemann-Gleichungen}
\frac{\partial u}{\partial x}=\frac{\partial v}{\partial y},
\quad \frac{\partial u}{\partial y}=-\frac{\partial v}{\partial x}
\qquad\text{bei}\;(x_0,y_0)
\end{equation}
gelten. Bei
\begin{equation}
\uv v := (u,-v) = (v_x,v_y) = v_x\uv e_x+v_y\uv e_y
\end{equation}
handelt es sich um ein Vektorfeld
auf dem Koordinatenraum. Die Gleichungen
\eqref{eq:Cauchy-Riemann-Gleichungen} lassen sich nun als
Quellenfreiheit
\begin{equation}
0=\langle\nabla,\uv v\rangle = \frac{\partial v_x}{\partial x}+\frac{\partial v_y}{\partial y}
\end{equation}
und Rotationsfreiheit
\begin{equation}
0=\nabla\wedge \uv v = \bigg(\frac{\partial v_y}{\partial x}
-\frac{\partial v_x}{\partial y}\bigg)\,\uv e_x\wedge\uv e_y
\end{equation}
interpretieren.

Für das totale Differential
\begin{equation}
\mathrm df = \frac{\partial f}{\partial x}\mathrm dx+\frac{\partial f}{\partial y}\mathrm dy
\end{equation}
gibt es die Umformulierung
\begin{equation}\label{eq:Differential-Wirtinger-Operatoren}
\mathrm df = \frac{\partial f}{\partial z}\mathrm dz+\frac{\partial f}{\partial\overline z}\mathrm d\overline z.
\end{equation}
Hierbei ist $\mathrm dz=\mathrm dx+\ui\,\mathrm dy$ und $\mathrm d\overline{z}=\mathrm dx-\ui\,\mathrm dy$.

Die Ableitungsoperatoren
\begin{align}
\frac{\partial f}{\partial z}
&:= \frac{1}{2}\bigg(\frac{\partial f}{\partial x}-\ui\frac{\partial f}{\partial y}\bigg),\\
\frac{\partial f}{\partial\overline z}
&:= \frac{1}{2}\bigg(\frac{\partial f}{\partial x}+\ui\frac{\partial f}{\partial y}\bigg)
\end{align}
mit $\partial f=\partial u+\ui\,\partial v$ heißen \emdef{Wirtinger-Operatoren}.

Die Gleichungen \eqref{eq:Cauchy-Riemann-Gleichungen} lassen sich nun
zur Gleichung%
\begin{equation}\label{eq:Cauchy-Riemann-Wirtinger}
\frac{\partial f}{\partial\overline z}(z_0)=0
\end{equation}
zusammenfassen. Für holomorphe Funktionen reduziert sich das
Differential \eqref{eq:Differential-Wirtinger-Operatoren} wegen
\eqref{eq:Cauchy-Riemann-Wirtinger} auf die Form%
\begin{equation}
\mathrm df = \frac{\partial f}{\partial z}\mathrm dz.
\end{equation}

\newpage
\section{Harmonische Funktionen}
\begin{definition}[Harmonische Funktion]
Sei $U\subseteq\R^2$ eine offene Menge.
Eine Funktion $\Phi\colon U\to\R$ heißt \emdef{harmonisch}
an der Stelle $(x_0,y_0)$, wenn die \emdef{Laplace-Gleichung}
$(\Delta\Phi)(x_0,y_0)=0$ mit dem \emdef{Laplace-Operator}%
\begin{equation}
\Delta\Phi := \frac{\partial^2\Phi}{\partial x\partial x}+\frac{\partial^2\Phi}{\partial y\partial y}
\end{equation}
erfüllt ist.
\end{definition}

\noindent
Ist $f=u+v\ui$ an der Stelle $z_0$ holomorph, so sind der
Realteil $u$ und der Imaginärteil $v$
an der Stelle $(x_0,y_0)=(\real z_0,\imag z_0)$ harmonisch.
Das heißt es gilt%
\begin{equation}
(\Delta u)(x_0,y_0) = 0,\quad (\Delta v)(x_0,y_0)=0.
\end{equation}
Ist eine Funktion $u$ auf einem einfach zusammenhängenden Gebiet
harmonisch, so lässt sich stets eine harmonische Funktion $v$
finden, so dass $f=u+v\ui$ holomorph ist. Die Funktion $v$ ist
bis auf eine additive reelle Konstante $c$ eindeutig bestimmt.
Das heißt, $v$ darf auch durch $v+c$ ersetzt werden.

Die Funktion $v$ wird die \emdef{harmonisch Konjugierte}
zu $u$ genannt. An jeder Stelle $(x_0,y_0)$ treffen die Linien%
\begin{align}
&\{(x,y)\mid u(x,y)=u(x_0,y_0)\},\\
&\{(x,y)\mid v(x,y)=v(x_0,y_0)\} 
\end{align}
senkrecht aufeinander.

\section{Wegintegrale}
\strong{Integral einer komplexwertigen Funktion.}\\
Für $f\colon [a,b]\to\C$ mit $f=u+\ui v$ ist
\begin{equation}
\int_a^b f(t)\,\mathrm dt
= \int_a^b u(t)\,\mathrm dt+\ui\int_a^b v(t)\,\mathrm dt,
\end{equation}
wenn $u$ und $v$ integrierbar sind.

\begin{definition}[Kurvenintegral]
Für $f\colon U\to\C$ mit $U\subseteq\C$:%
\begin{equation}
\int_\gamma f(z)\,\mathrm dz := \int_a^b f(\gamma(t))\,\gamma'(t)\,\mathrm dt,
\end{equation}
wobei $\gamma\colon [a,b]\to U$ ein (zumindest stückweise)
differenzierbarer Weg \eqref{eq:Parameterkurve} ist.
\end{definition}

\noindent
\strong{Integralsatz von Cauchy.}
Ist $U$ ein einfach zusammenhängendes Gebiet und $f\colon U\to\C$
holomorph, so gilt für jeden Weg $\gamma$ von $\gamma(a)$ nach
$\gamma(b)$ die Formel%
\begin{equation}
\int_\gamma f(z)\,\mathrm dz = F(\gamma(b))-F(\gamma(a)),
\end{equation}
wobei die Funktion $F$ nicht vom gewählten Weg abhängig ist.
Außerdem ist $F$ eine Stammfunktion zu $f$, das heißt es gilt
$F'(z)=f(z)$ für alle $z\in U$.

Sind die Voraussetzungen für den Integralsatz erfüllt,
dann motiviert Wegunabhängigkeit die Definition%
\begin{equation}
\int_{z_1}^{z_2} f(z)\,\mathrm dz := F(z_2)-F(z_1),
\end{equation}
bei der auf Wege gänzlich verzichtet wird.

