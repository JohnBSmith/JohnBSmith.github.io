
\subsection{Umgebungen}\index{Umgebung}
Sei $(X,T)$ ein topologischer Raum und $x\in X$.
\begin{Definition}
\emdef{Umgebungsfilter}:\index{Umgebungsfilter}
\begin{equation}
\mathfrak U(x) := \{U\subseteq X\mid x\in O\land O\in T
\land O\subseteq U\}.
\end{equation}
Ein $U\in\mathfrak U(x)$ wird Umgebung von $x$ genannt.
\end{Definition}
\begin{Definition}
Eine Menge $\mathfrak B(x)\subseteq \mathfrak U(x)$
heißt \emdef{Umgebungsbasis} gdw.
\begin{equation}
\forall U{\in}\mathfrak U(x)\,\exists B{\in}\mathfrak B(x)\colon
B\subseteq U.
\end{equation}
\end{Definition}

\noindent
Sei $(X,d)$ ein metrischer Raum und $x\in X$.
\begin{Definition}
$\varepsilon$-\emdef{Umgebung}:
\begin{equation}\label{eq:epsilon-Umgebung}
U_\varepsilon(x) := \{y\in X\mid d(x,y)<\varepsilon\}.
\end{equation}
\emdef{Punktierte $\varepsilon$-Umgebung}:
\begin{equation}
\dot U_\varepsilon(x) := U_\varepsilon(x)\setminus\{x\}.
\end{equation}
\end{Definition}
\noindent
Bei
\begin{equation}
\mathfrak B(x) = \{U_\varepsilon(x)\mid\varepsilon>0\}
\end{equation}
handelt es sich um eine Umgebungsbasis.

Für einen normierten Raum ist durch $d(x,y):=\|x-y\|$ eine
Metrik gegeben. Speziell für $X=\R$ oder $X=\C$ wird fast immer
$d(x,y):=|x-y|$ verwendet.

