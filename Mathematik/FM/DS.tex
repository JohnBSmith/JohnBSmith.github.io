
\chapter{Dynamische Systeme}
\section{Grundbegriffe}

\begin{Definition}
Ein Tupel $(T,M,\Phi)$ mit $\Phi\colon T\times M\to M$ heißt
\emdef{dynamisches System}\index{dynamisches System},
wenn für alle $t_1,t_2\in T$ und $x\in M$ gilt:
\begin{align}
&\Phi(0,x)=x,\\
&\Phi(t_2,\Phi(t_1,x)) = \Phi(t_1+t_2,x).
\end{align}
Die Menge $T$ heißt \emdef{Zeitraum}.
Ein System mit $T=\N_0$ oder $T=\Z$ heißt \emdef{zeitdiskret},
eines mit $T=\R_0^{+}$ oder $T=\R$ heißt \emdef{zeitkontinuierlich}.
Ein System mit $T=\Z$ oder $T=\R$ heißt \emdef{invertierbar}.

Die Menge $M$ heißt \emdef{Zustandsraum}\index{Zustandsraum},
ihre Elemente werden \emdef{Zustände}\index{Zustand} genannt.
\end{Definition}
Für ein invertierbares System handelt es sich bei $\Phi$
um eine Gruppenaktion (s. \ref{Gruppenaktion})
der additiven Gruppe $(T,+)$.

Die Menge
\begin{equation}
\Phi(T,x) := \{\Phi(t,x)\mid t\in T\}
\end{equation}
heißt \emdef{Orbit} von $x$. S.\,a. \eqref{eq:Orbit}.



