
\chapter{Dynamische Systeme}
\section{Grundbegriffe}

{\definition}
Ein Tupel $(T,M,\Phi)$ mit $\Phi\colon T\times M\to M$ heißt
\emdef{dynamisches System}\index{dynamisches System},
wenn für alle $t_1,t_2\in T$ und $x\in M$ gilt:
\begin{align}
&\Phi(0,x)=x,\\
&\Phi(t_2,\Phi(t_1,x)) = \Phi(t_1+t_2,x).
\end{align}
Die Menge $T$ heißt \emdef{Zeitraum}.
Ein System mit $T=\N_0$ oder $T=\Z$ heißt \emdef{zeitdiskret},
eines mit $T=\R_0^{+}$ oder $T=\R$ heißt \emdef{zeitkontinuierlich}.
Ein System mit $T=\Z$ oder $T=\R$ heißt \emdef{invertierbar}.

Die Menge $M$ heißt \emdef{Zustandsraum}\index{Zustandsraum},
ihre Elemente werden \emdef{Zustände}\index{Zustand} genannt.

Für ein invertierbares System handelt es sich bei $\Phi$
um eine Gruppenaktion (s. \ref{Gruppenaktion})
der additiven Gruppe $(T,+)$.

Die Menge
\begin{equation}
\Phi(T,x) := \{\Phi(t,x)\mid t\in T\}
\end{equation}
heißt \emdef{Orbit}\index{Orbit!unter einem dynamischen System}
von $x$. S.\,a. \eqref{eq:Orbit}.

\section{Iterationen}

{\definition}
Für eine Selbstabbildung $\varphi\colon M\to M$ lassen sich
die \emdef{Iterationen} gemäß
\begin{equation}
\varphi^0:=\id,\quad \varphi^n:=\varphi^{n-1}\circ\varphi
\end{equation}
formulieren. Mit $\id$ ist die identische Abbildung
\begin{equation}
\id\colon M\to M,\quad \id(x):=x
\end{equation}
und mit $g\circ f$ die Komposition \eqref{eq:composition} gemeint.
Für ein bijektives $\varphi$ wird zusätzlich
\begin{equation}
\varphi^{-n}:=(\varphi^{-1})^n
\end{equation}
definiert.

Die Iterationen bilden ein dynamisches System gemäß%
\begin{equation}
\Phi(n,x):=\varphi^n(x),\quad\Phi\colon\N_0\times M\to M.
\end{equation}
Bei einem bijektiven $\varphi$ lässt sich das System zum invertierbaren
System
\begin{equation}
\Phi(n,x):=\varphi^n(x),\quad\Phi\colon\Z\times M\to M
\end{equation}
erweitern.

{\definition}
Für eine Funktion $\varphi\colon A\to A$ wird der Operator
\begin{equation}
C_\varphi (g) := g\circ\varphi,\quad C_\varphi\colon B^A\to B^A
\end{equation}
\emdef{Kompositionsoperator}\index{Kompositionsoperator} genannt

Wenn $B^A$ ein Funktionenraum ist, dann ist der Kompositionsoperator
ein linearer Operator.
