\documentclass[a4paper,10pt,fleqn,twocolumn,twoside,openany]{scrbook}
\usepackage[utf8]{inputenc}
\usepackage[T1]{fontenc}
\usepackage{lmodern}
% \usepackage{ngerman}
\usepackage[ngerman]{babel}
\usepackage{amsmath}
\usepackage{amssymb}
\usepackage{amsthm}
\usepackage{mdframed}

\usepackage{textcomp}
\usepackage{lipsum}
\usepackage{microtype}
\usepackage{enumitem}
\usepackage{graphicx}
% \usepackage{multicol}

\usepackage{libertine}
\usepackage[libertine,cmintegrals]{newtxmath}
\usepackage[scaled=0.80]{DejaVuSans}
\renewcommand{\ttdefault}{lmtt}
% \renewcommand\sfdefault{lmss}

\usepackage{color}
\definecolor{c1}{RGB}{0,40,80}
\definecolor{c2}{RGB}{20,60,100}
\definecolor{c3}{RGB}{80,120,180}
\definecolor{c4}{RGB}{140,0,60}
\definecolor{bc1}{RGB}{236,236,250}
\usepackage[colorlinks=true,linkcolor=c1]{hyperref}
\usepackage{geometry}
\geometry{a4paper,left=25mm,right=10mm,top=20mm,bottom=28mm}
\setlength{\columnsep}{4mm}

\usepackage[toc]{multitoc}
\setcounter{secnumdepth}{4}
\setcounter{tocdepth}{2}
\usepackage{tocloft}
\setlength{\cftsecindent}{0pt}
\setlength{\cftsubsecindent}{23pt}
\setlength{\cftsubsubsecindent}{55pt}
\renewcommand{\cftchapfont}{\normalfont\sffamily\bfseries}
\renewcommand{\cftsecfont}{\normalfont\sffamily}
\renewcommand{\cftsubsecfont}{\normalfont\sffamily}
\renewcommand{\cftsubsubsecfont}{\normalfont\sffamily}
\renewcommand\cftchappagefont{\normalfont\sffamily\bfseries}
\renewcommand\cftsecpagefont{\normalfont\sffamily}
\renewcommand\cftsubsecpagefont{\normalfont\sffamily}
\renewcommand\cftsubsubsecpagefont{\normalfont\sffamily}

\usepackage{titlesec}

% \titleformat{...}[hang]
% is broken in TeX Live Ubuntu 16.04 LTS.
% So a patch is used until an update is available.
% BEGIN PATCH
\usepackage{etoolbox}
\makeatletter
\patchcmd{\ttlh@hang}{\parindent\z@}{\parindent\z@\leavevmode}{}{}
\patchcmd{\ttlh@hang}{\noindent}{}{}{}
\makeatother
% END PATCH

\titleformat{\chapter}[hang]
  {\normalfont\sffamily\huge\bfseries}{\thechapter}{1em}{\Huge}
\titleformat{\section}[hang]
  {\normalfont\sffamily\Large\bfseries}{\thesection}{1em}{\Large}
\titleformat{\subsection}[hang]
  {\normalfont\sffamily\large\bfseries}{\thesubsection}{1em}{\large}
\titleformat{\subsubsection}[hang]
  {\normalfont\sffamily\large\bfseries}{\thesubsubsection}{1em}{\large}

\titlespacing*{\chapter}{0pt}{0pt}{10pt}
\titlespacing*{\section}{0pt}{4pt minus 2pt}{2pt minus 2pt}
\titlespacing*{\subsection}{0pt}{6pt minus 4pt}{2pt minus 2pt}
\titlespacing*{\subsubsection}{0pt}{6pt minus 4pt}{2pt minus 4pt}

\usepackage[justification=RaggedRight,singlelinecheck=off]{caption}

\numberwithin{equation}{chapter}

\renewcommand{\baselinestretch}{0.9}

\newenvironment{ttsection}{\ttfamily}{\par}
\newcommand{\strong}[1]{{\sffamily\bfseries #1}}
\newcommand{\bsf}[1]{{\sffamily\bfseries #1}}
\newcommand{\bitem}{\item[\scriptsize $\blacksquare$]}
\newcommand{\bulletbs}{\text{\scriptsize $\blacksquare$}\;\,}

% \newcommand{\definition}{\strong{Definition.}}
\newcommand{\theorem}[1]{\strong{#1.}}
\newenvironment{Definition}{\par\noindent\strong{Definition.}}{\par}
\newenvironment{Satz}{\par\noindent\strong{Satz.}}{\par}
\newcommand{\minisection}{\vspace{4pt plus 2pt minus 1pt}\par\noindent}

\newtheoremstyle{Definition}
  {0pt}{0pt}
  {}{}
  {\sffamily\bfseries}{\newline}
% {.2em}{\thmname{#1}\thmnumber{~#2}.~\thmnote{#3.}}
  {.2em}{\thmname{#1}.~\thmnote{#3.}}
\theoremstyle{Definition}
\newtheorem{definition}{Definition}[chapter]

%\newenvironment{definition}[1][]%
%  {\par\addvspace{4pt plus 0pt minus 4pt}\noindent\strong{Definition.~#1.}\newline\indent}%
%  {\par\addvspace{4pt plus 0pt minus 4pt}}

\definecolor{greenblue}{rgb}{0.0,0.32,0.4}
\definecolor{grayblue}{rgb}{0.2,0.2,0.4}
\definecolor{lightgray}{rgb}{0.4,0.4,0.4}
\definecolor{b1}{rgb}{0.6,0.6,0.54}

\surroundwithmdframed[topline=false,rightline=false,bottomline=false,%
  linecolor=greenblue, linewidth=3.0pt, innerleftmargin=4pt,%
  innertopmargin=1pt, innerbottommargin=1pt,%
  innerrightmargin=0pt%
]{definition}

\newcommand{\ibox}[1]{\par\hbox{\hsize=22em\hglue 1em\vbox{{\noindent}#1}}}

% parametric strut
\newcommand{\pstrut}[1]{\rule{0pt}{\dimexpr 8pt+#1}}

% emphasis, definierter Begriff
\newcommand{\emdef}[1]{\textit{#1}}
\newcommand{\bdef}[1]{\strong{#1}}

% table header bold font
\newcommand{\thbf}[1]{{\sffamily\bfseries #1}}

% \ui: imaginäre Einheit
% \ue: Einheitsvektor
% \ue: eulersche Zahl
% \uv{x}: unterstrichener Vektor

\newcommand{\ui}{\mathrm i}
\newcommand{\uj}{\mathrm j}
\newcommand{\uk}{\mathrm k}
\newcommand{\ue}{e}
\newcommand{\unit}[1]{\mathrm{#1}}
\newcommand{\ee}{\mathrm e}
\newcommand{\uv}[1]{\underline{#1}}
\newcommand{\bv}[1]{\mathbf{#1}}
\newcommand{\comp}{\textsf{c}}
\newcommand{\bright}{\texttt{[}}
\newcommand{\bleft}{\texttt{]}}

% A cdot that can be made bolder
\newcommand{\bcdot}{\cdot}

\newcommand{\N}{\mathbb N}
\newcommand{\Z}{\mathbb Z}
\newcommand{\Q}{\mathbb Q}
\newcommand{\R}{\mathbb R}
\newcommand{\C}{\mathbb C}
\newcommand{\K}{\mathbb K}

\DeclareMathOperator{\id}{id}
\DeclareMathOperator*{\sgn}{sgn}
\DeclareMathOperator*{\rg}{rg}
\DeclareMathOperator*{\diag}{diag}
\DeclareMathOperator*{\Eig}{Eig}
\DeclareMathOperator{\real}{Re}
\DeclareMathOperator{\imag}{Im}
\DeclareMathOperator{\tr}{tr}
\DeclareMathOperator{\arccot}{arccot}
\DeclareMathOperator{\arsinh}{arsinh}
\DeclareMathOperator{\arcosh}{arcosh}
\DeclareMathOperator{\artanh}{artanh}
\DeclareMathOperator{\arcoth}{arcoth}

\newcommand{\defiff}{\;:\Longleftrightarrow\;}


\usepackage{makeidx}
\renewcommand\indexname{Stichwortverzeichnis}
\makeindex

\begin{document}
% \setlength{\baselineskip}{11.0pt}
\setlength{\abovedisplayskip}{6pt}
\setlength{\belowdisplayskip}{6pt}
\setlength{\abovedisplayshortskip}{6pt}
\setlength{\belowdisplayshortskip}{6pt}
\setlength{\abovecaptionskip}{2pt plus 2pt minus 1pt}

\begin{titlepage}
\centering
\phantom{x}

\vspace{20em}
{\noindent\Huge\sffamily\textbf{Formelsammlung\\
Mathematik}}

\vspace{2em}
{\Large Februar 2018}\\
\end{titlepage}

\thispagestyle{empty}

\noindent
Dieses Buch ist unter der Lizenz\\
Creative Commons CC0 veröffentlicht.
\vspace{8em}

\noindent
\begin{ttsection}
\begin{tabular}{r|r|r|r}
 0 & 0000 & 0 &  0\\
 1 & 0001 & 1 &  1\\
 2 & 0010 & 2 &  2\\
 3 & 0011 & 3 &  3\\
\noalign{\vspace{1em}}
 4 & 0100 & 4 &  4\\
 5 & 0101 & 5 &  5\\
 6 & 0110 & 6 &  6\\
 7 & 0111 & 7 &  7\\
\noalign{\vspace{1em}}
 8 & 1000 & 8 & 10\\
 9 & 1001 & 9 & 11\\
10 & 1010 & A & 12\\
11 & 1011 & B & 13\\
\noalign{\vspace{1em}}
12 & 1100 & C & 14\\
13 & 1101 & D & 15\\
14 & 1110 & E & 16\\
15 & 1111 & F & 17
\end{tabular}
\end{ttsection}

\newpage
\noindent
$\!\begin{aligned}
\sin(-x) &= -\sin x\\
\cos(-x) &= \cos x
\end{aligned}$
\vspace{1em}

\noindent
$\!\begin{aligned}
\sin(x+y) &= \sin x\cos y + \cos x\sin y\\
\sin(x-y) &= \sin x\cos y - \cos x\sin y\\
\cos(x+y) &= \cos x\cos y - \sin x\sin y\\
\cos(x-y) &= \cos x\cos y + \sin x\sin y
\end{aligned}$
\vspace{1em}

\noindent
$\ee^{\ui\varphi}=\cos\varphi+\ui\sin\varphi$
\vspace{2em}

\noindent
\strong{Polarkoordinaten}\\
$x=r\cos\varphi$\\
$y=r\sin\varphi$\\
$\varphi\in(-\pi,\pi]$\\
$\det J=r$
\vspace{1em}

\noindent
\strong{Zylinderkoordinaten}\\
$x=r_{xy}\cos\varphi$\\
$y=r_{xy}\sin\varphi$\\
$z=z$\\
$\det J=r_{xy}$
\vspace{1em}

\noindent
\strong{Kugelkoordinaten}\\
$x=r\sin\theta\,\cos\varphi$\\
$y=r\sin\theta\,\sin\varphi$\\
$z=r\cos\theta$\\
$\varphi\in(-\pi,\pi],\;\theta\in[0,\pi]$\\
$\det J=r^2\sin\theta$
\vspace{1em}

\noindent
$\theta=\beta-\pi/2$\\
$\beta\in[-\pi/2,\pi/2]$\\
$\cos\theta=\sin\beta$\\
$\sin\theta=\cos\beta$

\renewcommand{\contentsname}{\sffamily Inhaltsverzeichnis}
\tableofcontents


\chapter{Grundlagen}
\section{Arithmetik}
\subsection{Zahlenbereiche}
Natürliche Zahlen ab null:
\begin{equation}
\N_0 := \{0,1,2,3,4,\ldots\}.
\end{equation}
Natürliche Zahlen ab eins:
\begin{equation}
\N_1 := \{1,2,3,4,5,\ldots\}.
\end{equation}
Natürliche Zahlen:
\begin{equation}
\begin{split}
&\text{$\N$, wenn es keine Rolle spielt,}\\
&\text{ob $\N:=\N_0$ oder $\N:=\N_1$}.
\end{split}
\end{equation}
Ganze Zahlen:
\begin{equation}
\Z := \{\ldots -3,-2,-1,0,1,2,3,\ldots\}.
\end{equation}
Rationale Zahlen:
\begin{equation}
\Q := \{\tfrac{z}{n}\mid z\in\Z,n\in\N_0\}.
\end{equation}
Reelle Zahlen:
\begin{equation}
\R := \overline{\Q}\enspace\text{bezüglich}\; d(x,y)=|x-y|.
\end{equation}
Positive reelle Zahlen:
\begin{equation}
\R^+ := \{x\in\R\mid x>0\}.
\end{equation}
Nichtnegative reelle Zahlen:
\begin{equation}
\R_0^+ := \{x\in\R\mid x\ge 0\}.
\end{equation}
Negative reelle Zahlen:
\begin{equation}
\R^- := \{x\in\R\mid x<0\}.
\end{equation}
Nichtpositive reelle Zahlen:
\begin{equation}
\R_0^- := \{x\in\R\mid x\le 0\}.
\end{equation}
Komplexe Zahlen:
\begin{equation}
\C := \{a+b\ui\mid a,b\in\R\}.
\end{equation}
Quaternionen:
\begin{equation}
\mathbb H := \{a+b\ui+c\uj+d\uk\mid a,b,c,d\in\R\}.
\end{equation}
Algebraische Zahlen:
\begin{equation}
\mathbb A := \{a\in\C\mid \exists P{\in}\Q[X]\colon P(a)=0\}.
\end{equation}
Irrationale Zahlen:
\begin{equation}
\R\setminus\Q = \{\sqrt{2},\sqrt{3},\pi,\ee,\ldots\}.
\end{equation}
Transzendente Zahlen:
\begin{equation}
\R\setminus\mathbb A = \{\pi,\ee,\ldots\}.
\end{equation}
Es gelten die folgenden Teilmengenbeziehungen:
\begin{equation}
\N\subset\Z\subset\Q\subset\R\subset\C\subset\mathbb H.
\end{equation}
Es gilt die folgende Abstufung der Mächtigkeit:
\begin{equation}
|\N| = |\Z| = |\Q| = |\mathbb A| < |\R| = |\C|.
\end{equation}

\newpage
\subsection{Intervalle}
Abgeschlossene Intervalle:
\begin{equation}
[a,b] := \{x\in\R\mid a\le x\le b\}.
\end{equation}
Offene Intervalle:
\begin{equation}
(a,b) := \{x\in\R\mid a<x<b\}.
\end{equation}
Halboffene Intervalle:
\begin{align}
(a,b] &:= \{x\in\R\mid a<x\le b\},\\
[a,b) &:= \{x\in\R\mid a\le x<b\}.
\end{align}
Unbeschränkte Intervalle:
\begin{align}
[a,\infty) &:= \{x\in\R\mid a\le x\},\\
(a,\infty) &:= \{x\in\R\mid a<x\},\\
(-\infty,b] &:= \{x\in\R\mid x\le b\},\\
(-\infty,b) &:= \{x\in\R\mid x<b\}.
\end{align}

\subsection{Summen}
\begin{Definition} Für eine Folge $(a_n)$:
\begin{align}
\sum_{k=m}^{m-1} a_k &:= 0,\qquad(\text{leere Summe})\\
\sum_{k=m}^n a_k &:= \sum_{k=m}^{n-1} a_k.\qquad(n\ge m)
\end{align}
\end{Definition}
\noindent
Für eine Konstante $c$ gilt:
\begin{equation}
\sum_{k=m}^n c = (n-m+1)\,c.
\end{equation}
Der Summierungsoperator ist linear:
\begin{align}
\sum_{k=m}^n (a_k+b_k) &= \sum_{k=m}^n a_k + \sum_{k=m}^n b_k,\\
\sum_{k=m}^n ca_k &= c\sum_{k=m}^n a_k.
\end{align}
Indexverschiebung ist möglich:
\begin{equation}
\sum_{k=m}^n a_k = \sum_{k=m-j}^{n-j} a_{k+j} = \sum_{k=m+j}^{n+j} a_{k-j}.
\end{equation}
Aufspaltung ist möglich:
\begin{equation}
\sum_{k=m}^n a_k = \sum_{k=m}^p a_k + \sum_{k=p+1}^n a_k.
\end{equation}
Vertauschung der Reihenfolge bei Doppelsummen:
\begin{equation}
\sum_{i=p}^m \sum_{j=q}^n a_{ij} = \sum_{j=q}^n \sum_{i=p}^m a_{ij}.
\end{equation}

\subsection{Binomischer Lehrsatz}\index{binomischer Lehrsatz}
Sei $R$ ein unitärer Ring. 
Für $a,b\in R$ mit $ab=ba$ gilt:%
\begin{equation}
(a+b)^n = \sum_{k=0}^n \binom{n}{k} a^{n-k} b^k
\end{equation}
und
\begin{equation}
(a-b)^n = \sum_{k=0}^n \binom{n}{k} (-1)^k a^{n-k} b^k.
\end{equation}
Die ersten Formeln sind:\index{binomische Formeln}
\begin{gather}
(a+b)^2 = a^2+2ab+b^2,\\
(a-b)^2 = a^2-2ab+b^2,\\
(a+b)^3 = a^3+3a^2 b+3ab^2+b^3,\\
(a-b)^3 = a^3-3a^2 b+3ab^2-b^3,\\
(a+b)^4 = a^4+4a^3 b+6a^2 b^2+4ab^3+b^4,\\
(a-b)^4 = a^4-4a^3 b+6a^2 b^2-4ab^3+b^4.
\end{gather}
\subsection{Potenzgesetze}
\begin{Definition}
Für $a\in\R, a>0$ und $x\in\C$:
\begin{equation}
a^x := \exp(\ln(a)\,x).
\end{equation}
\end{Definition}
\noindent
Für $a\in\R, a>0$ und $x,y\in\C$ gilt:
\begin{gather}
a^{x+y} = a^x a^y,\quad a^{x-y} = \frac{a^x}{a^y},
\quad a^{-x} = \frac{1}{a^x}.
\end{gather}

\section{Komplexe Zahlen}\index{komplexe Zahl}
\subsection{Rechenoperationen}

\begin{gather}
\frac{z_1}{z_2}
= \frac{z_1\overline z_2}{z_2\overline z_2}
= \frac{z_1\overline z_2}{|z_2|^2},\\
\frac{1}{z} = \frac{\overline z}{z\overline z}
= \frac{\overline z}{|z|^2}.
\end{gather}

\subsection{Betrag}\index{Betrag!einer komplexen Zahl}
Für alle $z_1,z_2\in\C$ gilt:
\begin{gather}
|z_1z_2| = |z_1|\,|z_2|,\\
z_2\ne 0\implies \Big|\frac{z_1}{z_2}\Big|
= \frac{|z_1|}{|z_2|},\\
z\,\overline z = |z|^2.
\end{gather}

\subsection{Konjugation}\index{Konjugation!einer komplexen Zahl}
Für alle $z_1,z_2\in\C$ gilt:
\begin{gather}
\overline{z_1+z_2} = \overline z_1+\overline z_2,\qquad
\overline{z_1-z_2} = \overline z_1-\overline z_2,\\
\overline{z_1 z_2} = \overline z_1\,\overline z_2,\qquad
z_2\ne 0 \implies \overline{\Big(\frac{z_1}{z_2}\Big)}
= \frac{\overline z_1}{\overline z_2},\\
\overline{\overline z}=z,\qquad
|\overline{z}| = |z|,\qquad
z\,\overline z = |z|^2,\\
\real(z) = \frac{z+\overline z}{2},\qquad
\imag(z) = \frac{z-\overline z}{2\ui},\\
\overline{\cos(z)} = \cos(\overline z),\qquad
\overline{\sin(z)} = \sin(\overline z),\\
\overline{\exp(z)} = \exp(\overline z).
\end{gather}

\begin{table*}[t]
\caption{Rechenoperationen}
\bgroup
\def\arraystretch{1.4}
\begin{tabular}{|l|r|l|l|}
\hline
  \thbf{Name}
& \thbf{Operation}
& \thbf{Polarform}
& \thbf{kartesische Form}\\
\hline
  Identität
& $z$ & $=r\ee^{\ui\varphi}$
& $= a+b\ui$\\
\hline
  Addition
& $z_1+z_2$ &
& $= (a_1+a_2)+(b_1+b_2)\ui$\\
\hline
  Subtraktion
& $z_1-z_2$ &
& $= (a_1-a_2)+(b_1-b_2)\ui$\\
\hline
  Multiplikation
& $z_1 z_2$
& $= r_1 r_2 \ee^{\ui(\varphi_1+\varphi_2)}$
& $= (a_1 a_2 - b_1 b_2)+(a_1 b_2+a_2 b_1)\ui$\\
\hline
  Division
& $\displaystyle\frac{z_1}{z_2}$
& $\displaystyle =\frac{r_1}{r_2}\ee^{\ui(\varphi_1-\varphi_2)}$
& $\displaystyle =\frac{a_1 a_2 + b_1 b_2}{a_2^2+b_2^2}
   + \frac{a_2 b_1 - a_1 b_2}{a_2^2+b_2^2}\ui$\\
\hline
  Kehrwert
& $\displaystyle\frac{1}{z}$
& $\displaystyle =\frac{1}{r}\ee^{-\ui\varphi}$
& $\displaystyle =\frac{a}{a^2+b^2}-\frac{b}{a^2+b^2}\ui$\\
\hline
  Realteil
& $\real(z)$
& $=\cos\varphi$
& $=a$\\
\hline
  Imaginärteil
& $\imag(z)$
& $=\sin\varphi$
& $=b$\\
\hline
  Konjugation
& $\overline{z}$
& $=r\ee^{-\varphi\ui}$
& $=a-b\ui$\\
\hline
Betrag
& $|z|$
& $=r$
& $=\sqrt{a^2+b^2}$\\
\hline
  Argument
& $\arg(z)$
& $=\varphi$
& $\displaystyle = s(b)\arccos\Big(\frac{a}{r}\Big)$\\
\hline
\end{tabular}
\egroup\\
\\
$s(b):=\begin{cases}
+1 & \text{if}\;b\ge 0,\\
-1 & \text{if}\;b<0
\end{cases}$
\end{table*}

\section{Logik}
\subsection{Aussagenlogik}\index{Aussagenlogik}
\subsubsection{Boolesche Algebra}\index{boolesche Algebra}
\begin{table*}[t]
\caption{Boolesche Algebra}
\begin{tabular}{l|l|l}
\thbf{Disjunktion} & \thbf{Konjunktion} &\\
  $A\lor A \Leftrightarrow A$
& $A\land A \Leftrightarrow A$
& Idempotenzgesetze\\
  $A\lor 0 \Leftrightarrow A$
& $A\land 1 \Leftrightarrow A$
& Neutralitätsgesetze\\
  $A\lor 1 \Leftrightarrow 1$
& $A\land 0 = 0$
& Extremalgesetze\\
  $A\lor \overline A \Leftrightarrow 1$
& $A\land \overline A \Leftrightarrow 0$
& Komplementärgesetze\\
\noalign{\vspace{1em}}
  $A\lor B \Leftrightarrow B\lor A$
& $A\land B \Leftrightarrow B\land A$
& Kommutativgesetze\\
  $(A\lor B)\lor C \Leftrightarrow A\lor (B\lor C)$
& $(A\land B)\land C \Leftrightarrow A\land (B\land C)$
& Assoziativgesetze\\
  $\overline{A\lor B} \Leftrightarrow \overline A\land\overline B$
& $\overline{A\land B} \Leftrightarrow \overline A\lor\overline B$
& De Morgansche Regeln\\
  $A\lor (A\land B) \Leftrightarrow A$
& $A\land (A\lor B) \Leftrightarrow A$
& Absorptionsgesetze\\
\end{tabular}
\end{table*}

\noindent
\strong{Distributivgesetze}:
\begin{gather}
A\lor (B\land C) \iff (A\lor B)\land (A\lor C),\\
A\land (B\lor C) \iff (A\land B)\lor (A\land C).
\end{gather}

\subsubsection{Zweistellige Funktionen}
Es gibt 16 zweistellige boolesche\\
Funktionen.

\begin{tabular}{r|l}
\textbf{\texttt{AB}} & \thbf{Wert}\\
\texttt{00} & \texttt{a}\\
\texttt{01} & \texttt{b}\\
\texttt{10} & \texttt{c}\\
\texttt{11} & \texttt{d}
\end{tabular}

\begin{tabular}{r|l|l|l}
\thbf{Nr.}& \textbf{\texttt{dcba}} & \thbf{Fkt.} & \thbf{Name}\\
 0 & \texttt{0000} & 0 & Kontradiktion\\
 1 & \texttt{0001} & $\overline{A\lor B}$ & NOR\\
 2 & \texttt{0010} & $\overline{B\Rightarrow A}$\\
 3 & \texttt{0011} & $\overline A$\\
 4 & \texttt{0100} & $\overline{A\Rightarrow B}$\\
 5 & \texttt{0101} & $\overline{B}$\\
 6 & \texttt{0110} & $A\oplus B$ & Kontravalenz\index{Kontravalenz}\\
 7 & \texttt{0111} & $\overline{A\land B}$ & NAND\\
 8 & \texttt{1000} & $A\land B$ & Konjunktion\index{Konjunktion}\\
 9 & \texttt{1001} & $A\Leftrightarrow B$ & Äquivalenz\\
10 & \texttt{1010} & $B$ & Projektion\\
11 & \texttt{1011} & $A\Rightarrow B$ & Implikation\\
12 & \texttt{1100} & $A$ & Projektion\\
13 & \texttt{1101} & $B\Rightarrow A$ & Implikation\\
14 & \texttt{1110} & $A\lor B$ & Disjunktion\index{Disjunktion}\\
15 & \texttt{1111} & $1$ & Tautologie
\end{tabular}

\subsubsection{Darstellung mit Negation, Konjunktion und Disjunktion}
\begin{gather}
A\Rightarrow B \iff \overline A\lor B,\\
(A\Leftrightarrow B) \iff
  (\overline A\land\overline B)\lor(A\land B),\\
A\oplus B \iff (\overline A\land B)\lor(A\land\overline B).
\end{gather}

\subsubsection{Tautologien}
Modus ponens:
\begin{equation}
(A\Rightarrow B)\land A\implies B.
\end{equation}
Modus tollens:
\begin{equation}
(A\Rightarrow B)\land\overline B\implies\overline A.
\end{equation}
Modus tollendo ponens:
\begin{equation}
(A\lor B)\land\overline A \implies B.
\end{equation}
Modus ponendo tollens:
\begin{equation}
\overline{A\land B}\land A\implies\overline B.
\end{equation}
Kontraposition:\index{Kontraposition}
\begin{equation}
A\Rightarrow B \iff \overline B\Rightarrow \overline A.
\end{equation}
Beweis durch Widerspruch:\index{Widerspruch}
\begin{equation}
(\overline A\Rightarrow B\land\overline B)\implies A.
\end{equation}
Zerlegung einer Äquivalenz:
\begin{equation}
(A\Leftrightarrow B) \iff (A\Rightarrow B)\land(B\Rightarrow A).
\end{equation}
Kettenschluss:
\begin{equation}
(A\Rightarrow B)\land(B\Rightarrow C)\implies (A\Rightarrow C).
\end{equation}
Ringschluss:
\begin{equation}
\begin{split}
&(A\Rightarrow B)\land (B\Rightarrow C)\land(C\Rightarrow A)\\
&\implies (A\Leftrightarrow B)\land(A\Leftrightarrow C)\land(B\Leftrightarrow C).
\end{split}
\end{equation}
Ringschluss, allgemein:
\begin{equation}
\begin{split}
& (A_1{\Rightarrow }A_2)\land\ldots\land(A_{n-1}{\Rightarrow}A_n)
\land(A_n{\Rightarrow}A_1)\\
& \implies \forall i,j\,[A_i\Leftrightarrow A_j].
\end{split}
\end{equation}
Ersetzungsregel:

Für jede Funktion $P\colon\{0,1\}\to\{0,1\}$ gilt:
\begin{equation}
P(A)\land (A\Leftrightarrow B)\implies P(B).
\end{equation}
Regel zur Implikation:
\begin{equation}
A\land B\Rightarrow C \iff A\Rightarrow (B\Rightarrow C).
\end{equation}
Vollständige Fallunterscheidung:
\begin{gather}
(A\Rightarrow C)\land (B\Rightarrow C)\implies (A\oplus B\Rightarrow C),\\
(A\Rightarrow C)\land (B\Rightarrow C)\iff (A\lor B\Rightarrow C).
\end{gather}
Vollständige Fallunterscheidung, allgemein:
\begin{gather}
\textstyle \forall k[A_k\Rightarrow C]
\implies (\bigoplus_{k=1}^n A_k\Rightarrow C),\\
\forall k[A_k\Rightarrow C]
\iff (\exists k[A_k]\Rightarrow C).
\end{gather}

\newpage
\subsection{Prädikatenlogik}
\subsubsection{Rechenregeln}
Verneinung (De Morgansche Regeln):
\begin{gather}
\overline{\forall x[P(x)]}\iff \exists x[\overline{P(x)}],\\
\overline{\exists x[P(x)]}\iff \forall x[\overline{P(x)}].
\end{gather}
Verallgemeinerte Distributivgesetze:
\begin{gather}
P\lor\forall x[Q(x)] \iff \forall x[P\lor Q(x)],\\
P\land\exists x[Q(x)] \iff \exists x[P\land Q(x)].
\end{gather}
Verallgemeinerte Idempotenzgesetze:
\begin{gather}
\begin{split}
\exists x{\in}M\,[P] & \iff
(M\ne\{\})\land P\\
& \iff\begin{cases}
P & \text{wenn}\; M\ne\{\},\\
0 & \text{wenn}\; M=\{\}.
\end{cases}
\end{split}\\
\begin{split}
\forall x{\in}M\,[P]& \iff
(M=\{\})\lor P\\
&\iff\begin{cases}
P & \text{wenn}\; M\ne\{\},\\
1 & \text{wenn}\; M=\{\}.
\end{cases}
\end{split}
\end{gather}
\newpage\noindent
Äquivalenzen:
\begin{gather}
\hspace{-2em}\forall x\forall y[P(x,y)] \iff \forall y\forall x[P(x,y)],\\
\hspace{-2em}\exists x\exists y[P(x,y)] \iff \exists y\exists x[P(x,y)],\\
\hspace{-2em}\forall x[P(x)\land Q(x)] \iff \forall x[P(x)]\land\forall x[Q(x)],\\
\hspace{-2em}\exists x[P(x)\lor Q(x)] \iff \exists x[P(x)]\lor\exists x[Q(x)],\\
\hspace{-2em}\forall x[P(x)\Rightarrow Q] \iff \exists x[P(x)]\Rightarrow Q,\\
\hspace{-2em}\forall x[P\Rightarrow Q(x)] \iff P\Rightarrow\forall x[Q(x)],\\
\hspace{-2em}\exists x[P(x)\Rightarrow Q(x)]
  \iff\forall x[P(x)]\Rightarrow\exists x[Q(x)].
\end{gather}
% \newpage\noindent
Implikationen:
\begin{gather}
\hspace{-2em}\exists x\forall y[P(x,y)]\implies \forall y\exists x[P(x,y)],\\
\hspace{-2em}\forall x[P(x)]\lor\forall x[Q(x)]\implies\forall x[P(x)\lor Q(x)],\\
\hspace{-2em}\exists x[P(x)\land Q(x)]\implies
  \exists x[P(x)]\land \exists x[Q(x)],\\
\hspace{-2em}\forall x[P(x)\Rightarrow Q(x)]\implies
  (\forall x[P(x)]\Rightarrow\forall x[Q(x)]),\\
\hspace{-2em}\forall x[P(x)\Leftrightarrow Q(x)]\implies
  (\forall x[P(x)]\Leftrightarrow\forall x[Q(x)]).
\end{gather}

\subsubsection{Endliche Mengen}
Sei $M=\{x_1,\ldots,x_n\}$. Es gilt:
\begin{gather}
\forall x{\in}M\,[P(x)]\iff P(x_1)\land\ldots\land P(x_n),\\
\exists x{\in}M\,[P(x)]\iff P(x_1)\lor\ldots\lor P(x_n).
\end{gather}

\subsubsection{Beschränkte Quantifizierung}
\begin{gather}
\begin{split}
& \forall x{\in}M\,[P(x)]\;:\Longleftrightarrow\;\forall x[x\notin M\lor P(x)]\\
& \quad\iff\forall x[x\in M\Rightarrow P(x)],
\end{split}\\
\exists x{\in}M\,[P(x)]\;:\Longleftrightarrow\;\exists x[x\in M\land P(x)],\\
\forall x{\in}M{\setminus}N\,[P(x)]\iff \forall x[x\notin N\Rightarrow P(x)].
\end{gather}

\subsubsection[Quantifizierung über Produktmengen]{\\
Quantifizierung über Produktmengen}
\begin{gather}
\forall(x,y)\,[P(x,y)]\iff \forall x\forall y[P(x,y)],\\
\exists(x,y)\,[P(x,y)]\iff \exists x\exists y[P(x,y)].
\end{gather}
Analog gilt
\begin{gather}
\forall(x,y,z)\,\iff \forall x\forall y\forall z,\\
\exists(x,y,z)\,\iff \exists x\exists y\exists z
\end{gather}
usw.

\subsubsection{Alternative Darstellung}
Sei $P\colon G\to\{0,1\}$ und $M\subseteq G$.
Mit $P(M)$ ist die Bildmenge von $P$ bezüglich $M$ gemeint.
Es gilt
\begin{equation}
\begin{split}
&\forall x{\in}M\,[P(x)] \iff P(M)=\{1\}\\
& \iff M\subseteq\{x{\in}G\mid P(x)\}
\end{split}
\end{equation}
und
\begin{equation}
\begin{split}
& \exists x{\in}M\,[P(x)] \iff \{1\}\subseteq P(M)\\
& \iff M\cap\{x{\in}G\mid P(x)\}\ne\{\}.
\end{split}
\end{equation}

\subsubsection{Eindeutigkeit}
Quantor für eindeutige Existenz:
\begin{equation}
\begin{split}
&\exists!x\,[P(x)]\\
&:\Longleftrightarrow\; \exists x\,[P(x)\land \forall y\,[P(y)\Rightarrow x=y]]\\
&\iff \exists x\,[P(x)]\land \forall x\forall y[P(x)\land P(y)\Rightarrow x=y].
\end{split}
\end{equation}

\newpage
\section{Mengenlehre}
\subsection{Definitionen}
Aufzählende Notation:
\begin{equation}
a\in\{x_1,\ldots,x_n\} :\Leftrightarrow a=x_1\lor\ldots\lor a=x_n.
\end{equation}
Beschreibende Notation:
\begin{gather}
a\in\{x\mid P(x)\}\;:\Longleftrightarrow\; P(a),\\
\{x\in M\mid P(x)\} := \{x\mid x\in M\land P(x)\},\\
\{f(x)\mid P(x)\} := \{y\mid y=f(x)\land P(x)\}.
\end{gather}
Teilmengenrelation:
\begin{equation}
A\subseteq B\;:\Longleftrightarrow\; \forall x\,[x\in A\implies x\in B].
\end{equation}
Gleichheit:
\begin{equation}
A=B\;:\Longleftrightarrow\; \forall x\,[x\in A\iff x\in B].
\end{equation}
Vereinigungsmenge:
\begin{equation}
A\cup B:=\{x\mid x\in A\lor x\in B\}.
\end{equation}
Schnittmenge:
\begin{equation}
A\cap B:=\{x\mid x\in A\land x\in B\}.
\end{equation}
Differenzmenge:
\begin{equation}
A\setminus B:=\{x\mid x\in A\land x\not\in B\}.
\end{equation}
Symmetrische Differenz:
\begin{equation}
A\triangle B:=\{x\mid x\in A\oplus x\in B\}.
\end{equation}
Komplementärmenge:
\begin{equation}
A^\comp := G\setminus A.\qquad (\text{$G$: Grundmenge})
\end{equation}
Vereinigung über indizierte Mengen:
\begin{equation}
\bigcup_{i\in I} A_i := \{x\mid\exists i{\in}I\,[x\in A_i]\}.
\end{equation}
Schnitt über indizierte Mengen:
\begin{equation}
\bigcap_{i\in I} A_i := \{x\mid\forall i{\in}I\,[x\in A_i]\}.
\end{equation}


\subsection{Boolesche Algebra}
\begin{table*}[t]
\caption{Boolesche Algebra}
\begin{tabular}{l|l|l}
\thbf{Vereinigung} & \thbf{Schnitt} &\\
  $A\cup A = A$
& $A\cap A = A$
& Idempotenzgesetze\\
  $A\cup \{\} = A$
& $A\cap G = A$
& Neutralitätsgesetze\\
  $A\cup G = G$
& $A\cap \{\} = \{\}$
& Extremalgesetze\\
  $A\cup \overline A = G$
& $A\cap \overline A = \{\}$
& Komplementärgesetze\\
\noalign{\vspace{1em}}
  $A\cup B = B\cup A$
& $A\cap B = B\cap A$
& Kommutativgesetze\\
  $(A\cup B)\cup C = A\cup (B\cup C)$
& $(A\cap B)\cap C = A\cap (B\cap C)$
& Assoziativgesetze\\
  $\overline{A\cup B} = \overline A\cap\overline B$
& $\overline{A\cap B} = \overline A\cup\overline B$
& De Morgansche Regeln\\
  $A\cup (A\cap B) = A$
& $A\cap (A\cup B) = A$
& Absorptionsgesetze\\
\end{tabular}\\
\\
$G$: Grundmenge
\end{table*}

\noindent
\strong{Distributivgesetze}:
\begin{gather}
M\cup (A\cap B) = (M\cup A)\cap (M\cup B),\\
M\cap (A\cup B) = (M\cap A)\cup (M\cap B).
\end{gather}

\subsection{Teilmengenrelation}
Zerlegung der Gleichheit:
\begin{equation}
A=B \iff A\subseteq B \land B\subseteq A.
\end{equation}
Umschreibung der Teilmengenrelation:
\begin{equation}
\begin{split}
A\subseteq B &\iff A\cap B=A\\
& \iff A\cup B=B\\
& \iff A\setminus B=\{\}.
\end{split}
\end{equation}
Kontraposition:
\begin{equation}
A\subseteq B = B^\comp\subseteq A^\comp.
\end{equation}

\subsection{Natürliche Zahlen}
\subsubsection{Von-Neumann-Modell}
Mengentheoretisches Modell der natürlichen Zahlen:
\begin{equation}
\begin{split}
& 0:=\{\},\quad 1:=\{0\},\quad 2:=\{0,1\},\\
& 3:=\{0,1,2\},\quad \text{usw.}
\end{split}
\end{equation}
Nachfolgerfunktion:
\begin{equation}
x' := x\cup\{x\}.
\end{equation}
\subsubsection{Vollständige Induktion}
Ist $A(n)$ mit $n\in\N$
eine Aussageform, so gilt:
\begin{equation}
\begin{split}
& A(n_0)\land \forall n\ge n_0\,[A(n)\Rightarrow A(n+1)]\\
& \implies \forall n\ge n_0\,[A(n)].
\end{split}
\end{equation}
Die Aussage $A(n_0)$ ist der \emph{Induktionsanfang}.

Die Implikation
\begin{equation}
A(n)\Rightarrow A(n+1)
\end{equation}
heißt \emph{Induktionsschritt}. Beim Induktionsschritt muss
$A(n+1)$ gezeigt werden, wobei $A(n)$ als gültig vorausgesetzt werden
darf.

\newpage
\subsection{ZFC-Axiome}

Axiom der Bestimmtheit:
\begin{equation}
\forall A\forall B\,[A=B\iff\forall x\,[x\in A\Leftrightarrow x\in B]].
\end{equation}
Axiom der leeren Menge:
\begin{equation}
\exists M\forall x\,[x\notin M].
\end{equation}
Axiom der Paarung:
\begin{equation}
\forall x\forall y\exists M\forall a\,[a\in M\iff x=a\lor y=a].
\end{equation}
Axiom der Vereinigung:
\begin{equation}
\forall S\exists M\forall x\,[x\in M\iff\exists A{\in}S\,[x\in A]].
\end{equation}
Axiom der Aussonderung:
\begin{equation}
\forall A\exists M\forall x\,[x\in M\iff x\in A\land\varphi(x)].
\end{equation}
Axiom des Unendlichen:
\begin{equation}
\exists M\,[\{\}\in M\land\forall x{\in}M\,[x\cup\{x\}\in M]].
\end{equation}
Axiom der Potenzmenge:
\begin{equation}
\forall A\exists M\forall T\,[T\in M\iff T\subseteq A].
\end{equation}
Axiom der Ersetzung:
\begin{equation}
\begin{split}
&\forall a{\in}A\;\exists^{=1} b\,[\varphi(a,b)]\\
&\implies\exists B\,\forall b\,[b\in B\iff\exists a{\in}A\,[\varphi(a,b)]].
\end{split}
\end{equation}
Axiom der Fundierung:
\begin{equation}
\forall A\,[A\ne\{\}\implies\exists x{\in}A\,[x\cap A=\{\}]].
\end{equation}
Auswahlaxiom:
\begin{equation}
\begin{split}
&\forall x,y{\in}A\,[x\ne y\implies x\cap y=\{\}]\\
&\quad\land\forall x{\in}A\,[x\ne\{\}]\\
&\implies\exists M\;\forall x{\in}A\;\exists^{=1}u{\in}x\,[u\in M].
\end{split}
\end{equation}

\newpage
\subsection{Kardinalität}
\begin{Definition}
Zwei Mengen $M,N$ heißen \emdef{gleichmächtig}, notiert als
$|M|=|N|$, wenn es eine bijektive Abbildung $f\colon M\to N$ gibt.

Eine Menge $M$ heißt \emdef{weniger mächtig oder gleichmächtig},
notiert als $|M|\le|N|$, wenn es eine injektive Abbildung
$f\colon M\to N$ gibt. Äquivalent dazu ist, dass es eine
surjektive Abbildung $g\colon N\to M$ gibt.

Eine Menge heißt \emdef{abzählbar unendlich}, wenn sie gleichmächtig
zu den natürlichen Zahlen ist.
\end{Definition}
Gleichmächtigkeit ist eine Äquivalenzrelation.
\begin{Definition}
Die Äquivalenzklassen
\begin{equation}
|M| := \{N\mid\;{\scriptstyle |M|=|N|}\}
\end{equation}
heißen \emdef{Kardinalzahlen}.
\end{Definition}

\strong{Satz von Cantor-Bernstein.}

Aus $|M|\le |N|$ und $|N|\le |M|$ folgt $|M|=|N|$.

\subsubsection{Potenzmengen}

\strong{Satz von Cantor.}
Für jede Menge gilt $|M|<|2^M|$.

Ist $M$ endlich, dann gilt $|M|=2^{|M|}$.


\section{Funktionen}
\subsection{Surjektionen}\index{surjektiv}
\begin{Definition}
Eine Funktion $f\colon A\to B$ heißt \emdef{surjektiv},\\
wenn $f(A)=B$ ist. Damit ist gemeint, dass jedes Element
der Zielmenge wenigstens einmal der Funktionswert von einem
Element der Definitionsmenge ist.
\end{Definition}

\subsection{Injektionen}\index{injektiv}
\begin{Definition}
Eine Funktion $f\colon A\to B$ heißt \emdef{injektiv},\\
wenn
\begin{equation}
\forall x_1,x_2\in A\,[f(x_1)=f(x_2)\implies x_1=x_2]
\end{equation}
gilt.
\end{Definition}

\subsection{Bijektionen}\index{bijektiv}
\begin{Definition}
Eine Funktion $f\colon A\to B$ heißt \emdef{bijektiv},
wenn sie injektiv und surjektiv ist.

Eine Funktion $f\colon A\to B$ ist genau dann bijektiv, wenn es
ein $g$ mit
\begin{equation}
g\circ f = \id_A\quad\text{und}\quad f\circ g = \id_B
\end{equation}
gibt. Wenn $f$ bijektiv ist, so gibt es $g$ genau einmal und
$g$ wird die \emph{Umkehrfunktion}\index{Umkehrfunktion}
oder \emph{Inverse}
von $f$ genannt und als $f^{-1}$ notiert.
\end{Definition}

\subsection{Komposition}\index{Komposition}
\begin{Definition} Für zwei Funktionen $f\colon A\to B$
und $g\colon B\to C$ ist die \emdef{Komposition}
($g$ nach $f$)
durch
\begin{equation}\label{eq:composition}
g\circ f\colon A\to C,\quad (g\circ f)(x) := g(f(x))
\end{equation}
definiert.
\end{Definition}
Für die Komposition gilt das Assozativgesetz:
\begin{equation}
(f\circ g)\circ h = f\circ(g\circ h).
\end{equation}

Die Komposition von Injektionen ist eine Injektion.

Die Komposition von Surjektionen ist eine Surjektion.

Die Komposition von Bijektionen ist eine Bijektion.

Sind $f,g$ Bijektionen, so gilt
\begin{equation}
(g\circ f)^{-1} = f^{-1}\circ g^{-1}.
\end{equation}

Ist $g\circ f$ injektiv, so ist $f$ injektiv.

Ist $g\circ f$ surjektiv, so ist $g$ surjektiv.

Ist $g\circ f$ bijektiv, so ist $f$ injektiv und $g$ surjektiv.

\begin{Definition}
Für eine Funktion $\varphi\colon A\to A$ wird
\begin{equation}
\varphi^0:=\operatorname{id}_A,\quad \varphi^{n+1}:=\varphi^n\circ\varphi
\end{equation}
\emdef{Iteration}\index{Iteration} von $\varphi$ genannt.
\end{Definition}

\subsection{Einschränkung}\index{Einschränkung}
\begin{Definition} Sei $f\colon A\to B$ und $M\subseteq A$.
Die Funktion $g(x)=f(x)$ mit $g\colon M\to B$ wird \emdef{Einschränkung}
von $f$ genannt und mit $f|_M$ notiert.
\end{Definition}
Sei $f\colon A\to B$ und $M\subseteq A$.
Mit der Inklusionsabbildung $i(x):=x$ mit $i\colon M\to A$ gilt:
\begin{equation}
f|_M = f\circ i.
\end{equation}
Es gilt
\begin{equation}
g\circ (f|_M) = (g\circ f)|_M.
\end{equation}
%\newpage
\subsection{Bild}\index{Bild}
\begin{Definition} Ist $f\colon A\to B$ und $M\subseteq A$, so wird
\begin{equation}
f(M) := \{f(x)\mid x\in M\}
\end{equation}
das \emdef{Bild} von $M$ unter $f$ genannt.
\end{Definition}
Es gilt
\begin{align}
&f(M\cup N) = f(M)\cup f(N),\\
&f(M\cap N) = f(M)\cap f(N),\\
&f\Big(\bigcup_{i\in I}M_i\Big) = \bigcup_{i\in I} f(M_i),\\
&I\ne\emptyset\implies f\Big(\bigcap_{i\in I} M_i\Big) = \bigcap_{i\in I} f(M_i),\\
&M\subseteq N\implies f(M)\subseteq f(N),\\
&f(\emptyset) = \emptyset,\\
&(g\circ f)(M) = g(f(M)).
\end{align}

\subsection{Urbild}\index{Urbild}
\begin{Definition}
Ist $f\colon A\to B$, so wird
\begin{equation}
f^{-1}(M) := \{x\in A\mid f(x)\in M\}.
\end{equation}
das \emdef{Urbild} von $M$ unter $f$ genannt.
\end{Definition}
Es gilt
\begin{align}
& f^{-1}(M\cup N) = f^{-1}(M)\cup f^{-1}(N),\\
& f^{-1}(M\cap N) = f^{-1}(M)\cap f^{-1}(N),\\
& f^{-1}\Big(\bigcup_{i\in I}M_i\Big) = \bigcup_{i\in I} f^{-1}(M_i),\\
& I\ne\emptyset\implies f^{-1}\Big(\bigcap_{i\in I} M_i\Big) = \bigcap_{i\in I}f^{-1}(M_i),\\
& M\subseteq N\implies f^{-1}(M)\subseteq f^{-1}(N),\\
& f^{-1}(\emptyset) = \emptyset,\\
& f^{-1}(B) = A,\\
& f^{-1}(M\setminus N) = f^{-1}(M)\setminus f^{-1}(N),\\
& f^{-1}(B\setminus M) = B\setminus f^{-1}(M),\\
& (g\circ f)^{-1}(M) = f^{-1}(g^{-1}(M)),\\
& (f|_M)^{-1}(N) = M\cap f^{-1}(N).
\end{align}

\newpage
\section{Mathematische Strukturen}\label{sec:Strukturen}
\subsubsection*{Axiome}

\noindent\bsf{E:} Abgeschlossenheit.
\ibox{Die Verknüpfung führt nicht aus der Menge heraus.}

\noindent\bsf{A:} Assoziativgesetz.
\ibox{$\forall a,b,c\bright (a*b)*c = a*(b*c)\bleft$.}

\noindent\bsf{N:} Existenz des neutralen Elements.
\ibox{$\exists e\forall a\bright e*a=a*e=a\bleft$.}

\noindent\bsf{I:} Existenz der inversen Elemente.
\ibox{$\forall a\exists b\bright a*b=b*a=e\bleft$.}

\noindent\bsf{K:} Kommutativgesetz.
\ibox{$\forall a,b\bright a*b=b*a\bleft.$}

\noindent
\bsf{I*:} Existenz der multiplikativ inversen Elemente.
\ibox{$\forall a{\ne}0\;\exists b\bright a*b=b*a=1\bleft$.}

\noindent\bsf{Dl:} Linksdistributivgestz.
\ibox{$\forall a,x,y\bright a*(x+y) = a*x+a*y\bleft$.}

\noindent\bsf{Dr:} Rechtsdistributivgesetz.
\ibox{$\forall a,x,y\bright (x+y)*a = x*a+y*a\bleft$.}

\noindent\bsf{D:} Distributivgesetze.
\ibox{Dl und Dr.}

\noindent\bsf{T:} Nullteilerfreiheit.
\ibox{$\forall a,b\bright a\ne 0\land b\ne 0\implies a*b\ne 0\bleft$}
\ibox{bzw. die Kontraposition}
\ibox{$\forall a,b\bright a*b=0\implies a=0\lor b=0\bleft$.}

\noindent\bsf{U:} Unterscheibarkeit von Null- und Einselement.
\ibox{Die neutralen Elemente bezüglich Addition und}
\ibox{Multiplikation sind unterschiedlich.}

\subsubsection*{Strukturen}
Strukturen mit einer inneren Verknüpfung:\\
\begin{tabular}{l|l}
\bsf{EA} & Halbgruppe\\
\bsf{EAN} & Monoid\\
\bsf{EANI} & Gruppe\\
\bsf{EANIK} & abelsche Gruppe
\end{tabular}

\noindent
Strukturen mit zwei inneren Verknüpfungen:\\
\begin{tabular}{l|l}
\bsf{EANIK, EA, D}\dotfill & Ring\\
\bsf{EANIK, EAK, D}\dotfill & kommutativer Ring\\
\bsf{EANIK, EAN, D}\dotfill & unitärer Ring\\
\bsf{EANIK, EANK, DTU} & Integritätsring\\
\bsf{EANIK, EANI*K, DTU} & Körper
\end{tabular}

\newpage
\subsubsection*{Axiome für Relationen}

\noindent\bsf{R:} Reflexivität.
\ibox{$\forall a\,(a R a)$.}

\noindent\bsf{S:} Symmetrie.
\ibox{$\forall a,b\,(aRb\iff bRa)$.}

\noindent\bsf{T:} Transitivität.
\ibox{$\forall a,b,c\,(aRb\land bRc\implies aRc)$.}

\noindent\bsf{An:} Antisymmetrie.
\ibox{$\forall a,b\,(aRb\land bRa\implies a=b)$.}

\noindent\bsf{L:} Linearität.
\ibox{$\forall a,b\,(aRb\lor bRa)$.}

\noindent\bsf{Ri:} Irrreflexivität.
\ibox{$\forall a\,(\neg aRa)$.}

\noindent\bsf{A:} Asymmetrie.
\ibox{$\forall a,b\,(aRb\implies \neg bRa)$.}

\noindent\bsf{Min:} Existenz der Minimalelemente.
\ibox{$\forall T{\subseteq}M, T{\ne}\emptyset\;\exists x{\in}T\;\forall y{\in}T{\setminus}\{x\}\,(x<y)$.}

\subsubsection*{Relationen}
\begin{tabular}{l|l}
\bsf{RST}\dotfill & Äquivalenzrelation\\
\bsf{RAnT}\dotfill & Halbordnung\\
\bsf{RAnTL}\dotfill & Totalordnung\\
\bsf{RiAT}\dotfill & strenge Halbordnung\\
\bsf{RiATL}\dotfill & strenge Totalordnung\\
\bsf{RiATLMin} & Wohlordnung
\end{tabular}


\chapter{Funktionen}
\section{Elementare Funktionen}
\subsection{Exponentialfunktion}
\begin{Definition}
$\exp\colon\C\to\C$ mit
\begin{equation}
\exp(x) := \sum_{k=0}^{\infty} \frac{x^k}{k!}.
\end{equation}
\end{Definition}
\noindent
Die Einschränkung von $\exp$ auf $\R$ ist injektiv und
hat die Bildmenge $\{x{\in}\R\mid x>0\}$.

Für alle $x,y\in\C$ gilt:
\begin{gather}
\exp(x+y) = \exp(x)\exp(y),\\
\exp(x-y) = \frac{\exp(x)}{\exp(y)},\\
\exp(-x) = \frac{1}{\exp(x)}.
\end{gather}
\strong{Eulersche Formel.} Für alle $x\in\C$ gilt:
\begin{equation}
\ee^{\ui x} = \cos x+\ui\sin x.
\end{equation}

\subsection{Winkelfunktionen}\index{Winkelfunktion}
\begin{Definition}
\emdef{Kosinus}\index{Kosinus}\index{Cosinus}: $\C\to\C$,
\begin{equation}
\cos(x) := \sum_{k=0}^\infty \frac{x^{2k}}{(2k)!}.
\end{equation}
\emdef{Sinus}\index{Sinus}: $\C\to\C$,
\begin{equation}
\sin(x) := \sum_{k=0}^\infty \frac{x^{2k+1}}{(2k+1)!}.
\end{equation}
\emdef{Tangens}\index{Tangens}: $\C\setminus\{k\pi+\pi/2\mid k\in\Z\}\to\C$,
\begin{equation}
\tan(x) := \frac{\sin(x)}{\cos(x)}.
\end{equation}
\emdef{Kotangens}\index{Kotangens}: $\C\setminus\{k\pi\mid k\in\Z\}\to\C$,
\begin{equation}
\cot(x) := \frac{\cos(x)}{\sin(x)}.
\end{equation}
\emdef{Sekans}\index{Sekans}: $\C\setminus\{k\pi+\pi/2\mid k\in\Z\}\to\C$,
\begin{equation}
\sec(x) := \frac{1}{\cos(x)}.
\end{equation}
\emdef{Kosekans}\index{Kosekans}: $\C\setminus\{k\pi\mid k\in\Z\}\to\C$,
\begin{equation}
\csc(x) := \frac{1}{\sin(x)}.
\end{equation}
\end{Definition}
\noindent
Darstellung durch die Exponentialfunktion:\\
Für alle $x\in\C$ gilt:
\begin{align}
\cos x &= \Real(\ee^{\ui x}) = \frac{\ee^{\ui x}+\ee^{-\ui x}}{2},\\
\sin x &= \Imag(\ee^{\ui x}) = \frac{\ee^{\ui x}-\ee^{-\ui x}}{2\ui}.
\end{align}

\subsubsection{Symmetrie und Periodizität}
Für alle $x\in\C$ gilt:
\begin{align}
\sin(-x) &= -\sin x,\enspace(\text{Punktsymmetrie})\\
\cos(-x) &= \cos x,\quad\;(\text{Achsensymmetrie})\\
\sin(x+2\pi) &= \sin x,\\
\cos(x+2\pi) &= \cos x,\\
\sin(x+\pi)  &=-\sin x,\\
\cos(x+\pi)  &=-\cos x,\\
\sin\Big(x+\frac{\pi}{2}\Big) &= \cos x = -\sin\Big(x-\frac{\pi}{2}\Big),\\
\cos\Big(x+\frac{\pi}{2}\Big) &= -\sin x = -\cos\Big(x-\frac{\pi}{2}\Big).
\end{align}

\subsubsection{Additionstheoreme}
\index{Additionstheoreme}

Für alle $x,y\in\C$ gilt:
\begin{align}
\sin(x+y) &= \sin x\cos y+\cos x\sin y,\\
\sin(x-y) &= \sin x\cos y-\cos x\sin y,\\
\cos(x+y) &= \cos x\cos y-\sin x\sin y,\\
\cos(x-y) &= \cos x\cos y+\sin x\sin y.
\end{align}

\subsubsection{Trigonometrischer Pythagoras}
Für alle $x\in\C$ gilt:
\begin{equation}
\sin^2 x+\cos^2 x=1.
\end{equation}

\subsubsection{Produkte}
Für alle $x,y\in\C$ gilt:
\begin{align}
2\sin x\sin y &= \cos(x-y)-\cos(x+y),\\
2\cos x\cos y &= \cos(x-y)+\cos(x+y),\\
2\sin x\cos y &= \sin(x-y)+\sin(x+y).
\end{align}

\subsubsection{Summen und Differenzen}
Für alle $x,y\in\C$ gilt:
\begin{align}
\sin x+\sin y &= 2\sin\frac{x+y}{2}\cos\frac{x-y}{2},\\
\sin x-\sin y &= 2\cos\frac{x+y}{2}\sin\frac{x-y}{2},\\
\cos x+\cos y &= 2\cos\frac{x+y}{2}\cos\frac{x-y}{2},\\
\cos x-\cos y &= 2\sin\frac{x+y}{2}\sin\frac{y-x}{2}.
\end{align}

\subsubsection{Winkelvielfache}
Für alle $x\in\C$ gilt:
\begin{align}
\sin(2x) &= 2\sin x\cos x,\\
\cos(2x) &= \cos^2 x-\sin^2 x,\\
\sin(3x) &= 3\sin x-4\sin^3 x,\\
\cos(3x) &= 4\cos^3 x-3\cos x.
\end{align}


\chapter{Analysis}
\section{Ableitungen}
\subsection{Differentialquotient}
Sei $U\subseteq\mathbb R$ ein offenes Intervall
und sei $f\colon U\to\mathbb R$. Die Funktion $f$ heißt
differenzierbar an der Stelle $x_0\in U$, falls der Grenzwert
\begin{equation}
\begin{split}
&\lim_{x\to x_0} \frac{f(x)-f(x_0)}{x-x_0}
= \lim_{h\to 0}\frac{f(x+h)-f(x)}{h}
\end{split}
\end{equation}
existiert. Dieser Grenzwert heißt
Differentialquotient oder Ableitung
von $f$ an der Stelle $x_0$. Notation:
\begin{equation}
f'(x_0),\,\qquad (Df)(x_0),\qquad \frac{\mathrm df(x)}{\mathrm dx}\Big|_{x=x_0}.
\end{equation}



\chapter{Lineare Algebra}
\section{Grundbegriffe}
\subsection{Norm}\index{Norm}
\begin{Definition}
Eine Abbildung $v\mapsto\|v\|$ von einem
Vektorraum $V$ über dem Körper $K$ in die nichtnegativen reellen
Zahlen heißt \emdef{Norm}, wenn für alle $v,w\in V$ und $a\in K$
die drei Axiome%
\begin{gather}
\|v\|=0 \implies v=0,\\
\|av\| = |a|\,\|v\|,\\
\|v+w\| \le \|v\|+\|w\|
\end{gather}
erfüllt sind.
\end{Definition}

Eigenschaften:
\begin{gather}
\|v\|=0\iff v=0,\\
\|-v\|=\|v\|,\\
\|v\|\ge 0.
\end{gather}
Dreiecksungleichung nach unten:
\begin{equation}
|\|v\|-\|w\||\le \|v-w\|.
\end{equation}

\subsection{Skalarprodukt}\index{Skalarprodukt}

\subsubsection{Axiome}
{\definition}
Eine Abbildung heißt \emdef{Skalarprodukt},
wenn folgende Axiome erfüllt sind.

Axiome für $v,w$ aus einem reellen Vektorraum und $\lambda$ ein Skalar:
\begin{gather}
\langle v,w\rangle = \langle w,v\rangle,\\
\langle v,\lambda w\rangle = \lambda\langle v,w\rangle,\\
\langle v,w_1+w_2\rangle = \langle v,w_1\rangle +\langle v,w_2\rangle,\\
\langle v,v\rangle\ge 0,\\
\langle v,v\rangle=0 \iff v=0.
\end{gather}
Axiome für $v,w$ aus einem komplexen Vektorraum und $\lambda$ ein Skalar:
\begin{gather}
\langle v,w\rangle = \overline{\langle w,v\rangle},\\
\langle \lambda v,w\rangle = \overline{\lambda}\langle v,w\rangle,\\
\langle v,\lambda w\rangle = \lambda\langle v,w\rangle,\\
\langle v,w_1+w_2\rangle = \langle v,w_1\rangle +\langle v,w_2\rangle,\\
\langle v,v\rangle\ge 0,\\
\langle v,v\rangle=0 \iff v=0.
\end{gather}

\subsubsection{Eigenschaften}
Das reelle Skalarprodukt ist eine symmetrische bilineare Abbildung.

\subsubsection{Winkel und Längen}
{\definition}
Der \emdef{Winkel} $\varphi$ zwischen $v$ und $w$
ist definiert durch die Beziehung:
\begin{equation}
\langle v,w\rangle = \|v\|\,\|w\|\,\cos\varphi.
\end{equation}
{\definition}
\emdef{Orthogonal}:\index{Orthogonal}
\begin{equation}
v\perp w \;:\Longleftrightarrow\; \langle v,w\rangle=0.
\end{equation}
Ein Skalarprodukt $\langle v,w\rangle$ induziert die Norm
\begin{equation}
\|v\| := \sqrt{\langle v,v\rangle}.
\end{equation}

\subsubsection{Orthonormalbasis}\label{sec:ONB}
\index{Orthogonalsystem}\index{Orthogonalbasis}
\index{Orthonormalsystem}\index{Orthonormalbasis}
Sei $B=(b_k)_{k=1}^n$ eine Basis eines endlichdimensionalen
Vektorraumes über den reellen oder komplexen Zahlen.

{\definition}
Gilt $\langle b_i,b_j\rangle=0$
für alle $i,j$ mit $i\ne j$, so wird $B$ \emdef{Orthogonalbasis}
genannt. Ist $B$ nicht unbedingt eine Basis, so spricht man von einem 
\emdef{Orthogonalsystem}.

{\definition}
Ist $B$ eine Orthogonalbasis und gilt
zusätzlich $\langle b_k,b_k\rangle=1$ für alle $k$, so wird
$B$ \emdef{Orthonormalbasis} (ONB) genannt. Ist $B$ nicht unbedingt
eine Basis,  so spricht man von einem \emdef{Orthonormalsystem}.

Sei $v=\sum_k v_kb_k$ und $w=\sum_k w_kb_k$.
Mit $\sum_k$ ist immer $\sum_{k=1}^n$ gemeint.

Ist $B$ eine Orthonormalbasis, so gilt:
\begin{equation}
\langle v,w\rangle = \sum_k \overline{v_k}\,w_k.
\end{equation}
Ist $B$ nur eine Orthogonalbasis, so gilt:
\begin{equation}
\langle v,w\rangle = \sum_k \langle b_k,b_k\rangle \overline{v_k}\,w_k
\end{equation}
Allgemein gilt:
\begin{equation}
\langle v,w\rangle = \sum_{i,j} g_{ij} \overline{v_i}\,w_j
\end{equation}
mit $g_{ij}=\langle b_i,b_j\rangle$. In reellen Vektorräumen
ist die komplexe Konjugation wirkungslos und kann somit entfallen.

Ist $B$ eine Orthogonalbasis und $v=\sum_k v_k b_k$, so gilt:
\begin{equation}
v_k = \frac{\langle b_k,v\rangle}{\langle b_k,b_k\rangle}.
\end{equation}
Ist $B$ eine Orthonormalbasis, so gilt speziell:
\begin{equation}
v_k = \langle b_k,v\rangle.
\end{equation}


\subsubsection{Orthogonale Projektion}
Orthogonale Projektion von $v$ auf $w$:
\begin{equation}
P[w](v) := \frac{\langle v,w\rangle}{\langle w,w\rangle}\,w.
\end{equation}
\subsubsection{Gram-Schmidt-Verfahren}
Für linear unabhängige Vektoren $v_1,\ldots,v_n$
wird durch%
\begin{equation}
w_k := v_k - \sum_{i=1}^{k-1} P[w_i](v_k)
\end{equation}
ein Orthogonalsystem $w_1,\ldots,w_n$ berechnet.

Speziell für zwei nicht kollineare Vektoren $v_1,v_2$ gilt
\begin{gather}
w_1=v_1,\\
w_2=v_2-P[w_1](v_2).
\end{gather}
\section{Koordinatenvektoren}
\subsection{Koordinatenraum}
Addition von $a,b\in K^n$:
\begin{equation}\label{eq:Koordinatenraum-Addition}
\begin{bmatrix}
a_1\\
\vdots\\
a_n
\end{bmatrix}
+\begin{bmatrix}
b_1\\
\vdots\\
b_n
\end{bmatrix}
:= \begin{bmatrix}
a_1+b_1\\
\vdots\\
a_n+b_n
\end{bmatrix}.
\end{equation}
Subtraktion:
\begin{equation}
\begin{bmatrix}
a_1\\
\vdots\\
a_n
\end{bmatrix}
-\begin{bmatrix}
b_1\\
\vdots\\
b_n
\end{bmatrix}
:= \begin{bmatrix}
a_1-b_1\\
\vdots\\
a_n-b_n
\end{bmatrix}.
\end{equation}
Skalarmultiplikation von $\lambda\in K$ mit $a\in K^n$:
\begin{align}\label{eq:Koordinatenraum-Skalarmultiplikation}
\lambda\begin{bmatrix}
a_1\\
\vdots\\
a_n
\end{bmatrix}
:= \begin{bmatrix}
\lambda a_1\\
\vdots\\
\lambda a_n
\end{bmatrix}.
\end{align}
Ist $K$ ein Körper, so bildet die Menge
\begin{equation}
K^n = \{(a_1,\ldots,a_n)\mid \forall k\colon a_k\in K\}
\end{equation}
bezüglich der Addition \eqref{eq:Koordinatenraum-Addition}
und der Multiplikation \eqref{eq:Koordinatenraum-Skalarmultiplikation}
einen Vektorraum, der \emdef{Koordinatenraum} genannt wird.
Das Tupel $E_n=(e_1,\ldots,e_n)$ mit
\begin{equation}\label{eq:kanonische-Basis}
\begin{split}
e_1 &:= (1,0,0,0,\ldots, 0),\\
e_2 &:= (0,1,0,0,\ldots, 0),\\
e_3 &:= (0,0,1,0,\ldots, 0),\\
\ldots\\
e_n &:= (0,0,0,0,\ldots, 1)
\end{split}
\end{equation}
bildet eine geordnete Basis von $K^n$, die \emdef{kanonische Basis}
genannt wird. Es gilt
\begin{equation}
a = (a_1,\ldots,a_n) = a_1 e_1+\ldots+a_n e_n.
\end{equation}

\subsection{Kanonisches Skalarprodukt}
\begin{Definition}
Für $a,b\in\R^n$:
\begin{equation}
\langle a,b\rangle := \sum_{k=1}^n a_k b_k.
\end{equation}
Für $a,b\in\C^n$:
\begin{equation}
\langle a,b\rangle := \sum_{k=1}^n \overline{a_k}\,b_k
\end{equation}
\end{Definition}
\noindent
Die kanonische Basis \eqref{eq:kanonische-Basis} ist eine
Orthonormalbasis bezüglich diesem Skalarprodukt, s. \ref{sec:ONB}.
Das Skalarprodukt induziert die Norm
\begin{equation}
|a| := \sqrt{\langle a,a\rangle} = \sqrt{\textstyle \sum_{k=1}^n |a_k|^2},
\end{equation}
die \emdef{Vektorbetrag} genannt wird.

Jedem Koordinatenvektor $a\ne 0$ lässt sich ein Einheitsvektor
$\hat a:=\frac{a}{|a|}$ zuordnen, der in Richtung von $a$ zeigt
und die Eigenschaft $|\hat a|=1$ besitzt.


\section{Matrizen}\index{Matrix}
\subsection{Quadratische Matrizen}%
\index{Matrix!quadratische}\index{quadratische Matrix}
\subsubsection{Matrizenring}%
\index{Ring!Matrizenring}\index{Matrizenring}
Mit $K^{n\times n}$ wird die Menge quadratischen Matrizen
\begin{equation}
(a_{ij}) = \begin{bmatrix}
a_{11} & \ldots & a_{1n}\\
\ldots & \ddots & \ldots\\
a_{n1} & \ldots & a_{nn}
\end{bmatrix}
\end{equation}
mit Einträgen $a_{ij}$ aus dem Körper $K$ bezeichnet.

Die Menge $K^{n\times n}$ bildet bezüglich Addition
und Multiplikation von Matrizen einen Ring (s. \ref{sec:Strukturen}).

Das neutrale Element der Multiplikation
ist die Einheitsmatrix
\begin{equation}
E_n = (\delta_{ij}),\quad
\delta_{ij}:=\begin{cases}
1 & \text{wenn}\;i=j,\\
0 & \text{sonst}.
\end{cases}
\end{equation}
Das sind
\begin{equation}
E_2 = \begin{bmatrix}
1 & 0\\
0 & 1
\end{bmatrix},\quad
E_3 = \begin{bmatrix}
1 & 0 & 0\\
0 & 1 & 0\\
0 & 0 & 1
\end{bmatrix},
\quad\text{usw.}
\end{equation}

\subsubsection{Symmetrische Matrizen}
Eine quadratiche Matrix $A=(a_{ij})$ heißt
symmetrisch\index{symmetrische Matrix},
falls gilt $a_{ij}=a_{ji}$ bzw. $A^T=A$.

Jede reelle symmetrische Matrix besitzt ausschließlich reelle
Eigenwerte und die algebraischen Vielfachheiten stimmen mit den
geometrischen Vielfachheiten überein.

Jede reelle symmetrische Matrix $A$ ist diagonalisierbar, d.\,h. es gibt
eine invertierbare Matrix $T$ und eine Diagonalmatrix $D$, so dass
$A=TDT^{-1}$ gilt.

Sei $V$ ein $K$-Vektorraum und $(b_k)_{k=1}^n$ eine Basis von $V$.
Für jede symmetrische Bilinearform\index{symmetrische Bilinearform}
$f\colon V^2\to K$ ist die
Darstellungsmatrix
\begin{equation}
A = (f(b_i,b_j))
\end{equation}
symmetrisch. Ist $A\in K^{n\times n}$ eine symmetrische Matrix, so
ist
\begin{equation}\label{eq:symmBf}
f(x,y) = x^T A y.
\end{equation}
eine symmetrische Bilinearform für  $x,y\in K^n$.
Ist $K=\R$ und $A$ positiv definit, so ist
\eqref{eq:symmBf} ein Skalarprodukt auf $\R^n$.

\subsubsection{Reguläre Matrizen}\index{inverse Matrix}
Eine quadratische Matrix $A\in K^{n\times n}$ heißt \emdef{regulär}
oder \emdef{invertierbar}, wenn es eine inverse Matrix $A^{-1}$ gibt,
so dass
\begin{equation}
A^{-1}A = E_n \quad (\iff AA^{-1} = E_n)
\end{equation}
gilt, wobei mit $E_n$ die Einheitsmatrix gemeint ist. Jede
reguläre Matrix besitzt genau eine inverse Matrix. Eine Matrix $A$
ist genau dann regulär, wenn $\det(A)\ne 0$ gilt. Die Menge
der regulären Matrizen bildet bezüglich Matrizenmultiplikation
eine Gruppe, die
\emdef{allgemeine lineare Gruppe}\index{allgemeine lineare Gruppe}
\begin{equation}
\operatorname{GL}(n,K) := \{A\in K^{n\times n}\mid\det(A)\ne 0\}.
\end{equation}
Ist $V$ ein Vektorraum über dem Körper $K$, so bilden die
Automorphismen bezüglich Verkettung eine Gruppe, die
\emph{Automorphismengruppe}
\begin{equation}
\operatorname{GL}(V) = \operatorname{Aut}(V).
\end{equation}
Ein \emdef{Endomorphismus}\index{Endomorphismus!auf einem Vektorraum}
ist eine lineare Abbildung, welche eine Selbstabbildung ist.
Ein \emdef{Automorphismus}\index{Automorphismus!auf einem Vektorraum}
ist eine bijektiver Endomorphismus.

Wählt man auf $V$ eine Basis
$B$, so ist die Zuordnung der Darstellungsmatrix
\begin{equation}
M_B^B\colon \operatorname{Aut}(V)\to\operatorname{GL}(\dim V,K)
\end{equation}
eine Gruppenisomorphismus.

Eine quadratische Matrix, die nicht regulär ist, heißt
\emdef{singulär}. Endomorphismen, die nicht bijektiv sind, lassen
die Dimension ihrer Definitionsmenge schrumpfen:
\begin{equation}
f{\in}\operatorname{End}(V){\setminus}\operatorname{Aut}(V)
\Longleftrightarrow \dim f(V)<\dim V.
\end{equation}
Für Matrizen $A\in K^{n\times n}$ bedeutet das, dass sie nicht
den vollen Rang besitzen:
\begin{equation}
\det A=0\iff \operatorname{rk}(A) < n = \dim K^n.
\end{equation}
Inversionsformel:
\begin{equation}
\begin{bmatrix}
a & b\\
c & d
\end{bmatrix}^{-1}
= \frac{1}{ad-bc}\begin{bmatrix}
d & -b\\
-c & a
\end{bmatrix}.
\end{equation}
\begin{Definition}
Wird in der Matrix $A$ die Zeile $i$ und die Spalte $j$ entfernt,
so entsteht eine neue Matrix $[A]_{ij}$, die
\emdef{Streichungsmatrix}\index{Streichungsmatrix}
von $A$ genannt wird.
\end{Definition}
Laplacescher Entwicklungssatz:
\begin{align}
\det A = \sum_{i=1}^n (-1)^{i+j}a_{ij}\det([A]_{ij}),\\
\det A = \sum_{j=1}^n (-1)^{i+j}a_{ij}\det([A]_{ij}).
\end{align}

\subsection{Determinanten}\index{Determinante}
Für Matrizen $A,B\in K^{n\times n}$ und $r\in K$ gilt:
\begin{gather}
\det(AB) = \det(A)\det(B),\\
\det(A^T) = \det(A),\\
\det(rA) = r^n\det(A),\\
\det(A^{-1}) = \det(A)^{-1}.
\end{gather}
Für eine Diagonalmatrix $D=\diag(d_1,\ldots,d_n)$ gilt:
\begin{gather}
\det(D) = \prod_{k=1}^n d_k.
\end{gather}
Eine linke Dreiecksmatrix ist eine Matrix der Form
$(a_{ij})$ mit $a_{ij}=0$ für $i<j$. Eine rechte
Dreiecksmatrix ist die Transponierte einer linken
Dreiecksmatrix.

Für eine linke oder rechte Dreiecksmatrix $A=(a_{ij})$ gilt:
\begin{gather}
\det(A) = \prod_{k=1}^n a_{kk}.
\end{gather}

\subsection{Eigenwerte}
\strong{Eigenwertproblem:}\index{Eigenwert}
Für eine gegebene quadratische Matrix $A$ bestimme
\begin{equation}
\{(\lambda,v)\mid Av = \lambda v,\,v\ne 0\}.
\end{equation}
Das homogene lineare Gleichungssystem
\begin{equation}
Av=\lambda v \iff (A-\lambda E_n)v=0
\end{equation}
besitzt Lösungen $v\ne 0$ gdw.
\begin{equation}
p(\lambda)=\det(A-\lambda E_n)=0.
\end{equation}
Bei $p(\lambda)$ handelt es sich um ein normiertes Polynom
vom Grad $n$, das \emdef{charakeristisches Polynom}
\index{charakteristisches Polynom}
genannt wird.

\strong{Eigenraum:}\index{Eigenraum}
\begin{equation}
\Eig(A,\lambda):=\{v\mid Av=\lambda v\}.
\end{equation}
Die Dimension $\dim\Eig(A,\lambda)$ wird
\emdef{geometrische Vielfachheit}\index{geometrische Vielfachheit}
von $\lambda$ genannt.

\strong{Spektrum:}\index{Spektrum}
\begin{equation}
\sigma(A) := \{\lambda\mid \exists v\ne 0\colon Av=\lambda v\}.
\end{equation}

\section{Lineare Gleichungssysteme}
\index{lineares Gleichungssytem}
Ein lineares Gleichungssystem mit $m$ Gleichungen und $n$ Unbekannten
hat die Form:
\begin{equation}\label{eq:LGS}
\begin{split}
a_{11} x_1 + a_{12} x_2 + \ldots + a_{1n} x_n &= b_1,\\
a_{21} x_1 + a_{22} x_2 + \ldots + a_{2n} x_n &= b_2,\\
&\;\;\vdots\\
a_{m1} x_1 + a_{m2} x_2 + \ldots + a_{mn} x_n &= b_n.
\end{split}
\end{equation}
Das System lässt sich durch
\begin{equation}
A:=\begin{bmatrix}
a_{11} & a_{12} & \ldots & a_{1n}\\
a_{21} & a_{22} & \ldots & a_{2n}\\
\vdots & \vdots & \ddots & \vdots\\
a_{m1} & a_{m1} & \ldots & a_{mn}
\end{bmatrix}
\end{equation}
und
\begin{equation}
x:=\begin{bmatrix}
x_1 \\ x_2 \\ \vdots \\ x_n
\end{bmatrix},\quad
b:=\begin{bmatrix}
b_1 \\ b_2 \\ \vdots \\ b_n
\end{bmatrix}
\end{equation}
zusammenfassen.

Äquivalente Matrixform von \eqref{eq:LGS}:
\begin{equation}
Ax=b.
\end{equation}
Erweiterte Koeffizientenmatrix:\index{erweiterte Koeffizientenmatrix}
\begin{equation}
(A\,|\,b) := \left[\begin{array}{ccc|c}
a_{11} & \ldots & a_{1n} & b_1\\
\vdots & \ddots & \vdots & \vdots\\
a_{m1} & \ldots & a_{mn} & b_n
\end{array}\right].
\end{equation}
Lösungskriterium:
\begin{equation}
\exists x[Ax=b] \iff \rg(A)=\rg(A\,|\,b).
\end{equation}
Eindeutige Lösung (bei $n$ Unbekannten):
\begin{equation}
\exists! x[Ax=b] \iff\rg(A)=\rg(A\,|\,b)=n.
\end{equation}
Im Fall $m=n$ gilt:
\begin{equation}
\begin{split}
&\exists! x[Ax=b] \iff A\in\operatorname{GL}(n,K)\\
&\iff \rg(A)=n \iff \det(A)\ne 0.
\end{split}
\end{equation}

\newpage
\section{Multilineare Algebra}
\subsection{Äußeres Produkt}
Sei $V$ ein Vektorraum und sei $v_k\in V$ für alle $k$.

Sind $a=\sum_{k=1}^n a_k v_k$
und $b=\sum_{k=1}^n b_k v_k$ beliebige
Linearkombinationen, so gilt
\begin{equation}
\begin{split}
a\wedge b &= \sum_{i,j} a_i b_j\,v_i\wedge v_j\\
&= \sum_{1\le i<j\le n} (a_i b_j-a_j b_i)\,v_i\wedge v_j
\end{split}
\end{equation}
und
\begin{equation}
\begin{split}
a\wedge b &= a\otimes b-b\otimes a\\
&= \sum_{i,j} (a_i b_j-a_j b_i)\,v_i\otimes v_j\\
&= \sum_{i,j} a_i b_j (v_i\otimes v_j-v_j\otimes v_i).
\end{split}
\end{equation}
\subsubsection{Alternator}\index{Alternator}
Für $a_k\in V$ ist
$\operatorname{Alt}_p\colon T^p(V)\to A^p(V)\subseteq T^p(V)$
mit
\begin{equation}
\begin{split}
& \operatorname{Alt}_p (a_1\otimes\ldots\otimes a_p)\\
&:= \frac{1}{p!}\sum_{\sigma\in S_{\scriptstyle p}}
\sgn(\sigma)\,(a_{\sigma(1)}\otimes\ldots\otimes a_{\sigma(p)}).
\end{split}
\end{equation}
Es ist $\Lambda^p(V)$ isomorph zu $A^p(V)$ und man setzt:
\begin{equation}
a_1\wedge\ldots\wedge a_p
= p!\operatorname{Alt}_p(a_1\otimes\ldots\otimes a_p).
\end{equation}
Speziell gilt
\begin{equation}
\operatorname{Alt}_2 (a\otimes b) := \frac{1}{2}(a\otimes b-b\otimes a).
\end{equation}
und
\begin{equation}
a\wedge b = 2\operatorname{Alt}_2(a\otimes b).
\end{equation}

\subsubsection{Äußere Algebra}\index{aussere Algebra@äußere Algebra}
Darstellung als Quotientenraum:
\begin{equation}
\Lambda^2(V) = T^2(V)/\{v\otimes v\mid v\in V\}.
\end{equation}
Dimension: Ist $\dim(V)=n$, so gilt
\begin{equation}
\dim(\Lambda^k(V)) = \binom{n}{k}.
\end{equation}

\clearpage
\section{Analytische Geometrie}
\subsection{Geraden}\index{Gerade}
\subsubsection{Parameterdarstellung}
\index{Parameterdarstellung!einer Geraden}

\strong{Punktrichtungsform:}\index{Punktrichtungsform}
\begin{equation}
p(t) = p_0+t\underline v,
\end{equation}
$p_0$: Stützpunkt, $\underline v$: Richtungsvektor.
Die Gerade ist dann die Menge $g=\{p(t)\mid t\in\R\}$.

Der Vektor $\underline v$ repräsentiert außerdem die Geschwindigkeit,
mit der diese Parameterdarstellung durchlaufen wird:
$p'(t)=\underline v$.

\strong{Gerade durch zwei Punkte:}
Sind zwei Punkte $p_1,p_2$ mit $p_1\ne p_2$ gegeben, so ist
durch die beiden Punkte eine Gerade gegeben. Für diese Gerade ist
\begin{equation}
p(t) = p_1+t(p_2-p_1)
\end{equation}
eine Punktrichtungsform\index{Punktrichtungsform}.
Durch Umformung ergibt sich die \strong{Zweipunkteform:}
\begin{equation}\label{eq:Zweipunkteform}
p(t) = (1-t)p_1+tp_2.
\end{equation}
Bei \eqref{eq:Zweipunkteform} handelt es sich um eine
Affinkombination. Gilt $t\in[0,1]$, so ist \eqref{eq:Zweipunkteform}
eine Konvexkombination: eine Parameterdarstellung für die Strecke
von $p_1$ nach $p_2$.

\subsubsection{Parameterfreie Darstellung}
\strong{Hesse-Form:}
\begin{equation}\label{eq:Hesse-Form}
g = \{p\mid\langle \uv n,p-p_0\rangle = 0\},
\end{equation}
$p_0$: Stützpunkt, $\uv n$: Normalenvektor.

Die Hesse-Form ist nur in der Ebene möglich.
Form \eqref{eq:Hesse-Form} hat in Koordinaten
die Form
\begin{equation}
\begin{split}
g &= \{(x,y)\mid n_x(x-x_0)+n_y(y-y_0)=0\}\\
&= \{(x,y)\mid n_x x+n_y y = n_x x_0+n_y y_0\}.
\end{split}
\end{equation}

\strong{Hesse-Normalform:} \eqref{eq:Hesse-Form} mit $|\uv n|=1$.


Sei $\uv v\wedge\uv w$ das äußere Produkt.

\strong{Plückerform:}
\begin{equation}
g = \{p\mid (p-p_0)\wedge \underline v=0\}.
\end{equation}
Die Größe $\underline m = p_0\wedge\underline v$ heißt Moment.
Beim Tupel $(\underline v:\underline m)$ handelt es sich um
Plückerkoordinaten für die Gerade.

In der Ebene gilt speziell:
\begin{equation}\label{eq:Gerade-Ebene}
g = \{(x,y)\mid (x-x_0)\Delta y = (y-y_0)\Delta x\}
\end{equation}
mit $\underline v=(\Delta x,\Delta y)$.

Sei $a:=\Delta y$ und $b:=-\Delta x$ und $c:=ax_0+by_0$.
Aus \eqref{eq:Gerade-Ebene} ergibt sich:
\begin{equation}
g = \{(x,y)\mid ax+by=c\}.
\end{equation}
Im Raum ergibt sich ein Gleichungssystem:
\begin{equation}
g = \{\begin{pmatrix}x\\ y\\ z\end{pmatrix}
\mid
\begin{vmatrix}
(x-x_0)\Delta y = (y-y_0)\Delta x\\
(y-y_0)\Delta z = (z-z_0)\Delta y\\
(x-x_0)\Delta z = (z-z_0)\Delta x
\end{vmatrix}\}
\end{equation}
mit $\underline v=(\Delta x,\Delta y,\Delta z)$.

\subsubsection{Abstand Punkt zu Gerade}
Sei $p(t):=p_0+t\underline v$ die Punktrichtungsform einer Geraden und
sei $q$ ein weiterer Punkt. Bei $\underline d(t):=p(t)-q$ handelt
es sich um den Abstandsvektor in Abhängigkeit von $t$.

Ansatz: Es gibt genau ein $t$, so dass gilt:
\begin{equation}
\langle\underline d,\underline v\rangle=0.
\end{equation}
Lösung:
\begin{equation}
t = \frac
  {\langle\underline v,q{-}p_0\rangle}
  {\langle\underline v,\underline v\rangle}.
\end{equation}

\subsection{Ebenen}\index{Ebene}
\subsubsection{Parameterdarstellung}
\index{Parameterdarstellung!einer Ebene}
Seien $\uv u, \uv v$ zwei nicht kollineare Vektoren.

Punktrichtungsform:
\begin{equation}\label{eq:Ebene-Punktrichtungsform}
p(s,t) = p_0+s\uv u+t\uv v.
\end{equation}

\subsubsection{Parameterfreie Darstellung}
Seien $\uv v, \uv w$ zwei nicht kollineare Vektoren.
Durch
\begin{equation}
E = \{p\mid (p-p_0)\wedge\uv v\wedge\uv w=0\}.
\end{equation}
wird eine Ebene beschrieben.

\strong{Hesse-Form:}
\begin{equation}
E = \{p\mid \langle\uv n,p-p_0\rangle=0\},
\end{equation}
$p_0$: Stützpunkt, $\uv n$: Normalenvektor. Die Hesse-Form einer
Ebene ist nur im dreidimensionalen Raum möglich.
Den Normalenvektor bekommt man aus \eqref{eq:Ebene-Punktrichtungsform}
mit $\uv n = \uv u\times\uv v$.

\subsubsection{Abstand Punkt zu Ebene}
Sei $p(s,t):=p_0+s\uv u+t\uv v$ die Punktrichtungsform einer Ebene
und sei $q$ ein weiterer Punkt. Bei $\uv d(s,t):=p-q$ handelt es sich um
den Abstandsvektor in Abhängigkeit von $(s,t)$.

Ansatz: Es gibt genau ein Tupel $(s,t)$, so dass gilt:
\begin{equation}
\langle\uv d,\uv u\rangle=0\enspace\text{und}\enspace
\langle\uv d,\uv v\rangle=0.
\end{equation}
Lösung: Es ergibt sich ein LGS:
\begin{equation}
\begin{bmatrix}
\langle\uv u,\uv v\rangle & \langle\uv v,\uv v\rangle\\
\langle\uv v,\uv v\rangle & \langle\uv u,\uv v\rangle
\end{bmatrix}
\begin{bmatrix}
s\\ t
\end{bmatrix}
= \begin{bmatrix}
\langle\uv v,q{-}p_0\rangle\\
\langle\uv u,q{-}p_0\rangle
\end{bmatrix}.
\end{equation}
Bemerkung: Die Systemmatrix $g_{ij}$ ist der metrische Tensor für die
Basis $B=(\uv u,\uv v)$. Die Lösung des LGS ist:
\begin{gather}
s = \frac
  {\langle g_{12}\uv v-g_{12}\uv u, q{-}p_0\rangle}
  {g_{11}^2-g_{12}^2},\\
t = \frac
  {\langle g_{12}\uv u-g_{12}\uv v, q{-}p_0\rangle}
  {g_{11}^2-g_{12}^2}.
\end{gather}



\chapter{Differentialgeometrie}
\section{Kurven}
\subsection{Parameterkurven}\index{Weg}\index{Kurve}
\strong{Definition.} Sei $X$ ein topologischer Raum und
$I$ ein reelles Intervall, auch offen oder halboffen, auch unbeschränkt.
Eine stetige Funktion
\begin{equation}
f\colon I\to X
\end{equation}
heißt \emdef{Parameterdarstellung einer Kurve}, kurz
\emdef{Parameterkurve}. Die Bildmenge $f(I)$ heißt \emdef{Kurve}.

Eine Parameterdarstellung mit einem kompakten Intervall $I=[a,b]$
heißt \emdef{Weg}.

Für einen Weg mit $I=[a,b]$ heißt $f(a)$ \emdef{Anfangspunkt}
und $f(b)$ \emdef{Endpunkt}. Ein Weg mit $f(a)=f(b)$
heißt \emdef{geschlossen}. Ein Weg dessen Einschränkung auf $[a,b)$
injektiv ist, heißt \emdef{einfach}, auch \emdef{doppelpunktfrei} oder
\emdef{Jordan-Weg}.

Bsp. für einen einfachen geschlossenen Weg:
\begin{equation}
f\colon [0,2\pi]\to \R^2,\quad
f(t):=\begin{bmatrix}
\cos t\\
\sin t
\end{bmatrix}.
\end{equation}
Die Kurve ist der Einheitskreis.

Bsp. für einen geschlossenen Weg mit Doppelpunkt:
\begin{equation}
f\colon [0,2\pi]\to \R^2,\quad
f(t):=\begin{bmatrix}
2\cos t\\
\sin(2t)
\end{bmatrix}.
\end{equation}
Die Kurve ist eine Achterschleife.

\subsection{Differenzierbare Parameterkurven}
\strong{Definition.} Eine Parameterkurve $f\colon (a,b)\to\R^n$ heißt
\emdef{differenzierbar}, wenn die Ableitung $f'(t)$ an jeder Stelle
$t$ existiert. Die Ableitung $f'(t)$ wird
\emdef{Tangentialvektor} an die Kurve an der Stelle $t$ genannt.

Ein \emdef{$C^k$-Kurve} ist ein Parameterkurve, dessen $k$-te Ableitung
eine stetige Funktion ist. Ein unendlich oft differenzierbare
Parameterkurve heißt \emdef{glatt}.

Eine Parameterkurve heißt \emdef{regulär}, wenn:
\begin{equation}
\forall t\colon f'(t)\ne 0.
\end{equation}








\chapter{Funktionentheorie}
\section{Holomorphe Funktionen}

\begin{definition}[Holomorphe Funktion]
Sei $U\subseteq\C$ eine offene Menge und $f\colon U\to\C$.
Die Funktion $f$ wird \emdef{holomorph}\index{holomorph} an der
Stelle $z_0\in U$ genannt, wenn der Grenzwert
\begin{equation}
f'(z_0) := \lim_{z\to z_0} \frac{f(z)-f(z_0)}{z-z_0}
\end{equation}
existiert.
\end{definition}

\noindent
Das Argument und Bild von $f$ werden nun in Real- und Imaginärteil
zerlegt. Das sind die Zerlegungen $z=x+y\ui$ und $f(z)=u(x,y)+v(x,y)\ui$.
Die Funktion $f(z)$ ist genau dann holomorph an der Stelle
$z_0=x_0+y_0\ui$, wenn bei $(x_0,y_0)$ die partiellen Ableitungen
stetig sind und die \emdef{Cauchy-Riemann-Gleichungen}
\begin{equation}\label{eq:Cauchy-Riemann-Gleichungen}
\frac{\partial u}{\partial x}=\frac{\partial v}{\partial y},
\quad \frac{\partial u}{\partial y}=-\frac{\partial v}{\partial x}
\qquad\text{bei}\;(x_0,y_0)
\end{equation}
gelten. Bei
\begin{equation}
\uv v := (u,-v) = (v_x,v_y) = v_x\uv e{}_x+v_y\uv e{}_y
\end{equation}
handelt es sich um ein Vektorfeld
auf dem Koordinatenraum. Die Gleichungen
\eqref{eq:Cauchy-Riemann-Gleichungen} lassen sich nun als
Quellenfreiheit
\begin{equation}
0=\langle\nabla,\uv v\rangle = \frac{\partial v_x}{\partial x}+\frac{\partial v_y}{\partial y}
\end{equation}
und Rotationsfreiheit
\begin{equation}
0=\nabla\wedge \uv v = \bigg(\frac{\partial v_y}{\partial x}
-\frac{\partial v_x}{\partial y}\bigg)\,\uv e{}_x\wedge\uv e{}_y
\end{equation}
interpretieren.

Für das totale Differential
\begin{equation}
\mathrm df = \frac{\partial f}{\partial x}\mathrm dx+\frac{\partial f}{\partial y}\mathrm dy
\end{equation}
gibt es die Umformulierung
\begin{equation}\label{eq:Differential-Wirtinger-Operatoren}
\mathrm df = \frac{\partial f}{\partial z}\mathrm dz+\frac{\partial f}{\partial\overline z}\mathrm d\overline z.
\end{equation}
Hierbei ist $\mathrm dz=\mathrm dx+\ui\,\mathrm dy$ und $\mathrm d\overline{z}=\mathrm dx-\ui\,\mathrm dy$.

Die Ableitungsoperatoren
\begin{align}
\frac{\partial f}{\partial z}
&:= \frac{1}{2}\bigg(\frac{\partial f}{\partial x}-\ui\frac{\partial f}{\partial y}\bigg),\\
\frac{\partial f}{\partial\overline z}
&:= \frac{1}{2}\bigg(\frac{\partial f}{\partial x}+\ui\frac{\partial f}{\partial y}\bigg)
\end{align}
mit $\partial f=\partial u+\ui\,\partial v$ heißen \emdef{Wirtinger-Operatoren}.

Die Gleichungen \eqref{eq:Cauchy-Riemann-Gleichungen} lassen sich nun
zur Gleichung%
\begin{equation}\label{eq:Cauchy-Riemann-Wirtinger}
\frac{\partial f}{\partial\overline z}(z_0)=0
\end{equation}
zusammenfassen. Für holomorphe Funktionen reduziert sich das
Differential \eqref{eq:Differential-Wirtinger-Operatoren} wegen
\eqref{eq:Cauchy-Riemann-Wirtinger} auf die Form%
\begin{equation}
\mathrm df = \frac{\partial f}{\partial z}\mathrm dz.
\end{equation}

\newpage
\section{Harmonische Funktionen}
\begin{definition}[Harmonische Funktion]
Sei $U\subseteq\R^2$ eine offene Menge.
Eine Funktion $\Phi\colon U\to\R$ heißt \emdef{harmonisch}
an der Stelle $(x_0,y_0)$, wenn die \emdef{Laplace-Gleichung}
$(\Delta\Phi)(x_0,y_0)=0$ mit dem \emdef{Laplace-Operator}%
\begin{equation}
\Delta\Phi := \frac{\partial^2\Phi}{\partial x\partial x}+\frac{\partial^2\Phi}{\partial y\partial y}
\end{equation}
erfüllt ist.
\end{definition}

\noindent
Ist $f=u+v\ui$ an der Stelle $z_0$ holomorph, so sind der
Realteil $u$ und der Imaginärteil $v$
an der Stelle $(x_0,y_0)=(\real z_0,\imag z_0)$ harmonisch.
Das heißt es gilt%
\begin{equation}
(\Delta u)(x_0,y_0) = 0,\quad (\Delta v)(x_0,y_0)=0.
\end{equation}
Ist eine Funktion $u$ auf einem einfach zusammenhängenden Gebiet
harmonisch, so lässt sich stets eine harmonische Funktion $v$
finden, so dass $f=u+v\ui$ holomorph ist. Die Funktion $v$ ist
bis auf eine additive reelle Konstante $c$ eindeutig bestimmt.
Das heißt, $v$ darf auch durch $v+c$ ersetzt werden.

Die Funktion $v$ wird die \emdef{harmonisch Konjugierte}
zu $u$ genannt. An jeder Stelle $(x_0,y_0)$ treffen die Linien%
\begin{align}
&\{(x,y)\mid u(x,y)=u(x_0,y_0)\},\\
&\{(x,y)\mid v(x,y)=v(x_0,y_0)\} 
\end{align}
senkrecht aufeinander.

\section{Wegintegrale}
\strong{Integral einer komplexwertigen Funktion.}\\
Für $f\colon [a,b]\to\C$ mit $f=u+\ui v$ ist
\begin{equation}
\int_a^b f(t)\,\mathrm dt
= \int_a^b u(t)\,\mathrm dt+\ui\int_a^b v(t)\,\mathrm dt,
\end{equation}
wenn $u$ und $v$ integrierbar sind.

\begin{definition}[Kurvenintegral]
Für $f\colon U\to\C$ mit $U\subseteq\C$:%
\begin{equation}
\int_\gamma f(z)\,\mathrm dz := \int_a^b f(\gamma(t))\,\gamma'(t)\,\mathrm dt,
\end{equation}
wobei $\gamma\colon [a,b]\to U$ ein (zumindest stückweise)
differenzierbarer Weg \eqref{eq:Parameterkurve} ist.
\end{definition}

\noindent
\strong{Integralsatz von Cauchy.}
Ist $U$ ein einfach zusammenhängendes Gebiet und $f\colon U\to\C$
holomorph, so gilt für jeden Weg $\gamma$ von $\gamma(a)$ nach
$\gamma(b)$ die Formel%
\begin{equation}
\int_\gamma f(z)\,\mathrm dz = F(\gamma(b))-F(\gamma(a)),
\end{equation}
wobei die Funktion $F$ nicht vom gewählten Weg abhängig ist.
Außerdem ist $F$ eine Stammfunktion zu $f$, das heißt es gilt
$F'(z)=f(z)$ für alle $z\in U$.

Sind die Voraussetzungen für den Integralsatz erfüllt,
dann motiviert Wegunabhängigkeit die Definition%
\begin{equation}
\int_{z_1}^{z_2} f(z)\,\mathrm dz := F(z_2)-F(z_1),
\end{equation}
bei der auf Wege gänzlich verzichtet wird.



\chapter{Dynamische Systeme}
\section{Grundbegriffe}

\begin{definition}[Dynamisches System]
Ein Tupel $(T,M,\Phi)$ mit $\Phi\colon T\times M\to M$ heißt
\emdef{dynamisches System}\index{dynamisches System},
wenn für alle $t_1,t_2\in T$ und $x\in M$ gilt:
\begin{align}
&\Phi(0,x)=x,\\
&\Phi(t_2,\Phi(t_1,x)) = \Phi(t_1+t_2,x).
\end{align}
Die Menge $T$ heißt \emdef{Zeitraum}.
Ein System mit $T=\N_0$ oder $T=\Z$ heißt \emdef{zeitdiskret},
eines mit $T=\R_0^{+}$ oder $T=\R$ heißt \emdef{zeitkontinuierlich}.
Ein System mit $T=\Z$ oder $T=\R$ heißt \emdef{invertierbar}.

Die Menge $M$ heißt \emdef{Zustandsraum}\index{Zustandsraum},
ihre Elemente werden \emdef{Zustände}\index{Zustand} genannt.

Für ein invertierbares System handelt es sich bei $\Phi$
um eine Gruppenaktion (s. \ref{Gruppenaktion})
der additiven Gruppe $(T,+)$.

Die Menge
\begin{equation}
\Phi(T,x) := \{\Phi(t,x)\mid t\in T\}
\end{equation}
heißt \emdef{Orbit}\index{Orbit!unter einem dynamischen System}
von $x$. S.\,a. \eqref{eq:Orbit}.
\end{definition}

\section{Iterationen}

\begin{definition}[Iteration]
Für eine Selbstabbildung $\varphi\colon M\to M$ lassen sich
die \emdef{Iterationen} gemäß
\begin{equation}
\varphi^0:=\id,\quad \varphi^n:=\varphi^{n-1}\circ\varphi
\end{equation}
formulieren. Mit $\id$ ist die identische Abbildung
\begin{equation}
\id\colon M\to M,\quad \id(x):=x
\end{equation}
und mit $g\circ f$ die Komposition \eqref{eq:composition} gemeint.
Für ein bijektives $\varphi$ wird zusätzlich
\begin{equation}
\varphi^{-n}:=(\varphi^{-1})^n
\end{equation}
definiert.
\end{definition}

\noindent
Die Iterationen bilden ein dynamisches System gemäß%
\begin{equation}
\Phi(n,x):=\varphi^n(x),\quad\Phi\colon\N_0\times M\to M.
\end{equation}
Bei einem bijektiven $\varphi$ lässt sich das System zum invertierbaren
System
\begin{equation}
\Phi(n,x):=\varphi^n(x),\quad\Phi\colon\Z\times M\to M
\end{equation}
erweitern.

\begin{definition}[Kompositionsoperator]
Für eine Funktion $\varphi\colon A\to A$ wird der Operator
\begin{equation}
C_\varphi (g) := g\circ\varphi,\quad C_\varphi\colon B^A\to B^A
\end{equation}
\emdef{Kompositionsoperator}\index{Kompositionsoperator} genannt.
\end{definition}

\noindent
Wenn $B^A$ ein Funktionenraum ist, dann ist der Kompositionsoperator
ein linearer Operator.


\chapter{Kombinatorik}
\section{Kombinatorische Funktionen}
\subsection{Faktorielle}\index{Faktorielle}
\subsubsection{Fakultät}\index{Fakultät}
\strong{Definition.} Für $n\in\mathbb Z, n\ge 0$:
\begin{equation}
n! := \prod_{k=1}^n k.
\end{equation}
Rekursionsgleichung:
\begin{equation}
(n+1)! = n!\,(n+1)
\end{equation}
Die Gammafunktion ist eine Verallgemeinerung der Fakultät:
\begin{equation}
n! = \Gamma(n+1).
\end{equation}

\subsubsection{Fallende Faktorielle}
\strong{Definition.} Für $a\in\mathbb C$ und $k\ge 0$:
\begin{equation}\label{eq:FF}
a^{\underline k} := \prod_{j=0}^{k-1} (a-j).
\end{equation}
Für $n\ge k$ und $k\ge 0$ gilt:
\begin{equation}
n^{\underline k} = \frac{n!}{(n-k)!}.
\end{equation}
Für $a\in\mathbb C\setminus\{-1,-2,\ldots\}$
und $k\in\mathbb C$ gilt:
\begin{equation}
a^{\underline k} = \frac{\Gamma(a+1)}{\Gamma(a-k+1)}.
\end{equation}

\subsubsection{Steigende Faktorielle}
\strong{Definition.} Für $a\in\mathbb C$ und $k\ge 0$:
\begin{equation}
a^{\overline k} := \prod_{j=0}^{k-1} (a+j).
\end{equation}
Für $n\ge 1$ und $n+k\ge 1$ gilt:
\begin{equation}
n^{\overline k} = \frac{(n+k-1)!}{(n-1)!}.
\end{equation}

\subsection{Binomialkoeffizienten}\index{Binomialkoeffizient}
\strong{Definition.} Für $a\in\mathbb C$
und $k\in\mathbb Z$:
\begin{equation}
\binom{a}{k} := \begin{cases}
\frac{a^{\underline k}}{k!} & \text{wenn}\;k>0,\\
1 & \text{wenn}\;k=0,\\
0 & \text{wenn}\;k<0.
\end{cases}
\end{equation}
Für $a\in\mathbb C\setminus\{-1,-2,\ldots\}$ und $b\in\mathbb C$:
\begin{equation}\label{eq:bc-allg}
\binom{a}{b} := \frac{\Gamma(a+1)}{\Gamma(b+1)\Gamma(a-b+1)}.
\end{equation}
Es gilt die Symmetriebeziehung
\begin{equation}
\binom{a}{b} = \binom{a}{a-b}
\end{equation}
und die Rekursionsgleichung
\begin{equation}
\binom{a+1}{b+1} = \binom{a}{b+1}+\binom{a}{b}.
\end{equation}
Für $a\in\mathbb C$ und $k\in\mathbb Z$ gilt:
\begin{equation}
\binom{-a}{k} = (-1)^k \binom{a+k-1}{k}.
\end{equation}

\section{Formale Potenzreihen}
\subsection{Binomische Reihe}
\strong{Definition.} Für $a\in\mathbb C$:
\begin{equation}
(1+X)^a := \sum_{k=0}^\infty \binom{a}{k} X^k
\end{equation}
Es gilt:
\begin{equation}
(1+X)^{a+b} = (1+X)^a (1+X)^b 
\end{equation}
und
\begin{equation}
(1+X)^{ab} = ((1+X)^a)^b.
\end{equation}


\chapter{Algebra}
\section{Gruppentheorie}
\subsection{Grundbegriffe}
\begin{definition}[Gruppenhomomorphismus]
Sind $(G,*)$ und $(H,\bullet)$ zwei Gruppen, so
heißt $\varphi\colon G\to H$ \emdef{Gruppenhomomorphismus}%
\index{Gruppenhomomorphismus}, wenn
\begin{gather}
\forall g_1,g_2\in G\colon
  \varphi(g_1*g_2) = \varphi(g_1)\bullet\varphi(g_2)
\end{gather}
gilt. Ein \emdef{Gruppenisomorphismus}\index{Isomorphismus!zwischen Gruppen}
ist ein bijektiver Gruppenhomomorphismus, da die Umkehrabbildung
auch wieder ein Gruppenhomomorphismus ist.
\end{definition}
\begin{definition}[Direktes Produkt]
\emdef{Direktes Produkt}\index{direktes Produkt}:
\begin{gather}
G\times H := \{(g,h)\mid g\in G, h\in H\},\\
(g_1,h_1)*(g_2,h_2) := (g_1*g_2, h_1*h_2).
\end{gather}
\end{definition}
\noindent
\strong{Satz von Lagrange.} Für Gruppen $G,H$ gilt:
\begin{equation}
H\le G\implies |G| = |G/H|\cdot |H|.
\end{equation}

\subsection{Gruppenaktionen}\label{Gruppenaktion}
\begin{definition}[Gruppenaktion]
Eine Funktion $f\colon G\times X\to X$ heißt
\emdef{Gruppenaktion}\index{Gruppenaktion}, wenn
\begin{gather}
\hspace{-1em}\forall g_1,g_2{\in}G, x{\in}X\colon f(g_1,f(g_2,x)) = f(g_1 g_2,x),\\
\hspace{-1em}\forall x\in X\colon f(e,x) = x
\end{gather}
gilt, wobei mit $e$ das neutrale Element von $G$ gemeint ist.
Anstelle von $f(g,x)$ wird üblicherweise kurz $gx$ (oder
$g+x$ bei einer Gruppe $(G,+)$) geschrieben.

Anstelle von \emdef{Linksaktionen} kommen auch \emdef{Rechtsaktionen}
vor, die sich von Linksaktionen in der Reihenfolge unterscheiden.
Eine Rechtsaktion $f\colon X\times G\to X$ genügt den Regeln
\begin{gather}
\hspace{-1em}\forall g_1,g_2{\in}G, x{\in}X\colon f(f(x,g_1),g_2) = f(x,g_1 g_2),\\
\hspace{-1em}\forall x\in X\colon f(x,e) = x.
\end{gather}
\end{definition}

\begin{definition}[Orbit, Stabilisator]
Für ein $x\in X$ wird
\begin{equation}\label{eq:Orbit}
Gx := \{gx\mid g\in G\}
\end{equation}
\emdef{Bahn}\index{Bahn} oder
\emdef{Orbit}\index{Orbit!unter einer Gruppenaktion} genannt.
Die Menge
\begin{equation}
G_x := \{g\in G\mid gx=x\}
\end{equation}
wird \emdef{Fixgruppe}\index{Fixgruppe}
oder \emdef{Stabilisator}\index{Stabilisator} genannt.
Die Menge
\begin{equation}
X^g := \{x\in X\mid gx=x\}
\end{equation}
heißt \emdef{Fixpunktmenge}.
\end{definition}

\noindent
Fixgruppen sind immer Untergruppen:
\begin{equation}
\forall x\colon G_x\le G.
\end{equation}
Bahnen sind Äquivalenzklassen, die Quotientenmenge
\begin{equation}
X/G := \{Gx\mid x\in X\}
\end{equation}
wird \emdef{Bahnenraum}\index{Bahnenraum} genannt.

\strong{Bahnformel.}\index{Bahnformel}
Ist $G$ eine endliche Gruppe, so gilt:
\begin{equation}
|G| = |Gx|\cdot |G_x|.
\end{equation}
\strong{Lemma von Burnside.}\index{Lemma von Burnside}
Für eine endliche Gruppe $G$ gilt:%
\begin{equation}
|X/G| = \frac{1}{|G|}\sum_{g\in G}|X^g|.
\end{equation}

\section{Ringe}\index{Ring}
Ist $(R,+,*)$ ein Ring, so gilt für alle $a,b\in R$:
\begin{align}
0*a &= a*0 = 0,\\
(-a)*b &= a*(-b) = -(a*b),\\
(-a)*(-b) &= -(a*b).
\end{align}
\begin{definition}[Ringhomomorphismus]
Sind $(R,+,*)$ und $(R',+',*')$ Ringe, so wird
$\varphi\colon R\to R'$ als \emdef{Ringhomomorphismus}
bezeichnet, wenn
\begin{align}
\varphi(a+b) &= \varphi(a)+'\varphi(b),\\
\varphi(a*b) &= \varphi(a)*'\varphi(b),
\end{align}
für alle $a,b\in R$ gilt und $\varphi(1)=1$ ist.
\end{definition}

\subsection{Polynome}\index{Polynom}
\begin{definition}[Polynom, Polynomring, Koeffizienten]
Sei $R$ ein kommutativer unitärer Ring.
Mit $R[X]$ bezeichnen wir die Menge der unendlichen Folgen
\begin{equation}
(a_k) = (a_0,a_1,\ldots,a_n,0,0,0,\ldots)
\end{equation}
mit $a_k\in R$, bei denen ab einem Index alle Folgenglieder null sind.

Für zwei Folgen aus $R[X]$ wird nun die Addition
\begin{equation}
(a_k) + (b_k) := (a_k+b_k)
\end{equation}
und die Multiplikation
\begin{equation}\label{eq:Faltung}
(a_i)*(b_j) = \bigg(\sum_{i=0}^k a_i b_{k-i}\bigg)
\end{equation}
erklärt. In der Form \eqref{eq:Faltung} wird die Operation auch
\emdef{Faltung}\index{Faltung!von zwei Folgen}
der Folgen $(a_i)$ und $(b_j)$ genannt.

Die Menge $R[X]$ bildet mit der Addition und Multiplikation
einen kommutativen unitären Ring, den \emdef{Polynomring}
mit Koeffizienten in $R$. Ein Element von $R[X]$ wird
\emdef{Polynom} genannt.

Man definiert nun
\begin{equation}
X:=(0,1,0,0,0,\ldots),
\end{equation}
womit sich jedes Polynom in der Form
\begin{equation}\textstyle
(a_k) = \sum_{k=0}^n a_k X^k
\end{equation}
schreiben lässt. Die $a_k$ nennt man \emdef{Koeffizienten}
des Polynoms.
\end{definition}

\noindent
Die Addition bekommt nun die Form
\begin{equation}
\sum_{k=0}^m a_k X^k + \sum_{k=0}^n b_k X^k
:= \sum_{k=0}^p (a_k+b_k)X^k.
\end{equation}
mit $p=\max(m,n)$. Die Multiplikation lässt sich nun in der Form
\begin{equation}
\bigg(\sum_{i=0}^m a_i X^i\bigg)\bigg(\sum_{j=0}^n b_j X^j\bigg)
:= \sum_{k=0}^{m+n}\bigg(\sum_{i=0}^k a_i b_{k-i}\bigg) X^k.
\end{equation}
schreiben. Die Multiplikation von Polynomen ist das gewöhnlichen
Ausmultiplizieren der Polynome, wobei $X^i X^j=X^{i+j}$ gilt.

Die $X^k$ können als Vektorraumbasis betrachtet
werden und dienen dabei dazu, die $a_k$ auseinanderzuhalten.
Zwei Polynome $\sum_{k=0}^m a_k X^k$ und $\sum_{k=0}^n b_k X^k$
sind genau dann gleich, wenn $a_k=b_k$ für alle $k\le\max(m,n)$ gilt.

Da $R[X]$ wieder ein kommutativer unitärer Ring ist,
ist auch $R[X][Y]$ ein Polynomring. Man definiert
\begin{equation}
R[X,Y] := R[X][Y].
\end{equation}
Polynome aus $R[X,Y]$ lassen sich in der Form
\begin{equation}
\sum_{j=0}^n \bigg(\sum_{i=0}^m a_{ij}X^i\bigg)Y^j
= \sum_{i=0}^m\sum_{j=0}^n a_{ij} X^i Y^j
\end{equation}
mit $a_{ij}\in R$ schreiben.

Allgemein ist die Menge
\begin{equation}
R[X_1,\ldots,X_q] := X[X_1,\ldots,X_{q-1}][X_q]
\end{equation}
ein kommutativer unitärer Ring. Die Polynome lassen sich in der Form
\begin{equation}
\sum_{k\in\N_0^q} a_k X^k\quad (a_k\in R)
\end{equation}
mit
\[k=(k_1,\ldots,k_q)\quad\text{und}\quad X^k:=\prod_{i=1}^q X_i^{k_i}\]
schreiben.

\begin{definition}[Grad]
Für ein Polynom $f=\sum_{k=0}^n a_k X^k$ mit $a_n\ne 0$ wird
$n$ als \emdef{Grad} von $f$ bezeichnet. Man schreibt $n=\deg f$.

Für ein Monom $a_{ij} X^i Y^j$ mit $a_{ij}\ne 0$ heißt $i+j$
\emdef{Totalgrad}. Der \emdef {Grad} eines Polynoms
\begin{equation}
\textstyle\sum_{i=1}^m\sum_{j=1}^n a_{ij} X^i Y^j
\end{equation}
ist der maximale Totalgrad aller Monome mit $a_{ij}\ne 0$.
Für Polynome in mehr als zwei Variablen ist die Definition analog.
\end{definition}

\strong{Regeln.}\\
Für zwei Polynome $f,g\in R[X_1,\ldots,X_q]$ gilt:
\begin{align}
\deg(f+g)&\le \max(\deg f,\deg g),\\
\deg(fg)&\le (\deg f)(\deg g).
\end{align}
Für zwei Polynome $f,g$ mit $\deg f\ne\deg g$ gilt:
\begin{equation}
\deg(f+g) = \max(\deg f,\deg g).
\end{equation}
Ist $R$ ein Integritätsring, so gilt für $f,g\in R[X_1,\ldots,X_q]$:%
\begin{equation}
\deg(fg) = (\deg f)(\deg g).
\end{equation}
Jeder Körper, z.\,B. $\R$ oder $\C$ ist ein Integritätsring.
Auch die ganzen Zahlen $\Z$ bilden einen Integritätsring.
Ein Polynomring ist genau dann ein Integritätsring, wenn die
Koeffizienten aus einem Integritätsring entstammen.

\begin{definition}[Einsetzungshomomorphismus]
Seien $R,R'$ kommutative unitäre Ringe. Sei $R'$ eine Ringerweiterung
von $R$ und sei $r\in R'$. Die Abbildung $\varphi_r\colon R[X]\to R'$
mit
\begin{equation}\textstyle
\varphi_r(p) = p(r) := \sum_{k=0}^n a_k r^k 
\end{equation}
für jedes Polynom
\[\textstyle p = \sum_{k=0}^n a_k X^k\]
ist ein Ringhomomorphismus. Man bezeichnet $p(r)$ als \emdef{Einsetzung}
von $r$ in $p$ und $\varphi_r$ als
\emdef{Einsetzungshomomorphismus}\index{Einsetzungshomomorphimus}.
\end{definition}

Man kann auch $R'=R$ und $r=X$ setzen, dann gilt $p=p(X)$.
Ein Polynom stimmt also mit der Einsetzung seiner eigenen formalen
Variablen überein.

\begin{definition}[Polynomfunktion]
Für ein festes $p\in R[X]$ wird die Funktion
\begin{equation}
f\colon R'\to R',\quad f(x):=p(x)
\end{equation}
als \emdef{Polynomfunktion} bezeichnet.
\end{definition}

\noindent
In einigen Ringen können unterschiedliche Polynome zur selben
Polynomfunktion führen. Handelt es sich bei $R$ jedoch um einen
unendlichen Körper, z.\,B. $R=\R$ oder $R=\C$, dann gibt es zu jeder
Polynomfunktion nur ein einziges Polynom.

\section{Körper}
\begin{definition}[Körperhomomorphismus]
Sind $(K,+,\bullet)$ und $(K',+',\bullet')$ Körper, so
wird $\varphi\colon K\to K'$ als \emph{Körperhomomorphismus}
bezeichnet, wenn
\begin{align}
\varphi(a+b) &= \varphi(a)+'\varphi(b),\\
\varphi(a\bullet b) &= \varphi(a)\bullet'\varphi(b)
\end{align}
für alle $a,b\in K$ gilt und $\varphi(1)=1$ ist.
\end{definition}



\input{Wahrscheinlichkeitsrechnung.tex}

\onecolumn
\chapter{Tabellen}
\section{Kombinatorik}
\subsection{Binomialkoeffizienten}\index{Binomialkoeffizient!Tabelle}

\hspace{-6pt}\begin{tabular}{ll}
\begin{tabular}[t]{l|ccccccccccc}
\toprule
& $k=0$ & $k=1$ & $k=2$ & $k=3$ & $k=4$ & $k=5$ & $k=6$ & $k=7$ & $k=8$ & $k=9$ & $\hspace{-2pt}k=10\hspace{-2pt}$\\
\midrule%
$n=\phantom{1}0$ & 1 & 0 & 0 & 0 & 0 & 0 & 0 & 0 & 0 & 0 & 0\\
$n=\phantom{1}1$ & 1 & 1 & 0 & 0 & 0 & 0 & 0 & 0 & 0 & 0 & 0\\
$n=\phantom{1}2$ & 1 & 2 & 1 & 0 & 0 & 0 & 0 & 0 & 0 & 0 & 0\\
$n=\phantom{1}3$ & 1 & 3 & 3 & 1 & 0 & 0 & 0 & 0 & 0 & 0 & 0\\
\midrule%
$n=\phantom{1}4$ & 1 & 4 & 6 & 4 & 1 & 0 & 0 & 0 & 0 & 0 & 0\\
$n=\phantom{1}5$ & 1 & 5 & 10 & 10 & 5 & 1 & 0 & 0 & 0 & 0 & 0\\
$n=\phantom{1}6$ & 1 & 6 & 15 & 20 & 15 & 6 & 1 & 0 & 0 & 0 & 0\\
$n=\phantom{1}7$ & 1 & 7 & 21 & 35 & 35 & 21 & 7 & 1 & 0 & 0 & 0\\
\midrule%
$n=\phantom{1}8$ & 1 & 8 & 28 & 56 & 70 & 56 & 28 & 8 & 1 & 0 & 0\\
$n=\phantom{1}9$ & 1 & 9 & 36 & 84 & 126 & 126 & 84 & 36 & 9 & 1 & 0\\
$n=10$ & 1 & 10 & 45 & 120 & 210 & 252 & 210 & 120 & 45 & 10 & 1\\
$n=11$ & 1 & 11 & 55 & 165 & 330 & 462 & 462 & 330 & 165 & 55 & 11\\
\midrule%
$n=12$ & 1 & 12 & 66 & 220 & 495 & 792 & 924 & 792 & 495 & 220 & 66\\
$n=13$ & 1 & 13 & 78 & 286 & 715 & 1287 & 1716 & 1716 & 1287 & 715 & 286\\
$n=14$ & 1 & 14 & 91 & 364 & 1001 & 2002 & 3003 & 3432 & 3003 & 2002 & 1001\\
$n=15$ & 1 & 15 & 105 & 455 & 1365 & 3003 & 5005 & 6435 & 6435 & 5005 & 3003\\
\midrule%
$n=16$ & 1 & 16 & 120 & 560 & 1820 & 4368 & 8008 & 11440 & 12870 & 11440 & 8008\\
$n=17$ & 1 & 17 & 136 & 680 & 2380 & 6188 & 12376 & 19448 & 24310 & 24310 & 19448\\
$n=18$ & 1 & 18 & 153 & 816 & 3060 & 8568 & 18564 & 31824 & 43758 & 48620 & 43758\\
$n=19$ & 1 & 19 & 171 & 969 & 3876 & 11628 & 27132 & 50388 & 75582 & 92378 & 92378\\
\bottomrule
\end{tabular}
& \begin{tabular}[t]{l}
\\
$\dbinom{n}{k}$
\end{tabular}
\end{tabular}

\vspace{3em}\noindent
\begin{tabular}[t]{l|cccccccccc}
\toprule
& $k=0$ & $k=1$ & $k=2$ & $k=3$ & $k=4$ & $k=5$ & $k=6$ & $k=7$ & $k=8$ & $k=9$\\
\midrule%
$n=-15$ & $1$ & $-15$ & $120$ & $-680$ & $3060$ & $-11628$ & $38760$ & $-116280$ & $319770$ & $-817190$\\
$n=-14$ & $1$ & $-14$ & $105$ & $-560$ & $2380$ & $-8568$ & $27132$ & $-77520$ & $203490$ & $-497420$\\
$n=-13$ & $1$ & $-13$ & $91$ & $-455$ & $1820$ & $-6188$ & $18564$ & $-50388$ & $125970$ & $-293930$\\
$n=-12$ & $1$ & $-12$ & $78$ & $-364$ & $1365$ & $-4368$ & $12376$ & $-31824$ & $75582$ & $-167960$\\
\midrule%
$n=-11$ & $1$ & $-11$ & $66$ & $-286$ & $1001$ & $-3003$ & $8008$ & $-19448$ & $43758$ & $-92378$\\
$n=-10$ & $1$ & $-10$ & $55$ & $-220$ & $715$ & $-2002$ & $5005$ & $-11440$ & $24310$ & $-48620$\\
$n=\phantom{1}{-9}$ & $1$ & $-9$ & $45$ & $-165$ & $495$ & $-1287$ & $3003$ & $-6435$ & $12870$ & $-24310$\\
$n=\phantom{1}{-8}$ & $1$ & $-8$ & $36$ & $-120$ & $330$ & $-792$ & $1716$ & $-3432$ & $6435$ & $-11440$\\
\midrule%
$n=\phantom{1}{-7}$ & $1$ & $-7$ & $28$ & $-84$ & $210$ & $-462$ & $924$ & $-1716$ & $3003$ & $-5005$\\
$n=\phantom{1}{-6}$ & $1$ & $-6$ & $21$ & $-56$ & $126$ & $-252$ & $462$ & $-792$ & $1287$ & $-2002$\\
$n=\phantom{1}{-5}$ & $1$ & $-5$ & $15$ & $-35$ & $70$ & $-126$ & $210$ & $-330$ & $495$ & $-715$\\
$n=\phantom{1}{-4}$ & $1$ & $-4$ & $10$ & $-20$ & $35$ & $-56$ & $84$ & $-120$ & $165$ & $-220$\\
\midrule%
$n=\phantom{1}{-3}$ & $1$ & $-3$ & $6$ & $-10$ & $15$ & $-21$ & $28$ & $-36$ & $45$ & $-55$\\
$n=\phantom{1}{-2}$ & $1$ & $-2$ & $3$ & $-4$ & $5$ & $-6$ & $7$ & $-8$ & $9$ & $-10$\\
$n=\phantom{1}{-1}$ & $1$ & $-1$ & $1$ & $-1$ & $1$ & $-1$ & $1$ & $-1$ & $1$ & $-1$\\
$n=\phantom{-1}0$ & $1$ & $\phantom{-}0$ & $0$ & $\phantom{-}0$ & $0$
 & $\phantom{-}0$ & $0$ & $\phantom{-}0$ & $0$ & $\phantom{-}0$\\
\bottomrule
\end{tabular}

\vspace{2em}
\[\dbinom{n+1}{k+1} = \dbinom{n}{k}+\dbinom{n}{k+1},\]
\[\dbinom{n}{k} = \dbinom{n}{n-k} = \frac{n!}{k!\,(n-k)!}\qquad (0\le k\le n)\]

\newpage
\vglue 4em
\subsection{Stirling-Zahlen erster Art}\index{Stirling-Zahlen!Tabelle}
$\begin{bmatrix}n\\ k\end{bmatrix}$

\vspace{4pt}
\noindent
\begin{tabular}{r|rrrrrrrrrrrrrrr}
\toprule%
& $k=0$ & $k=1$ & $k=2$ & $k=3$ & $k=4$ & $k=5$ & $k=6$ & $k=7$ & $k=8$ & $k=9$\\
\midrule%
$n= 0$ &     1 &     0 &     0 &     0 &     0 &     0 &     0 &     0 &     0 &     0\\
$n= 1$ &     0 &     1 &     0 &     0 &     0 &     0 &     0 &     0 &     0 &     0\\
$n= 2$ &     0 &     1 &     1 &     0 &     0 &     0 &     0 &     0 &     0 &     0\\
$n= 3$ &     0 &     2 &     3 &     1 &     0 &     0 &     0 &     0 &     0 &     0\\
\midrule%
$n= 4$ &     0 &     6 &    11 &     6 &     1 &     0 &     0 &     0 &     0 &     0\\
$n= 5$ &     0 &    24 &    50 &    35 &    10 &     1 &     0 &     0 &     0 &     0\\
$n= 6$ &     0 &   120 &   274 &   225 &    85 &    15 &     1 &     0 &     0 &     0\\
$n= 7$ &     0 &   720 &  1764 &  1624 &   735 &   175 &    21 &     1 &     0 &     0\\
\midrule%
$n= 8$ &     0 &  5040 & 13068 & 13132 &  6769 &  1960 &   322 &    28 &     1 &     0\\
$n= 9$ &     0 & 40320 &109584 &118124 & 67284 & 22449 &  4536 &   546 &    36 &     1\\
$n=10$ &     0 &362880 &1026576&1172700&723680 &269325 & 63273 &  9450 &   870 &    45\\
$n=11$ &     0 &\!3628800 &\!\!10628640 &\!\!12753576 &\!8409500 &\!3416930 &\!902055 &\!157773 & 18150 &  1320\\
\bottomrule
\end{tabular}

\vspace{4em}
\subsection{Stirling-Zahlen zweiter Art}
$\begin{Bmatrix}n\\ k\end{Bmatrix}$

\vspace{4pt}
\noindent
\begin{tabular}{r|rrrrrrrrrr}
\toprule%
& $k=0$ & $k=1$ & $k=2$ & $k=3$ & $k=4$ & $k=5$ & $k=6$ & $k=7$ & $k=8$ & $k=9$\\
\midrule%
$n= 0$ &     1 &     0 &     0 &     0 &     0 &     0 &     0 &     0 &     0 &     0\\
$n= 1$ &     0 &     1 &     0 &     0 &     0 &     0 &     0 &     0 &     0 &     0\\
$n= 2$ &     0 &     1 &     1 &     0 &     0 &     0 &     0 &     0 &     0 &     0\\
$n= 3$ &     0 &     1 &     3 &     1 &     0 &     0 &     0 &     0 &     0 &     0\\
\midrule%
$n= 4$ &     0 &     1 &     7 &     6 &     1 &     0 &     0 &     0 &     0 &     0\\
$n= 5$ &     0 &     1 &    15 &    25 &    10 &     1 &     0 &     0 &     0 &     0\\
$n= 6$ &     0 &     1 &    31 &    90 &    65 &    15 &     1 &     0 &     0 &     0\\
$n= 7$ &     0 &     1 &    63 &   301 &   350 &   140 &    21 &     1 &     0 &     0\\
\midrule%
$n= 8$ &     0 &     1 &   127 &   966 &  1701 &  1050 &   266 &    28 &     1 &     0\\
$n= 9$ &     0 &     1 &   255 &  3025 &  7770 &  6951 &  2646 &   462 &    36 &     1\\
$n=10$ &     0 &     1 &   511 &  9330 & 34105 & 42525 & 22827 &  5880 &   750 &    45\\
$n=11$ &     0 &     1 &  1023 & 28501 &145750 &246730 &179487 & 63987 & 11880 &  1155\\
\bottomrule
\end{tabular}

\newpage
\section{Zahlentheorie}
\subsection{Primzahlen}\index{Primzahlen!Tabelle}

\begin{tabular}{rrrrrrrrrrrrrr@{\;\;\vrule width \heavyrulewidth\;}r}
\toprule
0 & 40 & 80 & 120 & 160 & 200 & 240 & 280 & 320 & 360 & 400 & 440 & 480 & 520 &\\
\midrule[\heavyrulewidth]
   2 &  179 &  419 &  661 &  947 & 1229 & 1523 & 1823 & 2131 & 2437 & 2749 & 3083 & 3433 & 3733 &    1\\
   3 &  181 &  421 &  673 &  953 & 1231 & 1531 & 1831 & 2137 & 2441 & 2753 & 3089 & 3449 & 3739 &    2\\
   5 &  191 &  431 &  677 &  967 & 1237 & 1543 & 1847 & 2141 & 2447 & 2767 & 3109 & 3457 & 3761 &    3\\
   7 &  193 &  433 &  683 &  971 & 1249 & 1549 & 1861 & 2143 & 2459 & 2777 & 3119 & 3461 & 3767 &    4\\
  11 &  197 &  439 &  691 &  977 & 1259 & 1553 & 1867 & 2153 & 2467 & 2789 & 3121 & 3463 & 3769 &    5\\
&&&&&&&&&&&&&&\\
  13 &  199 &  443 &  701 &  983 & 1277 & 1559 & 1871 & 2161 & 2473 & 2791 & 3137 & 3467 & 3779 &    6\\
  17 &  211 &  449 &  709 &  991 & 1279 & 1567 & 1873 & 2179 & 2477 & 2797 & 3163 & 3469 & 3793 &    7\\
  19 &  223 &  457 &  719 &  997 & 1283 & 1571 & 1877 & 2203 & 2503 & 2801 & 3167 & 3491 & 3797 &    8\\
  23 &  227 &  461 &  727 & 1009 & 1289 & 1579 & 1879 & 2207 & 2521 & 2803 & 3169 & 3499 & 3803 &    9\\
  29 &  229 &  463 &  733 & 1013 & 1291 & 1583 & 1889 & 2213 & 2531 & 2819 & 3181 & 3511 & 3821 &   10\\
&&&&&&&&&&&&&&\\
  31 &  233 &  467 &  739 & 1019 & 1297 & 1597 & 1901 & 2221 & 2539 & 2833 & 3187 & 3517 & 3823 &   11\\
  37 &  239 &  479 &  743 & 1021 & 1301 & 1601 & 1907 & 2237 & 2543 & 2837 & 3191 & 3527 & 3833 &   12\\
  41 &  241 &  487 &  751 & 1031 & 1303 & 1607 & 1913 & 2239 & 2549 & 2843 & 3203 & 3529 & 3847 &   13\\
  43 &  251 &  491 &  757 & 1033 & 1307 & 1609 & 1931 & 2243 & 2551 & 2851 & 3209 & 3533 & 3851 &   14\\
  47 &  257 &  499 &  761 & 1039 & 1319 & 1613 & 1933 & 2251 & 2557 & 2857 & 3217 & 3539 & 3853 &   15\\
&&&&&&&&&&&&&&\\
  53 &  263 &  503 &  769 & 1049 & 1321 & 1619 & 1949 & 2267 & 2579 & 2861 & 3221 & 3541 & 3863 &   16\\
  59 &  269 &  509 &  773 & 1051 & 1327 & 1621 & 1951 & 2269 & 2591 & 2879 & 3229 & 3547 & 3877 &   17\\
  61 &  271 &  521 &  787 & 1061 & 1361 & 1627 & 1973 & 2273 & 2593 & 2887 & 3251 & 3557 & 3881 &   18\\
  67 &  277 &  523 &  797 & 1063 & 1367 & 1637 & 1979 & 2281 & 2609 & 2897 & 3253 & 3559 & 3889 &   19\\
  71 &  281 &  541 &  809 & 1069 & 1373 & 1657 & 1987 & 2287 & 2617 & 2903 & 3257 & 3571 & 3907 &   20\\
&&&&&&&&&&&&&&\\
  73 &  283 &  547 &  811 & 1087 & 1381 & 1663 & 1993 & 2293 & 2621 & 2909 & 3259 & 3581 & 3911 &   21\\
  79 &  293 &  557 &  821 & 1091 & 1399 & 1667 & 1997 & 2297 & 2633 & 2917 & 3271 & 3583 & 3917 &   22\\
  83 &  307 &  563 &  823 & 1093 & 1409 & 1669 & 1999 & 2309 & 2647 & 2927 & 3299 & 3593 & 3919 &   23\\
  89 &  311 &  569 &  827 & 1097 & 1423 & 1693 & 2003 & 2311 & 2657 & 2939 & 3301 & 3607 & 3923 &   24\\
  97 &  313 &  571 &  829 & 1103 & 1427 & 1697 & 2011 & 2333 & 2659 & 2953 & 3307 & 3613 & 3929 &   25\\
&&&&&&&&&&&&&&\\
 101 &  317 &  577 &  839 & 1109 & 1429 & 1699 & 2017 & 2339 & 2663 & 2957 & 3313 & 3617 & 3931 &   26\\
 103 &  331 &  587 &  853 & 1117 & 1433 & 1709 & 2027 & 2341 & 2671 & 2963 & 3319 & 3623 & 3943 &   27\\
 107 &  337 &  593 &  857 & 1123 & 1439 & 1721 & 2029 & 2347 & 2677 & 2969 & 3323 & 3631 & 3947 &   28\\
 109 &  347 &  599 &  859 & 1129 & 1447 & 1723 & 2039 & 2351 & 2683 & 2971 & 3329 & 3637 & 3967 &   29\\
 113 &  349 &  601 &  863 & 1151 & 1451 & 1733 & 2053 & 2357 & 2687 & 2999 & 3331 & 3643 & 3989 &   30\\
&&&&&&&&&&&&&&\\
 127 &  353 &  607 &  877 & 1153 & 1453 & 1741 & 2063 & 2371 & 2689 & 3001 & 3343 & 3659 & 4001 &   31\\
 131 &  359 &  613 &  881 & 1163 & 1459 & 1747 & 2069 & 2377 & 2693 & 3011 & 3347 & 3671 & 4003 &   32\\
 137 &  367 &  617 &  883 & 1171 & 1471 & 1753 & 2081 & 2381 & 2699 & 3019 & 3359 & 3673 & 4007 &   33\\
 139 &  373 &  619 &  887 & 1181 & 1481 & 1759 & 2083 & 2383 & 2707 & 3023 & 3361 & 3677 & 4013 &   34\\
 149 &  379 &  631 &  907 & 1187 & 1483 & 1777 & 2087 & 2389 & 2711 & 3037 & 3371 & 3691 & 4019 &   35\\
&&&&&&&&&&&&&&\\
 151 &  383 &  641 &  911 & 1193 & 1487 & 1783 & 2089 & 2393 & 2713 & 3041 & 3373 & 3697 & 4021 &   36\\
 157 &  389 &  643 &  919 & 1201 & 1489 & 1787 & 2099 & 2399 & 2719 & 3049 & 3389 & 3701 & 4027 &   37\\
 163 &  397 &  647 &  929 & 1213 & 1493 & 1789 & 2111 & 2411 & 2729 & 3061 & 3391 & 3709 & 4049 &   38\\
 167 &  401 &  653 &  937 & 1217 & 1499 & 1801 & 2113 & 2417 & 2731 & 3067 & 3407 & 3719 & 4051 &   39\\
 173 &  409 &  659 &  941 & 1223 & 1511 & 1811 & 2129 & 2423 & 2741 & 3079 & 3413 & 3727 & 4057 &   40\\
\bottomrule
\end{tabular}

\twocolumn



\chapter{Anhang}
\section{Griechisches Alphabet}

\begin{tabular}{l|l}
\begin{tabular}[t]{lll}
$\mathrm A$ & $\alpha$   & Alpha\\
$\mathrm B$ & $\beta$    & Beta\\
$\Gamma$    & $\gamma$   & Gamma\\
$\Delta$    & $\delta$   & Delta\\
\noalign{\vspace{1em}}
$\mathrm E$ & $\varepsilon$ & Epsilon\\
$\mathrm Z$ & $\zeta$    & Zeta\\
$\mathrm H$ & $\eta$     & Eta\\
$\Theta$    & $\theta$   & Theta\\
\noalign{\vspace{1em}}
$\mathrm I$ & $\iota$    & Jota\\
$\mathrm K$ & $\kappa$   & Kappa\\
$\Lambda$   & $\lambda$  & Lambda\\
$\mathrm M$ & $\mu$      & My
\end{tabular}
&
\begin{tabular}[t]{lll}
$\mathrm N$ & $\nu$      & Ny\\
$\Xi$       & $\xi$      & Xi\\
$\mathrm O$ & $o$        & Omikron\\
$\Pi$       & $\pi$      & Pi\\
\noalign{\vspace{1em}}
$\mathrm R$ & $\varrho$  & Rho\\
$\Sigma$    & $\sigma$   & Sigma\\
$\mathrm T$ & $\tau$     & Tau\\
$\mathrm Y$ & $\upsilon$ & Ypsilon\\
\noalign{\vspace{1em}}
$\Phi$      & $\varphi$  & Phi\\
$\mathrm X$ & $\chi$     & Chi\\
$\Psi$      & $\psi$     & Psi\\
$\Omega$    & $\omega$   & Omega 
\end{tabular}
\end{tabular}

\section{Frakturbuchstaben}
\begin{tabular}{l|l}
\begin{tabular}[t]{l@{\hskip 2pt}ll@{\hskip 2pt}l}
A & a & $\mathfrak A$ & $\mathfrak a$\\
B & b & $\mathfrak B$ & $\mathfrak b$ \\
C & c & $\mathfrak C$ & $\mathfrak c$\\
D & d & $\mathfrak D$ & $\mathfrak d$\\
\noalign{\vspace{1em}}
E & e & $\mathfrak E$ & $\mathfrak e$\\
F & f & $\mathfrak F$ & $\mathfrak f$\\
G & g & $\mathfrak G$ & $\mathfrak g$\\
H & h & $\mathfrak H$ & $\mathfrak h$\\
\noalign{\vspace{1em}}
I & i & $\mathfrak I$ & $\mathfrak i$\\
J & j & $\mathfrak J$ & $\mathfrak j$\\
K & k & $\mathfrak K$ & $\mathfrak k$\\
L & l & $\mathfrak L$ & $\mathfrak l$\\
\noalign{\vspace{1em}}
M & m & $\mathfrak M$ & $\mathfrak m$\\
N & n & $\mathfrak N$ & $\mathfrak n$
\end{tabular}
&
\begin{tabular}[t]{l@{\hskip 2pt}ll@{\hskip 2pt}l}
O & o & $\mathfrak O$ & $\mathfrak o$\\
P & p & $\mathfrak P$ & $\mathfrak p$\\
Q & q & $\mathfrak Q$ & $\mathfrak q$\\
R & r & $\mathfrak R$ & $\mathfrak r$\\
\noalign{\vspace{1em}}
S & s & $\mathfrak S$ & $\mathfrak s$\\
T & t & $\mathfrak T$ & $\mathfrak t$\\
U & u & $\mathfrak U$ & $\mathfrak u$\\
V & v & $\mathfrak V$ & $\mathfrak v$\\
\noalign{\vspace{1em}}
W & w & $\mathfrak W$ & $\mathfrak w$\\
X & x & $\mathfrak X$ & $\mathfrak x$\\
Y & y & $\mathfrak Y$ & $\mathfrak y$\\
Z & z & $\mathfrak Z$ & $\mathfrak z$
\end{tabular}
\end{tabular}

\section{Mathematische Konstanten}
\begin{enumerate}
\item Kreiszahl\\
$\pi = 3{,}14159\;26535\;89793\;23846\;26433\;83279\ldots$

\item Eulersche Zahl\\
$\ee = 2{,}71828\;18284\;59045\;23536\;02874\;71352\ldots$

\item Euler-Mascheroni-Konstante\\
$\gamma = 0{,}57721\;56649\;01532\;86060\;65120\;90082\ldots$

\item Goldener Schnitt, $(1+\sqrt{5})/2$\\
$\varphi = 1{,}61803\;39887\;49894\;84820\;45868\;34365\ldots$

\item 1. Feigenbaum-Konstante\\
$\delta = 4{,}66920\;16091\;02990\;67185\;32038\;20466\ldots$

\item 2. Feigenbaum-Konstante\\
$\alpha = 2{,}50290\;78750\;95892\;82228\;39028\;73218\ldots$
\end{enumerate}

\newpage
\section{Physikalische Konstanten}
\begin{enumerate}
\item Lichtgeschwindigkeit im Vakuum\\
$c=299\;792\;458\;\unit{m/s}$

\item Elektrische Feldkonstante\\
$\varepsilon_0 = 8{,}854\;187\;817\;620\;39\times 10^{-12}\:\unit{F/m}$

\item Magnetische Feldkonstante\\
$\mu_0 = 4\pi\times 10^{-7}\:\unit{H/m}$

\item Elementarladung\\
$e = 1{,}602\;176\;6208\;(98)\times 10^{-19}\,\unit{C}$

\item Gravitationskonstante\\
$G = 6{,}674\;08\;(31)\times 10^{-11}\,\unit{m^3/(kg\,s^2)}$

\item Avogadro-Konstante\\
$N_A = 6{,}022\;140\;857\;(74)\times 10^{23}/\unit{mol}$

\item Boltzmann-Konstante\\
$k_B = 1{,}380\;648\;52\;(79)\times 10^{-23}\,\unit{J/K}$

\item Universelle Gaskonstante\\
$R = 8{,}314\;4598\;(48)\:\unit{J/(mol\,K)}$

\item Plancksches Wirkungsquantum\\
$h = 6{,}626\;070\;040\;(81)\times 10^{-34}\,\unit{Js}$

\item Reduziertes planksches Wirkungsquantum\\
$\hbar = 1{,}054\;571\;800\;(13)\times 10^{-34}\,\unit{Js}$

\item Masse des Elektrons\\
$m_e = 9{,}109\;383\;56\;(11)\times 10^{-31}\,\unit{kg}$

\item Masse des Neutrons\\
$m_n = 1{,}674\;927\;471\;(21)\times 10^{-27}\,\unit{kg}$

\item Masse des Protons\\
$m_p = 1{,}672\;621\;898\;(21)\times 10^{-27}\,\unit{kg}$
\end{enumerate}

\newpage
\section{Einheiten}
\subsection{SI-System}
Newton (Kraft):
\begin{equation}
\unit{N}=\unit{kg\,m/s^2}.
\end{equation}
Watt (Leistung):
\begin{equation}
\unit{W}=\unit{kg\,m^2/s^3}=\unit{VA}.
\end{equation}
Joule (Energie):
\begin{equation}
\unit{J}=\unit{kg\,m^2/s^2}=\unit{Nm}=\unit{Ws}=\unit{VAs}.
\end{equation}
Pascal (Druck):
\begin{equation}
\unit{Pa}=\unit{N/m^2} = 10^{-5}\,\unit{bar}.
\end{equation}
Hertz (Frequenz):
\begin{equation}
\unit{Hz} = \unit{1/s}.
\end{equation}
Coulomb (Ladung):
\begin{equation}
\unit{C} = \unit{As}.
\end{equation}
Volt (Spannung):
\begin{equation}
\unit{V} = \unit{kg\,m^2/(A\,s^3)}
\end{equation}
Tesla (magnetische Flussdichte):
\begin{equation}
\unit{T} = \unit{N/(A\,m)} = \unit{Vs/m^2}.
\end{equation}

\subsection{Nicht-SI-Einheiten}
\begin{tabular}{l|r|l}
\hline
\thbf{Einheit} & \thbf{Symbol} & \thbf{Umrechnung}\pstrut{1pt}\\
\hline
\multicolumn{3}{l}{\thbf{Zeit:}\pstrut{4pt}}\\
\hline
Minute & min & = 60\,s\pstrut{2pt}\\
Stunde & h & = 60\,min = 3600\,s\\
Tag & d & = 24\,h = 86\,400\,s\\
Jahr & a & = 356,25\,d\\
\hline
\multicolumn{3}{l}{\thbf{Druck:}\pstrut{4pt}}\\
\hline
bar & bar & $= 10^5\,\unit{Pa}$\pstrut{2pt}\\
mmHg & mmHg & = 133,322 Pa\\
\hline
\multicolumn{3}{l}{\thbf{Fläche:}\pstrut{4pt}}\\
\hline
Ar & a & $= 100\,\unit{m^2}$\pstrut{2pt}\\
Hektar & ha & = 100\,a = $10\,000\,\unit{m^2}$\\
\hline
\multicolumn{3}{l}{\thbf{Masse:}\pstrut{4pt}}\\
\hline
Tonne & t & = 1000\,kg\pstrut{2pt}\\
\hline
\multicolumn{3}{l}{\thbf{Länge:}\pstrut{4pt}}\\
\hline
Liter & L & = $10^{-3}\,\unit{m^3}$\pstrut{2pt}
\end{tabular}

\subsection{Britische Einheiten}
\begin{tabular}{l|r|l}
\thbf{Einheit} & \thbf{Abk.} & \thbf{Umrechnung}\\
inch & in. & = 2,54\,cm\\
foot & ft. & = 12\,in. = 30,48\,cm\\
yard & yd. & = 3\,ft. = 91,44\,cm\\
chain & ch. & = 22\,yd. = 20,1168\,m\\
&\\[-4pt]
furlong & fur. & = 10\,ch. = 201,168\,m\\
mile & mi. & = 1760\,yd. = 1609,3440\,m
\end{tabular}




\printindex

\end{document}


