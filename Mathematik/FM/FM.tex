\documentclass[a4paper,10pt,fleqn,twocolumn,twoside,openany]{scrbook}
\usepackage[utf8]{inputenc}
\usepackage[T1]{fontenc}
\usepackage{lmodern}
% \usepackage{ngerman}
\usepackage[ngerman]{babel}
\usepackage{amsmath}
\usepackage{amssymb}
\usepackage{amsthm}
\usepackage{mdframed}

\usepackage{textcomp}
\usepackage{lipsum}
\usepackage{microtype}
\usepackage{enumitem}
\usepackage{graphicx}
% \usepackage{multicol}

\usepackage{libertine}
\usepackage[cmintegrals]{newtxmath}
\usepackage[scaled=0.80]{DejaVuSans}
\renewcommand{\ttdefault}{lmtt}
% \renewcommand\sfdefault{lmss}

\usepackage{color}
\definecolor{c1}{RGB}{0,40,80}
\definecolor{c2}{RGB}{20,60,100}
\definecolor{c3}{RGB}{80,120,180}
\definecolor{c4}{RGB}{140,0,60}
\definecolor{bc1}{RGB}{236,236,250}
\usepackage[colorlinks=true,linkcolor=c1]{hyperref}
\usepackage{geometry}
\geometry{a4paper,left=25mm,right=10mm,top=20mm,bottom=28mm}
\setlength{\columnsep}{4mm}

\usepackage[toc]{multitoc}
\setcounter{secnumdepth}{4}
\setcounter{tocdepth}{2}
\usepackage{tocloft}
\setlength{\cftsecindent}{0pt}
\setlength{\cftsubsecindent}{23pt}
\setlength{\cftsubsubsecindent}{55pt}
\renewcommand{\cftchapfont}{\normalfont\sffamily\bfseries}
\renewcommand{\cftsecfont}{\normalfont\sffamily}
\renewcommand{\cftsubsecfont}{\normalfont\sffamily}
\renewcommand{\cftsubsubsecfont}{\normalfont\sffamily}
\renewcommand\cftchappagefont{\normalfont\sffamily\bfseries}
\renewcommand\cftsecpagefont{\normalfont\sffamily}
\renewcommand\cftsubsecpagefont{\normalfont\sffamily}
\renewcommand\cftsubsubsecpagefont{\normalfont\sffamily}

\usepackage{titlesec}

% \titleformat{...}[hang]
% is broken in TeX Live Ubuntu 16.04 LTS.
% So a patch is used until an update is available.
% BEGIN PATCH
\usepackage{etoolbox}
\makeatletter
\patchcmd{\ttlh@hang}{\parindent\z@}{\parindent\z@\leavevmode}{}{}
\patchcmd{\ttlh@hang}{\noindent}{}{}{}
\makeatother
% END PATCH

\titleformat{\chapter}[hang]
  {\normalfont\sffamily\huge\bfseries}{\thechapter}{1em}{\Huge}
\titleformat{\section}[hang]
  {\normalfont\sffamily\Large\bfseries}{\thesection}{1em}{\Large}
\titleformat{\subsection}[hang]
  {\normalfont\sffamily\large\bfseries}{\thesubsection}{1em}{\large}
\titleformat{\subsubsection}[hang]
  {\normalfont\sffamily\large\bfseries}{\thesubsubsection}{1em}{\large}

\titlespacing*{\chapter}{0pt}{0pt}{10pt}
\titlespacing*{\section}{0pt}{4pt minus 2pt}{2pt minus 2pt}
\titlespacing*{\subsection}{0pt}{6pt minus 4pt}{2pt minus 2pt}
\titlespacing*{\subsubsection}{0pt}{6pt minus 4pt}{2pt minus 4pt}

\usepackage[justification=RaggedRight,singlelinecheck=off]{caption}

\numberwithin{equation}{chapter}

\renewcommand{\baselinestretch}{0.9}

\newenvironment{ttsection}{\ttfamily}{\par}
\newcommand{\strong}[1]{{\sffamily\bfseries #1}}
\newcommand{\bsf}[1]{{\sffamily\bfseries #1}}
\newcommand{\bitem}{\item[\scriptsize $\blacksquare$]}
\newcommand{\bulletbs}{\text{\scriptsize $\blacksquare$}\;\,}

% \newcommand{\definition}{\strong{Definition.}}
\newcommand{\theorem}[1]{\strong{#1.}}
\newenvironment{Definition}{\par\noindent\strong{Definition.}}{\par}
\newenvironment{Satz}{\par\noindent\strong{Satz.}}{\par}
\newcommand{\minisection}{\vspace{4pt plus 2pt minus 1pt}\par\noindent}

\newtheoremstyle{Definition}
  {0pt}{0pt}
  {}{}
  {\sffamily\bfseries}{\newline}
% {.2em}{\thmname{#1}\thmnumber{~#2}.~\thmnote{#3.}}
  {.2em}{\thmname{#1}.~\thmnote{#3.}}
\theoremstyle{Definition}
\newtheorem{definition}{Definition}[chapter]

%\newenvironment{definition}[1][]%
%  {\par\addvspace{4pt plus 0pt minus 4pt}\noindent\strong{Definition.~#1.}\newline\indent}%
%  {\par\addvspace{4pt plus 0pt minus 4pt}}

\definecolor{greenblue}{rgb}{0.0,0.32,0.4}
\definecolor{grayblue}{rgb}{0.2,0.2,0.4}
\definecolor{lightgray}{rgb}{0.4,0.4,0.4}
\definecolor{b1}{rgb}{0.6,0.6,0.54}

\surroundwithmdframed[topline=false,rightline=false,bottomline=false,%
  linecolor=greenblue, linewidth=3.0pt, innerleftmargin=4pt,%
  innertopmargin=1pt, innerbottommargin=1pt,%
  innerrightmargin=0pt%
]{definition}

\newcommand{\ibox}[1]{\par\hbox{\hsize=22em\hglue 1em\vbox{{\noindent}#1}}}

% parametric strut
\newcommand{\pstrut}[1]{\rule{0pt}{\dimexpr 8pt+#1}}

% emphasis, definierter Begriff
\newcommand{\emdef}[1]{\textit{#1}}
\newcommand{\bdef}[1]{\strong{#1}}

% table header bold font
\newcommand{\thbf}[1]{{\sffamily\bfseries #1}}

% \ui: imaginäre Einheit
% \ue: Einheitsvektor
% \ue: eulersche Zahl
% \uv{x}: unterstrichener Vektor

\newcommand{\ui}{\mathrm i}
\newcommand{\uj}{\mathrm j}
\newcommand{\uk}{\mathrm k}
\newcommand{\ue}{e}
\newcommand{\unit}[1]{\mathrm{#1}}
\newcommand{\ee}{\mathrm e}
\newcommand{\uv}[1]{\underline{#1}}
\newcommand{\bv}[1]{\mathbf{#1}}
\newcommand{\comp}{\textsf{c}}
\newcommand{\bright}{\texttt{[}}
\newcommand{\bleft}{\texttt{]}}

% A cdot that can be made bolder
\newcommand{\bcdot}{\cdot}

\newcommand{\N}{\mathbb N}
\newcommand{\Z}{\mathbb Z}
\newcommand{\Q}{\mathbb Q}
\newcommand{\R}{\mathbb R}
\newcommand{\C}{\mathbb C}
\newcommand{\K}{\mathbb K}

\DeclareMathOperator{\id}{id}
\DeclareMathOperator*{\sgn}{sgn}
\DeclareMathOperator*{\rg}{rg}
\DeclareMathOperator*{\diag}{diag}
\DeclareMathOperator*{\Eig}{Eig}
\DeclareMathOperator{\real}{Re}
\DeclareMathOperator{\imag}{Im}
\DeclareMathOperator{\tr}{tr}
\DeclareMathOperator{\arccot}{arccot}
\DeclareMathOperator{\arsinh}{arsinh}
\DeclareMathOperator{\arcosh}{arcosh}
\DeclareMathOperator{\artanh}{artanh}
\DeclareMathOperator{\arcoth}{arcoth}

\newcommand{\defiff}{\;:\Longleftrightarrow\;}
\renewcommand{\coloneq}{:=}

\newcounter{bibcounter}
\newenvironment{bibenumerate}[1]%
  {\begingroup\renewcommand{\chapter}[2]{}%
    \begin{thebibliography}{#1}%
      \setcounter{enumiv}{\value{bibcounter}}}%
  {\setcounter{bibcounter}{\value{enumiv}}%
    \end{thebibliography}\endgroup}

\newcommand{\bibtitle}[1]{»#1«}

\usepackage{makeidx}
\renewcommand\indexname{Stichwortverzeichnis}
\makeindex

\begin{document}
% \setlength{\baselineskip}{11.0pt}
\setlength{\abovedisplayskip}{6pt}
\setlength{\belowdisplayskip}{6pt}
\setlength{\abovedisplayshortskip}{6pt}
\setlength{\belowdisplayshortskip}{6pt}
\setlength{\abovecaptionskip}{2pt plus 2pt minus 1pt}

\begin{titlepage}
\centering
\phantom{x}

\vspace{20em}
{\noindent\Huge\sffamily\textbf{Formelsammlung\\
Mathematik}}

\vspace{2em}
{\Large Juli 2018}\\
\end{titlepage}

\thispagestyle{empty}

\noindent
Dieses Buch ist unter der Lizenz\\
Creative Commons CC0 veröffentlicht.
\vspace{8em}

\noindent
\begin{ttsection}
\begin{tabular}{r|r|r|r}
 0 & 0000 & 0 &  0\\
 1 & 0001 & 1 &  1\\
 2 & 0010 & 2 &  2\\
 3 & 0011 & 3 &  3\\
\noalign{\vspace{1em}}
 4 & 0100 & 4 &  4\\
 5 & 0101 & 5 &  5\\
 6 & 0110 & 6 &  6\\
 7 & 0111 & 7 &  7\\
\noalign{\vspace{1em}}
 8 & 1000 & 8 & 10\\
 9 & 1001 & 9 & 11\\
10 & 1010 & A & 12\\
11 & 1011 & B & 13\\
\noalign{\vspace{1em}}
12 & 1100 & C & 14\\
13 & 1101 & D & 15\\
14 & 1110 & E & 16\\
15 & 1111 & F & 17
\end{tabular}
\end{ttsection}

\newpage
\noindent
$\!\begin{aligned}
\sin(-x) &= -\sin x\\
\cos(-x) &= \cos x
\end{aligned}$
\vspace{1em}

\noindent
$\!\begin{aligned}
\sin(x+y) &= \sin x\cos y + \cos x\sin y\\
\sin(x-y) &= \sin x\cos y - \cos x\sin y\\
\cos(x+y) &= \cos x\cos y - \sin x\sin y\\
\cos(x-y) &= \cos x\cos y + \sin x\sin y
\end{aligned}$
\vspace{1em}

\noindent
$\ee^{\ui\varphi}=\cos\varphi+\ui\sin\varphi$
\vspace{2em}

\noindent
\strong{Polarkoordinaten}\\
$x=r\cos\varphi$\\
$y=r\sin\varphi$\\
$\varphi\in(-\pi,\pi]$\\
$\det J=r$
\vspace{1em}

\noindent
\strong{Zylinderkoordinaten}\\
$x=r_{xy}\cos\varphi$\\
$y=r_{xy}\sin\varphi$\\
$z=z$\\
$\det J=r_{xy}$
\vspace{1em}

\noindent
\strong{Kugelkoordinaten}\\
$x=r\sin\theta\,\cos\varphi$\\
$y=r\sin\theta\,\sin\varphi$\\
$z=r\cos\theta$\\
$\varphi\in(-\pi,\pi],\;\theta\in[0,\pi]$\\
$\det J=r^2\sin\theta$
\vspace{1em}

\noindent
$\theta=\beta-\pi/2$\\
$\beta\in[-\pi/2,\pi/2]$\\
$\cos\theta=\sin\beta$\\
$\sin\theta=\cos\beta$

\renewcommand{\contentsname}{\sffamily Inhaltsverzeichnis}
\tableofcontents


\chapter{Grundlagen}
\section{Arithmetik}
\subsection{Zahlenbereiche}
Natürliche Zahlen ab null:
\begin{equation}
\N_0 := \{0,1,2,3,4,\ldots\}.
\end{equation}
Natürliche Zahlen ab eins:
\begin{equation}
\N_1 := \{1,2,3,4,5,\ldots\}.
\end{equation}
Natürliche Zahlen:
\begin{equation}
\begin{split}
&\text{$\N$, wenn es keine Rolle spielt,}\\
&\text{ob $\N:=\N_0$ oder $\N:=\N_1$}.
\end{split}
\end{equation}
Ganze Zahlen:
\begin{equation}
\Z := \{\ldots,-3,-2,-1,0,1,2,3,\ldots\}.
\end{equation}
Rationale Zahlen:
\begin{equation}
\Q := \{\tfrac{z}{n}\mid z\in\Z,n\in\N_0\}.
\end{equation}
Reelle Zahlen:
\begin{equation}
\R := \overline{\Q}\enspace\text{bezüglich}\; d(x,y)=|x-y|.
\end{equation}
Positive reelle Zahlen:
\begin{equation}
\R^+ := \{x\in\R\mid x>0\}.
\end{equation}
Nichtnegative reelle Zahlen:
\begin{equation}
\R_0^+ := \{x\in\R\mid x\ge 0\}.
\end{equation}
Negative reelle Zahlen:
\begin{equation}
\R^- := \{x\in\R\mid x<0\}.
\end{equation}
Nichtpositive reelle Zahlen:
\begin{equation}
\R_0^- := \{x\in\R\mid x\le 0\}.
\end{equation}
Komplexe Zahlen:
\begin{equation}
\C := \{a+b\ui\mid a,b\in\R\}.
\end{equation}
Quaternionen:
\begin{equation}
\mathbb H := \{a+b\ui+c\uj+d\uk\mid a,b,c,d\in\R\}.
\end{equation}
Algebraische Zahlen:
\begin{equation}
\mathbb A := \{a\in\C\mid \exists P{\in}\Q[X]\colon P(a)=0\}.
\end{equation}
Irrationale Zahlen:
\begin{equation}
\R\setminus\Q = \{\sqrt{2},\sqrt{3},\pi,\ee,\ldots\}.
\end{equation}
Transzendente Zahlen:
\begin{equation}
\R\setminus\mathbb A = \{\pi,\ee,\ldots\}.
\end{equation}
Es gelten die folgenden Teilmengenbeziehungen:
\begin{equation}
\N\subset\Z\subset\Q\subset\R\subset\C\subset\mathbb H.
\end{equation}
Es gilt die folgende Abstufung der Mächtigkeit:
\begin{equation}
|\N| = |\Z| = |\Q| = |\mathbb A| < |\R| = |\C|.
\end{equation}

\newpage
\subsection{Intervalle}
Abgeschlossene Intervalle:
\begin{equation}
[a,b] := \{x\in\R\mid a\le x\le b\}.
\end{equation}
Offene Intervalle:
\begin{equation}
(a,b) := \{x\in\R\mid a<x<b\}.
\end{equation}
Halboffene Intervalle:
\begin{align}
(a,b] &:= \{x\in\R\mid a<x\le b\},\\
[a,b) &:= \{x\in\R\mid a\le x<b\}.
\end{align}
Unbeschränkte Intervalle:
\begin{align}
[a,\infty) &:= \{x\in\R\mid a\le x\},\\
(a,\infty) &:= \{x\in\R\mid a<x\},\\
(-\infty,b] &:= \{x\in\R\mid x\le b\},\\
(-\infty,b) &:= \{x\in\R\mid x<b\}.
\end{align}

\subsection{Summen}
\begin{definition}[Summe]
Für eine Folge $(a_n)$:
\begin{align}
\sum_{k=m}^{m-1} a_k &:= 0,\qquad(\text{leere Summe})\\
\sum_{k=m}^n a_k &:= a_n+\sum_{k=m}^{n-1} a_k.\qquad(n\ge m)
\end{align}
\end{definition}
\noindent
Für eine Konstante $c$ gilt:
\begin{equation}
\sum_{k=m}^n c = (n-m+1)\,c.
\end{equation}
Der Summierungsoperator ist linear:
\begin{align}
\sum_{k=m}^n (a_k+b_k) &= \sum_{k=m}^n a_k + \sum_{k=m}^n b_k,\\
\sum_{k=m}^n ca_k &= c\sum_{k=m}^n a_k.
\end{align}
Indexverschiebung ist möglich:
\begin{equation}
\sum_{k=m}^n a_k = \sum_{k=m-j}^{n-j} a_{k+j} = \sum_{k=m+j}^{n+j} a_{k-j}.
\end{equation}
Aufspaltung ist möglich:
\begin{equation}
\sum_{k=m}^n a_k = \sum_{k=m}^p a_k + \sum_{k=p+1}^n a_k.
\end{equation}
Vertauschung der Reihenfolge bei Doppelsummen:
\begin{equation}
\sum_{i=p}^m \sum_{j=q}^n a_{ij} = \sum_{j=q}^n \sum_{i=p}^m a_{ij}.
\end{equation}

\subsection{Produkte}
\begin{definition}[Produkt]
Für eine Folge $(a_n)$:
\begin{align}
\prod_{k=m}^{m-1} a_k &:= 1,\qquad(\text{leeres Produkt})\\
\prod_{k=m}^n a_k &:= a_n\prod_{k=m}^{n-1} a_k.\qquad(n\ge m)
\end{align}
\end{definition}

\noindent
Für eine Konstante $c$ gilt:
\begin{equation}
\prod_{k=m}^n c = c^{n-m+1}.
\end{equation}
Unter Voraussetzung des Kommutativgesetzes gilt
\begin{align}
\label{eq:product-product}
\prod_{k=m}^n (a_k b_k) &= \bigg(\prod_{k=m}^n a_k\bigg)\bigg(\prod_{k=m}^n b_k\bigg),\\
\label{eq:product-power}
\prod_{k=m}^n a_k^c &= \bigg(\prod_{k=m}^n a_k\bigg)\Big.^c.\qquad (c\in\N_0)
\end{align}

Formel \eqref{eq:product-power} gilt auch für $a_k\in\R^+$ und $c\in\C$.

Formel \eqref{eq:product-product} ist ein Spezialfall von
\begin{equation}
\prod_{i=p}^m \prod_{j=q}^n a_{ij} = \prod_{j=q}^n \prod_{i=p}^m a_{ij}.
\end{equation}
Indexverschiebung ist möglich:
\begin{equation}
\prod_{k=m}^n a_k = \prod_{k=m-j}^{n-j} a_{k+j} = \prod_{k=m+j}^{n+j} a_{k-j}.
\end{equation}
Aufspaltung ist möglich:
\begin{equation}
\prod_{k=m}^n a_k = \bigg(\prod_{k=m}^p a_k\bigg)\bigg(\prod_{k=p+1}^n a_k\bigg).
\end{equation}
Für $a_k\in\R^+$ gilt
\begin{equation}
\prod_{k=m}^n a_k = \exp\bigg(\sum_{k=m}^n \ln(a_k)\bigg).
\end{equation}

\subsection{Binomischer Lehrsatz}\index{binomischer Lehrsatz}
Sei $R$ ein unitärer Ring, z.\,B. $R=\R$ oder $R=\C$.\\
Für $a,b\in R$ mit $ab=ba$ gilt:%
\begin{equation}
(a+b)^n = \sum_{k=0}^n \binom{n}{k} a^{n-k} b^k
\end{equation}
und
\begin{equation}
(a-b)^n = \sum_{k=0}^n \binom{n}{k} (-1)^k a^{n-k} b^k.
\end{equation}
Die ersten Formeln sind:\index{binomische Formeln}
\begin{gather}
(a+b)^2 = a^2+2ab+b^2,\\
(a-b)^2 = a^2-2ab+b^2,\\
(a+b)^3 = a^3+3a^2 b+3ab^2+b^3,\\
(a-b)^3 = a^3-3a^2 b+3ab^2-b^3,\\
(a+b)^4 = a^4+4a^3 b+6a^2 b^2+4ab^3+b^4,\\
(a-b)^4 = a^4-4a^3 b+6a^2 b^2-4ab^3+b^4.
\end{gather}
\subsection{Potenzgesetze}
\begin{definition}[Potenz]
Für $a$ aus einem Monoid und $n\in\Z,n\ge 1$:
\begin{align}
a^0 &:= 1,\\
a^n &:= a^{n-1}\cdot a.
\end{align}

Für $a\in\R, a>0$ und $x\in\C$:
\begin{equation}
a^x := \exp(\ln(a)\,x).
\end{equation}
\end{definition}
\noindent
Für $a\in\R, a>0$ und $x,y\in\C$ gilt:
\begin{gather}
a^{x+y} = a^x a^y,\quad a^{x-y} = \frac{a^x}{a^y},
\quad a^{-x} = \frac{1}{a^x}.
\end{gather}

\section{Gleichungen}
\begin{definition}[Bestimmungsgleichung]
Sind $f,g$ auf der Grundmenge $G$ definierte Funktionen, so nennt
man
\begin{equation}
f(x) = g(x)\\
\end{equation}
eine \emdef{Bestimmungsgleichung}\index{Bestimmungsgleichung},
wenn die Lösungemenge
\begin{equation}
L = \{x\in G\mid f(x)=g(x)\}
\end{equation}
gesucht ist.
\end{definition}
Bei den $x\in G$ kann es sich auch um Tupel $x=(x_1,x_2)$ oder
$x=(x_1,x_2,x_3)$ usw. handeln. Man spricht in diesem Fall
von einer Gleichung \emdef{in mehreren Variablen}.

Handelt es sich bei den Funktionswerten von $f,g$ um Tupel,
dann spricht man von einem
\emdef{Gleichungssystem}\index{Gleichungssystem}.

\subsection{Äquivalenzumformungen}

Äquivalenzumformungen lassen die Lösungsmenge einer Gleichung
unverändert. Seien $A(x),B(x)$ zwei Aussageformen bzw. zwei
Gleichungen. Aus
\begin{equation}
\forall x{\in}G\,[A(x)\Longleftrightarrow B(x)]
\end{equation}
folgt
\begin{equation}
\{x\in G\mid A(x)\} = \{x\in G\mid B(x)\}.
\end{equation}
Aus
\begin{equation}
\forall x{\in} G\,[A(x)\Longrightarrow B(x)]
\end{equation}
folgt jedoch nur noch
\begin{equation}
\{x\in G\mid A(x)\}\subseteq\{x\in G\mid B(x)\}.
\end{equation}
Seien $f,g,h$ Funktionen mit Definitionsmenge $G$ und
Zielmenge $Z=\R$ oder $Z=\C$.

Für alle $x$ gilt:
\begin{align}
f(x)=g(x) &\Longleftrightarrow f(x){+}h(x)=g(x){+}h(x),\\
f(x)=g(x) &\Longleftrightarrow f(x){-}h(x)=g(x){-}h(x).
\end{align}
Besitzt $h(x)$ keine Nullstellen, dann gilt für alle $x$:
\begin{align}
f(x)=g(x) &\iff f(x)h(x)=g(x)h(x),\\
f(x)=g(x) &\iff \frac{f(x)}{h(x)}=\frac{f(x)}{h(x)}.
\end{align}
Besitzt $h(x)$ aber Nullstellen, dann gilt immerhin noch für alle $x$:
\begin{equation}
f(x)=g(x) \implies f(x)h(x)=g(x)h(x).
\end{equation}
Sei $f,g\colon G\to Z$. Sei $\varphi_x\colon Z\to Z'$ eine Injektion
für jedes $x\in G$. Es gilt
\begin{equation}
f(x)=g(x) \iff \varphi_x(f(x))=\varphi_x(g(x))
\end{equation}
für alle $x\in G$.

Bei einer Kette von Äquivalenzumformungen wird links das
Äquivalenzzeichen geschrieben, in der Mitte die Gleichung
und rechts hinter einem senkrechten Strich die Operation
$\varphi_x(w)$, welche als nächstes auf beide Seiten der Gleichung
angwendet werden soll.

Beispiel:
\begin{equation*}\setlength{\arraycolsep}{2pt}
\begin{array}{rrl@{\qquad}l}
& 2x+4 &= 2x^2-8x+2 &\mid w/2\\[2pt]
\Longleftrightarrow& x+2 &= x^2-4x+1 &\mid w-2\\[2pt]
\Longleftrightarrow& x &= x^2-4x-1 &\mid w-x\\[2pt]
\Longleftrightarrow& 0 &= x^2-7x-1.
\end{array}
\end{equation*}
Am Anfang befinden sich eventuell Bedingungen für $x$.
Bei Fallunterscheidungen wird eine Verschärfung der Bedingungen
vorgenommen, so dass es zur Verkleinerung der Grundmenge kommt.
Nach einer Fallunterscheidung ergeben sich unter Umständen neue
Injektionen.

\subsection{Quadratische Gleichungen}
\begin{definition}[Quadratische Gleichung]
Eine Gleichung der Form $ax^2+bx+c=0$ mit $a\ne 0$ heißt
\emdef{quadratische Gleichung}.
\end{definition}

Wegen $a\ne 0$ lässt sich die Gleichung durch $a$ dividieren
und es ensteht die äquivalente Normalform $x^2+px+q=0$
mit $p:=b/a$ und $q:=c/a$.

\strong{Lösung.}
Seien nun die $a,b,c$ reelle Zahlen. Die Zahl
\begin{equation}
D = p^2-4q
\end{equation}
heißt \emdef{Diskriminante}. Für $D>0$ gibt es zwei reelle Lösungen:
\begin{align}
x_1 &= \frac{-p-\sqrt{D}}{2} = \frac{-b-\sqrt{b^2-4ac}}{2a},\\
x_2 &= \frac{-p+\sqrt{D}}{2} = \frac{-b+\sqrt{b^2-4ac}}{2a}.
\end{align}
Für $D=0$ fallen beiden Lösungen zu einer \emdef{doppelten Lösung}
zusammen:
\begin{equation}
x_1 = x_2 = -\frac{p}{2} = -\frac{b}{2a}.
\end{equation}
Für $D<0$ gibt es keine reelle Lösung. Aber es gibt zwei komplexe
Lösungen, die zueinander konjugiert sind:
\begin{equation}
x_1 = \frac{-p-\ui\sqrt{|D|}}{2},\quad
x_2 = \frac{-p+\ui\sqrt{|D|}}{2}.
\end{equation}
In jedem Fall gelten die Formeln von Vieta:
\begin{equation}
p = -(x_1+x_2),\qquad q = x_1 x_2.
\end{equation}

\section{Komplexe Zahlen}\index{komplexe Zahl}
\subsection{Rechenoperationen}

\begin{gather}
\frac{z_1}{z_2}
= \frac{z_1\overline z_2}{z_2\overline z_2}
= \frac{z_1\overline z_2}{|z_2|^2},\\
\frac{1}{z} = \frac{\overline z}{z\overline z}
= \frac{\overline z}{|z|^2}.
\end{gather}

\subsection{Betrag}\index{Betrag!einer komplexen Zahl}
Für alle $z_1,z_2\in\C$ gilt:
\begin{gather}
|z_1z_2| = |z_1|\,|z_2|,\\
z_2\ne 0\implies \Big|\frac{z_1}{z_2}\Big|
= \frac{|z_1|}{|z_2|},\\
z\,\overline z = |z|^2.
\end{gather}

\subsection{Konjugation}\index{Konjugation!einer komplexen Zahl}
Für alle $z_1,z_2\in\C$ gilt:
\begin{gather}
\overline{z_1+z_2} = \overline z_1+\overline z_2,\qquad
\overline{z_1-z_2} = \overline z_1-\overline z_2,\\
\overline{z_1 z_2} = \overline z_1\,\overline z_2,\qquad
z_2\ne 0 \implies \overline{\Big(\frac{z_1}{z_2}\Big)}
= \frac{\overline z_1}{\overline z_2},\\
\overline{\overline z}=z,\qquad
|\overline{z}| = |z|,\qquad
z\,\overline z = |z|^2,\\
\real(z) = \frac{z+\overline z}{2},\qquad
\imag(z) = \frac{z-\overline z}{2\ui},\\
\overline{\cos(z)} = \cos(\overline z),\qquad
\overline{\sin(z)} = \sin(\overline z),\\
\overline{\exp(z)} = \exp(\overline z).
\end{gather}

\begin{table*}[t]
\caption{Rechnen mit komplexen Zahlen}
\bgroup
\def\arraystretch{1.4}
\begin{tabular}{|l|r|l|l|}
\hline
  \thbf{Name}
& \thbf{Operation}
& \thbf{Polarform}
& \thbf{kartesische Form}\\
\hline
  Identität
& $z$ & $=r\ee^{\ui\varphi}$
& $= a+b\ui$\\
\hline
  Addition
& $z_1+z_2$ &
& $= (a_1+a_2)+(b_1+b_2)\ui$\\
\hline
  Subtraktion
& $z_1-z_2$ &
& $= (a_1-a_2)+(b_1-b_2)\ui$\\
\hline
  Multiplikation
& $z_1 z_2$
& $= r_1 r_2 \ee^{\ui(\varphi_1+\varphi_2)}$
& $= (a_1 a_2 - b_1 b_2)+(a_1 b_2+a_2 b_1)\ui$\\
\hline
  Division
& $\displaystyle\frac{z_1}{z_2}$
& $\displaystyle =\frac{r_1}{r_2}\ee^{\ui(\varphi_1-\varphi_2)}$
& $\displaystyle =\frac{a_1 a_2 + b_1 b_2}{a_2^2+b_2^2}
   + \frac{a_2 b_1 - a_1 b_2}{a_2^2+b_2^2}\ui$\\
\hline
  Kehrwert
& $\displaystyle\frac{1}{z}$
& $\displaystyle =\frac{1}{r}\ee^{-\ui\varphi}$
& $\displaystyle =\frac{a}{a^2+b^2}-\frac{b}{a^2+b^2}\ui$\\
\hline
  Realteil
& $\real(z)$
& $=\cos\varphi$
& $=a$\\
\hline
  Imaginärteil
& $\imag(z)$
& $=\sin\varphi$
& $=b$\\
\hline
  Konjugation
& $\overline{z}$
& $=r\ee^{-\varphi\ui}$
& $=a-b\ui$\\
\hline
Betrag
& $|z|$
& $=r$
& $=\sqrt{a^2+b^2}$\\
\hline
  Argument
& $\arg(z)$
& $=\varphi$
& $\displaystyle = s(b)\arccos\Big(\frac{a}{r}\Big)$\\
\hline
\end{tabular}
\egroup\\
\\
$s(b):=\begin{cases}
+1 & \text{if}\;b\ge 0,\\
-1 & \text{if}\;b<0
\end{cases}$
\end{table*}

\section{Logik}
\subsection{Aussagenlogik}\index{Aussagenlogik}
\subsubsection{Boolesche Algebra}\index{boolesche Algebra}
\begin{table*}[t]
\caption{Boolesche Algebra}
\begin{tabular}{l|l|l}
\thbf{Disjunktion} & \thbf{Konjunktion} &\\
  $A\lor A \Leftrightarrow A$
& $A\land A \Leftrightarrow A$
& Idempotenzgesetze\\
  $A\lor 0 \Leftrightarrow A$
& $A\land 1 \Leftrightarrow A$
& Neutralitätsgesetze\\
  $A\lor 1 \Leftrightarrow 1$
& $A\land 0 = 0$
& Extremalgesetze\\
  $A\lor \overline A \Leftrightarrow 1$
& $A\land \overline A \Leftrightarrow 0$
& Komplementärgesetze\\
\noalign{\vspace{1em}}
  $A\lor B \Leftrightarrow B\lor A$
& $A\land B \Leftrightarrow B\land A$
& Kommutativgesetze\\
  $(A\lor B)\lor C \Leftrightarrow A\lor (B\lor C)$
& $(A\land B)\land C \Leftrightarrow A\land (B\land C)$
& Assoziativgesetze\\
  $\overline{A\lor B} \Leftrightarrow \overline A\land\overline B$
& $\overline{A\land B} \Leftrightarrow \overline A\lor\overline B$
& De Morgansche Regeln\\
  $A\lor (A\land B) \Leftrightarrow A$
& $A\land (A\lor B) \Leftrightarrow A$
& Absorptionsgesetze\\
\end{tabular}
\end{table*}

\noindent
\strong{Distributivgesetze}:
\begin{gather}
A\lor (B\land C) \iff (A\lor B)\land (A\lor C),\\
A\land (B\lor C) \iff (A\land B)\lor (A\land C).
\end{gather}

\subsubsection{Zweistellige Funktionen}
Es gibt 16 zweistellige boolesche\\
Funktionen.

\begin{tabular}{r|l}
\textbf{\texttt{AB}} & \thbf{Wert}\\
\texttt{00} & \texttt{a}\\
\texttt{01} & \texttt{b}\\
\texttt{10} & \texttt{c}\\
\texttt{11} & \texttt{d}
\end{tabular}

\begin{tabular}{r|l|l|l}
\thbf{Nr.}& \textbf{\texttt{dcba}} & \thbf{Fkt.} & \thbf{Name}\\
 0 & \texttt{0000} & 0 & Kontradiktion\\
 1 & \texttt{0001} & $\overline{A\lor B}$ & NOR\\
 2 & \texttt{0010} & $\overline{B\Rightarrow A}$\\
 3 & \texttt{0011} & $\overline A$\\
 4 & \texttt{0100} & $\overline{A\Rightarrow B}$\\
 5 & \texttt{0101} & $\overline{B}$\\
 6 & \texttt{0110} & $A\oplus B$ & Kontravalenz\index{Kontravalenz}\\
 7 & \texttt{0111} & $\overline{A\land B}$ & NAND\\
 8 & \texttt{1000} & $A\land B$ & Konjunktion\index{Konjunktion}\\
 9 & \texttt{1001} & $A\Leftrightarrow B$ & Äquivalenz\\
10 & \texttt{1010} & $B$ & Projektion\\
11 & \texttt{1011} & $A\Rightarrow B$ & Implikation\\
12 & \texttt{1100} & $A$ & Projektion\\
13 & \texttt{1101} & $B\Rightarrow A$ & Implikation\\
14 & \texttt{1110} & $A\lor B$ & Disjunktion\index{Disjunktion}\\
15 & \texttt{1111} & $1$ & Tautologie
\end{tabular}

\subsubsection[Darstellung mit Negation, Konjunktion und Disjunktion]%
{Darstellung mit Negation,\newline Konjunktion und Disjunktion}
\begin{gather}\label{eq:implication-definition}
A\Rightarrow B \iff \overline A\lor B,\\
(A\Leftrightarrow B) \iff
  (\overline A\land\overline B)\lor(A\land B),\\
A\oplus B \iff (\overline A\land B)\lor(A\land\overline B).
\end{gather}

\subsubsection{Tautologien}
Modus ponens:
\begin{equation}\label{eq:modus-ponens}
(A\Rightarrow B)\land A\implies B.
\end{equation}
Modus tollens:
\begin{equation}
(A\Rightarrow B)\land\overline B\implies\overline A.
\end{equation}
Modus tollendo ponens:
\begin{equation}
(A\lor B)\land\overline A \implies B.
\end{equation}
Modus ponendo tollens:
\begin{equation}
\overline{A\land B}\land A\implies\overline B.
\end{equation}
Kontraposition:\index{Kontraposition}
\begin{equation}
A\Rightarrow B \iff \overline B\Rightarrow \overline A.
\end{equation}
Beweis durch Widerspruch:\index{Widerspruch}
\begin{equation}
(\overline A\Rightarrow B\land\overline B)\implies A.
\end{equation}
Zerlegung einer Äquivalenz:
\begin{equation}
(A\Leftrightarrow B) \iff (A\Rightarrow B)\land(B\Rightarrow A).
\end{equation}
Kettenschluss:
\begin{equation}
(A\Rightarrow B)\land(B\Rightarrow C)\implies (A\Rightarrow C).
\end{equation}
Ringschluss:
\begin{equation}
\begin{split}
&(A\Rightarrow B)\land (B\Rightarrow C)\land(C\Rightarrow A)\\
&\implies (A\Leftrightarrow B)\land(A\Leftrightarrow C)\land(B\Leftrightarrow C).
\end{split}
\end{equation}
Ringschluss, allgemein:
\begin{equation}
\begin{split}
& (A_1{\Rightarrow }A_2)\land\ldots\land(A_{n-1}{\Rightarrow}A_n)
\land(A_n{\Rightarrow}A_1)\\
& \implies \forall i,j\,[A_i\Leftrightarrow A_j].
\end{split}
\end{equation}
Ersetzungsregel:

Für jede Funktion $P\colon\{0,1\}\to\{0,1\}$ gilt:
\begin{equation}
P(A)\land (A\Leftrightarrow B)\implies P(B).
\end{equation}
Regel zur Implikation:
\begin{equation}
A\land B\Rightarrow C \iff A\Rightarrow (B\Rightarrow C).
\end{equation}
Vollständige Fallunterscheidung:
\begin{gather}
(A\Rightarrow C)\land (B\Rightarrow C)\implies (A\oplus B\Rightarrow C),\\
(A\Rightarrow C)\land (B\Rightarrow C)\iff (A\lor B\Rightarrow C).
\end{gather}
Vollständige Fallunterscheidung, allgemein:
\begin{gather}
\textstyle \forall k[A_k\Rightarrow C]
\implies (\bigoplus_{k=1}^n A_k\Rightarrow C),\\
\forall k[A_k\Rightarrow C]
\iff (\exists k[A_k]\Rightarrow C).
\end{gather}

\subsubsection{Schlussregeln}
\strong{Ersetzungsregel.} Sei $p(\varphi)$ eine aussagenlogische
Formel in expliziter Abhängigkeit von der Formelvariablen $\varphi$.
Es gilt
\begin{equation}
\{p(\varphi),\varphi\leftrightarrow\psi\}\vdash p(\psi).
\end{equation}
\strong{Beispiel.} Betrachte $\varphi\land A\rightarrow B$ mit
$\varphi:=(A\rightarrow B)$, was expandiert wird zu
\[(A\rightarrow B)\land A\rightarrow B.\qquad\text{(s. \eqref{eq:modus-ponens})}\]
Nun gilt nach \eqref{eq:implication-definition} aber
\[A\rightarrow B\leftrightarrow \overline A\lor B.\]
Daher lässt sich folgern:
\[(\overline A\lor B)\land A\rightarrow B.\]

\subsubsection{Metatheoreme}
\strong{Korrektheit der Aussagenlogik.}

Für die Aussagenlogik gilt:
\begin{equation}
(\Gamma\vdash\psi)\implies (\Gamma\models\psi).
\end{equation}

\noindent
\strong{Vollständigkeit der Aussagenlogik.}

Für die Aussagenlogik gilt:
\begin{equation}
(\Gamma\models\psi)\implies (\Gamma\vdash\psi).
\end{equation}

\noindent
\strong{Deduktionstheorem (syntaktisch).}

Für die Aussagenlogik gilt:
\begin{equation}
(\Gamma\cup\{\varphi\}\vdash\psi)\iff (\Gamma\vdash\varphi\rightarrow\psi).
\end{equation}

Infolge gilt auch:
\begin{equation}
\begin{split}
&(\{\varphi_1,\ldots,\varphi_n\}\vdash\psi)\\
&\iff (\vdash \varphi_1\land\ldots\land\varphi_n\rightarrow\psi).
\end{split}
\end{equation}

\noindent
\strong{Deduktionstheorem (semantisch).}

Für die Aussagenlogik gilt:
\begin{equation}
(\Gamma\cup\{\varphi\}\models\psi)\iff (\Gamma\models\varphi\rightarrow\psi).
\end{equation}

Infolge gilt auch:
\begin{equation}
\begin{split}
&(\{\varphi_1,\ldots,\varphi_n\}\models\psi)\\
&\iff (\models \varphi_1\land\ldots\land\varphi_n\rightarrow\psi).
\end{split}
\end{equation}

\noindent
\strong{Einsetzungsregel.}

Sei $v$ eine metasprachliche Variable, die für eine beliebige
objektsprachliche Variable steht. Dann gilt:
\begin{equation}
(\models\varphi) \implies (\models\varphi[v:=\psi]).
\end{equation}
D.\,h. wenn in der tautologischen Formel $\varphi$ jedes auftreten
der Variable $v$ gegen die Formel $\psi$ ersetzt wird, ergibt
sich wieder eine tautologische Formel.

%\newpage
\subsection{Prädikatenlogik}
\subsubsection{Rechenregeln}
Verneinung (De Morgansche Regeln):
\begin{gather}
\overline{\forall x[P(x)]}\iff \exists x[\overline{P(x)}],\\
\overline{\exists x[P(x)]}\iff \forall x[\overline{P(x)}].
\end{gather}
Verallgemeinerte Distributivgesetze:
\begin{gather}
P\lor\forall x[Q(x)] \iff \forall x[P\lor Q(x)],\\
P\land\exists x[Q(x)] \iff \exists x[P\land Q(x)].
\end{gather}
Verallgemeinerte Idempotenzgesetze:
\begin{gather}
\begin{split}
\exists x{\in}M\,[P] & \iff
(M\ne\{\})\land P\\
& \iff\begin{cases}
P & \text{wenn}\; M\ne\{\},\\
0 & \text{wenn}\; M=\{\}.
\end{cases}
\end{split}\\
\begin{split}
\forall x{\in}M\,[P]& \iff
(M=\{\})\lor P\\
&\iff\begin{cases}
P & \text{wenn}\; M\ne\{\},\\
1 & \text{wenn}\; M=\{\}.
\end{cases}
\end{split}
\end{gather}
%\newpage\noindent
Äquivalenzen:
\begin{gather}
\hspace{-2em}\forall x\forall y[P(x,y)] \iff \forall y\forall x[P(x,y)],\\
\hspace{-2em}\exists x\exists y[P(x,y)] \iff \exists y\exists x[P(x,y)],\\
\hspace{-2em}\forall x[P(x)\land Q(x)] \iff \forall x[P(x)]\land\forall x[Q(x)],\\
\hspace{-2em}\exists x[P(x)\lor Q(x)] \iff \exists x[P(x)]\lor\exists x[Q(x)],\\
\hspace{-2em}\forall x[P(x)\Rightarrow Q] \iff \exists x[P(x)]\Rightarrow Q,\\
\hspace{-2em}\forall x[P\Rightarrow Q(x)] \iff P\Rightarrow\forall x[Q(x)],\\
\hspace{-2em}\exists x[P(x)\Rightarrow Q(x)]
  \iff\forall x[P(x)]\Rightarrow\exists x[Q(x)].
\end{gather}
% \newpage\noindent
Implikationen:
\begin{gather}
\hspace{-2em}\exists x\forall y[P(x,y)]\implies \forall y\exists x[P(x,y)],\\
\hspace{-2em}\forall x[P(x)]\lor\forall x[Q(x)]\implies\forall x[P(x)\lor Q(x)],\\
\hspace{-2em}\exists x[P(x)\land Q(x)]\implies
  \exists x[P(x)]\land \exists x[Q(x)],\\
\hspace{-2em}\forall x[P(x)\Rightarrow Q(x)]\implies
  (\forall x[P(x)]\Rightarrow\forall x[Q(x)]),\\
\hspace{-2em}\forall x[P(x)\Leftrightarrow Q(x)]\implies
  (\forall x[P(x)]\Leftrightarrow\forall x[Q(x)]).
\end{gather}

\newpage
\subsubsection{Endliche Mengen}
Sei $M=\{x_1,\ldots,x_n\}$. Es gilt:
\begin{gather}
\forall x{\in}M\,[P(x)]\iff P(x_1)\land\ldots\land P(x_n),\\
\exists x{\in}M\,[P(x)]\iff P(x_1)\lor\ldots\lor P(x_n).
\end{gather}

\subsubsection{Beschränkte Quantifizierung}
\begin{gather}
\begin{split}
& \forall x{\in}M\,[P(x)]\defiff\forall x[x\notin M\lor P(x)]\\
& \quad\iff\forall x[x\in M\Rightarrow P(x)],
\end{split}\\
\exists x{\in}M\,[P(x)]\defiff\exists x[x\in M\land P(x)],\\
\forall x{\in}M{\setminus}N\,[P(x)]\iff \forall x[x\notin N\Rightarrow P(x)].
\end{gather}

\subsubsection[Quantifizierung über Produktmengen]%
{Quantifizierung über\newline Produktmengen}
\begin{gather}
\forall(x,y)\,[P(x,y)]\iff \forall x\forall y[P(x,y)],\\
\exists(x,y)\,[P(x,y)]\iff \exists x\exists y[P(x,y)].
\end{gather}
Analog gilt
\begin{gather}
\forall(x,y,z)\,\iff \forall x\forall y\forall z,\\
\exists(x,y,z)\,\iff \exists x\exists y\exists z
\end{gather}
usw.

\subsubsection{Alternative Darstellung}
Sei $P\colon G\to\{0,1\}$ und $M\subseteq G$.
Mit $P(M)$ ist die Bildmenge von $P$ bezüglich $M$ gemeint.
Es gilt
\begin{equation}
\begin{split}
&\forall x{\in}M\,[P(x)] \iff P(M)=\{1\}\\
& \iff M\subseteq\{x{\in}G\mid P(x)\}
\end{split}
\end{equation}
und
\begin{equation}
\begin{split}
& \exists x{\in}M\,[P(x)] \iff \{1\}\subseteq P(M)\\
& \iff M\cap\{x{\in}G\mid P(x)\}\ne\{\}.
\end{split}
\end{equation}

\subsubsection{Eindeutigkeit}
Quantor für eindeutige Existenz:
\begin{equation}
\begin{split}
&\exists!x\,[P(x)]\\
&:\Longleftrightarrow\; \exists x\,[P(x)\land \forall y\,[P(y)\Rightarrow x=y]]\\
&\iff \exists x\,[P(x)]\land \forall x\forall y[P(x)\land P(y)\Rightarrow x=y].
\end{split}
\end{equation}

\newpage
\section{Mengenlehre}
\subsection{Definitionen}
Aufzählende Notation:
\begin{equation}
\hspace{-1em} a\in\{x_1,\ldots,x_n\} :\Leftrightarrow a=x_1\lor\ldots\lor a=x_n.
\end{equation}
Beschreibende Notation:
\begin{gather}
a\in\{x\mid P(x)\}\defiff P(a),\\
\{x\in M\mid P(x)\} := \{x\mid x\in M\land P(x)\},\\
\hspace{-1em}\{f(x)\mid P(x)\} := \{y\mid \exists x(y=f(x)\land P(x))\}.
\end{gather}
Teilmengenrelation:
\begin{equation}
A\subseteq B\defiff \forall x\,(x\in A\implies x\in B).
\end{equation}
Gleichheit:
\begin{equation}
A=B\defiff \forall x\,(x\in A\iff x\in B).
\end{equation}
Vereinigungsmenge:
\begin{equation}
A\cup B:=\{x\mid x\in A\lor x\in B\}.
\end{equation}
Schnittmenge:
\begin{equation}
A\cap B:=\{x\mid x\in A\land x\in B\}.
\end{equation}
Differenzmenge:
\begin{equation}
A\setminus B:=\{x\mid x\in A\land x\not\in B\}.
\end{equation}
Symmetrische Differenz:
\begin{equation}
A\triangle B:=\{x\mid x\in A\oplus x\in B\}.
\end{equation}
Komplementärmenge:
\begin{equation}
A^\comp := G\setminus A.\qquad (\text{$G$: Grundmenge})
\end{equation}
Vereinigung über indizierte Mengen:
\begin{equation}
\bigcup_{i\in I} A_i := \{x\mid\exists i{\in}I\,(x\in A_i)\}.
\end{equation}
Schnitt über indizierte Mengen:
\begin{equation}
\bigcap_{i\in I} A_i := \{x\mid\forall i{\in}I\,(x\in A_i)\}.
\end{equation}


\subsection{Boolesche Algebra}
\begin{table*}[t]
\caption{Boolesche Algebra}
\begin{tabular}{l|l|l}
\thbf{Vereinigung} & \thbf{Schnitt} &\\
  $A\cup A = A$
& $A\cap A = A$
& Idempotenzgesetze\\
  $A\cup \{\} = A$
& $A\cap G = A$
& Neutralitätsgesetze\\
  $A\cup G = G$
& $A\cap \{\} = \{\}$
& Extremalgesetze\\
  $A\cup \overline A = G$
& $A\cap \overline A = \{\}$
& Komplementärgesetze\\
\noalign{\vspace{1em}}
  $A\cup B = B\cup A$
& $A\cap B = B\cap A$
& Kommutativgesetze\\
  $(A\cup B)\cup C = A\cup (B\cup C)$
& $(A\cap B)\cap C = A\cap (B\cap C)$
& Assoziativgesetze\\
  $\overline{A\cup B} = \overline A\cap\overline B$
& $\overline{A\cap B} = \overline A\cup\overline B$
& De Morgansche Regeln\\
  $A\cup (A\cap B) = A$
& $A\cap (A\cup B) = A$
& Absorptionsgesetze\\
\end{tabular}\\
\\
$G$: Grundmenge
\end{table*}

\noindent
\strong{Distributivgesetze}:
\begin{gather}
M\cup (A\cap B) = (M\cup A)\cap (M\cup B),\\
M\cap (A\cup B) = (M\cap A)\cup (M\cap B).
\end{gather}

\subsection{Teilmengenrelation}
Zerlegung der Gleichheit:
\begin{equation}
A=B \iff A\subseteq B \land B\subseteq A.
\end{equation}
Umschreibung der Teilmengenrelation:
\begin{equation}
\begin{split}
A\subseteq B &\iff A\cap B=A\\
& \iff A\cup B=B\\
& \iff A\setminus B=\{\}.
\end{split}
\end{equation}
Kontraposition:
\begin{equation}
A\subseteq B = B^\comp\subseteq A^\comp.
\end{equation}

\subsection{Natürliche Zahlen}
\subsubsection{Von-Neumann-Modell}
Mengentheoretisches Modell der natürlichen Zahlen:
\begin{equation}
\begin{split}
& 0:=\{\},\quad 1:=\{0\},\quad 2:=\{0,1\},\\
& 3:=\{0,1,2\},\quad \text{usw.}
\end{split}
\end{equation}
Nachfolgerfunktion:
\begin{equation}
x' := x\cup\{x\}.
\end{equation}
\subsubsection{Vollständige Induktion}
Ist $A(n)$ mit $n\in\N$
eine Aussageform, so gilt:
\begin{equation}
\begin{split}
& A(n_0)\land \forall n\ge n_0\,[A(n)\Rightarrow A(n+1)]\\
& \implies \forall n\ge n_0\,[A(n)].
\end{split}
\end{equation}
Die Aussage $A(n_0)$ ist der \emph{Induktionsanfang}.

Die Implikation
\begin{equation}
A(n)\Rightarrow A(n+1)
\end{equation}
heißt \emph{Induktionsschritt}. Beim Induktionsschritt muss
$A(n+1)$ gezeigt werden, wobei $A(n)$ als gültig vorausgesetzt werden
darf.

% \newpage
\subsection{ZFC-Axiome}

Axiom der Bestimmtheit:
\begin{equation}
\forall A\forall B\,[A=B\iff\forall x\,[x\in A\Leftrightarrow x\in B]].
\end{equation}
Axiom der leeren Menge:
\begin{equation}
\exists M\forall x\,[x\notin M].
\end{equation}
Axiom der Paarung:
\begin{equation}
\forall x\forall y\exists M\forall a\,[a\in M\iff x=a\lor y=a].
\end{equation}
Axiom der Vereinigung:
\begin{equation}
\forall S\exists M\forall x\,[x\in M\iff\exists A{\in}S\,[x\in A]].
\end{equation}
Axiom der Aussonderung:
\begin{equation}
\forall A\exists M\forall x\,[x\in M\iff x\in A\land\varphi(x)].
\end{equation}
Axiom des Unendlichen:
\begin{equation}
\exists M\,[\{\}\in M\land\forall x{\in}M\,[x\cup\{x\}\in M]].
\end{equation}
Axiom der Potenzmenge:
\begin{equation}
\forall A\exists M\forall T\,[T\in M\iff T\subseteq A].
\end{equation}
Axiom der Ersetzung:
\begin{equation}
\begin{split}
&\forall a{\in}A\;\exists^{=1} b\,[\varphi(a,b)]\\
&\implies\exists B\,\forall b\,[b\in B\iff\exists a{\in}A\,[\varphi(a,b)]].
\end{split}
\end{equation}
Axiom der Fundierung:
\begin{equation}
\forall A\,[A\ne\{\}\implies\exists x{\in}A\,[x\cap A=\{\}]].
\end{equation}
Auswahlaxiom:
\begin{equation}
\begin{split}
&\forall x,y{\in}A\,[x\ne y\implies x\cap y=\{\}]\\
&\quad\land\forall x{\in}A\,[x\ne\{\}]\\
&\implies\exists M\;\forall x{\in}A\;\exists^{=1}u{\in}x\,[u\in M].
\end{split}
\end{equation}

\newpage
\subsection{Kardinalität}
\begin{definition}[Gleichmächtigkeit]
Zwei Mengen $M,N$ heißen \emdef{gleichmächtig}, notiert als
$|M|=|N|$, wenn es eine bijektive Abbildung $f\colon M\to N$ gibt.

Eine Menge $M$ heißt \emdef{weniger mächtig oder gleichmächtig},
notiert als $|M|\le|N|$, wenn es eine injektive Abbildung
$f\colon M\to N$ gibt. Äquivalent dazu ist, dass es eine
surjektive Abbildung $g\colon N\to M$ gibt.

Eine Menge heißt \emdef{abzählbar unendlich}, wenn sie gleichmächtig
zu den natürlichen Zahlen ist.
\end{definition}
Gleichmächtigkeit ist eine Äquivalenzrelation.
\begin{definition}[Kardinalzahl]
Die Äquivalenzklassen
\begin{equation}
|M| := \{N\mid\;{\scriptstyle |M|=|N|}\}
\end{equation}
heißen \emdef{Kardinalzahlen}.
\end{definition}

\strong{Satz von Cantor-Bernstein.}

Aus $|M|\le |N|$ und $|N|\le |M|$ folgt $|M|=|N|$.

\subsubsection{Potenzmengen}

\strong{Satz von Cantor.}
Für jede Menge gilt $|M|<|2^M|$.

Ist $M$ endlich, dann gilt $|M|=2^{|M|}$.


\section{Funktionen}
\subsection{Injektionen}\index{injektiv}
\begin{definition}[Injektion]
Eine Funktion $f\colon A\to B$ heißt \emdef{injektiv},
wenn
\begin{equation}
\forall x_1,x_2\in A\,[f(x_1)=f(x_2)\implies x_1=x_2]
\end{equation}
gilt.
\end{definition}

\begin{definition}[Linksinverse]
Sei $f\colon A\to B$. Eine Funktion $g\colon B\to A$ mit
\begin{equation}
g\circ f = \id_A
\end{equation}
heißt \emdef{Linksinverse} von $f$.
\end{definition}
Eine Funktion ist genau dann injektiv, wenn sie eine Linksinverse
besitzt. Zu einer Injektion kann es aber mehrere unterschiedliche
Linksinverse geben.

\subsection{Surjektionen}\index{surjektiv}
\begin{definition}[Surjektion]
Eine Funktion $f\colon A\to B$ heißt \emdef{surjektiv},\\
wenn $f(A)=B$ ist. Damit ist gemeint, dass jedes Element
der Zielmenge wenigstens einmal der Funktionswert von einem
Element der Definitionsmenge ist.
\end{definition}

\newpage
\begin{definition}[Rechtsinverse]
Sei $f\colon A\to B$. Eine Funktion $g\colon B\to A$ mit
\begin{equation}
f\circ g = \id_B
\end{equation}
heißt \emdef{Rechtsinverse} von $f$.
\end{definition}
Eine Funktion ist genau dann surjektiv, wenn sie eine Rechtsinverse
besitzt. Zu einer Surjektion kann es aber mehrere unterschiedliche
Rechtsinverse geben.

\subsection{Bijektionen}\index{bijektiv}
\begin{definition}[Bijektion]
Eine Funktion $f\colon A\to B$ heißt \emdef{bijektiv},
wenn sie injektiv und surjektiv ist.

Eine Funktion $f\colon A\to B$ ist genau dann bijektiv, wenn es
ein $g$ mit
\begin{equation}
g\circ f = \id_A\quad\text{und}\quad f\circ g = \id_B
\end{equation}
gibt. Wenn $f$ bijektiv ist, so gibt es $g$ genau einmal und
$g$ wird die \emph{Umkehrfunktion}\index{Umkehrfunktion}
oder \emph{Inverse}
von $f$ genannt und als $f^{-1}$ notiert.
\end{definition}

\subsection{Komposition}\index{Komposition}
\begin{definition}[Komposition]
Für zwei Funktionen $f\colon A\to B$
und $g\colon B\to C$ ist die \emdef{Komposition}
($g$ nach $f$)
durch
\begin{equation}\label{eq:composition}
g\circ f\colon A\to C,\quad (g\circ f)(x) := g(f(x))
\end{equation}
definiert.
\end{definition}
Für die Komposition gilt das Assozativgesetz:
\begin{equation}
(f\circ g)\circ h = f\circ(g\circ h).
\end{equation}

Die Komposition von Injektionen ist eine Injektion.

Die Komposition von Surjektionen ist eine Surjektion.

Die Komposition von Bijektionen ist eine Bijektion.

Sind $f,g$ Bijektionen, so gilt
\begin{equation}
(g\circ f)^{-1} = f^{-1}\circ g^{-1}.
\end{equation}

Ist $g\circ f$ injektiv, so ist $f$ injektiv.

Ist $g\circ f$ surjektiv, so ist $g$ surjektiv.

Ist $g\circ f$ bijektiv, so ist $f$ injektiv und $g$ surjektiv.

\begin{definition}[Iteration]
Für eine Funktion $\varphi\colon A\to A$ wird
\begin{equation}
\varphi^0:=\operatorname{id}_A,\quad \varphi^{n+1}:=\varphi^n\circ\varphi
\end{equation}
\emdef{Iteration}\index{Iteration} von $\varphi$ genannt.
\end{definition}

\newpage
\subsection{Einschränkung}\index{Einschränkung}
\begin{definition}[Einschränkung]
Sei $f\colon A\to B$ und $M\subseteq A$.
Die Funktion $g(x)=f(x)$ mit $g\colon M\to B$ wird \emdef{Einschränkung}
von $f$ genannt und mit $f|_M$ notiert.
\end{definition}
Sei $f\colon A\to B$ und $M\subseteq A$.
Mit der Inklusionsabbildung $i(x):=x$ mit $i\colon M\to A$ gilt:
\begin{equation}
f|_M = f\circ i.
\end{equation}
Es gilt
\begin{equation}
g\circ (f|_M) = (g\circ f)|_M.
\end{equation}

\subsection{Bild}\index{Bild}
\begin{definition}[Bild]
Ist $f\colon A\to B$ und $M\subseteq A$, so wird
\begin{equation}
f(M) := \{f(x)\mid x\in M\}
\end{equation}
das \emdef{Bild} von $M$ unter $f$ genannt.
\end{definition}
Es gilt
\begin{align}
&f(M\cup N) = f(M)\cup f(N),\\
&f(M\cap N) \subseteq f(M)\cap f(N),\\
&f\Big(\bigcup_{i\in I}M_i\Big) = \bigcup_{i\in I} f(M_i),\\
&I\ne\emptyset\implies f\Big(\bigcap_{i\in I} M_i\Big) \subseteq \bigcap_{i\in I} f(M_i),\\
&M\subseteq N\implies f(M)\subseteq f(N),\\
&f(\emptyset) = \emptyset,\\
&(g\circ f)(M) = g(f(M)),\\
&f(M) = \bigcup_{x\in M} f(\{x\}).
\end{align}

\subsection{Urbild}\index{Urbild}
\begin{definition}[Urbild]
Ist $f\colon A\to B$, so wird
\begin{equation}
f^{-1}(M) := \{x\in A\mid f(x)\in M\}.
\end{equation}
das \emdef{Urbild} von $M$ unter $f$ genannt.
\end{definition}
Es gilt
\begin{align}
& f^{-1}(M\cup N) = f^{-1}(M)\cup f^{-1}(N),\\
& f^{-1}(M\cap N) = f^{-1}(M)\cap f^{-1}(N),\\
& f^{-1}\Big(\bigcup_{i\in I}M_i\Big) = \bigcup_{i\in I} f^{-1}(M_i),\\
& I\ne\emptyset\implies f^{-1}\Big(\bigcap_{i\in I} M_i\Big) = \bigcap_{i\in I}f^{-1}(M_i),\\
& M\subseteq N\implies f^{-1}(M)\subseteq f^{-1}(N),\\
& f^{-1}(\emptyset) = \emptyset,\\
& f^{-1}(B) = A,\\
& f^{-1}(M\setminus N) = f^{-1}(M)\setminus f^{-1}(N),\\
& f^{-1}(B\setminus M) = B\setminus f^{-1}(M),\\
& (g\circ f)^{-1}(M) = f^{-1}(g^{-1}(M)),\\
& (f|_M)^{-1}(N) = M\cap f^{-1}(N).
\end{align}

\newpage
\phantom{x}

\newpage
\section{Formale Systeme}
\subsection{Formale Sprachen}
\begin{definition}[Formale Sprache]
Eine \emdef{formale Sprache} $L$ ist eine Teilmenge der kleenschen
Hülle über einer Menge $\Sigma$, kurz $L\subseteq\Sigma^*$.
Die Menge $\Sigma$ wird \emdef{Alphabet} genannt,
ihre Elemente heißen \emdef{Symbole}.

Die kleensche Hülle $\Sigma^*$ besteht aus allen möglichen
Konkatenationen von Symbolen aus $\Sigma$. Die Konkatenationen
von $\Sigma^*$ heißen \emdef{Wörter}. Die leere Konkatenation ist
zulässig und wird mit $\varepsilon$ notiert. Die Elemente von $L$ heißen
\emdef{wohlgeformte Wörter} oder \emdef{wohlgeformte Formeln},
engl. \emdef{well formed formulas}, kurz \emdef{wff}.
\end{definition}

Ein Wort $a$ ist ein Tupel
\begin{equation}
a = (a_1,\ldots, a_m).\qquad (a_k\in\Sigma)
\end{equation}
Sind $a,b$ zwei Wörter, dann ist mit $ab$ deren Konkatenation
gemeint:
\begin{equation}
ab := (a_1,\ldots,a_m,b_1,\ldots b_n).
\end{equation}
Es gilt $\varepsilon a=a$ und $a\varepsilon=a$.
Bei $\varepsilon$ handelt es sich um das leere Tupel.

\begin{definition}[Konkatenation von Sprachen]
\emdef{Konkatenation} von $L_1$ und $L_2$:
\begin{equation}
L_1\circ L_2 := \{ab\mid a\in L_1, b\in L_2\}.
\end{equation}
\end{definition}

\begin{definition}[Potenz einer Sprache]
\emdef{Potenzen} von $L$:
\begin{align}
L^0 &:= \{\varepsilon\},\\
L^n &:= L^{n-1}\circ L.
\end{align}
\end{definition}

\begin{definition}[Kleensche Hülle einer Sprache]
\emdef{Kleensche Hülle} von $L$:
\begin{equation}
L^* := \bigcup_{k\in\N_0} L^k.
\end{equation}

\emdef{Positive Hülle} von $L$:
\begin{equation}
L^+ := \bigcup_{k\in\N_1} L^k.
\end{equation}
\end{definition}

% \newpage
\subsection{Formale Grammatiken}
\begin{definition}[Formale Grammatik]
Eine \emdef{formale Grammatik} ist ein Tupel $(N,\Sigma,P,S)$,
wobei $N$ die \emdef{Nonterminalsymbolen}\index{Nonterminalsymbol},
$\Sigma$ die \emdef{Terminalsymbolen}\index{Terminalsymbol},
$P$ die \emdef{Produktionsregeln}\index{Produktionsregel} sind
und $S$ ein \emdef{Startsymbol}\index{Startsymbol} ist.
Die Mengen $N,\Sigma,P$ müssen endlich sein. Die Mengen $N$ und
$\Sigma$ müssen disjunkt sein. Bei $\Sigma$ handelt es sich um
ein Alphabet. Das Startsymbol ist ein Element $S\in N$.

Bei $P$ handelt es sich um eine Relation
\begin{equation}\label{eq:einfache-Produktionsregeln}
P\subseteq N\times (N\cup\Sigma)^*
\end{equation}
oder allgemeiner
\begin{equation}
P\subseteq (N\cup\Sigma)^*\setminus\Sigma^*\times (N\cup\Sigma)^*.
\end{equation}
Produktionsregeln werden in der Form $n\to w$ notiert und drücken aus,
dass in jedem Wort das Nonterminalsymbol $n$ durch das Wort $w$ ersetzt
werden darf. Allgemeiner bedeutet $t\to w$, dass ein Teilwort $t$
durch $w$ ersetzt werden darf.

Die Produktionsregeln werden ausgehend vom Startsymbol immer weiter
angewendet bis keine Nonterminalsymbole mehr vorhanden sind.
Die Menge aller möglichen Produktionen bildet
eine formale Sprache $L\subseteq\Sigma^*$.
\end{definition}

Für Produktionsregeln der Form \eqref{eq:einfache-Produktionsregeln}
wurde eine Kurznotation geschaffen, die EBNF:

\begin{tabular}{l|l}
\verb|Symbol| & Nonterminalsymbol\\
\verb|"Symbol"| & Terminalsymbol\\
\verb|w1, w2| & $w_1w_2$ (Konkatenation)\\
\verb/n = w1 | w2./ & $n\to w_1,\; n\to w_2$\\
\verb|n = {w}.| & $n\to \varepsilon,\; n\to wn$\\
\verb|n = [w].| & $n\to w,\; n\to wn$
\end{tabular}

\subsection{Formale Systeme}
\begin{definition}[Formales System]
Ein \emdef{formales System} ist ein Tupel $(\Sigma,L,A,R)$, wobei
$\Sigma$ ein Alphabet, $L$ eine formale Sprache über
dem Alphabet, $A$ eine Menge von Axiomen und $R$ eine Menge von
Ableitungsrelationen ist. Die Menge der \emdef{Axiome} ist eine
beliebige Teilmenge von $L$. 
Eine \emdef{Ableitungsrelation} ist eine zwei oder mehrstellige
Relation über $L$, die
\begin{equation}
a_1,\ldots,a_n\vdash b
\end{equation}
geschrieben wird. Eine wohlgeformte Formel wird $\emdef{Satz}$
genannt, wenn sie ein Axiom ist oder über eine Kette von
Ableitungen aus den Axiomen folgt.
\end{definition}

\subsection{Semantik}
\begin{definition}[Interpretation (Aussagenlogik)]
Eine \emph{Interpretation}\index{Interpretation}
$I\colon V\to\{0,1\}$ ist eine Abbildung,
welche jeder logischen Variablen einen Wahrheitswert zuordnet.

Eine \emph{Interpretation} $I\colon F\to\{0,1\}$ erweitert den
Definitionsbereich einer Interpretation wie folgt auf die
Menge aller wohlgeformten Formeln:
\begin{gather}
I(\varphi\land\psi) = (I(\varphi)\land I(\psi)),\\
I(\varphi\lor\psi) = (I(\varphi)\lor I(\psi)),\\
I(\varphi\rightarrow\psi) = (I(\varphi)\rightarrow I(\psi)),\\
I(\varphi\leftrightarrow\psi) = (I(\varphi)\leftrightarrow I(\psi)),\\
I(\neg\varphi) = (\neg I(\varphi)).
\end{gather}
Die rechten Seiten werden hierbei entsprechend den Wertetabellen
ausgewertet.
\end{definition}

\begin{definition}[Modellrelation]
Sei $\Gamma=\{\varphi_1,\ldots,\varphi_n\}$ eine endliche Menge
von Formeln und sei $\psi$ eine Formel. Die Formelmenge $\Gamma$
\emph{modelliert} $\psi$, wenn jede Interpretation, die alle
Formeln in $\Gamma$ erfüllt, auch $\psi$ erfüllt. Kurz:
\begin{equation}
(\Gamma\models\psi) \defiff \forall I[\forall\varphi{\in}\Gamma(I(\varphi))\Rightarrow I(\psi)].
\end{equation}
\end{definition}

\newpage
\section{Mathematische Strukturen}\label{sec:Strukturen}
\subsubsection*{Axiome}

\noindent\bsf{E:} Abgeschlossenheit.
\ibox{Die Verknüpfung führt nicht aus der Menge heraus.}

\noindent\bsf{A:} Assoziativgesetz.
\ibox{$\forall a,b,c\bright (a*b)*c = a*(b*c)\bleft$.}

\noindent\bsf{N:} Existenz des neutralen Elements.
\ibox{$\exists e\forall a\bright e*a=a*e=a\bleft$.}

\noindent\bsf{I:} Existenz der inversen Elemente.
\ibox{$\forall a\exists b\bright a*b=b*a=e\bleft$.}

\noindent\bsf{K:} Kommutativgesetz.
\ibox{$\forall a,b\bright a*b=b*a\bleft.$}

\noindent
\bsf{I*:} Existenz der multiplikativ inversen Elemente.
\ibox{$\forall a{\ne}0\;\exists b\bright a*b=b*a=1\bleft$.}

\noindent\bsf{Dl:} Linksdistributivgestz.
\ibox{$\forall a,x,y\bright a*(x+y) = a*x+a*y\bleft$.}

\noindent\bsf{Dr:} Rechtsdistributivgesetz.
\ibox{$\forall a,x,y\bright (x+y)*a = x*a+y*a\bleft$.}

\noindent\bsf{D:} Distributivgesetze.
\ibox{Dl und Dr.}

\noindent\bsf{T:} Nullteilerfreiheit.
\ibox{$\forall a,b\bright a\ne 0\land b\ne 0\implies a*b\ne 0\bleft$}
\ibox{bzw. die Kontraposition}
\ibox{$\forall a,b\bright a*b=0\implies a=0\lor b=0\bleft$.}

\noindent\bsf{U:} Unterscheibarkeit von Null- und Einselement.
\ibox{Die neutralen Elemente bezüglich Addition und}
\ibox{Multiplikation sind unterschiedlich.}

\subsubsection*{Strukturen}
Strukturen mit einer inneren Verknüpfung:\\
\begin{tabular}{l|l}
\bsf{EA} & Halbgruppe\\
\bsf{EAN} & Monoid\\
\bsf{EANI} & Gruppe\\
\bsf{EANIK} & abelsche Gruppe
\end{tabular}

\noindent
Strukturen mit zwei inneren Verknüpfungen:\\
\begin{tabular}{l|l}
\bsf{EANIK, EA, D}\dotfill & Ring\\
\bsf{EANIK, EAK, D}\dotfill & kommutativer Ring\\
\bsf{EANIK, EAN, D}\dotfill & unitärer Ring\\
\bsf{EANIK, EANK, DTU} & Integritätsring\\
\bsf{EANIK, EANI*K, DTU} & Körper
\end{tabular}

\newpage
\subsubsection*{Axiome für Relationen}

\noindent\bsf{R:} Reflexivität.
\ibox{$\forall a\,(a R a)$.}

\noindent\bsf{S:} Symmetrie.
\ibox{$\forall a,b\,(aRb\iff bRa)$.}

\noindent\bsf{T:} Transitivität.
\ibox{$\forall a,b,c\,(aRb\land bRc\implies aRc)$.}

\noindent\bsf{An:} Antisymmetrie.
\ibox{$\forall a,b\,(aRb\land bRa\implies a=b)$.}

\noindent\bsf{L:} Linearität.
\ibox{$\forall a,b\,(aRb\lor bRa)$.}

\noindent\bsf{Ri:} Irrreflexivität.
\ibox{$\forall a\,(\neg aRa)$.}

\noindent\bsf{A:} Asymmetrie.
\ibox{$\forall a,b\,(aRb\implies \neg bRa)$.}

\noindent\bsf{Min:} Existenz der Minimalelemente.
\ibox{$\forall T{\subseteq}M, T{\ne}\emptyset\;\exists x{\in}T\;\forall y{\in}T{\setminus}\{x\}\,(x<y)$.}

\subsubsection*{Relationen}
\begin{tabular}{l|l}
\bsf{RST}\dotfill & Äquivalenzrelation\\
\bsf{RAnT}\dotfill & Halbordnung\\
\bsf{RAnTL}\dotfill & Totalordnung\\
\bsf{RiAT}\dotfill & strenge Halbordnung\\
\bsf{RiATL}\dotfill & strenge Totalordnung\\
\bsf{RiATLMin} & Wohlordnung
\end{tabular}


\chapter{Funktionen}
\section{Elementare Funktionen}
\subsection{Exponentialfunktion}
\begin{Definition}
$\exp\colon\C\to\C$ mit
\begin{equation}
\exp(x) := \sum_{k=0}^{\infty} \frac{x^k}{k!}
= 1+x+\frac{x^2}{2!}+\frac{x^3}{3!}+\ldots
\end{equation}
\end{Definition}
\noindent
Die Einschränkung von $\exp$ auf $\R$ ist injektiv und
hat die Bildmenge $\{x{\in}\R\mid x>0\}$.

Für alle $x,y\in\C$ gilt:
\begin{gather}
\exp(x+y) = \exp(x)\exp(y),\\
\exp(x-y) = \frac{\exp(x)}{\exp(y)},\\
\exp(-x) = \frac{1}{\exp(x)}.
\end{gather}
\strong{Eulersche Formel.} Für alle $x\in\C$ gilt:
\begin{equation}
\ee^{\ui x} = \cos x+\ui\sin x.
\end{equation}

\subsection{Winkelfunktionen}\index{Winkelfunktion}
\begin{Definition}
\emdef{Kosinus}\index{Kosinus}\index{Cosinus}: $\C\to\C$,
\begin{equation}
\cos(x) := \sum_{k=0}^\infty \frac{x^{2k}}{(2k)!}
= 1+\frac{x^2}{2!}+\frac{x^4}{4!}+\ldots
\end{equation}
\emdef{Sinus}\index{Sinus}: $\C\to\C$,
\begin{equation}
\sin(x) := \sum_{k=0}^\infty \frac{x^{2k+1}}{(2k+1)!}
= x+\frac{x^3}{3!}+\frac{x^5}{5!}+\ldots
\end{equation}
\emdef{Tangens}\index{Tangens}: $\C\setminus\{k\pi+\pi/2\mid k\in\Z\}\to\C$,
\begin{equation}
\tan(x) := \frac{\sin(x)}{\cos(x)}.
\end{equation}
\emdef{Kotangens}\index{Kotangens}: $\C\setminus\{k\pi\mid k\in\Z\}\to\C$,
\begin{equation}
\cot(x) := \frac{\cos(x)}{\sin(x)}.
\end{equation}
\emdef{Sekans}\index{Sekans}: $\C\setminus\{k\pi+\pi/2\mid k\in\Z\}\to\C$,
\begin{equation}
\sec(x) := \frac{1}{\cos(x)}.
\end{equation}
\emdef{Kosekans}\index{Kosekans}: $\C\setminus\{k\pi\mid k\in\Z\}\to\C$,
\begin{equation}
\csc(x) := \frac{1}{\sin(x)}.
\end{equation}
\end{Definition}
\noindent
Darstellung durch die Exponentialfunktion:\\
Für alle $x\in\C$ gilt:
\begin{align}
\cos x &= \Real(\ee^{\ui x}) = \frac{\ee^{\ui x}+\ee^{-\ui x}}{2},\\
\sin x &= \Imag(\ee^{\ui x}) = \frac{\ee^{\ui x}-\ee^{-\ui x}}{2\ui}.
\end{align}

\subsubsection{Symmetrie und Periodizität}
Für alle $x\in\C$ gilt:
\begin{align}
\sin(-x) &= -\sin x,\enspace(\text{Punktsymmetrie})\\
\cos(-x) &= \cos x,\quad\;(\text{Achsensymmetrie})\\
\sin(x+2\pi) &= \sin x,\\
\cos(x+2\pi) &= \cos x,\\
\sin(x+\pi)  &=-\sin x,\\
\cos(x+\pi)  &=-\cos x,\\
\sin\Big(x+\frac{\pi}{2}\Big) &= \cos x = -\sin\Big(x-\frac{\pi}{2}\Big),\\
\cos\Big(x+\frac{\pi}{2}\Big) &= -\sin x = -\cos\Big(x-\frac{\pi}{2}\Big).
\end{align}

\subsubsection{Additionstheoreme}
\index{Additionstheoreme}

Für alle $x,y\in\C$ gilt:
\begin{align}
\sin(x+y) &= \sin x\cos y+\cos x\sin y,\\
\sin(x-y) &= \sin x\cos y-\cos x\sin y,\\
\cos(x+y) &= \cos x\cos y-\sin x\sin y,\\
\cos(x-y) &= \cos x\cos y+\sin x\sin y.
\end{align}

\subsubsection{Trigonometrischer Pythagoras}
Für alle $x\in\C$ gilt:
\begin{equation}
\sin^2 x+\cos^2 x=1.
\end{equation}

\subsubsection{Produkte}
Für alle $x,y\in\C$ gilt:
\begin{align}
2\sin x\sin y &= \cos(x-y)-\cos(x+y),\\
2\cos x\cos y &= \cos(x-y)+\cos(x+y),\\
2\sin x\cos y &= \sin(x-y)+\sin(x+y).
\end{align}

\subsubsection{Summen und Differenzen}
Für alle $x,y\in\C$ gilt:
\begin{align}
\sin x+\sin y &= 2\sin\frac{x+y}{2}\cos\frac{x-y}{2},\\
\sin x-\sin y &= 2\cos\frac{x+y}{2}\sin\frac{x-y}{2},\\
\cos x+\cos y &= 2\cos\frac{x+y}{2}\cos\frac{x-y}{2},\\
\cos x-\cos y &= 2\sin\frac{x+y}{2}\sin\frac{y-x}{2}.
\end{align}

\subsubsection{Winkelvielfache}
Für alle $x\in\C$ gilt:
\begin{align}
\sin(2x) &= 2\sin x\cos x,\\
\cos(2x) &= \cos^2 x-\sin^2 x,\\
\sin(3x) &= 3\sin x-4\sin^3 x,\\
\cos(3x) &= 4\cos^3 x-3\cos x.
\end{align}


\chapter{Analysis}
\section{Folgen}
\subsection{Konvergenz}

\begin{Definition}[open-ep-ball: offene Epsilon-Umgebung]%
\index{Epsilon-Umgebung}\index{offene Epsilon-Umgebung}
Sei $(M,d)$ ein metrischer Raum. Unter der offenen Epsilon-Umgebung
von $a\in M$ versteht man:%
\[U_\varepsilon(a) := \{x\mid d(x,a)<\varepsilon\}.\]
\end{Definition}

\noindent
Man setze zunächst speziell $d(x,a):=|x-a|$ bzw. $d(x,a):=\|x-a\|$.

\begin{Definition}[lim: konvergente Folge, Grenzwert]%
\label{def:lim}\index{konvergente Folge}\index{Grenzwert}
Eine Folge $(a_n)$ heißt konvergent gegen einen Grenzwert $a$, wenn es
zu jedem noch so kleinen $\varepsilon$ einen Index $n_0$ gibt, so dass
ab diesem Index sämtliche ihrer Werte in der Umgebung
$U_\varepsilon(a)$ liegen. Formal:
\[\lim_{n\to\infty} a_n = a
\defiff \forall\varepsilon{>}0\colon\exists n_0\colon\forall n{\ge}n_0\colon a_n\in U_\varepsilon(a)\]
bzw.
\[\lim_{n\to\infty} a_n = a
\defiff \forall\varepsilon{>}0\colon\exists n_0\colon\forall n{\ge}n_0\colon \|a_n-a\|<\varepsilon.\]
\end{Definition}

\begin{Definition}[bseq: beschränkte Folge]%
\label{def:bseq}\index{beschreankte Folge@beschränkte Folge}
Eine Folge $(a_n)$ mit $a_n\in\R$ heißt genau dann beschränkt,
wenn es eine reelle Zahl $S$ gibt mit $|a_n|<S$ für alle $n$.

Eine Folge $(a_n)$ von Punkten eines normierten Raums heißt genau
dann beschränkt, wenn es eine reelle Zahl $S$ gibt mit $\|a_n\|<S$
für alle $n$.
\end{Definition}

\begin{Satz}[Grenzwert bei Konvergenz eindeutig bestimmt]\newlinefirst
Eine konvergente Folge von Elementen eines metrischen Raumes
besitzt genau einen Grenzwert.
\end{Satz}

\begin{Beweis}
Sei $(a_n)$ eine konvergente Folge mit $a_n\to g_1$. Sei weiterhin
$g_1\ne g_2$. Es wird nun gezeigt, dass $g_2$ kein Grenzwert von $a_n$
sein kann. Wir müssen also zeigen:
\[\neg\lim_{n\to\infty} a_n=g_2 \iff
\exists\varepsilon{>}0\colon\forall n_0\colon\exists n{\ge}n_0\colon
a_n\notin U_\varepsilon(g_2)\]
mit $a_n\notin U_\varepsilon(g_2)\iff d(a_n,g_2)\ge\varepsilon$.

Um dem Existenzquantor zu genügen, wählt man nun
$\varepsilon = \frac{1}{2}d(g_1,g_2)$.
Nach Def. \ref{metric-space} (metric-space) gilt 
$d(g_1,g_2)>0$, daher ist auch $\varepsilon>0$. Nach Satz
\ref{construction-disjoint-ep-balls} sind die Umgebungen
$U_\varepsilon(g_1)$ und $U_\varepsilon(g_2)$ disjunkt.
Wegen $a_n\to g_1$ gibt es ein $n_0$ mit $a_n\in U_\varepsilon(g_1)$ für alle
$n\ge n_0$. Dann gibt es für jedes beliebig große $n_0$ aber auch
$n\ge n_0$ mit $a_n\notin U_\varepsilon(g_2)$.\,\qedsymbol
\end{Beweis}

\begin{Satz}[lim-scaled-ep: skaliertes Epsilon]\label{lim-scaled-ep}
Es gilt:
\[\lim_{n\to\infty} a_n=a \iff
\forall\varepsilon{>}0\colon\exists n_0\colon\forall n{\ge}n_0\colon \|a_n-a\|<R\varepsilon,\]
wobei $R>0$ ein fester aber beliebieger Skalierungsfaktor ist.
\end{Satz}

\begin{Beweis}
Betrachte $\varepsilon>0$ und multipliziere auf beiden Seiten
mit $R$. Dabei handelt es sich um eine Äquivalenzumformung.
Setze $\varepsilon':=R\varepsilon$. Demnach gilt:
\[\varepsilon>0 \iff \varepsilon'>0.\]
Nach der Ersetzungsregel düfen wir die Teilformel $\varepsilon>0$
nun ersetzen. Es ergibt sich die äquivalente Formel
\[\lim_{n\to\infty} a_n=a \iff
\forall\varepsilon'{>}0\colon\exists n_0\colon\forall n{\ge}n_0\colon
\|a_n-a\|<\varepsilon'.\]
Das ist aber genau Def. \ref{def:lim} (lim).\,\qedsymbol
\end{Beweis}

\begin{Satz}
Es gilt:
\[\lim_{n\to\infty} a_n = a\implies \lim_{n\to\infty} \|a_n\| = \|a\|.\]
\end{Satz}

\begin{Beweis}
Nach Satz \ref{rev-tineq} (umgekehrte Dreiecksungleichung) gilt:
\[|\|a_n\|-\|a\|| \le \|a_n-a\| < \varepsilon.\]
Dann ist aber erst recht $|\|a_n\|-\|a\||<\varepsilon$.\,\qedsymbol
\end{Beweis}

\begin{Satz}\label{zero-seq-bounded}
Ist $(a_n)$ eine Nullfolge und $(b_n)$ eine beschränkte Folge,
dann ist auch $(a_n b_n)$ eine Nullfolge.
\end{Satz}

\begin{Beweis}
Wenn $(b_n)$ beschränkt ist, dann existiert nach
Def. \ref{def:bseq} (bseq) eine Schranke $S$ mit
$|b_n|<S$ für alle $n$. Man multipliziert nun auf beiden Seiten
mit $|a_n|$ und erhält
\[|a_n b_n| = |a_n| |b_n| < |a_n| S.\]
Wenn $a_n\to 0$, dann muss für jedes $\varepsilon$
ein $n_0$ existieren mit $|a_n|<\varepsilon$ für $n\ge n_0$.
Multipliziert man auf beiden Seiten mit $S$, und ergibt sich
\[|a_n b_n-0| = |a_n b_n| < |a_n| S < S\varepsilon.\]
Nach Satz \ref{lim-scaled-ep} (lim-scaled-ep) gilt dann
aber $a_n b_n\to 0$.\,\qedsymbol
\end{Beweis}

\begin{Satz}
Sind $(a_n)$ und $(b_n)$ Nullfolgen,
dann ist auch $(a_n b_n)$ eine Nullfolge.
\end{Satz}

\begin{Beweis}[Beweis 1]
Wenn $(b_n)$ eine Nullfolge ist, dann ist $(b_n)$ auch beschränkt.
Nach Satz \ref{zero-seq-bounded} gilt dann die Behauptung.
\end{Beweis}

\begin{Beweis}[Beweis 2]
Sei $\varepsilon>0$ beliebig.
Es gibt ein $n_0$, so dass
$|a_n|<\varepsilon$ und $|b_n|<\varepsilon$ für $n\ge n_0$.
Demnach ist
\[|a_n b_n| = |a_n| |b_n|< |a_n|\varepsilon <\varepsilon^2.\]
Wegen $\varepsilon>0\iff\varepsilon'>0$ mit
$\varepsilon'=\varepsilon^2$ gilt
\[\forall\varepsilon'{>}0\colon\exists n_0\colon\forall n{\ge}n_0\colon
|a_n b_n|<\varepsilon'.\]
Nach Def. \ref{def:lim} (lim) gilt somit die Behauptung.\,\qedsymbol
\end{Beweis}

\newpage
\begin{Satz}[Grenzwertsatz zur Addition]%
\label{lim-add}\index{Grenzwertsaetze@Grenzwertsätze}
Seien $(a_n)$, $(b_n)$ Folgen von Vektoren eines normierten Raumes.
Es gilt:
\[\lim_{n\to\infty} a_n = a\land \lim_{n\to\infty} b_n
= b \implies \lim_{n\to\infty} a_n+b_n = a+b.\]
\end{Satz}

\begin{Beweis}
Dann gibt es ein $n_0$, so dass für $n\ge n_0$ sowohl
$\|a_n-a\|<\varepsilon$ als auch $\|b_n-b\|<\varepsilon$.
Addition der beiden Ungleichungen führt zu
\[\|a_n-a\| + \|b_n-b\| < 2\varepsilon.\]
Laut der Dreiecksungleichung, das ist Axiom (N3) in Def.
\ref{def:normed-space} (normed-space), gilt nun aber die Abschätzung
\[\|(a_n+b_n)-(a+b)\| = \|(a_n-a)+(b_n-b)\| \le \|a_n-a\|+\|b_n-b\|.\]
Somit gilt erst recht
\[\|(a_n+b_n)-(a+b)\| < 2\varepsilon.\]
Nach Satz \ref{lim-scaled-ep} (lim-scaled-ep)
folgt die Behauptung.\,\qedsymbol
\end{Beweis}

\begin{Satz}[Grenzwertsatz zur Skalarmultiplikation]\label{lim-smult}
Sei $(a_n)$ eine Folge von Vektoren eines normierten Raumes
und sei $r\in\R$ oder $r\in\C$. Es gilt:
\[\lim_{n\to\infty} a_n = a\implies \lim_{n\to\infty} ra_n\to ra.\]
\end{Satz}

\begin{Beweis}
Sei $\varepsilon>0$ fest aber beliebig. Es gibt nun ein $n_0$, so
dass $\|a_n-a\|<\varepsilon$ für $n\ge n_0$.
Multipliziert man auf beiden Seiten
mit $|r|$ und zieht Def. \ref{def:normed-space} (normed-space)
Axiom (N2) heran, dann ergibt sich
\[\|ra_n-ra\| = |r|\,\|a_n-a\|<|r|\varepsilon.\]
Nach Satz \ref{lim-scaled-ep} (lim-scaled-ep)
folgt die Behauptung.\,\qedsymbol
\end{Beweis}

\begin{Satz}[Grenzwertsatz zum Produkt]\newlinefirst
Seien $(a_n)$ und $(b_n)$ Folgen
reeller Zahlen. Es gilt:
\[\lim_{n\to\infty} a_n=a\land\lim_{n\to\infty} b_n=b\implies
\lim_{n\to\infty} a_n b_n = ab.\]
\end{Satz}

\begin{Beweis}
Nach Voraussetzung sind $a_n-a$ und $b_n-b$ Nullfolgen.
Da das Produkt von Nullfolgen wieder eine Nullfolge ist, gilt
\[(a_n-a)(b_n-b) = a_n b_n-a_n b-ab_n+ab\to 0.\]
Da nach Satz \ref{lim-smult} aber $a_n b\to ab$ und $ab_n\to ab$,
ergibt sich nach Satz \ref{lim-add} nun
\[(a_n-a)(b_n-b)+a_n b+ab_n = a_n b_n+ab\to 2ab.\]
Addiert man nun noch die konstante Folge $-2ab$
und wendet nochmals Satz \ref{lim-add} an, dann ergibt sich
die Behauptung
\[a_n b_n\to ab.\,\qedsymbol\]
\end{Beweis}

\newpage
\begin{Satz}\label{cont-seqcont}%
\index{folgenstetig}\index{stetig!folgenstetig}
Sei $M$ ein metrischer Raum und $X$ ein topologischer Raum.
Eine Abbildung $f\colon M\to X$ ist genau dann stetig, wenn
sie folgenstetig ist.
\end{Satz}

\begin{Satz}[Satz zur Fixpunktgleichung]\index{Fixpunktgleichung}
Sei $M$ ein metrischer Raum und sei $f\colon M\to M$.
Sei $x_{n+1}:=f(x_n)$ eine Fixpunktiteration. Wenn die Folge
$(x_n)$ zu einem Startwert $x_0$ gegen ein $x\in M$ konvergiert, und
wenn $f$ eine stetige Abbildung ist, dann muss der Grenzwert $x$ die
Fixpunktgleichung $x=f(x)$ erfüllen.
\end{Satz}

\begin{Beweis}
Wenn $x_n\to x$, dann gilt trivialerweise auch $x_{n+1}\to x$.
Weil $f$ stetig ist, ist $f$ nach Satz \ref{cont-seqcont}
auch folgenstetig. Daher gilt $\lim f(a_n) = f(\lim a_n)$ für jede
konvergente Folge $(a_n)$. Somit gilt:
\[x=\lim_{n\to\infty} x_{n+1} = \lim_{n\to\infty} f(x_n)
= f(\lim_{n\to\infty} x_n) = f(x).\;\qedsymbol\]
\end{Beweis}

\subsection{Wachstum und Landau-Symbole}
\begin{Definition}\label{Landau-O}
Seien $f,g\colon D\to\R$ mit $D=\N$ oder $D=\R$. Man sagt, die
Funktion $f$ wächst nicht wesentlich schneller als $g$, kurz
$f\in\mathcal O(g)$, genau dann, wenn
\[\exists c{>}0\colon\,\exists x_0\colon\,\forall x{>}x_0\colon\, |f(x)|\le c|g(x)|.\]
\end{Definition}

\begin{Korollar}
Ist $r\in\R$ mit $r\ne 0$ eine Konstante, dann gilt
$\mathcal O(rg)=\mathcal O(g)$.
\end{Korollar}
\begin{Beweis}
Nach Def. \ref{Landau-O} ist
\[f\in\mathcal O(rg) \iff 
\exists c{>}0\colon\,\exists x_0\colon\,\forall x{>}x_0\colon\,|f(x)|\le c|rg(x)|.\]
Man hat nun
\[|f(x)|\le c|rg(x)| = c\cdot |r|\cdot |g(x)|.\]
Wegen $r\ne 0$ ist $|r|>0$ und daher auch $c>0\iff c|r|>0$. Sei
$c':=r|c|$. Also gilt $c>0\iff c'>0$. Nach der Ersetzungsregel
darf $c>0$ gegen $c'>0$ ersetzt werden und man erhält die
äquivalente Bedingung
\[\exists c'{>}0\colon\,\exists x_0\colon\,
\forall x{>}x_0\colon\,|f(x)|\le c'|g(x)|.\]
Nach Def. \ref{Landau-O} ist das gerade $f\in\mathcal O(g)$.\;\qedsymbol
\end{Beweis}

\begin{Korollar}
Sind $f_1,f_2\in\mathcal O(g)$, ist auch $f_1+f_2\in\mathcal O(g)$.
\end{Korollar}
\begin{Beweis}
Als Prämissen liegen Zeugen $c'>0,x_0'$ und $c''>0,x_0''$ für
\begin{gather*}
\forall x>x_0'\colon |f_1(x)|\le c'|g(x)|,\\
\forall x>x_0''\colon |f_2(x)|\le c''|g(x)|
\end{gather*}
vor. Mit der Dreiecksungleichung findet sich
\[|f_1(x)+f_2(x)|\le |f_1(x)|+|f_2(x)| \le c'|g(x)|+c''|g(x)| = (c'+c'')|g(x)|\]
für $x>\max(x_0',x_0'')$. Ergo sind $x_0:=\max(x_0',x_0'')$ und $c:=c'+c''$
Zeugen für
\[\exists c>0\colon\exists x_0\colon\forall x>x_0\colon |f_1(x)+f_2(x)|\le c|g(x)|.\,\qedsymbol\]
\end{Beweis}

\newpage
\section{Stetige Funktionen}

\begin{Definition}[Grenzwert einer Funktion]\label{fn-lim}
Sei $f\colon D\to\R$ mit $D\subseteq\R$ und sei $p$ ein
Häufungspunkt von $D$. Die Funktion $f$ heißt konvergent
gegen $L$ für $x\to p$, wenn%
\[\forall \varepsilon{>}0\colon\,\exists \delta{>}0\colon\,\forall x{\in}D\colon\,
(0<|x-x_0|<\delta\implies |f(x)-L|<\varepsilon).\]
Bei Konvergenz schreibt man $L=\lim\limits_{x\to p} f(x)$ und nennt $L$ den Grenzwert.
\end{Definition}

\begin{Definition}[cont: stetig]\label{cont}
Eine Funktion $f\colon D\to\R$ mit $D\subseteq\R$ heißt stetig an der
Stelle $x_0\in D$, wenn
\[\forall \varepsilon{>}0\colon\,\exists \delta{>}0\colon\,\forall x{\in}D\colon\,
(|x-x_0|<\delta\implies |f(x)-f(x_0)|<\varepsilon).\]
\end{Definition}

\begin{Definition}[Lipschitz-stetig]\newlinefirst
Eine Funktion $f\colon D\to\R$ mit $D\subseteq\R$ heißt
Lipschitz"=stetig, wenn eine Konstante $L$ existiert, so dass
\[|f(b)-f(a)|\le L|b-a|\]
für alle $a,b\in D$.
\end{Definition}

\begin{Definition}[Lipschitz-stetig an einer Stelle]%
\label{Lipschitz-cont-at}\newlinefirst
Eine Funktion $f\colon D\to\R$ mit $D\subseteq\R$ heißt
Lipschitz"=stetig an der Stelle $x_0\in D$, wenn eine Konstante $L$
existiert, so dass
\[|f(x_0)-(a)|\le L|x_0-a|\]
für alle $a\in D$.
\end{Definition}

\begin{Korollar}
Eine Funktion ist genau dann Lipschitz"=stetig, wenn sie an jeder
Stelle Lipschitz"=stetig ist und die Menge der optimalen
Lipschitz"=Konstanten dabei beschränkt.
\end{Korollar}
\begin{Beweis}
Eine Lipschitz"=stetige Funktion ist trivialerweise an jeder Stelle
Lipschitz"=stetig. Ist $f\colon D\to\R$ an der Stelle $b$ Lipschitz"=stetig,
dann existiert eine Lipschitz"=Konstante $L_b$ mit%
\[\forall a\in D\colon |f(b)-f(a)|\le L_b |b-a|.\]
Nach Voraussetzung ist $L=\sup_{b\in D} L_b$ endlich. Alle $L_b$ können
nun zu $L$ abgeschwächt werden und es ergibt sich%
\[\forall b\in D\colon\forall a\in D\colon |f(b)-f(a)|\le L|b-a|.\;\qedsymbol\]
\end{Beweis}


\begin{Definition}[lokal Lipschitz-stetig]\newlinefirst
Eine Funktion $f\colon D\to\R$ mit $D\subseteq\R$ heißt lokal
Lipschitz"=stetig in der Nähe einer Stelle $x_0\in D$, wenn es eine
Epsilon"=Umgebung $U_\varepsilon(x_0)$ gibt, so dass die Einschränkung
von $f$ auf diese Umgebung Lipschitz"=stetig ist. Die Funktion heißt
lokal Lipschitz"=stetig, wenn sie in der Nähe jeder Stelle
Lipschitz"=stetig ist.
\end{Definition}

\begin{Satz}\label{diff-nh-Lipschitz-cont-at}
Ist die Funktion $f\colon D\to\R$ an der Stelle $x_0$ differenzierbar,
dann gibt es ein $\delta>0$, so dass die Einschränkung von $f$
auf $U_\delta(x_0)$ an der Stelle $x_0$ Lipschitz"=stetig ist.
\end{Satz}

\begin{Beweis}
Def. \ref{fn-lim} wird in Def. \ref{diff} (diff) eingesetzt.
Es ergibt sich:
\[0<|x-x_0|<\delta\implies
\left|\frac{f(x)-f(x_0)}{x-x_0}-f'(x_0)\right|<\varepsilon.\]
Nach der umgekehrten Dreiecksungleichung \ref{rev-tineq} gilt
\[\left|\frac{f(x)-f(x_0)}{x-x_0}\right|-|f'(x_0)| \le
\left|\frac{f(x)-f(x_0)}{x-x_0}-f'(x_0)\right|
< \varepsilon.\]
Daraus ergibt sich
\[|f(x)-f(x_0)| < (|f'(x_0)|+\varepsilon)\cdot |x-x_0|\]
und somit erst recht
\[|f(x)-f(x_0)| \le (|f'(x_0)|+\varepsilon)\cdot |x-x_0|,\]
wobei jetzt auch $x=x_0$ erlaubt ist. Demnach wird Def.
\ref{Lipschitz-cont-at} erfüllt:
\[\exists \delta{>}0\colon\,\forall x\in U_\delta(x_0)\colon\,
|f(x)-f(x_0)| \le (|f'(x_0)|+\varepsilon)\cdot |x-x_0|.\;\qedsymbol\]
\end{Beweis}

\begin{Satz}\label{diff-bounded-Lipschitz-cont}
Eine differenzierbare Funktion ist genau dann Lipschitz"=stetig,
wenn ihre Ableitung beschränkt ist.
\end{Satz}
\begin{Beweis}
Wenn $f\colon I\to\R$ Lipschitz"=stetig ist, dann gibt es $L$ mit
\[\left|\frac{f(b)-f(a)}{b-a}\right|\le L\]
für alle $a,b\in D$ mit $a\ne b$. Daraus folgt
\[|f'(a)| = \left|\lim_{b\to a} \frac{f(b)-f(a)}{b-a}\right|
= \lim_{b\to a} \left|\frac{f(b)-f(a)}{b-a}\right|
\le L.\]
Demnach ist die Ableitung beschränkt.

Sei nun umgekehrt die Ableitung beschränkt. Für $a,b\in I$ mit $a\ne b$
gibt es nach dem Mittelwertsatz ein $x_0\in(a,b)$, so dass
\[|f'(x_0)| = \left|\frac{f(b)-f(a)}{b-a}\right|.\]
Da die Ableitung beschränkt ist gibt es ein Supremum
$L = \sup_{x\in I} |f'(x)|$. Demnach ist $|f'(x)|\le L$ für alle $x$.
Es ergibt sich
\[\left|\frac{f(b)-f(a)}{b-a}\right|\le L|b-a| \implies |f(b)-f(a)|\le L|b-a|.\]
Nun darf auch $a=b$ gewählt werden.\;\qedsymbol
\end{Beweis}

\begin{Satz}\label{diff-compact-Lipschitz-cont}
Eine auf einem kompakten Intervall $[a,b]$ definierte stetig
differenzierbare Funktion ist Lipschitz"=stetig.
\end{Satz}
\begin{Beweis}
Sei $f\colon [a,b]\to\R$ stetig differenzierbar. Dann ist $f'(x)$ stetig.
Nach dem Satz vom Minimum und Maximum ist $|f'(x)|$ beschränkt. Nach
Satz \ref{diff-bounded-Lipschitz-cont} muss $f$ Lipschitz"=stetig
sein.\;\qedsymbol
\end{Beweis}

\begin{Korollar}
Eine stetig differenzierbare Funktion ist lokal Lipschitz"=stetig.
\end{Korollar}
\begin{Beweis}
Sei $f\colon D\to\R$ stetig differenzierbar. Sei $[a,b]\in D$. Sei
$x_0\in [a,b]$. Die Einschränkung von $f$ auf $[a,b]$ ist
Lipschitz"=stetig nach Satz \ref{diff-compact-Lipschitz-cont}.
Dann ist auch die Einschränkung von $f$ auf
$U_\varepsilon(x_0)\subseteq [a,b]$ Lipschitz"=stetig.\;\qedsymbol
\end{Beweis}

\begin{Satz}
Es gibt differenzierbare Funktionen, die nicht überall lokal
Lipschitz"=stetig sind.
\end{Satz}
\begin{Beweis}
Aus Satz \ref{diff-bounded-Lipschitz-cont} ergibt sich also
Kontraposition, dass eine Funktion mit unbeschränkter Ableitung
nicht Lipschitz"=stetig sein kann.

Ist $f\colon D\to\R$ an jeder Stelle differenzierbar und ist $f'$
in jeder noch so kleinen Umgebung der Stelle $x_0$ unbeschränkt, dann
kann $f$ also in der Nähe dieser Stelle auch nicht lokal
Lipschitz"=stetig sein.

Ein Beispiel für eine solche Funktion ist $f\colon{}[0,\infty)\to\R$
mit
\[f(0):=0\quad \text{und}\quad f(x):=x^{3/2}\cos\Big(\tfrac{1}{x}\Big).\]
Einerseits gilt
\[f'(0) = \lim_{h\to 0}\frac{f(0+h)-f(0)}{h} = \lim_{h\to 0}\frac{f(h)}{h}
= \lim_{h\to 0} (h^{1/2}\cos\Big(\tfrac{1}{h}\Big)) = 0.\]
Die Funktion ist also an der Stelle $x=0$ differenzierbar.
Andererseits gilt nach den Ableitungsregeln%
\[f'(x) = \frac{3}{2}\sqrt{x}\cos\Big(\tfrac{1}{x}\Big)+\frac{1}{\sqrt{x}}\sin\Big(\tfrac{1}{x}\Big).\]
für $x>0$. Der Term $\tfrac{1}{\sqrt{x}}$ erwirkt für $x\to 0$ immer
größere Maxima von $|f'(x)|$. Daher kann $f$ in der Nähe von $x=0$ nicht
lokal Lipschitz"=stetig sein.\;\qedsymbol
\end{Beweis}

\begin{Satz}
Sei $f\colon\R\to\R$ differenzierbar und $f(x)$ konvergent
für $x\to\infty$. Ist außerdem $f'$  Lipschitz-stetig,
zieht dies $f'(x)\to 0$ für $x\to\infty$ nach sich.
\end{Satz}
\begin{Beweis}
Gemäß dem cauchyschen Konvergenzkriterium gibt es zu jedem
$\varepsilon>0$ eine Stelle $x_0$, so dass
\begin{equation}
|f(b)-f(a)| < \varepsilon
\end{equation}
für alle $a,b$ mit $x_0 < a \le b$. Nun ist $f'$ aufgrund
der Lipschitz-Stetigkeit erst recht stetig, womit
\begin{equation}
\bigg|\int_a^b f'(x)\,\mathrm dx\bigg| = |f(b)-f(a)|
\end{equation}
laut dem Fundamentalsatz gilt. Gezeigt wird nun, dass $|f'(a)|$
beschränkt ist. Sei dazu $L$ die Lipschitz-Konstante. Ohne
Beschränkung der Allgemeinheit sei $f'(a)>0$. Fallen darf $f'$ maximal
mit dem Anstieg $-L$. Geschieht dies linear bis zur Nullstelle $b$,
ergibt sich ein rechtwinkliges Dreieck mit dem Flächeninhalt
\begin{equation}
\frac{1}{2L} f'(a)^2 = \int_a^b f'(x)\,\mathrm dx < \varepsilon.
\end{equation}
Demnach ist $f'(a) < \sqrt{2L\varepsilon}$. Weil dies für alle $a>x_0$
gilt, muss $f'$ jede Beschränkung unterbieten, womit
der Beweis der Behauptung erbracht ist.\;\qedsymbol
\end{Beweis}

\noindent
Die Diskussion Gegenbeispiels $f(0):=0$, $f(x):=\sin(x^2)/x$ macht
ersichtlich, dass die Aussage ohne Lipschitz-Stetigkeit nicht einmal
für glatte Funktionen gilt.

\newpage
\section{Differentialrechnung}

\subsection{Ableitungsregeln}

\begin{Definition}[diff: differenzierbar, Ableitung]%
\label{diff}\index{differenzierbar}\index{Ableitung}
Eine Funktion $f\colon D\to\R$ heißt differenzieraber an der Stelle
$x_0\in D$, wenn der Grenzwert%
\[f'(x_0) = \lim_{x\to x_0}\frac{f(x)-f(x_0)}{x-x_0}
= \lim_{h\to 0}\frac{f(x_0+h)-f(x_0)}{h}\]
existiert. Man nennt $f'(x_0)$ die Ableitung von $f$ an der Stelle
$x_0$.
\end{Definition}

\begin{Satz}\index{Produktregel}
Sei $I$ ein Intervall und $f,g\colon I\to\R$. Sind $f,g$
differenzierbar an der Stelle $x\in I$, dann ist auch%
\begin{align}
f+g&\;\text{dort differenzierbar mit}\;(f+g)'(x)=f'(x)+g'(x),\\
f-g&\;\text{dort differenzierbar mit}\;(f-g)'(x)=f'(x)-g'(x),\\
\label{eq:diff-mul}
fg&\;\text{dort differenzierbar mit}\;(fg)'(x)=f'(x)g(x)+f(x)g'(x).
\end{align}
\end{Satz}

\begin{Beweis} Es gilt
\begin{gather}
(f+g)'(x)
= \lim_{h\to 0}\frac{(f+g)(x+h)-(f+g)(x)}{h}\\
= \lim_{h\to 0}\frac{(f(x+h)+g(x+h))-(f(x)+g(x))}{h}\\
= \lim_{h\to 0}\bigg(\frac{f(x+h)-f(x)}{h}+\frac{g(x+h)-g(x)}{h}\bigg)\\
= \lim_{h\to 0}\frac{f(x+h)-f(x)}{h}+\lim_{h\to 0}\frac{g(x+h)-g(x)}{h}
= f'(x)+g'(x).
\end{gather}
Da die Grenzwerte auf der rechten Seite nach Voraussetzung existieren,
muss auch der Grenzwert der Summe existieren.
Die Rechnung für die Subtraktion ist analog.

Bei der Multiplikation wird ein Nullsummentrick angewendet:
\begin{gather}
g(x)f'(x)+f(x)g'(x)
= g(x)\lim_{h\to 0}\frac{f(x+h)-f(x)}{h}
+ f(x)\lim_{h\to 0}\frac{g(x+h)-g(x)}{h}\\
= \lim_{h\to 0}\bigg[g(x+h)\frac{f(x+h)-f(x)}{h}\bigg]
+ \lim_{h\to 0}\bigg[f(x)\frac{g(x+h)-g(x)}{h}\bigg]\\
= \lim_{h\to 0}\frac{f(x+h)g(x+h)-f(x)g(x+h)}{h}
+ \lim_{h\to 0}\frac{f(x)g(x+h)-f(x)g(x)}{h}\\
= \lim_{h\to 0}\frac{f(x+h)g(x+h)-f(x)g(x+h)+f(x)g(x+h)-f(x)g(x)}{h}\\
= \lim_{h\to 0}\frac{f(x+h)g(x+h)-f(x)g(x)}{h}
= \lim_{h\to 0}\frac{(fg)(x+h)-(fg)(x)}{h}
= (fg)'(x).
\end{gather}
Hierbei wurde $\lim_{h\to 0}g(x+h)=g(x)$ benutzt, was richtig ist,
weil $g$ an der Stelle $x$ differenzierbar ist und dort somit ganz
sicher stetig.\;\qedsymbol
\end{Beweis}

\newpage
\begin{Satz}
Sei $I$ ein Intervall. Sind $f,g\colon I\to\R$ an der Stelle
$x$ differenzierbar und ist $g(x)\ne 0$, dann
ist auch $f/g$ differenzierbar und es gilt
\begin{equation}
\bigg(\frac{f}{g}\bigg)'(x) = \frac{f'(x)g(x)-f(x)g'(x)}{g(x)^2}.
\end{equation}
\end{Satz}
\begin{Beweis}
Nach der Produktregel \eqref{eq:diff-mul} gilt
\begin{equation}
0 = 1' = \bigg(g\cdot\frac{1}{g}\bigg)'
= g'\cdot\frac{1}{g}+g\cdot \bigg(\frac{1}{g}\bigg)'.
\end{equation}
Umstellen bringt $(1/g)'(x)=-g'(x)/g(x)^2$. Nochmalige Anwendung der
Produktregel \eqref{eq:diff-mul} bringt
\begin{align}
\bigg(\frac{f}{g}\bigg)'(x)
&= \bigg(f\cdot\frac{1}{g}\bigg)'(x)
= f'(x)\cdot\frac{1}{g(x)}+f(x)\bigg(\frac{1}{g}\bigg)'(x)\\
&= \frac{f'(x)}{g(x)}-\frac{f(x)g'(x)}{g(x)^2}
= \frac{f'(x)g(x)-f(x)g'(x)}{g(x)^2}.\;\qedsymbol
\end{align}
\end{Beweis}

\begin{Satz}\label{diff-power}
Für $f\colon\R\to\R$, $f(x):=x^n$ mit $n\in\N$ gilt
$f'(x)=nx^{n-1}$.
\end{Satz}
\begin{Beweis}[Beweis 1]
Heranziehung des binomischen Lehrsatzes bringt
\begin{align}
f'(x) &= \lim_{h\to 0}\frac{(x+h)^n-x^n}{h}
= \lim_{h\to 0}\frac{\sum_{k=0}^n\binom{n}{k}x^{n-k} h^k-x^n}{h}\\
&= \lim_{h\to 0}\bigg(nx^{n-1}+\sum_{k=2}^n\binom{n}{k}x^{n-k}h^{k-1}\bigg)
= nx^{n-1}.\;\qedsymbol
\end{align}
\end{Beweis}
\begin{Beweis}[Beweis 2]
Induktiv. Der Induktionsanfang $\tfrac{\mathrm d}{\mathrm dx}x=1$ ist klar.
Induktionsschritt mittels Produktregel \eqref{eq:diff-mul}:
\begin{align}
\tfrac{\mathrm d}{\mathrm dx} x^n = \tfrac{\mathrm d}{\mathrm dx} (x\cdot x^{n-1})
= x^{n-1}+x\tfrac{\mathrm d}{\mathrm dx}x^{n-1}
= x^{n-1}+(n-1)x^{n-1} = nx^{n-1}.\;\qedsymbol
\end{align}
\end{Beweis}

\begin{Satz}
Für $f\colon\R{\setminus}\{0\}\to\R$, $f(x):=x^n$ mit $n\in\Z$
gilt $f'(x)=nx^{n-1}$.
\end{Satz}
\begin{Beweis}
Der Fall $n=0$ ist trivial und $n\ge 1$ wurde schon in Satz
\ref{diff-power} gezeigt. Sei nun $a\in\N$ und $n=-a$. Nach der
Produktregel \eqref{eq:diff-mul} und Satz \ref{diff-power} gilt
\begin{equation}
0 = \tfrac{\mathrm d}{\mathrm dx} 1
= \tfrac{\mathrm d}{\mathrm dx} (x^a x^{-a})
= x^{-a}\tfrac{\mathrm d}{\mathrm dx} x^a+x^a\tfrac{\mathrm d}{\mathrm dx} x^{-a}
= x^{-a}ax^{a-1}+x^a\tfrac{\mathrm d}{\mathrm dx} x^{-a}.
\end{equation}
Dividiert man nun durch $x^a$ und formt um, dann ergibt sich
\begin{equation}
\tfrac{\mathrm d}{\mathrm dx} x^{-a} = -ax^{-a-1}
\implies \tfrac{\mathrm d}{\mathrm dx} x^n = nx^{n-1}.\;\qedsymbol
\end{equation}
\end{Beweis}

\newpage
\subsection{Glatte Funktionen}

\begin{Satz}
Sei $f\colon\R\to\R$ eine Funktion mit der Eigenschaft
$f(x)=0$ für $x\le 0$ und $f(x)>0$ für $x>0$. Es gibt glatte Funktionen
mit dieser Eigenschaft, jedoch keine analytischen.
\end{Satz}

\begin{Beweis}
Wegen $f(x)=0$ für $x\le 0$ muss die linksseitige $n$-te Ableitung
an der Stelle $x=0$ immer verschwinden. Wenn die $n$-te Ableitung
stetig sein soll, muss auch die rechtsseitige Ableitung bei $x=0$
verschwinden. Da die Funktion glatt sein soll, muss das für jede
Ableitung gelten. Daher verschwindet die Taylorreihe an der Stelle
$x=0$. Da aber $f(x)>0$ für $x>0$, gibt es keine noch so kleine
Umgebung mit Übereinstimmung von $f$ und ihrer Taylorreihe.
Daher kann $f$ an der Stelle $x=0$ nicht analytisch sein.

Eine glatte Funktion lässt sich jedoch konstruieren:
\[f(x):=\begin{cases}
\ee^{-1/x}&\text{wenn}\;x>0,\\
0&\text{wenn}\;x\le 0.
\end{cases}\]
Ist nämlich $g(x)$ an einer Stelle glatt, dann ist
es nach Kettenregel, Produktregel und Summenregel auch $\ee^{g(x)}$.
Die $n$-te Ableitung lässt sich immer in der Form%
\[\sum\nolimits_k e^{g(x)}{r_k(x)}
= e^{g(x)}\sum\nolimits_k r_k(x) = e^{g(x)}r(x)\]
darstellen, wobei die $r_k(x)$ bzw. $r(x)$ in diesem Fall rationale
Funktionen mit Polstelle bei $x=0$ sind. Da aber $e^{-1/x}$ für
$x\to 0$ schneller fällt als jede rationale Funktion steigen kann,
muss die rechtsseitige Ableitung an der Stelle $x=0$ immer
verschwinden.\;\qedsymbol
\end{Beweis}

\subsection{Richtungsableitung}

\begin{Definition}[Richtungsableitung]
Sei $U\subseteq\R^n$ offen, $x\in U$ eine Stelle und $v\in\R^n$
ein Vektor. Man betrachte für ein kleines $\varepsilon>0$
die Parametergerade
\[\gamma\colon(-\varepsilon,\varepsilon)\to U,\quad \gamma(t):=x+tv.\]
Für eine Funktion $f\colon U\to\R$ ist die Zahl
\[D_v f(x) := (f\circ\gamma)'(0) = \lim_{h\to 0}\frac{f(x+hv)-f(x)}{h},\]
falls sie existiert, die Richtungsableitung von $f$ an der Stelle $x$ in
Richtung $v$.
\end{Definition}
\begin{Korollar}
Die Funktionen $f,g$ seien an der Stelle $x$ in Richtung $v$
differenzierbar. Sei $c$ eine reelle Zahl. Dann sind auch
$f+g$, $f-g$, $cf$, $fg$ differenzierbar und es gelten die
den üblichen Ableitungsregeln analogen Regeln
\begin{align*}
D_v(f+g)(x) &= D_v f(x)+D_v g(x),\\
D_v(f-g)(x) &= D_v f(x)+D_v g(x),\\
D_v(cf)(x) &= cD_v f(x),\\
D_v(fg)(x) &= g(x)D_v f(x) + f(x)D_v g(x).
\end{align*}
\end{Korollar}
\begin{Beweis}
Die Ableitungsregeln werden über die Definition
auf die Ableitungsregeln für gewöhnliche reelle Funktionen
zurückgeführt. So ist
\begin{align*}
D_v(f+g)(x) &= ((f+g)\circ\gamma)'(0)
= ((f\circ\gamma)+(g\circ\gamma))'(0)\\
&= (f\circ\gamma)'(0)+(g\circ\gamma)'(0)
= D_v f(x) + D_v g(x).
\end{align*}
Der Beweis der restlichen Regeln ist analog.\,\qedsymbol
\end{Beweis}

\begin{Korollar}[Kettenregel]\newlinefirst
Sei $g\colon\R\to\R$ differenzierbar und
$f$ differenzierbar an der Stelle $x$ in Richtung $v$. Dann ist
auch $g\circ f$ entsprechend differenzierbar, und es gilt
\[D_v(g\circ f)(x) = (g'\circ f)(x)\cdot D_v f(x).\]
\end{Korollar}
\begin{Beweis}
Die Regel ist gemäß der Definition auf die gewöhnliche Kettenregel
zurückführbar. Man bekommt
\begin{align*}
D_v(g\circ f)(x) = (g\circ f\circ\gamma)'(0)
= g'(f(\gamma(0)))\cdot (f\circ\gamma)'(0)
= g'(f(x))\cdot D_v f(x).\;\qedsymbol
\end{align*}
\end{Beweis}

\begin{Definition}[Partielle Ableitung]\newlinefirst
Sei $(\mathbf e_1,\ldots,\mathbf e_n)$ die Standardbasis. 
Die partielle Ableitung $\partial_k f(x)$ ist definiert als
die Richtungsableitung $D_v f(x)$ bezüglich $v=\mathbf e_k$.
\end{Definition}

\begin{Korollar}
Zur jeder gewöhnlichen Ableitungsregel besitzt die
Richtungsableitung eine analoge Regel.
\end{Korollar}
\strong{Vorbereitung.}
Sei $f=(f_1,\ldots,f_n)$ ein Tupel von Funktionen aus einem
Funktionenraum und sei entsprechend
$f(x):=(f_1(x),\ldots, f_n(x))$. Sei $p$ eine beliebige mehrstellige
Operation. Sei $\eta_p(f)(x)$ die punktweise Anwendung von $p$.
Ein Beispiel ist die Addition $p(y_1,y_2):=y_1+y_2$. Dann ist
$\eta_p(f_1,f_2)(x)=f_1(x)+f_2(x)$. Sei%
\[F(T)(f) := (T f_1,\ldots ,T f_n)\]
die komponentenweise Anwendung eines Operators $T$.
Sei $C_\gamma$ der durch $C_\gamma f := f\circ\gamma$ definierte
Kompositionsoperator. Allgemein gilt%
\[C_\gamma\circ\eta_p = \eta_p\circ F(C_\gamma).\]

\begin{Beweis}
Prämisse ist, dass der gewöhnliche Ableitungsoperator $D$ die Regel
\[D(\eta_p(f))(x) = (D\circ\eta_p)(f)(x) = R(f(x),F(D)(f)(x))\]
erfüllt. Für die Richtungsableitung von $\eta_p(f)$ gilt dann
\begin{gather*}
D_v(\eta_p(f))(x) = (\eta_p(f)\circ\gamma)'(0)
= (D\circ C_\gamma\circ\eta_p)(f)(0)
= (D\circ\eta_p\circ F(C_\gamma))(f)(0)\\
= (D\circ\eta_p)(F(C_\gamma)(f))(0)
= R(F(C_\gamma)(f)(0),F(D)(F(C_\gamma)(f))(0))\\
= R(f(x),F(D\circ C_\gamma)(f)(0))
= R(f(x),F(D_v)(f)(x)).\;\qedsymbol
\end{gather*}
\end{Beweis}

\noindent
Beispiele sind
\begin{gather*}
\begin{array}{ll}
p(y_1,y_2) = y_1+y_2, & R((y_1,y_2),(y_1',y_2')) = y_1'+y_2',\\[4pt]
p(y_1,y_2) = y_1 y_2, & R((y_1,y_2),(y_1',y_2')) = y_1'y_2 + y_1y_2',\\[4pt]
p(y) = cy, & R(y,y') = cy',\\[4pt]
p(y) = g(y), & R(y,y') = g'(y)y'.
\end{array}
\end{gather*}

\newpage
\section{Fixpunkt-Iterationen}%
\index{Fixpunkt-Iteration}

\begin{Definition}[Kontraktion]\index{Kontraktion}
Sei $(M,d)$ ein vollständiger metrischer Raum. Eine Abbildung
$\varphi\colon M\to M$ heißt Kontraktion, wenn sie
Lipschitz"=stetig mit Lipschitz"=Konstante $L<1$ ist, d.\,h.
\[d(\varphi(x),\varphi(y))<L\,d(x,y)\]
für alle $x,y\in M$.
\end{Definition}

\begin{Satz}[Fixpunktsatz von Banach]\label{Banach-fixed-point-theorem}%
\index{Fixpunktsatz von Banach}\index{Banach!Fixpunktsatz von}
Sei $(M,d)$ ein nichtleerer vollständiger metrischer Raum
und sei $\varphi\colon M\to M$ eine Kontraktion. Es gibt genau
einen Fixpunkt $x\in M$ mit $x=\varphi(x)$ und die Folge
$(x_n)\colon\N\to M$ mit $x_{n+1}=\varphi(x_n)$ konvergiert
gegen den Fixpunkt, unabhängig vom Startwert $x_0$.
\end{Satz}

\begin{Satz}[Hinreichendes Konvergenzkriterium]\label{diff-fixed-point-iter}
Sei $M=[a,b]$. Ist $\varphi\colon M\to M$ differenzierbar und gibt es
eine Zahl $r$ mit $|\varphi'(x)|<r<1$ für alle $x\in M$, dann
hat $\varphi$ genau einen Fixpunkt und die Folge $(x_n)$ mit $x_{n+1}=\varphi(x_n)$
konvergiert für jeden Startwert $x_0\in M$ gegen diesen Fixpunkt.
\end{Satz}
\begin{Beweis}
Nach Satz \ref{diff-bounded-Lipschitz-cont} ist eine differenzierbare
Funktion $\varphi$ mit beschränkter Ableitung auch Lipschitz"=stetig,
und $L=\sup_{x\in M}|\varphi'(x)|$ eine Lipschitz"=Konstante.
Wegen $|\varphi'(x)|<r$ muss $L\le r$ sein, und somit $L<1$.
D.\,h., $\varphi$ ist eine Kontraktion. Die Konvergenz der Folge
$(x_n)$ ist gemäß Satz \ref{Banach-fixed-point-theorem}
gewährleistet.\;\qedsymbol
\end{Beweis}

\begin{Satz}[Hinreichendes Konvergenzkriterium zum Newton-Verfahren]%
\index{Newton-Verfahren}\newlinefirst
Sei $f\colon [a,b]\to\R$ zweimal stetig differenzierbar und
$f'(x)\ne 0$ für alle $x$. Sei%
\[\varphi\colon [a,b]\to [a,b],\quad \varphi(x):=x-\frac{f(x)}{f'(x)}.\]
Man beachte $\varphi([a,b])\subseteq [a,b]$. Gilt für alle $x$ die Ungleichung%
\[|\varphi'(x)| = \bigg|\frac{f(x)f''(x)}{f'(x)^2}\bigg| < 1,\]
dann besitzt $f$ genau eine Nullstelle und die Folge $(x_n)$ mit
$x_{n+1}=\varphi(x_n)$ konvergiert gegen diese Nullstelle.
\end{Satz}

\begin{Beweis}
Gemäß den Ableitungsregeln ist $\varphi$ stetig differenzierbar
und es gilt%
\[\varphi'(x) = 1-\frac{f'(x)f'(x)-f(x)f''(x)}{f'(x)^2}
= \frac{f(x)f''(x)}{f'(x)^2}.\]
Da $|\varphi'(x)|$ stetig ist, gibt es nach dem Satz vom Minimum
und Maximum ein Maximum $M$ und nach Voraussetzung ist $M<1$.
Man setze nun $r:=(M+1)/2$. Dann ist $|\varphi'(x)|<r<1$.
Gemäß Satz \ref{diff-fixed-point-iter} konvergiert die Iteration
$(x_n)$ gegen den einzigen Fixpunkt von $\varphi$. Wegen $f'(x)\ne 0$
gilt dabei%
\[x = \varphi(x) = x-\frac{f(x)}{f'(x)} \iff \frac{f(x)}{f'(x)}=0\iff f(x)=0.\]
Der Fixpunkt von $\varphi$ ist also die einzige Nullstelle von $f$.\;\qedsymbol
\end{Beweis}



\chapter{Lineare Algebra}
\section{Grundbegriffe}
\subsection{Norm}\index{Norm}
\begin{definition}[Norm]\mbox{}\newline
Eine Abbildung $v\mapsto\|v\|$ von einem
Vektorraum $V$ über dem Körper $K$ in die nichtnegativen reellen
Zahlen heißt \emdef{Norm}, wenn für alle $v,w\in V$ und $a\in K$
die drei Axiome%
\begin{gather}
\|v\|=0 \implies v=0,\\
\|av\| = |a|\,\|v\|,\\
\|v+w\| \le \|v\|+\|w\|
\end{gather}
erfüllt sind.
\end{definition}

\noindent
Eigenschaften:
\begin{gather}
\|v\|=0\iff v=0,\\
\|-v\|=\|v\|,\\
\|v\|\ge 0.
\end{gather}
Dreiecksungleichung nach unten:
\begin{equation}
|\|v\|-\|w\||\le \|v-w\|.
\end{equation}

\subsection{Skalarprodukt}\index{Skalarprodukt}

Ein Vektorraum über dem Körper $\R$ heißt \emdef{reeller Vektorraum},
einer über dem Körper $\C$ heißt \emdef{komplexer Vektorraum}.

\subsubsection{Definition}
Sei $V$ ein reeller Vektorraum. Eine Abbildung $f\colon V^2\to\R$
mit $f(x,y)=\langle x,y\rangle$ heißt \emdef{Skalarprodukt}, wenn
folgende Axiome erfüllt sind. Für $v,w\in V$ und $\lambda\in\R$ gilt:
\begin{gather}
\langle v,w\rangle = \langle w,v\rangle,\\
\langle v,\lambda w\rangle = \lambda\langle v,w\rangle,\\
\langle v,w_1+w_2\rangle = \langle v,w_1\rangle +\langle v,w_2\rangle,\\
\langle v,v\rangle\ge 0,\\
\langle v,v\rangle=0 \iff v=0.
\end{gather}
Sei $V$ ein komplexer Vektorraum und $f\colon V^2\to\C$.\\
Für $v,w\in V$ und $\lambda\in\R$ gilt:
\begin{gather}
\langle v,w\rangle = \overline{\langle w,v\rangle},\\
\langle \lambda v,w\rangle = \overline{\lambda}\langle v,w\rangle,\\
\langle v,\lambda w\rangle = \lambda\langle v,w\rangle,\\
\langle v,w_1+w_2\rangle = \langle v,w_1\rangle +\langle v,w_2\rangle,\\
\langle v,v\rangle\ge 0,\\
\langle v,v\rangle=0 \iff v=0.
\end{gather}

\subsubsection{Eigenschaften}
Das reelle Skalarprodukt ist eine symmetrische bilineare Abbildung.

\subsubsection{Winkel und Längen}
\begin{definition}[Winkel, orgthogonale Vektoren]\mbox{}\newline
Der \emdef{Winkel} $\varphi$ zwischen $v$ und $w$
ist definiert durch die Beziehung:
\begin{equation}
\langle v,w\rangle = \|v\|\,\|w\|\,\cos\varphi.
\end{equation}
\emdef{Orthogonal}:\index{Orthogonal}
\begin{equation}
v\perp w \;:\Longleftrightarrow\; \langle v,w\rangle=0.
\end{equation}
\end{definition}

Ein Skalarprodukt $\langle v,w\rangle$ induziert die Norm
\begin{equation}
\|v\| := \sqrt{\langle v,v\rangle}.
\end{equation}

\subsubsection{Orthonormalbasis}\label{sec:ONB}
\index{Orthogonalsystem}\index{Orthogonalbasis}
\index{Orthonormalsystem}\index{Orthonormalbasis}
Sei $B=(b_k)_{k=1}^n$ eine Basis eines endlichdimensionalen
Vektorraumes über den reellen oder komplexen Zahlen.

\begin{definition}[Orthogonalbasis]\mbox{}\newline
Gilt $\langle b_i,b_j\rangle=0$
für alle $i,j$ mit $i\ne j$, so wird $B$ \emdef{Orthogonalbasis}
genannt. Ist $B$ nicht unbedingt eine Basis, so spricht man von einem 
\emdef{Orthogonalsystem}.
\end{definition}

\begin{definition}[Orthonormalbasis]\mbox{}\newline
Ist $B$ eine Orthogonalbasis und gilt
zusätzlich $\langle b_k,b_k\rangle=1$ für alle $k$, so wird
$B$ \emdef{Orthonormalbasis} (ONB) genannt. Ist $B$ nicht unbedingt
eine Basis,  so spricht man von einem \emdef{Orthonormalsystem}.
\end{definition}

Sei $v=\sum_k v_kb_k$ und $w=\sum_k w_kb_k$.
Mit $\sum_k$ ist immer $\sum_{k=1}^n$ gemeint.

Ist $B$ eine Orthonormalbasis, so gilt:
\begin{equation}
\langle v,w\rangle = \sum_k \overline{v_k}\,w_k.
\end{equation}
Ist $B$ nur eine Orthogonalbasis, so gilt:
\begin{equation}
\langle v,w\rangle = \sum_k \langle b_k,b_k\rangle \overline{v_k}\,w_k
\end{equation}
Allgemein gilt:
\begin{equation}
\langle v,w\rangle = \sum_{i,j} g_{ij} \overline{v_i}\,w_j
\end{equation}
mit $g_{ij}=\langle b_i,b_j\rangle$. In reellen Vektorräumen
ist die komplexe Konjugation wirkungslos und kann somit entfallen.

Ist $B$ eine Orthogonalbasis und $v=\sum_k v_k b_k$, so gilt:
\begin{equation}
v_k = \frac{\langle b_k,v\rangle}{\langle b_k,b_k\rangle}.
\end{equation}
Ist $B$ eine Orthonormalbasis, so gilt speziell:
\begin{equation}
v_k = \langle b_k,v\rangle.
\end{equation}


\subsubsection{Orthogonale Projektion}
Orthogonale Projektion von $v$ auf $w$:
\begin{equation}
P[w](v) := \frac{\langle v,w\rangle}{\langle w,w\rangle}\,w.
\end{equation}
\subsubsection{Gram-Schmidt-Verfahren}
Für linear unabhängige Vektoren $v_1,\ldots,v_n$
wird durch%
\begin{equation}
w_k := v_k - \sum_{i=1}^{k-1} P[w_i](v_k)
\end{equation}
ein Orthogonalsystem $w_1,\ldots,w_n$ berechnet.

Speziell für zwei nicht kollineare Vektoren $v_1,v_2$ gilt
\begin{gather}
w_1=v_1,\\
w_2=v_2-P[w_1](v_2).
\end{gather}

\newpage
\subsubsection{Musikalische Isomorphismen}
\begin{definition}[Musikalische Isomorphismen]%
\index{musikalische Isomorphismen}\mbox{}\newline
Sei $V$ ein eindlichdimensionaler Vektorraum
mit Skalarprodukt und $V^*$ sein Dualraum.
Die lineare Abbildung%
\begin{equation}
\Phi\colon V\to V^*,\quad \Phi(u)(v):=\langle u,v\rangle
\end{equation}
ist ein kanonischer Isomorphismus.%
\index{kanonischer Isomorphismus!musikalische Isomorphismen}
Man nennt $u^\flat:=\Phi(u)$
und $\omega^\sharp:=\Phi^{-1}(\omega)$ die \emdef{musikalischen
Isomorphismen}.
\end{definition}

\section{Koordinatenvektoren}
\subsection{Koordinatenraum}
Addition von $a,b\in K^n$:
\begin{equation}\label{eq:Koordinatenraum-Addition}
\begin{bmatrix}
a_1\\
\vdots\\
a_n
\end{bmatrix}
+\begin{bmatrix}
b_1\\
\vdots\\
b_n
\end{bmatrix}
:= \begin{bmatrix}
a_1+b_1\\
\vdots\\
a_n+b_n
\end{bmatrix}.
\end{equation}
Subtraktion:
\begin{equation}
\begin{bmatrix}
a_1\\
\vdots\\
a_n
\end{bmatrix}
-\begin{bmatrix}
b_1\\
\vdots\\
b_n
\end{bmatrix}
:= \begin{bmatrix}
a_1-b_1\\
\vdots\\
a_n-b_n
\end{bmatrix}.
\end{equation}
Skalarmultiplikation von $\lambda\in K$ mit $a\in K^n$:
\begin{align}\label{eq:Koordinatenraum-Skalarmultiplikation}
\lambda\begin{bmatrix}
a_1\\
\vdots\\
a_n
\end{bmatrix}
:= \begin{bmatrix}
\lambda a_1\\
\vdots\\
\lambda a_n
\end{bmatrix}.
\end{align}
Ist $K$ ein Körper, so bildet die Menge
\begin{equation}
K^n = \{(a_1,\ldots,a_n)\mid \forall k\colon a_k\in K\}
\end{equation}
bezüglich der Addition \eqref{eq:Koordinatenraum-Addition}
und der Multiplikation \eqref{eq:Koordinatenraum-Skalarmultiplikation}
einen Vektorraum, der \emdef{Koordinatenraum} genannt wird.
Das Tupel $E_n=(e_1,\ldots,e_n)$ mit
\begin{equation}\label{eq:kanonische-Basis}
\begin{split}
e_1 &:= (1,0,0,0,\ldots, 0),\\
e_2 &:= (0,1,0,0,\ldots, 0),\\
e_3 &:= (0,0,1,0,\ldots, 0),\\
\ldots\\
e_n &:= (0,0,0,0,\ldots, 1)
\end{split}
\end{equation}
bildet eine geordnete Basis von $K^n$, die \emdef{kanonische Basis}
genannt wird. Es gilt
\begin{equation}
a = (a_1,\ldots,a_n) = a_1 e_1+\ldots+a_n e_n.
\end{equation}

\newpage
\subsection{Kanonisches Skalarprodukt}
\begin{definition}[Kanonisches Skalarprodukt]\mbox{}\newline
Für $a,b\in\R^n$:
\begin{equation}
\langle a,b\rangle := \sum_{k=1}^n a_k b_k.
\end{equation}
Für $a,b\in\C^n$:
\begin{equation}
\langle a,b\rangle := \sum_{k=1}^n \overline{a_k}\,b_k.
\end{equation}
\end{definition}
\noindent
Die kanonische Basis \eqref{eq:kanonische-Basis} ist eine
Orthonormalbasis bezüglich diesem Skalarprodukt, s. \ref{sec:ONB}.
Das Skalarprodukt induziert die Norm
\begin{equation}
|a| := \sqrt{\langle a,a\rangle} = \sqrt{\textstyle \sum_{k=1}^n |a_k|^2},
\end{equation}
die \emdef{Vektorbetrag}\index{Vektorbetrag} genannt wird.

Jedem Koordinatenvektor $a\ne 0$ lässt sich ein Einheitsvektor\index{Einheitsvektor}
$\hat a:=\frac{a}{|a|}$ zuordnen, der in Richtung von $a$ zeigt
und die Eigenschaft $|\hat a|=1$ besitzt.

Es gilt
\begin{align}
a\perp b &\iff \langle a,b\rangle=0,\\
a\uparrow\uparrow b &\iff \langle a,b\rangle = |a|\,|b|,\\
a\uparrow\downarrow b &\iff \langle a,b\rangle = -|a|\,|b|.
\end{align}
Allgemein gilt
\begin{equation}
\langle a,b\rangle = |a|\,|b|\cos\varphi.\qquad(\varphi=\angle (a,b))
\end{equation}

\subsection{Vektorprodukt}
Für $a,b\in\R^3$:\index{Vektorprodukt}
\begin{equation}
a\times b = \begin{bmatrix}
a_x\\ a_y\\ a_z
\end{bmatrix}\times\begin{bmatrix}
b_x\\ b_y\\ b_z
\end{bmatrix}
= \begin{vmatrix}
e_x & a_x & b_x\\
e_y & a_y & b_y\\
e_z & a_z & b_z
\end{vmatrix}
= \begin{bmatrix}
a_y b_z - a_z b_y\\
a_z b_x - a_x b_z\\
a_x b_y - a_y b_x
\end{bmatrix}.
\end{equation}
Rechenregeln für $a,b,c\in\R^3$ und $r\in\R$:
\begin{gather}
a\times (b+c) = a\times b+a\times c,\\
(a+b)\times c = a\times c+b\times c,\\
(ra)\times b = r(a\times b) = a\times(rb),\\
a\times b = -b\times a,\\
a\times a = 0.
\end{gather}
Für den Betrag gilt:
\begin{equation}
|a\times b| = |a|\,|b|\,\sin\varphi.\qquad(\varphi=\angle (a,b))
\end{equation}
Beziehung zur Determinante:
\begin{equation}
\langle a,b\times c\rangle = \det(a,b,c).
\end{equation}
Jacobi-Identität:%
\index{Identität!Jacobi-Identität}\index{Jacobi-Identität}
\begin{equation}
a\times(b\times c) = b\times (a\times c) - c\times (a\times b). 
\end{equation}
Graßmann-Identität:%
\index{Identität!Graßmann-Identität}\index{Graßmann-Identität}
\begin{equation}
a\times(b\times c) = b\langle a,c\rangle - c\langle a,b\rangle.
\end{equation}
Cauchy-Binet-Identität:
\index{Identität!Cauchy-Binet-Identität}\index{Cauchy-Binet-Identität}
\begin{equation}
\langle a\times b, c\times d\rangle
= \langle a,c\rangle\langle b,d\rangle
- \langle b,c\rangle\langle a,d\rangle.
\end{equation}
Lagrange-Identität:%
\index{Identität!Lagrange-Identität}\index{Lagrange-Identität}
\begin{equation}
|a\times b|^2 = |a|^2 |b|^2 - \langle a,b\rangle^2.
\end{equation}

\newpage
\section{Matrizen}\index{Matrix}
\subsection{Quadratische Matrizen}%
\index{Matrix!quadratische}\index{quadratische Matrix}
\subsubsection{Matrizenring}%
\index{Ring!Matrizenring}\index{Matrizenring}
Mit $K^{n\times n}$ wird die Menge quadratischen Matrizen
\begin{equation}
(a_{ij}) = \begin{bmatrix}
a_{11} & \ldots & a_{1n}\\
\ldots & \ddots & \ldots\\
a_{n1} & \ldots & a_{nn}
\end{bmatrix}
\end{equation}
mit Einträgen $a_{ij}$ aus dem Körper $K$ bezeichnet.

Die Menge $K^{n\times n}$ bildet bezüglich Addition
und Multiplikation von Matrizen einen Ring (s. \ref{sec:Strukturen}).

Das neutrale Element der Multiplikation
ist die Einheitsmatrix
\begin{equation}
E_n = (\delta_{ij}),\quad
\delta_{ij}:=\begin{cases}
1 & \text{wenn}\;i=j,\\
0 & \text{sonst}.
\end{cases}
\end{equation}
Das sind
\begin{equation}
E_2 = \begin{bmatrix}
1 & 0\\
0 & 1
\end{bmatrix},\quad
E_3 = \begin{bmatrix}
1 & 0 & 0\\
0 & 1 & 0\\
0 & 0 & 1
\end{bmatrix},
\quad\text{usw.}
\end{equation}

\subsubsection{Symmetrische Matrizen}
Eine quadratiche Matrix $A=(a_{ij})$ heißt
symmetrisch\index{symmetrische Matrix}\index{Matrix!symmetrische},
falls gilt $a_{ij}=a_{ji}$ bzw. $A^T=A$.

Jede reelle symmetrische Matrix besitzt ausschließlich reelle
Eigenwerte und die algebraischen Vielfachheiten stimmen mit den
geometrischen Vielfachheiten überein.

Jede reelle symmetrische Matrix $A$ ist diagonalisierbar, d.\,h. es gibt
eine invertierbare Matrix $T$ und eine Diagonalmatrix $D$, so dass
$A=TDT^{-1}$ gilt.

Sei $V$ ein $K$-Vektorraum und $(b_k)_{k=1}^n$ eine Basis von $V$.
Für jede symmetrische Bilinearform\index{symmetrische Bilinearform}
$f\colon V^2\to K$ ist die
Darstellungsmatrix
\begin{equation}
A = (f(b_i,b_j))
\end{equation}
symmetrisch. Ist $A\in K^{n\times n}$ eine symmetrische Matrix, so
ist
\begin{equation}\label{eq:symmBf}
f(x,y) = x^T A y.
\end{equation}
eine symmetrische Bilinearform für  $x,y\in K^n$.
Ist $K=\R$ und $A$ positiv definit, so ist
\eqref{eq:symmBf} ein Skalarprodukt auf $\R^n$.

\subsubsection{Reguläre Matrizen}\index{inverse Matrix}
\begin{definition}[Reguläre Matrix, singuläre Matrix]\mbox{}\newline
Eine quadratische Matrix $A\in K^{n\times n}$ heißt
\emdef{regulär}\index{reguläre Matrix}\index{Matrix!reguläre}
oder \emdef{invertierbar}, wenn es eine inverse Matrix $A^{-1}$ gibt,
so dass
\begin{equation}
A^{-1}A = E_n \quad (\iff AA^{-1} = E_n)
\end{equation}
gilt, wobei mit $E_n$ die Einheitsmatrix gemeint ist.
Eine quadratische Matrix die nicht regulär ist, heißt
\emdef{singulär}\index{singuläre Matrix}\index{Matrix!singuläre}.
\end{definition}

\strong{Kriterien.} Für eine Matrix $A\in K^{n\times n}$ gilt:
\begin{gather}
A\text{ ist regulär}\iff \exists B (BA=E_n)\\
\iff \det(A)\ne 0 \iff \operatorname{rk}(A)=n\\
\iff 0\text{ ist kein Eigenwert von }A\\
\iff \ker(A)=\{0\}.
\end{gather}

\noindent
\strong{Eigenschaften.}
Jede reguläre Matrix besitzt genau eine inverse Matrix. 
Die Menge der regulären Matrizen bildet bezüglich Matrizenmultiplikation
eine Gruppe, die
\emdef{allgemeine lineare Gruppe}\index{allgemeine lineare Gruppe}
\begin{equation}
\operatorname{GL}(n,K) := \{A\in K^{n\times n}\mid\det(A)\ne 0\}.
\end{equation}
Ist $V$ ein Vektorraum über dem Körper $K$, so bilden die
Automorphismen bezüglich Verkettung eine Gruppe, die
\emph{Automorphismengruppe}
\begin{equation}
\operatorname{GL}(V) = \operatorname{Aut}(V).
\end{equation}
Ein \emdef{Endomorphismus}\index{Endomorphismus!auf einem Vektorraum}
ist eine lineare Abbildung, welche eine Selbstabbildung ist.
Ein \emdef{Automorphismus}\index{Automorphismus!auf einem Vektorraum}
ist eine bijektiver Endomorphismus.

Wählt man auf $V$ eine Basis
$B$, so ist die Zuordnung der Darstellungsmatrix
\begin{equation}
M_B^B\colon \operatorname{Aut}(V)\to\operatorname{GL}(\dim V,K)
\end{equation}
eine Gruppenisomorphismus.

Endomorphismen die nicht bijektiv sind, bzw. singuläre Matrizen,
lassen die Dimension ihrer Definitionsmenge schrumpfen:
\begin{equation}
f{\in}\operatorname{End}(V){\setminus}\operatorname{Aut}(V)
\Longleftrightarrow \dim f(V)<\dim V.
\end{equation}
Für Matrizen $A\in K^{n\times n}$ bedeutet das, dass sie nicht
den vollen Rang besitzen:
\begin{equation}
\det A=0\iff \operatorname{rk}(A) < n = \dim K^n.
\end{equation}
\strong{Bestimmung der inversen Matrix.}

Für eine Matrix $A\in K^{2\times 2}$ gilt:
\begin{equation}
A^{-1} = \begin{bmatrix}
a & b\\
c & d
\end{bmatrix}^{-1}
= \frac{1}{ad-bc}\begin{bmatrix}
d & -b\\
-c & a
\end{bmatrix},
\end{equation}
wenn $\det A\ne 0$ mit $\det A = ad-bc$.

\begin{definition}[Streichungsmatrix]\mbox{}\newline
Wird in der Matrix $A$ die Zeile $i$ und die Spalte $j$ entfernt,
so entsteht eine neue Matrix $[A]_{ij}$, die
\emdef{Streichungsmatrix}\index{Streichungsmatrix}
von $A$ genannt wird.
\end{definition}
Laplacescher Entwicklungssatz:
\begin{align}
\det A = \sum_{i=1}^n (-1)^{i+j}a_{ij}\det([A]_{ij}),\\
\det A = \sum_{j=1}^n (-1)^{i+j}a_{ij}\det([A]_{ij}).
\end{align}

\subsubsection{Determinanten}\index{Determinante}
Für Matrizen $A,B\in K^{n\times n}$ und $r\in K$ gilt:
\begin{gather}
\det(AB) = \det(A)\det(B),\\
\det(A^T) = \det(A),\\
\det(rA) = r^n\det(A),\\
\det(A^{-1}) = \det(A)^{-1}.
\end{gather}
Für eine Diagonalmatrix $D=\diag(d_1,\ldots,d_n)$ gilt:
\begin{gather}
\det(D) = \prod_{k=1}^n d_k.
\end{gather}
Eine linke Dreiecksmatrix ist eine Matrix der Form
$(a_{ij})$ mit $a_{ij}=0$ für $i<j$. Eine rechte
Dreiecksmatrix ist die Transponierte einer linken
Dreiecksmatrix.

Für eine linke oder rechte Dreiecksmatrix $A=(a_{ij})$ gilt:
\begin{gather}
\det(A) = \prod_{k=1}^n a_{kk}.
\end{gather}

\newpage
\subsubsection{Eigenwerte}
\strong{Eigenwertproblem:}\index{Eigenwert}
Für eine gegebene quadratische Matrix $A$ bestimme
\begin{equation}
\{(\lambda,v)\mid Av = \lambda v,\,v\ne 0\}.
\end{equation}
Das homogene lineare Gleichungssystem
\begin{equation}
Av=\lambda v \iff (A-\lambda E_n)v=0
\end{equation}
besitzt Lösungen $v\ne 0$ gdw.
\begin{equation}
p(\lambda):=\det(A-\lambda E_n)=0.
\end{equation}
Bei $p(\lambda)$ handelt es sich um ein normiertes Polynom
vom Grad $n$, das \emdef{charakeristisches Polynom}
\index{charakteristisches Polynom}
genannt wird.

\strong{Eigenraum:}\index{Eigenraum}
\begin{equation}
\Eig(A,\lambda):=\{v\mid Av=\lambda v\}.
\end{equation}
Die Dimension $\dim\Eig(A,\lambda)$ wird
\emdef{geometrische Vielfachheit}\index{geometrische Vielfachheit}
von $\lambda$ genannt.

\strong{Spektrum:}\index{Spektrum}
\begin{equation}
\sigma(A) := \{\lambda\mid \exists v\ne 0\colon Av=\lambda v\}.
\end{equation}

\subsubsection{Nilpotente Matrizen}
\begin{definition}[Nilpotente Matrix]\mbox{}\newline
Eine quadratische Matrix $A\in K^{n\times n}$ heißt \emdef{nilpotent},
wenn es eine Zahl $k\in\N, k\ge 1$ gibt, so dass gilt:
\begin{equation}
A^k=0.
\end{equation}
Die erste solche Zahl heißt \emdef{Nilpotenzgrad} der Matrix $A$.

Eine äquivalente Bedingung ist:
\begin{equation}
p_A(\lambda):=\det(\lambda E-A)=\lambda^n.
\end{equation}
\end{definition}

\noindent
\strong{Eigenschaften.}
Sei $A$ eine nilpotente Matrix. Es gilt:
\begin{itemize}[itemsep=0pt, leftmargin=3em]
\bitem $A$ besitzt nur den Eigenwert $\lambda=0$.
\bitem $\det(A)=\tr(A)=0$.
\bitem $E-A$ ist invertierbar.
\end{itemize}

\subsection{Matrixfunktionen}
\subsubsection{Matrixexponential}
\begin{definition}[Matrixexponential]\mbox{}\\*
Für eine beliebige Matrix  $X\in \C^{n\times n}$ konvergiert%
\begin{equation}
\exp(X) := \sum_{k=0}^\infty \frac{X^k}{k!}.
\end{equation}
\end{definition}
Für jede Matrix $X$ und $a,b\in\C$ gilt
\begin{gather}
\exp(-X) = \exp(X)^{-1},\\
\exp(X^H) = \exp(X)^H,\\
\exp((a+b)X) = \exp(aX)\exp(bX),\\
\exp(\diag(d_1,\ldots,d_n)) = \diag(\ee^{d_1},\ldots,\ee^{d_n}),\\
\det(\exp(X)) = \ee^{\tr(X)}.
\end{gather}
Für $XY=XY$ gilt
\begin{equation}
\exp(X+Y)=\exp(X)\exp(Y).
\end{equation}
Das Exponential einer Matrix ist immer invertierbar und
jede Matrix aus $\operatorname{GL}(n,\C)$ kann als Matrixexponential
dargestellt werden. D.\,h.
$\exp\colon \C^{n\times n}\to\operatorname{GL}(n,\C)$
ist surjektiv.

\newpage
\subsubsection{Allgemein}

Matrizen bilden bezüglich Matrizenmultiplikation zusammen
mit der Frobeniusnorm oder einer Operatornorm eine assoziative
Banachalgebra mit Einselement.
Man betrachte nun die formale Potenzreihe%
\begin{equation}
f(X) := \sum_{k=0}^\infty a_k X^k,\quad a_k\in \C.
\end{equation}
Besitzt die Einsetzung $f(z)$ für $z\in\C$ den Konvergenzradius
$r>0$ und ist $A$ ein Element einer Banachalgebra
mit Einselement mit $\|A\|<r$,
dann ist $f(A)$ absolut konvergent. Gemäß%
\begin{equation}
f\colon \{\|A\|<r\mid A\in\C^{n\times n}\}\to\C^{n\times n}
\end{equation}
ist daher eine Matrixfunktion definiert.

Ist $A$ diagonalisierbar mit $A=TDT^{-1}$, dann gilt
\begin{equation}
f(A) = Tf(D)T^{-1},
\end{equation}
wobei $f(D)$ gemäß
\begin{equation}
f(\diag(d_1,\ldots,d_n)) = \diag(f(d_1),\ldots,f(d_n))
\end{equation}
berechnet wird.

\strong{Sylvesters Formel.} Allgemein gilt%
\begin{equation}\label{eq:Sylvesters-Formel}
f(A) = \sum_{i=1}^s A_i \sum_{k=0}^{m_i-1} \frac{f^{(k)}(\lambda_i)}{k!}(A-\lambda_i E)^k
\end{equation}
mit
\begin{equation}
A_i := \prod_{j=1,\,j\ne i}^s \frac{1}{\lambda_i-\lambda_j}(A-\lambda_j E).
\end{equation}
Hierbei ist $s$ die Anzahl der unterschiedlichen Eigenwerte und $m_i$
die algebraische Vielfachheit von $\lambda_i$.

Bei einer diagonalisierbaren Matrix vereinfacht sich die Formel zu
\begin{equation}
f(A) = \sum_{i=1}^n A_i f(\lambda_i).
\end{equation}
Speziell für $2\times 2$-Matrizen gilt
\begin{equation}
f(A) = pA+qE
\end{equation}
mit
\begin{align}
p &= \frac{f(\lambda_2)-f(\lambda_1)}{\lambda_2-\lambda_1},\\
q &= f(\lambda_1)-p\lambda_1 = f(\lambda_2)-p\lambda_2.
\end{align}
Im Fall $\lambda_1=\lambda_2$ ist $p=f'(\lambda_1)$.

\strong{Als Cauchy-Integral.} Sei $f\colon U\to\C$ holomorph
und $G\subseteq U$ abgeschlossen und einfach zusammenhängend.
Liegen alle Eigenwerte von $A$ im Inneren von $G$, dann gilt%
\begin{equation}\label{eq:Matrixfunktion-Cauchy}
f(A) = \frac{1}{2\pi\ui}\int_{\partial G} f(z)(zE-A)^{-1}\,\mathrm dz.
\end{equation}
Die Formeln \eqref{eq:Sylvesters-Formel} und
\eqref{eq:Matrixfunktion-Cauchy} liefern außerdem
zwei miteinander verträgliche Verallgemeinerungen der Definition
der Matrixfunktion.

\newpage
\section{Lineare Gleichungssysteme}
\index{lineares Gleichungssytem}
Ein lineares Gleichungssystem mit $m$ Gleichungen und $n$ Unbekannten
hat die Form:
\begin{equation}\label{eq:LGS}
\begin{split}
a_{11} x_1 + a_{12} x_2 + \ldots + a_{1n} x_n &= b_1,\\
a_{21} x_1 + a_{22} x_2 + \ldots + a_{2n} x_n &= b_2,\\
&\;\;\vdots\\
a_{m1} x_1 + a_{m2} x_2 + \ldots + a_{mn} x_n &= b_n.
\end{split}
\end{equation}
Das System lässt sich durch
\begin{equation}
A:=\begin{bmatrix}
a_{11} & a_{12} & \ldots & a_{1n}\\
a_{21} & a_{22} & \ldots & a_{2n}\\
\vdots & \vdots & \ddots & \vdots\\
a_{m1} & a_{m1} & \ldots & a_{mn}
\end{bmatrix}
\end{equation}
und
\begin{equation}
x:=\begin{bmatrix}
x_1 \\ x_2 \\ \vdots \\ x_n
\end{bmatrix},\quad
b:=\begin{bmatrix}
b_1 \\ b_2 \\ \vdots \\ b_n
\end{bmatrix}
\end{equation}
zusammenfassen.

Äquivalente Matrixform von \eqref{eq:LGS}:
\begin{equation}
Ax=b.
\end{equation}
Erweiterte Koeffizientenmatrix:\index{erweiterte Koeffizientenmatrix}
\begin{equation}
(A\,|\,b) := \left[\begin{array}{ccc|c}
a_{11} & \ldots & a_{1n} & b_1\\
\vdots & \ddots & \vdots & \vdots\\
a_{m1} & \ldots & a_{mn} & b_n
\end{array}\right].
\end{equation}
Lösungskriterium:
\begin{equation}
\exists x[Ax=b] \iff \rg(A)=\rg(A\,|\,b).
\end{equation}
Eindeutige Lösung (bei $n$ Unbekannten):
\begin{equation}
\exists! x[Ax=b] \iff\rg(A)=\rg(A\,|\,b)=n.
\end{equation}
Im Fall $m=n$ gilt:
\begin{equation}
\begin{split}
&\exists! x[Ax=b] \iff A\in\operatorname{GL}(n,K)\\
&\iff \rg(A)=n \iff \det(A)\ne 0.
\end{split}
\end{equation}

% \newpage
\section{Multilineare Algebra}
\subsection{Äußeres Produkt}
Sei $V$ ein Vektorraum und sei $v_k\in V$ für alle $k$.

Sind $a=\sum_{k=1}^n a_k v_k$
und $b=\sum_{k=1}^n b_k v_k$ beliebige
Linearkombinationen, so gilt
\begin{equation}
\begin{split}
a\wedge b &= \sum_{i,j} a_i b_j\,v_i\wedge v_j\\
&= \sum_{1\le i<j\le n} (a_i b_j-a_j b_i)\,v_i\wedge v_j
\end{split}
\end{equation}
und
\begin{equation}
\begin{split}
a\wedge b &= a\otimes b-b\otimes a\\
&= \sum_{i,j} (a_i b_j-a_j b_i)\,v_i\otimes v_j\\
&= \sum_{i,j} a_i b_j (v_i\otimes v_j-v_j\otimes v_i).
\end{split}
\end{equation}
\subsubsection{Alternator}\index{Alternator}
Für $a_k\in V$ ist
$\operatorname{Alt}_p\colon T^p(V)\to T^p(V)$
mit
\begin{equation}
\begin{split}
& \operatorname{Alt}_p (a_1\otimes\ldots\otimes a_p)\\
&:= \frac{1}{p!}\sum_{\sigma\in S_{\scriptstyle p}}
\sgn(\sigma)\,(a_{\sigma(1)}\otimes\ldots\otimes a_{\sigma(p)}).
\end{split}
\end{equation}
Mit $A^p(V)$ wird die Bildmenge des Alternators bezeichnet.
Der Raum $\Lambda^p(V)$ wird kanonisch mit $A^p(V)$ identifiziert, indem
\begin{equation}
a_1\wedge\ldots\wedge a_p
= p!\operatorname{Alt}_p(a_1\otimes\ldots\otimes a_p)
\end{equation}
gesetzt wird. Hierdurch wird ein kanonischer Isomorphismus%
\index{kanonischer Isomorphismus!Alternator} zwischen
den Algebren $\Lambda(V)$ und $A(V)$ induziert.

Speziell gilt
\begin{equation}
\operatorname{Alt}_2 (a\otimes b) := \frac{1}{2}(a\otimes b-b\otimes a).
\end{equation}
und
\begin{equation}
a\wedge b = 2\operatorname{Alt}_2(a\otimes b).
\end{equation}

\subsubsection{Äußere Algebra}\index{aussere Algebra@äußere Algebra}
Darstellung als Quotientenraum:
\begin{equation}
\Lambda^2(V) = T^2(V)/\{v\otimes v\mid v\in V\}.
\end{equation}
Dimension: Ist $\dim(V)=n$, so gilt
\begin{equation}
\dim(\Lambda^k(V)) = \binom{n}{k}.
\end{equation}

\clearpage
\section{Analytische Geometrie}
\subsection{Geraden}\index{Gerade}
\subsubsection{Parameterdarstellung}
\index{Parameterdarstellung!einer Geraden}

\strong{Punktrichtungsform:}\index{Punktrichtungsform}
\begin{equation}
p(t) = p_0+t\underline v,
\end{equation}
$p_0$: Stützpunkt, $\underline v$: Richtungsvektor.
Die Gerade ist dann die Menge $g=\{p(t)\mid t\in\R\}$.

Der Vektor $\underline v$ repräsentiert außerdem die Geschwindigkeit,
mit der diese Parameterdarstellung durchlaufen wird:
$p'(t)=\underline v$.

\strong{Gerade durch zwei Punkte:}
Sind zwei Punkte $p_1,p_2$ mit $p_1\ne p_2$ gegeben, so ist
durch die beiden Punkte eine Gerade gegeben. Für diese Gerade ist
\begin{equation}
p(t) = p_1+t(p_2-p_1)
\end{equation}
eine Punktrichtungsform\index{Punktrichtungsform}.
Durch Umformung ergibt sich die \strong{Zweipunkteform:}
\begin{equation}\label{eq:Zweipunkteform}
p(t) = (1-t)p_1+tp_2.
\end{equation}
Bei \eqref{eq:Zweipunkteform} handelt es sich um eine
Affinkombination. Gilt $t\in[0,1]$, so ist \eqref{eq:Zweipunkteform}
eine Konvexkombination: eine Parameterdarstellung für die Strecke
von $p_1$ nach $p_2$.

\subsubsection{Parameterfreie Darstellung}
\strong{Hesse-Form:}
\begin{equation}\label{eq:Hesse-Form}
g = \{p\mid\langle \uv n,p-p_0\rangle = 0\},
\end{equation}
$p_0$: Stützpunkt, $\uv n$: Normalenvektor.

Die Hesse-Form ist nur in der Ebene möglich.
Form \eqref{eq:Hesse-Form} hat in Koordinaten
die Form
\begin{equation}
\begin{split}
g &= \{(x,y)\mid n_x(x-x_0)+n_y(y-y_0)=0\}\\
&= \{(x,y)\mid n_x x+n_y y = n_x x_0+n_y y_0\}.
\end{split}
\end{equation}

\strong{Hesse-Normalform:} \eqref{eq:Hesse-Form} mit $|\uv n|=1$.


Sei $\uv v\wedge\uv w$ das äußere Produkt.

\strong{Plückerform:}
\begin{equation}
g = \{p\mid (p-p_0)\wedge \underline v=0\}.
\end{equation}
Die Größe $\underline m = p_0\wedge\underline v$ heißt Moment.
Beim Tupel $(\underline v:\underline m)$ handelt es sich um
Plückerkoordinaten für die Gerade.

In der Ebene gilt speziell:
\begin{equation}\label{eq:Gerade-Ebene}
g = \{(x,y)\mid (x-x_0)\Delta y = (y-y_0)\Delta x\}
\end{equation}
mit $\underline v=(\Delta x,\Delta y)$.

Sei $a:=\Delta y$ und $b:=-\Delta x$ und $c:=ax_0+by_0$.
Aus \eqref{eq:Gerade-Ebene} ergibt sich:
\begin{equation}
g = \{(x,y)\mid ax+by=c\}.
\end{equation}
Im Raum ergibt sich ein Gleichungssystem:
\begin{equation}
g = \{\begin{pmatrix}x\\ y\\ z\end{pmatrix}
\mid
\begin{vmatrix}
(x-x_0)\Delta y = (y-y_0)\Delta x\\
(y-y_0)\Delta z = (z-z_0)\Delta y\\
(x-x_0)\Delta z = (z-z_0)\Delta x
\end{vmatrix}\}
\end{equation}
mit $\underline v=(\Delta x,\Delta y,\Delta z)$.

\subsubsection{Abstand Punkt zu Gerade}
Sei $p(t):=p_0+t\underline v$ die Punktrichtungsform einer Geraden und
sei $q$ ein weiterer Punkt. Bei $\underline d(t):=p(t)-q$ handelt
es sich um den Abstandsvektor in Abhängigkeit von $t$.

Ansatz: Es gibt genau ein $t$, so dass gilt:
\begin{equation}
\langle\underline d,\underline v\rangle=0.
\end{equation}
Lösung:
\begin{equation}
t = \frac
  {\langle\underline v,q{-}p_0\rangle}
  {\langle\underline v,\underline v\rangle}.
\end{equation}

\subsection{Ebenen}\index{Ebene}
\subsubsection{Parameterdarstellung}
\index{Parameterdarstellung!einer Ebene}
Seien $\uv u, \uv v$ zwei nicht kollineare Vektoren.

Punktrichtungsform:
\begin{equation}\label{eq:Ebene-Punktrichtungsform}
p(s,t) = p_0+s\uv u+t\uv v.
\end{equation}

\subsubsection{Parameterfreie Darstellung}
Seien $\uv v, \uv w$ zwei nicht kollineare Vektoren.
Durch
\begin{equation}
E = \{p\mid (p-p_0)\wedge\uv v\wedge\uv w=0\}.
\end{equation}
wird eine Ebene beschrieben.

\strong{Hesse-Form:}
\begin{equation}
E = \{p\mid \langle\uv n,p-p_0\rangle=0\},
\end{equation}
$p_0$: Stützpunkt, $\uv n$: Normalenvektor. Die Hesse-Form einer
Ebene ist nur im dreidimensionalen Raum möglich.
Den Normalenvektor bekommt man aus \eqref{eq:Ebene-Punktrichtungsform}
mit $\uv n = \uv u\times\uv v$.

Es gilt:
\begin{equation}
\langle\uv n,p-p_0\rangle\iff \langle\uv n,p\rangle = \langle\uv n,p_0\rangle.
\end{equation}
Über den Zusammenhang $\uv n=(a,b,c)$, $p=(x,y,z)$ und $d=\langle\uv n,p_0\rangle$
ergibt sich die

\strong{Koordinatenform:}
\begin{equation}
E = \{(x,y,z)\mid ax+by+cz = d\}.
\end{equation}

\subsubsection{Abstand Punkt zu Ebene}
Sei $p(s,t):=p_0+s\uv u+t\uv v$ die Punktrichtungsform einer Ebene
und sei $q$ ein weiterer Punkt. Bei $\uv d(s,t):=p-q$ handelt es sich um
den Abstandsvektor in Abhängigkeit von $(s,t)$.

Ansatz: Es gibt genau ein Tupel $(s,t)$, so dass gilt:
\begin{equation}
\langle\uv d,\uv u\rangle=0\enspace\text{und}\enspace
\langle\uv d,\uv v\rangle=0.
\end{equation}
Lösung: Es ergibt sich ein LGS:
\begin{equation}
\begin{bmatrix}
\langle\uv u,\uv v\rangle & \langle\uv v,\uv v\rangle\\
\langle\uv v,\uv v\rangle & \langle\uv u,\uv v\rangle
\end{bmatrix}
\begin{bmatrix}
s\\ t
\end{bmatrix}
= \begin{bmatrix}
\langle\uv v,q{-}p_0\rangle\\
\langle\uv u,q{-}p_0\rangle
\end{bmatrix}.
\end{equation}
Bemerkung: Die Systemmatrix $g_{ij}$ ist der metrische Tensor für die
Basis $B=(\uv u,\uv v)$. Die Lösung des LGS ist:
\begin{gather}
s = \frac
  {\langle g_{12}\uv v-g_{12}\uv u, q{-}p_0\rangle}
  {g_{11}^2-g_{12}^2},\\
t = \frac
  {\langle g_{12}\uv u-g_{12}\uv v, q{-}p_0\rangle}
  {g_{11}^2-g_{12}^2}.
\end{gather}



\chapter{Differentialgeometrie}
\section{Kurven}
\subsection{Parameterkurven}\index{Weg}\index{Kurve}
\begin{Definition}
Sei $X$ ein topologischer Raum und
$I$ ein reelles Intervall, auch offen oder halboffen, auch unbeschränkt.
Eine stetige Funktion
\begin{equation}
f\colon I\to X
\end{equation}
heißt \emdef{Parameterdarstellung einer Kurve}, kurz
\emdef{Parameterkurve}. Die Bildmenge $f(I)$ heißt \emdef{Kurve}.
\end{Definition}

Eine Parameterdarstellung mit einem kompakten Intervall $I=[a,b]$
heißt \emdef{Weg}.

Für einen Weg mit $I=[a,b]$ heißt $f(a)$ \emdef{Anfangspunkt}
und $f(b)$ \emdef{Endpunkt}. Ein Weg mit $f(a)=f(b)$
heißt \emdef{geschlossen}. Ein Weg, dessen Einschränkung auf $[a,b)$
injektiv ist, heißt \emdef{einfach}, auch \emdef{doppelpunktfrei} oder
\emdef{Jordan-Weg}.

Bsp. für einen einfachen geschlossenen Weg:
\begin{equation}
f\colon [0,2\pi]\to \R^2,\quad
f(t):=\begin{bmatrix}
\cos t\\
\sin t
\end{bmatrix}.
\end{equation}
Die Kurve ist der Einheitskreis.

Bsp. für einen geschlossenen Weg mit Doppelpunkt:
\begin{equation}
f\colon [0,2\pi]\to \R^2,\quad
f(t):=\begin{bmatrix}
2\cos t\\
\sin(2t)
\end{bmatrix}.
\end{equation}
Die Kurve ist eine Achterschleife.

\subsection{Differenzierbare Parameterkurven}
\begin{Definition}
Eine Parameterkurve $f\colon (a,b)\to\R^n$ heißt
\emdef{differenzierbar}, wenn die Ableitung $f'(t)$ an jeder Stelle
$t$ existiert. Die Ableitung $f'(t)$ wird
\emdef{Tangentialvektor} an die Kurve an der Stelle $t$ genannt.
\end{Definition}

Ein \emdef{$C^k$-Kurve} ist ein Parameterkurve, dessen $k$-te Ableitung
eine stetige Funktion ist. Ein unendlich oft differenzierbare
Parameterkurve heißt \emdef{glatt}.

Eine Parameterkurve heißt \emdef{regulär}, wenn:
\begin{equation}
\forall t\colon f'(t)\ne 0.
\end{equation}

\section{Koordinatensysteme}
\subsection{Polarkoordinaten}\index{Polarkoordinaten}
Polarkoordinaten $r,\varphi$ sind gegeben durch
die Abbildung%
\begin{equation}
\begin{bmatrix}x\\ y\end{bmatrix}
=f(r,\varphi)
:=\begin{bmatrix}
r\cos\varphi\\
r\sin\varphi
\end{bmatrix}
\end{equation}
mit $r>0$ und $0\le\varphi<2\pi$.

Umkehrabbildung für $(x,y)\ne (0,0)$:
\begin{equation}
\begin{bmatrix}r\\ \varphi\end{bmatrix}
= f^{-1}(x,y)
= \begin{bmatrix}
r\\
\displaystyle s(y)\arccos\Big(\frac{x}{r}\Big)
\end{bmatrix}
\end{equation}
mit $r=\sqrt{x^2+y^2}$\\
und $s(y)=\sgn(y)+1-|\sgn(y)|$.

Jacobi-Determinante:
\begin{equation}
\det J = \det((Df)(r,\varphi)) =r.
\end{equation}

\pagebreak[3]\noindent
Darstellung des metrischen Tensors in Polarkoordinaten:%
\begin{equation}
(g_{ij}) = J^T J = \begin{bmatrix}
1 & 0\\
0 & r^2
\end{bmatrix}.
\end{equation}

\section{Mannigfaltigkeiten}
\subsection{Grundbegriffe}
\begin{Definition}
Seien $U,V$ offene Mengen. Eine Abbildung
\begin{equation}
\varphi\colon (U\subseteq\R^n)\to(V\subseteq\R^m)
\end{equation}
heißt \emdef{regulär}, wenn
\begin{equation}
\forall u\in U\colon \operatorname{rg}((D\varphi)(u))=\min(m,n)
\end{equation}
gilt. Mit $(D\varphi)(u)$ ist dabei die Jacobi-Matrix an der Stelle
$u$ gemeint:
\begin{equation}
((D\varphi)(u))_{ij} := \frac{\partial\varphi_i(u)}{\partial u_j}.
\end{equation}
\end{Definition}
\noindent
Für $(D\varphi)(u)\colon\R^n\to\R^m$ gilt:
\begin{gather}
n{\ge}m\implies\forall u\colon (D\varphi)(u)\;\text{ist surjektiv},\\
n{<}m\implies\forall u\colon (D\varphi)(u)\;\text{ist injektiv}.
\end{gather}

\begin{Definition}
Sei $m,n\in\N, n<m$ und sei $M\subseteq\R^m$.
Eine Abbildung $\varphi$ von einer offenen Menge $U'\subseteq\R^n$
in eine offene Menge $U\subseteq M$ heißt \emdef{Karte},
wenn $\varphi$ ein Homöomorphismus und $\varphi\colon U'\to\R^m$
eine reguläre Abbildung ist. Ist $U$ eine offene Umgebung von
$p\in M$, so heißt $\varphi$ \emdef{lokale Karte} bezüglich $p$.
\end{Definition}
\pagebreak[1]
\begin{Definition}
Sei $m,n\in\N, n<m$. Eine Menge $M\subseteq\R^m$ heißt
\emdef{$n$-dimensionale Untermannigfaltigkeit} des $\R^m$, wenn
es zu jedem Punkt $p\in M$ eine lokale Karte
\begin{equation}
\varphi\colon (U'\subseteq R^n)\to (U\subseteq M\subseteq\R^m)
\end{equation}
gibt.
\end{Definition}
\begin{Definition} Ein \emdef{Atlas} für eine Mannigfaltigkeit $M$
ist eine Menge von Karten, deren Bildmengen $M$ überdecken.
\end{Definition}
Sei $M$ eine glatte Mannigfaltigkeit.
\begin{Definition}
Eine Abbildung $f\colon M\to\R$ ist ($k$ mal) (stetig)
\emdef{differenzierbar}
gdw. für jede Karte $\varphi\colon U'\to (U\subseteq M)$ das
Kompositum $f\circ\varphi$ ($k$ mal) (stetig) differenzierbar ist.
Es genügt der Nachweis für alle Karten aus einem Atlas.
\end{Definition}
Seien $M,N$ zwei glatte Mannigfaltigkeiten.
\begin{Definition} Eine Abbildung $f\colon M\to N$ heißt \emdef{glatt}
gdw. für alle Karten $\varphi\colon U'\to (U\subseteq M)$ und
$\psi\colon V'\to (V\subseteq N)$ das Kompositum
$\psi^{-1}\circ f\circ\varphi$ eine glatte Abbildung ist.
Es genügt bereits der Nachweis für alle Karten aus jeweils einem
Atlas für $M$ und $N$.
\end{Definition}


\subsection{Vektorfelder}
\subsubsection{Tangentialräume}
\begin{Definition} \emdef{Tangentialbündel}\index{Tangentialbündel}:
\begin{equation}
TM := \bigsqcup_{p\in M} T_p M = \bigcup_{p\in M} \{p\}\times T_p M.
\end{equation}
\emdef{Kotangentialbündel}\index{Kotangentialbündel}:
\begin{equation}
T^*M := \bigsqcup_{p\in M} T_p^* M.
\end{equation}
\emdef{Natürliche Projektion}\index{natürliche Projektion}:
\begin{equation}
\pi(p,v):=p,\quad\pi\colon TM\to M.
\end{equation}
\end{Definition}
\noindent
Das Tangentialbündel einer glatten Mannigfaltigkeit ist eine
glatte Mannigfaltigkeit.

\subsubsection{Christoffel-Symbole}\index{Christoffel-Symbole}
Sei $(M,g)$ eine pseudo-riemannsche Mannigfaltigkeit.

Es gilt:
\begin{align}
\Gamma_{ab}^k &= \frac{1}{2} g^{kc}
(\partial_a g_{bc}+\partial_b g_{ac}-\partial_c g_{ab}),\\
\Gamma_{cab} &= \frac{1}{2}
(\partial_a g_{bc}+\partial_b g_{ac}-\partial_c g_{ab}),\\
\partial_a g_{bc} &= \Gamma_{bac}+\Gamma_{cab},\\
\Gamma_{ab}^k &= \Gamma_{ba}^k.
\end{align}




\chapter{Funktionentheorie}
\section{Holomorphe Funktionen}

{\definition}
Sei $U\subseteq\C$ eine offene Menge und $f\colon U\to\C$.
Die Funktion $f$ wird \emdef{holomorph}\index{holomorph} an der
Stelle $z_0\in U$ genannt, wenn der Grenzwert
\begin{equation}
f'(z_0) := \lim_{z\to z_0} \frac{f(z)-f(z_0)}{z-z_0}
\end{equation}
existiert.

Das Argument und Bild von $f$ werden nun in Real- und Imaginärteil
zerlegt. Das sind die Zerlegungen $z=x+y\ui$ und $f(z)=u(x,y)+v(x,y)\ui$.
Die Funktion $f(z)$ ist genau dann holomorph an der Stelle
$z_0=x_0+y_0\ui$, wenn bei $(x_0,y_0)$ die partiellen Ableitungen
stetig sind und die \emdef{Cauchy-Riemann-Gleichungen}
\begin{equation}\label{eq:Cauchy-Riemann-Gleichungen}
\frac{\partial u}{\partial x}=\frac{\partial v}{\partial y},
\quad \frac{\partial u}{\partial y}=-\frac{\partial v}{\partial x}
\qquad\text{bei}\;(x_0,y_0)
\end{equation}
gelten. Bei
\begin{equation}
\uv v := (u,-v) = (v_x,v_y) = v_x e_x+v_y e_y
\end{equation}
handelt es sich um ein Vektorfeld
auf dem Koordinatenraum. Die Gleichungen
\eqref{eq:Cauchy-Riemann-Gleichungen} lassen sich nun als
Quellenfreiheit
\begin{equation}
0=\langle\nabla,\uv v\rangle = \frac{\partial v_x}{\partial x}+\frac{\partial v_y}{\partial y}
\end{equation}
und Rotationsfreiheit
\begin{equation}
0=\nabla\wedge \uv v = \Big(\frac{\partial v_y}{\partial x}-\frac{\partial v_x}{\partial y}\Big)\,e_x\wedge e_y
\end{equation}
interpretieren.

Für totale Differential
\begin{equation}
\mathrm df = \frac{\partial f}{\partial x}\mathrm dx+\frac{\partial f}{\partial y}\mathrm dy
\end{equation}
gibt es die Umformulierung
\begin{equation}\label{eq:Differential-Wirtinger-Operatoren}
\mathrm df = \frac{\partial f}{\partial z}\mathrm dz+\frac{\partial f}{\partial\overline z}\mathrm d\overline z.
\end{equation}
Hierbei ist $\mathrm dz=\mathrm dx+\ui\,\mathrm dy$ und $\mathrm d\overline{z}=\mathrm dx-\ui\,\mathrm dy$.

Die Ableitungsoperatoren
\begin{align}
\frac{\partial f}{\partial z}
&:= \frac{1}{2}\bigg(\frac{\partial f}{\partial x}-\ui\frac{\partial f}{\partial y}\bigg),\\
\frac{\partial f}{\partial\overline z}
&:= \frac{1}{2}\bigg(\frac{\partial f}{\partial x}+\ui\frac{\partial f}{\partial y}\bigg)
\end{align}
mit $\partial f=\partial u+\ui\,\partial v$ heißen \emdef{Wirtinger-Operatoren}.

Die Gleichungen \eqref{eq:Cauchy-Riemann-Gleichungen} lassen sich nun
zur Gleichung
\begin{equation}\label{eq:Cauchy-Riemann-Wirtinger}
\frac{\partial f}{\partial\overline z}(z_0)=0
\end{equation}
zusammenfassen. Für holomorphe Funktionen reduziert sich das
Differential \eqref{eq:Differential-Wirtinger-Operatoren} wegen
\eqref{eq:Cauchy-Riemann-Wirtinger} auf die Form
\begin{equation}
\mathrm df = \frac{\partial f}{\partial z}\mathrm dz.
\end{equation}

\section{Harmonische Funktionen}
\begin{Definition}
Sei $U\subseteq\R^2$ eine offene Menge.
Eine Funktion $\Phi\colon U\to\R$ heißt \emdef{harmonisch}
an der Stelle $(x_0,y_0)$, wenn die \emdef{Laplace-Gleichung}
$(\Delta\Phi)(x_0,y_0)=0$ mit dem \emdef{Laplace-Operator}
\begin{equation}
\Delta\Phi := \frac{\partial^2\Phi}{\partial x\partial x}+\frac{\partial^2\Phi}{\partial y\partial y}
\end{equation}
erfüllt ist.
\end{Definition}

Ist $f=u+v\ui$ an der Stelle $z_0$ holomorph, so sind der
Realteil $u$ und der Imaginärteil $v$
an der Stelle $(x_0,y_0)=(\real z_0,\imag z_0)$ harmonisch.
Das heißt es gilt
\begin{equation}
(\Delta u)(x_0,y_0) = 0,\quad (\Delta v)(x_0,y_0)=0.
\end{equation}
Ist eine Funktion $u$ auf einem einfach zusammenhängenden Gebiet
harmonisch, so lässt sich stets eine harmonische Funktion $v$
finden, so dass $f=u+v\ui$ holomorph ist. Die Funktion $v$ ist
bis auf eine additive reelle Konstante $c$ eindeutig bestimmt.
Das heißt, $v$ darf auch durch $v+c$ ersetzt werden.

Die Funktion $v$ wird die \emdef{harmonisch Konjugierte}
zu $u$ genannt. An jeder Stelle $(x_0,y_0)$ treffen die Linien
\begin{align}
&\{(x,y)\mid u(x,y)=u(x_0,y_0)\},\\
&\{(x,y)\mid v(x,y)=v(x_0,y_0)\} 
\end{align}
senkrecht aufeinander.

\section{Wegintegrale}
\strong{Integral einer komplexwertigen Funktion.}

Für $f\colon [a,b]\to\C$ mit $f=u+\ui v$ ist
\begin{equation}
\int_a^b f(t)\,\mathrm dt
= \int_a^b u(t)\,\mathrm dt+\ui\int_a^b v(t)\,\mathrm dt,
\end{equation}
wenn $u$ und $v$ integrierbar sind.

{\definition}
Ist $\gamma\colon [a,b]\to\C$ ein differenzierbarer Weg
\eqref{eq:Parameterkurve}, so wird
\begin{equation}
\int_\gamma f(z)\,\mathrm dz := \int_a^b f(\gamma(t))\,\gamma'(t)\,\mathrm dt
\end{equation}
das \emdef{Kurvenintegral} über $f$ entlang von $\gamma$ genannt.

\strong{Integralsatz von Cauchy.}
Ist $G$ ein einfach zusammenhängendes Gebiet und $f\colon G\to\C$
holomorph, so gilt für jeden Weg $\gamma$ von $\gamma(a)$ nach
$\gamma(b)$ die Formel
\begin{equation}
\int_\gamma f(z)\,\mathrm dz = F(\gamma(b))-F(\gamma(a)),
\end{equation}
wobei die Funktion $F$ nicht vom gewählten Weg abhängig ist.
Außerdem ist $F$ eine Stammfunktion zu $f$, das heißt es gilt
$F'(z)=f(z)$ für alle $z\in G$.

Sind die Voraussetzungen für den Integralsatz erfüllt,
dann motiviert Wegunabhängigkeit die Definition
\begin{equation}
\int_{z_1}^{z_2} f(z)\,\mathrm dz := F(z_2)-F(z_1),
\end{equation}
bei der auf Wege gänzlich verzichtet wird.



\chapter{Dynamische Systeme}
\section{Grundbegriffe}

{\definition}
Ein Tupel $(T,M,\Phi)$ mit $\Phi\colon T\times M\to M$ heißt
\emdef{dynamisches System}\index{dynamisches System},
wenn für alle $t_1,t_2\in T$ und $x\in M$ gilt:
\begin{align}
&\Phi(0,x)=x,\\
&\Phi(t_2,\Phi(t_1,x)) = \Phi(t_1+t_2,x).
\end{align}
Die Menge $T$ heißt \emdef{Zeitraum}.
Ein System mit $T=\N_0$ oder $T=\Z$ heißt \emdef{zeitdiskret},
eines mit $T=\R_0^{+}$ oder $T=\R$ heißt \emdef{zeitkontinuierlich}.
Ein System mit $T=\Z$ oder $T=\R$ heißt \emdef{invertierbar}.

Die Menge $M$ heißt \emdef{Zustandsraum}\index{Zustandsraum},
ihre Elemente werden \emdef{Zustände}\index{Zustand} genannt.

Für ein invertierbares System handelt es sich bei $\Phi$
um eine Gruppenaktion (s. \ref{Gruppenaktion})
der additiven Gruppe $(T,+)$.

Die Menge
\begin{equation}
\Phi(T,x) := \{\Phi(t,x)\mid t\in T\}
\end{equation}
heißt \emdef{Orbit}\index{Orbit!unter einem dynamischen System}
von $x$. S.\,a. \eqref{eq:Orbit}.

\section{Iterationen}

{\definition}
Für eine Selbstabbildung $\varphi\colon M\to M$ lassen sich
die \emdef{Iterationen} gemäß
\begin{equation}
\varphi^0:=\id,\quad \varphi^n:=\varphi^{n-1}\circ\varphi
\end{equation}
formulieren. Mit $\id$ ist die identische Abbildung
\begin{equation}
\id\colon M\to M,\quad \id(x):=x
\end{equation}
und mit $g\circ f$ die Komposition \eqref{eq:composition} gemeint.
Für ein bijektives $\varphi$ wird zusätzlich
\begin{equation}
\varphi^{-n}:=(\varphi^{-1})^n
\end{equation}
definiert.

Die Iterationen bilden ein dynamisches System gemäß%
\begin{equation}
\Phi(n,x):=\varphi^n(x),\quad\Phi\colon\N_0\times M\to M.
\end{equation}
Bei einem bijektiven $\varphi$ lässt sich das System zum invertierbaren
System
\begin{equation}
\Phi(n,x):=\varphi^n(x),\quad\Phi\colon\Z\times M\to M
\end{equation}
erweitern.

{\definition}
Für eine Funktion $\varphi\colon A\to A$ wird der Operator
\begin{equation}
C_\varphi (g) := g\circ\varphi,\quad C_\varphi\colon B^A\to B^A
\end{equation}
\emdef{Kompositionsoperator}\index{Kompositionsoperator} genannt

Wenn $B^A$ ein Funktionenraum ist, dann ist der Kompositionsoperator
ein linearer Operator.


\chapter{Kombinatorik}

\section{Endliche Mengen}

\subsection{Indikatorfunktion}

\begin{Definition}[Iverson-Klammer]\newlinefirst
Für eine Aussage $A$ der klassischen Aussagenlogik definiert man
\[[A] := \begin{cases}
1 &\text{wenn}\;A,\\
0 &\text{sonst}.
\end{cases}\]
\end{Definition}

\begin{Satz}\label{iverson-basic-rules}
Es gilt
\begin{gather*}
[A\land B] = [A][B],\\
[A\lor B] = [A]+[B]-[A][B],\\
[\neg A] = 1-[A],\\
[A\to B] = 1-[A](1-[B]).
\end{gather*}
\end{Satz}
\begin{Beweis} Trivial mittels Wertetabelle.\,\qedsymbol
\end{Beweis}

\begin{Satz}\label{indicator-set-op}
Für die Indikatorfunktion $1_M(x):=[x\in M]$ gilt
\begin{gather*}
1_{A\cap B} = 1_A 1_B,\\
1_{A\cup B} = 1_A + 1_B - 1_{A\cap B}.
\end{gather*}
\end{Satz}
\begin{Beweis}
Gemäß Satz \ref{iverson-basic-rules} gelten die
Rechnungen
\begin{align*}
1_{A\cap B}(x) = [x\in A\cap B]
= [x\in A\land x\in B] = [x\in A][x\in B] = 1_A(x)1_B(x)
\end{align*}
und
\begin{align*}
1_{A\cup B}(x) &= [x\in A\cup B] = [x\in A\lor x\in B]
= [x\in A] + [x\in B] - [x\in A][x\in B]\\
&= 1_A(x) + 1_B(x) - 1_{A\cap B}(x).\,\qedsymbol
\end{align*}
\end{Beweis}

\begin{Satz}
Für endliche Mengen $A,B$ gilt $|A\cup B| = |A|+|B|-|A\cap B|$.
\end{Satz}
\begin{Beweis}
Gemäß Satz \ref{indicator-set-op} darf man rechnen
\begin{align*}
|A\cup B| &= \sum_{x\in G} 1_{A\cup B}(x)
= \sum_{x\in G} (1_A(x) + 1_B(x) - 1_{A\cap B}(x))\\
&= \sum_{x\in G} 1_A(x) + \sum_{x\in G} 1_B(x) - \sum_{x\in G} 1_{A\cap B}(x)
= |A| + |B| - |A\cap B|.\,\qedsymbol
\end{align*}
\end{Beweis}

\begin{Satz}
Für endliche Mengen $A,B$ mit $A\subseteq B$ gilt $|A|\le |B|$.
\end{Satz}
\begin{Beweis}
Mithilfe der Indiktorfunktion findet sich
\begin{gather*}
A\subseteq B \iff (\forall x\colon 1_A(x)\le 1_B(x))
\iff (\forall x\colon 0\le 1_B(x)-1_A(x))\\
\implies 0\le \sum_{x\in B}(1_B(x)-1_A(x))
= \sum_{x\in B} 1_B(x) - \sum_{x\in B} 1_A(x) = |B| - |A|\\
\implies |A|\le |B|.\,\qedsymbol
\end{gather*}
\end{Beweis}

\newpage
\subsection{Endliche Abbildungen}

\begin{Satz}[Anzahl der Abbildungen]\newlinefirst
Seien $X,Y$ endliche Mengen mit $|X| = k$ und $|Y|=n$. Die Menge
der Abbildungen $X\to Y$ enthält $n^k$ Elemente.
\end{Satz}
\begin{Beweis}
Induktion über $k$. Im Anfang $k=0$ ist $X=\emptyset$. Es gibt genau
eine Abbildung $\emptyset\to Y$, nämlich die leere Abbildung.
Gleichermaßen ist $n^0=1$.

Zum Induktionsschritt. Induktionsvoraussetzung sei die Gültigkeit
für $k-1$. Es  sei $|X|=k$ und $|Y|=n$. Gesucht ist die Anzahl
der Möglichkeiten zur Festlegung der Abbildung $f\colon X\to Y$.
Sei $x\in X$ fest. Für die Festlegung $f(x)=y$ bestehen nun genau $n$
Möglichkeiten, nämlich so viele, wie es Elemente $y\in Y$ gibt.
Für die Festlegung der übrigen Werte betrachtet man $f$ als Abbildung%
\[f\colon X\setminus\{x\}\to Y,\]
von denen es laut Voraussetzung $n^{k-1}$ gibt. Wir haben also
$n$ mal $n^{k-1}$ Möglichkeiten, das sind $n^k$.\,\qedsymbol
\end{Beweis}

\begin{Satz}[Anzahl der Bijektionen]\newlinefirst
Seien $X,Y$ endliche Mengen, wobei $|X|=|Y|=n$ gelte.
Die Menge der Bijektionen $X\to Y$ enthält $n!$ Elemente.
\end{Satz}
\begin{Beweis}
Induktion über $n$. Im Anfang $n=0$ ist $X=\emptyset$
und $Y=\emptyset$. Es existiert genau eine Bijektion
$\emptyset\to\emptyset$, nämlich die leere Abbildung.
Bei der Fakultät gilt ebenfalls $0! = 1$ laut
Def. \ref{def:factorial}.

Zum Induktionsschritt. Induktionsvoraussetzung sei die Gültigkeit für
$n-1$. Es sei $|X|=n$. Gesucht ist die Anzahl der Möglichkeiten zur
Festlegung der Bijektion $f\colon X\to Y$. Sei $x\in X$ fest. Für
die Festlegung $f(x)=y$ bestehen genau $n$ Möglichkeiten, nämlich
so viele, wie es Elemente $y\in Y$ gibt. Bei der Festlegung der übrigen
Werte entfällt $y$ aufgrund der Injektivität von $f$. Für die Festlegung
betrachtet man $f$ daher als Bijektion%
\[f\colon X\setminus\{x\}\to Y\setminus\{y\},\]
von denen es laut Voraussetzung $(n-1)!$ gibt. Wir haben also
$n$ mal $(n-1)!$ Möglichkeiten, was gemäß Def. \ref{def:factorial}
gleich $n!$ ist.\,\qedsymbol
\end{Beweis}

\begin{Satz}[Anzahl der Injektionen]\newlinefirst
Seien $X,Y$ endliche Mengen, wobei $|X|=k$ und $|Y|=n$ gelte.
Die Menge der Injektionen $X\to Y$ enthält $n^{\underline k}$
Elemente.
\end{Satz}
\begin{Beweis}
Induktion über $k$. Im Anfang $k=0$ ist $X=\emptyset$. Es gibt
genau eine Injektion $\emptyset\to Y$, nämlich die leere Abbildung.
Gleichermaßen gilt $n^{\underline 0} = 1$.

Zum Schritt. Voraussetzung sei die Gültigkeit
für $k-1$. Es sei $|X|=k$ und $|Y|=n$. Gesucht ist die Anzahl
der Möglichkeiten zur Festlegung der Injektion $f\colon X\to Y$.
Sei $x\in X$ fest. Für die Festlegung $f(x)=y$ bestehen genau
$n$ Möglichkeiten, nämlich so viele, wie es Elemente $y\in Y$ gibt.
Bei der Festlegung der übrigen entfällt $y$ aufgrund der Injektivität
von $f$. Für die Festlegung betrachtet man $f$ daher als Injektion%
\[f\colon X\setminus\{x\}\to Y\setminus\{y\},\]
von denen es laut Voraussetzung $(n-1)^{\underline{k-1}}$ gibt. Es sind
also $n$ mal $(n-1)^{\underline{k-1}}$ Möglichkeiten, was gleich
$n^{\underline k}$ ist.\,\qedsymbol
\end{Beweis}

\newpage
\begin{Satz}\label{bijection-from-k-subsets-to-orbits}
Seien $X,Y$ endliche Mengen und sei $|X|=k$. Sei $C_k(Y)$
die Menge der $k$-elementigen Teilmengen von $Y$. Für zwei
Injektionen $X\to Y$ sei ferner die Äquivalenzrelation
\[f\sim g \defiff \exists\pi\in S_k\colon f=g\circ \pi\]
definiert, wobei mit den $\pi\in S_k$ Permutationen gemeint sind.
Zwischen der Quotientenmenge $\operatorname{Inj}(X, Y)/S_k$
und $C_k(Y)$ besteht eine kanonische Bijektion.
\end{Satz}
\begin{Beweis}
Wir definieren diese Bijektion als
\[\varphi\colon \operatorname{Inj}(X, Y)/S_k\to C_k(Y),
\quad \varphi([f]) := f(X),\]
wobei $[f]=f\circ S_k$ die Äquivalenzklasse des Repräsentanten $f$
bezeichne. Sie ist wohldefiniert, denn für $f\sim g$ gilt
\[f(X) = (g\circ\pi)(X) = g(\pi(X)) = g(X).\]
Für die Injektivität von $\varphi$ ist zu zeigen, dass $f(X) = g(X)$
die Aussage $[f]=[g]$ impliziert, also die Existenz einer Permutation
$\pi$ mit $f=g\circ\pi$. Weil $g$ injektiv ist, existiert eine
Linksinverse $g^{-1}$, so dass wir $\pi:=g^{-1}\circ f$ wählen
können. Es verbleibt somit die Gleichung $f=g\circ g^{-1}\circ f$ zu
zeigen. Zwar ist $g^{-1}$ im Allgemeinen keine Rechtsinverse, ihre
Einschränkung auf $g(X)$ aber schon. Wegen $f(X)=g(X)$ hebt sich
$g\circ g^{-1}$ daher wie gewünscht auf $f(X)$ weg.

Zur Surjektivität von $\varphi$. Hier ist zu zeigen, dass es zu jeder
Menge $B\in C_k(Y)$ eine Injektion $f$ mit $f(X)=B$ gibt. Betrachten
wir sie als Bijektion $f\colon X\to B$. Eine solche besteht,
weil $X$ und $B$ gleichmächtig sind.\,\qedsymbol
\end{Beweis}

\begin{Satz}[Anzahl der Kombinationen]\newlinefirst
Sei $Y$ eine $|Y|=n$ Elemente enthaltende endliche Menge und $C_k(Y)$
die Menge der $k$-elementigen Teilmengen von $Y$.
Es gilt $|C_k(Y)| = \binom{n}{k}$.
\end{Satz}
\begin{Beweis}[Beweis 1]
Sei $X$ eine Menge mit $|X|=k$. Es gilt
\[|C_k(Y)|
\stackrel{\text{(1)}}= |\operatorname{Inj}(X,Y)/S_k|
\stackrel{\text{(2)}}= \frac{|\operatorname{Inj}(X,Y)|}{|S_k|}
= \frac{n^{\underline k}}{k!} = \binom{n}{k}.\]
Gleichung (1) gilt hierbei laut Satz
\ref{bijection-from-k-subsets-to-orbits}.
Die Einsicht von (2) erhält man mit der folgenden Überlegung.
Für jede Gruppe $G$ gilt die Bahnformel $|G| = |f\circ G|\cdot |G_f|$.
Ist nun die Fixgruppe $G_f$ trivial, ist $|G_f|=1$ und
infolge $|f\circ G|=|G|$. Dies ist bei der symmetrischen Gruppe
$G=S_k$ der Fall. Aus diesem Grund enthält jede Bahn $f\circ S_k$
gleich viele Elemente, $|S_k|$ an der Zahl. Weil die Bahnen außerdem
paarweise disjunkt sind, erhält man die Faktorisierung
\[|\operatorname{Inj}(X,Y)| = |S_k|\cdot |\operatorname{Inj}(X,Y)/S_k|.\,\qedsymbol\]
\end{Beweis}
\begin{Beweis}[Beweis 2]
Induktion über $(n, k)$. Im Anfang ist $k=0$ oder $k=n$. Der abstruse
Fall $k=0$ sucht nach Teilmengen ohne Elemente. Es existiert genau eine
solche Menge, nämlich die leere Menge, womit $C_0(Y)=1$ ist. Der Fall
$k=n$ sucht nach Teilmengen, die so viele Elemente haben wie $Y$.
Dies kann nur $Y$ selbst sein, womit $C_n(Y)=1$ gilt.
Gleichermaßen gilt $\binom{n}{0}=1$ und $\binom{n}{n}=1$.

Induktionsvoraussetzung sei die Gültigkeit für $(n-1, k-1)$ und
$(n-1, k)$. Man nimmt nun ein Element $y$ aus $Y$ heraus,
womit darin $n-1$ verbleiben. Entscheidet man sich,
$y$ zur Teilmenge hinzuzufügen, verbleiben noch $k-1$ Elemente
auszuwählen. Entscheidet man sich dagegen, verbleibt die Teilmenge
unverändert, womit nach wie vor $k$ Elemente auszuwählen sind.
Die Anzahl der Möglichkeiten ist somit
\[|C_k(Y)| = |C_{k-1}(Y\setminus\{y\})| + |C_k(Y\setminus\{y\})|
\stackrel{\mathrm{IV}}=\binom{n-1}{k-1} + \binom{n-1}{k}
= \binom{n}{k}.\,\qedsymbol\]
\end{Beweis}

\begin{Satz}[Gitterweg-Interpretation]\newlinefirst
Ein Gitterweg\index{Gitterweg} auf dem Gitter $\Z\times\Z$ heißt
monoton, wenn von $(x,y)$ aus lediglich der Schritt nach $(x+1,y)$
oder der Schritt nach $(x,y+1)$ gewährt ist. Die Anzahl der monotonen
Gitterwege von $(0,0)$ zu $(x,y)$ beträgt
\[\frac{(x+y)!}{x!y!} = \binom{x+y}{x} = \binom{x+y}{y}.\]
\end{Satz}
\begin{Beweis}[Beweis 1]
Alle Gitterwege besitzen dieselbe Länge $x+y$. Die Knoten des jeweiligen
Wegs nummerieren wir der Reihe nach mit Ausnahme des letzten. Nun
ist von den $x+y$ Nummern eine Teilmenge von $y$ Nummern auszuwählen,
an denen ein Schritt nach oben stattfinden soll. Dafür gibt es
$\binom{x+y}{y}$ Möglichkeiten.\,\qedsymbol
\end{Beweis}

\begin{Beweis}[Beweis 2]
Es bezeichne $f(x,y)$ die Anzahl der Wege von $(0,0)$
zu $(x,y)$. Zum Erreichen eines Randpunktes besteht immer nur eine
einzige Möglichkeit, womit $f(x,0)=1$ und $f(0,y)=1$ gilt. Der nicht
auf dem Rand befindliche Punkt $(x,y)$ kann nun von $(x-1,y)$ oder von
$(x,y-1)$ aus erreicht werden, womit
\[f(x,y) = f(x-1,y) + f(x,y-1)\]
gelten muss. Man sieht nun, dass diese Rekursion ein gedrehtes
pascalsches Dreieck erzeugt. Wir setzen daher $f(x,y) = C(x+y,x)$ und
führen die Koordinatentransformation $x+y=n$ und $x=k$ aus. Die
Rekurrenz nimmt damit die Form
\begin{gather*}
C(x+y,x) = C(x-1+y,x-1) + C(x+y-1,x)\\
\iff C(n,k) = C(n-1,k-1) + C(n-1,k).
\end{gather*}
an. Die Randbedingungen führen zu $C(n,n)=1$ und $C(n,0)=1$. Durch diese
Rekurrenz ist eindeutig der Binomialkoeffizient $C(n,k)=\binom{n}{k}$
bestimmt, womit
\[f(x,y) = C(x+y,x) = \binom{x+y}{x}\]
gelten muss.\,\qedsymbol
\end{Beweis}

\begin{Satz}[Rekursionsformel der Potenzmengenabbildung]\newlinefirst
Für $x\notin M$ gilt $\mathcal P(M\cup\{x\}) = \mathcal P(M)\cup\{A\cup\{x\}\mid A\in\mathcal P(M)\}$.
\end{Satz}
\begin{Beweis}
Die Gleichung ist äquivalent zu
\[T\subseteq M\cup\{x\}\iff T\subseteq M\lor\exists A\subseteq M\colon T=A\cup\{x\}.\]
Nehmen wir die rechte Seite an. Im Fall $T\subseteq M$ gilt erst
recht $T\subseteq M\cup\{x\}$. Im anderen Fall liegt ein $A\subseteq M$
vor, womit $A\cup\{x\}\subseteq M\cup\{x\}$ gilt. Wegen $T=A\cup\{x\}$
gilt also ebenfalls $T\subseteq M\cup\{x\}$.

Nehmen wir die linke Seite an. Mit $T\subseteq M\cup\{x\}$ und
$x\notin M$ folgt per Satz \ref{subseteq-diff}
\[T\setminus\{x\}\subseteq (M\cup\{x\})\setminus\{x\} = M,
\;\text{also}\; T\setminus\{x\}\subseteq M.\]
Im Fall $x\notin T$ ist
$T=T\setminus\{x\}$, womit man $T\subseteq M$
erhält. Im Fall $x\in T$ wird $A:=T\setminus\{x\}$
als Zeuge der Existenzaussage gewählt.\,\qedsymbol
\end{Beweis}

\newpage
\section{Endliche Summen}

\subsection{Allgemeine Regeln}

\begin{Definition}[Summe]
Sei $(G,+,0)$ eine kommutative Gruppe und $a_k\in G$. Die Summe ist
rekursiv definiert als
\[\sum_{k=m}^{m-1} a_k := 0,\quad \sum_{k=m}^n a_k
:= a_n + \sum_{k=m}^{n-1} a_k.\]
\end{Definition}

\begin{Satz}\label{sum-add}
Es gilt
\[\sum_{k=m}^n (a_k + b_k) = \sum_{k=m}^n a_k + \sum_{k=m}^n b_k.\]
\end{Satz}
\begin{Beweis} Induktion über $n$. Im Anfang $n=m-1$ haben
beide Seiten der Gleichung den Wert null. Induktionsschritt:
\begin{align*}
\sum_{k=m}^n (a_k+b_k) &= a_n + b_n + \sum_{k=m}^{n-1} (a_k+b_k)
\stackrel{\mathrm{IV}}= a_n + b_n + \sum_{k=m}^{n-1} a_k + \sum_{k=m}^{n-1} b_k\\
&= \sum_{k=m}^n a_k + \sum_{k=m}^n b_k.\,\qedsymbol
\end{align*}
\end{Beweis}

\begin{Satz}\label{sum-scale}
Sei $R$ ein Ring und $c,a_k\in R$. Sei $c$ eine
Konstante. Es gilt
\[\sum_{k=m}^n ca_k = c\sum_{k=m}^n a_k.\]
\end{Satz}
\begin{Beweis} Induktion über $n$. Im Anfang $n=m-1$ haben beide
Seiten der Gleichung den Wert null. Induktionsschritt:
\[\sum_{k=m}^n ca_k = ca_n + \sum_{k=m}^{n-1} ca_k
\stackrel{\mathrm{IV}}= ca_n + c\sum_{k=m}^{n-1} a_k
= c(a_n + \sum_{k=m}^{n-1} a_k) = c\sum_{k=m}^n a_k.\,\qedsymbol\]
\end{Beweis}

\begin{Satz}[Aufteilung einer Summe]\label{sum-split}
Für $m\le p\le n$ gilt
\[\sum_{k=m}^n a_k = \sum_{k=m}^{p-1} a_k + \sum_{k=p}^n a_k.\]
\end{Satz}
\begin{Beweis} Induktion über $n$. Im Induktionsanfang ist $n=p$
und folglich:
\[\sum_{k=m}^p a_k = \sum_{k=m}^{p-1} a_k + p_k
= \sum_{k=m}^{p-1} a_k + \sum_{k=p}^p a_k.\]
Induktionsschritt:
\[\sum_{k=m}^n a_k = a_n + \sum_{k=m}^{n-1} a_k
\stackrel{\mathrm{IV}}= a_n + \sum_{k=m}^{p-1} a_k + \sum_{k=p}^{n-1} a_k
= \sum_{k=m}^{p-1} a_k + \sum_{k=p}^n a_k.\,\qedsymbol\]
\end{Beweis}

\begin{Satz}[Indexshift]\label{sum-indexshift}\newlinefirst
Für die Indexverschiebung der Distanz $d\in\Z$ gilt
\[\textstyle\sum_{k=m}^n a_k = \sum_{k=m+d}^{n+d} a_{k-d}.\]
\end{Satz}
\begin{Beweis}[Beweis 1]
Induktion über $n$. Im Anfang $n = m-1$ haben beide Seiten
der Gleichung den Wert null. Induktionsschritt:
\[\sum_{k=m}^n a_k = a_n+\sum_{k=m}^{n-1}a_k \stackrel{\mathrm{IV}}=
a_{(n+d)-d}+\sum_{k=m+d}^{n+d-1}a_{k-d}
= \sum_{k=m+d}^{n+d}a_{k-d}.\,\qedsymbol\]
\end{Beweis}
\begin{Beweis}[Beweis 2]
Mit der Substitution $k=k'-d$ findet sich die Umformung
\[\sum_{k=m}^n a_k \stackrel{\text{(1)}}= \sum_{m\le k\le n} a_k
\stackrel{\text{(2)}}= \sum_{m\le k'-d\le n} a_{k'-d}
\stackrel{\text{(3)}}= \sum_{m+d\le k'\le n+d} a_{k'-d}
\stackrel{\text{(4)}}= \sum_{k'=m+d}^{n+d} a_{k'-d},\]
wobei (1), (4) gemäß Satz \ref{sum-set-is-range} gelten
und (2), (3) eine andere Schreibweise für die Substitutionsregel
\ref{sum-set-subs} ist.\,\qedsymbol
\end{Beweis}
\strong{Bemerkung.} Der zweite Beweis ist eigentlich zirkulär,
weil der Beweis der Substitutionsregel über den Beweis von
Satz \ref{sum-set-well-defined} in transitiver Abhängigkeit zum
generalisierten Kommutativgesetz \ref{sum-perm-index} steht, dessen
Beweis einen Indexshift enthält.

\begin{Satz} Es gilt
\[\sum_{i=m}^n \sum_{j=m'}^{n'} a_{ij} = \sum_{j=m'}^{n'}\sum_{i=m}^n a_{ij}.\]
\end{Satz}
\begin{Beweis}
Induktion über $n$ und $n'$. Im Anfang bei $n=m-1$ und $n'=m-1$
haben beide Seiten der Gleichung den Wert null. Induktionsschritt für $n$:
\[\sum_{i=m}^n\sum_{j=m'}^{n'} a_{ij}
= \!\!\sum_{j=m'}^{n'} a_{nj}
+ \!\sum_{i=m}^{n-1}\sum_{j=m'}^{n'} a_{ij}
\stackrel{\mathrm{IV}}=
\!\sum_{j=m'}^{n'} a_{nj}
+ \!\sum_{j=m'}^{n'}\sum_{i=m}^{n-1} a_{ij}
= \!\!\sum_{j=m'}^{n'} (a_{nj}+\sum_{i=m}^{n-1} a_{ij})
= \!\!\sum_{j=m'}^{n'} \sum_{i=m}^n a_{ij}.\]
Induktionsschritt für $n'$:
\[\sum_{i=m}^n\sum_{j=m'}^{n'} a_{ij}
= \!\!\sum_{i=m}^n (a_{in'}+\!\!\sum_{j=m'}^{n'-1}a_{ij})
= \!\!\sum_{i=m}^n a_{in'}+\!\!\sum_{i=m}^n\sum_{j=m'}^{n'-1}a_{ij}
\stackrel{\mathrm{IV}}=
\!\sum_{i=m}^n a_{in'}+\sum_{j=m'}^{n'-1}\sum_{i=m}^n a_{ij}
= \!\!\sum_{j=m'}^{n'}\sum_{i=m}^n a_{ij}.\]
Weil immer ein Schritt nach rechts oder ein Schritt nach oben durchführbar ist,
werden alle Punkte $(n,n')$ im Gitter $\Z_{\ge m-1}\times\Z_{\ge m'-1}$ erreicht.\,\qedsymbol
\end{Beweis}

\begin{Satz}[Umkehrung der Reihenfolge]\label{sum-rev}\newlinefirst
Es gilt $\sum_{k=0}^n a_k = \sum_{k=0}^n a_{n-k}$.
\end{Satz}
\begin{Beweis}
Induktion über $n$. Im Anfang $n=-1$ haben beide Seiten der Gleichung
den Wert null. Der Induktionsschritt ist
\begin{align*}
\sum_{k=0}^n a_{n-k} &= a_{n-n} + \sum_{k=0}^{n-1} a_{n-k}
\stackrel{\mathrm{IV}}= a_0+\sum_{k=0}^{n-1} a_{n-(n-1-k)}a_k\\
&= a_0+\sum_{k=0}^{n-1} a_{k-1}
\stackrel{\text{(1)}}= \sum_{k=0}^0 a_k+\sum_{k=1}^n a_k
\stackrel{\text{(2)}}= \sum_{k=0}^n a_k,
\end{align*}
wobei (1) gemäß Indexshift \ref{sum-indexshift} und
(2) gemäß Aufteilung \ref{sum-split} gilt.\,\qedsymbol
\end{Beweis}

\newpage
\begin{Satz}[Generalisiertes Kommutativgesetz]%
\label{sum-perm-index}\newlinefirst
Sei $M=\{k\in\Z\mid m\le k\le n\}$. Für jede Permutation $\pi\colon M\to M$ gilt
\[\sum_{k=m}^n a_k = \sum_{k=m}^n a_{\pi(k)}.\]
\end{Satz}
\begin{Beweis} Induktiv. Sei ohne Beschränkung der Allgemeinheit $m=1$.
Im Induktionsanfang $n=0$ und $n=1$ ist die Gleichung offenkundig erfüllt.

Induktionsschritt. Induktionsvoraussetzung sei die Gültigkeit für $M$.
Zu zeigen ist die Gültigkeit für $M\cup\{n+1\}$.

Sei $t$ ein fester Parameter mit $1\le t\le n+1$.
Im Fall $\pi(t) = n+1$ geht man wie folgt vor.
Man setze $\sigma(k):=\pi(k)$ für $1\le k\le t-1$. Man setze
$\sigma(k):=\pi(k+1)$ für $t\le k\le n$. Weil $n+1$ kein Wert von
$\sigma$ ist, muss $\sigma$ eine Permutation $\sigma\colon M\to M$ sein.
Ergo gilt
\begin{align*}
\sum_{k=1}^{n+1} a_{\pi(k)} &= \sum_{k=1}^{t-1} a_{\pi(k)}
+ a_{\pi(t)} + \sum_{k=t+1}^{n+1} a_{\pi(k)}
= a_{\pi(t)} + \sum_{k=1}^{t-1} a_{\pi(k)}
+ \sum_{k=t}^n a_{\pi(k+1)}\\
&= a_{n+1} + \sum_{k=1}^{t-1} a_{\sigma(k)}
+ \sum_{k=t}^n a_{\sigma(k)}
= a_{n+1} + \sum_{k=1}^n a_{\sigma(k)}\\
&\stackrel{\mathrm{IV}}= a_{n+1} + \sum_{k=1}^n a_k
= \sum_{k=1}^{n+1} a_k.
\end{align*}
Man beachte, dass in den beiden Randfällen $t=1$ und $t=n+1$ die
jeweilige Randsumme den Wert null hat und somit verschwindet.\,\qedsymbol
\end{Beweis}

\begin{Definition}\label{def:sum-set}
Für eine endliche Menge $M$ definiert man
\[\sum_{k\in M} a_k := \sum_{i=m}^n a_{f(i)},\]
wobei $f\colon \{m,\ldots,n\}\to M$ eine frei wählbare Bijektion ist.
\end{Definition}

\begin{Satz}\label{sum-set-well-defined}
Der Wert Summe auf der rechten Seite von Def. \ref{def:sum-set}
ist unabhängig von der gewählten Bijektion.
\end{Satz}
\begin{Beweis} Seien $f,g$ zwei solche Bijektionen. Dann existiert
$\pi$ mit $f=g\circ\pi$, womit%
\[\sum_{i=m}^n a_{f(i)} = \sum_{i=m}^n a_{g(\pi(i))} =
\sum_{i=m}^n a_{g(i)}\]
laut Satz \ref{sum-perm-index} gilt.\,\qedsymbol
\end{Beweis}

\begin{Satz}\label{sum-set-is-range}
Für $M = \{k\in\Z\mid m\le k\le n\}$ gilt
\[\sum_{m\le k\le n} a_k := \sum_{k\in M} a_k = \sum_{k=m}^n a_k.\]
\end{Satz}
\begin{Beweis} Es gilt
$\sum_{k\in M} a_k = \sum_{k=m}^n a_{\id(k)} = \sum_{k=m}^n a_k$.\,\qedsymbol
\end{Beweis}

\begin{Satz}[Substitutionsregel]\label{sum-set-subs}
Ist $\varphi\colon M'\to M$ eine Bijektion, gilt
\[\sum_{k\in M} a_k = \sum_{k'\in M'} a_{\varphi(k')}.\]
\end{Satz}
\begin{Beweis} Zur Bijektion $f\colon\{1,\ldots,|M|\}\to M$ existiert die Bijektion
$g$ mit $f = \varphi\circ g$.

Infolge gilt
\[\sum_{k\in M} a_k = \sum_{i=1}^{|M|} a_{f(i)}
= \sum_{i=1}^{|M|} a_{\varphi(g(i))}
= \sum_{k'\in M'} a_{\varphi(k')}.\,\qedsymbol\]
\end{Beweis}

\begin{Satz} Es gilt
$\sum\limits_{k\in M} ca_k = c\sum\limits_{k\in M} a_k$ und
$\sum\limits_{k\in M} (a_k + b_k)
= \sum\limits_{k\in M} a_k + \sum\limits_{k\in M} b_k$.
\end{Satz}
\begin{Beweis}
Laut Definition gilt
\begin{gather*}
\sum_{k\in M} ca_k = \sum_{i=1}^{|M|} ca_{f(i)}
= c\sum_{i=1}^{|M|} a_{f(i)} = c\sum_{k\in M} a_k,\\
\sum_{k\in M} (a_k+b_k) = \sum_{i=1}^{|M|} (a_{f(i)}+b_{f(i)}) =
\sum_{i=1}^{|M|} a_{f(i)} + \sum_{i=1}^{|M|} b_{f(i)}
= \sum_{k\in M} a_k + \sum_{k\in M} b_k.\,\qedsymbol
\end{gather*}
\end{Beweis}

\begin{Satz} Es gilt
\[\sum_{k\in M}\sum_{l\in N} a_{kl} = \sum_{l\in N}\sum_{k\in M} a_{kl}.\]
\end{Satz}
\begin{Beweis}
Laut Definition gilt
\[\sum_{k\in M}\sum_{l\in N} a_{kl}
= \sum_{i=1}^{|M|}\sum_{j=1}^{|N|} a_{f(i),g(j)}
= \sum_{j=1}^{|N|}\sum_{i=1}^{|M|} a_{f(i),g(j)}
= \sum_{l\in N}\sum_{k\in M} a_{k,l}.\]
\end{Beweis}

\begin{Satz}\label{sum-set-disjoint}
Für $M\cap N=\emptyset$ gilt
\[\sum_{k\in M\cup N} a_k = \sum_{k\in M} a_k + \sum_{k\in N} a_k.\]
\end{Satz}
\begin{Beweis}
Sei $m:=|M|$ und $n:=|N|$. Laut Prämisse existiert eine Bijektion
$f\colon \{1,\ldots, m+n\}\to M\cup N$ mit $f(i)\in M$ für
$1\le i\le m$ und $f(i)\in N$ für $m+1\le i\le m+n$. Das macht
\[\sum_{k\in M\cup N} a_k = \sum_{i=1}^{m+n} a_{f(i)}
= \sum_{i=1}^m a_{f(i)} + \sum_{i=m+1}^{m+n} a_{f(i)}
= \sum_{k\in M} a_k + \sum_{k\in N} a_k.\,\qedsymbol\]
\end{Beweis}

\begin{Satz}\label{sum-partition}
Für eine disjunkte Zerlegung $M = \bigcup_{i\in I} M_i$ gilt
\[\sum_{k\in M} a_k = \sum_{i\in I}\sum_{k\in M_i} a_k.\]
\end{Satz}
\begin{Beweis}
Induktion über $I$. Im Anfang $I=\emptyset$ haben beide Seiten
den Wert null. Induktionsvoraussetzung sei die Gültigkeit
für $I$. Zu zeigen ist die Gültigkeit für $I\cup\{n\}$ mit $n\notin I$.
Der Induktionsschritt ist
\[\sum_{k\in M_n\cup M} a_k
= \sum_{k\in M_n} a_k + \sum_{k\in M} a_k
\stackrel{\mathrm{IV}}= \sum_{k\in M_n} a_k +
\sum_{i\in I}\sum_{k\in M_i} a_k
= \sum_{i\in I\cup\{n\}}\sum_{k\in M_i} a_k.\,\qedsymbol\]
\end{Beweis}

\begin{Satz}
Es gilt
\[\sum_{t\in M\times N} a_t = \sum_{k\in M}\sum_{l\in N} a_{(k,l)}.\]
\end{Satz}
\begin{Beweis}
Es ist $M = \bigcup_{k\in M} \{k\}$ und weiter $M\times N =
\bigcup_{k\in M} (\{k\}\times N)$ eine disjunkte Zerlegung. Hiermit
findet sich die Umformung
\[\sum_{t\in M\times N} a_t
\stackrel{\text{(1)}}= \sum_{k\in M}\;\sum_{t\in \{k\}\times N} a_t
\stackrel{\text{(2)}}= \sum_{k\in M}\sum_{l\in N} a_{(k,l)},\]
wobei (1) laut Satz \ref{sum-partition} gilt und (2) per Substitutionsregel
\ref{sum-set-subs} mit der Bijektion $\varphi\colon N\to\{k\}\times N$
mit $\varphi(l):=(k,l)$ und $t=\varphi(l)$.
\end{Beweis}

\begin{Satz} Mit der Indikatorfunktion $1_A\colon M\to\{0,1\}$
für $A\subseteq M$ gilt
\[\sum_{k\in M} 1_A(k)a_k = \sum_{k\in A} a_k.\]
\end{Satz}
\begin{Beweis}
Mit disjunkter Zerlegung $M=A\cup (M\setminus A)$
und Satz \ref{sum-set-disjoint} gilt
\[\sum_{k\in M} 1_A(k)a_k = \sum_{k\in A} \underbrace{1_A(k)}_{1} a_k
+ \sum_{k\in M\setminus A}\underbrace{1_A(k)}_{0} a_k
= \sum_{k\in A} a_k.\,\qedsymbol\]
\end{Beweis}

\begin{Satz} Allgemein gilt
\[\sum_{k\in A\cup B} a_k = \sum_{k\in A} a_k + \sum_{k\in B} a_k
- \sum_{k\in A\cap B} a_k.\]
\end{Satz}
\begin{Beweis} Sei $G=A\cup B$ die Grundmenge. Gemäß Satz
\ref{indicator-set-op} darf man rechnen
\begin{align*}
\sum_{k\in G} a_k &= \sum_{k\in G} 1_{A\cup B}(k)a_k
= \sum_{k\in G} 1_A(k)a_k + \sum_{k\in G} 1_B(k)a_k
- \sum_{k\in G} 1_{A\cap B}(k)a_k\\
&= \sum_{k\in A} a_k + \sum_{k\in B} a_k - \sum_{k\in A\cap B} a_k.\,\qedsymbol
\end{align*}
\end{Beweis}

\begin{Definition}[Differenzenfolge]\index{Differenzenfolge}
Zu einer Folge $(a_k)$ definiert man
\[(\Delta a)_k := a_{k+1} - a_k.\]
\end{Definition}

\begin{Satz}[Teleskopsumme]\label{sum-tele} Es gilt
\[\sum_{k=m}^{n-1} (\Delta a)_k = \sum_{k=m}^{n-1} (a_{k+1} - a_k) = a_n - a_m.\]
\end{Satz}
\begin{Beweis}[Beweis 1] Induktion über $n$. Im Anfang $n=m$ haben beide Seiten
der Gleichung den Wert null. Induktionsschritt:
\[\sum_{k=m}^n (a_{k+1} - a_k) = (a_{n+1} - a_n) + \!\!\sum_{k=m}^{n-1} (a_{k+1} - a_n)
\stackrel{\mathrm{IV}}= a_{n+1} - a_n + a_n - a_m = a_{n+1} - a_m.\,\qedsymbol\]
\end{Beweis}
\begin{Beweis}[Beweis 2]
Per Indexshift \ref{sum-indexshift}
gilt $\sum\limits_{k=m}^{n-1} a_{k+1} = \!\!\sum\limits_{k=m+1}^n\!\! a_k
= a_n - a_m + \sum\limits_{k=m}^{n-1} a_k$.
Somit ist%
\[\sum_{k=m}^{n-1} (a_{k+1} - a_k) = \sum_{k=m}^{n-1} a_{k+1} - \sum_{k=m}^{n-1} a_k
= a_n - a_m + \sum_{k=m}^{n-1} a_k - \sum_{k=m}^{n-1} a_k = a_n - a_m.\,\qedsymbol\]
\end{Beweis}

\begin{Satz}
Zum Beweis einer Formel
\[\sum_{k=m}^{n-1} a_k = s_n\]
genügt es, $s_m=0$ und $(\Delta s)_n = a_n$ zu zeigen.
\end{Satz}
\begin{Beweis}[Beweis 1]
Induktion über $n$. Im Anfang $n=m$ haben beide Seiten der Gleichung laut
der Prämisse den Wert null. Induktionsschritt:
\[\sum_{k=m}^n a_k = a_n + \sum_{k=m}^{n-1} a_k
\stackrel{\mathrm{IV}}= a_n + s_n
= (\Delta s)_n + s_n = s_{n+1} - s_n + s_n = s_{n+1}.\,\qedsymbol\]
\end{Beweis}
\begin{Beweis}[Beweis 2]
Spezialisierung von Satz \ref{sum-tele}.\,\qedsymbol
\end{Beweis}

\begin{Satz} Der Differenzoperator ist linear. Das heißt,
für alle Folgen $(a_n), (b_n)$ und jede Konstante $c$ gilt
\begin{align*}
& \Delta(a+b) = \Delta a + \Delta b, && ((a+b)_n := a_n + b_n)\\
& \Delta(ca) = c\Delta a. && ((ca)_n := ca_n)
\end{align*}
\end{Satz}
\begin{Beweis} Man findet
\begin{align*}
(\Delta(a+b))_n &= (a+b)_{n+1} - (a+b)_n
= (a_{n+1}+b_{n+1}) - (a_n+b_n)\\
&= a_{n+1}-a_n + b_{n+1}-b_n = (\Delta a)_n + (\Delta b)_n
= (\Delta a + \Delta b)_n
\end{align*}
und
\[(\Delta(ca))_n = (ca)_{n+1} - (ca)_n = ca_{n+1} - ca_n
= c(a_{n+1} - a_n) = c(\Delta a)_n = (c\Delta a)_n.\,\qedsymbol\]
\end{Beweis}

\begin{Definition}[Shiftoperator] Man definiert
\[(Ta)_n := a_{n+1}.\]
\end{Definition}

\begin{Satz}[Iterierter Differenzoperator]\newlinefirst
Für jede Folge $(a_n)$ und $m\in\Z_{\ge 0}$ gilt
\[(\Delta^m a)_n = (-1)^m\sum_{k=0}^m\binom{m}{k} (-1)^k a_{n+k}.\]
\end{Satz}
\begin{Beweis} Es gilt $\Delta = T-\id$. Weil $T$ und $\id$ kommutieren,
ist der binomische Lehrsatz anwendbar. Es ergibt sich
\[\Delta^m = (T-\id)^m = \sum_{k=0}^m\binom{m}{k} (-1)^{m-k} T^k\id^{m-k}
= (-1)^m \sum_{k=0}^m\binom{m}{k} (-1)^k T^k.\,\qedsymbol\]
\end{Beweis}

\begin{Satz}\label{delta-deg}
Sei $f$ eine Polynomfunktion. Dann ist $\Delta_h f$ eine
Polynomfunktion mit niedrigerem Grad.
\end{Satz}
\begin{Beweis} Für $f(x)=\sum_{n=0}^m a_n x^n$ gilt
\begin{gather*}
\Delta_h f(x) = f(x+h) - f(x) = \sum_{n=0}^m a_n (x+h)^n - \sum_{n=0}^m a_n x^n
= \sum_{n=0}^m a_n ((x+h)^n - x^n)\\
= \sum_{n=0}^m a_n (x^n + \sum_{k=0}^{n-1}\binom{n}{k}x^k h^{n-k} - x^n)
= \sum_{n=0}^m a_n \sum_{k=0}^{n-1}\binom{n}{k}x^k h^{n-k}.
\end{gather*}
In der Summe treten nur Monome bis $x^{m-1}$ auf.\,\qedsymbol
\end{Beweis}

\begin{Satz} Sei $f$ ein Polynom vom Grad $N$. Für $n,a\in\Z$ und $n\ge a$ gilt
\[f(n) = \sum_{k=0}^N \frac{(\Delta^k f)(a)}{k!}(n-a)^{\underline k}
= \sum_{k=0}^N \binom{n-a}{k}(\Delta^k f)(a).\]
\end{Satz}
\begin{Beweis}
Es gilt $T=\Delta+\id$. Für jede nichtnegative ganze Zahl $m$ gilt
\[T^m = (\Delta+\id)^m = \sum_{k=0}^m\binom{m}{k}\Delta^k\]
mit dem binomischen Lehrsatz, da $\Delta$ und $\id$ kommutieren. Das macht
\[f(a + m) = \sum_{k=0}^m\binom{m}{k}(\Delta^k f)(a).\]
Man substituiere nun $n = a+m$. Für $n\ge a$ gilt dann
\[f(n) = \sum_{k=0}^{n-a}\binom{n-a}{k}(\Delta^k f)(a)
= \sum_{k=0}^N\binom{n-a}{k}(\Delta^k f)(a).\]
Der Indexbereich der Summierung durfte auf bis $k=N$ geändert werden, weil
$\Delta^k f = 0$ für $k>N$ laut Satz \ref{delta-deg} gilt. Dass nun
Summanden mit $k>n-a$ auftreten können, ist nicht weiter schlimm, weil
in diesem Fall $\binom{n-a}{k}=0$ ist.\,\qedsymbol
\end{Beweis}

\newpage
\subsection{Klassische Partialsummen}

\begin{Satz}[Partialsummen der konstanten Folge]%
\label{sum-const}\newlinefirst
Es gilt $\displaystyle\sum_{k=m}^n 1 = n-m+1$.
\end{Satz}
\begin{Beweis}
Induktion über $n$. Im Anfang $n=m-1$ haben beide Seiten
der Gleichung den Wert null. Induktionsschritt:
\[\sum_{k=m}^n 1 = 1 + \sum_{k=m}^{n-1}
\stackrel{\mathrm{IV}}= 1 + n-1-m+1 = n-m+1.\,\qedsymbol\]
\end{Beweis}

\begin{Satz}[Partialsummen der arithmetischen Folge]%
\index{arithmetische Folge}\newlinefirst
Es gilt $\displaystyle\sum_{k=0}^n k = \frac{n}{2}(n+1)$.
\end{Satz}
\begin{Beweis}[Beweis 1]
Induktion über $n$. Im Anfang $n=-1$ haben beide Seiten der Gleichung
den Wert null. Induktionsschritt:
\[\sum_{k=0}^n k = n + \sum_{k=0}^{n-1} k
\stackrel{\mathrm{IV}}= n + \frac{n-1}{2}(n-1+1)
= \frac{n}{2}(2 + n-1) = \frac{n}{2}(n+1).\,\qedsymbol\]
\end{Beweis}
\begin{Beweis}[Beweis 2]
Klassischer Beweis. Man findet die Umformung
\[2\!\sum_{k=0}^n k = \!\sum_{k=0}^n k + \!\sum_{k=0}^n k
\stackrel{\text{(1)}}= \!\sum_{k=0}^n k + \!\sum_{k=0}^n (n-k)
\stackrel{\text{(2)}}= \!\sum_{k=0}^n (k+n-k)
= \!\sum_{k=0}^n n \stackrel{\text{(3)}}= n\!\sum_{k=0}^n 1
\stackrel{\text{(4)}}= n(n+1),\]
wobei (1), (2), (3), (4) gemäß Satz
\ref{sum-rev}, \ref{sum-add}, \ref{sum-scale}, \ref{sum-const}
gelten.\,\qedsymbol
\end{Beweis}

\begin{Satz}[Partialsummen der geometrischen Folge]%
\label{sum-geom-seq}\index{geometrische Folge}\newlinefirst
Für $m\ge 0$ und $z\in\C\setminus\{1\}$ gilt
$\displaystyle\sum_{k=m}^{n-1} z^k = \frac{z^n-z^m}{z-1}$.
\end{Satz}
\begin{Beweis}
Induktion über $n$. Im Anfang $n=m-1$ haben beiden Seiten der Gleichung
den Wert null. Induktionsschritt:
\[\sum_{k=m}^n z^k = z^n + \sum_{k=m}^{n-1} z^k
\stackrel{\mathrm{IV}}= z^n + \frac{z^n-z^m}{z-1}
= \frac{(z-1)z^n+z^n-z^m}{z-1}
= \frac{z^{n+1}-z^m}{z-1}.\,\qedsymbol\]
\end{Beweis}

\begin{Satz}
Für $m\ge 0$ und $z\in\C\setminus\{1\}$ gilt
\[\sum_{k=m}^{n-1} kz^k
= \frac{(nz^n-mz^m)(z-1) - (z^n-z^m)z}{(z-1)^2}.\]
\end{Satz}
\begin{Beweis}
Die Gleichung von Satz \ref{sum-geom-seq} für $m\ge 1$ auf beiden
Seiten nach $z$ ableiten und anschließend beide Seiten mit $z$
multiplizieren. Den Fall $m=0$ und in diesem den Summand zu $k=0$
explizit betrachten, sonst aber auf dieselbe Weise vorgehen.\,\qedsymbol
\end{Beweis}

\newpage
\begin{Satz} Es gilt
\[\sum_{k=1}^n (-1)^k k = (-1)^n\left\lfloor\frac{n+1}{2}\right\rfloor.\]
\end{Satz}
\begin{Beweis}
Induktion über $n$. Im Anfang $n=0$ haben beiden Seiten den Wert null.

Induktionsschritt:
\[\sum_{k=1}^n (-1)^k k = (-1)^n n + \sum_{k=1}^{n-1} (-1)^k k
\stackrel{\mathrm{IV}}= (-1)^n n + (-1)^{n-1}\left\lfloor\frac{n}{2}\right\rfloor
= (-1)^n (n-\left\lfloor\frac{n}{2}\right\rfloor).\]
Zu zeigen verbleibt die Gleichung
\[n-\left\lfloor\frac{n}{2}\right\rfloor = \left\lfloor\frac{n+1}{2}\right\rfloor
\iff n = \left\lfloor\frac{n}{2}\right\rfloor + 
\left\lfloor\frac{n+1}{2}\right\rfloor.\]
Wir nehmen die Fallunterscheidung zwischen geraden und ungeraden
Zahlen vor, um Satz \ref{floor-add-int} und \ref{floor-is-zero}
nutzen zu können. Im geraden Fall $n=2k$ bestätigt sich
\[\left\lfloor\frac{2k}{2}\right\rfloor +  \left\lfloor\frac{2k+1}{2}\right\rfloor
= \lfloor k\rfloor + \left\lfloor k + \frac{1}{2}\right\rfloor = k + k = 2k.\]
Im ungeraden Fall $n=2k+1$ bestätigt sich
\[\left\lfloor\frac{2k+1}{2}\right\rfloor + \left\lfloor\frac{2k+1+1}{2}\right\rfloor
= \left\lfloor k + \frac{1}{2}\right\rfloor + \left\lfloor k+1\right\rfloor
= k + k + 1 = 2k + 1.\,\qedsymbol\]
\end{Beweis}

\section{Funktionen}

\subsection{Floor und Ceil}

\begin{Definition}[Floorfunktion]%
\label{def:floor}\index{Floorfunktion}
Für $x\in\R$ definiert man
\[y = \lfloor x\rfloor\defiff y\in\Z\land 0\le x-y < 1.\]
\end{Definition}

\begin{Definition}[Ceilfunktion]%
\label{def:ceil}\index{Ceilfunktion}
Für $x\in\R$ definiert man
\[y = \lceil x\rceil\defiff y\in\Z\land 0\le y-x < 1.\]
\end{Definition}

\begin{Satz}\label{floor-add-int}
Für jede ganze Zahl $k$ gilt $\lfloor k + x\rfloor = k + \lfloor x\rfloor$.
\end{Satz}
\begin{Beweis} Aufgrund der Prämisse $k\in\Z$ ist $y\in\Z$ äquivalent
zu $y-k\in\Z$. Unter dieser Gegebenheit findet sich mit
Def. \ref{def:floor} die äquivalente Umformung
\begin{align*}
y = \lfloor k+x\rfloor &\iff y\in\Z\land 0\le (k+x)-y < 1\iff y-k\in\Z\land 0\le x-(y-k) < 1\\
&\iff y - k = \lfloor x\rfloor \iff y = k + \lfloor x\rfloor.\,\qedsymbol
\end{align*}
\end{Beweis}

\begin{Satz}\label{floor-is-zero}
Für $0\le x < 1$ gilt $\lfloor x\rfloor = 0$.
\end{Satz}
\begin{Beweis}
Dies folgt unmittelbar aus Def. \ref{def:floor}.\,\qedsymbol
\end{Beweis}

\newpage
\subsection{Faktorielle}

\begin{Definition}[Fakultät]%
\label{def:factorial}\index{Fakultaet@Fakultät}\newlinefirst
Für eine nichtnegative ganze Zahl $n$ definiert man $n!$ rekursiv durch
\[0! := 1,\quad (n+1)! := (n+1)n!.\]
\end{Definition}

\begin{Definition}[Fallende Faktorielle]\label{def:falling-factorial}%
\index{Faktorielle!fallende}\index{fallende Faktorielle}\newlinefirst
Für $k\in\Z_{\ge 0}$ und $n\in\Z$ (oder allgemeiner $n\in\C$)
definiert man $n^{\underline k}$ rekursiv durch
\[n^{\underline 0} := 1,\quad n^{\underline {k+1}}:=n(n-1)^{\underline k}.\]
\end{Definition}

\begin{Definition}[Steigende Faktorielle]\label{def:raising-factorial}%
\index{Faktorielle!steigende}\index{steigende Faktorielle}\newlinefirst
Für $k\in\Z_{\ge 0}$ und $n\in\Z$ (oder allgemeiner $n\in\C$)
definiert man $n^{\overline k}$ rekursiv durch
\[n^{\overline 0} := 1,\quad n^{\overline {k+1}}:=n(n+1)^{\overline k}.\]
\end{Definition}

\begin{Satz}\label{relation-ff-factorial}
Für $n,k\in\Z_{\ge 0}$ und $k\le n$ gilt
\[n^{\underline k} = \frac{n!}{(n-k)!}.\]
\end{Satz}
\begin{Beweis}
Induktion über $k$. Im Anfang $k=0$ resultieren beide Seiten der
Gleichung im gleichen Wert~1. Der Induktionsschritt ist
\[n^{\underline k} = n(n-1)^{\underline{k-1}}
\stackrel{\text{IV}}= n\frac{(n-1)!}{((n-1)-(k-1))!}
= \frac{n(n-1)!}{(n-k)!}
= \frac{n!}{(n-1)!}.\,\qedsymbol\]
\end{Beweis}

\begin{Satz}
Für ganze Zahlen $n,k$ mit $n\ge 1$ und $k\ge 1-n$ gilt
\[n^{\overline k} = \frac{(n+k-1)!}{(n-1)!}.\]
\end{Satz}
\begin{Beweis}
Induktion über $k$. Im Anfang $k=0$ resultieren beide Seiten der
Gleichung im gleichen Wert~1. Der Induktionsschritt ist
\begin{align*}
n^{\overline k} &= n(n+1)^{\overline{k-1}}
\stackrel{\text{IV}}= n\frac{(n+1+k-1-1)!}{(n+1-1)!}
= \frac{n(n+k-1)!}{n!}\\
&= \frac{n(n+k-1)!}{n(n-1)!}
= \frac{(n+k-1)!}{(n-1)!}.\,\qedsymbol
\end{align*}
\end{Beweis}

\begin{Satz}
Für jedes $n\in\Z_{\ge 0}$ gilt $n!\le n^n$.
\end{Satz}
\begin{Beweis}
Induktion über $n$. Im Induktionsanfang $n=0$ hat man $0! = 1$ und $0^0=1$.
Zum Induktionsschritt unternimmt man zunächst die äquivalente Umformung
\[(n+1)!\le (n+1)^{n+1} \iff (n+1)n!\le (n+1)(n+1)^n
\iff n!\le (n+1)^n.\]
Diese Ungleichung bestätigt die Rechnung
\[(n+1)^n = \sum_{k=0}^n\binom{n}{k}n^k =
n^n+\sum_{k=0}^{n-1}\binom{n}{k}n^k\ge n^n
\stackrel{\mathrm{IV}}\ge n!.\,\qedsymbol\]
\end{Beweis}

\newpage
\subsection{Binomialkoeffizient}

\begin{Definition}[Binomialkoeffizient]%
\label{def:binom}\index{Binomialkoeffizient}\newlinefirst
Für $k\in\Z_{\ge 0}$ und $n\in\Z$ (oder allgemeiner $n\in\C$)
definiert man
\[\binom{n}{k} := \frac{n^{\underline k}}{k!}.\]
\end{Definition}

\begin{Satz}
Für $n\in\Z$ mit $k\le n$ gilt
\[\binom{n}{k} = \frac{n!}{k!(n-k)!}\]
\end{Satz}
\begin{Beweis}
Folgt direkt aus Def. \ref{def:binom} und Satz
\ref{relation-ff-factorial}.\,\qedsymbol
\end{Beweis}

\begin{Satz}
Für $k\ge 1$ und $n\in\Z$ (oder allgemeiner $n\in\C$) gilt
\[\binom{n}{k} = \frac{n}{k}\binom{n-1}{k-1}.\]
\end{Satz}
\begin{Beweis}
Es findet sich die Umformung
\[\binom{n}{k} = \frac{n^{\underline k}}{k!}
= \frac{n(n-1)^{\underline {k-1}}}{k(k-1)!}
= \frac{n}{k}\binom{n-1}{k-1}.\,\qedsymbol\]
\end{Beweis}

\begin{Satz}
Für $k\ge 1$ und $n\in\Z$ (oder allgemeiner $n\in\C$) gilt
\[\binom{n}{k} = \binom{n-1}{k-1} + \binom{n-1}{k}.\]
\end{Satz}
\begin{Beweis}
Es findet sich die Umformung
\begin{align*}
\binom{n-1}{k-1} + \binom{n-1}{k}
&= \frac{(n-1)^{\underline{k-1}}}{(k-1)!} + \frac{(n-1)^{\underline k}}{k!}
= \frac{k(n-1)^{\underline{k-1}}}{k!} + \frac{(n-1)^{\underline{k-1}}(n-k)}{k!}\\
&= \frac{(n-1)^{\underline{k-1}}}{k!}(k + n -k)
= \frac{n(n-1)^{\underline{k-1}}}{k!}
= \frac{n^{\underline k}}{k!} = \binom{n}{k}.\,\qedsymbol
\end{align*}
\end{Beweis}


\chapter{Algebra}

\section{Gruppentheorie}

\subsection{Grundlagen}

\begin{Definition}[Gruppe]
Das Tupel $(G,*)$ bestehend aus einer Menge $G$ und
Abbildung $*\colon G\times G\to\Omega$ heißt Gruppe, wenn die folgenden
Axiome erfüllt sind:
\begin{enumerate}
\item[(G1)] Für alle $a,b\in G$ gilt $a*b\in G$. D.\,h., man darf $G=\Omega$ setzen.
\item[(G2)] Es gilt das Assoziativgesetz: für alle $a,b,c\in G$ gilt $(a*b)*c=a*(b*c)$.
\item[(G3)] Es gibt ein Element $e\in G$, so dass $e*g=g=g*e$ für jedes $g\in G$ gilt.
\item[(G4)] Zu jedem $g\in G$ gibt es ein $g^{-1}\in G$ so dass $g*g^{-1}=e=g^{-1}*g$ gilt.
\end{enumerate}
Das Element $e$ wird neutrales Element der Gruppe genannt.
Das Element $g^{-1}$ wird inverses Element zu $g$ genannt.
Anstelle von $a*b$ schreibt man auch kurz $ab$. Ist $(G,+)$ eine
Gruppe, dann schreibt man immer $a+b$, und $-g$ anstelle von $g^{-1}$.
\end{Definition}

\begin{Korollar}
Das neutrale Element einer Gruppe $G$ ist eindeutig bestimmt.
D.\,h., es gibt keine zwei unterschiedlichen neutralen Elemente. 
\end{Korollar}
\begin{Beweis}
Seien $e,e'$ zwei neutrale Elemente von $G$. Nach Axiom (G3)
gilt dann $e=e'e$, und weiter $e'e=e'$ bei nochmaliger Anwendung
von (G3). Daher ist $e=e'$.\;\qedsymbol
\end{Beweis}

\begin{Korollar}
Sei $G$ eine Gruppe. Zu jedem Element $g\in G$ ist das inverse
Element $g^{-1}$ eindeutig bestimmt. D.\,h., es kann keine zwei
unterschiedlichen inversen Elemente zu $g$ geben.
\end{Korollar}
\begin{Beweis}
Seien $a,b$ zwei inverse Elemente zu $g$. Nach Axiom (G3), Axiom (G2)
und Axiom (G4) gilt
\[a \stackrel{(G3)}= ae \stackrel{(G4)} = a(gb) \stackrel{(G2)}
= (ag)b \stackrel{(G4)}= eb \stackrel{(G3)}= b.\]
Daher ist $a=b$.\;\qedsymbol
\end{Beweis}

\begin{Definition}[Untergruppe]
Sei $(G,*)$ eine Gruppe. Eine Teilmenge $U\subseteq G$ heißt
Untergruppe von $G$, kurz $U\le G$, wenn $U$ bezüglich derselben
Verknüpfung $*$ selbst eine Gruppe $(U,*)$ bildet.
\end{Definition}

\begin{Korollar}
Jede Gruppe $G$ besitzt die Untergruppen $\{e\}\le G$ und $G\le G$,
wobei $e\in G$ das neutrale Element ist. Man spricht von den
trivialen Untergruppen.
\end{Korollar}
\begin{Beweis}
Die Aussage $G\le G$ ist trivial, denn $G\subseteq G$ ist allgemeingültig
und $(G,*)$ bildet nach Voraussetzung eine Gruppe. Zu (G1):
Es gilt $ee=e$. Da es nur diese eine Möglichkeit gibt, sind damit alle
überprüft.
Zu (G2): Das Assoziativgesetz wird auf Elemente der Teilmenge vererbt.
Zu (G3): Das neutrale Element ist in $\{e\}$ enthalten.
Zu (G4): Das neutrale Element ist gemäß $ee=e$ zu sich selbst invers.
Da $e$ das einzige Element von $\{e\}$ ist, sind damit alle
überprüft.\;\qedsymbol
\end{Beweis}

\section{Ringtheorie}

\subsection{Grundlagen}

\begin{Definition}[Ring]
Eine Struktur $(R,+,\cdot)$ heißt genau dann Ring, wenn die folgenden
Axiome erfüllt sind
\begin{enumerate}
\item[1.] $(R,+)$ ist eine kommutative Gruppe.
\item[2.] $(R,\cdot)$ ist eine Halbgruppe.
\item[3.] Für alle $a,b,c\in R$ gilt $a(b+c) = ab+ac$. (Linksdistributivgesetz)
\item[4.] Für alle $a,b,c\in R$ gilt $(a+b)c = ac+bc$. (Rechtsdistributivgesetz)
\end{enumerate}
\end{Definition}
Bemerkung: Das neutrale Element von $(R,+)$ wird als Nullelement
bezeichnet und meist $0$ geschrieben.

\begin{Definition}[Ring mit Eins]
Ein Ring $R$ heißt genau dann Ring mit Eins, wenn $(R,\cdot)$ ein
Monoid ist. Monoid heißt, es gibt ein Element $e\in R$, so dass
$e\cdot a = a$ und $a\cdot e = a$ für alle $a\in R$.
\end{Definition}
Bemerkung: Man bezeichnet $e$ als Einselement des Rings.

\begin{Korollar}
Sei $R$ ein Ring und $0\in R$ das Nullelement.\\
Für jedes $a\in R$ gilt $0\cdot a = 0$ und $a\cdot 0 = 0$.
\end{Korollar}
\begin{Beweis} Man rechnet
\[0a = 0a+0 = 0a+0a-0a = (0+0)a-0a = 0a-0a = 0.\]
\end{Beweis}
Die Rechnung für $a\cdot 0$ ist analog.\;\qedsymbol

\begin{Korollar}\label{Minus-vorziehen}
Sei $R$ ein Ring und $a,b\in R$, dann gilt $(-a)b = -(ab) = a(-b)$.
\end{Korollar}
\begin{Beweis}
Man rechnet
\begin{align*}
(-a)b &= (-a)b+0 = (-a)b+ab-(ab) = ((-a)+a)b-(ab)\\
&= 0b-(ab) = 0-(ab) = -(ab).\;\qedsymbol
\end{align*}
\end{Beweis}

\begin{Korollar}[»Minus mal minus macht plus«]\mbox{}\\*
Sei $R$ ein Ring und $a,b\in R$, dann gilt
$(-a)(-b) = ab$.
\end{Korollar}
Beachtung von $-(-x)=x$ nach zweifacher Anwendung von
Korollar \ref{Minus-vorziehen} bringt
\[(-a)(-b) = -((-a)b) = -(-(ab)) = ab.\;\qedsymbol\]

\subsection{Ringhomomorphismen}

\begin{Definition}[Ringhomomorphismus]
Seien $R,R'$ Ringe. Eine Abbildung $\varphi\colon R\to R'$ heißt
Ringhomomorphismus, falls
\begin{gather*}
\varphi(x+y) = \varphi(x)+\varphi(y),\\
\varphi(xy) = \varphi(x)\varphi(y)
\end{gather*}
für alle $x,y\in R$ gilt. Liegen Ringe mit Eins vor, und gilt
zusätzlich $\varphi(1)=1$, dann spricht man von einem unitären
Ringhomomorphismus.
\end{Definition}

\begin{Korollar}
Bei jedem Ringhomomorphismus $\varphi$ gilt
$\varphi(kx) = k\varphi(x)$ für $k\in\Z$.
\end{Korollar}
\strong{Beweis.} Für $k>0$ ist
\[\varphi(kx) = \varphi(\sum_{i=1}^k x)
= \sum_{i=1}^k \varphi(x) = k\varphi(x).\]
Nun der Fall $k=0$. Man rechnet $f(0) = f(0+0) = f(0)+f(0)$.
Subtraktion von $f(0)$ auf beiden Seiten ergibt $f(0)=0$.
Schließlich bleibt noch $f(-kx)=-kf(x)$ für $k>0$ zu zeigen. Hier
rechnet man zunächst
\[0 = f(0) = f(-x+x) = f(-x)+f(x).\]
Subtraktion von $f(x)$ auf beiden Seiten ergibt $f(-x) = -f(x)$.
Somit gilt
\[f(-kx) = -f(kx) = -kf(x).\;\qedsymbol\]

\section{Polynomringe}

\subsection{Einsetzungshomomorphismus}

\begin{Satz}
Die Abbildung $\Phi\colon\R[X]\to\Abb(\R,\R)$ mit $\Phi(f)(x):=f(x)$
ist injektiv.
\end{Satz}
\begin{Beweis} Sei $f=\sum_{k=0}^n a_k X^k$
und $g=\sum_{k=0}^n b_k X^k$, wobei $n=\max(\deg f,\deg g)$.
Zu zeigen ist
\[(\forall x\in\R\colon \Phi(f)(x)=\Phi(g)(x))\implies f=g,\]
das heißt
\[(\forall x\in\R\colon \sum_k a_k x^k = \sum_k b_k x^k)\implies (\forall k\colon a_k=b_k).\]
Die Umformung der Voraussetzung ergibt $\sum_k (b_k-a_k)x^k = 0$.
D.\,h., jedes der $(b_k-a_k)$ muss verschwinden. Zu zeigen ist also
lediglich
\[(\forall x\in\R\colon \sum_{k=0}^n c_k x^k = 0)\implies (\forall k\colon c_k=0).\]
Wenn $f(x)=0$ für alle $x$ ist, muss auch die Ableitung $D^m f(x)=0$
sein. Es gilt $D^k x^k = k!$, und daher
\[D^n\sum_{k=0}^n c_k x^k = n!\cdot c_n = 0 \implies c_n=0.\]
Demnach ergibt sich dann aber auch
\[D^{n-1}\sum_{k=0}^n c_k x^k = (n-1)!\cdot c_{n-1} = 0\implies c_{n-1}=0\]
usw. Man erhält $c_k=0$ für alle $k$.\;\qedsymbol
\end{Beweis}


\chapter{Wahrscheinlichkeitsrechnung}

\section{Diskrete Verteilungen}

\subsection{Diskreter Wahrscheinlichkeitsraum}

\begin{definition}[Ergebnis, Ereignis, Ergebnismenge, Ereignisraum,
unmögliches Ereignis, sicheres Ereignis]%
\index{Ergebnismenge}\index{Ereignisraum}%
\index{sicheres Ereignis}\index{unmögliches Ereignis}
Eine abzählbare \emdef{Ergebnismenge} $\Omega$ ist eine endliche
(oder abzählbar unendliche) Menge, die als Grundmenge verwendet wird.
Ein Element von $\Omega$ heißt \emdef{Ergebnis} oder
\emdef{Elementarereignis}.

Die Potenzmenge $2^\Omega$ heißt \emdef{Ereignisraum}, die
Elemente heißen \emdef{Ereignisse}.
Man nennt die leere Menge $\emptyset$ das \emdef{unmögliche} und $\Omega$
das \emdef{sichere} Ereignis.
\end{definition}

\begin{definition}[Diskreter Wahrscheinlichkeitsraum, Wahrscheinlichkeitsmaß]%
\index{Wahrscheinlichkeitsraum!diskreter}\index{Wahrscheinlichkeitsmaß!diskretes}%
\index{Verteilung!diskrete Wahrscheinlichkeitsverteilung}
Ein Paar $(\Omega,P)$ heißt \emdef{diskreter Wahrscheinlichkeitsraum}, wenn
$\Omega$ eine abzählbare Ergebnismenge ist und%
\begin{equation}
P(A):=\sum_{\omega\in A} P(\{\omega\}),\quad P\colon 2^\Omega\to [0,1]
\end{equation}
die Eigenschaft
\begin{equation}
\sum_{\omega\in\Omega} P(\{\omega\})=1
\end{equation}
besitzt. Die Abbildung $P$ heißt (das von den
Einzelwahrscheinlichkeiten induzierte) \emdef{Wahrscheinlichkeitsmaß}.
Man spricht auch von einer \emdef{Verteilung} auf $\Omega$.
\end{definition}


\subsection{Axiome von Kolmogorow}
\begin{definition}[Wahrscheinlichkeitsmaß (Axiome von Kolmogorow)]%
\index{Wahrscheinlichkeitsmaß!Axiome von Kolmogorow}%
\index{Axiome von Kolmogorow}
Gegeben ist ein Messraum $(\Omega,\Sigma)$. Man nennt $P$ ein
\emdef{Wahrscheinlichkeitsmaß}, wenn gilt:

1. $P$ ist eine Funktion $P\colon\Sigma\to [0,1]$.

2. $P(\Omega)=1$.

3. Ist $I$ eine abzählbare Indexmenge und sind die $A_i$
für $i\in I$ paarweise disjunkte Ereignisse, so gilt
\begin{equation}
P\Big(\bigcup_{i\in I} A_i\Big) = \sum_{i\in I}P(A_i).
\end{equation}
\end{definition}

\noindent
Bei einem diskreten Wahrscheinlichkeitsraum $(\Omega,P)$ mit
$\Sigma=2^\Omega$ sind die Axiome erfüllt.

\subsection{Rechenregeln}
Aus den Axiomen von Kolmogorow folgen folgende
Rechenregeln für ein Wahrscheinlichkeitsmaß $P$:
\begin{gather}
P(\emptyset) = 0,\\
P(\Omega) = 1,\\
P(A\cup B) = P(A)+P(B)-P(A\cap B).
\end{gather}
Man nennt $A^\comp:=\Omega\setminus A$ das
\emdef{komplementäre Ereignis}\index{komplementäres Ereignis}
zu $A$. Es gilt:
\begin{gather}
A\cup A^\comp = \Omega,\\
A\cap A^\comp = \emptyset,\\
P(A\cup A^\comp) = P(A)+P(A^\comp) = 1.
\end{gather}
\strong{Mehrstufige Experimente.}
Ein zweistufiges Zufallsexperiment mit einem ersten Ergebnis aus
$\Omega_1$ und einem zweiten aus $\Omega_2$ lässt sich als
Zufallsexperiment modellieren, bei dem die Ergebnismenge das
kartesische Produkt $\Omega=\Omega_1\times\Omega_2$ ist. Bei einem
$n$-stufigen Experiment gilt
\begin{equation}
\Omega = \Omega_1\times\ldots\times\Omega_n.
\end{equation}
Erste Pfadregel: Sei $a\in\Omega_1$, $b\in\Omega_2$, $A=\{a\}\times\Omega_2$
und $B=\Omega_1\times\{b\}$. Es gilt
\begin{equation}
P(\{(a,b)\}) = P(A\cap B) = P(A)\,P(B|A).
\end{equation}
Das Ereignis $\{(a,b)\}$ tritt ein, wenn zuerst
der Pfad $A$ eingetreten ist, und dann auch der Pfad $B$.
Die Wahrscheinlichkeit ist das Produkt der Pfadwahrscheinlichkeiten
$P(A)$ und $P(B|A)$.

Zweite Pfadregel: Sind $a,b\in\Omega$ zwei unterschiedliche
Ergebnisse, dann gilt
\begin{equation}
P(\{a\}\cup\{b\}) = P(\{a\})+P(\{b\}).
\end{equation}
Wenn die Teilexperimente eines mehrstufigen Experiments
stochastisch unabhängig sind, dann gilt nach der ersten Pfadregel
die Formel
\begin{equation}
P(\{(a_1,\ldots,a_n)\}) = \prod_{k=1}^n P(A_k),
\end{equation}
wobei $A_k$ der Pfad zu $a_k$ ist.
Für den Fall, dass die einzelnen
Experimente alle Laplace-Experimente sind, gilt speziell
\begin{equation}
P(\{(a_1,\ldots,a_n)\}) = \frac{1}{|\Omega|} = \prod_{k=1}^n \frac{1}{|\Omega_k|}
\end{equation}
mit $\Omega=\Omega_1\times\ldots\times\Omega_n$ und $(a_1,\ldots,a_n)\in\Omega$.

Führt man immer wieder das selbe Laplace-Experiment aus, gilt
mit $t\in\Omega$ und $\Omega=\Omega_1^n$ die Regel
\begin{equation}
P(t) = \frac{1}{|\Omega|} = \frac{1}{|\Omega_1|^n}.
\end{equation}
Würfelt man z.\,B. $n$-mal hintereinander, dann gibt es $6^n$ Pfade
und für jeden Pfad ergibt sich eine Wahrscheinlichkeit von $(1/6)^n$.

\subsection{Bedingte Wahrscheinlichkeit}
\begin{definition}[Bedingte Wahrscheinlichkeit]%
\index{bedingte Wahrscheinlichkeit}
Für zwei Ereignisse $A,B$ mit $P(B)>0$ nennt man
\begin{equation}
P(A|B) := \frac{P(A\cap B)}{P(B)}
\end{equation}
die \emdef{bedingte Wahrscheinlichkeit} von $A$, vorausgesetzt $B$.
\end{definition}
Bei
\begin{equation}
P'(A) := P(A|B),\quad P'\colon 2^B\to [0,1]
\end{equation}
handelt es sich wieder um ein Wahrscheinlichkeitsmaß.

\strong{Satz von Bayes.} Für $P(A)>0$ und $P(B)>0$ gilt
\begin{equation}
P(A|B) = \frac{P(B|A)\, P(A)}{P(B)}.
\end{equation}

\subsection{Unabhängige Ereignisse}
\begin{definition}[Stochastische Unabhängigkeit]%
\index{stochastisch unabhängig}
Zwei Ereignisse $A,B$ heißen \emdef{stochastisch unabhängig}, wenn
\begin{equation}
P(A\cap B) = P(A)\, P(B)
\end{equation}
gilt.
\end{definition}

\subsection{Gleichverteilung}
\begin{definition}[Gleichverteilung (Laplace-Verteilung)]%
\index{Gleichverteilung}\index{Laplace-Verteilung}
Sei $\Omega$ eine endliche Ergebnismenge. Mann nennt $P$ eine
\emdef{Gleichverteilung} oder \emdef{Laplace-Verteilung}, wenn
\begin{equation}
P(\{\omega\}) = \frac{1}{|\Omega|}
\end{equation}
für alle Ergebnisse $\omega\in\Omega$ gilt.
\end{definition}

Für eine Gleichverteilung gilt
\begin{equation}
P(A) = \frac{|A|}{|\Omega|}.
\end{equation}

\subsection{Zufallsvariablen}
\begin{definition}[Zufallsvariable]
Sei $(\Omega,P)$ ein diskreter Wahrscheinlichkeitsraum. Jede Funktion
\begin{equation}
X\colon\Omega\to\R
\end{equation}
heißt \emdef{Zufallsvariable}. Die Funktionswerte $x=X(\omega)$ heißen
\emdef{Realisationen} der Zufallsvariable.
\end{definition}

Eine Zufallsvariable $X$ ordent dem Raum $(\Omega,P)$
einen neuen Wahrscheinlichkeitsraum $(\R,P_X)$ zu, wobei
\begin{equation}
P_X\colon 2^{X(\Omega)}\to [0,1],\; P_X(A):=P(X^{-1}(A))
\end{equation}
definiert wird. Mit
\begin{equation}
X^{-1}(A) := \{\omega\in\Omega\mid X(\omega)\in A\}
\end{equation}
ist das Urbild von $A$ gemeint.
Die folgenden Kurzschreibweisen haben sich
eingebürgert:
\begin{align}
P(X\in A) &:= P(\{\omega\mid X(\omega)\in A\}),\\
P(X=x) &:= P(\{\omega\mid X(\omega)=x\}),\\
P(X\le x) &:= P(\{\omega\mid (X\omega)\le x\}).
\end{align}

\begin{definition}[Verteilungsfunktion]
Für eine Zufallsvariable $X$ wird
\begin{equation}
F(x):=P(X\le x),\quad F\colon\R\to [0,1]
\end{equation}
\emdef{Verteilungsfunktion} von $X$ genannt.
\end{definition}

\strong{Eigenschaften von Verteilungsfunktionen.}\\
Für eine Verteilungsfunktion $F$ gilt:
\begin{gather}
\bulletbs F\text{ ist monoton wachsend},\\
\bulletbs F\text{ ist rechtsseitig stetig},\\
\bulletbs \lim\limits_{x\to -\infty} F(x) = 0,\\
\bulletbs \lim\limits_{x\to\infty} F(x)=1,\\
\bulletbs P(a<X\le b) = F(b)-F(a).
\end{gather}



\onecolumn
\chapter{Tabellen}

\section{Lineare Algebra}
\subsection{Lineare Abbildungen}
\begin{tabular}{lcccc}
\toprule
\strong{Endomorphismus}
& \strong{Matrix}
& \strong{Inverse}
& \strong{Eigenwerte}\\
\midrule
Identität
& $E=\begin{bmatrix}1 & 0\\ 0 & 1\end{bmatrix}$
& $E^{-1}=E$
& $+1,+1$\\
\midrule
Skalierung
& $rE=\begin{bmatrix}r & 0\\ 0 & r\end{bmatrix}$
& $(rE)^{-1} = \tfrac{1}{r}E = \begin{bmatrix}1/r & 0\\ 0 & 1/r\end{bmatrix}$
& $r,r$\\
\midrule
Skalierung der $x$"=Achse
& $V_x = \begin{bmatrix}r & 0\\ 0 & 1\end{bmatrix}$
& $V_x^{-1} = \begin{bmatrix}1/r & 0\\ 0 & 1\end{bmatrix}$
& $r,1$\\
\midrule
Skalierung der $y$"=Achse
& $V_y = \begin{bmatrix}1 & 0\\ 0 & r\end{bmatrix}$
& $V_y^{-1} = \begin{bmatrix}1 & 0\\ 0 & 1/r\end{bmatrix}$
& $r,1$\\
\midrule
Spiegelung an der $x$"=Achse
& $S_x = \begin{bmatrix}1 & 0\\ 0 & -1\end{bmatrix}$
& $S_x^{-1} = S_x$
& $\pm 1$\\
\midrule
Spiegelung an der $y$"=Achse
& $S_y = \begin{bmatrix}-1 & 0\\ 0 & 1\end{bmatrix}$
& $S_y^{-1} = S_y$
& $\pm 1$\\
\midrule
\begin{tabular}[c]{@{}l@{}}Spiegelung an der Achse\\
des Vektors $v=(a,b)$\end{tabular}
& $S_v = \frac{1}{a^2{+}b^2}\begin{bmatrix}a^2{-}b^2 & 2ab\\ 2ab & b^2{-}a^2\end{bmatrix}$
& $S_v^{-1} = S_v$
& $\pm 1$\\
\midrule
Spiegelung am Ursprung
& $S_0 = \begin{bmatrix}-1 & 0\\ 0 & -1\end{bmatrix}$
& $S_0^{-1} = S_0$
& $-1,-1$\\
\midrule
Projektion auf die $x$"=Achse
& $P_x = \begin{bmatrix}1 & 0\\ 0 & 0\end{bmatrix}$
& nicht vorhanden
& $0,+1$\\
\midrule
Projektion auf die $y$"=Achse
& $P_y = \begin{bmatrix}0 & 0\\ 0 & 1\end{bmatrix}$
& nicht vorhanden
& $0,+1$\\
\midrule
\begin{tabular}[c]{@{}l@{}}Projektion auf die Achse\\
des Vektors $v=(a,b)$\end{tabular}
& $P_v = \tfrac{1}{a^2+b^2}\begin{bmatrix}a^2 & ab\\ ab & b^2\end{bmatrix}$
& nicht vorhanden
& $0,+1$\\
\midrule
Scherung an der $x$"=Achse
& $M_x = \begin{bmatrix}1 & m\\ 0 & 1\end{bmatrix}$
& $M_x^{-1} = \begin{bmatrix}1 & -m\\ 0 & 1\end{bmatrix}$
& $+1,+1$\\
\midrule
Scherung an der $y$"=Achse
& $M_y = \begin{bmatrix}1 & 0\\ m & 1\end{bmatrix}$
& $M_y^{-1} = \begin{bmatrix}1 & 0\\ -m & 1\end{bmatrix}$
& $+1,+1$\\
\midrule
\begin{tabular}[c]{@{}l@{}}Rotation um den Winkel $\varphi$\\
gegen den Uhrzeigersinn\end{tabular}
& $R(\varphi) = \begin{bmatrix}\cos\varphi & -\sin\varphi\\ \cos\varphi & \sin\varphi\end{bmatrix}$
& $R(\varphi)^{-1} = R(\varphi)^T = R(-\varphi)$
& $\cos\varphi\pm\ui\sin\varphi$\\
\midrule
\begin{tabular}[c]{@{}l@{}}Rotation um $90^\circ$ gegen\\
den Uhrzeigersinn\end{tabular}
& $R(\tfrac{\pi}{4}) = \begin{bmatrix}0 & -1\\ 1 & 0\end{bmatrix}$
& $R(\tfrac{\pi}{4})^{-1} = R(-\tfrac{\pi}{4})$
& $\pm\ui$\\
\midrule
\begin{tabular}[c]{@{}l@{}}Rotation um $90^\circ$ im\\
Uhrzeigersinn\end{tabular}
& $R(-\tfrac{\pi}{4}) = \begin{bmatrix}0 & 1\\ -1 & 0\end{bmatrix}$
& $R(-\tfrac{\pi}{4})^{-1} = R(\tfrac{\pi}{4})$
& $\pm\ui$\\
\midrule
\begin{tabular}[c]{@{}l@{}}Drehskalierung, entspricht\\
der komplexen Zahl $a+b\ui$\end{tabular}
& $Z = \begin{bmatrix}a & -b\\ b & a\end{bmatrix}$
& $Z^{-1} = \tfrac{1}{a^2+b^2}\begin{bmatrix}a & b\\ -b & a\end{bmatrix}$
& $a\pm b\ui$\\
\midrule
\begin{tabular}[c]{@{}l@{}}Drehskalierung, entspricht\\
der komplexen Zahl $r\ee^{\ui\varphi}$\end{tabular}
& $Z = rR(\varphi)$
& $Z^{-1} = \tfrac{1}{r}R(-\varphi)$
& $r\cos\varphi\pm\ui r\sin\varphi$\\
\midrule
\begin{tabular}[c]{@{}l@{}}Allgemeiner\\
Endomorphismus\end{tabular}
& $A = \begin{bmatrix}a & b\\ c & d\end{bmatrix}$
& $A^{-1} = \tfrac{1}{ad-bc}\begin{bmatrix}d & -b\\ -c & a\end{bmatrix}$
& $\tfrac{a{+}d{\pm}\sqrt{(a{-}d)^2{+}4bc}}{2}$\\
\bottomrule
\end{tabular}

\newpage
\section{Kombinatorik}
\subsection{Binomialkoeffizienten}\index{Binomialkoeffizient!Tabelle}

\hspace{-6pt}\begin{tabular}{ll}
\begin{tabular}[t]{l|ccccccccccc}
\toprule
& $k=0$ & $k=1$ & $k=2$ & $k=3$ & $k=4$ & $k=5$ & $k=6$ & $k=7$ & $k=8$ & $k=9$ & $\hspace{-2pt}k=10\hspace{-2pt}$\\
\midrule%
$n=\phantom{1}0$ & 1 & 0 & 0 & 0 & 0 & 0 & 0 & 0 & 0 & 0 & 0\\
$n=\phantom{1}1$ & 1 & 1 & 0 & 0 & 0 & 0 & 0 & 0 & 0 & 0 & 0\\
$n=\phantom{1}2$ & 1 & 2 & 1 & 0 & 0 & 0 & 0 & 0 & 0 & 0 & 0\\
$n=\phantom{1}3$ & 1 & 3 & 3 & 1 & 0 & 0 & 0 & 0 & 0 & 0 & 0\\
\midrule%
$n=\phantom{1}4$ & 1 & 4 & 6 & 4 & 1 & 0 & 0 & 0 & 0 & 0 & 0\\
$n=\phantom{1}5$ & 1 & 5 & 10 & 10 & 5 & 1 & 0 & 0 & 0 & 0 & 0\\
$n=\phantom{1}6$ & 1 & 6 & 15 & 20 & 15 & 6 & 1 & 0 & 0 & 0 & 0\\
$n=\phantom{1}7$ & 1 & 7 & 21 & 35 & 35 & 21 & 7 & 1 & 0 & 0 & 0\\
\midrule%
$n=\phantom{1}8$ & 1 & 8 & 28 & 56 & 70 & 56 & 28 & 8 & 1 & 0 & 0\\
$n=\phantom{1}9$ & 1 & 9 & 36 & 84 & 126 & 126 & 84 & 36 & 9 & 1 & 0\\
$n=10$ & 1 & 10 & 45 & 120 & 210 & 252 & 210 & 120 & 45 & 10 & 1\\
$n=11$ & 1 & 11 & 55 & 165 & 330 & 462 & 462 & 330 & 165 & 55 & 11\\
\midrule%
$n=12$ & 1 & 12 & 66 & 220 & 495 & 792 & 924 & 792 & 495 & 220 & 66\\
$n=13$ & 1 & 13 & 78 & 286 & 715 & 1287 & 1716 & 1716 & 1287 & 715 & 286\\
$n=14$ & 1 & 14 & 91 & 364 & 1001 & 2002 & 3003 & 3432 & 3003 & 2002 & 1001\\
$n=15$ & 1 & 15 & 105 & 455 & 1365 & 3003 & 5005 & 6435 & 6435 & 5005 & 3003\\
\midrule%
$n=16$ & 1 & 16 & 120 & 560 & 1820 & 4368 & 8008 & 11440 & 12870 & 11440 & 8008\\
$n=17$ & 1 & 17 & 136 & 680 & 2380 & 6188 & 12376 & 19448 & 24310 & 24310 & 19448\\
$n=18$ & 1 & 18 & 153 & 816 & 3060 & 8568 & 18564 & 31824 & 43758 & 48620 & 43758\\
$n=19$ & 1 & 19 & 171 & 969 & 3876 & 11628 & 27132 & 50388 & 75582 & 92378 & 92378\\
\bottomrule
\end{tabular}
& \begin{tabular}[t]{l}
\\
$\dbinom{n}{k}$
\end{tabular}
\end{tabular}

\vspace{3em}\noindent
\begin{tabular}[t]{l|cccccccccc}
\toprule
& $k=0$ & $k=1$ & $k=2$ & $k=3$ & $k=4$ & $k=5$ & $k=6$ & $k=7$ & $k=8$ & $k=9$\\
\midrule%
$n=-15$ & $1$ & $-15$ & $120$ & $-680$ & $3060$ & $-11628$ & $38760$ & $-116280$ & $319770$ & $-817190$\\
$n=-14$ & $1$ & $-14$ & $105$ & $-560$ & $2380$ & $-8568$ & $27132$ & $-77520$ & $203490$ & $-497420$\\
$n=-13$ & $1$ & $-13$ & $91$ & $-455$ & $1820$ & $-6188$ & $18564$ & $-50388$ & $125970$ & $-293930$\\
$n=-12$ & $1$ & $-12$ & $78$ & $-364$ & $1365$ & $-4368$ & $12376$ & $-31824$ & $75582$ & $-167960$\\
\midrule%
$n=-11$ & $1$ & $-11$ & $66$ & $-286$ & $1001$ & $-3003$ & $8008$ & $-19448$ & $43758$ & $-92378$\\
$n=-10$ & $1$ & $-10$ & $55$ & $-220$ & $715$ & $-2002$ & $5005$ & $-11440$ & $24310$ & $-48620$\\
$n=\phantom{1}{-9}$ & $1$ & $-9$ & $45$ & $-165$ & $495$ & $-1287$ & $3003$ & $-6435$ & $12870$ & $-24310$\\
$n=\phantom{1}{-8}$ & $1$ & $-8$ & $36$ & $-120$ & $330$ & $-792$ & $1716$ & $-3432$ & $6435$ & $-11440$\\
\midrule%
$n=\phantom{1}{-7}$ & $1$ & $-7$ & $28$ & $-84$ & $210$ & $-462$ & $924$ & $-1716$ & $3003$ & $-5005$\\
$n=\phantom{1}{-6}$ & $1$ & $-6$ & $21$ & $-56$ & $126$ & $-252$ & $462$ & $-792$ & $1287$ & $-2002$\\
$n=\phantom{1}{-5}$ & $1$ & $-5$ & $15$ & $-35$ & $70$ & $-126$ & $210$ & $-330$ & $495$ & $-715$\\
$n=\phantom{1}{-4}$ & $1$ & $-4$ & $10$ & $-20$ & $35$ & $-56$ & $84$ & $-120$ & $165$ & $-220$\\
\midrule%
$n=\phantom{1}{-3}$ & $1$ & $-3$ & $6$ & $-10$ & $15$ & $-21$ & $28$ & $-36$ & $45$ & $-55$\\
$n=\phantom{1}{-2}$ & $1$ & $-2$ & $3$ & $-4$ & $5$ & $-6$ & $7$ & $-8$ & $9$ & $-10$\\
$n=\phantom{1}{-1}$ & $1$ & $-1$ & $1$ & $-1$ & $1$ & $-1$ & $1$ & $-1$ & $1$ & $-1$\\
$n=\phantom{-1}0$ & $1$ & $\phantom{-}0$ & $0$ & $\phantom{-}0$ & $0$
 & $\phantom{-}0$ & $0$ & $\phantom{-}0$ & $0$ & $\phantom{-}0$\\
\bottomrule
\end{tabular}

\vspace{2em}
\[\dbinom{n+1}{k+1} = \dbinom{n}{k}+\dbinom{n}{k+1},\]
\[\dbinom{n}{k} = \dbinom{n}{n-k} = \frac{n!}{k!\,(n-k)!}\qquad (0\le k\le n)\]

\newpage
\vglue 4em
\subsection{Stirling-Zahlen erster Art}\index{Stirling-Zahlen!Tabelle}
$\begin{bmatrix}n\\ k\end{bmatrix}$

\vspace{4pt}
\noindent
\begin{tabular}{r|rrrrrrrrrrrrrrr}
\toprule%
& $k=0$ & $k=1$ & $k=2$ & $k=3$ & $k=4$ & $k=5$ & $k=6$ & $k=7$ & $k=8$ & $k=9$\\
\midrule%
$n= 0$ &     1 &     0 &     0 &     0 &     0 &     0 &     0 &     0 &     0 &     0\\
$n= 1$ &     0 &     1 &     0 &     0 &     0 &     0 &     0 &     0 &     0 &     0\\
$n= 2$ &     0 &     1 &     1 &     0 &     0 &     0 &     0 &     0 &     0 &     0\\
$n= 3$ &     0 &     2 &     3 &     1 &     0 &     0 &     0 &     0 &     0 &     0\\
\midrule%
$n= 4$ &     0 &     6 &    11 &     6 &     1 &     0 &     0 &     0 &     0 &     0\\
$n= 5$ &     0 &    24 &    50 &    35 &    10 &     1 &     0 &     0 &     0 &     0\\
$n= 6$ &     0 &   120 &   274 &   225 &    85 &    15 &     1 &     0 &     0 &     0\\
$n= 7$ &     0 &   720 &  1764 &  1624 &   735 &   175 &    21 &     1 &     0 &     0\\
\midrule%
$n= 8$ &     0 &  5040 & 13068 & 13132 &  6769 &  1960 &   322 &    28 &     1 &     0\\
$n= 9$ &     0 & 40320 &109584 &118124 & 67284 & 22449 &  4536 &   546 &    36 &     1\\
$n=10$ &     0 &362880 &1026576&1172700&723680 &269325 & 63273 &  9450 &   870 &    45\\
$n=11$ &     0 &\!3628800 &\!\!10628640 &\!\!12753576 &\!8409500 &\!3416930 &\!902055 &\!157773 & 18150 &  1320\\
\bottomrule
\end{tabular}

\vspace{4em}
\subsection{Stirling-Zahlen zweiter Art}
$\begin{Bmatrix}n\\ k\end{Bmatrix}$

\vspace{4pt}
\noindent
\begin{tabular}{r|rrrrrrrrrr}
\toprule%
& $k=0$ & $k=1$ & $k=2$ & $k=3$ & $k=4$ & $k=5$ & $k=6$ & $k=7$ & $k=8$ & $k=9$\\
\midrule%
$n= 0$ &     1 &     0 &     0 &     0 &     0 &     0 &     0 &     0 &     0 &     0\\
$n= 1$ &     0 &     1 &     0 &     0 &     0 &     0 &     0 &     0 &     0 &     0\\
$n= 2$ &     0 &     1 &     1 &     0 &     0 &     0 &     0 &     0 &     0 &     0\\
$n= 3$ &     0 &     1 &     3 &     1 &     0 &     0 &     0 &     0 &     0 &     0\\
\midrule%
$n= 4$ &     0 &     1 &     7 &     6 &     1 &     0 &     0 &     0 &     0 &     0\\
$n= 5$ &     0 &     1 &    15 &    25 &    10 &     1 &     0 &     0 &     0 &     0\\
$n= 6$ &     0 &     1 &    31 &    90 &    65 &    15 &     1 &     0 &     0 &     0\\
$n= 7$ &     0 &     1 &    63 &   301 &   350 &   140 &    21 &     1 &     0 &     0\\
\midrule%
$n= 8$ &     0 &     1 &   127 &   966 &  1701 &  1050 &   266 &    28 &     1 &     0\\
$n= 9$ &     0 &     1 &   255 &  3025 &  7770 &  6951 &  2646 &   462 &    36 &     1\\
$n=10$ &     0 &     1 &   511 &  9330 & 34105 & 42525 & 22827 &  5880 &   750 &    45\\
$n=11$ &     0 &     1 &  1023 & 28501 &145750 &246730 &179487 & 63987 & 11880 &  1155\\
\bottomrule
\end{tabular}

\newpage
\section{Zahlentheorie}
\subsection{Primzahlen}\index{Primzahlen!Tabelle}

\begin{tabular}{rrrrrrrrrrrrrr@{\;\;\vrule width \heavyrulewidth\;}r}
\toprule
0 & 40 & 80 & 120 & 160 & 200 & 240 & 280 & 320 & 360 & 400 & 440 & 480 & 520 &\\
\midrule[\heavyrulewidth]
   2 &  179 &  419 &  661 &  947 & 1229 & 1523 & 1823 & 2131 & 2437 & 2749 & 3083 & 3433 & 3733 &    1\\
   3 &  181 &  421 &  673 &  953 & 1231 & 1531 & 1831 & 2137 & 2441 & 2753 & 3089 & 3449 & 3739 &    2\\
   5 &  191 &  431 &  677 &  967 & 1237 & 1543 & 1847 & 2141 & 2447 & 2767 & 3109 & 3457 & 3761 &    3\\
   7 &  193 &  433 &  683 &  971 & 1249 & 1549 & 1861 & 2143 & 2459 & 2777 & 3119 & 3461 & 3767 &    4\\
  11 &  197 &  439 &  691 &  977 & 1259 & 1553 & 1867 & 2153 & 2467 & 2789 & 3121 & 3463 & 3769 &    5\\
&&&&&&&&&&&&&&\\
  13 &  199 &  443 &  701 &  983 & 1277 & 1559 & 1871 & 2161 & 2473 & 2791 & 3137 & 3467 & 3779 &    6\\
  17 &  211 &  449 &  709 &  991 & 1279 & 1567 & 1873 & 2179 & 2477 & 2797 & 3163 & 3469 & 3793 &    7\\
  19 &  223 &  457 &  719 &  997 & 1283 & 1571 & 1877 & 2203 & 2503 & 2801 & 3167 & 3491 & 3797 &    8\\
  23 &  227 &  461 &  727 & 1009 & 1289 & 1579 & 1879 & 2207 & 2521 & 2803 & 3169 & 3499 & 3803 &    9\\
  29 &  229 &  463 &  733 & 1013 & 1291 & 1583 & 1889 & 2213 & 2531 & 2819 & 3181 & 3511 & 3821 &   10\\
&&&&&&&&&&&&&&\\
  31 &  233 &  467 &  739 & 1019 & 1297 & 1597 & 1901 & 2221 & 2539 & 2833 & 3187 & 3517 & 3823 &   11\\
  37 &  239 &  479 &  743 & 1021 & 1301 & 1601 & 1907 & 2237 & 2543 & 2837 & 3191 & 3527 & 3833 &   12\\
  41 &  241 &  487 &  751 & 1031 & 1303 & 1607 & 1913 & 2239 & 2549 & 2843 & 3203 & 3529 & 3847 &   13\\
  43 &  251 &  491 &  757 & 1033 & 1307 & 1609 & 1931 & 2243 & 2551 & 2851 & 3209 & 3533 & 3851 &   14\\
  47 &  257 &  499 &  761 & 1039 & 1319 & 1613 & 1933 & 2251 & 2557 & 2857 & 3217 & 3539 & 3853 &   15\\
&&&&&&&&&&&&&&\\
  53 &  263 &  503 &  769 & 1049 & 1321 & 1619 & 1949 & 2267 & 2579 & 2861 & 3221 & 3541 & 3863 &   16\\
  59 &  269 &  509 &  773 & 1051 & 1327 & 1621 & 1951 & 2269 & 2591 & 2879 & 3229 & 3547 & 3877 &   17\\
  61 &  271 &  521 &  787 & 1061 & 1361 & 1627 & 1973 & 2273 & 2593 & 2887 & 3251 & 3557 & 3881 &   18\\
  67 &  277 &  523 &  797 & 1063 & 1367 & 1637 & 1979 & 2281 & 2609 & 2897 & 3253 & 3559 & 3889 &   19\\
  71 &  281 &  541 &  809 & 1069 & 1373 & 1657 & 1987 & 2287 & 2617 & 2903 & 3257 & 3571 & 3907 &   20\\
&&&&&&&&&&&&&&\\
  73 &  283 &  547 &  811 & 1087 & 1381 & 1663 & 1993 & 2293 & 2621 & 2909 & 3259 & 3581 & 3911 &   21\\
  79 &  293 &  557 &  821 & 1091 & 1399 & 1667 & 1997 & 2297 & 2633 & 2917 & 3271 & 3583 & 3917 &   22\\
  83 &  307 &  563 &  823 & 1093 & 1409 & 1669 & 1999 & 2309 & 2647 & 2927 & 3299 & 3593 & 3919 &   23\\
  89 &  311 &  569 &  827 & 1097 & 1423 & 1693 & 2003 & 2311 & 2657 & 2939 & 3301 & 3607 & 3923 &   24\\
  97 &  313 &  571 &  829 & 1103 & 1427 & 1697 & 2011 & 2333 & 2659 & 2953 & 3307 & 3613 & 3929 &   25\\
&&&&&&&&&&&&&&\\
 101 &  317 &  577 &  839 & 1109 & 1429 & 1699 & 2017 & 2339 & 2663 & 2957 & 3313 & 3617 & 3931 &   26\\
 103 &  331 &  587 &  853 & 1117 & 1433 & 1709 & 2027 & 2341 & 2671 & 2963 & 3319 & 3623 & 3943 &   27\\
 107 &  337 &  593 &  857 & 1123 & 1439 & 1721 & 2029 & 2347 & 2677 & 2969 & 3323 & 3631 & 3947 &   28\\
 109 &  347 &  599 &  859 & 1129 & 1447 & 1723 & 2039 & 2351 & 2683 & 2971 & 3329 & 3637 & 3967 &   29\\
 113 &  349 &  601 &  863 & 1151 & 1451 & 1733 & 2053 & 2357 & 2687 & 2999 & 3331 & 3643 & 3989 &   30\\
&&&&&&&&&&&&&&\\
 127 &  353 &  607 &  877 & 1153 & 1453 & 1741 & 2063 & 2371 & 2689 & 3001 & 3343 & 3659 & 4001 &   31\\
 131 &  359 &  613 &  881 & 1163 & 1459 & 1747 & 2069 & 2377 & 2693 & 3011 & 3347 & 3671 & 4003 &   32\\
 137 &  367 &  617 &  883 & 1171 & 1471 & 1753 & 2081 & 2381 & 2699 & 3019 & 3359 & 3673 & 4007 &   33\\
 139 &  373 &  619 &  887 & 1181 & 1481 & 1759 & 2083 & 2383 & 2707 & 3023 & 3361 & 3677 & 4013 &   34\\
 149 &  379 &  631 &  907 & 1187 & 1483 & 1777 & 2087 & 2389 & 2711 & 3037 & 3371 & 3691 & 4019 &   35\\
&&&&&&&&&&&&&&\\
 151 &  383 &  641 &  911 & 1193 & 1487 & 1783 & 2089 & 2393 & 2713 & 3041 & 3373 & 3697 & 4021 &   36\\
 157 &  389 &  643 &  919 & 1201 & 1489 & 1787 & 2099 & 2399 & 2719 & 3049 & 3389 & 3701 & 4027 &   37\\
 163 &  397 &  647 &  929 & 1213 & 1493 & 1789 & 2111 & 2411 & 2729 & 3061 & 3391 & 3709 & 4049 &   38\\
 167 &  401 &  653 &  937 & 1217 & 1499 & 1801 & 2113 & 2417 & 2731 & 3067 & 3407 & 3719 & 4051 &   39\\
 173 &  409 &  659 &  941 & 1223 & 1511 & 1811 & 2129 & 2423 & 2741 & 3079 & 3413 & 3727 & 4057 &   40\\
\bottomrule
\end{tabular}

\twocolumn



\chapter{Anhang}
\section{Mathematische Konstanten}
\begin{enumerate}
\item Kreiszahl\\
$\pi = 3.14159\;26535\;89793\;23846\;26433\;83279\ldots$

\item Eulersche Zahl\\
$\ee = 2.71828\;18284\;59045\;23536\;02874\;71352\ldots$

\item Euler-Mascheroni-Konstante\\
$\gamma = 0.57721\;56649\;01532\;86060\;65120\;90082\ldots$

\item Goldener Schnitt, $(1+\sqrt{5})/2$\\
$\varphi = 1.61803\;39887\;49894\;84820\;45868\;34365\ldots$

\item 1. Feigenbaum-Konstante\\
$\delta = 4.66920\;16091\;02990\;67185\;32038\;20466\ldots$

\item 2. Feigenbaum-Konstante\\
$\alpha = 2.50290\;78750\;95892\;82228\;39028\;73218\ldots$
\end{enumerate}

\section{Physikalische Konstanten}

\begin{enumerate}
\item Lichtgeschwindigkeit im Vakuum\\
$c=299\:792\:458\:\unit{m/s}$

\item Elektrische Feldkonstante\\
$\varepsilon_0 = 8.854\:187\:817\:620\:39\times 10^{-12}\:\unit{F/m}$

\item Magnetische Feldkonstante\\
$\mu_0 = 4\pi\times 10^{-7}\:\unit{H/m}$

\item Elementarladung\\
$e = 1.602\:176\:6208(98)\times 10^{-19}\:\unit{C}$
\end{enumerate}

\newpage
\section{Griechisches Alphabet}

\begin{tabular}{l|l}
\begin{tabular}[t]{lll}
$\mathrm A$ & $\alpha$   & Alpha\\
$\mathrm B$ & $\beta$    & Beta\\
$\Gamma$    & $\gamma$   & Gamma\\
$\Delta$    & $\delta$   & Delta\\
\noalign{\vspace{1em}}
$\mathrm E$ & $\varepsilon$ & Epsilon\\
$\mathrm Z$ & $\zeta$    & Zeta\\
$\mathrm H$ & $\eta$     & Eta\\
$\Theta$    & $\theta$   & Theta\\
\noalign{\vspace{1em}}
$\mathrm I$ & $\iota$    & Jota\\
$\mathrm K$ & $\kappa$   & Kappa\\
$\Lambda$   & $\lambda$  & Lambda\\
$\mathrm M$ & $\mu$      & My
\end{tabular}
&
\begin{tabular}[t]{lll}
$\mathrm N$ & $\nu$      & Nu\\
$\Xi$       & $\xi$      & Xi\\
$\mathrm O$ & $o$        & Omikron\\
$\Pi$       & $\pi$      & Pi\\
\noalign{\vspace{1em}}
$\mathrm R$ & $\varrho$  & Rho\\
$\Sigma$    & $\sigma$   & Sigma\\
$\mathrm T$ & $\tau$     & Tau\\
$\mathrm Y$ & $y$        & Ypsilon\\
\noalign{\vspace{1em}}
$\Phi$      & $\varphi$  & Phi\\
$\mathrm X$ & $\chi$     & Chi\\
$\Psi$      & $\psi$     & Psi\\
$\Omega$    & $\omega$   & Omega 
\end{tabular}
\end{tabular}

\section{Frakturbuchstaben}
\begin{tabular}{l|l}
\begin{tabular}[t]{l@{\hskip 2pt}ll@{\hskip 2pt}l}
A & a & $\mathfrak A$ & $\mathfrak a$\\
B & b & $\mathfrak B$ & $\mathfrak b$ \\
C & c & $\mathfrak C$ & $\mathfrak c$\\
D & d & $\mathfrak D$ & $\mathfrak d$\\
\noalign{\vspace{1em}}
E & e & $\mathfrak E$ & $\mathfrak e$\\
F & f & $\mathfrak F$ & $\mathfrak f$\\
G & g & $\mathfrak G$ & $\mathfrak g$\\
H & h & $\mathfrak H$ & $\mathfrak h$\\
\noalign{\vspace{1em}}
I & i & $\mathfrak I$ & $\mathfrak i$\\
J & j & $\mathfrak J$ & $\mathfrak j$\\
K & k & $\mathfrak K$ & $\mathfrak k$\\
L & l & $\mathfrak L$ & $\mathfrak l$\\
\noalign{\vspace{1em}}
M & m & $\mathfrak M$ & $\mathfrak m$\\
N & n & $\mathfrak N$ & $\mathfrak n$
\end{tabular}
&
\begin{tabular}[t]{l@{\hskip 2pt}ll@{\hskip 2pt}l}
O & o & $\mathfrak O$ & $\mathfrak o$\\
P & p & $\mathfrak P$ & $\mathfrak p$\\
Q & q & $\mathfrak Q$ & $\mathfrak q$\\
R & r & $\mathfrak R$ & $\mathfrak r$\\
\noalign{\vspace{1em}}
S & s & $\mathfrak S$ & $\mathfrak s$\\
T & t & $\mathfrak T$ & $\mathfrak t$\\
U & u & $\mathfrak U$ & $\mathfrak u$\\
V & v & $\mathfrak V$ & $\mathfrak v$\\
\noalign{\vspace{1em}}
W & w & $\mathfrak W$ & $\mathfrak w$\\
X & x & $\mathfrak X$ & $\mathfrak x$\\
Y & y & $\mathfrak Y$ & $\mathfrak y$\\
Z & z & $\mathfrak Z$ & $\mathfrak z$
\end{tabular}
\end{tabular}



\begin{thebibliography}{00}

\bibitem{Gentzen1935}
Gerhard Gentzen: \emph{Untersuchungen über das logische Schließen}.
In: \emph{Mathematische Zeitschrift}. Band~39, 1935, S.~176--210,
S.~405--431.

\bibitem{Gentzen1936}
Gerhard Gentzen: \emph{Die Widerspruchsfreiheit der reinen
Zahlentheorie}. In: \emph{Mathematische Annalen}. Band~112,
1936, S.~493--565.

\bibitem{Gentzen1938}
Gerhard Gentzen: \emph{Die gegenwärtige Lage in der mathematischen
Grundlagenforschung. Neue Fassung des Widerspruchsfreiheitsbeweises für
die reine Zahlentheorie}. In: \emph{Forschungen zur Logik und zur
Grundlegung der exakten Wissenschaften}. Heft~4, S.~Hirzel,
Leipzig 1938.

\bibitem{Menzler-Trott}
Eckart Menzler-Trott: \emph{Gentzens Problem. Mathematische Logik
im nationalsozialistischen Deutschland}. Birkhäuser, Basel 2001.

\bibitem{Johansson}
Ingebrigt Johansson: \emph{Der Minimalkalkül, ein reduzierter
intuitionistischer Formalismus}. In: \emph{Compositio Mathematica}.
Band~4, 1937, S.~119–136.

\bibitem{Diener}
Hannes Diener, Maarten McKubre-Jordens: \emph{Classifying Material
Implications over Minimal Logic}. In: \emph{Archive for Mathematical
Logic}. Band~59, 2020, S.~905--924.
\href{https://doi.org/10.1007/s00153-020-00722-x}%
{doi:10.1007/s00153-020-00722-x}.

\bibitem{von-Plato-Reasoning}
Jan von Plato: \emph{Elements of Logical Reasoning}.
Cambridge University Press, Cambridge 2013.

\bibitem{von-Plato}
Jan von Plato: \emph{Gentzen's Logic}. In: \emph{Handbook of The
History of Logic}. Band~5, North-Holland, 2009.

\bibitem{Hazen}
Francis Jeffry Pelletier, Allen P. Hazen: \emph{A History of Natural
Deduction}.\\ In: \emph{Handbook of The History of Logic}.
Band 11, North-Holland, 2012.

\bibitem{Hazen-online} Francis Jeffry Pelletier, Allen Hazen:
\href{https://plato.stanford.edu/entries/natural-deduction/}%
{\emph{Natural Deduction Systems in Logic}}.\\
In: \emph{The Stanford Encyclopedia of Philosophy}.

\bibitem{Indrzejczak}
Andrzej Indrzejczak:
\href{https://iep.utm.edu/natural-deduction/}{\emph{Natural Deduction}}.
In: \emph{The Internet Encyclopedia of Philosophy}.

\bibitem{Mimram}
Samuel Mimram:
\emph{\href{https://www.lix.polytechnique.fr/Labo/Samuel.Mimram/publications/}%
{Program = Proof}}.
Laboratoire d'Informatique de l'Ecole polytechnique, Palaiseau 2020.

\bibitem{Hoffmann}
Dirk W. Hoffmann: \emph{Grenzen der Mathematik}.
Springer, Berlin 2011.

\bibitem{Ebbinghaus-Logik}
Heinz-Dieter Ebbinghaus, Jörg Flum, Wolfgang Thomas:
\emph{Einführung in die mathematische Logik}.
Springer Spektrum, 1978, 6. Auflage 2018.\\
\href{https://doi.org/10.1007/978-3-662-58029-5}%
{doi:10.1007/978-3-662-58029-5}.

\bibitem{OpenLogic} Open Logic Project:
\emph{The Open Logic Text}. Complete Build, Oktober 2022.

\bibitem{Avigad} Jeremy Avigad:
\emph{Mathematical Logic and Computation}.\\
Cambridge University Press, 2023.

\bibitem{Avigad-Foundations} Jeremy Avigad: \emph{Foundations}.
September 2021. \href{https://arxiv.org/abs/2009.09541v4}{arXiv:2009.09541v4}.

\bibitem{Wansing} Heinrich Wansing (Hrsg.):
\emph{Dag Prawitz on Proofs and Meaning}.
In: \emph{Outstanding Contributions to Logic}. Springer, 2015.
\href{https://doi.org/10.1007/978-3-319-11041-7}{doi:10.1007/978-3-319-11041-7}.

\bibitem{Enderton} Herbert B. Enderton:
\emph{A Mathematical Introduction to Logic}.
Academic Press, New York 1972, 2. Auflage 2001.

\bibitem{Taylor} Paul Taylor:
\emph{Practical Foundations of Mathematics}.\\
Cambridge University Press, 1999.

\bibitem{Blackburn} Patrick Blackburn, Johan van Benthem:
\emph{Modal Logic: A Semantic Perspective}.
In: \emph{Studies in Logic and Practical Reasoning}.
Band 3, 2007, S. 1--84.

\bibitem{HoTT}
\emph{Homotopy Type Theory. Univalent Foundations of Mathematics}.
The Univalent Foundations Program, 2013.

\bibitem{Deiser-Grundbegriffe}
Oliver Deiser:
\emph{\href{https://www.aleph1.info/?call=Puc&permalink=grundbegriffe}%
{Grundbegriffe der Mathematik}. Sprache, Zahlen und erste
Erkundungen}. März 2021, letzte Version Oktober 2022.

\bibitem{Cantor} Georg Cantor:
\emph{Beiträge zur Begründung der transfiniten Mengenlehre}.\\
In: \emph{Mathematische Annalen}. Band~46, 1895, S.~481.

\bibitem{Fraenkel} Abraham Adolf Fraenkel:
\emph{Einleitung in die Mengenlehre}.\\
Springer, Berlin 1919, 3. Auflage 1928.

\bibitem{Deiser} Oliver Deiser:
\emph{Einführung in die Mengenlehre}.
Springer, 2002, 3. Auflage 2010.

\bibitem{Ebbinghaus-Mengen} Heinz-Dieter Ebbinghaus:
\emph{Einführung in die Mengenlehre}.\\
Springer Spektrum, 5. Auflage 2021.

\bibitem{Enderton-sets} Herbert B. Enderton:
\emph{Elements of Set Theory}. Academic Press, New York 1977.

\bibitem{Shulman} Michael A. Shulman:
\emph{Set theory for category theory}.
Oktober 2008.\\
\href{https://arxiv.org/abs/0810.1279}{arXiv:0810.1279}.

\bibitem{Jech} Thomas Jech: \emph{Set Theory: The Third Millennium
Edition, revised and expanded}. Springer, 2002.
\href{https://doi.org/10.1007/3-540-44761-X}{doi:10.1007/3-540-44761-X}.

\bibitem{Siraphob} Ben Siraphob:
\href{https://siraben.dev/2021/06/27/classical-math-coq.html}{%
\emph{A non-trivial trivial theorem: doing classical mathematics in Coq}}.
Blogpost, 27. Juni 2021.

\bibitem{Stach} Stefan Müller-Stach:
\emph{Richard Dedekind: Was sind und was sollen die Zahlen?
Stetigkeit und Irrationale Zahlen}. Springer, 2017.

\bibitem{Lorenzen} Paul Lorenzen:
\emph{Die Definition durch vollständige Induktion}.
In: \emph{Monatshefte für Mathematik und Physik}.
Band~47, 1939, S.~356--358.

\bibitem{Halmos} Paul R. Halmos:
\emph{Naive Set Theory}. Springer, 1974.

\bibitem{Mainzer} Klaus Mainzer:
\emph{Natürliche, ganze und rationale Zahlen}.
In: Heinz-Dieter Ebbinghaus u. a.: \emph{Zahlen}.
Springer, 1983, 3. Auflage 1992.

\bibitem{Lamm} Christoph Lamm:
\emph{Karl Grandjot und der Dedekindsche Rekursionssatz}.\\
In: \emph{Mitteilungen der DMV}. Band~24, 2016, S.~37--45.

\bibitem{Glosauer} Tobias Glosauer:
\emph{Elementar(st)e Gruppentheorie}.
Springer, 2016.

\bibitem{Henze} Norbert Henze: \emph{Stochastik für Einsteiger}.
Springer, 1997, 12. Auflage 2018.

\bibitem{Winskel} Glynn Winskel:
\emph{The Formal Semantics of Programming Languages}.
The MIT Press, Cambridge (Massachusetts) 1993.

\bibitem{Harel} David Harel, Dexter Kozen, Jerzy Tiuryn:
\emph{Dynamic Logic}.
The MIT Press, Cambridge (Massachusetts) 2000.
\end{thebibliography}


\printindex

\end{document}


