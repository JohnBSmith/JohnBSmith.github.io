\documentclass[a4paper,10pt,fleqn,twocolumn,twoside]{article}
\usepackage[utf8]{inputenc}
\usepackage[T1]{fontenc}
\usepackage{lmodern}
\usepackage{ngerman}
\usepackage{amsmath}
\usepackage{amssymb}
\usepackage{color}
\definecolor{c1}{RGB}{60,00,40}
\usepackage[colorlinks=true,linkcolor=c1]{hyperref}
\usepackage{geometry}
\geometry{a4paper,left=25mm,right=15mm,top=20mm,bottom=28mm}
\setlength{\columnsep}{6mm}
\begin{document}
%\thispagestyle{empty}

\begin{huge}
\noindent
\textbf{Der mathematische\\
Werkzeugkasten}
\par
\end{huge}
\vspace{1em}
\noindent
John Smith, 2014--2015\\
Lizenz: CC0

\tableofcontents

%\newpage
%\section*{Vorwort}
%Dieses Heft war ursprünglich für Freunde gedacht, um ihnen
%mathematische Formalismen zu erläutern, die in der Physik
%von Bedeutung sind. Zusätzliche Themen kamen hinzu und es wuchs
%noch ein wenig an. Aus diesem Grund fehlen Beweise, sofern es sich
%nicht um einfache algebraische Umformungen handelt, denn das würde
%sehr schnell den Rahmen sprengen.


\newpage
\section{Lineare Algebra}
\subsection{Komplexe Zahlen}

Die Winkelfunktionen stehen zur Exponentialfunktion
in Beziehung. Es ist
\[\sin x = \frac{1}{2i}(\mathrm{e}^{ix}-\mathrm{e}^{-ix}),\]
\[\cos x = \frac{1}{2}(\mathrm{e}^{ix}+\mathrm{e}^{-ix}).\]
%
Mit diesen Formeln lassen sich die Ableitungen der Winkelfunktionen
gewinnen. Z.B. ist
\begin{gather*}
\sin' x = \frac{1}{2i}(\mathrm{e}^{ix}-\mathrm{e}^{-ix})'
= \frac{1}{2i}(i\mathrm{e}^{ix}+i\mathrm{e}^{-ix})\\
= \frac{1}{2}(\mathrm{e}^{ix}+\mathrm{e}^{-ix})
= \cos x.
\end{gather*}
Auch die Additionstheoreme lassen sich über die Exponentialfunktion
herleiten. Die Sinusschwingung
\[u = \hat u\sin(\omega t+\varphi_0)\]
%
lässt sich alternativ schreiben als
\[\underline u = \hat u\,\mathrm{e}^{i\omega t+i\varphi_0}.\]
%
Die ursprüngliche Schwingung ist dann gegeben durch den Imaginärteil
\[u = \mathrm{Im}(\underline u).\]
%
Die Schwingung lässt sich damit als Drehzeiger interpretieren, der
sich mit der Kreisfrequenz $\omega$ entgegen dem Uhrzeigersinn
dreht. Die ursprüngliche Sinus"=Funktion ist einfach die Projektion
auf die $y$"=Achse.

Dieser Formalismus ist in dem Sinn vorteilhaft, als dass sich hiermit
zwei Schwingungen gleicher Kreisfrequenz addieren lassen, indem man
die Drehzeiger wie Vektoren addiert. Es entsteht wieder eine
Sinus"=Schwingung mit der gleichen Kreisfrequenz.

Man betrachtet die Zeiger (welche komplexe Zahlen sind) einfach zum
Zeitpunkt $t=0$. Selbst diese Zeiger kodieren sämtliche Information
über die Schwingung, nämlich die Anfangsphase und die Amplitude. Man
muss also lediglich die Summe
\[\underline{u}_1(0)+\underline{u}_2(0)
= \hat u_1\mathrm{e}^{i\varphi_1}+\hat u_2\mathrm{e}^{i\varphi_2}\]
berechnen. Dazu ist es notwendig die komplexen Zahlen von der
Polarform in die algebraische Form umzurechnen.
Dann kann man sie einfach addieren. Die Summe muss anschließend wieder
in die Polarform umgerechnet werden. In dieser Form können Amplitude
und Anfangsphase abgelesen werden und man ist fertig.

Was haben wir jetzt gelernt? Wir haben gelernt, dass sich Schwingungen
als Vektoren in Polarkoordinaten interpretieren lassen. Will man diese
addieren, so geht man einen Weg, den man in der Mathematik häufiger
finden kann. Man transformiert erst von Polarkoordinaten in
euklidische Koordinaten, führt die Addition aus und transformiert
anschließend zurück in Polarkoordinaten.

Um dies möglichst zügig auszuführen, sollte man die
Transformationsformeln immer parat haben. Die Transformationsformeln
von Polarkoordinaten in euklidische Koordinaten sind
\begin{gather*}
x = r\cos\varphi,\\
y = r\sin\varphi.
\end{gather*}
\noindent
Die Transformationsformeln von euklidischen Koordinaten in
Polarkoordinaten sind
\begin{gather*}
r = \sqrt{x^2+y^2},\\
\varphi = \mathrm{sgn}(y)\arccos(x/r).
\end{gather*}
Die letzte Gleichung ist nur für $\varphi=\pi$ ungültig.

Zwei Dinge spielen beim Rechnen mit komplexen Zahlen eine
wesentliche Rolle. Das ist zum einen die Umwandlung von kartesischen
Koordinaten in Polarkoordinaten und umgekehrt. Zum anderen ist es
die Verwendung der eulerschen Formel.

Man kann die Berechnung von Funktionswerten der komplexen
Exponentialfunktion auf die Berechnung von Funktionswerten
der reellen Exponentialfunktion zurückführen.
Sei $a:=\mathrm{Re}(z)$ und $b:=\mathrm{Im}(z)$. Es ist
\[e^z = e^{(a+bi)} = e^a e^{bi} = e^a(\cos b+i\sin b).\]
Damit hat man also $\mathrm{Re}(e^z) = e^a\cos b$ und
$\mathrm{Im}(e^z) = e^a\sin b$ gewonnen. Realteil und Imaginärteil
können separat ausgerechnet werden, ohne komplexe Zahlen benutzen
zu müssen. Die eulersche Formel hat uns diese Zurückfüh"-rung
ermöglicht.

Die Berechnung des komplexen Logarithmus lässt sich ebenfalls
auf die Berechnung des reellen Logarithmus zurückführen. Dazu stellt
man $z$ in Polarform dar. Sei also $r=|z|$ und
$\varphi = \mathrm{arg}(z)$. Es ist
\begin{gather*}
w = \ln(z) = \ln(re^{i\varphi})
= \ln(r)+\ln(e^{i\varphi})\\
= \ln(r)+i\varphi.
\end{gather*}
Ersetzt man $\varphi$ gegen $\varphi+2\pi$, so erhält man
ebenfalls $z$, jedoch ein anderes $w$. Das Ergebnis der
Logarithmusfunktion ist also nicht eindeutig. Sie hat unendlich
viele sogenannte Zweige. Man verlangt nun $-\pi<\varphi$ und
$\varphi\le\pi$. Dann definiert man den Hauptzweig als
\[\ln(z):=\ln(r)+i\varphi.\]
Die mengenwertige Logarithmusfunktion mit allen Zweigen ist dann
\[\mathrm{Ln}(z):=\ln(z)+2\pi ki.\quad(k\in\mathbf Z)\]
Man muss beachten, dass die Regeln $\ln(xy)=\ln(x)+\ln(y)$
und $\ln(x^y)=y\ln(x)$ im Allgemeinen nicht mehr gültig sind.
Die hier dargestellte Rechnung für den Logarithmus funktioniert aber.
Man kann sich durch die Probe überzeugen. Es ist
\[e^w = e^{\ln(r)+i\varphi+2\pi ki}
= e^{\ln(r)}e^{i\varphi}e^{2\pi ki} = re^{i\varphi} = z.\]

\subsection{Das Skalarprodukt}

Die Menge der komplexen Zahlen mit der Addition zusammen ist
isomorph zum Vektorraum
$\mathbf{R}^2 = \mathbf{R}\times\mathbf{R}$.
Das heißt \textit{alles}, was man mit Vektoren machen kann, ist auch
mit komplexen Zahlen möglich. Zu jedem Punkt in der Menge
$\mathbf{R}\times\mathbf{R}$ gehört genau ein Vektor, der vom
Koordinatenursprung auf diesen Punkt zeigt. Zu jedem Punkt gehört
außerdem genau eine komplexe Zahl.

Mit einer Orthonormalbasis $B=(\underline e_1,\underline e_2)$
können wir jeden Vektor als
\[\underline a = a_1\underline e_1+a_2\underline e_2\]
darstellen. Dabei sind $\underline e_1$ und $\underline e_2$ die
Basisvektoren. Diesen Vektor kann man als komplexe Zahl
\[\underline a = a_1+a_2i\]
interpretieren. Die Basisvektoren sind 1 und $i$.

Wenn wir eine Orthonormalbasis verwenden, so ist das Skalarprodukt
zweier Vektoren
\[\langle \underline a, \underline b\rangle = a_1b_2+a_2b_2.\]
\noindent
Allgemeiner gilt die Formel
\[\langle \underline a, \underline b\rangle
= |\underline a||\underline b|\cos\varphi\]
wobei $\varphi$ der Winkel zwischen den beiden Vektoren ist.
Die Betrag eines Vektors ist
\[|\underline a| = \sqrt{\langle \underline a, \underline a\rangle}
= \sqrt{a_1^2+a_2^2}.\]
Der Betrag eines Vektors ist die Länge dieses Vektors. Die Formel wird
durch den Satz des Pythagoras motiviert. Der Betrag einer komplexen
Zahl stimmt mit dem Betrag des Vektors überein, wenn man die komplexe
Zahl als Vektor interpretiert. Es ist der Abstand vom
Koordinatenursprung zu dem Punkt, welcher durch die komplexe Zahl
gegeben ist.

Der Abstand von zwei Punkten $\underline a$ und $\underline b$
ist gegeben durch
\[d = |\underline a-\underline b|.\]
\noindent
Den hier vorgestellten Formalismus kann man nun sofort auf den
$n$"=dimen"-sionalen euklidischen Vektorraum $\mathbf{R}^n$
verallgemeinern. Jeder Vektor lässt sich als Linearkombination
\[\underline a = a_1\underline e_1+\dots+a_n\underline e_n\]
darstellen. Das Skalarprodukt ist
\[\langle \underline a, \underline b\rangle = a_1b_2+\dots+a_nb_n.\]
Die Formeln
$\langle \underline a, \underline b\rangle
= |\underline a||\underline b|\cos\varphi$
und
$|\underline a| = \sqrt{\langle \underline a, \underline a\rangle}$
sowie $d = |\underline a-\underline b|$
bleiben weiterhin gültig.

Für das Skalarprodukt gibt es wichtige Rechenregeln. Es ist kommutativ
und bilinear, jedoch nicht assoziativ.
Was bedeutet bilinear? Nun ja, es bedeutet, dass eine Funktion linear
in jedem ihrer zwei Argumente ist. Allgemeiner ist multilinear auf
diese Art definiert. Der Ableitungsoperator ist linear. Das bestimmte
Integral ist linear. Das Summenzeichen ist linear. Proportionale
Funktionen sind linear. Lineare Funktionen sind in diesem Sinn im
Allgemeinen aber nicht linear.

Lineare Differentialgleichungen heißen linear,
weil sie sich in der Form
\[Df(x)=g(x)\]
schreiben lassen. Dabei ist $g$ eine gegebene Funktion und $f$
die gesuchte Funktion. Der Operator $D$ ist ein linearer
Differentialoperator. Z.B. lässt sich
\[f''(x) + 2f'(x) + f(x) = 4\]
auch als
\[Df(x) = \left(\frac{\mathrm d^2}{\mathrm dx^2}
+2\frac{\mathrm d}{\mathrm dx}
+1\right) f(x) = 4\]
schreiben.

\subsection{Das äußere Produkt}

Ein weiteres wichtiges Produkt ist das äußere Produkt.
Es lässt sich mit der Formel
\[\underline a\wedge\underline b
= (a_1b_2-a_2b_1)\underline e_1\wedge\underline e_2\]
berechnen. Das äußere Produkt ist antikommutativ, es ist
\[\underline a\wedge\underline b = -\underline b\wedge\underline a.\]
\noindent
Damit ergibt sich $\underline a\wedge\underline a = 0$.
Allgemeiner ist
$\underline a\wedge\underline b = 0$,
wenn $\underline a$ und $\underline b$ kollinear sind.
Zwei Vektoren sind kollinear, wenn sie in die selbe Richtung oder
die entgegengesetzte Richtung zeigen.

Das äußere Produkt ist außerdem bilinear und assoziativ.
Diese Rechenregeln genügen, um mit dem äußeren Produkt zu rechnen.
Ein Beispiel.
\begin{gather*}
\underline a\wedge\underline b
= (a_1e_1+a_2e_2) \wedge (b_1e_1+b_2e_2)\\
= a_1e_1\wedge (b_1e_1+b_2e_2) + a_2e_2\wedge (b_1e_1+b_2e_2)\\
= a_1e_1\wedge b_1e_1+a_1e_1\wedge b_2e_2\\
\quad + a_2e_2\wedge b_1e_1+a_2e_2\wedge b_2e_2\\
= 0+a_1e_1\wedge b_2e_2 + a_2e_2\wedge b_1e_1+0\\
= a_1b_2e_1\wedge e_2 + a_2b_1 e_2\wedge e_1\\
= a_1b_2e_1\wedge e_2 - a_2b_1 e_1\wedge e_2\\
= (a_1b_2-a_2b_1)e_1\wedge e_2
\end{gather*}
\noindent
Für das äußere Produkt gilt die Formel
\[\underline a\wedge\underline b= |\underline a||\underline b|
(\sin\varphi)\,\underline e_1\wedge\underline e_2.\]
\noindent
Das äußere Produkt ist der Flächeninhalt des Parallelogramms,
welches die beiden Vektoren aufspannen.

\subsection{Das geometrische Produkt}

Das geometrische Produkt ist definiert als
\[\underline a\underline b =
\langle\underline a,\underline b\rangle
+\underline a\wedge\underline b.\]
\noindent
Das geometrische Produkt ist bilinear und assoziativ. Es ist jedoch
weder kommutativ noch antikommutativ.
Für rechtwinklige Vektoren reduziert sich das geometrische Produkt
auf das äußere Produkt. Für kollineare Vektoren reduziert sich das
geometrische Produkt auf das Skalarprodukt.
Wenn man eine Orthonormalbasis $B=(e_1,e_2)$ wählt, gilt daher
\begin{gather*}
e_1e_2 = -e_2e_1,\\
e_1e_1=1,\\
e_2e_2=1.
\end{gather*}
\noindent
Allgemeiner rechnet man bei Orthonormalbasen nach dem folgenden
Prinzip. Führt man z.B. in
\[e_1e_3e_4e_1e_2e_2\]
eine Transposition von zwei \textit{ungleichen} Basisvektoren aus,
so ändert sich das Vorzeichen. Außerdem darf
man wegen des Assoziativgesetzes die Klammern beliebig setzen.
Zwei nebeneinander liegende gleiche Basisvektoren heben sich zu
eins auf. Z.B. ist
\begin{gather*}
e_1e_3e_4e_1e_2e_2 = e_1e_3e_4e_1\\
= -e_1e_1e_4e_3 = -e_4e_3 = e_3e_4.
\end{gather*}
Diese Vereinfachungen muss man ständig machen, wenn man mit dem
geometrischen Produkt rechnet. Multivektoren der Form
\[a+b\,e_1e_2\]
sind isomorph zu den komplexen Zahlen.
Es ist $e_1e_2 = e_1\wedge e_2 = i$. Z.B. ist
\begin{gather*}
i^2 = (e_1e_2)^2 = (e_1e_2)(e_1e_2) = e_1e_2e_1e_2\\
= -e_1e_1e_2e_2 = -1.
\end{gather*}
\noindent
Mit der eulerschen Formel kann man schreiben
\begin{gather*}
\underline a\underline b =
\langle\underline a,\underline b\rangle
+\underline a\wedge\underline b
= |\underline a||\underline b|\cos\varphi
+ e_1e_2|\underline a||\underline b|\sin\varphi\\
= |\underline a||\underline b|\cos\varphi
+ i|\underline a||\underline b|\sin\varphi
= |\underline a||\underline b|e^{i\varphi}.
\end{gather*}
\noindent
Mit dem äußeren Produkt können wir einen Formalismus aufbauen,
der alle assoziativen hyperkomplexen Zahlensysteme zu einem
einzigen System zusammenfasst.

Was das soll? Nun ja, man kann jetzt austesten, was passiert,
wenn man einen gewöhnlichen Vektor mit der imaginären Einheit
multipliziert. Es ist
\begin{gather*}
i(a_1e_1+a_2e_2) = e_1e_2(a_1e_1+a_2e_2)\\
= a_1e_1e_2e_1+a_2e_1e_2e_2 = -a_1e_1e_1e_2+a_2e_1\\
= -a_1e_2+a_2e_1 = a_2e_1-a_1e_2.
\end{gather*}
\noindent
bzw.
\[i\begin{pmatrix}a_1\\ a_2\end{pmatrix}
= \begin{pmatrix}a_2\\ -a_1\end{pmatrix}.\]
\noindent
Der Vektor wird um 90~Grad im Uhrzeigersinn gedreht. Das geometrische
Produkt ist nicht kommutativ. Wenn man von der anderen Seite mit $i$
multipliziert, so dreht man gegen den Uhrzeigersinn. Bei komplexen
Zahlen ist es egal, weil die Multiplikation ja kommutativ ist. Es ist
\begin{gather*}
i(a_1+a_2i) = e_1e_2(a_1+a_2e_1e_2)\\
= a_1e_1e_2-a_2 = a_1i-a_2 = -a_2+a_1i
\end{gather*}
\noindent
Die komplexe Zahl wird um 90~Grad gegen den Uhrzeigersinn gedreht.
Multipliziert man von der anderen Seite mit $i$, so erhält man das
selbe Ergebnis. Die komplexen Zahlen sind ein Spezialfall, für den
das geometrische Produkt kommutativ ist, obwohl es im Allgemeinen
nicht kommutativ ist.

Der Isomorphismus, der einen Vektor in eine komplexe Zahl überführt
ist das geometrische Produkt mit $e_1$. Denn es ist
\begin{gather*}
e_1\underline a = e_1(a_1e_1+a_2e_2) = a_1e_1e_1 + a_2e_1e_2\\
= a_1 + a_2e_1e_2 = a_1+a_2i.
\end{gather*}
\noindent
Was macht die komplexen Zahlen eigentlich aus? Bezüglich der Addition
sind es nur ganz gewöhnliche Vektoren. Völlig uninteressant.
Das besondere ist die seltsame Multiplikation. Mit der Multiplikation
von zwei können wir die Länge einer Zahl (interpretiert als Vektor vom
Ursprung auf einen Punkt im Zahlenstrahl) verdoppeln. Durch die Multiplikation
mit einer komplexen Zahl können wir eine Zahl nicht nur verlängern,
wir können sie auch Drehen.

Wenn man einen Vektorraum mit einer bilinearen Multiplikation
ausstattet, entsteht etwas, das man auch als Algebra bezeichnet.
Der euklidische Vektorraum zusammen mit dem geometrischen Produkt
wird dadurch zu einer assoziativen Algebra, die Clifford-Algebra
genannt wird. Mit dem geometrischen Produkt können wir Vektoren
drehen.

Das Vektorprodukt lässt sich mit
\[\underline a\times\underline b = (\underline a\wedge\underline b)I
= (\underline a\wedge\underline b)e_1e_2e_3\]
\noindent
berechnen. Diese Formel gilt nur, wenn man keine Rauminversion
zulässt. Das Ergebnis des Vektorproduktes ist eigentlich ein
sogenannter Pseudovektor. Dieser entsteht, weil man die
Bilinearität des Vektorproduktes fordert.

Ein Vektor ändert bei Rauminversion im $\mathbf R^3$ sein
Vorzeichen, ein Pseudovektor tut das nicht. Rauminversion bedeutet,
dass jeder Basisvektor $e_k$ gegen $-e_k$ ausgetauscht wird.
Das rechtshändige Koordinatensystem wird dadurch zu einem
linkshändigen. Bei einem Vektor ist nun
\[a_1(-e_1)+a_2(-e_2)+a_3(-e_3) = -\underline a.\]
Für $I$ bekommt man
\[(-e_1)(-e_2)(-e_3) = -e_1e_2e_3 = -I.\]
Das Vektorprodukt ist bilinear. Man erhält
\[(-\underline a)\times (-\underline b)
= (-1)(-1)\underline a\times\underline b
= \underline a\times\underline b.\]
Eigentlich müsste man aber
\[(-\underline a)\wedge (-\underline b)(-I)
= -(\underline a\wedge\underline b)I\]
erhalten. Das äußere Produkt ist auch bilinear. Allerdings besteht
hier kein Problem, weil ja die Basisbivektoren
$e_i\wedge e_j = (-e_i)\wedge(-e_j)$ im Ergebnis des Produktes
enthalten sind.

Abschließend sollen noch einige nützliche Definitionen formuliert
werden. Der Projektionsoperator $[A]_k$ projiziert den Anteil
vom Grad $k$ heraus. Vom Grad null sind Skalare, vom Grad eins
sind Vektoren und Kovektoren, vom Grad zwei sind Bivektoren
usw. Bei negativem $k$ kann man $[A]_k=0$ setzen.
Es ist z.B.
\begin{gather*}
[a+be_1+ce_1e_2]_0 = a,\\
[a+be_1+ce_1e_2]_1 = be_1,\\
[a+be_1+ce_1e_2]_2 = ce_1e_2.
\end{gather*}
Man definiert nun
\begin{gather*}
A\wedge B := \sum\nolimits_{i,j} [[A]_i[B]_j]_{i+j},\\
A\,\big\lrcorner\,B := \sum\nolimits_{i,j} [[A]_i[B]_j]_{j-i},\\
\langle A,B\rangle := \sum\nolimits_{i,j} [[A]_i[B]_j]_0,\\
A\,\big\llcorner\,B := \sum\nolimits_{i,j} [[A]_i[B]_j]_{i-j}.
\end{gather*}
Man projiziert Anteile heraus, bildet von diesen das geometrische
Produkt und projiziert dann wieder der Formel entsprechend den Anteil
heraus. Darüber wird aufsummiert.

Bei $A\wedge B$ handelt es sich um die Verallgemeinerung des
äußeren Produktes.
Das Produkt $A\,\big\lrcorner\,B$ wird Linkskontraktion genannt
und ist eine Verallgemeinerung des Skalarproduktes.
Das Produkt $A\,\big\llcorner\,B$ wird Rechtskontraktion genannt
und ist auch eine Verallgemeinerung des Skalarproduktes.
Das Produkt $\langle A,B\rangle$ ist eine Verallgemeinerung des
Skalarproduktes, die auch tatsächlich immer einen Skalar liefert.

Es gibt auch die Schreibweise
$i_{\!X} \omega := X\big\lrcorner\,\omega$. Mit der
Rechtskontraktion würde das keinen Sinn ergeben, weil $X$ ein
Vektor und $\omega$ z.B. vom Grad zwei sein soll. Mit der
Rechtskontraktion würde man dann einen negativen Index zur
Projektion erhalten.

In der Physik sind die Pauli-Matrizen von Bedeutung. Es ist
\[
\sigma_1 := \begin{bmatrix}0 & 1\\ 1 & 0\end{bmatrix},\quad
\sigma_2 := \begin{bmatrix}0 & -i\\ i & 0\end{bmatrix},\quad
\sigma_3 := \begin{bmatrix}1 & 0\\ 0 & -1\end{bmatrix}.
\]
Für die Multiplikation von Pauli-Matrizen muss man sich
eine Multiplikationstabelle merken. Wenn man das geometrische
Produkt zur Multiplikation verwendet, dann entsprechen
die Pauli-Matrizen $\sigma_k$ jedoch einfach den
Basisvektoren $e_k$. Die Einheitsmatrix entspricht der Zahl eins
und $i$ entspricht $e_1e_2e_3$. Der Vorteil ist,
dass man sich keine Multiplikationstabelle merken muss.

Verwendet man die Basis des Minkowskiraums mit dem metrischen Tensor
$g=\mathrm{diag}(1,-1,-1,-1)$, so lassen sich die Dirac-Matrizen
$\gamma^k$ ebenfalls gegen die Basisvektoren $e^k$ ersetzen.
Die Basisvektoren bilden jedoch keine Orthonormalbasis, man rechnet
stattdessen $(e_k)^2=g^{kk}$ mit $g^{kk}=g_{kk}$. Das Produkt
von Dirac-Matrizen entspricht dem geometrischen Produkt von
Basisvektoren. Da die Basisvektoren eine Orthogonalbasis bilden,
ist das geometrische Produkt der Basisvektoren antikommutativ.


\subsection{Das Tensorprodukt}

Das Tensorprodukt ist bilinear und assoziativ. Jedoch ist es
weder kommutativ noch antikommutativ. Mit diesen Rechenregeln
ist es möglich das Tensorprodukt von zwei Vektoren auszurechnen.
Es ergibt sich
\begin{gather*}
\underline a\otimes\underline b
= (a_1e_1+a_2e_2)\otimes (b_1e_1+b_2e_2)\\
= a_1e_1\otimes (b_1e_1+b_2e_2)
+ a_2e_2\otimes (b_1e_1+b_2e_2)\\
= a_1e_1\otimes b_1e_1+a_1e_1\otimes b_2e_2\\
+\; a_2e_2\otimes b_1e_1+a_2e_2\otimes b_2e_2\\
= a_1b_1 e_1\otimes e_1+a_1b_2 e_1\otimes e_2\\
+\; a_2b_1 e_2\otimes e_1+a_2b_2 e_2\otimes e_2.
\end{gather*}
\noindent
Das Tensorprodukt macht aus zwei Vektoren eine Matrix, die als Tensor
bezeichnet wird. Es ist
\[\begin{bmatrix}a_1\\ a_2\end{bmatrix}\otimes
\begin{bmatrix}b_1\\ b_2\end{bmatrix}
=\begin{bmatrix}
a_1b_1 & a_1b_2\\
a_2b_1 & a_2b_2
\end{bmatrix}.\]
\noindent
Dabei ist folgendes wichtig. Der Tensor selbst ist wie ein Vektor
unabhängig von der gewählten Basis. Wechselt man die Basis gegen
eine andere aus, so ändern sich die Einträge der Matrix. Der Tensor
selbst bleibt aber der gleiche.

Es verhält sich wie bei physikalischen Größen.
In einer anderen Einheit ergibt sich ein anderer Zahlenwert.
Die Größe, welche aus Zahlenwert und Einheit besteht, bleibt aber
gleich.

Sei $B=(e_1,e_2)$ eine Basis und sei $B'=(e_1',e_2')$ eine zweite
Basis. Ein und der selbe Vektor kann in zwei unterschiedlichen Basen
ausgedrückt werden. Seien $a_k$ die Koordinaten bezüglich der
Basis $B$ und $a_k'$ die Koordinaten bezüglich der Basis $B'$.
Es ist
\[\underline a = a_1e_1+a_2e_2 = a_1'e_1'+a_2'e_2'.\]
Dabei stellt sich natürlich die Frage, wie man von den alten
Koordinaten auf die neuen kommt. Dieses Problem wird unter dem
Stichwort Basiswechsel gelöst. Bei Tensoren kann man auch einen
Basiswechsel machen.

Nun ist es auch möglich
$\underline a\otimes\underline b\otimes \underline c$
zu bilden. Wie soll man sich das Ergebnis vorstellen? Nun ja,
es ist eine dreidimensionale Tabelle. Man kann sich ein Regal mit
Zeilen und Spalten vorstellen, bei dem man in jedem Fach noch
schrittweise in die Tiefe gehen kann.

Tensoren bilden mit der Addition wieder einen Vektorraum.
Alles was man mit Vektoren tun kann ist also auch mit Tensoren
möglich.

Anstelle von
\[T = \sum_{i,j} T_{ij}\,e_i\otimes e_j\]
schreibt man einen Tensor auch kürzer als $T_{ij}$.
Dabei hat man sich natürlich stillschweigend auf die
Basis $B=(e_1,e_2)$ festgelegt.
Wir wollen diese Kurzschreibweise im Folgenden auch verwenden.
Das Tensorprodukt wird nun kürzer
\[T_{ij} = a_ib_j\]
geschrieben. Wir können ein Tensorprodukt aus beliebig vielen
Vektoren bilden. Z.B.
\[T_{ijkl} = a_ib_jc_kd_l.\]
Auch aus Tensoren lassen sich Tensorprodukte bilden. Z.B.
\[T_{ijkl} = A_{ij}B_{kl}.\]
Man kann einen Tensor nun kontrahieren.
Bei der Kontraktion setzt man zwei Indizes gleich und summiert
anschließend über diesen Index auf. Die Kontraktion von $T_{ij}$
ist z.B.
\[\sum_{k} T_{kk} = T_{11}+T_{22}.\]
Das Skalarprodukt ist ein Spezialfall von Tensorprodukt gefolgt von
Kontraktion. Das Skalarprodukt von $a_i$ und $b_j$ ist
\[\sum_{k} a_kb_k.\]
Wenn wir keine Orthonormalbasis verwenden, dann lässt sich das
Skalarprodukt so nicht bilden. Man muss dann zuerst einen
Basiswechsel in eine Orthonormalbasis machen oder man verwendet
den Formalismus mit dem metrischen Tensor.

Der Formalismus mit dem metrischen Tensor lässt sich allgemein
bei Kontraktionen verwenden. Wenn man keine Orthonormalbasis hat,
muss man bei der Kontraktion bloß darauf achten, dass ein Index
oben und der andere unten steht. Ein beliebiger Index lässt sich
mit dem metrischen Tensor senken und mit dem inversen metrischen
Tensor heben.

Die Multiplikation von Matrizen ist ein Spezialfall
von Tensorprodukt gefolgt von Kontraktion.
Seien $a_{ik}$ und $b_{lj}$
zwei Matrizen. Das Produkt ist
\[c_{ij} = \sum_{k}a_{ik}b_{kj}.\]
%
Ein antisymmetrischer Tensor ist ein Tensor mit der Eigenschaft
$T_{ij}=-T_{ji}$. Die Hauptdiagonale von einem antisymmetrischen
Tensor besteht zwingend aus Nullen.
Von zwei Vektoren $\underline a,\underline b$ kann man
einen antisymmetrischen Tensor auf folgende Art bilden. Man rechnet
\begin{gather*}\underline a\otimes\underline b
- \underline b\otimes \underline a
= \begin{bmatrix}
a_1 b_1 & a_1 b_2\\
a_2 b_1 & a_2 b_2
\end{bmatrix}
- \begin{bmatrix}
a_1 b_1 & a_2 b_1\\
a_1 b_2 & a_2 b_2
\end{bmatrix}\\
= \begin{bmatrix}
0 & (a_1 b_2-a_2 b_1)\\
(a_2 b_1 - a_1 b_2) & 0
\end{bmatrix}\\
= (a_1 b_2 - a_2 b_1) \begin{bmatrix}
0 & 1\\
-1 & 0
\end{bmatrix}\\
= (a_1 b_2 - a_2 b_1) (e_1\otimes e_2 - e_2\otimes e_1).
\end{gather*}
Mit der Definition
\[e_i\wedge e_j := e_i\otimes e_j - e_j\otimes e_i\]
erhält man
\[\underline a\wedge\underline b
= \underline a\otimes\underline b
- \underline b\otimes \underline a.\]
Die Formel ist auch für Vektoren mit mehr als zwei Komponenten gültig.
Das bestätigt man durch die Rechnung
\begin{gather*}
\underline a\otimes\underline b
- \underline b\otimes \underline a
= \sum_{i,j} a_ib_j\,e_i\otimes e_j
- \sum_{i,j} b_ia_j\,e_i\otimes e_j\\
= \sum_{i,j} (a_ib_j-a_jb_i)\,e_i\otimes e_j\\
= \sum_{i<j} (a_ib_j-a_jb_i)(e_i\otimes e_j-e_j\otimes e_i)\\
= \sum_{i<j} (a_ib_j-a_jb_i)\,e_i\wedge e_j
= \underline a\wedge\underline b.
\end{gather*}
Ab jetzt sollen Vektoren nicht mehr unterstrichen werden.
Außerdem sollen die Vektoren jetzt mit einem Index nummeriert
werden. Das ist leider etwas verwirrend, da $a_k$ bisher Komponenten
waren und ab jetzt Vektoren sind.
Man definiert
\[\mathrm{Alt}_n(a_1\otimes\ldots\otimes a_n)
:= \frac{1}{n!}\sum_{\sigma\in S_n}\mathrm{sgn}(\sigma)\,
a_{\sigma(1)}\otimes\ldots\otimes a_{\sigma(n)}.\]
Für Vektoren $a_k$ gilt allgemein die Formel
\[a_1\wedge\ldots\wedge a_n
= n!\,\mathrm{Alt}_n(a_1\otimes\ldots\otimes a_n).\]
Wenn man das Epsilon-Symbol verwendet, dann kann man auch alternativ
schreiben
\[\mathrm{Alt}_n (a_1\otimes\ldots\otimes a_n)
= \frac{1}{n!} \sum_{i_1,\ldots,i_n} \varepsilon_{i_1\ldots i_n}
a_{i_1}\otimes\ldots\otimes a_{i_n}.\]
An dieser Stelle soll noch festhalten werden,
dass für einen antisymmetrischen Tensor
gilt
\[A = \sum_{i,j}a_{ij}e_i\otimes e_j
= \frac{1}{2}\sum_{i,j}a_{ij} e_i\wedge e_j
= \sum_{i<j}a_{ij}e_i\wedge e_j.\]
Allgemeiner ist
\begin{gather*}
A = \sum_{i_1,\ldots,i_n}\!\! a_{i_1,\ldots,i_n}
e_{i_1}\!\otimes\ldots\otimes e_{i_n}\\
= \frac{1}{n!} \sum_{i_1,\ldots,i_n}\!\! a_{i_1,\ldots,i_n}
e_{i_1}\!\wedge\ldots\wedge e_{i_n}\\
= \sum_{i_1<\ldots<i_n}\!\! a_{i_1,\ldots,i_n}
e_{i_1}\!\wedge\ldots\wedge e_{i_n}.
\end{gather*}
Für einen antisymmetrischen Tensor mit den Komponenten
$a[i_1,\ldots,i_n]:=a_{i_1\ldots i_n}$ gilt
\[a[i_1,\ldots,i_n]=\mathrm{sgn}(\sigma)
a[\sigma(i_1),\ldots,\sigma(i_n)].\]
%
Ein Tensor lässt sich nun auf einen Vektor anwenden, indem immer
über den letzten Index kontrahiert wird. Für
$A:=\sum_{ij} a_{ij}e_i\otimes e_j$ und $v:=\sum_k v_ke_k$
hat man nun
\[A(v) = \langle A,v\rangle = \sum_i a_{ik}v_k e_i.\]
Somit gilt
$\langle e_i\otimes e_j, e_k\rangle = e_i\delta_{jk}$.
Weiterhin gilt
\begin{gather*}
T(v,w) = \langle\langle T,w\rangle, v\rangle,\\
T(u,v,w) = \langle\langle\langle T,w\rangle,v\rangle,u\rangle
\end{gather*}
usw. Allgemein gilt dann die Formel
\[\langle T\otimes e_i,e_j\rangle = T\delta_{ij}.\]
Sinnvollerweise definiert man auch
\[\langle e_i,e_j\otimes T\rangle := T\delta_{ij}.\]
Für einen Tensor $A$ zweiter Stufe ergibt sich dann
\[\langle A,v\rangle = \langle v,A^T\rangle.\]
Einen Tensor zweiter Stufe transponiert man wie eine Matrix, d.h. man definiert
\[(\sum_{ij}a_{ij}e_i\otimes e_j)^T:=\sum_{ij} a_{ji}e_i\otimes e_j.\]
Allgemein verwendet man eine Permutation $\sigma$, welche die
Indizes permutiert. Man notiert am besten $T^\sigma$. Speziell
ist dann $A^T=A^{(21)}$.


\subsection{Allgemeine Basen}

Bei Orthonormalbasen stehen die Basisvektoren jeweils senkrecht
aufeinander und haben den Betrag eins. Die Basisvektoren einer
Orthonormalbasis $B=(e_1,e_2)$ wollen wir mit $e_k$ bezeichnen.

Wir lassen diese Forderung nun fallen und bezeichnen mit $f_k$ die
Basisvektoren einer allgemeinen Basis $F=(f_1,f_2)$. Es ist
möglich, den Vektor $f_k$ als Linearkombination aus den
$e_k$ auszudrücken. Es ist
\begin{gather*}
f_1 = f_{11}e_1+f_{12}e_2,\\
f_2 = f_{21}e_1+f_{22}e_2.
\end{gather*}
\noindent
Das können wir kompakt zusammenfassen zu
\[f_k = f_{k1}e_1+f_{k2}e_2\]
\noindent
bzw. noch kompakter zu
\[f_k = \sum_{i}f_{ki}e_i.\]
\noindent
Wir wollen jetzt ein Skalarprodukt berechnen. Es ergibt sich
\begin{gather*}
\langle\underline a,\underline b\rangle
= \langle a_1f_1+a_2f_2,b_1f_1+b_2f_2\rangle\\
= \langle a_1f_1,b_1f_1\rangle
+ \langle a_1f_1,b_2f_2\rangle\\
+\;\langle a_2f_2,b_1f_1\rangle
+ \langle a_2f_2,b_2f_2\rangle\\
= a_1b_1\langle f_1,f_1\rangle
+ a_1b_2\langle f_1,f_2\rangle\\
+\;a_2b_1\langle f_2,f_1\rangle
+ a_2b_2\langle f_2,f_2\rangle\\
= a_1b_1 g_{11} + a_1b_2 g_{12}
+ a_2b_1 g_{21} + a_2b_2 g_{22}
\end{gather*}
\noindent
Die Skalarprodukte $g_{ij} = \langle f_i,f_j\rangle$ kann man
zum metrischen Tensor zusammenfassen. Der metrische Tensor ist also
die Matrix
\[g = \begin{bmatrix}
g_{11} & g_{12}\\
g_{21} & g_{22}
\end{bmatrix}
= \begin{bmatrix}\langle f_1,f_1\rangle & \langle f_1,f_2\rangle\\
\langle f_2,f_1\rangle & \langle f_2,f_2\rangle
\end{bmatrix}.\]
\noindent
Mit $\underline a' = a_1e_1+a_2e_2$ wollen wir den
Koordinatenvektor zum Vektor $\underline a = a_1f_1+a_2f_2$
bezeichnen. Wir können das Skalarprodukt nicht mehr mit
\[\langle\underline a',\underline b'\rangle = \sum_{i}a_ib_i\]
\noindent
berechnen, sondern müssen es mit
\[\langle\underline a,\underline b\rangle
= \langle g\underline a',\underline b'\rangle
= \sum_{i,j} g_{ij}a_ib_j\]
\noindent
berechnen. Ab jetzt wollen wir die Indizes aller Koordinaten
nach oben stellen und schreiben
\begin{gather*}
\underline a = a^1f_1+a^2f_2,\\
\underline a'= a^1e_1+a^2e_2.
\end{gather*}
\noindent
Zu den Koordinaten $a^k$ des Vektors $\underline a$ bezüglich der
Basis $F$ gibt es duale Koordinaten $a_k$. Der Duale
Koordinatenvektor von $\underline a'$ ist $g\underline a'$.
Das bedeutet der Koordinatenvektor wird mit dem metrischen Tensor
multipliziert. Es ist also
\[a_k = \sum_{k,i}g_{ki}a^i.\]
\noindent
Außerdem gibt es eine zu $F$ duale Basis $F^\ast=(f^1,f^2)$.
Es ist
\begin{gather*}
f_k = \sum_{k,i}g_{ki}f^k,\\
f^k = \sum_{k,i}g^{ki}f_k.
\end{gather*}
\noindent
Dabei ist $g^{-1} = (g^{ij})$ die inverse Matrix von
$g=(g_{ij})$.\\
Hierfür gibt es die Formel
\[g^{-1} = \begin{bmatrix}
g^{11} & g^{12}\\
g^{21} & g^{22}
\end{bmatrix}
= \frac{1}{g_{11}g_{22}-g_{12}g_{21}}
\begin{bmatrix}
g_{22} & -g_{12}\\
-g_{21} & g_{11}
\end{bmatrix}.\]
\noindent
Mit diesem Formalismus lässt sich das Skalarprodukt auch durch
\[\langle\underline a,\underline b\rangle = a^1b_1+a^2b_2\]
\noindent
berechnen. Allgemeiner ist
\[\langle\underline a,\underline b\rangle
= \sum_{i} a^ib_i = \sum_{i,j}g_{ij}a^ib^j.\]
\noindent
Ein Vektor lässt sich sowohl als Linearkombinationen mit der Basis
als auch als Linearkombinationen mit der Dualbasis ausdrücken. Es ist
\[\underline a = a^1f_1+a^2f_2 = a_1f^1+a_2f^2.\]

\subsection{Basiswechsel}

Sei $B=(b_1,b_2)$ eine Basis und $B'=(b_1',b_2')$ eine zweite
Basis. Seien $x_k$ die Koordinaten des Vektors $\underline v$
bezüglich der Basis $B$ und $x_k'$ die Koordinaten bezüglich der
Basis $B'$. Mit $\underline x$ wollen wir den Koordinatenvektor
der Koordinaten $x_k$ bezeichnen. Das sind die Koordinaten $x_k$
als Tupel geschrieben, jedoch nicht der
eigentliche Vektor $\underline v$. Es stellt sich die Frage, wie man
von den Koordinaten $x_k$ zu den Koordinaten $x_k'$ gelangt.

Die Basisvektoren können als Linearkombination aus den gestrichenen
Basisvektoren ausgedrückt werden. Es ist
\[b_k = \sum_{i} a_{ik}b_i'.\]
Man kann das auch als Matrizenmultiplikation schreiben.
Sei $A=(a_{ij})$. Da man bei der Matrizenmultiplikation über den
zweiten Index summiert, müssen wir die Matrix transponieren. Kompakter
schreibt man nun
\[B^T = A^TB'^T.\]
Mit $B^T$ ist der Spaltenvektor zu $B$ gemeint, da es sich bei
bei $B$ um einen Zeilenvektor handelt.
Daraus erhält man mit der allgemeinen Regel $(AB)^T = B^TA^T$ auch
\[B = B'A.\]
Den Vektor können wir sowohl in der alten als auch in der neuen
Basis ausdrücken.
Es ist
\[\underline v = \sum_{k}x_kb_k = \sum_{k}x_k'b_k'.\]
Man ersetzt die $b_k$ nun und erhält
\begin{gather*}
\underline v = \sum_{k}x_k\sum_{i}a_{ik}b_i'
= \sum_{i}\sum_{k}a_{ik}x_k b_i'\\
= \sum_{k}\sum_{i}a_{ki}x_i b_k'.
\end{gather*}
Durch einen Koeffizientenvergleich stellt man fest
\[x_k' = \sum_{i} a_{ki}x_i.\]
Das kann man kompakt als
\[\underline x' = A\underline x.\]
formulieren. Die gesamte Herleitung lässt sich genauso kompakt
schreiben. Unter Verwendung von $B=B'A$ ergibt sich
\[\underline v = B'\underline x'
= B\underline x = B'A\underline x.\]
Durch einen Koeffizientenvergleich oder durch Multiplikation mit
$B'^{-1}$ erhält man wieder
\[\underline x' = A\underline x.\]
Wenn wir also $A$ kennen, so ist unser Problem schon gelöst.
Kennen wir $A^{-1}$, so müssen wir daraus noch $A$ bestimmen.

Es kann auch der Fall bestehen, dass man die Koordinaten der
Basisvektoren kennt. Man kann dann $B$ auch zu einer Matrix
expandieren. Es ist
\[B = (b_1,b_2)
= (\begin{bmatrix}b_{11}\\ b_{21}\end{bmatrix},
\begin{bmatrix}b_{12}\\ b_{22}\end{bmatrix})
= \begin{bmatrix}
b_{11} & b_{12}\\
b_{21} & b_{22}
\end{bmatrix}.\]
Die Gleichung $B'A=B$ lässt sich nun als lineares Gleichungssystem
interpretieren. Dabei sind $B'$ und $B$ gegeben und $A$ ist
gesucht. Man hat also
\[\left[\begin{array}{cc|cc}
b_{11}' & b_{12}' & b_{11} & b_{12}\\
b_{21}' & b_{22}' & b_{21} & b_{22}
\end{array}\right].\]
Mit dem Gauß-Jordan-Verfahren erhält man
\[\left[\begin{array}{cc|cc}
1 & 0 & a_{11} & a_{12}\\
0 & 1 & a_{21} & a_{22}
\end{array}\right]\]
und kann die Einträge von $A$ direkt ablesen.
Eigentlich handelt es sich um zwei Gleichungssysteme.
Das erste davon ist
\[\begin{bmatrix}
b_{11}' & b_{12}'\\
b_{21}' & b_{22}'
\end{bmatrix}
\begin{bmatrix}
a_{11}\\ a_{21}
\end{bmatrix}
= \begin{bmatrix}
b_{11}\\ b_{21}
\end{bmatrix}.\]
Alternativ kann man $B'^{-1}$ bestimmen und dann
\[A = B'^{-1}B\]
berechnen.

Ein besonders einfacher Fall ist der, dass der Basisvektor $b_k'$
zum Basisvektor $b_k$ um einen Winkel verdreht ist. Für jedes $k$
soll dabei der gleiche Drehwinkel $\varphi$ bestehen. Die Matrix
$A$ ist dann orthogonal. Man hat in diesem Fall $A^{-1}=A^T$. 

Die Transformationsmatrix $A$ von den
alten in die neuen Koordinaten nennt man auch $T_{B'}^B$.
Man schreibt dann $\underline x'=T_{B'}^B\underline x$.
Dass $B'$ im Index unten steht und $B$ im Index oben, ist sinnvoll.
Es ist ja $T_{B'}^B=B'^{-1}B$. Die Matrix $B'$ steht sozusagen
im Nenner.

\subsection{Eigenwerte}

Sei $A$ eine quadratische Matrix. Normalerweise dreht und skaliert
die Matrix $A$ einen Vektor $\underline v$. Man kann nun die Frage
nach Vektoren $\underline v$ stellen, für die sich die Anwendung
der Matrix wie eine Skalierung verhält, \textit{ohne} den Vektor zu
drehen. Gesucht sind also Vektoren $\underline v$, für die
\[A\underline v = \lambda\underline v\]
ist. Dieses Problem wird Eigenwertproblem genannt. Der Vektor
$\underline v$ wird Eigenvektor der Matrix $A$ genannt und
$\lambda$ ist der dazu gehörige Eigenwert.

Beide Seiten der Gleichung lassen sich mit einer Zahl multiplizieren.
Man hat dann $A(r\underline v)=\lambda(r\underline v)$.
Nun ist $\underline v'=r\underline v$ wieder ein Vektor.
Hat man einen Eigenvektor gefunden, so lassen sich also beliebig
viele produzieren, die alle kollinear sind.

Die Umformung der Gleichung ergibt
\[(A-\lambda E)\underline v=0.\]
Das lässt sich als homogenes LGS interpretieren. Ein homogenes LGS
$A\underline v=0$ hat nur dann nichttriviale Lösungen,
wenn $\det A=0$ ist. Es muss also
\[\det(A-\lambda E)=0\]
sein. Sei $A$ nun eine 2$\times$2-Matrix. Es ist dann
\[\det\left(\begin{bmatrix}a_{11} & a_{12}\\
a_{21} & a_{22}\end{bmatrix}-
\begin{bmatrix}\lambda & 0\\
0 & \lambda\end{bmatrix}\right)=0.\]
Damit erhält man das charakteristische Polynom
\[\lambda^2 -\mathrm{Spur}(A)\lambda+\det(A)=0.\]
In diesem Fall ist es eine quadratische Gleichung.
Mit den beiden Lösungen $\lambda_1,\lambda_2$ löst man nun
die Gleichungssysteme
\begin{gather*}
(A-\lambda_1 E)\underline v=0,\\
(A-\lambda_2 E)\underline v=0.
\end{gather*}
Da diese unterbestimmt sind, sucht man sich Lösungen
$\underline v_1,\underline v_2$ aus, wo z.B. $v_{11}=1$ ist.
Damit kann man leichter rechnen.

Jetzt kann man eine Eigenzerlegung $A=TDT^{-1}$ vornehmen.
Dabei ist $D=\mathrm{diag}(\lambda_1,\lambda_2)$.
Die Matrix $T$ ist
\[T=[\underline v_1, \underline v_2]=\begin{bmatrix}
v_{11} & v_{21}\\
v_{12} & v_{22}
\end{bmatrix}.\]
Man stellt nun fest, dass
\[A^2 = (TDT^{-1})^2 = TDT^{-1}TDT^{-1} = TD^2T^{-1}.\]
Im Allgemeinen hat man für eine natürliche Potenz
\[A^n = TD^nT^{-1}.\]
Dieses Prinzip funktioniert erstaunlicherweise sogar für reelle
Potenzen. Die Formel ist von praktischer Bedeutung, da sie es
ermöglicht, große Potenzen einer Matrix ohne großen Aufwand zu
berechnen. Es ist nämlich
$D^n=\mathrm{diag}(\lambda_1^n,\lambda_2^n)$.

Man kann sogar allgemein eine Funktion $f$ auf die Matrix $A$
anwenden. Sei $f$ eine Funktion, die sich als Potenzreihe
darstellen lässt. Man hat dann
\begin{gather*}
f(A) = \sum_{k=0}^{\infty} a_kA^n
= \sum_{k=0}^{\infty} a_kTD^nT^{-1}\\
= T\Big(\sum_{k=0}^{\infty} a_kD^n\Big)T^{-1}
= Tf(D)T^{-1}.
\end{gather*}
Außerdem ist $f(D)=\mathrm{diag}(f(\lambda_1),f(\lambda_2))$.

Auch mit dem Satz von Cayley-Hamilton lässt sich die Anwendung
einer Funktion auf eine Matrix berechnen. Der Satz von
Cayley"=Hamilton sagt aus, dass eine Matrix eine Nullstelle ihres
eigenen charakteristischen Polynoms ist. Es ist also
\[A^2 -\mathrm{Spur}(A)A+\det(A)E=0.\]
Mit dieser Formel hat man
\[A^2 =\mathrm{Spur}(A)A-\det(A)E = pA+qE.\]
Damit ergibt sich
\begin{gather*}
A^3 = A^2A = (pA+qE)A = pA^2+qA\\
= p(pA+qE)+qA = p^{2}A+qA+pqE\\
= (p^2+q)A+pqE = p'A+q'E.
\end{gather*}
Das setzt sich für höhere Potenzen so fort.
Es gibt zwei Zahlen $p,q$, sodass
\[A^n = pA+qE\]
ist. Bei Partialsummen von Potenzreihen kann man diese Methode
auch anwenden. Die Methode funktioniert sogar für Potenzreihen.
Sei $f$ eine Funktion, die man als Potenzreihe darstellen kann.
Es gibt zwei Zahlen $p,q$, sodass
\[f(A) = pA+qE\]
ist. Die Frage ist jetzt natürlich, wie man auf diese Zahlen $p,q$
kommt. Weil die gleiche Argumentation auch für die Eigenwerte
funktioniert, erhält man
\begin{gather*}
f(\lambda_1) = p\lambda_1+q,\\
f(\lambda_2) = p\lambda_2+q.
\end{gather*}
Das ist ein einfaches lineares Gleichungssystem in den Unbekannten
$p,q$. Man erhält die Lösungen
\begin{gather*}
p = \frac{f(\lambda_2)-f(\lambda_1)}{\lambda_2-\lambda_1},\\
q = f(\lambda_1)-p\lambda_1.
\end{gather*}
Wenn die beiden Eigenwerte zusammenfallen, kann man $p$ mit der
angegebenen Formel nicht berechnen. Man sieht aber, dass der Ausdruck
beim zusammenfallen zur Ableitung wird. Im Fall
$\lambda_1=\lambda_2$ ist daher
\[p = f'(\lambda).\]

\subsection{Multilinearformen}

Die Dachprodukte vom Grad $k$ bilden $k$"=Blades. Durch
Linearkombination von $k$"=Blades lassen sich alle $k$"=Vektoren
bilden. Solche $k$"=Vektoren kann man nun auch als alternierende
Multilinearformen auffassen.

Multilinearformen sind einfach lineare Abbildungen, wenn man alle
Argumente bis auf eines konstant hält. Alternierend bedeutet, dass
sich das Vorzeichen umdreht, wenn zwei Argumente vertauscht werden.

Eine Multilinearform ist nun aber eine Funktion, die Vektoren als
Argumente und eine Zahl als Wert hat. Wie kann man $k$"=Vektoren
auf Vektoren anwenden?

Dazu betrachtet man zunächst 1-Formen. Das Skalarprodukt
schreibt man $\langle u,v\rangle$. Wenn aber keine Orthonormalbasis
vorliegt, so muss man den Formalismus mit Kovektoren verwenden.
Es lässt sich dann die Formel
\[\langle\mathrm e^i,e_j\rangle = \delta_{ij}\]
verwenden. Man berechnet dann
\[\omega(v) := \langle\omega,v\rangle = \omega_1v^1 + \omega_2v^2\]
wobei $\omega_k$ die Komponenten von $\omega$ bezüglich der
Dualbasis und $v^k$ die Komponenten von $v$ bezüglich der normalen
Basis sind. Eben dieses $\omega$ ist die Linearform.

Ziel ist es nun, diesen Formalismus auf alternierende
Multilinearformen zu übertragen. Das ist natürlich völlig analog
zur Kontraktion von Tensoren.

Allgemein wird die Anwendung einer alternierenden
Multilinearform auf einen Vektor mit dem inneren Produkt formuliert.
Es gilt die Rechenregel
\[\omega(v_1,\ldots,v_n) = (v_1\lrcorner\omega)(v_2,\ldots,v_n).\]
Für eine 1-Form ist einfach
$v\lrcorner\omega = \langle\omega,v\rangle$.

Sei $a$ der Grad von $A$. Mit der Formel
\[v\lrcorner(A\wedge B)
= (v\lrcorner A)\wedge B + (-1)^a A\wedge(v\lrcorner B)\]
lässt sich jede Berechnung auf 1-Formen zurückführen.
Für eine 2-Form gilt einfach
\begin{gather*}
\omega(u,v) = (\omega_{12} e^1\wedge e^2)(u,v)\\
= \omega_{12} u^1 v^2 - \omega_{12} u^2 v^1.
\end{gather*}
Es gibt auch eine allgemeine Formel mit Permutationen
und Signum.

Allgemeiner kann man rein kovariante Tensoren als Multilinearformen
interpretieren. Man definiert dazu die Anwendung auf Vektoren über die
Tensorkontraktion. D.h. man definiert
\begin{gather*}
T(a,b) := \sum_{i,j} T_{ij}a^ib^j,\\
T(a,b,c) := \sum_{i,j,k} T_{ijk}a^ib^jc^k
\end{gather*}
usw. Die alternierenden Multilinearformen entsprechen dann genau
den antisymmetrischen Tensoren. Wenn also $\omega_{ij}=-\omega_{ji}$
und $\omega_{kk}=0$ ist, so ergibt sich z.B.
\begin{gather*}
\omega_{12}e^1\wedge e^2
= \omega_{12}e^1\otimes e^2-\omega_{12}e^1\otimes e^2\\
= \sum_{i,j}\omega_{ij}e^i\otimes e^j.
\end{gather*}
Damit erhält man
\[\omega(a,b) = \sum_{i,j}\omega_{ij}a^ib^j
= \omega_{12}a^1b^2-\omega_{21}a^2b^1.\]
Kontrahiert man nicht über alle Indizes so lassen sich Tensoren
allgemeiner als multilineare Abbildungen interpretieren. Wichtig ist
nur, dass bei der Kontraktion immer ein Index oben und der zweite
unten steht.

% \newpage
\section{Analysis}
\subsection{Vektorwertige Funktionen}

Das Ableiten einer vektorwertigen Funktion ist einfach, da man
die Linearität des Differentialoperators ausnutzen kann. Sind $e_1$
und $e_2$ die Basisvektoren, so lässt sich die vektorwertige
Funktion schreiben als
\[\underline v(x) = e_1v_1(x)+e_2v_2(x).\]
Es kann sich z.B. um eine Parameterkurve handeln. Die $e_k$ sind
wie konstante Faktoren zu behandeln. Somit ergibt sich
\begin{gather*}
D\underline v(x) = D(e_1v_1)+D(e_2v_2)\\
= e_1Dv_1+e_2Dv_2.
\end{gather*}
Ein Vektorfeld wird daher abgeleitet, indem es komponentenweise
abgeleitet wird. Falls die Basisvektoren $e_k$
auch von $x$ abhängig sind, muss man die Produktregel zu benutzen.
Es ist z.B.
\[D(e_1v_1) = (De_1)v_1+e_1Dv_1.\]
Da Matrizen ebenfalls Vektoren sind, können auch sie komponentenweise
abgeleitet werden. Das selbe gilt für Tensoren. Die Produktregel
gilt außerdem für Skalarprodukte. Es ist
\begin{gather*}D\langle\underline a,\underline b\rangle
= D\Big(\sum\nolimits_k a_kb_k\Big) = \sum\nolimits_k D(a_kb_k)\\
= \sum\nolimits_k [(Da_k)b_k+a_kDb_k]\\
= \sum\nolimits_k (Da_k)b_k + \sum\nolimits_k a_kDb_k\\
= \langle D\underline a,\underline b\rangle
+ \langle \underline a,D\underline b\rangle.
\end{gather*}
Weiterhin gilt die Produktregel für Tensorprodukte und damit auch
für äußere Produkte.

Da das bestimmte Integral auch linear ist, können vektorwertige
Funktionen auch komponentenweise Integriert werden, solange die
Basisvektoren konstant sind. Allgemein gilt die Argumentation für
beliebige lineare Operatoren.


\subsection[Funktionen mit mehreren Argumenten]
{Funktionen mit mehreren\\
Argumenten}

Man kann anstelle einer Funktion $f(x)$ mit einem Argument auch
eine Funktion mit zwei Argumenten $f(x_1,x_2)$ definieren. Wie
leitet man eine solche Funktion ab? Zunächst kann man die partiellen
Ableitungen von $f$ berechnen. Dabei tut man so, als wären alle
Variablen, bis auf eine, Konstanten. Dann leitet man die Funktion ganz
normal nach dieser Variable ab. Zum Beispiel ist
\begin{gather*}
\frac{\partial}{\partial x_1}f(x_1,x_2)
= \frac{\partial}{\partial x_1} (x_1^2x_2+x_1x_2^2)\\
= \frac{\mathrm d}{\mathrm dx} (x^2a+xa^2) = 2xa+a^2\\
= 2x_1x_2+x_2^2.
\end{gather*}
\noindent
Kurze Schreibweisen für partielle Ableitungen sind
\[D_1f = \partial_1f = \frac{\partial}{\partial x_1}f.\]
\noindent
Die partiellen Ableitungen lassen sich zu einem Vektor zusammenfassen.
Dieser Vektor wird dann als Nabla-Operator bezeichnet. Es ist
\[\nabla = D_1e_1+D_2e_2\]
\noindent
bzw. allgemeiner
\[\nabla = D_1e_1+\dots+D_ne_n.\]
\noindent
Die Argumente einer Funktion lassen sich auch zu einem Vektor
zusammenfassen. Man schreibt dann kurz
\[f(\underline x) = f(x_1,\dots,x_n).\]
\noindent
Eine Funktion kann einer Zahl auch einen Vektor zuordnen.
Ein Beispiel ist die Parameterkurve
\[\underline f(t) = \begin{bmatrix}\cos t\\ \sin t\end{bmatrix}.\]
\noindent
Es ergibt sich der Einheitskreis. Eine solche Funktion wird
abgeleitet, indem sie nach jeder Komponente ableitet wird.
Wertet man die Ableitung an der Stelle $t$ aus, so erhält man
einen Tangentialvektor.

Nun kann man doch auch eine Funktion definieren, welche sowohl
mehrere Argumente hat, als auch vektorwertig ist. Man schreibt dann
$\underline f(\underline x)$ oder auch einfach wieder
$f(x)$ bzw. $f(\underline x)$.

Wie leitet man eine solche Funktion ab? Man kann bezüglich einer
Komponente nach einer Variable partiell ableiten. Alle Ableitungen
fasst man zur Jacobi"=Matrix zusammen. Die Jacobi"=Matrix
verallgemeinert also den Nabla"=Operator. Die Jacobi"=Matrix ist
definiert durch
\[J = Df = (D_j f_i).\]
\noindent
Dabei sind $f_i(\underline x)$ die Komponenten der vektorwertigen
Funktion $f$.

Die Tangente der reellen Funktion $f$ an der Stelle $x_0$
ist durch die Gleichung
\[T(x) = f'(x_0)(x-x_0)+f(x_0)\]
%
gegeben. Allgemeiner berechnet man den Tangentialraum der Funktion $f$
an der Stelle $\underline x_0$ mit
\[T(\underline x)
= Df(\underline x_0)(\underline x-\underline x_0)+f(\underline x_0).\]
%
Der Tangentialraum ist eigentlich ein Vektorraum der seinen Ursprung
am Berührungspunkt hat. Daher sollte man hier vom \textit{affinen}
Tangentialraum sprechen.

Mit $Df(\underline x_0)$ ist die Jacobi-Matrix von $f$ an der
Stelle $x_0$ gemeint. Für eine reellwertige Funktion ergibt sich der
Spezialfall
\[T(\underline x) = \langle\nabla f(\underline x_0),\underline x
-\underline x_0\rangle+f(\underline x_0).\]
%
Mit den spitzen Klammern ist das Standardskalarprodukt gemeint.
Wenn man die kanonische Basis verwendet, dann ist das Skalarprodukt
ein Spezial"-fall der Matrizenmultiplikation.

Man kann eine Funktion $f{:}\,\, \mathbf R^2\rightarrow \mathbf R^2$
als Koordinatentransformation interpretieren. Dafür ist es wichtig,
dass die Jacobi-Matrix von $f$ an jeder Stelle invertierbar ist,
dass also die Determinante der Matrix nirgendwo verschwindet.
Invertierbare Matrizen bezeichnet man auch als regulär bzw.
nicht singulär.

Die partiellen Ableitungen $D_kf(\underline x)$ sind die
Tangentialvektoren des Koordinatennetzes an der Stelle
$\underline x$.

Ein Beispiel für eine solche Koordinatentransformation ist die
Transformation von Polarkoordinaten in rechtwinklige Koordinaten
\[f(r,\varphi)
= \begin{bmatrix}
f_1(r,\varphi)\\
f_2(r,\varphi)
\end{bmatrix}
= \begin{bmatrix}
r\cos\varphi\\
r\sin\varphi
\end{bmatrix}.\]
\noindent
Bei Skalarfeldern handelt es sich um Funktionen in mehreren
Variablen. Vektorfelder sind vektorwertige Funktionen in mehreren
Variablen. Parameterkurven sind vektorwertige Funktionen in einer
Variable. Parameterflächen sind vektorwertige Funktionen in zwei
Variablen.

\subsection{Die Kettenregel}

Sei $f(\underline x)$ eine reellwertige Funktion in mehreren
Variablen und sei $\underline u(t)$ eine dazu passende
vektorwertige Funktion in einer Variablen. Die Kettenregel lautet
\[[f(\underline u)]'(t) = \sum_k D_kf(\underline u)u_k'(t)\]
bzw. alternativ
\[\frac{\mathrm d}{\mathrm dt}[f(\underline u)]
= \sum_k \frac{\partial f}{\partial u_k}\frac{\mathrm du_k}{\mathrm dt}.\]
\noindent
Das lässt sich mit Nabla-Operator und Skalarprodukt auch kurz schreiben als
\[[f(\underline u)]'(t)
= \langle\nabla f(\underline u),\underline u'(t)\rangle.\]
Ist $\underline u$ auch eine Funktion von mehreren Variablen,
so lässt sich $\underline u$
als Funktion einer einzigen Variablen auffassen wenn man die entsprechende
partielle Ableitung bildet. Damit ergibt sich
\[\frac{\partial f}{\partial x_i}
= \sum_k\frac{\partial f}{\partial u_k}\frac{\partial u_k}{\partial x_i}
= \sum_k D_k f(\underline u) D_i u_k\]
bzw. kurz
\[D_i [f(\underline u)] = \langle\mathrm \nabla f(\underline u),
D_i\underline u\rangle.\]

\noindent
Am allgemeinsten kann man die Kettenregel mit Jacobi"=Matrizen
formulieren.
Es ist dann
\[D[f(u)] = Df(u)Du.\]
\noindent
Das ist völlig analog zur einfachen Kettenregel
\[[f(u)]' = f'(u)u'.\]

\subsection{Kurvenintegrale}

Man stelle sich eine Kurve in der Ebene vor. Diese Kurve kann durch
eine vektorwertige Funktion $\underline x(t)$ angegeben werden.
Das wird Parameterdarstellung der Kurve genannt, wobei es sich bei
$t$ um den Parameter handelt.

Als Bogenlänge wird der Weg bezeichnet, den man zurücklegt, wenn
man sich auf der Kurve bewegt. Man kann sich dabei einen Fußweg,
eine Straße, oder einen Fluss vorstellen. Wie kann man die
Bogenlänge einer solchen Kurve berechnen?
Unter einer stark vergrößernden Lupe wird die Kurve aussehen wie
eine Gerade. Das ist nicht immer der Fall, aber wir wollen es
voraussetzen. Bei einer Gerade kann man nun den Pythagoras benutzen.
Für die kleinen Geradenstücke erhält man also
\[\mathrm ds = \sqrt{\mathrm dx_1^2+\mathrm dx_2^2}.\]
Man dividiert auf beiden Seiten durch $\mathrm dt$ und erhält
\[s'(t) = \sqrt{x_1'(t)^2+x_2'(t)^2} = |\underline x'(t)|.\]
Integration auf beiden Seiten bringt
\[L = s_2-s_1 = \int_{t_1}^{t_2} |\underline x'(t)|\,\mathrm dt.\]
%
Als Beispiel soll die Bogenlänge eines Kreises berechnet werden.
Als Parameterkurve wird
\[\underline x(t) = \begin{bmatrix}\cos t\\ \sin t\end{bmatrix}\]
mit $0\le t\le 2\pi$ gewählt. Es ist
\[\underline x'(t) = \begin{bmatrix}-\sin t\\ \cos t\end{bmatrix}\]
und $|\underline x'(t)|=\sqrt{\sin^2 t+\cos^2 t}=1$.
Damit ergibt sich
\[L = \int_0^{2\pi} 1\,\mathrm dt=2\pi.\]
%
Man stelle sich nun eine Gerade vor, auf der sich gleichmäßig
verteilt Staub befindet. Ein Sternenstaubsammler bewegt sich nun auf
der Gerade entlang. Die Masse an eingesammeltem Staub soll
berechnet werden. Man multipliziert dazu die Länge $L$ mit der
konstanten Staubdichte. Wenn die Staubdichte $f(x)$ aber nicht
konstant ist, so berechnet man die eingesammelte Masse mit
\[m = \int_0^s f(x)\,\mathrm dx.\]
%
Was ist nun, wenn der Staub nicht auf einer Geraden liegt, sondern auf
einer Kurve? Jeder Punkt auf der Kurve ist durch den Weg $s(t)$
bestimmt, welchen man zurücklegen muss, um zu diesem Punkt zu
gelangen. Zu jedem Punkt gehört außerdem eine Staubdichte. Die
eingesammelte Masse berechnet man also mit
\[m = \int_0^{s(t)} f(s)\,\mathrm ds
= \int_0^t f(s(t))s'(t)\,\mathrm dt.\]
Der letzte Term ergibt sich durch die Substitutionsregel.

Mit der Feststellung $f(s) = f(\underline x)$ und der Formel
$s'(t) = |\underline x'(t)|$ ergibt sich schließlich
\[m = \int_0^t f(\underline x'(t))|\underline x'(t)|\,\mathrm dt.\]
Ein solches Integral über eine Kurve bezeichnet man, wie der Name
schon sagt, als Kurvenintegral. Die innere Geometrie der Kurve stimmt
ja mit der inneren Geometrie einer Gerade überein. Aus diesem Grund
stimmt das Kurvenintegral aus Sicht der Bogenlänge mit dem
gewöhnlichen Integral überein.

Mit dem Spezialfall $f(s)=1$ erhält
man die Bogenlänge.



\subsection{Kurvenintegrale zweiter Art}

Ein Kurvenintegral lässt sich so interpretieren, dass die Kurve
dabei in ein Skalarfeld eingebettet ist. Jedem Punkt der Kurve
kann damit ein Skalar zugeordnet werden. Was ist nun, wenn die
Kurve in ein Vektorfeld eingebettet ist? Jedem Punkt der Kurve
wird ein Vektor zugeordnet. Wie berechnet man dann das Kurvenintegral?

Nun man kann doch eine Analogiebetrachtung machen und einfach das
Produkt gegen ein Skalarprodukt austauschen. Dafür ist es auch
notwendig, die Betragsstriche zu entfernen, da man sonst eine
Skalarmultiplikation hätte. Sei $\underline v$ ein Vektorfeld.
Man definiert
\[\int_C \langle\underline v(\underline x),\mathrm d\underline x\rangle
:= \int_{a}^{b} \langle\underline v(\underline x(t)),
\underline x'(t)\rangle\,\mathrm dt.\]
Sei $U(\underline x)$ nun ein beliebiges Skalarfeld. Wenn
$\underline v=\nabla U$ ist, so heißt $\underline v$ Potentialfeld
und $U$ ist das dazu gehörige Potential. Man kann nun rechnen
\[\langle\underline v,\mathrm d\underline x\rangle
= \langle\nabla U,\mathrm d\underline x\rangle
= \nabla^\flat U
= \mathrm dU.\]
Die Kurve $C$ habe den Anfangspunkt $A=\underline x(a)$ und den
Endpunkt $B=\underline x(b)$. Nach dem Satz von Stokes ist
\[\int_C \mathrm dU = \int_{R(C)} U = U(B)-U(A).\]
Für ein Potentialfeld ist das Kurvenintegral also nur von Randpunkten
der Kurve abhängig, jedoch nicht von der Kurve selbst. Das Kurvenintegral
ist für ein Potentialfeld \textit{wegunabhängig}.

Damit folgt sofort das Verschwinden des Kurvenintegrals für eine
geschlossene Kurve, es ist
\[\oint_C \langle\nabla U,\mathrm d\underline x\rangle = 0.\]
In der Physik wird man noch eine Proportionalitätskonstante einschieben.
Das Potentialfeld ist dann $\underline v=c\nabla U$.
Die Formel lautet dementsprechend
\[\int_C \langle\underline v,\mathrm d\underline x\rangle
= cU(B)-cU(A).\]
Eine Potentialfeld ist rotationsfrei, denn es ist
\[\nabla\wedge\underline v = \nabla\wedge \nabla U
= (\nabla\wedge\nabla)U = 0.\]
Verlangt man nun zusätzlich die Gültigkeit der Laplace-Gleichung
$\Delta U = 0$, so ergibt sich mit
$\Delta U = \langle\nabla,\nabla U\rangle$ dann
\[\nabla\underline v
= \langle\nabla,\underline v\rangle + \nabla\wedge\underline v=0.\]
Wenn $\underline v$ ein Vektorfeld in der Ebene ist, so
stellt diese Bedingung gerade die Cauchy-Riemann-PDGn dar, wie
im Abschnitt Holomorphie erläutert wird. Ein elektrostatisches
Feld in einer ladungsfreien Ebene kann also durch eine holomorphe
Funktion beschrieben werden. Charakteristisch sind dabei die
Feldlinien, die überall rechtwinklig auf den Äquipotential-Isolinien
stehen.

Im Raum kann man komplexe Funktionen leider nicht mehr benutzen.
Da das Vektorfeld $\underline v$ aber die Gleichung
$\nabla\underline v=0$ erfüllt, muss es ähnliche Eigenschaften
haben. Unter anderem stehen die Feldlinien rechtwinklig auf den
Äquipotentialflächen.


\subsection{Variationsrechnung}

Die kürzeste Verbindung zweier Punkte in der Ebene ist eine Gerade.
Wie sieht es aber mit der kürzesten Verbindung zweier Punkte auf
einer gekrümmten Oberfläche aus? Zur Berechnung einer solchen
Verbindungskurve benötigt man die Variationsrechnung.

In der Ebene hat man eine Funktion $L(x,y,y') = \sqrt{1+y'^2}$,
welche die Ableitung der Bogenlänge $S(f)$ ist. Dabei soll
$y=f(x)$ sein. Die Funktion $L$ wird allgemein als
Lagrange"=Funktion bezeichnet. Die Bogenlänge ist gegeben durch
\[S(f) = \int_a^b L(x,y,y')\,\mathrm dx.\]
Wir fragen nun nach einer Funktion $f$, für welche $S(f)$ am
kleinsten ist.

Wir tun nun so, als ob wir diese Funktion $f$ schon kennen.
Sei $g$ eine Funktion, die auf $[a,b]$ zweimal stetig
differenzierbar ist. Weiter soll $g(a)=0$ und $g(b)=0$
sein. Sei $p$ eine Zahl, die auch negativ sein darf. Man
kann nun die Funktionen $h(x)=f(x)+pg(x)$ bilden, welche
die Funktion $f$ etwas variieren.

Was das soll? Wir können die Bedingung $S(h)=\mathrm{min}$ nun als
ein gewöhnliches Extremwertproblem $S(p)=\mathrm{min}$ ansehen.
Und für ein solches muss beim Minimum ja die notwendige Bedingung
$S'(p)=0$ gültig sein. Das Minimum erhält man für $p=0$.
Damit muss $S'(0)=0$ sein.

Wir benutzen nun die Rechenregel
\[\frac{\mathrm d}{\mathrm dp}\int_a^b f(x,p)\,\mathrm dx =
\int_a^b \frac{\partial}{\partial p}f(x,p)\,\mathrm dx.\]
Diese Rechenregel ist ein Spezialfall der Rechenregel für
Parameterintegrale. Bei $p$ handelt es sich um den Parameter.
In diesem Fall ist damit
\[S'(p) = \int_a^b \frac{\partial L}{\partial p}\,\mathrm dx.\]
Man benutzt nun die Kettenregel und erhält
\[S'(p) = \int_a^b \frac{\partial L}{\partial y}g
  +\frac{\partial L}{\partial y'}g'\,\mathrm dx.\]
Mit der Produktregel hat man nun
\begin{gather*}
\int_a^b \frac{\partial L}{\partial y'}\,g'\,\mathrm dx
= \Big[g\frac{\partial L}{\partial y'}\Big]_a^b
- \int_a^b g \frac{\mathrm d}{\mathrm dx}
\frac{\partial L}{\partial y'}\,\mathrm dx\\
= - \int_a^b g \frac{\mathrm d}{\mathrm dx}
\frac{\partial L}{\partial y'}\,\mathrm dx.
\end{gather*}
Insgesamt ist also
\[S'(p) = \int_a^b \left[\frac{\partial L}{\partial y}
-\frac{\mathrm d}{\mathrm dx}\frac{\partial L}{\partial y'}
\right]g\,\mathrm dx\]
Die Argumentation ist nun folgende. Da die Gleichung $S'(0)=0$
für jede Funktion $g$ gelten soll, muss notwendigerweise der andere
Faktor gleich null sein. Dass diese Argumentation korrekt ist, ist
Inhalt des Fundamentallemmas der Variationsrechnung.
Damit erhält man die

\textbf{Euler-Lagrange-Gleichung}
\[\frac{\partial L}{\partial y}
-\frac{\mathrm d}{\mathrm dx}\frac{\partial L}{\partial y'}
=0.\]
Diese Gleichung stellt eine notwendige Bedingung für
$S(f)=\mathrm{min}$ oder $S(f)=\mathrm{max}$ dar.

Man kann die Gleichung noch verallgemeinern. Sei $\underline v(t)$
eine vektorwertige Funktion und sei
$L(t,v_1,\ldots,v_n,v_1',\ldots,v_n')$ die Lagrange"=Funktion.
Die Gleichungen
\[\frac{\partial L}{\partial v_k}
-\frac{\mathrm d}{\mathrm dt}\frac{\partial L}{\partial v_k'}
=0\]
stellen eine notwendige Bedingung für
$S(\underline v)=\mathrm{min}$ oder
$S(\underline v)=\mathrm{max}$ dar.
Man kann die Euler"=Lagrange"=Gleichung also komponentenweise
formulieren. Trotzdem kann dabei aber ein System von gekoppelten
Differentialgleichungen entstehen. In einem solchen Fall ist es
nicht möglich, eine der Gleichungen ohne die anderen zu lösen.

Man kann z.B. die kürzeste Verbindung in der Ebene suchen. Man erhält
\begin{gather*}
\frac{\partial L}{\partial y} = 0,\qquad
\frac{\partial L}{\partial y'} = \frac{y'}{\sqrt{1+y'^2}},\\
\frac{\mathrm d}{\mathrm dx}\frac{\partial L}{\partial y'}
= y''\frac{\sqrt{1+y'^2}-y'^2(\sqrt{1+y'^2})^{-1}}{1+y'^2}\\
= y''\frac{1}{(1+y'^2)^{3/2}}.
\end{gather*}
Damit ergibt sich als notwendige Bedingung die Differentialgleichung
\[y''=0.\]
Man hätte anstelle des Bogenlängenfunktionals auch das
Energiefunktional verwenden können. Man kommt so mit weniger
Rechenaufwand auf die gleiche Differentialgleichung.
Die Lösungen dieser Gleichung sind die Funktionen
\[y=mx+n.\]
Ein Beispiel aus der Physik ist der freie Fall. Es ist $L=T-V$,
wobei mit $T$ die kinetische und mit $V$ die potentielle Energie
gemeint ist. Weiterhin ist $v=s'(t)$. Man hat also
\[L(t,s,v) = \frac{1}{2}mv^2 - mg(-s).\]
Man rechnet nun
\[
\frac{\partial L}{\partial s} = mg,\quad
\frac{\partial L}{\partial v} = mv,\quad
\frac{\mathrm d}{\mathrm dt}\frac{\partial L}{\partial v}
= mv' = ms''.
\]
Als notwendige Bedingung ergibt sich die Differentialgleichung
\[s''(t) = g.\]
Die Lösung dieser Gleichung ist
\[s=\frac{1}{2}gt^2+v_0t+s_0.\]
%
Auch für Felder gibt es eine Euler"=Lagrange"=Gleichung. Dazu ist es
notwendig, die Lagrangedichte $l$ von einem Feld
$\varphi(t,x_1,x_2,x_3)$ einzuführen. Es ist
\[L = \int_V l\,\mathrm dx_1\mathrm dx_2\mathrm dx_3.\]
Setzt man noch $x_0=ct$, so ergibt sich
\[S(\varphi) = \int_H l\,\mathrm d^4x.\]
Die Lagrangedichte ist von mehreren Argumenten abhängig, man hat
\[l(t,\varphi,D_0\varphi, D_1\varphi, D_2\varphi, D_3\varphi).\]
Mit $D_k\varphi$ ist die partielle Ableitung von $\varphi$
nach $x_k$ gemeint.

Die Euler-Lagrange-Gleichung ist
\[\frac{\partial l}{\partial\varphi}
- \sum_{k=0}^3 \frac{\partial}{\partial x_k}
\frac{\partial l}{\partial (D_k\varphi)}=0.\]
Als Beispiel soll das elektrische Potentialfeld dienen.
Bei der Energiedichte $w=(1/2)\varepsilon_0 E^2$ handelt es sich
um die Lagrangedichte. Es ist also $l=w$.
Als Berechnungsformel gibt es auch
\[l = -\frac{1}{4\mu_0} F^{\mu\nu}F_{\mu\nu}.\]
Die magnetische Feldkonstante $\mu_0$ hat natürlich nichts
mit dem Index~$\mu$ zu tun. Man könnte auch $F_{\alpha\beta}$
oder $F_{ij}$ schreiben. Es ist üblich, die Indizes von null bis
drei gehen zu lassen und nicht von eins bis vier.

Der Feldstärke-Tensor ist
\[F^{\mu\nu} = \begin{bmatrix}
0 & -E_1/c & -E_2/c & -E_3/c\\
E_1/c & 0 & 0 & 0\\
E_2/c & 0 & 0 & 0\\
E_3/c & 0 & 0 & 0
\end{bmatrix}.\]
Mit dem metrischen Tensor $g=\mathrm{diag}(1,-1,-1,-1)$ senkt man
die Indizes und erhält
\[F_{\mu\nu} = \begin{bmatrix}
0 & E_1/c & E_2/c & E_3/c\\
-E_1/c & 0 & 0 & 0\\
-E_2/c & 0 & 0 & 0\\
-E_3/c & 0 & 0 & 0
\end{bmatrix}.\]
Es ergibt sich
\[F^{\mu\nu}F_{\mu\nu} = \frac{2}{c^2} E^2
= \frac{2}{c^2} (E_1^2+E_2^2+E_3^2).\]
Nutzt man nun noch $\mu_0\varepsilon_0 c^2=1$ aus, so gelangt man zu
\[l = \frac{1}{2}\varepsilon_0 E^2
= \frac{1}{2}\varepsilon_0 |\nabla\varphi|^2.\]
Man rechnet nun
\begin{gather*}
\frac{\partial l}{\partial(D_1\varphi)}
= \frac{\varepsilon_0}{2} \frac{\partial}{\partial(D_1\varphi)}
\sum_{k=1}^3 (D_k\varphi)^2 = \varepsilon_0 D_1\varphi,\\
\frac{\partial}{\partial x_1}\frac{\partial l}{\partial(D_1\varphi)}
= \varepsilon_0 D_1^2\varphi,\qquad
\frac{\partial l}{\partial\varphi} = 0.
\end{gather*}
Die notwendige Bedingung ist also
\[\varepsilon_0 (D_1^2\varphi+D_2^2\varphi+D_3^2\varphi) = 0\]
oder kurz $\varepsilon_0 \Delta\varphi=0$. Es ist also
\[\Delta\varphi=0.\]
Diese partielle Differentialgleichung wird Laplace"=Gleichung genannt.
Es ist die Gleichung, welche elektrostatische Felder in einem
ladungsfreien Raum beschreibt, wenn das Potentialfeld am Rand dieses
Raumes bekannt ist. Auf jeden Fall wird das Potentialfeld die
Laplace"=Gleichung erfüllen, auch wenn es nicht auf dem Rand
bekannt ist.

Die Lagrangedichte für das elektromagnetische Feld im Vakuum ist
\[l = -\frac{1}{4\mu_0} F^{\mu\nu}F_{\mu\nu} -J_\mu A^\mu.\]
Hierbei ist
\[F_{\mu\nu} = D_\mu A_\nu - D_\nu A_\mu.\]
Man kann auf beiden Seiten zweimal mit dem inversen metrischen
Tensor multiplizieren, womit die Indizes gehoben werden. Man erhält
\[F^{\mu\nu} = D^\mu A^\nu - D^\nu A^\mu.\]
An dieser Stelle sollen noch einige Kurznotationen eingeführt werden.
Wenn man den Gradient nicht nach den $x_k$, sondern nach
den $a_k$ bilden will, dann beschreibt man das am besten durch
\[\nabla[a] = \frac{\partial}{\partial a}
= \sum_k e_k\frac{\partial}{\partial a_k}.\]
Analog kann man auch $D[a_k]=\frac{\partial}{\partial a_k}$
definieren.
Damit schreibt man kurz
\[\nabla[\nabla\varphi] = \frac{\partial}{\partial\nabla\varphi}
= \sum_k e_k\frac{\partial}{\partial D_k\varphi}.\]
Somit kann man schreiben
\[\langle\nabla,\nabla[\nabla\varphi]l\rangle
= \sum_k \frac{\partial}{\partial x_k}
\frac{\partial l}{\partial D_k\varphi}.\]
Die Euler-Lagrange-Gleichung lautet nun kurz
\[D[\varphi]l-\langle\nabla,\nabla[\nabla\varphi]l\rangle=0.\]
Die gewöhnliche Euler-Lagrange-Gleichung ist ein Spezialfall dieser
Gleichung. Man erhält sie, wenn man anstelle der Lagrangedichte
die gewöhnliche Lagrangefunktion einsetzt. Bei einem Vektorfeld
kann die allgemeine Euler-Lagrange-Gleichung wieder komponentenweise
formuliert werden.

\subsection{Extrema mit Nebenbedingungen}

Man denke sich einen Hügel. Der Funktionswert $y=f(x_1,x_2)$ kann
als Höhe des Hügels an der Position $(x_1,x_2)$ interpretiert
werden. Nun wird durch die Gleichung $g(x_1,x_2)=0$ ein Weg auf der
$(x_1,x_2)$-Ebene festgelegt, den man entlang wandert.
Man hätte auch $g_1(x_1,x_2)=c$
festlegen können. Aber dann kann man ja $g:=g_1-c$ definieren und
hat wieder $g(x_1,x_2)=0$. Dieser Weg wird nun auf die Fläche
$f(x_1,x_2)$ hinauf projiziert. Man fragt sich nun, wo das Maximum
des Weges auf der Fläche $f(x_1,x_2)$ ist.

Die Höhe beim Maximum sei $h$ und die Position des Maximumspunktes
sei $\underline a = (a_1,a_2)$. Durch $\underline a$ geht eine
Isolinie, die implizit durch $f(x_1,x_2)=h$ beschrieben wird.
Weiterhin geht durch $\underline a$ die
Isolinie, die implizit durch $g(x_1,x_2)=0$ beschrieben wird.
Man kann ja auch $g(x_1,x_2)$ als eine Fläche interpretieren.
Die beiden Isolinien berühren sich, d.h. sie
haben eine gemeinsame Tangente.

Da der Gradient immer rechtwinklig auf der Isolinie steht, sind die
beiden Gradienten $\nabla f(\underline a)$ und
$\nabla g(\underline a)$ kollinear zueinander. Man such also eine
Stelle $\underline x$ wo die Bedingung
\[\nabla f(\underline x) = -\lambda\nabla g(\underline x)\]
erfüllt ist. Das Minuszeichen ist nur Konvention und hat keine weitere
Bedeutung. Die Gradienten können
gleichsinnig parallel oder entgegengesetzt parallel sein.
Durch Umformen erhält man die Bedingung
\[\nabla (f+\lambda g)=0.\]
Diese Bedingung kann man in die beiden Gleichungen
\begin{gather*}
D_1(f+\lambda g)=0,\\
D_2(f+\lambda g)=0.
\end{gather*}
aufspalten. Zusammen mit der Nebenbedingung $g(x_1,x_2)=0$ bilden
die Gleichungen ein nichtlineares Gleichungssystem, welches die
notwendige Bedingung für einen Extrempunkt angibt.

Der Skalierungsfaktor $\lambda$ heißt Lagrange"=Multiplikator.
Er gehört auch zur Lösung des Gleichungssystems, ist aber
im Ergebnis nicht weiter wichtig.

Die Nebenbedingung kann man auch durch
$D_\lambda (f+\lambda g)=0$ beschreiben.
Man definiert nun die Funktion
\[L(x_1,x_2,\lambda) := f(x_1,x_2)+\lambda g(x_1,x_2).\]
Man kann das Gleichungssystem jetzt mit $\nabla L=0$
abkürzen, wobei der Gradient über alle drei Variablen zu bilden ist.
Die Funktion~$L$ wird Lagrangefunktion genannt.

Die drei Bedingungen, die erfüllt sein müssen, sind
\begin{gather*}
D_1 L = D_2 L = D_\lambda L = 0.
\end{gather*}
Dabei ist noch etwas zu beachten. Wenn $\nabla g=0$ und
$\nabla f\ne 0$ ist, können die Bedingungen
nicht erfüllt sein. Der Nullvektor ist ja zu jedem Vektor
rechtwinklig, kann also nicht zu $\nabla f$ parallel sein.
Trotzdem kann an einer solchen Stelle ein Extrempunkt vorliegen.

Das Verfahren funktioniert auch, wenn man eine Funktion von $m$
Variablen hat. Dann ist $\underline x=(x_1,\ldots,x_m).$
Man kann immer noch $\nabla L=0$ schreiben. Man erhält jedoch ein
System von $m+1$ Gleichungen. Die Bedingungen sind dann
\[D_1 L = D_2 L = \ldots = D_n L = D_\lambda L = 0.\]
%
Wenn man $n$ Nebenbedingungen $g_k(\underline x)=0$ hat,
dann kann man immer noch $\nabla L=0$ schreiben.
Man hat dann jedoch ein System von $m+n$ Gleichungen.

Die Lagrangefunktion ist
\begin{gather*}
L(x_1,\ldots,x_m,\lambda_1,\ldots,\lambda_n)\\
= f(x_1,\ldots,x_m)+\sum_{k=1}^n \lambda_k g_k(x_1,\ldots,x_m).
\end{gather*}
Man bekommt die Bedingungen
\begin{gather*}
\frac{\partial L}{\partial x_k}=0 \quad (1\le k\le m),\\
\frac{\partial L}{\partial \lambda_k}=0 \quad (1\le k\le n).
\end{gather*}
An dieser Stelle soll noch eine alternative Formulierungsweise
erwähnt werden. Da $\nabla f$ und $\nabla g$ kollinear sind,
gilt ja $\mathrm df\wedge\mathrm dg=0$. Das äußere Produkt muss
man ausmultiplizieren. Man erhält dann einen Ausdruck der Form
\[\sum_{i<j} a_{ij}\mathrm dx_i\wedge\mathrm dx_j=0.\]
Damit ergibt sich jeweils $a_{ij}=0$. Ein oberflächlicher Vorteil
ist dabei, dass kein lagrangescher Multiplikator vorkommt.
Ein tiefer greifender Vorteil ist, dass dieser Rechenweg manchmal
eleganter ist.

\subsection{Die Gâteaux-Ableitung}

Variationsaufgaben lassen sich auch mit der sogenannten
Variationsableitung formulieren. Bei dieser Ableitung wird ein
Funktional nicht nach einer reellen Variablen, sondern nach einer
Funktion abgeleitet.

Die Gâteaux"=Ableitung ist nun ein Konzept, das die Richtungsableitung
verallgemeinert. Wichtig ist dabei, dass Variationsableitungen auch
Gâteaux"=Ableitungen sind.

Die Richtungsableitung ist definiert durch
\begin{gather*}
D_v f := \lim_{h\rightarrow 0} \frac{f(x+hv)-f(x)}{h}\\
= \Big[\frac{\mathrm d}{\mathrm dh}f(x+hv)\Big]_{h=0}.
\end{gather*}
Wenn $f$ aber ein Funktional ist, so sind doch $x(t)$ und $v(t)$
Funktionen. In diesem Fall spricht man von der Gâteaux-Ableitung.
Diese sollte linear und stetig bezüglich $v$ sein und man schreibt dann
auch $\delta f v$ anstelle von $D_v f$. Die Notation suggeriert
somit
\[\delta f(v_1+v_2) = \delta fv_1 + \delta fv_2\]
und $\delta f(rv) = r\delta fv$. Im Fall eines endlichdimensionalen
Vektorraumes ist $\delta f v = \langle \nabla f,v\rangle$. Man
fragt sich jetzt natürlich sofort inwiefern sich das auf
Hilberträume übertragen lässt.

Die notwendige Bedingung für ein Maximum ist nun, dass die
Richtungsableitung in jede beliebige Richtung verschwindet. D.h.
wenn $x_0$ eine Maximumsstelle sein soll, so muss notwendig
$(D_v f)(x_0)=0$ für alle $v$ sein.

Analog lässt sich
$\delta fv=0$ für alle Funktionen $v(t)$ verlangen.
Damit ergibt sich der gleiche Ansatz wie bei der Herleitung
der Euler-Lagrange-Gleichung.


\newpage
\section{Differentialgeometrie}

\subsection{Integralsätze}

Der gaußsche Integralsatz in der Ebene (Satz von Green), der gaußsche
Integralsatz im Raum und der Integralsatz von Stokes. Diese
Sätze lassen sich zu einem einzigen vereinigen. Der Dafür notwendige
Formalismus soll hier erläutert werden.

Ab jetzt wollen wir Vektoren nicht mehr unterstreichen. Das ist
einerseits ungünstig, weil wir Skalarfelder nicht mehr so gut von
Vektorfeldern unterscheiden können. Andererseits spart es
Schreibaufwand.

Zur Vereinfachung wird zunächst eine Orthonormalbasis gewählt.
Wir verwenden nun die Operatoren
\begin{gather*}
v^\flat = (v_1e_1+v_2e_2)^\flat
= v_1\mathrm dx_1+v_2\mathrm dx_2,\\
\omega^\sharp
= (\omega_1\mathrm dx_1+\omega_2\mathrm dx_2)^\sharp
= \omega_1e_1+\omega_2e_2.
\end{gather*}
Außerdem wird $\mathrm dx_1$ mit $e_1$ und $\mathrm dx_2$
mit $e_2$ identifiziert. Wenn wir keine Orthonormalbasis haben,
dann sind die Differentiale jedoch die dualen Basisvektoren.
Die Umrechnung geht dann über den metrischen Tensor.

Der gaußsche Integralsatz in der Ebene lautet
\[\iint_B \frac{\partial v_2}{\partial x_1}
- \frac{\partial v_1}{\partial x_2}\,\mathrm dx_1\mathrm dx_2
= \oint_{R(B)} (v_1\mathrm dx_1+v_2\mathrm dx_2).\] 
Das lässt sich kurz schreiben. Es ist dann
\[\iint_B \nabla^\flat\wedge v^\flat = \oint_{R(B)}  v^\flat.\]
Sei $\omega= v^\flat$.
Der Integralsatz lautet mit dieser Ab"-kür"-zung
\[\iint_B \nabla^\flat\wedge\omega
= \oint_{R(B)} \omega.\]
%
Sei $x=(x_1,\ldots,x_n)$.
Die Cartan-Ableitung von einem Skalarfeld $f(x)$ ist
\[\mathrm df = \nabla^\flat f
= \sum_{k} D_k f\,\mathrm dx_k.
= \sum_{k} \frac{\partial f}{\partial x_k}\mathrm dx_k.\]
Außerdem gilt die Formel
\[\mathrm d(f\mathrm dx_1\wedge\ldots\wedge\mathrm dx_n)
= \mathrm df\wedge\mathrm dx_1\wedge\ldots\wedge\mathrm dx_n.\]
Diese beiden Rechenregeln genügen, um die Cartan"=Ableitung
zu berechnen. Wir rechnen nun
\begin{gather*}
\mathrm d\omega
= \mathrm d(v^\flat)
= \mathrm d(v_1\mathrm dx_1+v_2\mathrm dx_2)\\
= \mathrm dv_1\wedge\mathrm dx_1+\mathrm dv_2\wedge\mathrm dx_2\\
= (D_1v_1\mathrm dx_1+D_2v_1\mathrm dx_2)\wedge\mathrm dx_1\\
+\;(D_1v_2\mathrm dx_1+D_2v_2\mathrm dx_2)\wedge\mathrm dx_2\\
= D_2v_1\mathrm dx_2\wedge\mathrm dx_1
+ D_1v_2\mathrm dx_1\wedge\mathrm dx_2\\
= -D_2v_1\mathrm dx_1\wedge\mathrm dx_2
+ D_1v_2\mathrm dx_1\wedge\mathrm dx_2\\
= (D_1v_2-D_2v_1)\mathrm dx_1\wedge\mathrm dx_2.
\end{gather*}
Wenn $\omega$ ein 1-Kovektor (eine 1-Form) ist, dann ist also
\[\mathrm d\omega
= \nabla^\flat\wedge\omega.\]
Der Integralsatz von Gauß in der Ebene lässt sich nun kurz fassen zu
\[\iint_B \mathrm d\omega
= \oint_{R(B)}\omega.\]
%
Sei $v$ ein Vektorfeld. Der Integralsatz von Gauß im Raum lautet
\[\iiint_V \mathrm{div}\,v\,\mathrm dV
= \iint_{R(V)} \langle v,\mathrm dA\rangle\]
Dabei soll $\langle v,\mathrm dA\rangle$ das Standardskalarprodukt
sein. Ausgeschrieben lautet der Satz
\begin{gather*}
\iiint_V D_1v_1+D_2v_2+D_3v_3\,\,\mathrm dx_1\mathrm dx_2\mathrm dx_3\\
= \iint_{R(V)}(v_1\mathrm dx_2\mathrm dx_3+v_2\mathrm dx_3\mathrm dx_1
+v_3 \mathrm dx_1\mathrm dx_2).
\end{gather*}
Nun wird $\mathrm dx_1\mathrm dx_2$ gegen
$\mathrm dx_1\wedge dx_2$ ersetzt usw. und
$\mathrm dx_1\mathrm dx_2\mathrm dx_3$\\
gegen $\mathrm dx_1\wedge\mathrm dx_2\wedge\mathrm dx_3$
ersetzt. Nun ist z.B.
\begin{gather*}\mathrm d(v_1\mathrm dx_2\wedge\mathrm dx_3)
= \mathrm dv_1\wedge\mathrm dx_2\wedge\mathrm dx_3\\
= (D_1v_1\mathrm dx_1+D_2v_2\mathrm dx_2+D_3v_3\mathrm dx_3)
\wedge\mathrm dx_2\wedge\mathrm dx_3\\
= D_1v_1\mathrm dx_1\wedge\mathrm dx_2\wedge\mathrm dx_3+0+0\\
= D_1v_1\mathrm dx_1\wedge\mathrm dx_2\wedge\mathrm dx_3.
\end{gather*}
Dabei wurde $\mathrm dx_k\wedge\mathrm dx_k=0$ verwendet.
Mit
\[\omega = v_1\mathrm dx_2\wedge\mathrm dx_3
+v_2\mathrm dx_3\wedge\mathrm dx_1
+v_3\mathrm dx_1\wedge\mathrm dx_2\]
erhält man den Integralsatz in der Form
\[\iiint_V \mathrm d\omega = \iint_{R(V)} \omega.\]
Der Integralsatz von Stokes lautet
\[\iint_B \langle\mathrm{rot}\,v,\mathrm dA\rangle
= \oint_{R(B)} \langle v,\mathrm dr\rangle.\]
Es ist
$\mathrm dA = (\mathrm dx_2\mathrm dx_3,\mathrm dx_3\mathrm dx_1,
\mathrm dx_1\mathrm dx_2)$.
Wir können das umschreiben. Es ist
\begin{gather*}
\langle\mathrm{rot}\,v,\mathrm dA\rangle
= (D_2v_3-D_3v_2)\mathrm dx_2\wedge\mathrm dx_3\\
+ (D_3v_1-D_1v_3)\mathrm dx_3\wedge\mathrm dx_1\\
+ (D_1v_2-D_2v_1)\mathrm dx_1\wedge\mathrm dx_2\\
= (D_1\mathrm dx_1+D_2\mathrm dx_2+D_3\mathrm dx_3)\\
\wedge (v_1\mathrm dx_1+v_2\mathrm dx_2+v_3\mathrm dx_3)\\
= \nabla^\flat\wedge v^\flat
\end{gather*}
Auch hier kann man nachrechnen, dass gilt
\[\mathrm d(v^\flat) = \nabla^\flat\wedge v^\flat.\]
Außerdem ist
\[\langle v,\mathrm dr\rangle = v^\flat
= v_1\mathrm dx_1+v_2\mathrm dx_2+v_3\mathrm dx_3.\]
Definiert man noch $\omega = v^\flat$, so ergibt sich wieder
\[\iint_B \mathrm d\omega = \oint_{R(B)}\omega.\]
%
Der Hauptsatz der Analysis lautet
\[\int_a^b f'(x)\,\mathrm dx = f(b)-f(a).\]
Da es nur eine Variable gibt, stimmen gewöhnliche Ableitung und
partielle Ableitung überein. Es ist also $D_x f(x) = f'(x)$.
%
Damit ist
\[\mathrm df = D_x f(x)\,\mathrm dx = f'(x)\,\mathrm dx.\]
%
Außerdem ist wollen wir die Rechnung
\[\int_{R[a,b]} f = \int_{[b]-[a]}f = \int_{[b]} f + \int_{-[a]} f
= f(b)-f(a)\]
als gültig annehmen. Das ist wieder so ein Formalismus, der
\textit{Integration über Ketten} genannt wird.
%
Der Hauptsatz lässt sich jetzt umschreiben zu
\[\int_{[a,b]} \mathrm df = \int_{R[a,b]}f.\]
Der Hauptsatz und alle Integralsätze lassen sich also mit
\[\int_B \mathrm d\omega = \int_{R(B)}\omega\]
zusammenfassen.

An dieser Stelle sollen noch zwei wichtige Eigenschaften der
Cartan-Ableitung hergeleitet werden. Seien $A,B$ Differentialformen
und sei $A$ vom Grad $r$. Für die Cartan-Ableitung gilt
\[\mathrm d(A\wedge B)
= \mathrm dA\wedge B + (-1)^r A\wedge \mathrm dB.\]
Diese seltsame Regel sieht aus wie die Produktregel. Es handelt
sich tatsäch"-lich um die Produktregel. Um das einzusehen muss man
zunächst $\mathrm d\omega = \nabla^\flat\wedge\omega$ schreiben.
Sei $P$ vom Grad $p$ und $Q$ vom Grad $q$. Für das
äußere Produkt gilt die Regel
\[P\wedge Q = (-1)^{pq}Q\wedge P.\]
Man geht nun in Analogie zu $D(fg)=(Df)g+fDg$ für die gewöhnliche
Ableitung vor. Dazu ist es notwendig einmal $\nabla^\flat\wedge A$
zu verdrehen. Man macht die Rechnung
\begin{gather*}
\mathrm d(A\wedge B) = \nabla^\flat\wedge (A\wedge B)\\
= (\nabla^\flat\wedge A)\wedge B
+ (-1)^r A\wedge\nabla^\flat\wedge B\\
= \mathrm dA\wedge B + (-1)^r A\wedge \mathrm dB.
\end{gather*}
Für alle $\omega$ ist $\mathrm {dd}\omega=0$.
Sei $v$ vom Grad eins.
Mit $v\wedge v=0$ macht man die Rechnung
\begin{gather*}
\mathrm{dd}\omega
= \nabla^\flat\wedge\nabla^\flat\wedge\omega
= 0\wedge\omega = 0.
\end{gather*}
Sei $r$ ein Skalar. Man verwende nun die Rechenregeln $a\wedge r=ra$ und
$r(a\wedge b) = ra\wedge b = a\wedge rb$. Für eine null-Form ergibt sich
\[\nabla^\flat\wedge f = \nabla^\flat f = \mathrm df.\]
Für eine eins-Form $\omega = f\mathrm dx_1\wedge\mathrm dx_2$
erhält man
\begin{gather*}
\nabla^\flat\wedge\omega = \nabla^\flat\wedge f\mathrm dx_1\wedge dx_2
= \nabla^\flat f\wedge\mathrm dx_1\wedge\mathrm dx_2\\
= \mathrm df\wedge\mathrm dx_1\wedge\mathrm dx_2
= \mathrm d\omega.
\end{gather*}
Die Formel
\[\mathrm dA = \nabla^\flat\wedge A\]
gilt also im allgemeinen, wo $A$ eine null-Form, eins-Form,
zwei-Form usw. sein kann.

\subsection{Die Richtungsableitung}

Von dem Folgenden sollte man sich unbedingt eine Skizze auf einem
Blatt Papier machen. Die partiellen Ableitungen lassen sich leicht
erklären. Man stellt sich ein Skalarfeld $f(x_1,x_2)$ als eine
Oberfläche vor. Jeder stelle $\underline x=(x_1,x_2)$ wird dabei
eine Höhe zugeordnet. Nun macht man parallel zu einer der
Koordinatenachsen einen Schnitt, indem man eine der Variablen
konstant hält. Halten wir $x_2=b$ konstant. Es ergibt sich
eine Schnittkurve. Diese ist eine reelle Funktion $g(x)=f(x,b)$
mit der nicht festgehaltenen Koordinate als Variable.
Von der Funktion $g$ kann dann die Ableitung an der Stelle $a$
bestimmt werden. Das ist dann der Anstieg von $f$ an der Stelle
$\underline x=(a,b)$ in Richtung der Koordinatenachse $(x_1,0)$.
Dieser Anstieg ist die partielle Ableitung
\[D_1 f(a,b) = \frac{\partial f}{\partial x_1}(a,b).\]
%
Nun liegt es nicht fern sich zu fragen, wie der Anstieg in eine
beliebige Richtung bestimmt wird. Das wäre dann eine wesentliche
Verallgemeinerung der partiellen Ableitung.
%
Um die Ableitung in eine beliebige Richtung zu bilden dient, wie der
Name schon sagt, die Richtungsableitung. Sei $\underline v$
ein Einheitsvektor. Die Richtungsableitung ist definiert durch
\[D_{\underline v} f(\underline x) := \lim_{h\rightarrow 0}
\frac{f(\underline x+h\underline v)-f(\underline x)}{h}.\]
Es gibt eine Formel für die praktische Berechnung, nämlich
\[D_{\underline v} f(\underline x)
= \langle\nabla f(\underline x),\underline v\rangle.\]
Das ist eine überraschend einfache Formel. Der wesentliche
Berechnungsaufwand liegt in der Bestimmung des Gradienten
an der Stelle $\underline x$.

Die partiellen Ableitungen sind tatsächlich Spezialfälle der
Richtungsableitung. Sei z.B. $\underline v=e_1$. Man erhält dann
\begin{gather*}
D_{\underline v} f
= \langle e_1 D_1f+e_2 D_2 f, e_1\rangle\\
= \langle e_1, e_1\rangle D_1f  + \langle e_2, e_1\rangle D_2 f\\
= D_1f+0 = D_1f.
\end{gather*}

\subsection{Koordinatensysteme}

Polarkoordinaten können durch eine Transformation von
Polarkoordinaten in kartesische Koordinaten dargestellt werden.
Es ist
\[x = \begin{bmatrix}
x_1\\ x_2 \end{bmatrix}
= f(r,\varphi) = \begin{bmatrix}
r\cos\varphi\\
r\sin\varphi
\end{bmatrix}.\]
Seien $x=(x_1,x_2)$ die kartesischen Koordinaten und
$u=(r,\varphi)$ die Polarkoordinaten. Allgemein ist eine
Transformation von gekrümmten in kartesische Koordinaten gegeben durch
$x=f(u).$

Damit überall zurücktransformiert werden kann, muss die
Jacobi"=Matrix $J=Df(u)$ an jeder stelle regulär sein. Es muss also
überall $\det J\neq 0$ sein. Das ist bei den Polarkoordinaten nur
für $x\neq 0$ der Fall, denn dort lässt sich ja der Winkel nicht
bestimmen. Bei Polarkoordinaten ist $\det J=r$.

Die partiellen Ableitungen $D_k f(u)$ sind die Tangentialvektoren
des Koordinatennetzes. Die Tangentialvektoren spannen einen
Vektorraum, den sogenannten Tangentialraum auf. Die
Tangentialvektoren sind die Basisvektoren des Tangentialraums.
Wenn die Jacobi-Matrix singulär wird, so gelingt es den
Tangentialvektoren nicht, den Tangentialraum aufzuspannen.

Man kann jedem Punkt der Bildmenge $\mathrm{Bild}(f)$ also einen
Tangentialraum anheften. Das Problem dabei ist, dass die Basis im
Allgemeinen keine Orthonormalbasis sein wird. Uns würde es schon
reichen, wenn es eine Orthogonalbasis wäre, denn dann könnten wir sie
einfach normieren, indem wir jeden Basisvektor durch seinen Betrag
teilen. Aber nicht mal das ist der Fall. Es gibt jetzt zwei
Möglichkeiten. Entweder man verwendet ein Orthogonalisierungsverfahren
oder man verwendet den Formalismus mit dem metrischen Tensor.
Die zweite Möglichkeit scheint eleganter zu sein.

Wir werden mit
\[e_k = D_kx = \partial_kx = \frac{\partial x}{\partial u^k}\]
die Basisvektoren bezeichnen. Mit
\[e^k = \mathrm dx^k = D^kx = \partial^kx = \frac{\partial x}{\partial u_k}\]
werden die dualen Basisvektoren bezeichnet. Die Basisvektoren lassen
sich, wie schon gesagt mit
\[e_k = \frac{\partial x}{\partial u^k}.\]
berechnen. Die Jacobi-Matrix lautet
\[J_{ij} = \frac{\partial x_i}{\partial u^j}.\]
Die Jacobi-Matrix enthält also die gesamte Information über die
Basisvektoren und damit auch die gesamte Information über den
Tangentialraum. Die Basisvektoren sind die Spaltenvektoren der
Jacobi"=Matrix.

Die Einträge des metrischen Tensors sind ja die Skalarprodukte der
Basisvektoren. Es ist
\[g_{ij} = \langle D_i x, D_j x\rangle
=  \left\langle \frac{\partial x}{\partial u^i},
\frac{\partial x}{\partial u^j}\right\rangle.\]
Der metrische Tensor lässt sich auch als gramsche Matrix von $J$
berechnen. Es ist
\[g = \mathrm{Gram}(J) = J^T\!J.\]
Die Umkehrfunktion der Transformation $x(u)$ ist $u(x)$.
Mit dieser bestimmt man
\[(J^{-1})_{ij} = \frac{\partial u^i}{\partial x_j}.\]
Das Kronecker-Delta ist definiert durch
\[\delta_{ij} = [i=j]
= \begin{cases}
1 & \mathrm{wenn}\;i=j,\\
0 & \mathrm{sonst}.
\end{cases}\]
Die Kronecker-Deltas sind die Einträge der Einheitsmatrix.
Es ist noch zu überprüfen, ob $J^{-1}$ wirklich die inverse Matrix
von $J$ ist. Das schafft die Rechnung
\begin{gather*}
(J^{-1}J)_{ij} = \sum_{k}(J^{-1})_{ik}J_{kj}\\
= \sum_{k} \frac{\partial u^i}{\partial x_k}
\frac{\partial x_k}{\partial u^j}
= \frac{\partial u^i}{\partial u^j}
= \delta_{ij}.
\end{gather*}
Dabei haben wir uns die Kettenregel zu nutze gemacht. Diese Rechnung
lässt sich auch kompakt formulieren. Sei $u$ die Variable und
$f(u)$ die Funktion mit $x=f(u)$. Es ist
\[J^{-1}J = Df^{-1}(f)Df = D[f^{-1}(f)] = Du = E.\]
Es ist also tatsächlich
\[J^{-1}J = E.\]
Die inverse Matrix des metrischen Tensors lässt sich mit der inversen
Jacobi"=Matrix ausdrücken, denn es ist
\[g^{-1} = (J^T\!J)^{-1} = J^{-1}(J^T)^{-1} = J^{-1}(J^{-1})^T.\]
Das heißt
\[g^{-1} = \mathrm{Gram}((J^{-1})^T).\]
%
Mit den dualen Basisvektoren geht man analog vor und definiert
\[g^{ij} = \langle e^i, e^j\rangle.\]
Es ist nun so, dass ein Basisvektor zu jedem dualen Basisvektor außer
dem mit dem gleichen Index rechtwinklig ist. Haben Basisvektor und
dualer Basisvektor den gleichen Index, so ist ihr Skalarprodukt eins.
Man fasst diese beiden Feststellungen zusammen zu
\[\langle e^i,e_j\rangle = \delta_{ij}.\]
%
Weiterhin können wir Indizes mit dem metrischen Tensor ja heben und
senken. Daher ergibt sich
\begin{gather*}\langle e^i,e_j\rangle
= \langle \sum_{k} g^{ik}e_k,e_j\rangle\\
= \sum_{k}g^{ik}\langle e_k,e_j\rangle
= \sum_{k}g^{ik}g_{kj}.
\end{gather*}
In Matrizenschreibweise ist das
\[E = (\delta_{ij}) =  g'g.\]
Bei $g'=(g^{ij})$ muss es sich also um $g^{-1}$ handeln.

Für den metrischen Tensor haben wir die Formel
\[g_{ij} = \sum_{k} \frac{\partial x_k}{\partial u^i}
\frac{\partial x_k}{\partial u^j}.\]
Wir wollen so eine Formel auch für $g^{ij}$ haben.\\
Mit $g^{ij}=J^{-1}(J^{-1})^T$ bekommt man
\[g^{ij} = \sum_{k} \frac{\partial u^i}{\partial x_k}
\frac{\partial u^j}{\partial x_k}.\]

\subsection{Der Gradient}

Sei $a(u)$ ein Skalarfeld. In gekrümmten Koordinaten lässt sich
der Gradient nicht mehr wie in kartesischen Koordinaten berechnen.
Wir finden eine allgemeine Formel für den Gradient, indem wir ihn in
kartesischen Koordinaten betrachten. Sei $(b_k)$ eine
Orthonormalbasis. Es ist
\begin{gather*}\mathrm{grad}\,a
= \sum_{k}\frac{\partial a}{\partial x_k}b_k\\
= \sum_{k}\frac{\partial a}{\partial x_k}\frac{\partial u}{\partial x_k}
= \sum_{k,i}\frac{\partial a}{\partial u^i}\frac{\partial u^i}{\partial x_k}
\frac{\partial u}{\partial x_k}\\
= \sum_{k,i,j}\frac{\partial a}{\partial u^i}\frac{\partial u^i}{\partial x_k}
\frac{\partial u^j}{\partial x_k}e_j
= \sum_{i,j}\frac{\partial a}{\partial u^i} g^{ij}e_j.
\end{gather*}
%
Der Gradient soll als Beispiel in Zylinderkoordinaten
$u=(r,\varphi,z)$ berechnet werden. In Zylinderkoordinaten
berechnet man $g = \mathrm{diag}(1,r^2,1)$. Damit ergibt sich
$g^{-1} = \mathrm{diag}(1,r^{-2},1)$. Der Gradient ist also
\[\mathrm{grad}\,a = \frac{\partial a}{\partial r} e_r
+\frac{1}{r^2}\frac{\partial a}{\partial \varphi} e_\varphi
+\frac{\partial a}{\partial z} e_z.\]
Die Basisvektoren sind $e_r=(\cos\varphi,\sin\varphi,0)$
und $e_\varphi = (-r\sin\varphi,r\cos\varphi,0)$
sowie $e_z=(0,0,1)$. Damit ergeben sich die Beträge $|e_r|=1$
und $|e_\varphi|=r$ sowie $|e_z|=1$. Mit den normierten
Basisvektoren
\[\hat e_k = \frac{e_k}{|e_k|}\]
lautet der Gradient nun
\[\mathrm{grad}\,a = \frac{\partial a}{\partial r} \hat e_r
+\frac{1}{r}\frac{\partial a}{\partial \varphi}\hat e_\varphi
+\frac{\partial a}{\partial z} \hat e_z.\]
%
Eine elegante Formulierung des Gradienten soll noch vorgestellt
werden. Dazu erinnern wir uns an die musikalischen Operatoren.
Allgemein ist
\begin{gather*}
v^\flat = (\sum_k v^k e_k)^\flat
= \sum_{k,i} g_{ki}v^k\mathrm dx^i,\\
\omega^\sharp = (\sum_k \omega_k\mathrm dx^k)^\sharp
= \sum_{k,i} g^{ki}\omega_k\mathrm e_i.
\end{gather*}
Die Cartan-Ableitung eines Skalarfeldes $a$ berechnet man in
kartesischen Koordinaten mit
$\mathrm da = \nabla^\flat a$. 
Man kann jetzt auf beiden Seiten der Gleichung heben und erhält
\[\mathrm{grad}\,a = (\mathrm da)^\sharp.\]
Diese schöne Formel ist auch in gekrümmten Koordinaten gültig.
Es ergibt sich nämlich
\begin{gather*}
(\mathrm da)^\sharp
= \Big(\sum_i \frac{\partial a}{\partial u^i}
\,\mathrm dx^i\Big)^\sharp
= \sum_{i,j} \frac{\partial a}{\partial u^i}g^{ij}e_j.
\end{gather*}

\subsection{Volumina}

In kartesischen Koordinaten berechnet man Abstände mit dem Satz des
Pythagoras. In gekrümmten Koordinaten geht das nicht mehr. Man denke
sich eine Gerade in kartesischen Koordinaten. Nach der Transformation
in gekrümmte Koordinaten sieht die Gerade auch gekrümmt aus. Wie
berechnen wir nun davon die Länge? Nun ja, unter einer starken Lupe
sieht die Kurve ja noch gerade aus. Es ist nicht mehr
\[s^2 = \Delta x^2+\Delta y^2.\]
Jedoch haben wir noch
\[\mathrm ds^2 = \mathrm dx^2+\mathrm dy^2.\]
Die kleinen Teile muss man nur alle aufsummieren. Es ergibt sich
\begin{gather*}
s = \int \sqrt{\mathrm dx^2+\mathrm dy^2}
= \int_{t_1}^{t_2}\sqrt{\Big(\frac{\mathrm dx}{\mathrm dt}\Big)^2
+\Big(\frac{\mathrm dy}{\mathrm dt}\Big)^2}\,\mathrm dt.
\end{gather*}
Die allgemeine Formel lautet
\[s = \int_{t1}^{t2} |\underline x'(t)|\,\mathrm dt.\]
Das Quadrat des Wegelements ist
\[\mathrm ds^2 = \sum_k \mathrm dx_k^2.\]
Bisher haben wir uns immer noch in kartesischen Koordinaten bewegt.
Die Transformation ist $x(u)$. Zur Abkürzung wird die Schreibweise
$\dot x=x'(t)$ benutzt. Mit der Kettenregel ergibt sich
\begin{gather*}
\dot x_1^2+\dot x_2^2\\
= (D_1x_1\dot u_1+D_2x_1\dot u_2)^2
+ (D_1x_2\dot u_1+D_2x_2\dot u_2)^2\\
= (D_1x_1)^2\dot u_1^2+2D_1x_1D_2x_1\dot u_1\dot u_2
+ (D_2x_1)^2\dot u_2\\
+\; (D_1x_2)^2\dot u_1^2+2D_1x_2D_2x_2\dot u_1\dot u_2
+ (D_2x_2)^2\dot u_2\\
= [(D_1x_1)^2(D_1x_2)^2]\dot u_1^2\\
+\; 2[D_1x_1D_2x_1+D_1x_2D_2x_2]\dot u_1\dot u_2\\
+\; [(D_2x_1)^2+(D_2x_2)^2]\dot u_2\\
= g_{11}\dot u_1^2+2g_{12}\dot u_1\dot u_2+g_{22}\dot u_2^2.
\end{gather*}
Damit ist das Wegelement
\[\mathrm ds^2 = \sum_{i,j} g_{ij}\mathrm du^i\mathrm du^j.\]
Mit dem metrischen Tensor senkt man einen der Indizes
und erhält
\[\mathrm ds^2 = \sum_{k}\mathrm du^k\mathrm du_k.\]
Die Formel für die Bogenlänge ist
\[s = \int_{t1}^{t2}
\sqrt{\langle gu',u'\rangle}\,\mathrm dt.\]
%
Als Beispiel soll der Umfang eines Kreises in Polarkoordinaten
berechnet werden. Der metrische Tensor ist $g=(1,r^2)$.
Man rechnet
\begin{gather*}
u(t)=re_1+te_2,\quad
u'(t) = e_2,\\
gu'(t) = r^2e_2,\quad
\langle gu',u'\rangle = r^2.
\end{gather*}
Es ergibt sich
\[s = \int_0^{2\pi} r\,\mathrm dt = 2\pi r.\]
%
Sei $h(u)$ eine Koordinatentransformation mit Definitionsbereich
$M\subseteq\mathbf R^n$ und Zielmenge $\mathbf R^n$.
Sei $V=h(M)$. Wenn $h(u)$ bestimmte Voraussetzungen erfüllt,
dann ist nach dem Transformationssatz
\[V = \int_V \mathrm dV = \int_M |{\det Dh(u)}|\,\mathrm du
= \int_M |{\det J}|\,\mathrm du.\]
Der Transformationssatz kann eine Erleichterung bei der Berechnung
des Volumens bringen. Berechnen wir z.B. den Flächeninhalt eines
Kreises ohne Transformationssatz. Ein viertel Kreis hat den
Flächeninhalt
\begin{gather*}
\frac{A}{4} = \int_B \mathrm dx\mathrm dy
= \int_B \mathrm dy\mathrm dx
= \int_{0}^r\int_{0}^{y(x)} \mathrm dy\mathrm dx\\
= \int_{0}^r y(x)\,\mathrm dx
= \int_{0}^r \sqrt{r^2-x^2}\,\mathrm dx\\
= \frac{1}{2}\Big[r^2\mathrm{arcsin}\Big(\frac{x}{r}\Big)
+ x\sqrt{r^2-x^2}\Big]_0^r\\
= \frac{1}{2} r^2 \frac{\pi}{2}
= \frac{1}{4}\pi r^2.
\end{gather*}
Jetzt die Berechnung des Flächeninhalts mit Transformationssatz.
Bei Polarkoordinaten hat man $u_1=r$ und $u_2=\varphi$
sowie $\det h(u)=r$.
\begin{gather*}
A = \int_A \mathrm dx\mathrm dy
= \int_0^{2\pi}\int_0^r r\,\mathrm dr\mathrm d\varphi\\
= \int_0^{2\pi} \frac{1}{2}r^2\mathrm d\varphi
= \frac{1}{2}r^2\,2\pi = \pi r^2.
\end{gather*}
Es ist notwendig, zuvor die Determinante der Jacobi-Matrix zu
bestimmen. Der zentrale Punkt ist hier aber, dass man auf die
Stammfunktion von $\sqrt{r^2-x^2}$ verzichten kann.

Was ist, wenn man eine Parameterfläche in einem Raum hat und man den
Flächeninhalt mit einem Doppelintegral bestimmen möchte? Bei den
euklidischen Koordinaten wissen wir gar nicht, wie wir das integrieren
sollen. Es sind ja drei Differentiale, obwohl man für ein
Doppelintegral zwei benötigt.

Besser ist es, über die Parameter der Fläche zu integrieren.
Da hat man auch zwei Differentiale. Beim Transformationssatz ergibt
sich die Beschränkung, dass die Jacobimatrix quadratisch sein muss.
Wir machen nun die Rechnung
\begin{gather*}
|{\det J}| = \sqrt{\det J\det J}
= \sqrt{\det J^T\det J}\\
= \sqrt{\det(J^TJ)} = \sqrt{\det g}.
\end{gather*}
Man bekommt nun die allgemeine Formel
\[V = \int_M \sqrt{\det g}\,\mathrm du.\]
Man kann außerdem die Formel
\[|v_1\wedge\ldots\wedge v_n|^2 = \det(\langle v_i,v_j\rangle)\]
benutzen. Mit $x=h(u)$ erhält man
\[\sqrt{\det g} = |D_1 x\wedge\ldots\wedge D_n x|.\]
Wenn $h(u)$ nur von der Variable $u=t$ abhängig ist, dann ergibt
sich der Spezialfall $|D_1 x| = |x'(t)|$. Damit erhält man die schon
bekannte Formel für die Bogenlänge.

Der Transformationssatz ist eine Verallgemeinerung der
Substitutionsregel. Die Substitutionsregel erhält man wiederum aus
der Kettenregel. Sei $h(x)$ eine reelle Funktion.
Das Differential ist $\mathrm dh = h'(x)\mathrm dx$. Man erhält
\[\int_{h(a)}^{h(b)} f(h)\mathrm dh = \int_a^b f(x) h'(x)\,\mathrm dx.\]
Wenn wir die Rechnung auf höhere Dimensionen übertragen wollen, dann
müssen wir zunächst schreiben
\[\int_B \mathrm dx_1\mathrm dx_2 = \int_B \mathrm dx_1\wedge\mathrm dx_2.\]
Wenn $x_k(u)$ eine Funktion der $u_k$ ist, dann erhält man
\[\mathrm dx_k = D_1x_k\mathrm du_1+D_2x_k\mathrm du_2.\]
Einsetzen bringt
\begin{gather*}
\mathrm dx_1\wedge\mathrm dx_2
= (D_1x_1\mathrm du_1+D_2x_1\mathrm du_2)\\
\wedge\; (D_1x_2\mathrm du_1+D_2x_2\mathrm du_2)\\
= (D_1 x_1 D_2 x_2 - D_2 x_1 D_1 x_2)\,\mathrm du_1\wedge du_2\\
= \det(J)\,\mathrm du_1\wedge du_2.
\end{gather*}
Für eine quadratische Jacobi-Matrix bekommt man allgemein
\[\mathrm dx_1\wedge\ldots\wedge\mathrm dx_n
= \det(J)\,\mathrm du_1\wedge\ldots\wedge du_n.\]


\subsection{Vektorfelder}

Das Ableiten von Skalarfeldern ist unproblematisch. Beim Ableiten von
Vektorfeldern muss man jedoch beachten, dass die Basisvektoren nun
vom Ort abhängig sind. Daher verwendet man die Produktregel der
Differentialrechnung. Man erhält
\[D_i v = D_i\sum_{j}v^je_j
= \sum_{j}(e_jD_i v^j+v^jD_i e_j).\]
Man kommt jetzt auf die Idee, die Ableitungen $D_ie_j$ als
Linearkombination der Basisvektoren $e_k$ darzustellen. Das sollte
möglich sein, weil die Ableitung eins Vektors wieder ein Vektor sein
wird. Die gekrümmte Fläche $x(u)$ wird mit der Orthonormalbasis
$(b_k)$ dargestellt. Man schreibt
\begin{gather*}
e_j = D_jx = D_j\sum_{l} b_lx_l\\
= \sum_{l}b_l D_jx_l
= \sum_{l}\frac{\partial x}{\partial x_l}D_jx_l.
\end{gather*}
Mit der Kettenregel erhält man
\[D_ie_j = \sum_{k,l}\frac{\partial x}{\partial u^k}
\frac{\partial u^k}{\partial x_l}D_iD_jx_l
= \sum_{k,l}e_k\frac{\partial u^k}{\partial x_l}D_iD_jx_l.\]
Man definiert nun die Abkürzung
\[\Gamma_{ij}^k := \sum_{l}\frac{\partial u^k}
{\partial x_l}D_iD_jx_l.\]
Die $\Gamma_{ij}^k$ werden als Christoffelsymbole bezeichnet.
Damit ist die Linearkombination
\[D_ie_j = \sum_{k} \Gamma_{ij}^k e_k.\]
Die Ableitung des Vektorfeldes ist hiermit
\begin{gather*}
D_i v  = \sum_{j}(e_jD_i v^j+v^j\sum_{k} \Gamma_{ij}^k e_k)\\
= \sum_{j}e_jD_i v^j+\sum_{k,j}v^j \Gamma_{ij}^k e_k.
\end{gather*}
Da der Index $j$ in der ersten Summe durch die Summe gebunden ist,
kann man ihn durch $k$ austauschen. Man erhält
\begin{gather*}
D_i v = \sum_{k}e_kD_i v^k+\sum_{k,j}v^j \Gamma_{ij}^k e_k\\
= \sum_{k}e_k(D_i v^k+\sum_{j}v^j\Gamma_{ij}^k).
\end{gather*}
Man definiert nun die kovariante Ableitung
\[\nabla_i v^k := D_i v^k+\sum_{j}v^j\Gamma_{ij}^k.\]
Damit ist
\[D_i v = \sum_{k}e_k\nabla_i v^k.\]
Die Schreibweise $\nabla_i v^k$ ist etwas unglücklich, da die
Information des gesamten Vektorfeldes notwendig ist, um die Ableitung
bilden zu können. Die Komponente $v^k$ allein reicht dafür nicht
aus. Man muss $\nabla_i v^k=(\nabla_i v)^k$ lesen. Wenn man
$\nabla_i v^k=\nabla_i(v^k)$ liest, dann ergibt das keinen Sinn.

Die Richtungsableitung soll nun auf Vektorfelder verallgemeinert
werden. Sei $a$ ein Vektor. Für ein Skalarfeld $f$ ist die
Richtungsableitung
\begin{gather*}
{D_a}f = \langle a,\mathrm{grad}\,f\rangle
= \langle \sum_k a^ke_k,\sum_k e^k D_k f\rangle\\
= \sum_k a^k D_k f.
\end{gather*}
Obwohl der Gradient die $e^k\equiv\mathrm dx^k$ enthält, ist es bei
der Richtungsableitung wieder unerheblich, ob man kartesische
Koordinaten hat oder nicht.

Die Richtungsableitung von einem Vektorfeld in kartesischen
Koordinaten kann man bilden, indem man die Richtungsableitung
komponentenweise bildet. D.h. man kann die Komponenten eines
Vektorfeldes als Skalarfelder betrachten. Bei gekrümmten Koordinaten
macht meine eine Rechnung, die zur Rechnung oben analog ist.
Man erhält zunächst
\[D_a e_j = \sum_{k,i}a^i \Gamma_{ij}^k e_k.\]
Man definiert
\[\nabla_a v^k := D_a(v^k) + \sum_{i,j}a^i v^j \Gamma_{ij}^k.\]
Damit hat man
\[D_a v = \sum_k e_k \nabla_a v^k.\]
%
Man kann nachrechnen, dass $\nabla_{e_i}v^k=\nabla_i v^k$ ist.
Um das zu überprüfen macht man die Rechnung
$(e^i)^l = \langle e^i,e_l\rangle = \delta_{il}.$
Damit ist
\[\nabla_{e_i} v^k = D_{e_i}(v^k)
+ \sum_{l,j}(e^i)^l v^j \Gamma_{il}^k
= D_i v^k + \sum_{j} v^j\Gamma_{ij}^k.\]
Wir wollen nun herausfinden, wie man Kovektoren ableitet.
Mit der Formel $D_ie_j=\sum_k\Gamma_{ij}^k e_k$ lassen sich
umgekehrt die Christoffelsymbole als Projektion darstellen. Es ist
$\Gamma_{ij}^k = \langle e^k,D_ie_j\rangle.$
Mit der Produktregel ist weiterhin
\[0=D_i\langle e^k,e_j\rangle
= \langle D_ie^k,e_j\rangle + \langle e^k,D_ie_j\rangle.\]
Damit erhält man
\[D_ie^k = -\sum_{j}\Gamma_{ij}^ke^j\]
Mit dieser Formel findet man
\[\nabla_i v_k = D_iv_k-\sum_{j}v_j\Gamma_{ik}^j.\]

\subsection{Die Volumenform}

Wir wollen den Hodge-Stern-Operator definieren. Da dieser Operator
etwas kompliziert ist, wollen wir das zunächst auf Orthonormalbasen
beschränken. Der Hodge-Operator ist linear, es ist
$*(v+w) = *v+*w$ und $*(rv)=r(*v)$.

Sei $n$ die Dimension des Raums. Man definiert
\[*(e_{\sigma(1)}\wedge\ldots\wedge e_{\sigma(p)})
:= \mathrm{sgn}(\sigma)\,e_{\sigma(p+1)}\wedge\ldots\wedge e_{\sigma(n)}.\]
Es ergibt sich der Spezialfall
\[*(e_1\wedge\ldots\wedge e_p) = e_{p+1}\wedge\ldots\wedge e_n.\]
Ein Beispiel. Für $n=4$ soll $*(e_1\wedge e_4)$
berechnet werden. Man ergänzt die fehlenden Vektoren
auf der anderen Seite und erhält
\[*(e_1\wedge e_4) = se_2\wedge e_3.\]
Das Vorzeichen $s$ bestimmt man, indem man die Permutation
$(1423)$ per Transposition sortiert und bei jeder Transposition
das Vorzeichen wechselt. Man rechnet
\begin{gather*}
s = \mathrm{sgn}(1423)=-\mathrm{sgn}(1324) = \mathrm{sgn}(1234) = 1.
\end{gather*}
Man definiert
\[I_n := e_1e_2\ldots e_n = e_1\wedge\ldots\wedge e_n.\]
Damit definiert man noch $*1:=I_n$ und $*I_n:=1$.
Wenn sich $n$ aus dem Kontext ergibt, dann schreibt man auch $I:=I_n$.

Außerdem legt man $I_0:=1$ fest.

Sei $A$ vom Grad $r$ und $B$ vom Grad $s$. Dann gilt
\[A\wedge B = (-1)^{rs} B\wedge A.\]
Man kann $I_n$ in drei Teile zerlegen. Es ist
\[I_n = I_{k-1}e_k J_{n-k} = I_{k-1}\wedge e_k\wedge J_{n-k}.\]
Damit erhält man
\begin{gather*}
I_n e_k = (-1)^{n-k} I_{k-1}J_{n-k}e_ke_k\\
= (-1)^{n-k} I_{k-1}J_{n-k}.
\end{gather*}
Anders herum ist
\[e_k I_n =  (-1)^{k-1} e_ke_k I_{k-1}J_{n-k}
= (-1)^{k-1} I_{k-1}J_{n-k}.\]
Weiterhin ist
\[*e_k = \mathrm{sgn}(k1\ldots\hat k\ldots n)I_{k-1}J_{n-k}.\]
Das Zirkumflex bedeutet, dass $k$ dort ausgelassen wird.
Man rechnet nun
\[\mathrm{sgn}(1\ldots n) = (-1)^{k-1}
\mathrm{sgn}(k1\ldots\hat k\ldots n).\]
Ein Vergleich bringt
\[*e_k = e_k I_n = (-1)^{n-1} I_n e_k.\]
Für einen Vektor $v$ erhält man also
\[*v = vI_n = (-1)^{n-1}I_n v.\]
Sei $a_k$ ein Vektor aus $(e_k)$ und seien die $a_k$
paarweise verschieden.
Man definiert noch die Reversion
\[R(a_1\ldots a_n) := (a_n\ldots a_1).\]
Man stellt fest, dass $R(I_n)=(-1)^{(n/2)(n-1)}I_n$ ist.
Es ergibt sich weiterhin
\[R(I_p)I_n = R(I_p)I_p I_{n-p} = I_{n-p}.\]
Außerdem ist ja $*I_p = I_{n-p}$. Ein Vergleich bringt $*I_p = R(I_p)I_n$.
Wenn nun $A$ eine Permutation von $I_p$ ist, dann hat man $A=sI_p$,
wobei $s$ ein Vorzeichen ist. Man erhält also
\begin{gather*}
*A = *(sI_p) = s\,{*I_p} = sR(I_p)I_n\\
= R(sI_p)I_n = R(A)I_n.
\end{gather*}
Das legt den Verdacht nahe, dass die Formel
$*A=R(A)I_n$ allgemein gültig ist.
Mit der Formel $R(AB) = R(B)R(A)$ kann man auch rechnen
\[*(AB) = R(AB)I_n = R(B)R(A)I_n = R(B)\,{*A}.\]
Wir benutzen nun die Formel
\[e_{\sigma(1)}\wedge\ldots\wedge e_{\sigma(n)}
= \mathrm{sgn}(\sigma)\, e_1\wedge\ldots\wedge e_n.\]
Damit rechnet man
\begin{gather*}
(e_{\sigma(p)}\wedge\ldots\wedge e_{\sigma(1)})I_n\\
= (e_{\sigma(p)}\wedge\ldots\wedge e_{\sigma(1)})
(e_1\wedge\ldots\wedge e_n)\\
= \mathrm{sgn}(\sigma)(e_{\sigma(p)}\wedge\ldots\wedge e_{\sigma(1)})
(e_{\sigma(1)}\wedge\ldots\wedge e_{\sigma(n)})\\
= \mathrm{sgn}(\sigma)\, e_{\sigma(p+1)}\wedge\ldots\wedge e_{\sigma(n)}.
\end{gather*}
Damit ist die Formel $*A=R(A)I_n$ bestätigt.
Wenn $A$ vom Grad~$r$ ist, dann ist
$R(A) = (-1)^{(r/2)(r-1)}A$.

Außerdem ist
\[I_n^2 = (-1)^{(n/2)(n-1)} R(I_n)I_n = (-1)^{(n/2)(n-1)}.\]
Weiterhin ist
\[*{*A} = *(R(A)I_n) = R(I_n)AI_n = sR(I_n)I_nA = sA.
\]
Für das Vorzeichen gilt
\begin{gather*}
AI_n = sI_nA = sR(R(A)R(I_n))\\
= s(-1)^{r(r-1)/2}(-1)^{(n(n-1)/2} R(AI_n).
\end{gather*}
Der Grad von $AI_n$ ist $n-r$.
Multipliziert man auf beiden Seiten mit $AI_n$, so ergibt sich
\begin{gather*}
(AI_n)^2 = (-1)^{(n-r)(n-r-1)/2}\\
= s(-1)^{r(r-1)/2}(-1)^{n(n-1)/2}.
\end{gather*}
Man formt nach $s$ um. Den Exponenten vereinfacht man zu
\[n^2+r^2-n-nr = n(n-1) + r(r-n).\]
Damit erhält man
\[*{*A} = (-1)^{r(n-r)}A.\]
Wie berechnet man den Hodge-Operator nun, wenn man keine
Orthonormalbasis hat?

Ab jetzt sollen die $e_k$ nicht mehr unbedingt ein Orthonormalsystem
bilden. Die Formel $e_1e_2=e_1\wedge e_2=-e_2e_1$ können wir leider
nicht mehr benutzen. Diesem Problem soll im Folgenden Abhilfe
geschafft werden.

Das geometrische Produkt von Basisvektoren ist
\[e_1e_2 = \langle e_1,e_2\rangle + e_1\wedge e_2.\]
Da $e_1\wedge e_2$ senkrecht zu den beiden Basisvektoren steht, ist
\[|e_1|^2|e_2|^2 = \langle e_1,e_2\rangle^2+|e_1\wedge e_2|^2.\]
Mit dem metrischen Tensor erhält man
\[g_{11} g_{22} = g_{12} g_{12} + |e_1\wedge e_2|^2\]
bzw. $|e_1\wedge e_2|^2 = \det g.$
Der Basisbivektor hat also einen Flächeninhalt der von eins abweichen
kann. Man bildet deswegen den normierten Bivektor
\[I = \frac{1}{\sqrt{\det g}}e_1\wedge e_2.\]
Es gibt auch noch $I'=\sqrt{\det g}\,e^1\wedge e^2$. Welchen man
benutzt ist unerheblich, denn es ist $I'=I$. Man kann den Strich
also weglassen.

Für höherdimensionale Räume hat man
\[I = \frac{1}{\sqrt{\det g}}e_1\wedge\ldots\wedge e_n.\]
Alternativ ist
\[I' = \sqrt{\det g}\,e^1\wedge\ldots\wedge e^n.\]
Mit der Formel
\[a_1\wedge\ldots\wedge a_n
= \det(a_1,\ldots,a_n)e_1\wedge\ldots\wedge e_n\]
macht man die Rechnung
\begin{gather*}
\bigwedge_k e^k
= \bigwedge_k \sum_{i_k} g^{{i_k}k} e_{i_k}\\
= \det(g^{{i_1}1},\ldots,g^{{i_n}n})\bigwedge_k e_k
= \det(g)\bigwedge_k e_k.
\end{gather*}
Damit bekommt man $I'=I$.

Sei $v$ ein Vektorfeld. Das geometrische Produkt $vI$
soll berechnet werden. Es ist nun problematisch, dass die
Basisvektoren nicht rechtwinklig aufeinander stehen. Jedoch
steht $e_1$ rechtwinklig zu $e^2$ und $e_2$ zu $e^1$.
Das wollen wir ausnutzen. Man hat $e^ie_j=-e_je^i$ für $i\ne j$.

Der Vektor $e_2$ lässt sich als Linearkombination von $e_1$
und $e^2$ darstellen. Es ist ja $e^2=g^{21}e_1+g^{22}e_2$.
Umformen bringt
\[e_2 = \frac{1}{g^{22}}(e^2-g^{21}e_1).\]
Damit erhält man
\[e_1\wedge e_2
= \frac{1}{g^{22}}(e_1\wedge e^2-g^{21}e_1\wedge e_1)
= \frac{1}{g^{22}}e_1\wedge e^2.\]
Man rechnet nun
\begin{gather*}
e_1(e_1\wedge e_2) = \frac{1}{g^{22}}e_1(e_1\wedge e^2)
= \frac{1}{g^{22}}e_1e_1e^2\\
= \frac{1}{g^{22}}\langle e_1,e_1\rangle e^2
= \frac{g_{11}}{g^{22}}e^2.
\end{gather*}
Man benutzt nun die Formeln
\begin{gather*}
\begin{bmatrix}
g^{11} & g^{12}\\
g^{21} & g^{22}
\end{bmatrix}
= \frac{1}{\det(g)}
\begin{bmatrix}
g_{22} & -g_{12}\\
-g_{21} & g_{22}
\end{bmatrix},\\
\begin{bmatrix}
g_{11} & g_{12}\\
g_{21} & g_{22}
\end{bmatrix}
= \det(g)
\begin{bmatrix}
g^{22} & -g^{12}\\
-g^{21} & g^{22}
\end{bmatrix}.
\end{gather*}
Damit erhält man
\[e_1(e_1\wedge e_2) = \det(g)\,e^2.\]
Über analoge Rechnung erhält man die anderen Formeln
\begin{gather*}
e_2(e_1\wedge e_2) = -\det(g)\,e^1,\\
e^1(e^1\wedge e^2) = \det(g)^{-1}\,e_2,\\
e^2(e^1\wedge e^2) = -\det(g)^{-1}\,e_1.
\end{gather*}
Damit ergibt sich
\begin{gather*}
*v = vI = \sqrt{\det g}\,(v^1 e^2 - v^2 e^1)\\
= \sqrt{\det g}\,(v^1 \mathrm dx^2 - v^2 \mathrm dx^1).
\end{gather*}
Mann kann noch die Indizes senken und erhält
\begin{gather*}
*v = \sqrt{\det g}\sum_{i,j}\varepsilon_{ij}v^i \mathrm dx^j\\
= \sqrt{\det g}\sum_{i,j}
\sum_{k}\varepsilon_{ij}g^{ik}v_k \mathrm dx^j.
\end{gather*}
Weiterhin ist ja noch $*1 = 1I=I$.

Man verwendet nun
\[(e_1\wedge e_2)(e_1\wedge e_2) = -\det(e_1,e_2) = -\det(g).\]
Damit rechnet man
\begin{gather*}
*I = R(I)I = \frac{1}{\det g}(e_2\wedge e_1)(e_2\wedge e_1)\\
= -\frac{1}{\det g}(e_1\wedge e_2)(e_2\wedge e_1)\\
= -\frac{1}{\det g}(-1)\det(g) = 1.
\end{gather*}
%
Die Formel für Orthogonalbasen lässt sich sehr leicht herleiten.
Sei $b_i=\frac{e_i}{|e_i|}$. Die Rechnung ist
\begin{gather*}
*(e_{\sigma(1)}\wedge\ldots\wedge e_{\sigma(k)})\\
= |e_{\sigma(1)}|\ldots|e_{\sigma(k)}|\,
{*}(b_{\sigma(1)}\wedge\ldots\wedge b_{\sigma(k)})\\
= |e_{\sigma(1)}|\ldots|e_{\sigma(k)}|\,
\mathrm{sgn}(\sigma)(b_{\sigma(k+1)}
\wedge\ldots\wedge b_{\sigma(n)})\\
= \frac{|e_{\sigma(1)}|\ldots|e_{\sigma(k)}|}
{|e_{\sigma(k+1)}|\ldots|e_{\sigma(n)}|}\,
\mathrm{sgn}(\sigma)\\
(e_{\sigma(k+1)} \wedge\ldots\wedge e_{\sigma(n)}).
\end{gather*}
Die Reihenfolge der Basisvektoren ist bei der Determinante
des metrischen Tensors nicht wichtig. Man hat
\begin{gather*}
|e_{\sigma(k+1)}|\ldots|e_{\sigma(n)}|
= \frac{\sqrt{\det g}}{|e_{\sigma(1)}|\ldots|e_{\sigma(k)}|}.
\end{gather*}
Damit ergibt sich
\begin{gather*}
*(e_{\sigma(1)}\wedge\ldots\wedge e_{\sigma(k)})\\
= g_{\sigma(1)\sigma(1)}\ldots g_{\sigma(k)\sigma(k)}
\frac{1}{\sqrt{\det g}} \mathrm{sgn}(\sigma)\\
(e_{\sigma(k+1)} \wedge\ldots\wedge e_{\sigma(n)}).
\end{gather*}
Analog dazu ist
\begin{gather*}
*(e^{\sigma(1)}\wedge\ldots\wedge e^{\sigma(k)})\\
= g^{\sigma(1)\sigma(1)}\ldots g^{\sigma(k)\sigma(k)}
\sqrt{\det g}\;\mathrm{sgn}(\sigma)\\
(e^{\sigma(k+1)} \wedge\ldots\wedge e^{\sigma(n)}).
\end{gather*}

\subsection{Der Pullback}

Bei der Substitutionsregel der Integralrechnung wird eine
Substitution vorgenommen. Dabei wird die Variable $u$ gegen die Funktion
$u(x)$ ausgetauscht. Die Funktion $f(u)$ wird daher zur
Verkettung
\[(f\circ u)(x) = f(u(x)).\]
Beim Differential muss die Substitution natürlich auch ausgeführt
werden. Es ergibt sich
\[\mathrm du(x) = u'(x)\,\mathrm dx.\]
Verwendet man die Iverson"=Klammer bzw. charakteristische Funktion,
so sieht man, dass die Substitution auch für die Variable zwischen
den Grenzen vorgenommen werden muss. Es ist
\begin{gather*}
\int_{u(a)}^{u(b)} f(u)\,\mathrm du\\
= \int_{-\infty}^{\infty} [u(a)\le u][u\le u(b)] f(u)\,\mathrm du\\
= \int_{-\infty}^{\infty} [u(a)\le u(x)][u(x) \le u(b)] f(u(x))u'(x)\,\mathrm dx\\
= \int_{-\infty}^{\infty} [a\le x][x\le b] f(u(x)) u'(x)\,\mathrm dx\\
= \int_a^b f(u(x))u'(x)\,\mathrm dx.
\end{gather*}
Wenn $u(x)$ nicht monoton steigend ist, dann kann man die
Ungleichungen nicht so einfach kürzen. Die Substitutionsregel
funktioniert in einem solchen Fall aber trotzdem. Der kanonische
Beweis der Substitutionsregel ist äußerst elegant, liefert jedoch
keine so anschauliche Begründung für die Modifikation der Grenzen.

Die Substitution wird auch als Pullback bezeichnet, wenn der
Rechenformalismus auf Differentialformen übertragen wird. Für
eine Funktion ist der Pullback gegeben durch
\[u^{*}f = f\circ u.\]
Der Pullback ist linear. Es ist
\[u^{*}(\omega_1+\omega_2)
= u^{*}\omega_1+u^{*}\omega_2,\]
und wenn $r$ eine konstante Zahl ist, so ist
\[u^{*}(r\omega) = ru^{*}(\omega).\]
Der Pullback verteilt sich außerdem auf Faktoren des
äußeren Produktes. Man sagt, er ist multiplikativ. Es ist
\[u^{*}(\omega_1\wedge\omega_2)
= u^{*}\omega_1\wedge u^{*}\omega_2.\]
Wenn $f$ eine 1"=Form ist, so ist ja $f\wedge\omega = f\omega.$
Somit ergibt sich auch die Rechenregel
\[u^{*}(f\omega) = (u^{*}f)(u^{*}\omega)
= (f\circ u)u^{*}\omega.\] 
Eine weitere wichtige Rechenregel ist die Verträglichkeit
mit der Cartan"=Ableitung. Es ist
\[u^{*}(\mathrm d\omega) = \mathrm d(u^{*}\omega).\]
Schließlich gilt noch die Regel
\[(u_1\circ u_2)^{*} = u_2^{*}\circ u_1^{*}.\]
Setzt man $\omega:=f(u)\mathrm du$, so lässt sich die
Substitutionsregel nun umformulieren zu
\[\int_{u[a,b]} \omega = \int_{[a,b]} u^{*}\omega.\]
Wenn $u$ ein Diffeomorphismus ist, so gilt diese Formel
allgemein und enthält den Transformationssatz. Die allgemeine Formel
lautet
\[\int_{u(B)} \omega = \int_{B} u^{*}\omega.\]
Voraussetzung ist, dass die Orientierung erhalten bleibt, damit man
beim Transformationssatz die Betragsstriche entfernen kann.
Die Formel gilt auch, wenn $u^{-1}$ eine diffeomorphe Karte ist.
Da der $\mathbf R^n$ ein Koordinatensystem bezüglich einer
Orthonormalbasis ist, gilt dort
\[\int_B f(\underline x)\, dx_1\wedge\ldots\wedge dx_n
= \int_B f(\underline x)\, dx_1\ldots dx_n.\]
Integrale auf Mannigfaltigkeiten können damit auf ge"=wöhnliche
mehrdimensionale Integrale zurückgeführt werden.

\subsection{Der Pushforward}

Bei $v(f)$ handelt es sich um eine in der Differentialgeometrie
übliche Schreibweise für die Richtungsableitung $D_v f$. Man
interpretiert das als Anwendung des Vektors $v$ auf die Funktion
$f$. Es ist
\[v(f) = \mathrm df(v) = \langle\mathrm df,v\rangle
= \sum_k v^k D_k f.\]
Formal kann man aber auch gleich
\[v=\sum_k v^k\frac{\partial}{\partial x^k} = \sum_k v^k D_k\]
schreiben. Dann ist
\[v(f) = vf = (\sum_k v^k D_k)f = \sum_k v^k D_k f.\]
%
Seien $M$ und $N$ zwei Mannigfaltigkeiten und sei $f$ eine
Funktion auf $N$. Sei weiterhin $F$ eine Funktion von $M$ nach
$N$. Sei $v$ ein Vektor aus dem Tangentialraum von $M$ am Punkt
$p$. Der \textit{Pushforward} ist definiert durch
\[(F_\ast v)(f) := v(f\circ F).\]
Mit der Kettenregel rechnet man nun
\begin{gather*}
v(f\circ F) = \sum_k v^k D_k(f\circ F)\\
= \sum_{k,i} v^k D_i f(F_i)D_k F_i
= \sum_{k,i} v^k \frac{\partial f}{\partial F_i}
\frac{\partial F_i}{\partial x^k}.
\end{gather*}
Man kann den Pushforward als Verallgemeinerung der Richtungsableitung
ansehen. Wenn man $F=\mathrm{id}$ setzt, dann erhält man ja wieder
die Richtungsableitung.



% \newpage
\section{Funktionalanalysis}

\subsection{Bra-Ket-Notation}

Ein Vektor lässt sich als Linearkombination von Basisvektoren
darstellen. Es ist
\[\underline a = a_1e_1+a_2e_2.\]
Eine alternative Schreibweise dafür ist
\[|a\rangle = a_1|e_1\rangle+a_2|e_2\rangle.\]
Bezeichnet man mit $|1\rangle$ und $|2\rangle$ die beiden
Basisvektoren, so schreibt man auch
\[|a\rangle = a_1|1\rangle+a_2|2\rangle.\]
Die Zahlen $a_1$ und $a_2$ sollen nun komplexe Zahlen sein.
Damit wird der Vektorraum zu einem komplexen Vektorraum. Wir können
auch für einen komplexen Vektorraum ein Skalarprodukt definieren.
Das Skalarprodukt darf nie negativ werden, weil die Wurzel aus dem
Skalarprodukt immer ein anschauliches Ergebnis sein soll. Würde man
das Skalarprodukt als
\[\langle a,b\rangle = \sum_{k=1}^n a_kb_k\]
definieren, so wäre $\langle a,b\rangle\geq 0$ nicht immer
gewährleistet. Wir schaffen dem Abhilfe, indem die $a_k$ konjugiert
werden. Das heißt, der Imaginärteil der $a_k$ wird negiert.

Definition. Das Standardskalarprodukt von Vektoren aus einem komplexen
Vektorraum ist
\[\langle a,b\rangle = \sum_{k=1}^n \overline a_kb_k.\]
Das Skalarprodukt ist nach dieser Definition nicht bilinear, sondern
sesquilinear. Es ist auch nicht mehr symmetrisch, sondern hermitisch.
Es ist
\begin{gather*}
\langle ra,b\rangle = \overline r\langle a,b\rangle,\\
\langle a,rb\rangle = r\langle a,b\rangle,\\
\langle b,a\rangle = \overline{\langle a,b\rangle}.
\end{gather*}
Die Konjugation kann anstelle von $\overline z$ alternativ auch
durch $z^\ast$ ausgedrückt werden.

Man definiert auch
\begin{gather*}
\langle a| = [\overline a_1,\overline a_2],\qquad
|a\rangle = \begin{bmatrix}a_1\\ a_2\end{bmatrix}.
\end{gather*}
Das Skalarprodukt $\langle a|b\rangle$ lässt sich nun durch eine
Matrizenmultiplikation bilden. Man definiert außerdem die
Adjunktion $H$ als
\begin{gather*}
[\overline a_1,\overline a_2]^H
= \begin{bmatrix}a_1\\ a_2\end{bmatrix},\qquad
\begin{bmatrix}a_1\\ a_2\end{bmatrix}^H
= [\overline a_1,\overline a_2].
\end{gather*}
In Bra-Ket-Notation ist also $\langle a|^H=|a\rangle$ und
$|a\rangle^H=\langle a|$. Die Adjunktion soll auch für eine Matrix
definiert werden. Sei
\[A^H = (a_{ij})^H = (\overline a_{ji}).\]
Sei außerdem $\overline A=(\overline a_{ij})$. Man schreibt dann
auch $A^H = \overline A\,{}^T$. Die Transposition einer Matrix ist
der Spezialfall der Adjunktion. Die Adjunktion geht nämlich in die
Transposition über, wenn nur reelle Zahlen verwendet werden. Für eine
reelle Matrix ist nämlich $\overline A=A$.

Die Adjunktion hat fast die gleichen Rechenregeln, wie die
Transposition. Die Rechenregeln sind
\begin{gather*}
(A+B)^H = A^H+B^H\\
(AB)^H = B^H A^H\\
(rA)^H = \overline r A^H\\
(A^H)^{-1} = (A^{-1})^H\\
(A^H)^H = A\\
\det(A^H) = \overline{\det(A)}
\end{gather*}
Durch den Formalismus mit Zeilenvektoren und Spaltenvektoren kann man
das Skalarprodukt durch eine Matrizenmultiplikation von einem
Zeilenvektor mit einem Spaltenvektor ausdrücken.

Sei $A$ eine quadratische Matrix. Die Matrizenmultiplikation ist
ja assoziativ. Es ist also
\[(\langle v|A)|w\rangle = \langle v|(A|w\rangle).\]
Man kann das auch explizit nachrechnen. Wegen dem Assoziativgesetz
schreibt man anstelle von $a(bc)$ auch einfach $abc$. Das wollen
wir bei der Multiplikation von Matrizen auch machen. Wir schreiben also einfach
\[\langle v|A|w\rangle.\]
Für Matrizen ist ja
\[(ABC)^H = C^HB^HA^H.\]
Es ergibt sich daher
\[(\langle v|A|w\rangle)^H = |w\rangle^H A^H \langle v|^H
= \langle w|A^H|v\rangle.\]
Außerdem ist
\[(\langle v|w\rangle)^H = |w\rangle^H\langle v|^H
= \langle w|v\rangle.\]
Die Komponenten eines Vektors bekommt man durch Skalarprodukte mit
den Basisvektoren der Orthonormalbasis. Es ist
\[a_k = \langle e_k|a\rangle.\]
Damit ist
\[|a\rangle = \sum_{k=1}^n a_k |e_k\rangle
= \sum_{k=1}^n \langle e_k|a\rangle |e_k\rangle.\]
Da die Multiplikation mit einem Skalar kommutativ ist, können die
Faktoren in $a_k|e_k\rangle$ zu $|e_k\rangle a_k$ vertauscht
werden. Man klammert $|a\rangle$ dann aus der Summe aus.
Übrig bleibt der Operator
\[\mathrm I = \sum_{k=1}^n |e_k\rangle\langle e_k|.\]
Der Operator $\mathrm I$ ist der Identitätsoperator.
Es ist $\mathrm I|a\rangle = |a\rangle$.


\subsection{Funktionen}

Ein Vektor kann als Punkt im $\mathbf R^n$ angesehen werden.
Zu jedem Punkt gehört genau ein Vektor, welcher vom
Koordinatenursprung auf diesen Punkt zeigt. Dieser Punkt ist
ein Tupel $(a_1,\ldots,a_n)$ von Koordinaten. Dann kann doch auch
jede endliche Folge $(a_k)$ als Vektor angesehen werden, denn die
Folge ist ja auch ein Tupel.

Ein Tupel ist eine Familie mit der Indexmenge
$I=\{k\in\mathbf N|\,k\leq n\}.$ Wir können doch auch die
Indexmenge $I=\mathbf N$ wählen, oder auch $I=\mathbf R$.
Die Familien $(a_k)$ mit $I=\mathbf R$, das sind doch die
reellen Funktionen. Wir setzen $x=k$ und $f(x)=a_k$.

Sind Funktionen so etwas wie Vektoren?

Die Notation $f=x\mapsto 2x$ soll bedeuten, dass $f$ eine
Funktion mit $f(x)=2x$ ist. Eigentlich müsste man zu jeder Funktion
auch den Definitionsbereich und die Zielmenge angeben, aber wenn
wir verlangen, dass die Funktionen alle aus dem selben Funktionenraum
entstammen, so ist dies nicht notwendig.

Seien $f,g$ reelle Funktionen. Ein Skalarprodukt ist z.B.
\[\langle f,g\rangle = \int_a^b f(x)g(x)\,\mathrm dx.\]
Seien $f,g$ Funktionen mit komplexen Werten.
Ein Skalarprodukt ist z.B.
\[\langle f,g\rangle = \int_a^b \overline{f(x)}g(x)\,\mathrm dx.\]
Analog zu Vektoren wird mit dem Skalarprodukt die Norm
\[\|f\| = \sqrt{\langle f,f\rangle}\]
definiert. Analog zu Vektoren werden Funktionen $f,g$ mit
$\langle f,g\rangle=0$ als orthogonal zueinander bezeichnet.
Analog zu Vektoren wird eine Funktion~$f$ normiert genannt,
wenn $\|f\|=1$ ist.

In einem Hilbertraum gibt es Orthonormalbasen. Analog zu Vektoren
kann jede Funktion aus dem Raum in eine uneigentliche
Linearkombination
\[f = \sum_{k\in I} f_ke_k = \sum_{k\in I} \langle e_k,f\rangle e_k.\]
zerlegt werden. Die Funktionen $e_k(x)$ sind die Basisvektoren,
die $f_k$ sind reelle oder komplexe Zahlen.

Beispiel: Sei
\begin{gather*}
B=\{x\mapsto 1,\,x\mapsto\sqrt{2}\cos(k\omega x),\\
x\mapsto\sqrt{2}\sin(k\omega x)|\,\,k\in\mathbf N\}.
\end{gather*}
Die Menge $B$ ist eine Orthonormalbasis, wenn sie als unendliches
Tupel aufgeschrieben wird. Das Skalarprodukt ist
\[\langle f,g\rangle = \frac{1}{T}\int_0^T f(x)g(x)\,\mathrm dx\]
mit $T=2\pi/\omega$. Wir normieren jetzt z.B. auf $\omega=1$ und
rechnen nach
\begin{gather*}
\langle x\mapsto 1,\,x\mapsto\sqrt{2}\cos(x)\rangle
= \frac{1}{2\pi}\int_0^{2\pi} \sqrt{2}\cos x\,\mathrm dx\\
= \frac{1}{2\pi}\sqrt{2}[\sin x]_0^{2\pi}
= \frac{1}{2\pi}\sqrt{2}[\sin(2\pi)-\sin 0] = 0.
\end{gather*}
Die beiden Funktionen sind tatsächlich orthogonal. Außerdem ist z.B.
\begin{gather*}
\|x\mapsto 1\|^2= \frac{1}{2\pi}\int_0^{2\pi} \mathrm dx
= \frac{1}{2\pi}[x]_0^{2\pi}=1.
\end{gather*}
Eine Funktion aus dem Funktionenraum wird mit $B$ zerlegt in die
uneigentliche Linearkombination
\[f(x) = \frac{a_0}{2}+\sum_{k=1}^\infty
[\frac{a_k}{\sqrt{2}}\sqrt{2}\cos(k\omega x)
+\frac{b_k}{\sqrt{2}}\sqrt{2}\sin(k\omega x)].\]
Die Koeffizienten sind
\begin{gather*}
\frac{a_0}{2}
= \langle x\mapsto 1,\,f\rangle,\\
\frac{a_k}{\sqrt{2}}
= \langle x\mapsto \sqrt{2}\cos(k\omega x),\,f\rangle,\\
\frac{b_k}{\sqrt{2}}
= \langle x\mapsto \sqrt{2}\sin(k\omega x),\,f\rangle.
\end{gather*}
%
Sei der Vektor $v$ dargestellt als Linearkombination von
Basisvektoren
\[v = v_1e_1+v_2e_2.\]
und sei $B=(e_1,e_2)$ eine Orthonormalbasis.
Für Vektoren gilt der Satz des Pythagoras in der Form
\[|v|^2 = v_1^2+v_2^2.\]
Allgemeiner gilt
\[|v|^2 = \sum_{k=1}^n v_k^2.\]
Analog gibt es für Funktionen aus einem Hilbertraum einen Satz des
Pythagoras, der als
parsevalsche Gleichung bezeichnet wird. Es ist
\[\|f\|^2 = \sum_{k\in I} f_k^2\]
mit $f_k=\langle e_k,f\rangle$.

Bei der Multiplikation eines Vektors mit einer Matrix entsteht ein
neuer Vektor. Für Funktionen gibt es ein analoges Konzept. Wird ein
linearer Operator auf eine Funktion angewendet, so entsteht dabei
eine neue Funktion. Zu den linearen Operatoren gehören die
Integraltransformationen.

Ein Beispiel ist die Laplace-Transformation.
Die Laplace"=Transformation ist
\[L(f)(s) = \int_0^{\infty} f(x)e^{-sx}\,\mathrm dx.\]
Die Laplace-Transformation von $f(x)=1$ ist z.B.
\begin{gather*}
L(x\mapsto 1)(s) = \int_0^{\infty} e^{-sx}\,\mathrm dx
= -\frac{1}{s} \int_0^{-\infty} e^x\,\mathrm dx\\
= -\frac{1}{s} \lim_{a\rightarrow\infty} (e^{-a}-1)
= \frac{1}{s}.
\end{gather*}
Aus $x\mapsto 1$ macht die Transformation also
$s\mapsto 1/s$.



Drehmatrizen drehen Vektoren, ändern ihren Betrag jedoch nicht.
Das Analogon zu Drehmatrizen sind orthogonale bzw. unitäre Operatoren.
Unitäre Operatoren erhalten die Norm. Ein Beispiel dafür ist die
Fourier"=Transformation $F$. Es ist $\|F(f)\| = \|f\|.$

Für unitäre Operatoren ist $T^{-1}=T^H$, die inverse Transformation
ist die adjungierte Transformation. Z.B. ist die Fouriertransformation
\[F(f)(\omega) = \frac{1}{\sqrt{2\pi}}
\int_{-\infty}^{\infty} f(x)e^{-ix\omega}\,\mathrm dx.\]
Sei $z=re^{i\varphi}$ eine komplexe Zahl in Polarform.
Die konjugierte Zahl ist
$\overline z = re^{-i\varphi}$. Auf diese Art können wir auch
die Transformation adjungieren. Man erhält
\[F^{-1}(f)(x) = \frac{1}{\sqrt{2\pi}}
\int_{-\infty}^{\infty} f(\omega)e^{ix\omega}\,\mathrm d\omega.\]

\subsection{Schwingungen}

Eine harmonische Schwingung hat die Gleichung
\[u=\hat u\sin(\omega t+\varphi_0).\]
Für den Effektivwert ergibt sich $U_\mathrm{eff} = \|u\|$.
Mit dieser Interpretation des Effektivwertes hat man die Möglichkeit
geometrische Mittel zu benutzen. Z.B. kann man nun die
Dreiecksungleichung
\[\|u_1+u_2\| \le \|u_1\|+\|u_2\|.\]
verwenden. Die Summe zweier Schwingungen wird immer einen kleineren
Effektivwert haben, als die Summe der Effektivwerte der Schwingungen.
Dabei scheint es egal zu sein, ob diese Schwingungen harmonisch sind
oder nicht.

Weiterhin erhält man mit der ersten binomischen Formel
\[\|u_1+u_2\|^2 = \|u_1\|^2+2\langle u_1,u_2\rangle+\|u_2\|^2.\]
Wir definieren die Mischspannung $u_m := u+\overline u$,
wobei $\overline u$ der konstante Gleichanteil sein soll. Man sieht
leicht ein, dass $\langle u,\overline u\rangle=0$ ist.
Außerdem ist $\|\overline u\|=\overline u$. Man erhält also
\[\|u_m\|^2 = \|u\|^2+\overline u^2.\]
Wenn $u_1,u_2$ zwei harmonische Schwingungen mit gleicher
Kreisfrequenz sind, dann ist
\[\langle u_1,u_2\rangle = \|u_1\|\|u_2\|\cos\Delta\varphi.\]
Eine beliebige Schwingung kann nun als Fourierreihe von
Grundschwingungen dargestellt werden. Dann kann man den Effektivwert
auch mit der parsevalschen Gleichung berechnen.

Z.B. ist die Fourierreihe der Rechteckschwingung
\[u = \frac{4\hat u}{\pi} \sum_{k=1}^{\infty}
\frac{\sin((2k-1)\omega t)}{2k-1}.\]
Mit der parsevalschen Gleichung ist
\begin{gather*}
\|u\|^2 = \frac{16\hat u^2}{\pi^2} \sum_{k=1}^{\infty}
\left\|\frac{\sin((2k-1)\omega t)}{2k-1}\right\|^2\\
= \frac{16\hat u^2}{\pi^2} \sum_{k=1}^{\infty} \frac{1}{2(2k-1)^2}
= \frac{8\hat u^2}{\pi^2} \sum_{k=1}^{\infty} \frac{1}{(2k-1)^2}\\
= \frac{8\hat u^2}{\pi^2} \frac{\pi^2}{8} = \hat u^2.
\end{gather*}
Man erhält also $U_\mathrm{eff}=\hat u$.

Es wird nun klarer, warum bei einer harmonischen Schwingung
\[U_\mathrm{eff}=\frac{1}{\sqrt{2}}\hat u\]
ist. Es ist so, weil nicht die Schwingungen
$\sin(\omega t+\varphi_0)$ die normierten Basisvektoren darstellen,
sondern die Schwingungen $\sqrt{2}\sin(\omega t+\varphi_0)$. Das
sieht natürlich etwas komisch aus. Verwendet man aber die komplexe
Darstellung einer Schwingung, so hat man
$\exp(j\omega t+j\varphi_0)$ als normierte Basisvektoren.
Mit diesen ist
\[\underline u = \|\underline u\|e^{j\omega t+j\varphi_0}.\]
Man beachte aber, dass dies kein Amplitudenzeiger sondern ein
Effektivwertzeiger ist.

\subsection{Basen}

Bisher waren die Basen immer abzählbar. Was ist, wenn die Basis nicht
abzählbar ist? Die uneigentliche Linearkombination ist dann keine
Summe, sondern ein Integral.

Für die Formulierung benötigt man ein Hilfsmittel, das sich
Delta"=Distribution nennt. Die Delta"=Distribution $\delta(x)$ kann man
sich als eine Funktion vorstellen, die für $x\ne 0$ den Wert null
und bei $x=0$ den Wert unendlich hat. Der Flächeninhalt der Fläche
unter der Kurve (d.h. zwischen Graph und $x$-Achse) soll aber
trotzdem gleich eins sein.

Für die Delta"=Distribution gilt
\[f(a) = \int_{-\infty}^{\infty} f(x)\,\delta(x-a)\,\mathrm dx.\]
Man definiert
\[\delta_a(x) := \delta(x-a).\]
Das Skalarprodukt definiert man mit
\[\langle f,g\rangle
= \int_{-\infty}^{\infty} \overline{f(x)}g(x)\,\mathrm dx.\]
Man kann nun überprüfen, dass
$\langle\delta_a,\delta_b\rangle=\delta(a-b)$ ist. Das ist analog
zu $\langle e_i,e_j\rangle = \delta_{ij}$. Die Menge der
$\delta_a$ kann man als Orthonormalbasis interpretieren.
Man schreibt nun kürzer $f(a) = \langle\delta_a,f\rangle$.

Damit kann man eine uneigentliche Linearkombination formulieren.
Die Funktion $f$ zerlegt man zu
\[f = \int_{-\infty}^{\infty} f(a)\delta_a\,\mathrm da
= \int_{-\infty}^{\infty} \langle\delta_a,f\rangle\delta_a\,\mathrm da.\]
Wir können nun $f=|f\rangle$ aus dem Integral ausklammern und setzen\\
$\delta_a = |\delta_a\rangle$.
Übrig bleibt der Identitätsoperator
\[\mathrm I
= \int_{-\infty}^{\infty} |\delta_a\rangle\langle\delta_a|\,\mathrm da.\]
Dieser Operator verändert eine Funktion nicht. Es ist $\mathrm If=f$.

Um Schreibaufwand zu sparen, wollen wir das Intervall
$(-\infty,\infty)$ mit~$\Omega$ bezeichnen. Das Skalarprodukt
kann man auch auf folgende Art ausrechnen:
\begin{gather*}
\langle f,g\rangle = \langle f|\int_{\Omega} g(a)|\delta_a\rangle\,\mathrm da
= \int_{\Omega} g(a)\langle f,\delta_a\rangle\,\mathrm da\\
= \int_{\Omega} g(a)\overline{\langle\delta_a,f\rangle}\,\mathrm da
= \int_{\Omega} g(a)\overline{f(a)}\,\mathrm da.
\end{gather*}
Fast die gleiche Rechnung macht man, wenn man bei
$\langle f,g\rangle=\langle f|I|g\rangle$ für den
Identitätsoperator einsetzt.

Da die inverse Fourier-Transformation unitär ist, kann man sie auf
die Basisvektoren $\delta_a$ anwenden, um neue Basisvektoren zu
erhalten. Es ergibt sich
\[e_a(x) = F^{-1}(\delta_a)(x) = \frac{1}{\sqrt{2\pi}}e^{iax}.\]
Tatsächlich ist nun
\[\int_{-\infty}^{\infty} e^{-iax}e^{ibx}\mathrm dx = 2\pi\delta(a-b).\]
Man hat also $\langle e_a,e_b\rangle = \delta(a-b)$, d.h. die
Funktionen $e_a$ sind orthonormal.

Bei diagonalisierbaren Matrizen ist es möglich eine Eigenzerlegung
der Matrix vorzunehmen. Im Folgenden wird erläutert, wie das beim
Ableitungsoperator gemacht wird.

Der Differentialoperator $D(f)=x\mapsto f'(x)$ hat die
Eigenwerte $\lambda$. Zum Eigenwert $\lambda$ gehört der
Eigenvektor $e^{\lambda x}$. Der Operator $D$ muss nun zerlegt
werden zu
\[D=T\circ S\circ T^{-1}.\]
Wenn man einen Basisvektor einer Orthonormalbasis mit einer
Diagonalmatrix multipliziert, so wird der Basisvektor mit der
entsprechenden Komponente der Matrix multipliziert. Es ergibt sich
\[\mathrm{diag}(a_{11},\ldots,a_{nn})e_k = a_{kk}e_k.\]
Dieses Prinzip übertragen wir analog auf den diagonalen
Operator $S$. Man hat dann $S(\delta_r) = \lambda_r \delta_r$.
Wir ordnen die Eigenwerte nun in der Reihenfolge an, bei der
$\lambda_r=-r$ ist. Wenn man $S$ nun auf eine Funktion anwendet,
so erhält man
\[S(f) = x\mapsto (-x)f(x).\]
Für die Transformation $T$ muss dann
$T\delta_r = x\mapsto e^{-rx}$ sein.
Man rechnet nun
\begin{gather*}
Tf = T\int_{-\infty}^{\infty} f(a)\delta_a\,\mathrm da
= \int_{-\infty}^{\infty} f(a)T\delta_a\,\mathrm da\\
= \int_{-\infty}^{\infty} f(a)(x\mapsto e^{-ax})\,\mathrm da\\
= \int_{-\infty}^{\infty} f(a)\int_{-\infty}^{\infty} e^{-ab}\delta_b
\,\mathrm db\,\mathrm da\\
= \int_{-\infty}^{\infty}\int_{-\infty}^{\infty} f(a)e^{-ab}
\,\mathrm da\,\delta_b\,\mathrm db\\
= x\mapsto \int_{-\infty}^{\infty} f(a)e^{-ax}\,\mathrm da.
\end{gather*}
Den Definitionsbereich von $f$ schränkt man nun zu $x\ge 0$ ein.
Für $x<0$ kann man ja einfach $f(x)=0$ definieren. Unter dieser
Voraussetzung erhält man $T=L$ wobei mit $L$ die
Laplace-Transformation gemeint ist.

Z.B. soll $F(s)=1/s$ abgeleitet werden. Man erhält
\begin{gather*}
D\left\{\frac{1}{s}\right\}
= LSL^{-1}\left\{\frac{1}{s}\right\} = LS\{1\}\\
= L\{-x\} = -L\{x\} = -\frac{1}{s^2}.
\end{gather*}
Wenn man die Eigenwerte stattdessen in der Reihenfolge anordnet,
dass $\lambda_r=r$ ist, ergibt sich
$T\delta_r=x\mapsto e^{rx}$. In diesem Fall ist $T=L^{-1}$.
Sei $R(f)=x\mapsto xf(x)$. Die Zerlegung lautet dann
$D=L^{-1}RL$. Nun ist ja
\[D^2 = DD = L^{-1}RLL^{-1}RL = L^{-1}R^2L.\]
Analog zu Matrizen müsste das auch für beliebige Potenzen möglich
sein. Man rechnet nun
\begin{gather*}
D^{-a}\{f(x)\} = L^{-1}R^{-a}L\{f(x)\}\\
= L^{-1}\{s^{-a}L\{f(x)\}\}.
\end{gather*}
Nach den Rechenregeln der Laplace-Transformation ist aber
\[s^{-a} = \frac{1}{\Gamma(a)}L\{x^{a-1}\}.\]
Mit der Rechenregel $L(f)L(g) = L(f*g)$ ergibt sich
\[D^{-a}\{f(x)\}
= \frac{1}{\Gamma(a)}L^{-1}
\{L\{\int_0^x (x-t)^{a-1}f(t)\,\mathrm dt\}\}.\]
Somit ist
\[D^{-a}\{f(x)\}
= \frac{1}{\Gamma(a)}
\int_0^x (x-t)^{a-1}f(t)\,\mathrm dt.\]
Dieser Operator heißt Riemann-Liouville-Integral und stellt
einen fraktionalen Integraloperator dar.

\subsection{Integraltransformationen}

Sei $T$ eine Integraltransformation. Mit $T(f)$ oder auch
$Tf$ ist die Applikation von $T$ auf die Funktion $f$ gemeint.
Eine Integraltransformation hat die Form
\[Tf = x\mapsto \int_{\Omega} K(a,x)f(a)\,\mathrm da.\]
%
Die Funktion $K(a,x)$ wird Integralkern genannt. Die Transformation
kann man als uneigentliche Linearkombination darstellen.
Man erhält dann
\[Tf = \int_{D}\int_{\Omega} K(a,b)
|\delta_b\rangle f(a)\,\mathrm da\mathrm db.\]
%
Ersetzt man noch $f(a)$ gegen $\langle\delta_a,f\rangle$,
so erhält man
\[Tf = \int_{D}\int_{\Omega} K(a,b)
|\delta_b\rangle\langle\delta_a,f\rangle\,\mathrm da\mathrm db.\]
%
Man klammert jetzt $f=|f\rangle$ aus. Der Integraloperator bleibt
übrig. Es ergibt sich
\[T = \int_{D}\int_{\Omega} K(a,b)
|\delta_b\rangle\langle\delta_a|\,\mathrm da\mathrm db.\]
%
Man kann die Transformation auf die Delta"=Distribution anwenden.
Es ist
\[T\delta_r = x\mapsto K(r,x).\]

\subsection{Das Stieltjes-Integral}

Sei $g$ eine Funktion mit stetiger Ableitung. Dann ist
\[\int_a^b f(x)\,\mathrm dg(x) = \int_a^b f(x)g'(x)\,\mathrm dx.\]
Die Funktion $g$ wird Integrator genannt. Beim Stieltjes"=Integral
sind aber auch Integratoren möglich, die z.B. Treppenfunktionen sind.
Sei $g$ eine Treppenfunktion, die im Intervall $(a,b)$ an den
Stellen $x_k$ Sprünge der Höhe $c_k$ macht. Wenn $c_k$ negativ
ist, dann macht $g$ an der Stelle $x_k$ einen Sprung nach unten.
Das Stieltjes"=Integral ist
\[\int_a^b f(x)\,\mathrm dg(x) = \sum_{k=1}^n f(x_k)c_k.\]
Die verallgemeinerte Regel zur partiellen Integration ist
\[\int_a^b f(x)\,\mathrm dg(x) = [f(x)g(x)]_a^b - \int_a^b g(x)\,\mathrm df(x).\]
Das Stieltjes-Integral ist linear im Integranden und im Integrator. Es ist
\begin{gather*}
\mathrm d(g(x)+h(x)) = \mathrm dg(x)+\mathrm dh(x),\\
\mathrm d(cg(x)) = c\,\mathrm dg(x).
\end{gather*}
Damit kann man Treppenfunktionen mit stetig differenzierbaren
Funktionen überlagern, wobei verschiedene Regelfunktionen entstehen.
Wenn der Integrator Knickstellen hat, die Funktion $f$ aber stetig
differenzierbar ist, so kann man das Stieltjes"=Integral über
partielle Integration auf ein Riemann"=Integral zurückführen.

Sei $\Omega$ ein Intervall, das die Stelle null enthält, wobei
null nicht auf dem Rand von $\Omega$ liegen soll. Mit der
Heaviside"=Funktion $H(x)$ ist
\[\int_{\Omega} f(x)\,\mathrm dH(x) = f(0).\]
Man kann $\delta(x)\,\mathrm dx$ also als $\mathrm dH(x)$
interpretieren. Wenn die Heaviside"=Funktion eine Ableitung hätte,
dann wäre diese die Delta"=Distribution. Es ist nun ja
\[H(x) = \frac{1}{2}(\mathrm{sgn}\,x+1).\]
Damit ergibt sich $\mathrm dH(x) = (1/2)\mathrm d\mathrm{sgn}(x)$.
Außerdem beachtet man die Rechenregel
\[\mathrm{sgn}(ax) = \mathrm{sgn}(a)\,\mathrm{sgn}(x).\]
Man erhält
\[\mathrm dH(ax) = \mathrm{sgn}(a)\,\mathrm dH(x).\]
Es gilt auch die Rechenregel
\[|a| = \mathrm{sgn}(a)a = \frac{a}{\mathrm{sgn}(a)}.\]
Verwendet man nun noch $\mathrm d(ax) = a\,\mathrm dx$,
so erhält man
\[\delta(ax) = \frac{1}{|a|}\delta(x).\]

\subsection{Fraktionale Iterationen}

Sei $f$ eine Funktion, und man setze
$\mathrm{D}(f)=\mathrm{Bild}(f)$. Es lassen sich nun die
Iterationen rekursiv definieren. Sei
\begin{gather*}
f^0(x) := \mathrm{id}(x) = x,\\
f^n(x) := f^{n-1}(f(x)).
\end{gather*}
Die Iterationen sind Potenzen bezüglich des Verkettungsoperators.
Es ist $f^2=f\circ f$ usw. Man kann sich nun die Frage stellen,
ob sich so etwas wie $f^{1/2}$ bestimmen lässt. Allgemeiner $f^c$,
wobei $c$ eine komplexe Zahl ist.

Sei $C_f(g) := g\circ f$. Man rechnet nun
\begin{gather*}
C_f(rg) = (rg)\circ f = x\mapsto rg(f(x)) = r(g\circ f)
\end{gather*}
und
\begin{gather*}
C_f(g_1+g_2) = (g_1+g_2)\circ f\\
= x\mapsto g_1(f(x))+g_2(f(x))
= g_1\circ f+g_2\circ f
\\= C_f(g_1)+C_f(g_2).
\end{gather*}
Bei $C_f$ handelt es sich um einen linearen Operator, und
$C_f^n(g)$ ist $g\circ f^n$, da Verkettungen assoziativ sind.
Der Operator $C_f$ wird \textit{Kompositionsoperator} genannt.

Sei nun $\Psi$ ein Eigenvektor von $C_f$. Das zugehörige
Eigenwertproblem $C_f\Psi = \lambda\Psi$ wird
\textit{schrödersche Gleichung}
genannt und stellt eine Funktionalgleichung dar. Hat man $\Psi$
gefunden, so ergibt sich damit auch
\[f^r(x) = \Psi^{-1}(\lambda^r\Psi(x)).\]
Sei nun $u$ ein Fixpunkt von $f$ und sei $0<|f'(u)|<1$.
Unter bestimmten Umständen ist dann
\[\Psi(x) = \lim_{n\rightarrow\infty}\frac{f^n(x)-u}{f'(u)^n}\]
und $\lambda=f'(u)$. Die Eigenfunktion $\Psi(x)$ wird
dann \textit{Funktion von Koenigs} genannt.

Jede gewöhnliche Potenz kann als Iteration formuliert werden.
Sei z.B. $f(x)=ax$, dann ist $f^2(x)=a(a(x))=a^2 x$, da die
Multiplikation assoziativ ist. Zu jeder Funktion $f$ gehört nun
genau eine Zahl $a$, womit sich eine Bijektion ergibt. Die
linearen Funktionen und die reellen Zahlen sind also isomorph
bezüglich der {\glqq}Potenzierung{\grqq}.

Für eine reelle Zahl $r$ wäre dann aber
\[a^r x = f^r(x).\]
Diese Vorgehensweise hängt offensichtlich nur davon ab, dass
die Multiplikation assoziativ ist. Man kann auch $f(x)=Dx$ setzen,
wobei $D$ der Ableitungsoperator und $x(t)$ eine Funktion ist.
Somit können fraktionale Ableitungen als Spezialfall von
fraktionalen Iterationen betrachtet werden.

Wenn $u$ ein Fixpunkt von $f$ ist, dann ist $f(u)=u$ und
somit auch $f^n(u)=u$. Damit erscheint es vernünftig, auch
$f^r(u)=u$ für eine reelle Zahl $r$ zu verlangen.

In der Umgebung von $u$ sollte sich $f$ nun durch eine
affine Funktion $g(x)=b+ax$ approximieren lassen. Man hat
nun
\begin{gather*}
g^2(x) = b+ab+a^2x,\\
g^3(x) = b+ab+a^2b+a^3x,\\
g^4(x) = b+ab+a^2b+a^3b+a^4x
\end{gather*}
und daher
\begin{gather*}
g^n(x) = \bigg[\sum_{k=0}^{n-1}a^k\bigg]b+a^nx
= \frac{a^n-1}{a-1}b+a^nx.
\end{gather*}
Jetzt lässt sich wieder $g^r(x)$ für eine reelle Zahl $r$
berechnen. Für $r=-1$ ergibt sich die Umkehrfunktion von $g$,
womit die hier dargestellte Vorgehensweise eine gewisse Sinnhaftigkeit
erhält. Setzt man nun $g(x)=u+a(x-u)$ bzw. $b=u-au$, so ergibt
sich die äquivalente, aber wesentlich einfachere Formel
\[g^n(x) = u+a^n(x-u).\]
Allgemeiner kann man $f$ durch eine Taylorreihe approximieren.
Sei $a:=f'(u)$, $b:=f''(u)$. Man erhält weiter
\[g^n(x) = u+a^n(x-u)
+\frac{a^n-1}{a-1}a^{n-1}\frac{b}{2}(x-u)^2+\ldots\]
Die nächsten Summanden werden wesentlich komplizierter sein. Ohne
System wird man händisch noch ein bis zwei Summanden formulieren
können. Eine numerische Berechnung beliebiger Genauigkeit ist damit
nicht möglich.

Als nächstes soll eine weitere Methode vorgestellt werden. Die
Funktion $f$ soll sich nun als Potenzreihe mit
$f(x)=\sum_{k=0}^\infty a_kx^k$ darstellen lassen. Die Potenzreihe
ist aber als Skalarprodukt $\langle A_1,v(x)\rangle$ mit
$A_1=(a_0,a_1,\ldots)$ und $v(x)=(x^0,x^1,\ldots)$
interpretierbar. Die Idee ist nun, auch die Potenzen $f(x)^i$ damit
auszudrücken. Dann muss $A_0=(1,0,0,\ldots)$ sein. Sagen wir, die
$A_i$ sollen Spaltenvektoren sein, so lassen sie sich zur Matrix
$A$ zusammenfassen. Diese Matrix ist also von folgender Gestalt.
\[\begin{bmatrix}
1 & 0 & 0 & 0 & \ldots\\
a_0 & a_1 & a_2 & a_3 & \ldots\\
a_0^2 & 2a_0a_1 & a_1^2+2a_0a_1 & 2a_1a_2+2a_0a_3 &\ldots\\
a_0^3 & 3a_0^2a_1 & 3a_0a_1^2+3a_0^2a_2 &\ldots &\ldots\\
a_0^4 & 4a_0^3a_1 & 6a_0^2a_1^2+4a_0^3a_2 & \ldots & \ldots\\
\ldots &\ldots &\ldots &\ldots & \ldots
\end{bmatrix}\]
Allgemein ist
\[A_{jk} = \frac{1}{k!}[D^k f(x)^j]_{x=0}.\]
Umgekehrt hat man mit dieser Matrix
\[f(x)^j = \sum_{k=0}^\infty A_{kj} x^k.\]
Mit dieser Matrix ergibt sich nun
\[Av(x) = A[1,x,x^2,\ldots] = [1,f(x),f(x)^2,\ldots].\]
Also kurz $Av(x)=v(f(x))$. Damit ist aber
\[A^n v(x) = v(f^n(x)).\]
Diese Gleichung stellt eine Beziehung zwischen der Matrixpotenz und
der Iteration von $f$ dar. Die Matrix $A(f)$ wird
\textit{Carleman-Matrix}
der Funktion $f$ genannt. Anders ausgedrückt erhält man die Formeln
$A(f\circ g) = A(f)A(g)$ und $A(f^n)=A(f)^n$.
Wegen $f(x)=[A(f)]_1v(x)$ erhält man
\[f^n(x) = [A(f)^n]_1v(x).\]
Um eine Approximation zu erhalten, beschränkt man sich auf eine
quadratische Teilmatrix von $A$. Möchte man für $n$ eine reelle
Zahl einsetzen, so muss man nur die üblichen Methoden zur Berechnung
von Matrixpotenzen mit reellem Exponenten benutzen.

Unter Verwendung der binomischen Reihe und der binomischen Formel
ergibt sich
\begin{gather*}
x^m = (x-1+1)^m = \sum_{n=0}^\infty \binom{m}{n}(x-1)^n\\
= \sum_{m=0}^\infty \binom{m}{n}\sum_{n=0}^k\binom{n}{k}(-1)^{n-k}x^k.
\end{gather*}
Und für Iterationen erhält man analog
\begin{gather*}
f^m(x) = \sum_{m=0}^\infty \binom{m}{n}
\sum_{n=0}^k\binom{n}{k}(-1)^{n-k}f^k(x).
\end{gather*}
Um der Frage auf den Grund zu gehen, unter welchen Voraussetzungen
die Formeln konvergente Lösungen liefern und wann die Ergebnisse
übereinstimmen, benötigt man Kenntnisse in Funktionentheorie
und Funktionalanalysis.


\newpage
\section{Stochastik}
\subsection{Ereignisse}

Wenn man einen Würfel wirft, so erhält man eine der Zahlen aus
\[\{1,2,3,4,5,6\}\]
als Ergebnis. Man kann jetzt jeder Zahl die Wahrscheinlichkeit 1/6
zuordnen. Diese Vorgehensweise ist aber von einer gewissen
Mangelhaftigkeit betroffen, weil man so nicht alles elegant
formulieren kann, was denkbar ist. Es ist viel nützlicher, zuerst
sogenannte Ereignisse aus den Ergebnissen zu bilden. Zu den
Ergebnissen gehören die elementaren Ereignisse
\[\{\{1\},\{2\},\{3\},\{4\},\{5\},\{6\}\}.\]
Jedes elementare Ereignis $A$ hat nun die Wahrscheinlichkeit
$P(A)=1/6$. Wenn man sich jetzt nur dafür interessiert, ob eine
Eins oder eine Zwei gewürfelt wurde, so kann man die Vereinigungsmenge
dieser Ereignisse bilden. Man erhält das Ereignis
\[A = \{1\}\cup\{2\} = \{1,2\}.\]
Man sagt, ein Ereignis $A$ ist eingetreten, wenn eines der Elemente
aus $A$ das Ergebnis des Würfelwurfes ist. Für zwei disjunkte
Ereignisse gilt nun die Formel
\[P(A\cup B) = P(A)+P(B).\]
Damit erhält man in diesem Fall
\begin{gather*}
P(\{1,2\})=P(\{1\}\cup\{2\})=P(\{1\})+P(\{2\})\\
= \frac{1}{6}+\frac{1}{6} = \frac{1}{3}.
\end{gather*}
Ein Zufallsexperiment, bei dem jedem elementaren Ereignis die
gleiche Wahrscheinlichkeit zugeordnet wird, wird allgemein als
Laplace"=Experiment bezeichnet. Das klassische Würfeln und der
Münzwurf sind Laplace"=Experimente. Die Menge aller Ergebnisse
soll mit dem Buchstaben $\Omega$ bezeichnet werden.
Für Laplace"=Experimente gilt allgemein die Formel
\[P(A) = \frac{|A|}{|\Omega|}.\]
Dabei ist $|A|$ die Anzahl der Elemente von $A$ und $|\Omega|$
die Anzahl der Elemente von $\Omega$. In Wirklichkeit gibt es
eigentlich keine Laplace"=Experimente. In der Realität weichen
Wahrscheinlichkeitsverteilungen immer geringfügig von der
Gleichverteilung ab.

Wenn man zwei Würfel auf einmal wirft, so kann man die Ergebnismenge
aus den Tupeln $(e_1,e_2)$ bilden, wobei $e_1$ das Ergebnis des
ersten Würfels und $e_2$ das Ergebnis des zweiten Würfels ist.
Die Anzahl der Tupel ist
\[|\Omega\times\Omega| = |\Omega|\times|\Omega| = 6\times 6 = 36.\]
Jedes elementare Ereignis hat also die Wahrscheinlichkeit $1/36$.

Wenn sicher ist, dass das Ereignis $B$ eintritt, so schreibt
man $P(A|B)$ anstelle von $P(A)$. Man fragt also nach der
Wahrscheinlichkeit für das Eintreten von $A$, wenn man das Eintreten
von $B$ voraussetzen kann. Das wird als bedingte Wahrscheinlichkeit
bezeichnet. Es gilt die Formel
\[P(A|B) = \frac{P(A\cap B)}{P(B)}.\]
Zwei Ereignisse heißen stochastisch unabhängig, wenn
\[P(A\cap B) = P(A)P(B)\]
gilt. Das ist der Fall, wenn das Eintreten $A$ nichts mit dem
Eintreten von $B$ zu tun hat.

Wenn man das gleiche Zufallsexperiment mehrmals hintereinander
durch"-führt oder verschiedene Zufallsexperimente nacheinander
durchführt, dann spricht man von einem mehrstufigen
Zufallsexperiment. Für mehrstufige Zufallsexperimente kann man
Baumdiagramme als Hilfsmittel benutzen. Es gelten Pfadregeln.
Man kann z.B. nach der Wahrscheinlichkeit fragen, mit einer Münze
erst Zahl zu werfen und dann mit einem Würfel eine Sechs zu würfeln.
Die Wahrscheinlichkeit dafür ist $1/12$. Ein mehrstufiges
Zufallsexperiment ist also ein zusammengesetztes Zufallsexperiment,
welches aus mehreren kleinen Zufallsexperimenten besteht, die
nacheinander durchgeführt werden. Auch beim gleichzeitigen Werfen
von mehreren Würfeln kann man sich vorstellen, dass diese Würfel
in einer festen Reihenfolge nacheinander geworfen werden.

Ein Zufallsexperiment mit der Ergebnismenge $\{0,1\}$ wird als
Bernoulli-Versuch bezeichnet. Bei Bernoulli"=Versuchen schreibt man
kurz $p:=P(\{1\})$. Damit ergibt sich auch $P(\{0\})=1-p$.
Wenn man einen Bernoulli"=Versuch mehrmals hintereinander durchführt,
so spricht man von einer Bernoulli"=Kette. Eine Bernoulli"=Kette kann
als mehrstufiges Zufallsexperiment oder auch als ein sogenannter
stochastischer Prozess interpretiert werden.

Da die Ereignisse der Ereignismenge gewisse Axiome erfüllen,
bezeichnet man die Ereignismenge auch als Ereignisraum. Damit
sind nicht die Axiome von Kolmogorow gemeint, die gelten für
Wahrscheinlichkeiten und nicht für Ereignisse. Ereignisräume
sollen mit dem Buchstaben $\Sigma$ bezeichnet werden.

Ein Ereignisraum ist eine Teilmenge der Potenzmenge von $\Omega$.
Daher ist $2^{|\Omega|}$ die maximale Anzahl der Elemente in
$\Sigma$. Wenn man jedem Ergebnis ein dazugehöriges elementares
Ereignis gibt, dann wird diese Anzahl auch erreicht.

\subsection{Zufallsgrößen}

Es hat sich als fruchtbar erwiesen sogenannte Zufallsgrößen zu
betrachten. Das sind Funktionen von der Ergebnismenge in die reellen
Zahlen. Eine diskrete Teilmenge der reellen Zahlen ist natürlich auch
eine Ergebnismenge, aber man kommt auf sinnvolle Ideen, wenn man
stattdessen eine Funktion
\[X{:}\;\Omega\rightarrow\mathbf R\]
betrachtet. Anstelle von $y=f(x)$ schreibt man hier $x=X(\omega)$.
Wenn man schon eine diskrete Teilmenge der reellen Zahlen vorliegen
hat und falls man den Begriff der Zufallsgröße nicht braucht, so kann
man die identische Funktion $X=\mathrm{id}$ als Zufallsgröße wählen.

Jetzt stellt sich natürlich die Frage, wie man die
Wahrscheinlichkeit für das Eintreten von $x$ angibt.
Wahrscheinlichkeiten sind ja eigentlich nur für Ereignisse definiert,
und genau so muss man auch Vorgehen. Man wählt eine Bildmenge aus,
in diesem Fall $\{x\}$, und bestimmt die Urbildmenge.
Die Urbildmenge ist nun aber ein Ereignis, und von diesem kann man
die Wahrscheinlichkeit bestimmen. Man definiert also
\[P(\{x\}) := P(X^{-1}(\{x\})).\]
Allgemeiner definiert man
\[P(A\subseteq \mathrm{Bild}(X)) := P(X^{-1}(A)).\]
Nun kann man auch die sogenannte Verteilungsfunktion definieren.
Man definiert
\[F(x) := P(\{a{\in}\mathrm{Bild}(X)|\,a\le x\}).\]
Oft wird auch die Kurzschreibweise $F(x)=P(X\le x)$ verwendet.
Die Wahrscheinlichkeit einer Teilmenge der Bildmenge ist immer noch
ein sogenanntes Maß. Damit gilt für disjunkte Teilmengen $A,B$
der Bildmenge von $X$ auch jetzt noch die Formel
\[P(A\cup B) = P(A)+P(B).\]
Damit ergibt sich
\[F(x) = P(\{a|\,a\le x\})
= P(\bigcup_{a\le x}\{a\}) = \sum_{a\le x} P(\{a\}).\]


\subsection{Stochastische Prozesse}
Betrachten wir eine Menge von Zuständen $s_k$. Für jeden
Zustand gibt es nun Pfeile $a_{ki}:=(s_k,s_i)$ auf andere Zustände.
Ein Pfeil der auf den aktuellen Zustand zurückführt, ist auch
zugelassen. Jeder Pfeil wird nun mit einer Wahrscheinlichkeit
$p_{ki}$ behaftet, so dass die Summe aller Wahrscheinlichkeiten
bezüglich des Zustandes $s_k$ eins ist. Man beginnt nun bei einem
Anfangszustand. Durch den Zufall wird ein Pfeil ausgewählt, so dass
man bei einem neuen Zustand landet. Dann wird durch Zufall wieder
ein Pfeil ausgewählt und man landet bei einem neuen Zustand usw.
Ein solches Zufallsexperiment bezeichnet man als
\textit{Markov-Kette}. Auf ähnliche Art lassen sich auch
kompliziertere Systeme, wie endliche Automaten oder zelluläre
Automaten, mit Zufall behaften.

Eine solche Markov-Kette wird als Gedächtnislos bezeichnet, da
die Wahrscheinlichkeiten für den Übergang in einen anderen Zustand
nur vom aktuellen Zustand abhängig sind.

Ein Beispiel für eine Markov-Kette ist die Bernoulli-Kette.
Der Zustandsraum ist hier $Z=\mathbb{N}_0$. Die Zustände $s_k=k$
sind die Anzahl der Erfolge. Jeder Zustand $k$ hat zwei Pfeile,
einen auf sich selbst, und den anderen auf $k+1$. Die
Wahrscheinlichkeit für den Übergang zu $k+1$ ist für alle Zustände
gleich und soll mit $p$ bezeichnet werden. Die Wahrscheinlichkeit,
im aktuellen Zustand zu verbleiben, ist damit für alle
Zustände $p-1$.

Eine einfache Erweiterung wäre nun z.B. die ganzen Zahlen als
Zustände und Hinzufügung eines Abstieges von $k$ zu $k-1$.
Man erhält die sogenannte Irrfahrt auf den ganzen Zahlen.


\newpage
\section{Funktionentheorie}
\subsection{Holomorphie}

Ein Kriterium für die komplexe Differenzierbarkeit sind die
Cauchy"=Riemannschen Differentialgleichungen. Sei $x=\mathrm{Re}(z)$
und $y=\mathrm{Im}(z)$. Die komplexe Funktion $f(z)$ lässt sich
in den Realteil $u$ und den Imaginärteil $v$ auftrennen. Es ist
dann
\[f(z) = f(x+iy) = u(x,y)+iv(x,y).\]
Somit lässt sich $f$ als Vektorfeld auffassen, welches jedem Tupel
$(x,y)$ ein Tupel $(u,v)$ zuordnet. Die Funktion $f$ ist
holomorph, wenn sie als Vektorfeld total differenzierbar ist und
wenn $u,v$ die Cauchy-Riemannschen Differen"-tialgleichungen
erfüllen. Die Differentialgleichungen sind $D_x u=D_y v$ und
$D_y u = -D_x v$.

Sei nun $x_1=x$ und $x_2=y$. Sei außerdem $u_1=u$
und $u_2=-v$. Die Differentialgleichungen lassen sich schreiben als
\begin{gather*}
D_1u_1+D_2u_2=0,\\
D_1u_2-D_2u_1=0.
\end{gather*}
Damit erhält man die Bedingungen $\langle\nabla,u\rangle=0$ und
$\nabla\wedge u=0$. Bei $u$ handelt es sich also um ein
quellenfreies und rotationsfreies Vektorfeld. Mit
\[\nabla u = \langle\nabla, u\rangle+\nabla\wedge u\]
lassen sich die beiden Bedingungen zur Gleichung $\nabla u=0$
zusammenfassen.

\subsection{Integralsatz von Cauchy}

Man stelle sich eine komplexwertige Funktion $f(t)$ vor.
Wie berechnet man das bestimmte Integral einer solchen Funktion?
Nun ja, eine komplexwertige Funktion lässt sich doch als vektorwertige
Funktion interpretieren. Die Funktion $f(t)$ wird in Realteil und
Imaginärteil aufgetrennt, sie wird also in der Form
$f(t)=u(t)+iv(t)$ dargestellt.
Da das Integral ein linearer Operator ist,
lässt es sich vor die Bildung der Linearkombination ziehen.
D.h. man rechnet
\begin{gather*}
\int_{t_1}^{t_2} f(t)\,\mathrm dt
= \int_{t_1}^{t_2} u(t)+iv(t)\,\mathrm dt\\
= \int_{t_1}^{t_2} u(t)\,\mathrm dt + i\int_{t_1}^{t_2}
v(t)\,\mathrm dt.
\end{gather*}
Jetzt stellt sich natürlich sofort die Frage, ob es auch möglich ist,
für eine komplexe Funktion ein bestimmtes Integral zu berechnen.
Das ist etwas komplizierter, man muss ein sogenanntes komplexes
Kurvenintegral definieren. Dafür bekommt man jedoch einfache analoge
Rechenregeln, ein sehr nützliches Werkzeug und höchst interessante
Resultate. Die analogen Rechenregeln bekommt man deshalb,
weil das Wegintegral unter einfachen Voraussetzungen
wegunabhängig ist.

Sei nun $\gamma$ ein Weg. Dieser soll jedoch nicht durch eine
vektorwertige Funktion parametrisiert werden, sondern durch
eine komplexwertige. Das Integral wird analog zum Kurvenintegral
zweiter Art definiert durch
\[\int_\gamma f(z)\,\mathrm dz
:= \int_{t_1}^{t_2} f(\gamma(t))\gamma(t)\,\mathrm dt.\]
Der Unterschied ist, dass die Multiplikation beim Kurvenintegral
zweiter Art ein Skalarprodukt ist, es sich hier aber um die
Multiplikation komplexer Zahlen handelt.
Man kann auch immer $t_1=0$ und $t_2=1$ wählen.
Der Weg ist dann so parametrisiert, dass $\gamma(0)$
der Anfangspunkt ist und $\gamma(1)$ der Endpunkt. Das lässt
sich natürlich durch triviale Umparametrisierung immer erreichen.

Man rechnet nun
\[\mathrm dz = \mathrm d(x+iy) = \mathrm dx+i\mathrm dy.\]
Außerdem rechnet man
\begin{gather*}
f(z)\mathrm dz = (u+iv)(\mathrm dx+i\mathrm dy)\\
= u\mathrm dx-v\mathrm dy+i(u\mathrm dy+v\mathrm dx)\\
= u_1\mathrm dx_1+u_2\mathrm dx_2+i(u_1\mathrm dx_2-u_2\mathrm dx_1)\\
= (u_1,u_2)^\flat + i(-u_2,u_1)^\flat.
\end{gather*}
Man wendet jetzt den gaußschen Integralsatz in der Ebene rückwärts an
und erhält
\begin{gather*}
\int_{R(B)} f(z)\,\mathrm dz\\
= \int_B \nabla^\flat\wedge (u_1,u_2)^\flat
+ i\int_B \nabla^\flat\wedge (-u_2,u_1)^\flat
\end{gather*}
Umformen und ausnutzen der Quellenfreiheit
und Rotationsfreiheit bringt
\begin{gather*}
\nabla^\flat\wedge (u_1,u_2)^\flat
= (D_1u_2-D_2u_1)\,\mathrm dx_1\wedge\mathrm dx_2= 0,\\
\nabla^\flat\wedge (-u_2,u_1)^\flat = (D_1u_1+D_2u_2)\,
\mathrm dx_1\wedge\mathrm dx_2 = 0.
\end{gather*}
Damit erhält man schließlich
\[\oint_\gamma f(z)\,\mathrm dz = 0.\]
Dieses Resultat wird als Integralsatz von Cauchy bezeichnet.
Es ist wichtig, dass die Funktion $f$ auf einem einfach
zusammenhängenden Gebiet definiert ist, und dass die Kurve $\gamma$
innerhalb dieses Gebietes liegt. Wenn das nicht der Fall ist, dann
kann man den gaußschen Integralsatz nicht verwenden. In der Tat lassen
sich dann auch Gegenbeispiele finden.

Man denke sich jetzt einen \textit{nicht} geschlossenen Weg $\gamma$
mit Anfangspunkt $a$ und Endpunkt $b$. Außerdem denke man sich
einen konstanten Weg $\beta$, der den Weg $\gamma$
abschließt, d.h. die beiden Punkte verbindet. Den Weg $\gamma$
kann man nun variieren. Nach dem Integralsatz wird aber immer
\[\int_{\beta} f(z)\,\mathrm dz+\int_\gamma f(z)\,\mathrm dz
=K+\int_\gamma f(z)\,\mathrm dz=0\]
sein. Daher ist
\[\int_\gamma f(z)\,\mathrm dz=\mathrm{const}.\]
Das Wegintegral hängt nur von Anfangspunkt $a$ und Endpunkt $b$
ab, nicht jedoch von der Wahl des Weges. Es ist wegunabhängig. Daher
kann man auch schreiben
\[\int_a^b f(z)\,\mathrm dz:=\int_\gamma f(z)\,\mathrm dz,\]
wobei man sich einen Weg von $a$ nach $b$ aussuchen darf.

Die Wegunabhängigkeit von komplexen Wegintegralen für holomorphe
Funktionen scheint analog zur Wegunabhängigkeit von Wegintegralen
zweiter Art für Potentialfelder zu sein. Jedoch gibt es zwei wichtige
Unterschiede. Zum einen sind komplexe Wegintegrale vektorwertig, wenn
man die komplexen Zahlen als Vektoren interpretiert. Zum anderen
benötigt man als Voraussetzung zusätzlich zur Rotationsfreiheit
auch noch die Quellenfreiheit.

Es gibt also drei Arten von Wegintegralen. Eines mit skalarer
Multiplikation, eines mit Skalarprodukt und eines mit komplexer
Multiplikation.

Der Hauptsatz gilt auch im Komplexen, was äußerst
nützlich ist. Sei $F$ eine Stammfunktion von $f$, d.h. die
komplexe Ableitung von $F$ muss $f$ sein. Dann gilt
\[\int_a^b f(z)\,\mathrm dz = F(b)-F(a).\]
Außerdem ist eine Stammfunktion $F$ gegeben durch
\[F(z) = \int_a^z f(z)\,\mathrm dz.\]

\end{document}


