\documentclass[a4paper,12pt,fleqn,twoside]{article}
\usepackage[utf8]{inputenc}
\usepackage[T1]{fontenc}
\usepackage{lmodern}
\usepackage{ngerman}
\usepackage{amsmath}
\usepackage{amssymb}
\usepackage{amsthm}
\usepackage{color}
\definecolor{c1}{RGB}{00,40,80}
\usepackage[colorlinks=true,linkcolor=c1]{hyperref}
\usepackage{geometry}
\geometry{a4paper,left=35mm,right=20mm,top=28mm,bottom=46mm}
\setlength{\columnsep}{4mm}
\numberwithin{equation}{section}

\usepackage{titlesec}
\titleformat{\chapter}[block]
  {\normalfont\sffamily\huge\bfseries}{\thechapter}{1em}{\Huge}
\titleformat{\section}[block]
  {\normalfont\sffamily\Large\bfseries}{\thesection}{1em}{\Large}
\titleformat{\subsection}[block]
  {\normalfont\sffamily\large\bfseries}{\thesubsection}{1em}{\large}
\titleformat{\subsubsection}[block]
  {\normalfont\sffamily\large\bfseries}{\thesubsubsection}{1em}{\large}

\newcommand{\sgn}{\operatorname{sgn}}
\newcommand{\R}{\mathbb R}
\newcommand{\strong}[1]{{\sf\bfseries #1}}
\newcommand{\entspricht}{\;\;\hat =\;}

\theoremstyle{definition}
\newtheorem{Aufgabe}{\sffamily Aufgabe}[section]

\begin{document}
\thispagestyle{empty}

\noindent
{\huge\sf\bfseries Aufgaben}

\tableofcontents

\section{Analysis}
\subsection{Konvergenz}
\begin{Aufgabe}
Berechne
\[g = \lim_{x\to 0}\frac{\sum_{k=1}^n a_k x^k}{\sin(bx)}.
\qquad(\forall k\colon a_k\ne 0)\]
\end{Aufgabe}
\noindent
Lösung.
Wegen $x\ne 0$ kann der Bruch mit $\frac{bx}{bx}$ erweitert
werden. Damit ergibt sich
\[\frac{\sum_{k=1}^n a_k x^k}{\sin(bx)}
= \underbrace{\bigg(\frac{bx}{\sin(bx)}\bigg)}_{\to 1}
\underbrace{\bigg(\frac{a_1}{b}+\sum_{k=2}^n\frac{a_k}{b}x^{k-1}\bigg)}_{\to a_1/b}.
\]
Nach den Grenzwertsätzen ist der gesamte Ausdruck konvergent, wenn
die beiden Faktoren konvergent sind und $g$ ist das Produkt
der Grenzwerte der Faktoren. Somit ist $g=a_1/b$. $\Box$

Verwende alternativ die Regel von L'Hôpital.

\begin{Aufgabe}
Berechne
\[g = \lim_{x\to\frac{\pi}{2a}} \frac{1-\sin(ax)}{(\pi-2ax)^2}.
\qquad(a\ne 0)\]
\end{Aufgabe}
\noindent
Lösung.
Verwende die Substitution $x=\frac{\pi}{2a}-\frac{u}{a}$.
Nun ist
\[
\frac{1-\sin(ax)}{(\pi-2ax)^2}
= \frac{1-\sin(\frac{\pi}{2}-u)}{4u^2}
= \frac{1-\cos u}{4u^2}\\
= \frac{\frac{u^2}{2!}+\frac{u^4}{4!}+\ldots}{4u^2}
= \frac{1}{4} \Big(\frac{1}{2!}+\frac{u^2}{4!}+\ldots\Big).
\]
Wenn $x\to\pi/4$ geht, muss $u\to 0$ gehen.

Somit ist $g=1/8$. $\Box$

Verwende alternativ die Regel von L'Hôpital zweimal hintereinander.

\begin{Aufgabe}
Bestimme
\[g=\lim_{x\downarrow 0} x^x.\]
\end{Aufgabe}
\noindent
Lösung. Es ist $x^x=\exp(x\ln x)$. Wegen der Stetigkeit von $\exp$
gilt nun
\[\lim_{x\to 0}\exp(f(x)) = \exp(\lim_{x\to 0} f(x)).\]
Nun ist $x\ln x = (\ln x)/(1/x).$
Mit der Regel von L'Hôpital ergibt sich
\[\lim_{x\downarrow 0} \frac{\ln x}{\frac{1}{x}}
= \lim_{x\downarrow 0} \frac{\frac{1}{x}}{-\frac{1}{x^2}}
= \lim_{x\downarrow 0}\frac{x^2}{x}
= \lim_{x\downarrow 0} x = 0.\]
Somit ist $g=1$. $\Box$

\begin{Aufgabe}
Bestimme
\[g=\lim_{x\downarrow 0} x^{1/x}.\]
\end{Aufgabe}
\noindent
Lösung. Es ist $x^{1/x}=\exp(\frac{\ln x}{x})$.
Nun gilt
\[\lim_{x\downarrow 0}\frac{\ln x}{x}
\stackrel{\text{L'H}}= \lim_{x\downarrow 0}\frac{1}{x}
= -\infty = \lim_{x\downarrow -\infty} x.\]
Somit ist
\[g = \exp(\lim_{x\downarrow -\infty} x)
= \lim_{x\downarrow -\infty} \exp(x) = 0.\;\Box\]

\section{Kombinatorik}
\subsection{Rekursionsgleichungen}

\begin{Aufgabe}\label{qPotenzen}
Gegeben ist die Rekursionsgleichung $a_{n+1} = qa_n$
mit der Anfangsbedingung $a_0=A.$
Gesucht ist die explizite Form von $a_n$.
\end{Aufgabe}

\begin{Aufgabe}
Gegeben ist die Rekursionsgleichung
$a_{n+1} = qa_n+r$
mit der Anfangsbedingung
$a_0=A.$
Gesucht ist die explizite Form von $a_n$.
\end{Aufgabe}

\noindent
Bemerkung. Es gilt:
\[\sum_{k=0}^n q^{n-k} = \sum_{0\le k\le n} q^{n-k}
\quad\stackrel{k:=(n-k)}=\quad\sum_{0\le (n-k)\le n} q^{n-(n-k)}
= \sum_{0\le (n-k)\le n} q^k.
\]
Nun besteht aber $0\le n-k\le n$ aus den beiden Ungleichungen
\[0\le n-k\quad\text{und}\quad n-k\le n.\]
Multipliziert man beide Seiten einer Ungleichung mit $-1$, so dreht
sich das Relationszeichen um:
\[0\ge -(n-k)\quad\text{und}\quad -(n-k)\ge -n.\]
Somit ergibt sich:
\[0\ge k-n\quad\text{und}\quad k-n\ge -n.\]
Addiere jetzt $n$ auf beiden Seiten der jeweiligen Ungleichung:
\[n\ge k\quad\text{und}\quad k\ge 0.\]
Somit ergibt sich $0\le k\le n$ und daher
\[\sum_{k=0}^n q^{n-k} = \sum_{k=0}^n q^k.\]
Einfach ausgedrückt heißt das, dass die Reihenfolge egal ist:
\[\sum_{k=0}^3 q^{3-k} = q^3+q^2+q^1+q^0 = q^0+q^1+q^2+q^3 = \sum_{k=0}^3 q^k.\]
Voraussetzung ist, dass das Kommutativgesetz gilt. Bei unendlichen
Reihen darf man nur endliche Partialsummen umordnen, es sei denn
die Reihe ist absolut konvergent.

\strong{Ansatz.} Sei
\[s_b := \sum_{k=a}^{b-1} q^k.\]
Nun gilt:
\[qs_b = q\sum_{k=a}^{b-1} q^k = \sum_{k=a}^{b-1} q^{k+1}
\quad\stackrel{k:=k-1}=\quad\sum_{k=a+1}^b q^k.\]
Es ergibt sich:
\[qs_b-s_b = (q^{a+1}+q^{a+2}+\ldots+q^{b})-(q^a+q^{a+1}+\ldots+q^{b-1}) = q^b-q^a.\]
D.\,h. alle Summanden $q^{a+1}$ bis $q^{b-1}$ kommen sowohl im Minuend als auch im Subtrahend vor und
entfallen somit.

Mit $qs_b-s_b=(q-1)s_b$ ergibt sich nun
\[\sum_{k=a}^{b-1} q^k = \frac{q^b-q^a}{q-1}.\]
Bemerkung. Hinter diesem \emph{Trick} verbirgt sich ein mathematischer
Formalismus. Was eben beschrieben wurde, nennt sich
\emph{Teleskopsumme}. \emph{Teleskopieren} nennt man die Rechenregel:
\[\sum_{k=a}^{b-1} f_{k+1} - \sum_{k=a}^{b-1} f_k
= \sum_{k=a}^{b-1} (f_{k+1}-f_k) = f_b-f_a,\]
welche für eine beliebige Folge $f_k$ gilt. In diesem Fall ist
$f_k=q^k$. Man muss bestimmte Eigenschaften einer Partialsummen-Folge
ausnutzen, um sie in Teleskopform bringen zu können. Das ist aber nicht
immer möglich.

Hinter Teleskopsummen verbigt sich nun ein kleiner mathematischer
Formalimus. Zunächst definiere die \emph{Vorwärts-Differenz}:
\[\Delta f_k\equiv (\Delta f)_k := f_{k+1}-f_k.\]
Nun gilt:
\[\sum_{k=a}^{b-1} (\Delta f)_k = f_b-f_a.\]
In dieser Form ist die Teleskopsummen-Regel völlig analog zu
\[\int_a^b \frac{\mathrm df(x)}{\mathrm dx}\,\mathrm dx
= \int_a^b \mathrm df(x) = f(b)-f(a).\]
Es gibt weitere Rechenregeln. Man spricht von \emph{Differenzenrechnung}
(engl. \emph{finite calculus}). Dieser Kalkül ist unter anderem
im Buch »Concrete Mathematics« beschrieben.

\strong{Homogene Koordinaten.}
Ein alternatives Verfahren zur Lösung der Aufgabe zeigen.
Was im Gegensatz zu Aufgabe \ref{qPotenzen}
jetzt stört, ist der Summand $r$. Es gibt nun ein Verfahren, um
Additionen in Multiplikationen umzuwandeln, das allgemein für die
Addition von Vektoren funktioniert.

Zunächst führt man auf folgede Weise homogene Koordinaten ein:
\[ x\entspricht\begin{bmatrix}x\\ 1\end{bmatrix}.\]
Es ergibt sich nun
\[ qx\entspricht\begin{bmatrix}qx\\ 1\end{bmatrix}
= \begin{bmatrix}
q & 0\\
0 & 1
\end{bmatrix}
\begin{bmatrix}x\\ 1\end{bmatrix}
\quad\text{und}\quad
x+r\entspricht\begin{bmatrix}x+r\\ 1\end{bmatrix}
= \begin{bmatrix}
1 & r\\
0 & 1
\end{bmatrix}
\begin{bmatrix}x\\ 1\end{bmatrix}.\]
Beide Operationen zusammen:
\[\begin{bmatrix}qx+r\\ 1\end{bmatrix}
= \begin{bmatrix}
1 & r\\
0 & 1
\end{bmatrix}
\begin{bmatrix}
q & 0\\
0 & 1
\end{bmatrix}
\begin{bmatrix}x\\ 1\end{bmatrix}
= \begin{bmatrix}
q & r\\
0 & 1
\end{bmatrix}
\begin{bmatrix}x\\ 1\end{bmatrix}.\]
Die Aufgabe lässt sich nun in der Form
$\underline a_{n+1} = Q\underline a_n$
mit
\[ Q:=\begin{bmatrix}
q & r\\
0 & 1
\end{bmatrix},\quad
\underline a_n := \begin{bmatrix}a_n\\ 1\end{bmatrix}\]
formulieren, was aber Aufgabe~\ref{qPotenzen} entspricht. Die
Lösung ist demnach
$\underline a_n = Q^n\underline a_0.$
Jetzt muss man einen Weg finden, die Matrixpotenz $Q^n$ zu berechnen.
Dazu wird eine Diagonalzerlegung $Q = TDT^{-1}$ vorgenommen.
Bei
\[Q^n = QQQ\ldots Q = TDT^{-1} TDT^{-1} TDT^{-1} \ldots TDT^{-1}\]
können die Faktoren $T^{-1}T$ nämlich gekürzt werden. Man erhält
somit
\[Q^n = TD^n T^{-1}.\]
Zunächst bestimmt man die Eigenwerte von $Q$. Die Eigenwerte
sind die Lösungen der Gleichung
\[P(\lambda) = \det(Q-\lambda E)=0.\]
Man nennt $P(\lambda)$ das \emph{charakteristische Polynom}.

In diesem Fall ist
\[\begin{split}
P(\lambda) &= \det\left(\begin{bmatrix}
q & r\\
0 & 1
\end{bmatrix}-\lambda\begin{bmatrix}
1 & 0\\
0 & 1
\end{bmatrix}\right)
= \det\left(\begin{bmatrix}
q-\lambda & r\\
0 & 1-\lambda
\end{bmatrix}\right)\\
&= (q-\lambda)(1-\lambda)
= \lambda^2 -(q+1)\lambda+q.\end{split}
\]
Die Lösungen dieser quadratischen Gleichung sind
\[\lambda = \frac{1}{2}(q+1\pm\sqrt{(q+1)^2-4q})
= \frac{1}{2}(q+1\pm\sqrt{(q-1)^2}),\]
also $\lambda_1 = q$ und $\lambda_2=1.$

Nun ergeben sich aus dem Eigenwertproblem $Qv = \lambda v$ zwei linear
unabhängige Eigenvektoren, die den Eigenraum aufspannen. Diese beiden
Eigenvektoren sind die Spaltenvektoren der Transformationsmatrix $T$.

Aus dem Eigenwertproblem ergibt sich das Gleichungssystem
\[\left|
\begin{array}{rcl}
qx+ry &=& \lambda x\\
y &=& \lambda y
\end{array}\right|.\]
Die untere Gleichung lässt sich umformulieren:
\[y=\lambda y \iff y=0\lor \lambda=1.\]
Gehen wir nun von $y=0$ aus, so haben wir den Fall $\lambda_1=q$.
Für $x$ können wir uns etwas aussuchen und nehmen sinnvollerweise
$x=1$. Natürlich wäre $x=0$ noch schöner, aber das darf nicht sein,
weil beim Eigenwertproblem der Nullvektor verboten ist. Für
den zweiten Eigenvektor soll betrachten wir nun den Fall $\lambda_2=1$.
Hier ergibt sich die Gleichung $qx+ry=x$. Wählt man nun $y=1$, so
ergibt sich $x=r/(1-q)$. Somit ist
\[
Q = TDT^{-1}
= T\begin{bmatrix}
\lambda_1 & 0\\
0 & \lambda_2
\end{bmatrix}T^{-1}
= \begin{bmatrix}
1 & \frac{r}{1-q}\\
0 & 1
\end{bmatrix}\begin{bmatrix}
q & 0\\
0 & 1
\end{bmatrix}\begin{bmatrix}
1 & \frac{r}{1-q}\\
0 & 1
\end{bmatrix}^{-1}.
\]
Zur Matrix-Inversion einer $2{\times}2$-Matrix verwendet man nun noch
die Formel
\[
\begin{bmatrix}
a_{11} & a_{12}\\
a_{21} & a_{22}
\end{bmatrix}^{-1}
= \frac{1}{a_{11}a_{22}-a_{12}a_{21}}
\begin{bmatrix}
a_{22} & -a_{12}\\
-a_{21} & a_{11}
\end{bmatrix}.
\]
Es ergibt sich nun
\[Q^n = TD^nT^{-1}
= \begin{bmatrix}
1 & \frac{r}{1-q}\\
0 & 1
\end{bmatrix}\begin{bmatrix}
q^n & 0\\
0 & 1
\end{bmatrix}\begin{bmatrix}
1 & \frac{r}{q-1}\\
0 & 1
\end{bmatrix}
= \begin{bmatrix}
q^n & \frac{rq^n-r}{q-1}\\
0 & 1
\end{bmatrix}.
\]
Es ergibt sich
\[\underline a_n = Q^n\underline a_0
= \begin{bmatrix}
q^n & \frac{rq^n-r}{q-1}\\
0 & 1
\end{bmatrix}\begin{bmatrix} A\\ 1\end{bmatrix}
= \begin{bmatrix}
Aq^n + \frac{rq^n-r}{q-1}\\
1\end{bmatrix}.\]
Die Lösung ist somit
\[a_n = Aq^n + \frac{rq^n-r}{q-1}.\]
Jetzt muss man noch die pathologischen Fälle untersuchen und entsprechende
Fallunterscheidungen dazu vornehmen. In diesem Fall ist nur $q=1$
problematisch. $\Box$

Das wesentliche Vorgehen besteht hier also aus zwei Schritten:

1. Formulierung des Problems bezüglich homogenen Koordinaten.

2. Berechnung von Matrixpotenzen via Eigenzerlegung.

\strong{Erzeugende Funktionen.} Jetzt kommt noch ein Verfahren.
Für eine Folge $a_n$ definiert man die \emph{erzeugende Funktion}
\[G\{a_n\}(x) := \sum_{k=0}^\infty a_k x^k.\]
Man definiert außerdem den Translationsoperator
\[T^h\{a_n\} := a_{n+h}.\]
Der Operator $G$ ist linear:
\begin{align*}
G\{a_n+b_n\} &= G\{a_n\}+G\{b_n\},\\
G\{ra_n\} &= rG\{a_n\}.
\end{align*}
Es gilt außerdem
\[G\{T^h\{a_n\}\}(x) = G\{a_{n+h}\}(x)
= \sum_{k=0}^\infty a_{k+h} x^k.\]
Somit gilt
\[x^h G\{a_{n+h}\}(x) = \sum_{k=0}^\infty a_{k+h} x^{k+h}
= G\{a_n\}(x) - \sum_{k=0}^{h-1} a_k x^k.\]
Speziell gilt
\[xG\{a_{n+1}\}(x) = G\{a_n\}(x) - a_0.\]
Durch Polynomdivision findet man zunächst die grundlegende erzeugende Funktion
\[ G\{q^n\}(x) = \frac{1}{1-qx} = \sum_{k=0}^\infty q^k x^k\]
mit Spezialfall
\[ G\{1\}(x) = \frac{1}{1-x} = \sum_{k=0}^\infty x^k.\]
Jetzt betrachten wir die Rekursionsgleichung
\[ a_{n+1} = qa_n+r. \]
Auf beiden Seiten der Gleichung wendet man den Operator $G$ an:
\[ G\{a_{n+1}\}(x) = qG\{a_n\}(x) + rG\{1\}(x).\]
Auf beiden Seiten multipliziert man nun noch mit $x$ und erhält
\[ xG\{a_{n+1}\}(x) = qxG\{a_n\}(x) + rxG\{1\}(x).\]
Mit $y=G\{a_n\}(x)$ gilt nun
\[ y-a_0 = qxy+\frac{rx}{1-x}.\]
Umformen nach $y$ bringt
\[ y = \frac{a_0}{1-qx} + \frac{rx}{(1-x)(1-qx)}.\]
Jetzt appliziert man den Umkehroperator $G^{-1}$ auf beiden Seiten
der Gleichung. Es ergibt sich
\[ a_n = a_0 G^{-1}\bigg\{\frac{1}{1-qx}\bigg\}_n
+ rG^{-1}\bigg\{\frac{x}{(1-x)(1-qx)}\bigg\}_n.\]
Beachte nun die Regel
\[ G^{-1}\{xf(x)\}_n = T^{-1} G^{-1}\{f(x)\}_n = G^{-1}\{f(x)\}_{n-1}.\]
Für den übrigen Ausdruck muss eine Partialbruchzerlegung vorgenommen
werden. Der Ansatz ist
\[ \frac{1}{(1-x)(1-qx)} = \frac{A}{1-x} + \frac{B}{1-qx}.\]
Damit ist
\[ 1 = A(1-qx) + B(1-x) = A+B-Aqx-Bx = A+B-(Aq+B)x.\]
Koeffizientenvergleich von linker und rechter Seite bringt
$A+B=1$ und $Aq+B=0$. Beachte dabei $1=0x^0+1x^1$.

Die Lösungen dieses linearen Gleichungssystems
sind $A=1/(1-q)$ und $B=q/(q-1)$. Nun ergibt sich
\[ a_n = a_0q^n + rT^{-1} \underbrace{G^{-1}\bigg\{\frac{A}{1-x} + \frac{B}{1-qx}\bigg\}}_{A+Bq^n}.\]
Hierbei ist
\[ A+Bq^n = \frac{1}{1-q}+\frac{q}{q-1}q^n = \frac{q^{n+1}-1}{q-1}.\]
Insgesamt ergibt sich
\[ a_n = a_0q^n + r\frac{q^n-1}{q-1}.\;\Box\]

\vfill
Dieses Heft steht unter der Creative-Commons-Lizenz CC0.

\end{document}


