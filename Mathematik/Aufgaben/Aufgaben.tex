\documentclass[a4paper,10pt,fleqn,twocolumn,twoside]{article}
\usepackage[utf8]{inputenc}
\usepackage[T1]{fontenc}
\usepackage{lmodern}
\usepackage{ngerman}
\usepackage{amsmath}
\usepackage{amssymb}
\usepackage{amsthm}
\usepackage{color}
\definecolor{c1}{RGB}{00,40,80}
\usepackage[colorlinks=true,linkcolor=c1]{hyperref}
\usepackage{geometry}
\geometry{a4paper,left=25mm,right=10mm,top=24mm,bottom=32mm}
\setlength{\columnsep}{4mm}
\numberwithin{equation}{section}

\usepackage{titlesec}
\titleformat{\chapter}[block]
  {\normalfont\sffamily\huge\bfseries}{\thechapter}{1em}{\Huge}
\titleformat{\section}[block]
  {\normalfont\sffamily\Large\bfseries}{\thesection}{1em}{\Large}
\titleformat{\subsection}[block]
  {\normalfont\sffamily\large\bfseries}{\thesubsection}{1em}{\large}
\titleformat{\subsubsection}[block]
  {\normalfont\sffamily\large\bfseries}{\thesubsubsection}{1em}{\large}

\newcommand{\sgn}{\operatorname{sgn}}
\newcommand{\R}{\mathbb R}
\newcommand{\strong}[1]{\textbf{#1}}

\theoremstyle{definition}
\newtheorem{Aufgabe}{\sffamily Aufgabe}[section]

\begin{document}
\thispagestyle{empty}
\section*{Aufgaben}

Lizenz: Creative Commons CC0

\tableofcontents

\section{Analysis}
\subsection{Konvergenz}
\begin{Aufgabe}
Berechne
\[g = \lim_{x\to 0}\frac{\sum_{k=1}^n a_k x^k}{\sin(bx)}.
\qquad(\forall k\colon a_k\ne 0)\]
\end{Aufgabe}
\noindent
Lösung:
Wegen $x\ne 0$ kann der Bruch mit $\frac{bx}{bx}$ erweitert
werden. Damit ergibt sich
\[\frac{\sum_{k=1}^n a_k x^k}{\sin(bx)}
= \underbrace{\bigg(\frac{bx}{\sin(bx)}\bigg)}_{\to 1}
\underbrace{\bigg(\frac{a_1}{b}+\sum_{k=2}^n\frac{a_k}{b}x^{k-1}\bigg)}_{\to a_1/b}.
\]
Nach den Grenzwertsätzen ist der gesamte Ausdruck konvergent, wenn
die beiden Faktoren konvergent sind und $g$ ist das Produkt
der Grenzwerte der Faktoren. Somit ist $g=a_1/b$. $\Box$

Verwende alternativ die Regel von L'Hôpital.

\begin{Aufgabe}
Berechne
\[g = \lim_{x\to\frac{\pi}{2a}} \frac{1-\sin(ax)}{(\pi-2ax)^2}.
\qquad(a\ne 0)\]
\end{Aufgabe}
\noindent
Lösung:
Verwende die Substitution $x=\frac{\pi}{2a}-\frac{u}{a}$.
Nun ist
\[\begin{split}
&\frac{1-\sin(ax)}{(\pi-2ax)^2}
= \frac{1-\sin(\frac{\pi}{2}-u)}{4u^2}
= \frac{1-\cos u}{4u^2}\\
& = \frac{\frac{u^2}{2!}+\frac{u^4}{4!}+\ldots}{4u^2}
= \frac{1}{4} \Big(\frac{1}{2!}+\frac{u^2}{4!}+\ldots\Big).
\end{split}\]
Wenn $x\to\pi/4$ geht, muss $u\to 0$ gehen.

Somit ist $g=1/8$. $\Box$

Verwende alternativ die Regel von L'Hôpital zweimal hintereinander.

\begin{Aufgabe}
Bestimme
\[g=\lim_{x\downarrow 0} x^x.\]
\end{Aufgabe}
\noindent
Lösung: Es ist $x^x=\exp(x\ln x)$. Wegen der Stetigkeit von $\exp$
gilt nun
\[\lim_{x\to 0}\exp(f(x)) = \exp(\lim_{x\to 0} f(x)).\]
Nun ist
\[x\ln x = \frac{\ln x}{\frac{1}{x}}.\]
Mit der Regel von L'Hôpital ergibt sich
\[\lim_{x\downarrow 0} \frac{\ln x}{\frac{1}{x}}
= \lim_{x\downarrow 0} \frac{\frac{1}{x}}{-\frac{1}{x^2}}
= \lim_{x\downarrow 0}\frac{x^2}{x}
= \lim_{x\downarrow 0} x = 0.\]
Somit ist $g=1$. $\Box$

\begin{Aufgabe}
Bestimme
\[g=\lim_{x\downarrow 0} x^{1/x}.\]
\end{Aufgabe}
\noindent
Lösung: Es ist $x^{1/x}=\exp(\frac{\ln x}{x})$.
Nun gilt
\[\lim_{x\downarrow 0}\frac{\ln x}{x}
\stackrel{\text{L'H}}= \lim_{x\downarrow 0}\frac{1}{x}
= -\infty = \lim_{x\downarrow -\infty} x.\]
Somit ist
\[g = \exp(\lim_{x\downarrow -\infty} x)
= \lim_{x\downarrow -\infty} \exp(x) = 0.\;\Box\]

\end{document}


