
\chapter{Grundlagen}
\section{Aussagenlogik}\index{Aussagenlogik}

\begin{Satz}[mp: Modus ponens]
Es gilt $(A\Rightarrow B)\land A \implies B$.
\end{Satz}
\begin{Beweis}[Beweis 1 (Natürliches Schließen)]\\
Zu $\{A\Rightarrow B, A\}\vdash B$. Trivial, da eine Inferenzregel des Kalküls. Schematisch:
\[\infer{B}{A\Rightarrow B & A}\]
Programmterm:
\[(A\to B)\times A\to B,\; (f, a)\mapsto f(a).\,\qedsymbol\]
\end{Beweis}

\begin{Beweis}[Beweis 2 (LEM, boolesche Algebra)]
Man darf rechnen
\begin{align*}
(A\Rightarrow B)\land A \Rightarrow B &\equiv
\neg ((\neg A\lor B)\land A) \lor B
\equiv\neg (\neg A\lor B) \lor \neg A \lor B\\
&\equiv \neg\varphi\lor\varphi\equiv 1,
\end{align*}
wobei $\varphi :\equiv \neg A\lor B$.\;\qedsymbol
\end{Beweis}

\begin{Satz}[bool-cl: Kommutativgesetze]%
\label{bool-cl}\index{Kommutativgesetz!boolesche Algebra}
Es gilt
\begin{gather}
A\land B \iff B\land A,\\
A\lor B \iff B\lor A.
\end{gather}
\end{Satz}
\begin{Beweis}[Beweis (Natürliches Schließen)]\\
Zu $A\land B\vdash B\land A$. Schematisch:
\[\infer{B\land A}{
   \infer{A}{A\land B}
   & \infer{B}{A\land B}
}\]
Programmterm:
\[A\times B\to B\times A,\; (a,b)\mapsto (b,a).\]
Zu $A\lor B\vdash B\lor A$. Schematisch:
\[\infer[\infernote{1}]{B\lor A}{
  A\lor B
  & \infer{B\lor A}{\infer[\infernote{1}]{A}{}}
  & \infer{B\lor A}{\infer[\infernote{1}]{B}{}}
}\]
Programmterm:
\[A+B\to B+A,\; s\mapsto\match s\begin{cases}
\inl(a)\mapsto \inr(a),\\
\inr(b)\mapsto \inl(b).
\end{cases}\]
Vertauschen von $A,B$ erbringt jeweils die umgekehrte Implikation.\;\qedsymbol
\end{Beweis}

\begin{Satz}[bool-dl: Distributivgesetze]%
\label{bool-dl}\index{Distributivgesetz!boolesche Algebra}
Es gilt:
\begin{align}
A\land (B\lor C) &\iff A\land B\lor A\land C,\\
A\lor (B\land C) &\iff (A\lor B)\land (A\lor C).
\end{align}
\end{Satz}
\strong{Beweis (Natürliches Schließen).}\\
Zu $A\land (B\lor C) \vdash A\land B\lor A\land C$.
Programmterm:
\[A\times (B+C) \to A\times B + A\times C,\; (a, s)\mapsto \match s \begin{cases}
\operatorname{inl}(b)\mapsto \operatorname{inl}((a,b)),\\
\operatorname{inr}(c)\mapsto \operatorname{inr}((a,c)).
\end{cases}\]
Zu $A\land B\lor A\land C\vdash A\land (B\lor C)$.
Programmterm:
\[A\times B + A\times C\to A\times (B + C),\;
s\mapsto\match s\begin{cases}
\operatorname{inl}((a,b))\mapsto (a, \operatorname{inl}(b)),\\
\operatorname{inr}((a,c))\mapsto (a, \operatorname{inr}(c)).
\end{cases}
\]
Zu $A\lor (B\land C) \vdash (A\lor B)\land (A\lor C)$. Programmterm:
\[A+B\times C\to (A+B)\times (A+C),\;
s\mapsto\match s\begin{cases}
\inl(a)\mapsto (\inl(a),\inl(a)),\\
\inr((b,c))\mapsto (\inr(b),\inr(c)).
\end{cases} 
\]
Zu $(A\lor B)\land (A\lor C)\vdash A\lor (B\land C)$. Programmterm:
\[
(A+B)\times (A+C)\to A + B\times C,\;
t\mapsto\match t\begin{cases}
(\inl(a), s) \mapsto\inl(a),\\
(\inr(b), \inl(a)) \mapsto\inl(a),\\
(\inr(b), \inr(c)) \mapsto\inr((b, c)).
\end{cases}
\]
Sämtliche Teilaussagen sind bewiesen.\;\qedsymbol

\begin{Axiom}[PE: Principle of explosion]\label{PE}
Eine falsche Aussage impliziert jede beliebige Aussage, kurz
\[\vdash (0\implies A).\]
\end{Axiom}
Bemerkung: Dieses Prinzip erlaubt Programmterme mit leerem
Pattern matching, so dass ein Zeuge für $0\to A$ konstruiert
werden kann.

\begin{Axiom}[LEM: Satz vom ausgeschlossenen Dritten]\label{LEM}\newlinefirst
Entweder gilt eine Aussage, oder ihre Negation gilt, kurz
\[\vdash A\lor\neg A.\]
\end{Axiom}
Bemerkung: Zur Schaffung von Klarheit sollte ein Beweis die Markierung
LEM bekommen, wenn transitive Abhängigkeit zu diesem Axiom besteht.
Verzichtet keiner der Beweise eines Satzes auf LEM, sollte der Satz
ebenfalls mit LEM markiert werden.

\begin{Axiom}[Beseitigung der Doppelnegation]\label{DNE}\newlinefirst
Die Doppelnegation einer Aussage $A$ impliziert die Aussage $A$, kurz
\[\vdash (\neg\neg A\implies A).\]
\end{Axiom}

\begin{Satz} Das Axiom \ref{LEM} (LEM) zieht \ref{DNE} nach sich.
\end{Satz}
\begin{Beweis}[Beweis (Natürliches Schließen)]
Zu $\neg A \lor A\vdash (\neg\neg A \Rightarrow A)$.
Gemäß Axiom \ref{PE} (PE) existiert ein Zeuge $\operatorname{ex}(A)$
für $0\to A$. Damit lässt sich der Programmterm
\[(A\to 0) + A \to (((A\to 0)\to 0)\to A),\;
s\mapsto\match s\begin{cases}
\inl f\mapsto g\mapsto \operatorname{ex}(A)(g(f)),\\
\inr a\mapsto g\mapsto a.
\end{cases}\]
konstruieren.\;\qedsymbol
\end{Beweis}

\section{Prädikatenlogik}\index{Praedikatenlogik@Prädikatenlogik}

\begin{Definition}[bounded: beschränkte Quantifizierung]%
\label{def:bounded}
\begin{align}
(\forall x{\in}M\colon P(x)) &\defiff \forall x\colon (x\in M\implies P(x)),\\
(\exists x{\in}M\colon P(x)) &\defiff \exists x\colon (x\in M\land P(x)).
\end{align}
\end{Definition}

\begin{Satz}[general-dl: allgemeine Distributivgesetze]%
\label{general-dl}
Es gilt:
\begin{align}
A\land (\exists x\colon P(x)) &\iff \exists x\colon (A\land P(x)),\\
A\lor (\forall x\colon P(x)) &\iff \forall x\colon (A\lor P(x)).
\end{align}
\end{Satz}
\begin{Beweis}[Beweis (Natürliches Schließen)]
Schematisch. Zur ersten Äquivalenz:
\[\begin{array}{c@{\qquad}c}
\infer[\infernote{1}]{\exists x\colon (A\land P(x))}{
  \exists x\colon P(x)
  & \infer{\exists x\colon (A\land P(x))}{
      \infer{A\land P(a)}{A & \infer[\infernote{1}]{P(a)}{}}}
}
&
\infer[\infernote{1}]{A\land\exists x\colon P(x)}{
\exists x\colon (A\land P(x))
& \infer{A\land \exists x\colon P(x)}{
    \infer{A}{\infer[\infernote{1}]{A\land P(a)}{}}
  & \infer{\exists x\colon P(x)}{
      \infer{P(a)}{\infer[\infernote{1}]{A\land P(a)}{}}}
}
}
\end{array}\]
Zur zweiten Äquivalenz:
{\small
\[\begin{array}{c@{\quad}c}
\infer[\infernote{1}]{\forall x\colon (A\lor P(x))}{
A{\lor}(\forall x\colon P(x))
& \infer{\forall x\colon (A{\lor}P(x))}{
    \infer{A\lor P(x)}{\infer[\infernote{1}]{A}{}}
  }
& \infer{\forall x\colon (A{\lor}P(x))}{
    \infer{A\lor P(x)}{
      \infer{P(x)}{\infer[\infernote{1}]{\forall x\colon P(x)}{}}
    }
  }
}
&
\infer[\infernote{1}]{A\lor\forall x\colon P(x)}{
  \infer{A\lor P(x)}{\forall x\colon (A{\lor}P(x))}
  & \infer{A{\lor}\forall x\colon P(x)}{
      \infer[\infernote{1}]{A}{}}
  & \infer{A{\lor}\forall x\colon P(x)}{
      \infer{\forall x\colon P(x)}{
        \infer[\infernote{1}]{P(x)}{}}}
}
\end{array}\]
}

\noindent
Die linken Schemata zeigen jeweils die Implikation von links nach
rechts, die rechten jeweils die Implikation von rechts nach
links.\,\qedsymbol
\end{Beweis}

\begin{Satz}[exists-dl: Distributivgesetz]\label{exists-dl}
Es gilt:
\[(\exists x\colon P(x)\lor Q(x)) \iff
(\exists x\colon P(x))\lor(\exists x\colon Q(x)).\]
\end{Satz}
\begin{Beweis}[Beweis (Natürliches Schließen)] Schematisch.\\
Die Implikation von links nach rechts:
\[
\infer[\infernote{1}]{(\exists x\colon P(x))\lor (\exists x\colon Q(x))}{
  \exists x\colon P(x)\lor Q(x)
  &
  \infer[\infernote{2}]{(\exists x\colon P(x))\lor (\exists x\colon Q(x))}{
    \infer[\infernote{1}]{P(a)\lor Q(a)}{}
    & \infer{(\exists x\colon P(x)){\lor}(\exists x\colon Q(x))}{\infer{
        \exists x\colon P(x)}{\infer[\infernote{2}]{P(a)}{}}}
    & \infer{(\exists x\colon P(x)){\lor}(\exists x\colon Q(x))}{\infer{
        \exists x\colon Q(x)}{\infer[\infernote{2}]{Q(a)}{}}}
  }
}
\]
Die Implikation von rechts nach links:
\[\infer[\infernote{1}]{\exists x\colon P(x)\lor Q(x)}{
  (\exists x\colon P(x))\lor(\exists x\colon Q(x))
  & \infer[\infernote{2}]{\exists x\colon P(x)\lor Q(x)}{
      \infer[\infernote{1}]{\exists x\colon P(x)}{}
      &
      \infer{P(a)\lor Q(a)}{
        \infer[\infernote{2}]{P(a)}{}}}
  & \infer[\infernote{3}]{\exists x\colon P(x)\lor Q(x)}{
      \infer[\infernote{1}]{\exists x\colon Q(x)}{}
      &
      \infer{P(b)\lor Q(b)}{
        \infer[\infernote{3}]{Q(b)}{}}}
}\]
Man beachte, dass die Zeugen $a,b$ unterschiedlich sein können.\,\qedsymbol
\end{Beweis}

\begin{Satz}[exists-asym-dl: asymmetrisches Distributivgesetz]\label{exists-asym-dl}
Es gilt:
\[(\exists x\colon P(x)\land Q(x)) \implies (\exists x\colon P(x))\land (\exists x\colon Q(x)).\]
\end{Satz}
\begin{Beweis}[Beweis (Natürliches Schließen)] Schematisch:
\[\infer[\infernote{1}]{(\exists x\colon P(x))\land (\exists x\colon Q(x))}{
\exists x\colon P(x)\land Q(x)
& \infer{(\exists x\colon P(x))\land (\exists x\colon Q(x))}{
  \infer{\exists x\colon P(x)}{
       \infer{P(a)}{\infer[\infernote{1}]{P(a)\land Q(a)}{}}}
  & \infer{\exists x\colon Q(x)}{
       \infer{Q(a)}{\infer[\infernote{1}]{P(a)\land Q(a)}{}}}}
}\]
In Worten: Weil aufgrund der Prämisse ein Zeuge $a$ mit sowohl $P(a)$
als auch $Q(a)$ vorliegt, dürfen wir schließen, dass die
Existenzaussagen $\exists x\colon P(x)$ und $\exists x\colon Q(x)$
erfüllt sind.\,\qedsymbol
\end{Beweis}

\begin{Satz}[dm-g1: 1. allgemeines de morgansches Gesetz]%
\label{dm-g1} Es gilt
\[(\neg\exists x\colon P(x))\iff (\forall x\colon \neg P(x)).\]
\end{Satz}
\begin{Beweis} Unter Spezialisierung von Satz \ref{exists-implies-const}
findet sich die äquivalente Umformung%
\[\neg\exists x\colon P(x)\equiv (\exists x\colon P(x))\Rightarrow 0
\equiv \forall x\colon (P(x)\Rightarrow 0)
\equiv \forall x\colon \neg P(x).\,\qedsymbol\]
\end{Beweis}
\begin{Satz}[dm-g2: 2. allgemeines de morgansches Gesetz]%
\label{dm-g2} Es gilt
\[(\neg\forall x\colon P(x))\iff (\exists x\colon \neg P(x)).\]
\end{Satz}
\begin{Beweis}[Beweis (LEM)]
Nutzung von LEM und Satz \ref{dm-g1} gestattet die äquivalente
Umformung%
\[\neg\forall x\colon P(x)
\equiv\neg\forall x\colon \neg\neg P(x)
\equiv\neg\neg\exists x\colon \neg P(x)
\equiv\exists x\colon \neg P(x).\,\qedsymbol\]
\end{Beweis}

\begin{Satz}\label{exists-implies-const}
Es gilt:
\[(\forall x\colon (P(x)\implies A)) \iff ((\exists x\colon P(x))\implies A).\]
\end{Satz}
\begin{Beweis}[Beweis 1 (Natürliches Schließen)] Schematisch:\\
\[\begin{array}{c@{\qquad}c}
\infer[\infernote{1}]{(\exists x\colon P(x))\implies A}{
  \infer[\infernote{2}]{A}{
    \infer[\infernote{1}]{\exists x\colon P(x)}{}
    &
    \infer{A}{
      \infer{P(a)\implies A}{\forall x\colon (P(x)\implies A)}
      & \infer[\infernote{2}]{P(a)}{}}
  }
}
&
\infer{\forall x\colon (P(x)\implies A)}{
  \infer[\infernote{1}]{P(x)\implies A}{
    \infer{A}{
      (\exists x\colon P(x))\implies A
      & \infer{\exists x\colon P(x)}{
          \infer[\infernote{1}]{P(x)}{}}}}}
\end{array}
\]
Die linke Seite zeigt die Implikation von links nach rechts,
die rechte die Implikation von rechts nach links.\,\qedsymbol
\end{Beweis}
\begin{Beweis}[Beweis 2 (LEM, boolesche Algebra)]
Unter Nutzung von Satz \ref{general-dl} (general-dl)
und Satz \ref{dm-g1} (dm-g1) gilt
\begin{align*}
\forall x\colon (P(x)\Rightarrow A)
&\equiv \forall x\colon (\neg P(x)\lor A)
\equiv (\forall x\colon \neg P(x))\lor A\\
&\equiv \neg (\exists x\colon P(x))\lor A
\equiv (\exists x\colon P(x))\Rightarrow A.\,\qedsymbol
\end{align*}
\end{Beweis}

\begin{Satz}[exists-cl: Kommutativgesetz]\label{exists-cl}
Es gilt:
\[(\exists x\colon\exists y\colon P(x,y)) \iff (\exists y\colon\exists x\colon P(x,y)).\]
\end{Satz}
\begin{Beweis}[Beweis (Natürliches Schließen)]
Die Implikation von links nach rechts:
\[\infer[\infernote{1}]{\exists y\colon\exists x\colon P(x,y)}{
  \exists x\colon\exists y\colon P(x,y)
  &
  \infer{\exists y\colon \exists x\colon P(x, y)}{
    \infer[\infernote{1}]{P(a,b)}{}}
}\]
Die Implikation von rechts nach links geht analog.\,\qedsymbol
\end{Beweis}

\begin{Satz}[all-cl: Kommutativgesetz]\label{all-cl}
Es gilt:
\[(\forall x\colon\forall y\colon P(x,y))
\iff (\forall y\colon\forall x\colon P(x,y)).\]
\end{Satz}
\begin{Beweis}[Beweis (Natürliches Schließen)]
Die Implikations von links nach rechts:
\[\infer{\forall y\colon\forall x\colon P(x,y)}{
  \infer{\forall x\colon P(x,y)}{
    \infer{P(x,y)}{\forall x\colon\forall y\colon P(x,y)}}
}\]
Die Implikation von rechts nach links geht analog.\,\qedsymbol
\end{Beweis}

\begin{Satz}[bounded-general-dl: allgemeine Distributivgesetze]%
\label{bounded-general-dl}
Es gilt:
\begin{align}
A\land (\exists x{\in}M\colon P(x)) &\iff (\exists x{\in}M\colon A\land P(x)),\\
A\lor (\forall x{\in}M\colon P(x)) &\iff (\forall x{\in}M\colon A\lor P(x)).
\end{align}
\end{Satz}

\begin{Beweis}
Nach Def. \ref{def:bounded} (bounded)
und Satz \ref{general-dl} (general-dl) gilt:
\begin{gather*}
A\land \exists x{\in}M\colon P(x)
\equiv A\land \exists x\colon x\in M\land P(x)
\equiv \exists x\colon A\land x\in M\land P(x)\\
\equiv \exists x\colon x\in M\land A\land P(x)
\equiv \exists x{\in}M\colon A\land P(x).
\end{gather*}
Nach Def. \ref{def:bounded} (bounded)
und Satz \ref{general-dl} (general-dl) gilt:
\begin{gather*}
A\lor\forall x{\in}M\colon P(x)
\equiv A\lor\forall x\colon (x\in M\Rightarrow P(x))
\equiv A\lor\forall x\colon x\notin M\lor P(x)\\
\equiv \forall x\colon A\lor x\notin M\lor P(x)
\equiv \forall x\colon (x\in M\Rightarrow A\lor P(x))\\
\equiv \forall x{\in}M\colon A\lor P(x).\,\qedsymbol
\end{gather*}
\end{Beweis}

\begin{Satz}\label{bounded-exists-cl}
Es gilt:
\[(\exists x{\in}A\colon \exists y{\in}B\colon P(x,y))
\iff (\exists y{\in}B\colon \exists x{\in}A\colon P(x,y)).\]
\end{Satz}

\begin{Beweis} Nach Def. \ref{def:bounded} (bounded),
Satz \ref{general-dl} (general-dl)
und Satz \ref{exists-cl} (exists-cl) gilt:
\begin{gather*}
\exists x{\in}A\colon \exists y{\in}B\colon P(x,y)
\equiv \exists x\colon x\in A\land\exists y\colon y\in B\land P(x,y)\\
\equiv \exists x\colon \exists y\colon x\in A\land y\in B\land P(x,y)
\equiv \exists y\colon \exists x\colon y\in B\land x\in A\land P(x,y)\\
\equiv \exists y\colon y\in B\land \exists x\colon x\in A\land P(x,y)
\equiv \exists y{\in}B\colon\exists x{\in}A\colon P(x,y).\,\qedsymbol
\end{gather*}
\end{Beweis}

\begin{Satz}\label{bounded-all-cl}
Es gilt:
\[(\forall x{\in}A\colon \forall y{\in}B\colon P(x,y))
\iff (\forall y{\in}B\colon \forall x{\in}A\colon P(x,y)).\]
\end{Satz}
\begin{Beweis}[Beweis (LEM, boolesche Algebra)]\newlinefirst
Nach Def. \ref{def:bounded} (bounded),
Satz \ref{general-dl} (general-dl)
und Satz \ref{all-cl} (all-cl) gilt:
\begin{gather*}
\forall x{\in}A\colon \forall y{\in}B\colon P(x,y)
\equiv \forall x\colon x\in A\Rightarrow \forall y\colon y\in B\Rightarrow P(x,y)\\
\equiv \forall x\colon x\notin A\lor \forall y\colon y\notin B\lor P(x,y)
\equiv \forall x\colon \forall y\colon x\notin A\lor y\notin B\lor P(x,y)\\
\equiv \forall y\colon \forall x\colon y\notin B\lor x\notin A\lor P(x,y)
\equiv \forall y\colon y\notin B\lor \forall x\colon x\notin A\lor P(x,y)\\
\equiv \forall y\colon y\in B\Rightarrow \forall x\colon x\in A\Rightarrow P(x,y)
\equiv \forall y{\in}B\colon\forall x{\in}A\colon P(x,y).\,\qedsymbol
\end{gather*}
\end{Beweis}

\begin{Satz}
Für eine Aussage $P$, die nicht von $x$ abhängt, und ein nichtleeres
Diskursuniversum gilt:
\[(\exists x\colon P) \iff P.\]
\end{Satz}
\begin{Beweis}[Beweis]
Nach \ref{general-dl} (general-dl) gilt:
\[\exists x\colon P \equiv \exists x\colon (1\land P)
\equiv (\exists x\colon 1)\land P\equiv 1\land P\equiv P.\]
Im vorletzten Schritt wurde dabei ausgenutzt, dass
für ein nichtleeres Diskursuniversum immer $(\exists x\colon 1)\equiv 1$
gelten muss.\,\qedsymbol
\end{Beweis}

\begin{Satz}
Es gilt
\[(\exists x{\in}M\colon P) \iff (M\ne\emptyset)\land P.\]
\end{Satz}

\begin{Beweis}
Nach Def. \ref{def:bounded} (bounded) und Satz
\ref{general-dl} (general-dl) gilt:
\begin{gather*}
\exists x{\in}M\colon P \equiv \exists x\colon (x\in M\land P)
\equiv (\exists x\colon x\in M)\land P\equiv (M\ne\emptyset)\land P.\,\qedsymbol
\end{gather*}
\end{Beweis}

\newpage
\section{Mengenlehre}\index{Mengenlehre}
\subsection{Definitionen}

\begin{Definition}[seteq: Gleichheit von Mengen]
\label{def:seteq}\index{Gleichheit!von Mengen}
\[A=B \defiff \forall x\colon (x\in A\iff x\in B).\]
\end{Definition}

\begin{Definition}[subseteq: Teilmenge]%
\label{def:subseteq}\index{Teilmenge}
\[A\subseteq B \defiff \forall x\colon (x\in A\implies x\in B).\]
\end{Definition}

\begin{Definition}[filter: beschreibende Angabe]\label{def:filter}
\[a\in\{x\mid P(x)\} \defiff P(a).\]
\end{Definition}

\begin{Definition}[cap: Schnitt]%
\label{def:cap}\index{Schnittmenge}
\[A\cap B := \{x\mid x\in A\land x\in B\}.\]
\end{Definition}

\begin{Definition}[cup: Vereinigung]%
\label{def:cup}\index{Vereinigungsmenge}
\[A\cup B := \{x\mid x\in A\lor x\in B\}.\]
\end{Definition}

\begin{Definition}[intersection: Schnitt]\label{def:intersection}
\[\bigcap_{i\in I} A_i := \{x\mid \forall i{\in}I\colon x\in A_i\}
= \{x\mid \forall i\colon (i\in I\implies x\in A_i)\}.\]
\end{Definition}

\begin{Definition}[union: Vereinigung]\label{def:union}
\[\bigcup_{i\in I} A_i := \{x\mid \exists i{\in}I\colon x\in A_i\}
= \{x\mid \exists i\colon (i\in I\land x\in A_i)\}.\]
\end{Definition}

\begin{Definition}[cart: kartesisches Produkt]%
\label{def:cart}\index{kartesisches Produkt}
\[A\times B := \{(a,b)\mid a\in A\land b\in B\}
= \{t\mid\exists a\colon\exists b\colon (t=(a,b)\land a\in A\land b\in B)\}.\]
\end{Definition}

\subsection{Rechenregeln}

\begin{Satz}[Kommutativgesetze]\index{Kommutativgesetz!Mengen, boolesche Algebra}
Es gilt $A\cap B = B\cap A$ und $A\cup B = B\cup A$.
\end{Satz}

\begin{Beweis}
Nach Def. \ref{def:seteq} (seteq) expandieren:
\[\forall x\colon (x\in A\cap B \iff x\in B\cap A).\]
Nach Def. \ref{def:cap} (cap) und Def. \ref{def:filter} (filter) gilt:
\[x\in A\cap B \iff x\in A\land x\in B \iff x\in B\land x\in A
\iff x\in B\cap A.\]
Für die Vereinigung ist das analog.\,\qedsymbol
\end{Beweis}

\begin{Satz}[Assoziativgesetze]%
\index{Assoziativgesetz!Mengen, boolesche Algebra}
Es gilt $A\cap (B\cap C) = (A\cap B)\cap C$
und $A\cup (B\cup C) = (A\cup B)\cup C$.
\end{Satz}

\begin{Beweis}
Nach Def. \ref{def:seteq} (seteq) expandieren:
\[\forall x\colon [x\in A\cap (B\cap C) \iff x\in (A\cap B)\cap C].\]
Nach Def. \ref{def:cap} (cap) und Def. \ref{def:filter} (filter) gilt:
\begin{align*}
&x\in A\cap (B\cap C) \iff x\in A\land x\in B\cap C
\iff x\in A\land (x\in B\land x\in C)\\
&\iff (x\in A\land x\in B)\land x\in C
\iff x\in A\cap B\land x\in C
\iff x\in (A\cap B)\cap C.
\end{align*}
Für die Vereinigung ist das analog.\,\qedsymbol
\end{Beweis}

\begin{Satz}\label{cap-subseteq}
Es gilt $A\cap B\subseteq A$.
\end{Satz}
\begin{Beweis}
Expansion liefert die Formel $x\in A\land x\in B\implies x\in A$.
Gemäß boolescher Algebra gilt allgemein
\[\varphi\land\psi\Rightarrow\varphi
\equiv \neg(\varphi\land\psi)\lor\varphi
\equiv \neg\varphi\lor\neg\psi\lor\varphi
\equiv 1\lor\neg\psi\equiv 1.\]
Setze $\varphi:=(x\in A)$ und $\psi:=(x\in B)$.\;\qedsymbol
\end{Beweis}

\begin{Satz}\label{subseteq-char}
Es gilt $A\subseteq B\iff A\cap B=A$.
\end{Satz}
\begin{Beweis}
Aufgrund von Satz \ref{cap-subseteq} muss lediglich
$A\subseteq B\iff A\subseteq A\cap B$ gezeigt werden.
Expansion führt zur Formel
\[x\in A\Rightarrow x\in B\iff x\in A\Rightarrow x\in A\land x\in B.\]
Die Formel $\varphi\Rightarrow\psi\iff\varphi\Rightarrow\varphi\land\psi$
ist aber tautologisch, denn
\[\varphi\Rightarrow\varphi\land\psi\equiv\neg\varphi\lor(\varphi\land\psi)
\equiv (\neg\varphi\lor\varphi)\land(\neg\varphi\land\psi)
\equiv 1\land(\varphi\Rightarrow\psi)\equiv \varphi\Rightarrow\psi.\]
Setze $\varphi:=(x\in A)$ und $\psi:=(x\in B)$.\;\qedsymbol
\end{Beweis}

\begin{Satz}\label{eq-iff-all-iff}
Es gilt $a=b\iff \forall x\colon (x=a\iff x=b)$.
\end{Satz}

\begin{Beweis}
Die Implikation $a=b\implies\forall x\colon (x=a\iff x=b)$.
Wenn wir $a=b$ voraussetzen, kann $b$ gegen $a$ ersetzt werden
und es ergibt sich
\[(\forall x\colon (x=a\iff x=a))\iff (\forall x\colon 1)\iff 1.\]
Die andere Implikation bringen wir zunächst in ihre Kontraposition:
\[a\ne b\implies \exists x\colon ((x=a)\oplus (x=b)).\]
Auf einer leeren Grundmenge wird der Allquantifizierung
über $a,b$ immer genügt. Besitzt die Grundmenge nur ein Element,
dann muss $a=b$ sein, womit $a\ne b$ falsch ist und die Implikation
somit erfüllt. Wir setzen nun $a\ne b$ voraus. Wählt man nun
$x=a$, dann ist $x\ne b$, womit die Kontravalenz erfüllt wird.\;\qedsymbol
\end{Beweis}

\begin{Satz}
Es gilt $a=b\iff\{a\}=\{b\}$.
\end{Satz}

\begin{Beweis}
Es gilt:
\[\{a\}=\{b\}\iff \{x\mid x=a\}=\{x\mid x=b\}\iff \forall x\colon (x=a\iff x=b).\]
Nach Satz \ref{eq-iff-all-iff} ist das aber äquivalent zu $a=b$.\;\qedsymbol
\end{Beweis}

\begin{Satz}\label{eq-substitution}
Es gilt:
\[\forall x\colon\forall y\colon (x=y\land P(x)\iff P(y))\]
\end{Satz}

\begin{Satz}\label{all-cart}
Es gilt:
\[(\forall t \in A{\times}B\colon P(t)) \iff \forall a{\in}A\colon\forall b{\in}B\colon P(a,b).\]
\end{Satz}

\begin{Beweis}
Nach Def. \ref{def:cart} (cart) gilt:
\begin{align*}
&(\forall t \in A{\times}B\colon P(t))\iff (\forall t\colon t\in A\times B\implies P(t))\\
&\iff (\forall t\colon (\exists a\colon\exists b\colon t=(a,b)\land a\in A\land b\in B)\implies P(t)).
\end{align*}
Unter doppelter Anwendung von Satz \ref{exists-implies-const} gilt weiter:
\[\iff \forall t\colon\forall a\colon\forall b\colon [t=(a,b)\land a\in A\land b\in B\implies P(t)].\]
Substituiert man $t:=(a,b)$, dann ergibt sich:
\[\implies (\forall a\colon\forall b\colon a\in A\land B\in B\implies P(a,b))
\iff \forall a{\in}a\colon\forall b{\in}B\colon P(a,b),\]
wobei $P(a,b)$ eine Kurzschreibweise für $P((a,b))$ ist.
Von der Gegenrichtung bilden wir die Kontraposition:
\[(\exists t\colon\exists a\colon\exists b\colon t=(a,b)\land a\in A\land b\in B\land \overline{P(t)})
\implies \exists a\colon\exists b\colon a\in a\land b\in B\land \overline{P(a,b)}.\]
Dem $\exists t$ wird aber immer durch $t:=(a,b)$ genügt, so dass sich die
äquivalente Formel
\[(\exists a\colon\exists b\colon a\in A\land b\in B\land \overline{P(a,b)})
\implies \exists a\colon \exists b\colon a\in A\land b\in B\land \overline{P(a,b)}.\]
ergibt.\;\qedsymbol
\end{Beweis}

\begin{Satz}\label{exists-cart}
Es gilt:
\[(\exists t\in A{\times}B\colon P(t))
\iff (\exists a{\in}A\colon \exists b{\in}B\colon P(a,b)).\]
\end{Satz}

\begin{Beweis}
Nach Def. \ref{def:cart} (cart) gilt:
\begin{gather*}
(\exists t{\in}A{\times}B\colon P(t))
\iff (\exists t\colon P(t)\land t\in A\times B)\\
\iff (\exists t\colon P(t)\land \exists a\colon\exists b\colon t=(a,b)\land a\in A\land b\in B)\\
\iff (\exists t\colon \exists a\colon \exists b\colon P(t)\land a\in A\land b\in B\land t=(a,b))\\
\iff \exists a{\in}A\colon \exists b{\in}B\colon \exists t\colon P(t)\land t=(a,b).
\end{gather*}
Nun gilt aber ganz offensichtlich
\[(\exists t\colon P(t)\land t=(a,b))\iff P(a,b).\]
Nimmt man $P(a,b)$ an, dann lässt sich $\exists t\colon P(t)\land t=(a,b)$
durch Wahl von $t:=(a,b)$ bestätigen. Nimmt man umgekehrt
$\exists t\colon P(t)\land t=(a,b)$ an, lässt sich $P(a,b)$ daraus
unter Anwendung von Satz \ref{eq-substitution} ableiten.
Da $\exists t\colon P(t)\land t=(a,b)$ gegen $P(a,b)$ ersetzt
werden darf, folgt die Behauptung.\,\qedsymbol
\end{Beweis}

\newpage
\begin{Satz}\label{cup-cart}
Es gilt:
\[\bigcup_{t\in I\times J} A_t
= \bigcup_{i\in I}\bigcup_{j\in J} A_{ij}.\quad (t=(i,j))\]
\end{Satz}

\begin{Beweis}
Nach Def. \ref{def:union} (union) und Satz \ref{exists-cart} gilt:
\begin{gather*}
x\in \bigcup_{t\in I\times J} A_t
\iff (\exists t\in I{\times J}\colon x\in A_t)
\iff (\exists i{\in}I\colon\exists j{\in}J\colon x\in A_{ij})\\
\iff (\exists i{\in}I\colon x\in \bigcup_{j\in J} A_{ij})
\iff x\in\bigcup_{i\in I}\bigcup_{j\in J} A_{ij}.
\end{gather*}
Nach Def. \ref{def:seteq} (seteq) folgt die Behauptung.\,\qedsymbol
\end{Beweis}

\begin{Satz}
Es gilt:
\[\bigcup_{i\in I}\bigcup_{j\in J} A_{ij}
= \bigcup_{j\in J}\bigcup_{i\in I} A_{ij}.\]
\end{Satz}

\begin{Beweis}
Nach Def. \ref{def:union} (union) und Satz \ref{bounded-exists-cl} gilt:
\begin{gather*}
x\in\bigcup_{i\in I}\bigcup_{j\in J} A_{ij}
\iff (\exists i{\in}I\colon x\in\bigcup_{j\in J} A_{ij})
\iff (\exists i{\in}I\colon\exists j{\in}J\colon x\in A_{ij})\\
\iff (\exists j{\in}J\colon\exists i{\in}I\colon x\in A_{ij})
\iff (\exists j{\in}J\colon x\in \bigcup_{i\in I}A_{ij})
\iff x\in\bigcup_{j\in J}\bigcup_{i\in I} A_{ij}.
\end{gather*}
Nach Def. \ref{def:seteq} (seteq) folgt die Behauptung.\,\qedsymbol
\end{Beweis}

\begin{Satz}
Die Relation \emph{Teilmenge von} ist eine Partialordnung.
Im Einzelnen gilt%
\begin{align*}
\text{(1)}\quad & A\subseteq A, && \text{(Reflexivität)}\\
\text{(2)}\quad & A\subseteq B\land B\subseteq A \implies A = B, && \text{(Antisymmetrie)}\\
\text{(3)}\quad & A\subseteq B\land B\subseteq C \implies A\subseteq C. && \text{(Transitivität)}
\end{align*}
\end{Satz}
\begin{Beweis}
Jeweils Def. \ref{def:subseteq} nutzen.

Zu (1). Die Aussage $A\subseteq A$ ist 
äquivalent zu $\forall x\colon (x\in A\Rightarrow x\in A)$.
Eine Prämisse impliziert sich im Allgemeinen selbst.

Zu (2). Es findet sich die äquivalente Umformung
\begin{gather*}
A\subseteq B\land B\subseteq A\iff
(\forall x\colon x\in A\Rightarrow x\in B)
\land (\forall x\colon x\in B\Rightarrow x\in A)\\
\iff (\forall x\colon (x\in A\Rightarrow x\in B)\land (x\in B\Rightarrow x\in A))
\iff (\forall x\colon (x\in A\Leftrightarrow x\in B))\\
\iff A = B.
\end{gather*}

Zu (3). Zu zeigen ist $\forall x\colon (x\in A\Rightarrow x\in C)$.
Sei $x\in A$ fest, aber beliebig. Wegen $A\subseteq B$ muss $x\in B$
sein. Wegen $B\subseteq C$ muss infolge $x\in C$ sein.\,\qedsymbol
\end{Beweis}

\begin{Satz}
Die Aussage $A\subseteq B$ ist äquivalent zu $1_A\le 1_B$.
\end{Satz}
\begin{Beweis}
Es gelte $A\subseteq B$. Um $\forall x\colon 1_A(x)\le 1_B(x)$ zu
zeigen, wird eine Fallunterscheidung in drei Fälle vorgenommen.
Sei $x\notin B$. Dann ist $x\notin A$
und daher $1_A(x)=0$ und $1_B(x)=0$, womit $1_A(x)\le 1_B(x)$ gilt.
Sei $x\in A$. Dann ist $x\in B$, und daher $1_A(x)=1$ und $1_B(x)=1$,
womit $1_A(x)\le 1_B(x)$ gilt.
Sei $x\notin A$, aber $x\in B$. Dann ist $1_A(x)=0$ und $1_B(x)=1$,
womit $1_A(x)\le 1_B(x)$ gilt.

Es gelte $\forall x\colon 1_A(x)\le 1_B(x)$. Sei $x\in A$ fest,
aber beliebig. Wegen $1_A(x)=1$ ist $1\le 1_B(x)$. Weil somit
$1_B(x)\ne 0$ ist, verbleibt nur noch $1_B(x)=1$, was
gleichbedeutend mit $x\in B$ ist.\,\qedsymbol
\end{Beweis}

\newpage
\section{Relationen}\index{Relation}

\subsection{Allgemeine Gesetzmäßigkeiten}

\begin{Definition}[rel: Relation]\newlinefirst
Zu zwei Mengen $X,Y$ bezeichnet man
jede Menge $R\subseteq X\times Y$ als Relation.
\end{Definition}
\begin{Definition}[img: Bildmenge]\label{def:relation-img}
Zu einer Relation $R$ wird die Menge
\[R(M) := \{y\mid\exists x\in M\colon (x,y)\in R\}\]
als Bildmenge von $M$ unter $R$ bezeichnet.
\end{Definition}

\begin{Satz}
Sei $R$ eine Relation und seien $A,B$ beliebige Mengen.\\
Es gilt $R(A\cup B) = R(A)\cup R(B)$.
\end{Satz}
\begin{Beweis}
Expansion mit Def. \ref{def:relation-img} (img) führt zur Behauptung
\[(\exists x\in A\cup B\colon (x,y)\in R) \iff (\exists x\in A\colon (x,y)\in R)
\lor (\exists x\in B\colon (x,y)\in R).\]
Die linke Seite lässt sich gemäß Def. \ref{def:bounded} (bounded),
Def. \ref{def:cup} (cup) und Satz \ref{exists-dl} (exists-dl)
äquivalent in die rechte umformen.\;\qedsymbol
\end{Beweis}

\begin{Satz}
Sei $R$ eine Relation und seien $A,B$ beliebige Mengen.\\
Es gilt $R(A)\setminus R(B)\subseteq R(A\setminus B)$.
\end{Satz}
\begin{Beweis}
Expansion mit Def. \ref{def:relation-img} (img) führt zur Behauptung
\[(\exists x\in A\colon (x,y)\in R)\land (\forall x\in B\colon (x,y)\notin R)
\implies \exists x\in A\setminus B\colon (x,y)\in R.\]
Laut der ersten Prämisse existiert ein $x\in A$ mit $(x,y)\in R$. Die
zweite Prämisse ist äquivalent zur Kontraposition $(x,y)\in R\Rightarrow x\notin B$.
Infolge ist $x\in A\setminus B$. Somit bezeugt $x$ die
Existenzaussage auf der rechten Seite.\,\qedsymbol
\end{Beweis}

\subsection{Äquivalenzrelationen}

\begin{Definition}[Äquivalenzrelation]\newlinefirst
Sei $M$ eine Menge. Man nennt $R\subseteq M\times M$, notiert als
$R(x,y) = (x\sim y)$, eine Äquivalenzrelation auf $M$, wenn für alle
$x,y,z\in M$ erfüllt ist:%
\begin{align*}
& x\sim x, && \text{(Reflexivität)}\\
& x\sim y\implies y\sim x, && \text{(Symmetrie)}\\
& x\sim y\land y\sim z\implies x\sim y. && \text{(Transitivität)}
\end{align*}
\end{Definition}

\begin{Definition}[Äquivalenzklasse]\newlinefirst
Sei $M$ eine Menge und $x\sim y$ eine Äquivalenzrelation für $x,y\in M$.
Die Menge%
\[[a] := \{x\in M\mid x\sim a\}\]
nennt man Äquivalenzklasse zum Repräsentanten $a\in M$.
\end{Definition}

\begin{Definition}[Quotientenmenge]\newlinefirst
Die Menge $M/{\sim} := \{A\mid \exists a\in M\colon A = [a]\}$
aller Äquivalenzklassen heißt Quotientenmenge von $M$ bezüglich der
Äquivalenzlation ${\sim}$.
\end{Definition}

\begin{Definition}[Quotientenabbildung]\newlinefirst
Die Abbildung $\pi\colon M\to M/{\sim}$ mit $\pi(x):=[x]$
heißt Quotientenabbildung.
\end{Definition}

\begin{Satz}[Äquivalenzrelation induziert disjunkte Zerlegung]\newlinefirst
Eine Menge wird durch eine auf ihr definierte Äquivalenzrelation
in paarweise disjunkte Äquivalenzklassen zerlegt.
\end{Satz}
\begin{Beweis}
Es ist zu zeigen, dass zwei unterschiedliche Äquivalenzklassen 
kein Element gemeinsam haben. Wir zeigen die Kontraposition, dass die
Existenz eines $x$ mit $x\in [a]$ und $x\in [b]$ bereits $[a]=[b]$ impliziert.
Laut Prämisse ist $x\sim a$ und $x\sim b$, und wegen der Transitivtät
infolge $a\sim b$, was äquivalent zu $[a]=[b]$ ist.

Zu bestätigen verbleibt noch, dass die Quotientenabbildung eine
surjektive ist. Dies ist wahr, weil $M/{\sim}$ gerade so definiert ist, 
dass direkt $M/{\sim}=\pi(M)$ gilt.\,\qedsymbol
\end{Beweis}

\begin{Satz}[Disjunkte Zerlegung induziert Äquivalenzrelation]\newlinefirst
Sei $M$ eine Menge. Die Familie $(A_k)$ der $A_k\subseteq M$ sei
eine Zerlegung von $M$ in paarweise disjunkte Mengen. Dann definiert
\[x\sim y\defiff \exists k\colon x\in A_k\land y\in A_k\]
eine Äquivalenzrelation.
\end{Satz}
\begin{Beweis}
Da die $A_k$ die Menge $M$ überdecken, muss für jedes $x\in M$
ein $k$ mit $x\in A_k$ existieren, womit die Reflexivität
$x\sim x$ erfüllt ist.

Die Symmetrie folgt unmittelbar aus der Kommutativität der Konjunktion.

Zur Transitivität. Seien $x,y,z$ fest, aber beliebig. Zudem seien die
Prämissen $x\sim y$ und $y\sim z$ erfüllt. Wir haben daher einen Zeugen
$i$ mit $x\in A_i$ und $y\in A_i$ und einen Zeugen $j$ mit $y\in A_j$
und $z\in A_j$. Infolge ist $y\in A_i\cap A_j$. Wegen
$A_i\cap A_j=\emptyset$ für $i\ne j$ muss $i=j$ sein. Deshalb ist
$i$ ein Zeuge für $\exists i\colon x\in A_i\land z\in A_i$, womit
$x\sim z$ gilt.\,\qedsymbol
\end{Beweis}

\begin{Satz}[Charakterisierung von Äquivalenzklassen]\newlinefirst
Sei $\sim$ eine Äquivalenzrelation auf der Menge $M$. Eine
Teilmenge $A\subseteq M$ ist genau dann eine Äquivalenzklasse, wenn%
\begin{gather*}
\text{(1)}\quad A\ne\emptyset,\\
\text{(2)}\quad x,y\in A\implies x\sim y,\\
\text{(3)}\quad x\in A\land y\in M\land x\sim y\implies y\in A.
\end{gather*}
\end{Satz}
\begin{Beweis}
Sei $A$ eine Äquivalenzklasse. Dann existiert definitionsgemäß
ein $a$, so dass $A=[a]$ gilt. Ergo ist $a\in A$, womit
$A\ne\emptyset$ sein muss. Mit $x,y\in A$ ergibt sich $[x]=[y]$,
was äquivalent zu $x\sim y$ ist. Sei nun $x\in A$ und $y$ irgendein
Element in $M$ mit $x\sim y$. Dies bedeutet $A=[x]=[y]$, womit
$y\in A$ sein muss.

Umgekehrt seien die drei Eigenschaften erfüllt. Zu zeigen ist,
dass ein Zeuge $a$ mit $A=[a]$ existiert. Weil $A$ gemäß (1)
nichtleer ist, muss ein Element $a\in A$ existieren. Für jedes
weitere Element $x\in A$ ergibt sich $x\sim a$ aufgrund (2),
also $x\in [a]$, womit wir $A\subseteq [a]$ haben.
Es verbleibt $[a]\subseteq A$ zu zeigen. Sei also $x\in [a]$.
Wir haben damit die Situation $a\in A$ und $x\sim a$, womit
laut (3) ebenso $x\in A$ sein muss.\,\qedsymbol
\end{Beweis}

\newpage
\section{Abbildungen}\index{Abbildung}
\subsection{Definitionen}

\begin{Definition}[app: Applikation]\label{def:app}
Für eine Abbildung $f$ ist
\[y=f(x)\defiff (x,y)\in G_f.\]
\end{Definition}

\begin{Definition}[img: Bildmenge]%
\label{def:img}\index{Bildmenge}\newlinefirst
Für eine Abbildung $f\colon X\to Y$ und $A\subseteq X$
wird die Menge
\[f(A) := \{y\mid \exists x\in A\colon y=f(x)\}
= \{y\mid \exists x\colon (x\in A\land y=f(x))\}\]
als Bildmenge von $A$ unter $f$ bezeichnet.
\end{Definition}

\begin{Definition}[preimg: Urbildmenge]\label{def:preimg}\index{Urbildmenge}
Für eine Abbildung $f\colon X\to Y$ wird
\[f^{-1}(B) := \{x\mid f(x)\in B\} = \{x\mid \exists y\in B\colon y=f(x)\}\]
als Urbildmenge von $B$ unter $f$ bezeichnet.
\end{Definition}

\begin{Definition}[inj: Injektion]%
\label{def:inj}\index{Injektion}\newlinefirst
Eine Abbildung $f\colon X\to Y$ heißt genau dann injektiv, wenn gilt:%
\[\forall x_1\colon \forall x_2\colon (f(x_1)=f(x_2)\implies x_1=x_2)\]
bzw. äquivalent (Kontraposition)
\[\forall x_1\colon \forall x_2\colon (x_1\ne x_2\implies f(x_1)\ne f(x_2)).\]
\end{Definition}

\begin{Definition}[sur: Surjektion]%
\label{def:sur}\index{Surjektion}\newlinefirst
Eine Abbildung $f\colon X\to Y$ heißt genau dann surjektiv, wenn gilt:%
\[Y\subseteq f(X).\]
\end{Definition}

\begin{Definition}[composition: Verkettung]\label{def:composition}%
\index{Komposition}\index{Verkettung}\newlinefirst
Für Abbildungen $f\colon X\to Y$ und $g\colon Y\to Z$ heißt%
\[(g\circ f)\colon X\to Z,\quad (g\circ f)(x):=g(f(x))\]
Verkettung von $f$ und $g$, sprich »$g$ nach $f$«.
\end{Definition}

\subsection{Grundlagen}
\begin{Satz}[feq: Gleichheit von Abbildungen]%
\label{feq}\index{Gleichheit!von Abbildungen}
Zwei Abbildungen $f\colon X\to Y$ und $g\colon X'\to Y'$ sind genau
dann gleich, kurz $f=g$, wenn $X=X'$ und $Y=Y'$ und%
\[\forall x\colon f(x)=g(x).\]
\end{Satz}

\begin{Beweis}
Nach Definition gilt $f=g$ genau dann, wenn $(G_f,X,Y)=(G_g,X',Y')$,
was äquivalent zu $G_f=G_g\land X=X'\land Y=Y'$ ist. Nach Def.
\ref{def:seteq} (seteq) gilt%
\[G_f=G_g\iff \forall t\colon (t\in G_f\Leftrightarrow t\in G_g).\]
Nach Satz \ref{eq-iff-all-iff} und Def. \ref{def:app} (app) gilt
\begin{align*}
&(\forall x\colon f(x)=g(x)) \iff (\forall x\colon\forall y\colon (y=f(x)\Leftrightarrow y=g(x)))\\
&\iff (\forall x\colon\forall y\colon((x,y)\in G_f\Leftrightarrow (x,y)\in G_g))
\iff \forall t\colon (t\in G_f\Leftrightarrow t\in G_g).
\end{align*}
Da die Quantifizerung auf $x\in X$, $y\in Y$ und $t\in X\times Y$
beschränkt ist, konnte im letzten Schritt Satz \ref{all-cart}
angewendet werden.\;\qedsymbol
\end{Beweis}

\begin{Satz} Für eine Abbildung $f$ gilt
\[f(x)\in A\cap B\iff f(x)\in A\land f(x)\in B.\]
\end{Satz}
\begin{Beweis}
Es gelte $f(x)\in A\cap B$. Dann existiert laut Definition ein
$y\in A\cap B$ mit $y=f(x)$, womit $y\in A$ und $y\in B$ gilt.
Folglich gilt $f(x)\in A$ und $f(x)\in B$.

Es gelte $f(x)\in A$ und $f(x)\in B$. Dann existiert laut Definition
ein $y\in A$ mit $y=f(x)$ und ein $y'\in B$ mit $y'=f(x)$.
Weil $f$ dem $x$ genau ein Bild zuordnet, muss $y=y'$ gelten.
Folglich gilt $y\in A\cap B$, und somit $f(x)\in A\cap B$.\,\qedsymbol
\end{Beweis}

\begin{Satz} Für eine Abbildung $f$ gilt
\[f(x)\in A\cup B\iff f(x)\in A\lor f(x)\in B.\]
\end{Satz}
\begin{Beweis} Es findet sich die äquivalente Umformung
\begin{align*}
f(x)\in A\cup B &\iff (\exists y\colon y\in A\cup B\land y=f(x))\\
&\iff (\exists y\colon (y\in A\lor y\in B)\land y = f(x))\\
&\iff (\exists y\colon y\in A\land y=f(x)\lor y\in B\land y = f(x))\\
&\iff (\exists y\colon y\in A\land y=f(x))\lor (\exists y\colon y\in B\land y = f(x))\\
&\iff f(x)\in A\lor f(x)\in B.\,\qedsymbol
\end{align*}
\end{Beweis}

\begin{Satz}[preimg-dl: Distributivität der Urbildoperation]%
\label{preimg-dl}\index{Distributivgesetz!Urbildoperation}\newlinefirst
Für $f\colon X\to Y$ und beliebige Mengen $M_i$ gilt:%
\begin{align}
f^{-1}(M_1\cap M_2) &= f^{-1}(M_1)\cap f^{-1}(M_2),\\
f^{-1}(M_1\cup M_2) &= f^{-1}(M_1)\cup f^{-1}(M_2),\\
f^{-1}(\bigcap_{i\in I} M_i) &= \bigcap_{i\in I} f^{-1}(M_i),\\
f^{-1}(\bigcup_{i\in I} M_i) &= \bigcup_{i\in I} f^{-1}(M_i).
\end{align}
\end{Satz}

\begin{Beweis}
Nach Def. \ref{def:seteq} (seteq) expandieren:
\[\forall x\colon [x\in f^{-1}(M_1\cap M_2)\iff x\in f^{-1}(M_1)\cap f^{-1}(M_2)].\]
Nach Def. \ref{def:preimg} (preimg) und Def. \ref{def:cap} (cap)
zusammen mit Def. \ref{def:filter} (filter) gilt:
\begin{align*}
& x\in f^{-1}(M_1\cap M_2) \iff f(x)\in M_1\cap M_2
\iff f(x)\in M_1\land f(x)\in M_2\\
&\iff x\in f^{-1}(M_1)\land x\in f^{-1}(M_2)
\iff x\in f^{-1}(M_1)\cap f^{-1}(M_2).
\end{align*}
Für die Vereinigung ist das analog.

Schnitt von beliebig vielen Mengen.
Nach Def. \ref{def:seteq} (seteq) expandieren:
\[\forall x\colon [x\in f^{-1}(\bigcap_{i\in I}M_i)
\iff x\in \bigcap_{i\in I} f^{-1}(M_i)].\]
Nach Def. \ref{def:preimg} (preimg) und Def. \ref{def:intersection}
(intersection) zusammen mit Def. \ref{def:filter} (filter) gilt:
\begin{align*}
& x\in f^{-1}(\bigcap_{i\in I} M_i)\iff f(x)\in\bigcap_{i\in I} M_i
\iff \forall i(i\in I\implies f(x)\in M_i)\\
&\iff \forall i(i\in I\implies x\in f^{-1}(M_i))
\iff x\in \bigcap_{i\in I} f^{-1}(M_i).
\end{align*}
Für die Vereinigung ist das analog.\;\qedsymbol
\end{Beweis}

\begin{Satz}[img-cup-dl: Distributivität der Bildoperation über die Vereinigung]
Für $f\colon X\to Y$ und Mengen $M_i\subseteq X$ gilt:
\begin{align}
f(M_1\cup M_2) &= f(M_1)\cup f(M_2),\\
f(\bigcup_{i\in I} M_i) &= \bigcup_{i\in I} f(M_i).
\end{align}
\end{Satz}
\begin{Beweis}
Nach Def. \ref{def:seteq} (seteq) expandieren:
\[\forall y\colon (y\in f(M_1\cup M_2)\iff y\in f(M_1)\cup f(M_2)).\]
Nach Def. \ref{def:img} (img), Def. \ref{def:cup} (cup),
Satz \ref{bool-dl} (bool-dl) und Satz \ref{exists-dl} (exists-dl) gilt:
\begin{align*}
&y\in f(M_1\cup M_2) \iff (\exists x\colon x\in M_1\cup M_2\land y=f(x))\\
&\iff (\exists x\colon (x\in M_1\lor x\in M_2)\land y=f(x))\\
&\iff (\exists x\colon x\in M_1\land y=f(x)\lor x\in M_2\land y=f(x))\\
&\iff (\exists x\colon x\in M_1\land y=f(x))\lor (\exists x\colon x\in M_2\land y=f(x))\\
&\iff y\in f(M_1)\lor y\in f(M_2) \iff y\in f(M_1)\cup f(M_2).
\end{align*}
Nach Def. \ref{def:seteq} (seteq) expandieren:
\[\forall y\colon [y\in f(\bigcup_{i\in I} M_i)\iff y\in \bigcup_{i\in I} f(M_i)].\]
Nach Def. \ref{def:img} (img), Def. \ref{def:union} (union),
Satz \ref{general-dl} (general-dl)\\
und Satz \ref{exists-cl} (exists-cl) gilt:
\begin{align*}
& y\in f(\bigcup_{i\in I} M_i)
\iff (\exists x\colon x\in\bigcup_{i\in I} M_i\land y=f(x))\\
&\iff (\exists x\colon (\exists i\colon i\in I\land x\in M_i)\land y=f(x))
\iff (\exists x\colon \exists i\colon i\in I\land x\in M_i\land y=f(x))\\
&\iff (\exists i\exists x\colon i\in I\land x\in M_i\land y=f(x))
\iff (\exists i\colon i\in I\land\exists x(x\in M_i\land y=f(x))\\
&\iff (\exists i\colon i\in I\land y\in f(M_i))
\iff y\in\bigcup_{i\in I} f(M_i).\;\qedsymbol
\end{align*}
\end{Beweis}

\begin{Satz}\label{img-cap-semi-dl}
Es gilt:
\begin{align}
f(M_1\cap M_2) &\subseteq f(M_1)\cap f(M_2),\\
f(\bigcap_{i\in I} M_i) &\subseteq \bigcap_{i\in I} f(M_i).
\end{align}
\end{Satz}

\begin{Beweis}
Nach Def. \ref{def:subseteq} (subseteq) expandieren:
\[\forall y\colon (y\in f(M_1\cap M_2)\implies y\in f(M_1)\cap f(M_2)).\]
Nach Def. \ref{def:img} (img), Def. \ref{def:cap} (cap)
und Satz. \ref{exists-asym-dl} (exists-asym-dl) gilt:
\begin{align*}
& y\in f(M_1\cap M_2) \iff (\exists x\colon x\in M_1\cap x\in M_2\land y=f(x))\\
&\iff (\exists x\colon x\in M_1\land x\in M_2\land y=f(x))\\
&\iff (\exists x\colon x\in M_1\land y=f(x)\land x\in M_2\land y=f(x))\\
&\implies (\exists x\colon x\in M_1\land y=f(x))\land (\exists x\colon x\in M_2\land y=f(x))\\
&\iff y\in f(M_1)\land y\in f(M_2)\iff y\in f(M_1)\cap f(M_2).
\end{align*}
Nach Def. \ref{def:subseteq} (subseteq) expandieren:
\[\forall y\colon (y\in f(\bigcap_{i\in I} M_i)\implies y\in \bigcap_{i\in I} f(M_i))\]
Nach Def. \ref{def:img} (img) und Def. \ref{def:intersection} (intersection)
gilt:
\begin{align*}
& y\in f(\bigcap_{i\in I} M_i)\iff (\exists x\colon x\in\bigcap_{i\in I} M_i\land y=f(x))\\
& \iff (\exists x\colon (\forall i\colon i\in I\Rightarrow x\in M_i)\land y=f(x))\\
& \iff (\exists x\colon \forall i\colon i\in I\Rightarrow x\in M_i\land y=f(x))\\
& \implies (\forall i\colon\exists x\colon i\in I\Rightarrow x\in M_i\land y=f(x))\\
& \iff (\forall i\colon i\in I\Rightarrow \exists x\colon x\in M_i\land y=f(x))\\
& \iff (\forall i\colon i\in I\Rightarrow y\in f(M_i))
\iff y\in\bigcap_{i\in I} f(M_i).\;\qedsymbol
\end{align*}
\end{Beweis}

\begin{Satz}\label{disjoint-preimg}
Zwei disjunkte Mengen haben disjunkte Urbilder.
\end{Satz}
\begin{Beweis} Sei $A\cap B=\emptyset$. Gemäß Satz
\ref{preimg-dl} (preimg-dl) ist
\[f^{-1}(A)\cap f^{-1}(B) = f^{-1}(A\cap B) = f^{-1}(\emptyset)
= \emptyset.\;\qedsymbol\]
\end{Beweis}

\begin{Satz}
Es gilt $M\subseteq N\implies f^{-1}(M)\subseteq f^{-1}(N)$.
\end{Satz}
\begin{Beweis}[Beweis\;1]
Gemäß Satz \ref{subseteq-char} ist $M\subseteq N$ äquivalent zu
$M\cap N=M$. Man wendet die Urbildoperation $f^{-1}$ nun auf beide
Seiten der Gleichung an und erhält mittles Satz \ref{preimg-dl}
(preimg-dl) dann%
\[f^{-1}(M\cap N) = f^{-1}(M)\cap f^{-1}(N) = f^{-1}(M).\]
Nochmalige Anwendung von Satz \ref{subseteq-char} liefert
das gewünschte Resultat
\[f^{-1}(M)\subseteq f^{-1}(N).\;\qedsymbol\]
\end{Beweis}

\begin{Beweis}[Beweis\;2]
Die Expansion der Aussage bringt
\[(y\in M\Rightarrow y\in N)\implies (f(x)\in M\Rightarrow f(x)\in N).\]
Trivialerweise kann die Prämisse mit $y:=f(x)$ spezialisiert
werden werden.\;\qedsymbol
\end{Beweis}

\begin{Satz}\label{img-subseteq}
Es gilt $M\subseteq N\implies f(M)\subseteq f(N)$.
\end{Satz}
\begin{Beweis}
Gemäß Satz \ref{subseteq-char} ist $M\subseteq N$ äquivalent zu
$M\cap N=M$. Man wendet die Bildoperation nun auf beide Seiten
der Gleichung an und erhält mittels Satz \ref{img-cap-semi-dl}
dann%
\[f(M) = f(M\cap N)\subseteq f(M)\cap f(N).\]
Laut Satz \ref{cap-subseteq} ist folglich $f(M)=f(M)\cap f(N)$.
Nochmalige Anwendung von Satz \ref{subseteq-char} bringt das
gewünschte Resultat $f(M)\subseteq f(N)$.\;\qedsymbol
\end{Beweis}

\begin{Satz}\label{img-as-cup}
Es gilt:
\[f(M) = \bigcup_{x\in M} \{f(x)\}.\]
\end{Satz}

\begin{Beweis}
Nach Def. \ref{def:img} (img) und Def. \ref{def:union} (union) gilt:
\[y\in f(M) \iff (\exists x{\in}M\colon y=f(x))
\iff (\exists x{\in}M\colon y\in \{f(x)\})
\iff y\in\bigcup_{x\in M}\{f(x)\}.\]
Nach Def. \ref{def:seteq} (seteq) folgt dann die Behauptung.\,\qedsymbol
\end{Beweis}

\begin{Satz}\label{preimg-chain}
Es gilt $(g\circ f)^{-1}(M) = f^{-1}(g^{-1}(M))$.
\end{Satz}
\begin{Beweis}
Nach Def. \ref{def:preimg} (preimg) und Def. \ref{def:seteq} (seteq)
expandieren und Def. \ref{def:filter} (filter) anwenden:%
\[(g\circ f)(x)\in M \iff f(x)\in\{y\mid g(y)\in M\}.\]
Links Def. \ref{def:composition} (composition) anwenden und rechts
nochmals Def. \ref{def:filter} (filter):%
\[g(f(x))\in M \iff g(f(x))\in M.\;\qedsymbol\]
\end{Beweis}

\begin{Satz}\label{img-chain}
Es gilt $(g\circ f)(M) = g(f(M))$.
\end{Satz}
\begin{Beweis}
Nach Def. \ref{def:img} (img) und Def. \ref{def:seteq} expandieren,
dann \ref{def:filter} (filter) anwenden:%
\[(\exists x\colon x\in M\land z=(g\circ f)(x))
\iff (\exists y\colon y\in f(M)\land z=g(y)).\]
Die rechte Seite mit Def. \ref{def:img} (img) expandieren und
Def. \ref{def:filter} (filter) anwenden. Unter Anwendung von
Satz \ref{general-dl} (general-dl) und Satz \ref{exists-cl} (exists-cl)
ergibt sich%
\begin{gather*}
(\exists y\colon (\exists x\colon x\in M\land y=f(x))\land z=g(y))\\
\iff (\exists y\colon\exists x\colon x\in M\land y=f(x)\land z=g(y))\\
\iff (\exists x\colon x\in M\land\exists y\colon y=f(x)\land z=g(y))\\
\iff (\exists x\colon x\in M\land z=g(f(x)))\\
\iff (\exists x\colon x\in M\land z=(g\circ f)(x)).\,\qedsymbol
\end{gather*}
\end{Beweis}

\begin{Satz}\label{left-inverse}
Sei $f\colon A\to B$ eine Abbildung und $A\ne\emptyset$. Man nennt
eine Funktion $g\colon B\to A$ mit $g\circ f=\id_A$ Linksinverse
zu $f$. Die Abbildung $f$ ist genau dann injektiv, wenn eine
Linksinverse zu $f$ existiert.
\end{Satz}
\begin{Beweis}
Sei $f$ injektiv. Man wähle ein $a\in A$, das wegen $A\ne\emptyset$
existieren muss. Man definiert nun $g\colon B\to A$ mit%
\[g(y):=\begin{cases}
x\;\text{wobei}\;y=f(x),\;\text{wenn}\;y\in f(A),\\
a\;\text{wenn}\;y\notin f(A).
\end{cases}\]
Diese Funktion ist eindeutig definiert, weil $f$ injektiv ist.
Gemäß ihrer Definition gilt $g(f(x))=x$, bzw. $g\circ f = \id$.

Sei nun eine Linksinverse $g$ mit $g\circ f=\id$ gegeben. Dann gilt
\[f(a)=f(b) \implies g(f(a))=g(f(b))\]
und
\[g(f(a))=g(f(b))
\iff (g\circ f)(a)=(g\circ f)(a)
\iff \id(a)=\id(b)
\iff a=b.\]
Es ergibt sich
\[f(a)=f(b)\implies a=b.\,\qedsymbol\]
\end{Beweis}

\begin{Satz}\label{preimg-compl}
Es gilt $f^{-1}(A^\comp)=f^{-1}(A)^\comp$ bzw.
$f(x)\in A^\comp\Leftrightarrow \neg f(x)\in A$.
\end{Satz}
\begin{Beweis} Zufolge der Expansion von Def. \ref{def:preimg} (preimg) ist
\[(\exists y\in A^\comp\colon y=f(x))\iff \neg (\exists y\in A\colon y=f(x))
\iff (\forall y\in A\colon y\ne f(x))\]
zu zeigen. Weil $x$ ein Element des Definitionsbereichs ist, muss
der Funktionswert $f(x)$ in irgendeiner Menge liegen. Die Implikation
von rechts nach links. Weil $f(x)$ nicht in $A$ liegt, muss $f(x)$
in $A^\comp$ liegen. Die Implikaton von links nach rechts.
Gemäß Prämisse liegt $f(x)$ in $A^\comp$. Weil $x$ nur einen
Funktionswert besitzt, kann $f(x)$ nicht in $A$ liegen.\,\qedsymbol
\end{Beweis}

\begin{Satz}\label{preimg-setminus}
Für jede Abbildung $f$ gilt
$f^{-1}(A\setminus B) = f^{-1}(A)\setminus f^{-1}(B)$.
\end{Satz}
\begin{Beweis}[Beweis 1] Ergibt sich sofort gemäß Definition:
\begin{gather*}
f^{-1}(A)\setminus f^{-1}(B) = \{x\mid x\in f^{-1}(A)\land \neg x\in f^{-1}(B)\}\\
= \{x\mid f(x)\in A\land f(x)\notin B\}
= \{x\mid f(x)\in A\setminus B\}
= f^{-1}(A\setminus B).\,\qedsymbol
\end{gather*}
\end{Beweis}
\begin{Beweis}[Beweis 2] Gemäß Satz \ref{preimg-compl} und Satz
\ref{preimg-dl} (preimg-dl) ist
\begin{gather*}
f^{-1}(A)\setminus f^{-1}(B) = \{x\mid x\in f^{-1}(A)\land \neg x\in f^{-1}(B)\}\\
= \{x\mid x\in f^{-1}(A)\land x\in f^{-1}(B^\comp)\}
= f^{-1}(A)\cap f^{-1}(B^\comp)\\
= f^{-1}(A\cap B^\comp) = f^{-1}(A\setminus B).\,\qedsymbol
\end{gather*}
\end{Beweis}

\begin{Satz}\label{img-preimg}
Für jede Abbildung $f$ gilt $f(f^{-1}(N)\subseteq N$.
\end{Satz}
\begin{Beweis}
Gemäß Definition bekommt man
\begin{gather*}
y\in f(f^{-1}(N))
\iff (\exists x\colon x\in f^{-1}(N)\land y=f(x))
\iff (\exists x\colon f(x)\in N\land y=f(x)).
\end{gather*}
Leicht ersichtlich ist nun, dass
\begin{gather*}
(\exists x\colon f(x)\in N\land y=f(x)) \implies y\in N.\;\qedsymbol
\end{gather*}
\end{Beweis}

\begin{Satz}\label{img-preimg-imgset}
Für jede Abbildung $f\colon A\to B$ gilt $f(f^{-1}(N))=N$,
sofern $N\subseteq f(A)$ ist.
\end{Satz}
\begin{Beweis}
Laut Satz \ref{img-preimg} bleibt zu zeigen
\begin{gather*}
y\in N\implies (\exists x\in A\colon f(x)\in N\land y=f(x)).
\end{gather*}
Setzt man nun $N\subseteq f(A)$ voraus, dann ist $f(x)\in N$
allgemeingültig. Man bekommt
\begin{gather*}
(\exists x\in A\colon f(x)\in N\land y=f(x))
\iff (\exists x\in A\colon y=f(x))\iff y\in f(A).
\end{gather*}
Die Implikation $y\in N\implies y\in f(A)$ ist nun
wiederum definitionsgemäß äquivalent zu $N\subseteq f(A)$,
was Voraussetzung war.\;\qedsymbol
\end{Beweis}

\begin{Satz}
Für jede Abbildung $f\colon A\to B$ gilt
$(\exists M\colon f(M)=N)\iff N\subseteq f(A)$.
\end{Satz}
\begin{Beweis}
Hat man ein $M$ mit $f(M)=N$, dann ist trivialerweise
$f(M)\subseteq f(A)$, also $N\subseteq f(A)$. Liegt umgekehrt
eine Menge $N\subseteq f(A)$ vor, dann kann man $M:=f^{-1}(N)$
setzen, nach Satz \ref{img-preimg-imgset} gilt dann $f(M)=N$.\;\qedsymbol
\end{Beweis}

\begin{Satz}\label{inj-img-setminus}
Ist $f$ injektiv, dann gilt $f(A\setminus B)=f(A)\setminus f(B)$.
\end{Satz}
\begin{Beweis}
Da $f$ injektiv ist, gibt es nach Satz \ref{left-inverse} eine
Linksinverse $f^{-1}$. Nach Satz \ref{img-chain} ist für eine
beliebige Menge $M$ die Gleichung
\[f^{-1}(f(M)) = (f^{-1}\circ f)(M) = \id(M) = M\]
erfüllt. Unter Heranziehung von Satz \ref{preimg-setminus}
bekommt man
\begin{gather*}
f^{-1}(f(A)\setminus f(B)) = f^{-1}(f(A))\setminus f^{-1}(f(B))
= \id(A)\setminus\id(B) = A\setminus B.
\end{gather*}
Wendet man nun auf beide Seiten der Gleichung $f$ an, dann ergibt
sich nach Satz \ref{img-preimg-imgset} das gesuchte Resultat
$f(A)\setminus f(B) = f(A\setminus B).\;\qedsymbol$
\end{Beweis}

\begin{Satz}
Ist $f$ eine bijektive Abbildung und $f^{-1}$ die Umkehrabbildung
von $f$, dann stimmt das Urbild $f^{-1}(N)$ mit der Bildmenge
von $N$ unter der Umkehrabbildung -- zur Unterscheidung $(f^{-1})(N)$
geschrieben -- überein.
\end{Satz}
\begin{Beweis}
Expansion der Gleichung $f^{-1}(N)=(f^{-1})(N)$ führt zur Bedingung
\[f(x)\in N\iff (\exists y\colon y\in N\land x=f^{-1}(y)).\]
Da $f$ bijektiv ist, gilt $x=f^{-1}(y)\iff f(x)=f(f^{-1}(y))=y$.
Demnach ist
\[(\exists y\colon y\in N\land x=f^{-1}(y))\iff (\exists y\colon f(x)\in N)\iff f(x)\in N.\]
Die Bedingung ist daher immer erfüllt.\;\qedsymbol
\end{Beweis}
Es genügt nicht, wenn $f$ injektiv ist. Als Gegenbeispiel setze
\[f\colon\{0\}\to\{0,1\},\quad f(x):=x.\]
Hier ist $f^{-1}(\{1\})=\emptyset$. Jedoch ist
$(f^{-1})(\{1\})=\{0\}$.

\begin{Satz}[Rechtskürzbarkeit von Surjektionen]\newlinefirst
Ist $f\colon X\to Y$ eine surjektive Abbildung, dann gilt
\[g\circ f = h\circ f \implies g=h.\]
\end{Satz}
\begin{Beweis}
Laut Prämisse und Satz \ref{feq} (feq) ist $g(f(x)) = h(f(x))$
für jedes $x\in X$. Da $f$ surjektiv ist, lässt sich zu
jedem $y\in Y$ ein $x\in X$ finden, so dass $y=f(x)$. Demnach ist
$g(y)=h(y)$ für alle $y\in Y$, denn man kann immer mindestens
ein $x$ finden, so dass sich $y:=f(x)$ substituieren lässt.
Laut Satz \ref{feq} (feq) ist daher $g=h$.\;\qedsymbol
\end{Beweis}

\newpage
\subsection{Kardinalzahlen}

\begin{Axiom}[acc: abzählbares Auswahlaxiom]\label{acc}%
\index{abzählbares Auswahlaxiom}\index{Auswahlaxiom!abzählbares}
Sei $(A_n)_{n\in\N}$ eine Folge nichtleerer Mengen.
Dann existiert eine Funktion $f\colon\N\to\bigcup_{n\in\N} A_n$
mit $f(n)\in A_n$.
\end{Axiom}

\begin{Definition}[equipotent: Gleichmächtigkeit]%
\label{def:equipotent}\index{gleichmaechtig@gleichmächtig}
Zwei Mengen $A$, $B$ heißen genau dann gleichmächtig, wenn
eine Bijektion $f\colon A\to B$ existiert.
\end{Definition}

\begin{Satz}
Sei $M$ eine beliebige Menge. Die Potenzmenge $2^M$ ist zur
Menge $\{0,1\}^M$ gleichmächtig.
\end{Satz}

\begin{Beweis}
Für eine Aussage $A$ sei
\[[A] := \begin{cases}
1&\text{wenn $A$ gilt},\\
0&\text{sonst}.
\end{cases}\]
Für eine Menge $A\subseteq M$ betrachte man nun die
Indikatorfunktion\index{Indikatorfunktion}
\[1_A\colon M\to\{0,1\},\quad 1_A(x):=[x\in A].\]
Die Abbildung
\[\varphi\colon 2^M\to \{0,1\}^M,\quad \varphi(A):=1_A\]
ist eine kanonische Bijektion.

\strong{Zur Injektivität.}
Nach Def. \ref{def:inj} (inj) muss gelten:
\[\varphi(A)=\varphi(B)\implies A=B,\quad\text{d.\,h.}\quad
1_A=1_B \implies A=B.\]
Nach Satz \ref{feq} (feq) und Def. \ref{def:seteq} (seteq)
wird die Aussage expandiert zu:
\[(\forall x\colon 1_A(x)=1_B(x))\implies
(\forall x\colon x\in A\iff x\in B).\]
Es gilt aber nun:
\[1_A(x)=1_B(x)\iff [x\in A]=[x\in B] \iff (x\in A\iff x\in B).\]
\end{Beweis}
\strong{Zur Surjektivität.} Wir müssen nach Def. \ref{def:sur} (sur)
prüfen, dass $\{0,1\}^M\subseteq \varphi(2^M)$ gilt.
Expansion nach Def. \ref{def:subseteq} (subseteq) und
Def. \ref{def:img} (img)
ergibt:
\[\forall f\colon (f\in \{0,1\}^M\implies\exists A{\in}2^M\colon f=\varphi(A)).\]
Um dem Existenzquantor zu genügen, wähle
\[A := f^{-1}(\{1\}) = \{x\in M\mid f(x)\in \{1\}\} = \{x\in M\mid f(x)=1\}.\]
Es gilt $f=1_A$, denn
\[1_A(x) = [x\in A] = [x\in\{x\mid f(x)=1\}] = [f(x)=1] = f(x).\]
Da $\varphi$ eine Bijektion ist, müssen $2^M$ und $\{0,1\}^M$
nach Def. \ref{def:equipotent} (equipotent) gleichmächtig
sein.\,\qedsymbol

\newpage
\begin{Satz}\label{countable-union-countable}
Man setze Axiom \ref{acc} (acc) voraus.
Die Vereinigung von abzählbar vielen abzählbar unendlichen Mengen
ist abzählbar unendlich. Kurz $|\bigcup_{n\in\N} A_n| = |\N|$, wenn
$|A_n|=|\N|$ für jedes $n$.
\end{Satz}

\begin{Beweis}
Sei $B_n$ die Menge der Bijektionen aus $\Abb(\N,A_n)$. 
Nach Axiom \ref{acc} (acc) kann aus jeder Menge $B_n$
eine Bijektion $f_n\colon\N\to A_n$ ausgewählt werden.
Man betrachte nun
\[\varphi\colon\N\times\N\to\bigcup_{n\in\N} A_n,\quad
\varphi(n,m):=f_n(m).\]
Die Abbildung $\varphi$ ist surjektiv, denn nach
Satz \ref{img-as-cup} und Satz \ref{cup-cart} gilt
\begin{gather*}
\varphi(\N\times\N) = \bigcup_{(n,m)\in\N{\times}\N} \{f_n(m)\}
= \bigcup_{n\in\N}\bigcup_{m\in\N} \{f_n(m)\}\\
= \bigcup_{n\in\N} f_n(\bigcup_{m\in\N} \{m\})
= \bigcup_{n\in\N} f_n(\N) = \bigcup_{n\in\N} A_n.
\end{gather*}
Daher gilt $|\bigcup_{n\in\N} A_n|\le |\N\times \N| = |\N|$.
Für eine beliebige der Bijektionen $f_n\in B_n$ lässt sich die Zielmenge
erweitern, so dass man eine Injektion $f\colon\N\to\bigcup_{n\in\N} A_n$
erhält. Daher ist auch $|\N|\le |\bigcup_{n\in\N} A_n|$. Nach dem
Satz von Cantor-Bernstein gilt also
$|\bigcup_{n\in\N} A_n|=|\N|$.\,\qedsymbol
\end{Beweis}

\begin{Satz}\label{countable-polynomial-ring}
Wenn $R$ abzählbar ist, dann ist auch der Polynomring $R[X]$ abzählbar.
\end{Satz}

\begin{Beweis}
Zu jedem Polynom vom Grad $n\ge 1$ gehört auf kanonische Weise
genau ein Tupel aus $M_n:=R^{n-1}\times R\setminus\{0\}$. Da $R$
abzählbar ist, sind auch $R^{n-1}$ und $R\setminus\{0\}$ abzählbar.
Dann ist auch $M_n$ abzählbar. Nach Satz \ref{countable-union-countable}
gilt
\[|R[X]| = 1+|\bigcup_{n\in\N} M_n| = 1+|\N| = |\N|.\,\qedsymbol\]
\end{Beweis}

\begin{Satz}\index{algebraische Zahlen!Kardinalität}
Es gibt nur abzählbar unendlich viele algebraische Zahlen.
\end{Satz}

\begin{Beweis}[Beweis 1]
Zu zeigen ist $|\A|=|\N|$ mit
\[\A := \{a\in\C\mid \exists p(p\in \Q[X]\setminus\{0\}\land p(a)=0)\}.\]
Dass $\A$ unendlich ist, ist leicht ersichtlich, denn schon jede
rationale Zahl $q$, von denen es unendlich viele gibt, ist Nullstelle
von $p(X):=X-q$ und daher algebraisch.

Ein Polynom vom Grad $n$ kann höchstens $n$ Nullstellen besitzen.
Nach Satz \ref{countable-polynomial-ring} gilt $\Q[X]=|\N|$.
Für $\Q[X]$ lässt sich also eine Abzählung angeben.
Bei dieser Abzählung lässt sich für jedes Polynom $p$ die Liste der
Nullstellen von $p$ einfügen. Streicht man alle Nullstellen, die schon
einmal vorkamen, dann erhält man eine Abzählung der algebraischen
Zahlen. Demnach gilt $|\A|=|\N|$.\,\qedsymbol
\end{Beweis}

\begin{Beweis}[Beweis 2]
Jedem $p=\sum_{k=0}^n a_kX^k$ lässt sich eine Höhe
$h:=n+\sum_{k=0}^n |a_k|$ zuordnen. Zu einer festen Höhe kann es nur
endlich viele Polynome $p\in\Z[X]$ geben, wodurch man eine Abzählung der
Polynome erhält, wenn für $h=1$, $h=2$, $h=3$ usw. jeweils die Liste
der Polynome eingefügt wird. Für jedes Polynom $p$ lässt sich die
Liste der Nullstellen von $p$ einfügen. Streicht man alle Nullstellen,
die schon einmal vorkamen, dann erhält man eine Abzählung der
algebraischen Zahlen.\,\qedsymbol
\end{Beweis}

\begin{Beweis}[Beweis 3]
Für $n\in\N$ sei
\[\begin{split}
A_n := \{x\in\A\mid \;&\text{$x$ ist Nullstelle eines
$p\in\Z[X]\setminus\{0\}$ mit $\deg(p)=n$,}\\
& \text{dessen Koeffizienten $a_k$ alle $|a_k|\le n$
erfüllen}\}.
\end{split}\]
Alle $A_n$ sind endlich und es gilt $\A=\bigcup_{n\in\N} A_n$.
Daher muss $|\A|\le |\N|$ sein.\,\qedsymbol
\end{Beweis}

\begin{Definition}[Satz und Def. Multiplikation von Kardinalzahlen]\newlinefirst
Die Operation $|X|\cdot |Y|:=|X\times Y|$ ist wohldefiniert.
\end{Definition}
\begin{Beweis}
Zu zeigen ist, dass $|X\times Y|=|X'\times Y'|$
aus $|X|=|X'|$ und $|Y|=|Y'|$ folgt. Nach Voraussetzung gibt
es Bijektionen $f_1\colon X\to X'$ und $f_2\colon Y\to Y'$. Gesucht
ist mindestens eine Bijektion $f\colon X\times Y\to X'\times Y'$.
Diese erhält man gemäß folgender Konstruktion:
\[f(x,y) := (f_1(x),f_2(y)).\]
Die Abbildung $f$ ist injektiv, denn
\begin{gather*}
f(x_1,y_1)=f(x_2,y_2) \iff (f_1(x_1),f_2(y_1))=(f_1(x_2),f_2(y_2))\\
\iff f_1(x_1)=f_1(x_2)\land f_2(y_1)=f_2(y_2)\iff x_1=x_2\land y_1=y_2\\
\iff (x_1,y_1)=(x_2,y_2).
\end{gather*}
Für die Surjektivität muss es für jedes $(x',y')$ mindestens
ein $(x,y)$ mit $(x',y')=f(x,y)$ geben. Die Konstruktion ergibt
\[(x',y')=(f_1(x),f_2(y))\iff x'=f_1(x)\land y'=f_2(y).\]
Man findet $x=f_1^{-1}(x')$ und $y=f_2^{-1}(y')$.

Die Umkehrabbildung ist gegeben gemäß
\begin{gather*}
f^{-1}(x',y')=f^{-1}((x',y')):=((f_1^{-1}\circ\pi_1)(x',y'),(f_2^{-1}\circ\pi_2)(x',y'))\\
= (f_1^{-1}(x'),f_2^{-1}(y')).
\end{gather*}
Mit $\pi_k$ ist die Projektion auf die $k$-te Komponente gemeint.\;\qedsymbol
\end{Beweis}

\begin{Definition}[Satz und Def. Addition von Kardinalzahlen]\newlinefirst
Für $X\cap Y=\emptyset$ ist $|X|+|Y|:=|X\cup Y|$ wohldefiniert.
Das schließt den Spezialfall $|X|+|Y|:=|X\sqcup Y|$ mit
$X\sqcup Y:=(\{0\}\times X)\cup(\{1\}\times Y)$ ein.
\end{Definition}
\begin{Beweis}
Zu zeigen ist, dass $|X\cup Y|=|X'\cup Y'|$
aus $|X|=|X'|$ und $|Y|=|Y'|$ folgt. Nach Voraussetzung gibt
es Bijektionen $f_1\colon X\to X'$ und $f_2\colon Y\to Y'$, wobei
$X\cap Y=\emptyset$ und $X'\cap Y'=\emptyset$ gilt. Gesucht
ist mindestens eine Bijektion $f\colon X\cup Y\to X'\cup Y'$.
Diese erhält man gemäß folgender Konstruktion:
\[f(x):=\begin{cases}
f_1(x)&\text{für}\;x\in X,\\
f_2(x)&\text{für}\;x\in Y.
\end{cases}\]
Die Abbildung $f$ ist injektiv, denn entweder ist $x'\in X'$
und somit
\[x'=f(a)=f(b)\iff x'=f_1(a)=f_1(b)\iff a=b\]
oder $x'\in Y'$ und somit
\[x'=f(a)=f(b)\iff x'=f_2(a)=f_2(b)\iff a=b.\]
Zusammengefasst folgt $f(a)=f(b)\iff a=b$ für alle $a,b\in X\cup Y$.

Für die Surjektivität muss es für jedes $x'$ mindestens ein $x$ mit
$x'=f(x)$ geben. Entweder ist $x'\in X'$, dann ist $x'=f_1(x)$
und daher $x=f_1^{-1}(x')$. Oder es ist $x'\in Y'$, dann ist
$x'=f_2(x)$ und daher $x=f_2^{-1}(x')$.\;\qedsymbol
\end{Beweis}

\newpage
\begin{Definition}[Satz und Def. Potenz von Kardinalzahlen]\newlinefirst
Die Operation $|Y|^{|X|}:=|Y^X|$ ist wohldefiniert.
\end{Definition}
\begin{Beweis}
Zu zeigen ist, dass $|\Abb(X,Y)|=|\Abb(X',Y')|$
aus $|X|=|X'|$ und $|Y|=|Y'|$ folgt. Nach Voraussetzung gibt
es Bijektionen $f_1\colon X\to X'$ und $f_2\colon Y\to Y'$.
Gesucht ist eine Bijektion $F\colon\Abb(X,Y)\to\Abb(X',Y')$.
Diese erhält man gemäß folgender Konstruktion:
\[F(f) := f_2\circ f\circ f_1^{-1}.\]
Die Abbildung $F$ ist injektiv, da
\begin{gather*}
F(f)=F(g) \iff f_2\circ f\circ f_1^{-1} = f_2\circ f\circ f_1^{-1}
\iff f_2\circ f = f_2\circ g\iff f=g,
\end{gather*}
denn Bijektionen sind kürzbar. Für die Surjektivität muss
es für jedes $f'$ mindestens ein $f$ mit $f'=F(f)$ geben.
Das führt auf die Gleichung $f'=f_2\circ f\circ f_1^{-1}$.
Diese lässt sich Umformen zu $f_2^{-1}\circ f'=f\circ f_1^{-1}$.
Wendet man beide Seiten auf $f_1$ an, ergibt sich
$f=f_2^{-1}\circ f'\circ f_1$.\;\qedsymbol
\end{Beweis}

\begin{Definition}[Weniger mächtig]\newlinefirst
Eine Menge $A$ heißt weniger mächtig als eine Menge $B$, kurz
$|A|<|B|$, wenn es eine Injektion $A\to B$ gibt, aber keine
Bijektion $A\to B$.
\end{Definition}

\begin{Satz}[Satz von Cantor]\newlinefirst
Eine Menge ist stets weniger mächtig als ihre Potenzmenge. Kurz $|A|<|2^A|$. 
\end{Satz}

\begin{Beweis}
Eine Injektion $A\to 2^A$ können wir mit $x\mapsto\{x\}$ trivial angeben.
Nun wird die Annahme, es gäbe eine Surjektion $f\colon A\to 2^A$
zum Widerspruch geführt, womit es erst recht keine Bijektion gibt.
Dazu definiert man die Menge
\[D := \{x\in A\mid x\notin f(x)\}.\]
Weil $D\subseteq A$ ist, hat man $D\in 2^A$. Weil $f$ surjektiv sein
soll, muss es ein $x\in A$ geben, so dass $f(x) = D$. Gemäß Def.
\ref{def:seteq} (seteq) und Definition von $D$ muss also
\[\exists x\in A\colon \forall u\colon (u\in f(x) \iff u\in A\land u\notin f(u))\]
gelten. Die allquantifizierte Aussage ist allerdings für $u:=x$ nicht erfüllt,
denn unter Berücksichtigung $x\in A$ gelangt man zur widersprüchlichen
Aussage
\[x\in f(x) \iff x\notin f(x).\,\qedsymbol\]
\end{Beweis}

\begin{Satz}
Die Menge der endlichen Teilmengen der natürlichen Zahlen ist
abzählbar.
\end{Satz}

\begin{Beweis}
Zu jeder Teilmenge $A\subseteq\N_0$ gehört genau eine Indikatorfunktion
\[1_A\colon\N_0\to\{0,1\},\quad 1_A(n) := [n\in A].\]
Weil die Indikatorfunktion die natürlichen Zahlen als Definitionsbereich
besitzt, handelt es sich um eine Folge, die wie jede Folge als formale Potenzreihe.
\[p_A(X) = \sum_{n=0}^\infty 1_A(n)X^n = \sum_{n\in A} X^n\]
dargestellt werden kann. Ist die Teilmenge $A$ eine endliche, besitzt
die Indiktorfunktion eine endliche Nichtnullstellenmenge, womit sich
$p_A(X)$ zu einem Polynom reduziert. Wegen $1_A(n)<2$, kann man
$1_A$ als Kodierung der Zahl $p_A(2)$ auffassen, nämlich als Binärdarstellung
der Zahl $p_A(2)$ im Stellenwertsystem zur Basis $X=2$. Deshalb fungiert
die Abbildung
\[f(A) := p_A(2) = \sum_{n\in A} 2^n\]
als kanonische Bijektion zwischen der Menge der endlichen Teilmengen der
natürlichen Zahlen und der natürlichen Zahlen. Sie ist injektiv
aufgrund der Eigenheiten von Stellenwertsystemen. Sie ist surjektiv,
weil $1_A$ laut Prämisse jede beliebige Binärdarstellung sein darf,
und man somit aufgrund der Eigenheiten von Stellenwertsystemen jede
beliebige natürliche Zahl erhält.\,\qedsymbol
\end{Beweis}
