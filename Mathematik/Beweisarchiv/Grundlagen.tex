
\chapter{Grundlagen}
\section{Logisches Schließen}

\subsection{Natürliches Schließen}

\subsubsection{Logische Schlussregeln}
\begin{tabular}{lc@{\qquad}c}
\toprule
& \strong{Einführung} & \strong{Beseitigung}\\
\toprule
\strong{Konjunktion}
& $\dfrac{\Gamma\vdash A\qquad\Gamma'\vdash B}{\Gamma, \Gamma'\vdash A\land B}$
& $\dfrac{\Gamma\vdash A\land B}{\Gamma\vdash A}\quad\dfrac{\Gamma\vdash A\land B}{\Gamma\vdash B}$\\[14pt]
\strong{Disjunktion}
& $\dfrac{\Gamma\vdash A}{\Gamma\vdash A\lor B}\quad\dfrac{\Gamma\vdash B}{\Gamma\vdash A\lor B}$
& $\dfrac{\Gamma\vdash A\lor B\qquad\Gamma',A\vdash C\qquad\Gamma'',B\vdash C}{\Gamma,\Gamma',\Gamma''\vdash C}$\\[14pt]
\strong{Implikation}
& $\dfrac{\Gamma,A\vdash B}{\Gamma\vdash A\Rightarrow B}$
& $\dfrac{\Gamma\vdash A\Rightarrow B\qquad\Gamma'\vdash A}{\Gamma,\Gamma'\vdash B}$\\[14pt]
\strong{Negation}
& $\dfrac{\Gamma,A\vdash\bot}{\Gamma\vdash\neg A}$
& $\dfrac{\Gamma\vdash \neg A\qquad\Gamma'\vdash A}{\Gamma,\Gamma'\vdash\bot}$\\[14pt]
\strong{Äquivalenz}
& $\dfrac{\Gamma\vdash A\Rightarrow B\qquad\Gamma'\vdash B\Rightarrow A}{\Gamma,\Gamma'\vdash A\Leftrightarrow B}$
& $\dfrac{\Gamma\vdash A\Leftrightarrow B}{\Gamma\vdash A\Rightarrow B}\quad
   \dfrac{\Gamma\vdash A\Leftrightarrow B}{\Gamma\vdash B\Rightarrow A}$\\[14pt]
\strong{Universalq.}
& $\dfrac{\Gamma\vdash A}{\Gamma\vdash\forall x\colon A}(x\notin\mathrm{FV}(\Gamma))$
& $\dfrac{\Gamma\vdash\forall x\colon A}{\Gamma\vdash A[x:=t]}$\\[14pt]
\strong{Existenzq.}
& $\dfrac{\Gamma\vdash A[x:=t]}{\Gamma\vdash\exists x\colon A}$
& $\!\!\!\!\dfrac{\Gamma\vdash\exists x\colon A\qquad\Gamma', A\vdash B}{\Gamma,\Gamma'\vdash B}
(x\notin\mathrm{FV}(\Gamma,\Gamma',B))\!\!$\\[8pt]
\bottomrule
\end{tabular}

\subsubsection{Strukturelle Schlussregeln}
\begin{tabular}{ccc}
\toprule
\strong{Abschwächung} & \strong{Vertauschung} & \strong{Kontraktion}\\[2pt]
$\dfrac{\Gamma,\Gamma'\vdash B}{\Gamma,A,\Gamma'\vdash B}$
& $\dfrac{\Gamma,A,B,\Gamma'\vdash C}{\Gamma,B,A,\Gamma'\vdash C}$
& $\dfrac{\Gamma,A,A,\Gamma'\vdash B}{\Gamma,A,\Gamma'\vdash B}$\\[8pt]
\bottomrule
\end{tabular}

\newpage
\subsubsection{Logische Axiome}

\begin{Axiom}
Es gilt $\Gamma,A,\Gamma'\vdash A$.
\end{Axiom}

\begin{Axiom}
Es gilt $\Gamma\vdash\top$.
\end{Axiom}

\begin{Axiom}[EFQ: Ex falso quodlibet]%
\label{EFQ}\index{Ex falso quodlibet}
Eine falsche Aussage impliziert jede beliebige Aussage, kurz
\[\vdash \bot\Rightarrow A.\]
\end{Axiom}
Bemerkung: Dieses Prinzip erlaubt Programmterme mit leerem
Pattern matching, so dass ein Zeuge für $0\to A$ konstruiert
werden kann.

\begin{Axiom}[LEM: Satz vom ausgeschlossenen Dritten]%
\label{LEM}\index{Satz vom ausgeschlossenen Dritten}\newlinefirst
Entweder gilt eine Aussage, oder ihre Negation gilt, kurz
\[\vdash A\lor\neg A.\]
\end{Axiom}
Bemerkung: Zur Schaffung von Klarheit sollte ein Beweis die Markierung
LEM bekommen, wenn transitive Abhängigkeit zu diesem Axiom besteht.
Verzichtet keiner der Beweise eines Satzes auf LEM, sollte der Satz
ebenfalls mit LEM markiert werden.

\begin{Axiom}[DN: Beseitigung der Doppelnegation]\label{DNE}\newlinefirst
Die Doppelnegation einer Aussage $A$ impliziert die Aussage $A$, kurz
\[\vdash \neg\neg A\Rightarrow A.\]
\end{Axiom}

\subsubsection{Axiome zur Gleichheit}
\begin{Axiom}[Reflexivität]
Es gilt $\vdash x=x$.
\end{Axiom}

\begin{Axiom}[Symmetrie]
Es gilt $\vdash x=y\Rightarrow y=x$. Infolge gilt die Schlussregel
\[\dfrac{\Gamma\vdash x=y}{\Gamma\vdash y=x}.\]
\end{Axiom}

\begin{Axiom}[Transitivität]
Es gilt $\vdash x=y\land y=z\Rightarrow x=z$. Infolge gilt
\[\dfrac{\Gamma\vdash x=y\qquad\Gamma'\vdash y=z}{\Gamma,\Gamma'\vdash x=z}.\]
\end{Axiom}

\begin{Axiom}
Es gilt $\vdash x=y\land P(x)\Rightarrow P(y)$. Infolge gilt
\[\dfrac{\Gamma\vdash x=y\qquad\Gamma'\vdash P(x)}{\Gamma,\Gamma'\vdash P(y)}.\]
\end{Axiom}

\begin{Axiom}
Es gilt $\vdash x=y\Rightarrow f(x)=f(y)$. Infolge gilt
\[\dfrac{\Gamma\vdash x=y}{\Gamma\vdash f(x)=f(y)}.\]
\end{Axiom}

\newpage
\subsection{Gesetzmäßigkeiten}
\begin{Satz}[Deduktionstheorem für Sequenzen]%
\label{sequent-dt}\index{Deduktionstheorem}\newlinefirst
Es gilt $\Gamma,A\vdash B$ genau dann, wenn $\Gamma\vdash A\Rightarrow B$.
\end{Satz}
\begin{Beweis} Es findet sich:
\[\begin{array}{@{}l@{\qquad}l}
\infer{\Gamma\vdash A\Rightarrow B}{\Gamma,A\vdash B}
&
\infer{\Gamma,A\vdash B}{\Gamma\vdash A\Rightarrow B & \infer{A\vdash A}{}}
\;\qedsymbol
\end{array}
\]
\end{Beweis}

\begin{Satz}\label{eq-impl-eq-sequents}
Sofern $\vdash A\Leftrightarrow B$ gilt,
ist $\Gamma\vdash A$ äquivalent zu $\Gamma\vdash B$.
\end{Satz}
\begin{Beweis} Es findet sich:
\[\begin{array}{@{}l@{\qquad}l}
\infer{\Gamma\vdash B}{\Gamma\vdash A &
  \infer{\vdash A\Rightarrow B}{\vdash A\Leftrightarrow B}}
&
\infer{\Gamma\vdash A}{\Gamma\vdash B &
  \infer{\vdash B\Rightarrow A}{\vdash A\Leftrightarrow B}}\;\qedsymbol
\end{array}
\]
\end{Beweis}

\begin{Satz}
Die Sequenz $\Gamma, A\land B\vdash C$ gilt genau dann, wenn $\Gamma,A,B\vdash C$.
\end{Satz}
\begin{Beweis}[Beweis 1]
Laut den Beweisbäumen
\[
\begin{array}{@{}l@{\qquad}l@{}}
\infer{\Gamma,A,B\vdash C}{
  \infer{\Gamma\vdash A\land B\cond C}{
    \Gamma,A\land B\vdash C}
& \infer{A, B\vdash A\land B}{
    \infer{A\vdash A}{}
  & \infer{B\vdash B}{}}}
&
\infer{\Gamma,A\land B\vdash C}{
  \infer{\Gamma,A\land B\vdash B\cond C}{
    \infer{\Gamma\vdash A\cond (B\cond C)}{
      \infer{\Gamma,A\vdash B\cond C}{
        \Gamma,A,B\vdash C}}
  & \infer{A\land B\vdash A}{
      \infer{A\land B\vdash A\land B}{}}}
& \infer{A\land B\vdash B}{
    \infer{A\land B\vdash A\land B}{}}}
\end{array}
\]
gehen die beiden Sequenzen gegenseitig aus sich hervor.\,\qedsymbol
\end{Beweis}

\begin{Beweis}[Beweis 2]
Es gilt
\begin{align*}
& (\Gamma, A\land B\vdash C)\iff (\Gamma\vdash A\land B\Rightarrow C),
  &&\text{(Satz \ref{sequent-dt})}\\
& (\Gamma,A,B\vdash C)\iff (\Gamma\vdash A\Rightarrow B\Rightarrow C),
  &&\text{(Satz \ref{sequent-dt})}\\
& {\vdash (A\land B\Rightarrow C)\Leftrightarrow (A\Rightarrow B\Rightarrow C)}.
  &&\text{(Satz \ref{curry-impl})}
\end{align*}
Mit Satz \ref{eq-impl-eq-sequents} folgt somit die Behauptung.\,\qedsymbol
\end{Beweis}

\begin{Satz}[Sequenzen erzeugen Schlussregeln]\newlinefirst
Gilt $A\vdash B$, dann darf von $\Gamma\vdash A$ auf $\Gamma\vdash B$
geschlossen werden.
\end{Satz}
\begin{Beweis} Es findet sich:
\[\infer{\Gamma\vdash B}{
  \infer{\vdash A\Rightarrow B}{A\vdash B}
  & \Gamma\vdash A}\;\qedsymbol\]
\end{Beweis}

\begin{Satz}[Sequenzen erzeugen Schlussregeln]\newlinefirst
Gilt $A_1,\ldots,A_n\vdash B$, dann darf auf $\Gamma\vdash B$
geschlossen werden, falls $\Gamma\vdash A_k$ für jedes $k$
von $k=1$ bis $k=n$ gilt.
\end{Satz}
\begin{Beweis} Für $n=2$ findet sich sich:
\[\infer{\Gamma\vdash B}{
  \infer{\Gamma\vdash A_2\Rightarrow B}{
    \infer{\vdash A_1\Rightarrow A_2\Rightarrow B}{
      \infer{A_1\vdash A_2\Rightarrow B}{A_1,A_2\vdash B}}
    & \Gamma\vdash A_1}
  & \Gamma\vdash A_2}\]
Für höhere $n$ verläuft der Beweis analog.\,\qedsymbol
\end{Beweis}

\newpage
\begin{Satz}[Kontraposition, Modus tollens]%
\label{modus-tollens}\index{Kontraposition}\index{Modus tollens}
Es gelten die Schlussregeln
\[\dfrac{\Gamma\vdash A\Rightarrow B}{\Gamma\vdash\neg B\Rightarrow\neg A},\qquad
\dfrac{\Gamma\vdash A\Rightarrow B\qquad\Gamma'\vdash\neg B}{\Gamma,\Gamma'\vdash\neg A}.\]
\end{Satz}
\begin{Beweis} Es findet sich:
\[\begin{array}{@{}l@{\qquad\quad}l}
\infer{\Gamma\vdash\neg B\Rightarrow\neg A}{
  \infer{\Gamma,\neg B\vdash\neg A}{
    \infer{\Gamma,\neg B, A\vdash\bot}{
      \infer{\neg B\vdash\neg B}{}
      & \infer{\Gamma,A\vdash B}{
          \Gamma\vdash A\Rightarrow B
          & \infer{A\vdash A}{}}}}}
&
\infer{\Gamma,\Gamma'\vdash\neg A}{
  \infer{\Gamma\vdash\neg B\Rightarrow\neg A}{\Gamma\vdash A\Rightarrow B}
  & \Gamma'\vdash\neg B}\;\qedsymbol
\end{array}
\]
\end{Beweis}

\begin{Satz}[Modus tollendo ponens]\index{Modus tollendo ponens}
Es gilt die Schlussregel
\[\dfrac{\Gamma\vdash A\lor B\qquad\Gamma'\vdash \neg A}{\Gamma,\Gamma'\vdash B}.\]
\end{Satz}
\begin{Beweis}
Es findet sich:
\[\infer{\Gamma,\Gamma'\vdash B}{
  \Gamma\vdash A\lor B
  & \infer[\infernote{EFQ}]{\Gamma',A\vdash B}{
      \infer{\Gamma',A\vdash\bot}{
        \Gamma'\vdash\neg A & \infer{A\vdash A}{}}}
  & \infer{B\vdash B}{}}\]
Mit EFQ ist Axiom \ref{EFQ} gemeint.\,\qedsymbol
\end{Beweis}

\begin{Satz}[Kettenschluss]\index{Kettenschluss}
Es gilt die Schlussregel
\[\dfrac{\Gamma\vdash A\Rightarrow B\qquad\Gamma'\vdash B\Rightarrow C}
{\Gamma,\Gamma'\vdash A\Rightarrow C}.\]
\end{Satz}
\begin{Beweis} Es findet sich:
\[\infer{\Gamma,\Gamma'\vdash A\Rightarrow C}{
  \infer{\Gamma,\Gamma',A\vdash C}{
    \Gamma'\vdash B\Rightarrow C &
      \infer{\Gamma,A\vdash B}{
        \Gamma\vdash A\Rightarrow B
        & \infer{A\vdash A}{}}}}\;\qedsymbol
\]
\end{Beweis}

\begin{Satz}[Klassische Reductio ad absurdum]%
\index{Reductio ad absurdum} \strong{[LEM]}\\
Es gilt die Schlussregel
\[\dfrac{\Gamma,\neg A\vdash\bot}{\Gamma\vdash A}.\]
\end{Satz}
\begin{Beweis} Es findet sich:
\[\infer[\infernote{DN}]{\Gamma\vdash A}{
  \infer{\Gamma\vdash\neg\neg A}{
    \Gamma,\neg A\vdash\bot}}
\]
Die genutzte Regel DN gilt lediglich klassisch.
Zu DN aus LEM siehe Satz \ref{LEM-EFQ-implies-DN}.\,\qedsymbol
\end{Beweis}

\begin{Satz}[Konstruktives Dilemma]\label{destructive-dilemma}
Es gilt die Schlussregel
\[\dfrac{\Gamma\vdash A\lor C\qquad
\Gamma'\vdash A\Rightarrow B\qquad\Gamma''\vdash C\Rightarrow D}{
\Gamma,\Gamma',\Gamma''\vdash B\lor D}.\]
\end{Satz}
\begin{Beweis}
Es findet sich:
\[\infer{\Gamma,\Gamma',\Gamma''\vdash B\lor D}{
  \Gamma\vdash A\lor C
  & \infer{\Gamma',A\vdash B\lor D}{
      \infer{\Gamma',A\vdash B}{
        \Gamma'\vdash A\Rightarrow B
        & \infer{A\vdash A}{}}}
  & \infer{\Gamma'',C\vdash B\lor D}{
      \infer{\Gamma'',C\vdash D}{
        \Gamma''\vdash C\Rightarrow D
        & \infer{C\vdash C}{}}}}\;\qedsymbol
\]
\end{Beweis}

\begin{Satz}[Destruktives Dilemma]
Es gilt die Schlussregel
\[\dfrac{\Gamma\vdash \neg B\lor \neg D\qquad
\Gamma'\vdash A\Rightarrow B\qquad\Gamma''\vdash C\Rightarrow D}{
\Gamma,\Gamma',\Gamma''\vdash \neg A\lor \neg C}.\]
\end{Satz}
\begin{Beweis}
Es findet sich:
\[
\infer[\infernote{(3)}]{\Gamma,\Gamma',\Gamma''\vdash \neg A\lor \neg C}{
  \Gamma\vdash \neg B\lor \neg D
  & \infer[\infernote{(1)}]{\Gamma'\vdash\neg B\Rightarrow\neg A}{
      \Gamma'\vdash A\Rightarrow B}
  & \infer[\infernote{(2)}]{\Gamma''\vdash\neg D\Rightarrow\neg C}{
      \Gamma''\vdash C\Rightarrow D}}
\]
Hierbei gilt (1), (2) gemäß Satz \ref{modus-tollens} und (3) gemäß Satz
\ref{destructive-dilemma}.\,\qedsymbol
\end{Beweis}

\begin{Satz} Die beiden Schlussregeln
\[\dfrac{\Gamma\vdash x=y}{\Gamma\vdash P(x)\Leftrightarrow P(y)},\qquad
\dfrac{\Gamma\vdash x=y\qquad\Gamma'\vdash P(x)}{\Gamma,\Gamma'\vdash P(y)}\]
sind äquivalent.
\end{Satz}
\begin{Beweis}
Sie gehen aus sich hervor:
\[
\begin{array}{@{}l@{\qquad}l}
\infer{\Gamma,\Gamma'\vdash P(y)}{
  \infer{\Gamma\vdash P(x)\Rightarrow P(y)}{
    \infer{\Gamma\vdash P(x)\Leftrightarrow P(y)}{\Gamma\vdash x=y}}
& \Gamma'\vdash P(x)}
&
\infer{\Gamma\vdash P(x)\Leftrightarrow P(y)}{
  \infer{\Gamma\vdash P(x)\Rightarrow P(y)}{
    \infer{\Gamma, P(x)\vdash P(y)}{
      \Gamma\vdash x=y & \infer{P(x)\vdash P(x)}{}}}
& \infer{\Gamma\vdash P(y)\Rightarrow P(x)}{
    \infer{\Gamma,P(y)\vdash P(x)}{
      \Gamma\vdash x=y & \infer{P(y)\vdash P(y)}{}}}}\;\qedsymbol
\end{array}
\]
\end{Beweis}

\begin{Satz}\label{LEM-EFQ-implies-DN}
Die Axiome \ref{LEM} (LEM), \ref{EFQ} (EFQ)
implizieren Axiom \ref{DNE} (DN).
\end{Satz}
\begin{Beweis} Es findet sich:
\[
\infer{\neg\neg A\vdash A}{
  \infer{\vdash\neg A\lor A}{}
& \infer[\infernote{EFQ}]{\neg\neg A,\neg A\vdash A}{
    \infer{\neg\neg A,\neg A\vdash\bot}{
      \infer{\neg\neg A\vdash\neg\neg A}{}
    & \infer{\neg A\vdash\neg A}{}}}
& \infer{A\vdash A}{}}
\]
Zu $\vdash \neg A \lor A\Rightarrow (\neg\neg A \Rightarrow A)$.
Gemäß Axiom \ref{EFQ} (EFQ) existiert ein Zeuge $\operatorname{exfq}(A)$
für $0\to A$. Damit lässt sich der Programmterm
\[(A\to 0) + A \to ((A\to 0)\to 0)\to A,\quad
s\mapsto\match s\begin{cases}
\inl f\mapsto g\mapsto \operatorname{exfq}(A)(g(f)),\\
\inr a\mapsto g\mapsto a.
\end{cases}\]
konstruieren.\;\qedsymbol
\end{Beweis}

\newpage
\begin{Satz}[Resolution]
Es gilt die Schlussregel
\[\dfrac{\Gamma\vdash A\lor B\qquad\Gamma'\vdash\neg A\lor C}{\Gamma,\Gamma'\vdash B\lor C}.\]
\end{Satz}
\begin{Beweis}
Der Beweisbaum:
\[
\infer{\Gamma,\Gamma'\vdash B\lor C}{
  \Gamma\vdash A\lor B
& \infer{\Gamma', A\vdash B\lor C}{
    \infer{\Gamma', A\vdash C}{
      \Gamma'\vdash\neg A\lor C
    & \infer{A,\neg A\vdash C}{
        \infer{A,\neg A\vdash\bot}{
          \infer{\neg A\vdash \neg A}{}
        & \infer{A\vdash A}{}}}
    & \infer{C\vdash C}{}}}
& \infer{B\vdash B\lor C}{\infer{B\vdash B}{}}}\;\qedsymbol
\]
\end{Beweis}

\begin{Satz} \strong{[LEM]}
Es gilt die Schlussregel
\[\dfrac{\Gamma,\neg A\vdash B\qquad\Gamma', A\vdash B}{\Gamma,\Gamma'\vdash B}.\]
\end{Satz}
\begin{Beweis}
Es findet sich:
\[\infer{\Gamma,\Gamma'\vdash B}{
  \infer[\infernote{LEM}]{\vdash A\lor\neg A}{}
& \Gamma,\neg A\vdash B
& \Gamma', A\vdash B
}\;\qedsymbol\]
\end{Beweis}

\begin{Satz}
Die beiden Schlussregeln
\[\dfrac{\Gamma\vdash A\land B\qquad\Gamma',A\vdash C}{\Gamma,\Gamma'\vdash C},\qquad
\dfrac{\Gamma\vdash A\land B\qquad\Gamma',B\vdash C}{\Gamma,\Gamma'\vdash C}\]
sind zulässig.
\end{Satz}
\begin{Beweis}
Es findet sich:
\[
\begin{array}{@{}l@{\qquad\quad}l}
\infer{\Gamma,\Gamma'\vdash C}{
  \infer{\Gamma\vdash A}{\Gamma\vdash A\land B}
& \infer{\Gamma'\vdash A\Rightarrow C}{\Gamma',A\vdash C}}
&
\infer{\Gamma,\Gamma'\vdash C}{
  \infer{\Gamma\vdash B}{\Gamma\vdash A\land B}
& \infer{\Gamma'\vdash B\Rightarrow C}{\Gamma',B\vdash C}}\;\qedsymbol
\end{array}
\]
\end{Beweis}

\begin{Satz}
Die Schlussregel
\[\dfrac{\Gamma,A\vdash C\qquad\Gamma',B\vdash C}{
\Gamma,\Gamma',A\lor B\vdash C}\]
ist zulässig.
\end{Satz}
\begin{Beweis}
Es findet sich:
\[
\infer{\Gamma,\Gamma',A\lor B\vdash C}{
  \infer{A\lor B\vdash A\lor B}{}
& \Gamma,A\vdash C
& \Gamma',B\vdash C}\;\qedsymbol
\]
\end{Beweis}

\newpage
\begin{Satz}[Ersetzungsregel]\index{Ersetzungsregel}
Es gelten die beiden äquivalenten Regeln
\[\dfrac{\Gamma\vdash A\Leftrightarrow B}{\Gamma\vdash P(A)\Leftrightarrow P(B)},
\qquad\dfrac{\Gamma\vdash A\Leftrightarrow B\qquad\Gamma'\vdash P(A)}{\Gamma,\Gamma'\vdash P(B)}.
\]
\end{Satz}
\begin{Beweis}
Zur Äquivalenz der beiden Regeln:
\[
\begin{array}{@{}l@{\qquad}l}
\infer{\Gamma,\Gamma'\vdash P(B)}{
  \infer{\Gamma\vdash P(A)\Rightarrow P(B)}{
    \infer{\Gamma\vdash P(A)\Leftrightarrow P(B)}{
      \Gamma\vdash A\Leftrightarrow B}}
& \Gamma'\vdash P(A)}
&
\infer{\Gamma\vdash P(A)\Leftrightarrow P(B)}{
  \infer{\Gamma\vdash P(A)\Rightarrow P(B)}{
    \infer{\Gamma, P(A)\vdash P(B)}{
      \Gamma\vdash A\Leftrightarrow B
    & \infer{P(A)\vdash P(A)}{}}}
& \infer{\Gamma\vdash P(B)\Rightarrow P(A)}{
    \infer{\Gamma, P(B)\vdash P(A)}{
      \Gamma\vdash A\Leftrightarrow B
    & \infer{P(B)\vdash P(B)}{}}}}
\end{array}
\]
Wir führen nun eine strukturelle Induktion über den Formelaufbau durch.
Ist $P(X):=\varphi$ bezüglich der atomaren logischen Variable $X$
definiert, gilt $P(t):=\varphi[X{:=}t]$, wobei mit $\varphi[X{:=}t]$
die Substitution jedes freien Vorkommens von $X$ durch die Formel $t$
gemeint ist. Die Behauptung wird in der Form
\[\dfrac{\Gamma\vdash t\Leftrightarrow t'}{
\Gamma\vdash\varphi[X{:=}t]\Leftrightarrow\varphi[X{:=}t']}\]
geschrieben und es werden die Abkürzungen
\[A:=\varphi[X{:=}t],\quad A':=\varphi[X{:=}t'],\quad B:=\psi[X{:=}t],
\quad B':=\psi[X{:=}t']\]
definiert. Zunächst die Basisfälle. Die Formeln $\varphi:=\bot$,
$\varphi:=\top$ und $\varphi:=v$ mit atomarer Variable $v\ne X$
bleiben von der Substitution unbetroffen und sind offenkundig zu sich
selbst äquivalent. Für $v=X$ erhält man schlicht die Prämisse.

Zum Induktionsschritt. Man hat nun
\[(\varphi\land\psi)[X{:=}t]
\equiv \varphi[X{:=}t]\land\psi[X{:=}t]\equiv A\land B\]
usw. Induktionsvoraussetzung sei also $\Gamma\vdash A\Leftrightarrow A'$
und $\Gamma\vdash B\Leftrightarrow B'$. Zu zeigen ist,
dass dann ebenfalls
\begin{align*}
& \Gamma\vdash A\land B \Leftrightarrow A'\land B',
&& \Gamma\vdash (A\Rightarrow B) \Leftrightarrow (A'\Rightarrow B'),
&& \Gamma\vdash (\forall x\colon A)\Leftrightarrow (\forall x\colon A'),\\
& \Gamma\vdash A\lor B \Leftrightarrow A'\lor B',
&& \Gamma\vdash (A\Leftrightarrow B)\Leftrightarrow (A'\Leftrightarrow B'),
&& \Gamma\vdash (\exists x\colon A)\Leftrightarrow (\exists x\colon A')
\end{align*}
und $\Gamma\vdash\neg A\Leftrightarrow\neg A'$ gilt. Zur Negation:
\[\infer{\Gamma\vdash\neg A\Leftrightarrow\neg A'}{
  \infer{\Gamma\vdash\neg A\Rightarrow\neg A'}{
    \infer{\Gamma,\neg A\vdash\neg A'}{
      \infer{\Gamma\vdash\neg A\Rightarrow\neg A'}{
        \infer{\Gamma\vdash A'\Rightarrow A}{
          \Gamma\vdash A\Leftrightarrow A'}}
    & \infer{\neg A\vdash\neg A}{}}}
& \infer{\Gamma\vdash\neg A'\Rightarrow\neg A}{
    \infer{\Gamma,\neg A'\vdash\neg A}{
      \infer{\Gamma\vdash\neg A'\Rightarrow\neg A}{
        \infer{\Gamma\vdash A\Rightarrow A'}{
          \Gamma\vdash A\Leftrightarrow A'}}
    & \infer{\neg A'\vdash\neg A'}{}}}}
\]
Zur Konjunktion:
\[
\infer{\Gamma\vdash A\land B\Leftrightarrow A'\land B'}{
  \infer{\Gamma\vdash A\land B\Rightarrow A'\land B'}{
    \infer{\Gamma,A\land B\vdash A'\land B'}{
      \infer{\Gamma,A\land B\vdash A'}{
        \infer{\Gamma\vdash A\Rightarrow A'}{
          \Gamma\vdash A\Leftrightarrow A'}
      & \infer{A\land B\vdash A}{
          \infer{A\land B\vdash A\land B}{}}}
    & \infer{\Gamma,A\land B\vdash B'}{
        \infer{\Gamma\vdash B\Rightarrow B'}{
          \Gamma\vdash B\Leftrightarrow B'}
      & \infer{A\land B\vdash B}{
          \infer{A\land B\vdash A\land B}{}}}}}
& \infer{\Gamma\vdash A'\land B'\Rightarrow A\land B}{\text{analog}}}
\]
Zur Disjunktion:
\[
\infer{\Gamma\vdash A\lor B\Leftrightarrow A'\lor B'}{
& \infer{\Gamma\vdash A\lor B\Rightarrow A'\lor B'}{
    \infer{\Gamma, A\lor B\vdash A'\lor B'}{
      \infer{A\lor B\vdash A\lor B}{}
    & \infer{\Gamma, A\vdash A'\lor B'}{
        \infer{\Gamma, A\vdash A'}{
          \infer{\Gamma\vdash A\Rightarrow A'}{
            \Gamma\vdash A\Leftrightarrow A'}
        & \infer{A\vdash A}{}}}
    & \infer{\Gamma, B\vdash A'\lor B'}{
        \infer{\Gamma, B\vdash B'}{
          \infer{\Gamma\vdash B\Rightarrow B'}{
            \Gamma\vdash B\Leftrightarrow B'}
        & \infer{B\vdash B}{}}}}}
& \infer{\Gamma\vdash A'\lor B'\Rightarrow A\lor B}{\text{analog}}}
\]
Zur Implikation:
\[
\infer{\Gamma\vdash (A\Rightarrow B)\Leftrightarrow (A'\Rightarrow B')}{
  \infer{\Gamma\vdash (A\Rightarrow B)\Rightarrow (A'\Rightarrow B')}{
    \infer{\Gamma,A\Rightarrow B\vdash A'\Rightarrow B'}{
      \infer{\Gamma,A\Rightarrow B, A'\vdash B'}{
        \infer{\Gamma\vdash B\Rightarrow B'}{
          \Gamma\vdash B\Leftrightarrow B'}
      & \infer{\Gamma,A\Rightarrow B\vdash B}{
          \infer{A\Rightarrow B\vdash A\Rightarrow B}{}
        & \infer{\Gamma,A'\vdash A}{
            \infer{\Gamma\vdash A'\Rightarrow A}{
              \Gamma\vdash A\Leftrightarrow A'}
          & \infer{A'\vdash A'}{}}}}}}
& \infer{\Gamma\vdash (A'\Rightarrow B')\Rightarrow (A\Rightarrow B)}{\text{analog}}}
\]
Zur Äquivalenz:
\[
\infer{\Gamma\vdash (A\Leftrightarrow B)\Leftrightarrow (A'\Leftrightarrow B')}{
  \infer{\Gamma\vdash (A\Leftrightarrow B)\Rightarrow (A'\Leftrightarrow B')}{
    \infer{\Gamma, A\Leftrightarrow B\vdash A'\Leftrightarrow B'}{
      \infer{\Gamma, A\Leftrightarrow B\vdash A'\Rightarrow B'}{
        \Gamma\vdash A\Leftrightarrow A'
      & \Gamma\vdash B\Leftrightarrow B'}
    & \infer{\Gamma, A\Leftrightarrow B\vdash B'\Rightarrow A'}{
        \Gamma\vdash A\Leftrightarrow A'
      & \Gamma\vdash B\Leftrightarrow B'}}}
& \infer{\Gamma\vdash (A'\Leftrightarrow B')\Rightarrow (A\Leftrightarrow B)}{\text{analog}}}
\]
Zur Universalquantifizierung:
\[
\infer{\Gamma\vdash (\forall x\colon A)\Leftrightarrow (\forall x\colon A')}{
  \infer{\Gamma\vdash (\forall x\colon A)\Rightarrow (\forall x\colon A')}{
    \infer[(x\notin\mathrm{FV(\Gamma)})]
    {\Gamma,\forall x\colon A\vdash\forall x\colon A'}{
      \infer{\Gamma,\forall x\colon A\vdash A'}{
        \infer{\Gamma\vdash A\Rightarrow A'}{
          \Gamma\vdash A\Leftrightarrow A'}
      & \infer{\forall x\colon A\vdash A}{
          \infer{\forall x\colon A\vdash\forall x\colon A}{}}}}}
& \infer{\Gamma\vdash (\forall x\colon A')\Rightarrow (\forall x\colon A)}{
    \text{analog}}
}
\]
Zur Existenzquantifizierung:
\[
\infer{\Gamma\vdash (\exists x\colon A)\Leftrightarrow (\exists x\colon A')}{
  \infer{\Gamma\vdash (\exists x\colon A)\Rightarrow (\exists x\colon A')}{
    \infer[(x\notin\mathrm{FV}(\Gamma))]
    {\Gamma, \exists x\colon A\vdash\exists x\colon A'}{
      \infer{\exists x\colon A\vdash\exists x\colon A}{}
    & \infer{\Gamma, A\vdash\exists x\colon A'}{
        \infer{\Gamma, A\vdash A'}{
          \infer{\Gamma\vdash A\Rightarrow A'}{
            \Gamma\vdash A\Leftrightarrow A'}
        & \infer{A\vdash A}{}}}}}
& \infer{\Gamma\vdash (\exists x\colon A')\Rightarrow (\exists x\colon A)}{
    \text{analog}}
}
\]
Die Forderung $x\notin\mathrm{FV}(\Gamma)$ kann immer erfüllt werden,
indem die Variable an den Stellen, wo sie gebunden vorliegt,
in eine frische umbenannt wird.\,\qedsymbol
\end{Beweis}

\newpage
\section{Aussagenlogik}\index{Aussagenlogik}

\begin{Satz}[Tautologie zum Modus ponens]%
\index{Modus ponens}\newlinefirst
Es gilt $(A\Rightarrow B)\land A \Rightarrow B$.
\end{Satz}
\begin{Beweis}[Beweis 1]
Zur Abkürzung sei $\Gamma:=[(A\Rightarrow B)\land A]$. Der Beweisbaum:
\[
\begin{array}{l@{\qquad\qquad}l}
\infer{\vdash (A\Rightarrow B)\land A\Rightarrow B}{
  \infer{\Gamma\vdash B}{
    \infer{\Gamma\vdash A\Rightarrow B}{
      \infer{\Gamma\vdash (A\Rightarrow B)\land A}{}}
  & \infer{\Gamma\vdash A}{
      \infer{\Gamma\vdash (A\Rightarrow B)\land A}{}}}}
&
\infer[\infernote{$\sim$1}]{(A\Rightarrow B)\land A\Rightarrow B}{
  \infer{B}{
    \infer{A\Rightarrow B}{
      \infer[\infernote{1}]{(A\Rightarrow B)\land A}{}}
  & \infer{A}{
      \infer[\infernote{1}]{(A\Rightarrow B)\land A}{}}}}
\end{array}
\]
Der Programmterm:
\[(A\to B)\times A\to B,\quad (f, a)\mapsto f(a).\,\qedsymbol\]
\end{Beweis}

\begin{Beweis}[Beweis 2 (LEM, boolesche Algebra)]
Man darf rechnen
\begin{align*}
(A\Rightarrow B)\land A \Rightarrow B &\equiv
\neg ((\neg A\lor B)\land A) \lor B
\equiv\neg (\neg A\lor B) \lor \neg A \lor B\\
&\equiv \neg\varphi\lor\varphi\equiv 1,
\end{align*}
wobei $\varphi :\equiv \neg A\lor B$.\;\qedsymbol
\end{Beweis}

\begin{Satz}[Kommutativgesetze]%
\label{bool-cl}\index{Kommutativgesetz!boolesche Algebra}
Es gilt
\begin{gather*}
A\land B \iff B\land A,\\
A\lor B \iff B\lor A.
\end{gather*}
\end{Satz}
\begin{Beweis}[Beweis]
Zu $A\land B\Rightarrow B\land A$. Der Beweisbaum:
\[
\begin{array}{@{}l@{\qquad\qquad}l}
\infer{\vdash A\land B\Rightarrow B\land A}{
  \infer{A\land B\vdash B\land A}{
    \infer{A\land B\vdash B}{
      \infer{A\land B\vdash A\land B}{}}
    & \infer{A\land B\vdash A}{
      \infer{A\land B\vdash A\land B}{}}}}
&
\infer[\infernote{$\sim$1}]{A\land B\Rightarrow B\land A}{
  \infer{B\land A}{
     \infer{B}{\infer[\infernote{1}]{A\land B}{}}
     & \infer{A}{\infer[\infernote{1}]{A\land B}{}}}}
\end{array}
\]
Der Programmterm:
\[A\times B\to B\times A,\quad (a,b)\mapsto (b,a).\]
Zu $A\lor B\Rightarrow B\lor A$. Der Beweisbaum:
\[
\begin{array}{@{}l@{\qquad\qquad}l}
\infer{\vdash A\lor B\Rightarrow B\lor A}{
  \infer{A\lor B\vdash B\lor A}{
    \infer{A\lor B\vdash A\lor B}{}
    & \infer{A\vdash B\lor A}{
        \infer{A\vdash A}{}}
    & \infer{B\vdash B\lor A}{
        \infer{B\vdash B}{}}}}
&
\infer[\infernote{$\sim$2}]{A\lor B\Rightarrow B\lor A}{
  \infer[\infernote{$\sim$1}]{B\lor A}{
    \infer[\infernote{2}]{A\lor B}{}
    & \infer{B\lor A}{\infer[\infernote{1}]{A}{}}
    & \infer{B\lor A}{\infer[\infernote{1}]{B}{}}}}
\end{array}
\]
Der Programmterm:
\[A+B\to B+A,\quad s\mapsto\match s\begin{cases}
\inl(a)\mapsto \inr(a),\\
\inr(b)\mapsto \inl(b).
\end{cases}\]
Vertauschen von $A,B$ erbringt jeweils die umgekehrte Implikation.\;\qedsymbol
\end{Beweis}

\begin{Satz}[Distributivgesetze]%
\label{bool-dl}\index{Distributivgesetz!boolesche Algebra}
Es gilt
\begin{align*}
A\land (B\lor C) &\iff (A\land B)\lor (A\land C),\\
A\lor (B\land C) &\iff (A\lor B)\land (A\lor C).
\end{align*}
\end{Satz}
\begin{Beweis} Es sind vier Implikationen zu bestätigen.\\
Zu $A\land (B\lor C) \vdash A\land B\lor A\land C$.
Programmterm:
\[A\times (B+C) \to A\times B + A\times C,\; (a, s)\mapsto \match s \begin{cases}
\operatorname{inl}(b)\mapsto \operatorname{inl}((a,b)),\\
\operatorname{inr}(c)\mapsto \operatorname{inr}((a,c)).
\end{cases}\]
Zu $A\land B\lor A\land C\vdash A\land (B\lor C)$.
Programmterm:
\[A\times B + A\times C\to A\times (B + C),\;
s\mapsto\match s\begin{cases}
\operatorname{inl}((a,b))\mapsto (a, \operatorname{inl}(b)),\\
\operatorname{inr}((a,c))\mapsto (a, \operatorname{inr}(c)).
\end{cases}
\]
Zu $A\lor (B\land C) \vdash (A\lor B)\land (A\lor C)$. Programmterm:
\[A+B\times C\to (A+B)\times (A+C),\;
s\mapsto\match s\begin{cases}
\inl(a)\mapsto (\inl(a),\inl(a)),\\
\inr((b,c))\mapsto (\inr(b),\inr(c)).
\end{cases} 
\]
Zu $(A\lor B)\land (A\lor C)\vdash A\lor (B\land C)$. Programmterm:
\[
(A+B)\times (A+C)\to A + B\times C,\;
t\mapsto\match t\begin{cases}
(\inl(a), s) \mapsto\inl(a),\\
(\inr(b), \inl(a)) \mapsto\inl(a),\\
(\inr(b), \inr(c)) \mapsto\inr((b, c)).
\end{cases}
\]
Sämtliche Teilaussagen sind bewiesen.\;\qedsymbol
\end{Beweis}

\begin{Satz}\label{from-conj}
Es gilt $A\land B\Rightarrow B$ und $A\Rightarrow A\lor B$.
\end{Satz}
\begin{Beweis}[Beweis 1]
Die Beweisbäume:
\[
\begin{array}{@{}l@{\qquad\quad}l}
\infer{\vdash A\land B\Rightarrow B}{
  \infer{A\land B\vdash B}{
    \infer{A\land B\vdash A\land B}{}}}
&
\infer{\vdash A\Rightarrow A\lor B}{
  \infer{A\vdash A\lor B}{
    \infer{A\vdash A}{}}}
\end{array}
\]
Die Programmterme sind
\[
A\times B\to A,\quad (a,b)\mapsto a;\qquad
A\to A+B,\quad a\mapsto\inl(a).\,\qedsymbol\]
\end{Beweis}
\begin{Beweis}[Beweis 2 (LEM)]
Mit boolescher Algebra erhält man
\begin{gather*}
A\land B\Rightarrow A\equiv \neg(A\land B)\lor A
\equiv \neg A\lor\neg B\lor A
\equiv 1\lor\neg B\equiv 1,\\
A\Rightarrow A\lor B\equiv \neg A\lor A\lor B
\equiv 1\lor B\equiv 1.\,\qedsymbol
\end{gather*}
\end{Beweis}

\newpage

\begin{Satz}[Transitivgesetz der Implikation]\newlinefirst
Es gilt
$(A\Rightarrow B) \land (B\Rightarrow C) \Rightarrow (A\Rightarrow C)$.
\end{Satz}
\begin{Beweis}[Beweis]
Zur Abkürzung sei $\Gamma:=[(A\Rightarrow B) \land (B\Rightarrow C)]$.
Beweisbaum:
\[
\infer{\vdash (A\Rightarrow B) \land (B\Rightarrow C) \Rightarrow (A\Rightarrow C)}{
  \infer{\Gamma\vdash A\Rightarrow C}{
    \infer{\Gamma,A\vdash C}{
     \infer{\Gamma,A\vdash B}{\infer{A\vdash A}{}
     & \infer{\Gamma\vdash A\Rightarrow B}{
         \infer{\Gamma\vdash (A\Rightarrow B)\land (B\Rightarrow C)}{}}}
     & \infer{\Gamma\vdash B\Rightarrow C}{
         \infer{\Gamma\vdash (A\Rightarrow B)\land (B\Rightarrow C)}{}}}}}
\]
Programmterm:
\[(A\to B)\times (B\to C)\to (A\to C),\quad
(f, g)\mapsto (a\colon A)\mapsto g(f(a)).\,\qedsymbol\]
\end{Beweis}

\begin{Satz}[Tautologie zur Kontraposition]\label{contrapos-intro}\newlinefirst
Es gilt $(A\Rightarrow B)\Rightarrow (\neg B\Rightarrow\neg A)$.
\end{Satz}
\begin{Beweis}[Beweis] Beweisbaum:
\[
\begin{array}{@{}l@{\qquad\qquad}l}
\infer{A\Rightarrow B\vdash \neg B\Rightarrow\neg A}{
  \infer{A\Rightarrow B, \neg B\vdash\neg A}{
    \infer{A\Rightarrow B, \neg B, A\vdash\bot}{
      \infer{A\Rightarrow B, A\vdash B}{
        \infer{A\Rightarrow B\vdash A\Rightarrow B}{}
      & \infer{A\vdash A}{}}
    & \infer{\neg B\vdash\neg B}{}}}}
&
\infer[\infernote{$\sim$2}]{\neg B\Rightarrow\neg A}{
  \infer[\infernote{$\sim$1}]{\neg A}{
    \infer{\bot}{
      \infer{B}{A\Rightarrow B & \infer[\infernote{1}]{A}{}}
    & \infer[\infernote{2}]{\neg B}{}}}}
\end{array}
\]
Programmterm zu $(A\Rightarrow B)\Rightarrow (B\Rightarrow\bot)\Rightarrow (A\Rightarrow\bot)$:
\[(A\to B)\to (B\to 0)\to (A\to 0),\quad
f\mapsto g\mapsto (a\colon A)\mapsto g(f(a)).\,\qedsymbol\]
\end{Beweis}

\begin{Satz}[Tautologie zur klassischen Kontraposition] \strong{[LEM]}\newlinefirst
Es gilt $(A\Rightarrow B)\Leftrightarrow (\neg B\Rightarrow\neg A)$.
\end{Satz}
\begin{Beweis}[Beweis 1 (LEM, boolesche Algebra)]
Es findet sich
\[A\Rightarrow B \equiv \neg A\lor B\equiv B\lor\neg A
\equiv \neg\neg B\lor\neg A \equiv \neg B\Rightarrow \neg A.\,\qedsymbol\]
\end{Beweis}
\begin{Beweis}[Beweis 2 (LEM)]
Sei $\Gamma:=[\neg B\Rightarrow\neg A]$. Beweisbaum:
\[
\infer{\vdash (\neg B\Rightarrow\neg A)\Rightarrow (A\Rightarrow B)}{
  \infer{\Gamma\vdash A\Rightarrow B}{
    \infer{\Gamma,A\vdash B}{
      \infer[\infernote{LEM}]{\vdash B\lor \neg B}{}
      & \infer{B\vdash B}{}
      & \infer[\infernote{EFQ}]{\Gamma,A,\neg B\vdash B}{
        \infer{\Gamma,A,\neg B\vdash\bot}{
          \infer{A\vdash A}{}
          & \infer{\Gamma,\neg B\vdash\neg A}{
            \infer{\Gamma\vdash\neg B\Rightarrow\neg A}{}
            & \infer{\neg B\vdash\neg B}{}}}}}}}
\]
Die umgekehrte Richtung bestätigt Satz \ref{contrapos-intro}.\,\qedsymbol
\end{Beweis}

\begin{Satz}
Axiom \ref{DNE} (DN) impliziert
Axiom \ref{EFQ} (EFQ).
\end{Satz}
\begin{Beweis}[Beweis]
Beweisbaum:
\[
\begin{array}{@{}l@{\qquad\qquad}l}
\infer[\infernote{DN}]{\bot\vdash A}{
  \infer{\bot\vdash\neg\neg A}{
    \infer{\bot\vdash \neg A\Rightarrow\bot}{
      \infer{\bot,\neg A\vdash\bot}{}}}}
&
\infer[\infernote{DN}]{A}{
  \infer{\neg\neg A}{
    \infer{\neg A\Rightarrow\bot}{\bot}}}
\end{array}
\]
Vermittels $\operatorname{dn}(A):=((A\mapsto 0)\mapsto 0)\mapsto A$
findet sich der Programmterm
\[0\to A,\quad x\mapsto \operatorname{dn}(A)((f: A\to 0)\mapsto x).
\,\qedsymbol\]
\end{Beweis}

\begin{Satz}
Axiom \ref{DNE} (DN) impliziert Axiom \ref{LEM} (LEM).
\end{Satz}
\begin{Beweis}[Beweis]
Sei $\Gamma:=[\neg (A\lor\neg A)]$. Beweisbaum:
\[\infer[\infernote{DN}]{\vdash A\lor\neg A}{
  \infer{\vdash\neg\neg (A\lor\neg A)}{
    \infer{\Gamma\vdash\bot}{
      \infer{\Gamma\vdash\neg (A\lor\neg A)}{}
    & \infer{\Gamma\vdash A\lor\neg A}{
        \infer{\Gamma\vdash\neg A}{
          \infer{\Gamma,A\vdash\bot}{
            \infer{\Gamma\vdash\neg (A\lor\neg A)}{}
          & \infer{A\vdash A\lor\neg A}{\infer{A\vdash A}{}}}}}}}}
\]
Der diesbezügliche Programmterm ist
\[A + (A\to 0),\quad
\operatorname{dn}(A + (A\to 0))(
  f\mapsto f(\operatorname{inr}((a: A)\mapsto f(\operatorname{inl}(a))))
),\]
wobei $f\colon A + (A \to 0)\to 0$ sein soll.\,\qedsymbol
\end{Beweis}

\begin{Satz}\label{curry-impl}
Es gilt $(A\land B\Rightarrow C)\Leftrightarrow (A\Rightarrow B\Rightarrow C)$.
\end{Satz}
\begin{Beweis}
Für von links nach rechts findet sich:
\[
\infer{A\land B\Rightarrow C\vdash A\Rightarrow B\Rightarrow C}{
  \infer{A\land B\Rightarrow C, A, B\vdash C}{
    \infer{A\land B\Rightarrow C\vdash A\land B\Rightarrow C}{}
  & \infer{A, B\vdash A\land B}{
      \infer{A\vdash A}{} & \infer{B\vdash B}{}}}}
\]
Für von rechts nach links findet sich:
\[
\infer{A\Rightarrow B\Rightarrow C\vdash A\land B\Rightarrow C}{
  \infer{A\Rightarrow B\Rightarrow C, A\land B\vdash C}{
    \infer{A\Rightarrow B\Rightarrow C, A\land B\vdash B\Rightarrow C}{
      \infer{A\Rightarrow B\Rightarrow C\vdash A\Rightarrow B\Rightarrow C}{}
    & \infer{A\land B\vdash A}{\infer{A\land B\vdash A\land B}{}}}
  & \infer{A\land B\vdash B}{\infer{A\land B\vdash A\land B}{}}}}
\]
Die Programmterme sind Schönfinkeln (Currying)
\[(A\times B\to C)\to (A\to B\to C),\quad
f\mapsto a\mapsto b\mapsto f(a,b)\]
und Entschönfinkeln (Uncurrying)
\[(A\to B\to C)\to (A\times B\to C),\quad
f\mapsto (a, b)\mapsto f(a)(b).\,\qedsymbol\]
\end{Beweis}

\begin{Satz}
Es gilt $(A\Rightarrow B\Rightarrow C)\Rightarrow ((A\Rightarrow B)\Rightarrow (A\Rightarrow C))$.
\end{Satz}

\begin{Beweis}
Sei $\Gamma:=[A\Rightarrow B\Rightarrow C]$. Es findet sich:
\[\infer{\vdash (A\Rightarrow B\Rightarrow C)\Rightarrow ((A\Rightarrow B)\Rightarrow (A\Rightarrow C))}{
  \infer{\Gamma\vdash (A\Rightarrow B)\Rightarrow (A\Rightarrow C)}{
    \infer{\Gamma, A\Rightarrow B\vdash A\Rightarrow C}{
      \infer{\Gamma, A\Rightarrow B, A\vdash C}{
        \infer{\Gamma, A\vdash B\Rightarrow C}{
          \infer{\Gamma\vdash A\Rightarrow B\Rightarrow C}{} & \infer{A\vdash A}{}}
    & \infer{A\Rightarrow B, A\vdash B}{
        \infer{A\Rightarrow B\vdash A\Rightarrow B}{} & \infer{A\vdash A}{}}}}}}
\]
Der Programmterm:
\[(A\to B\to C)\to (A\to B)\to A\to C,\quad
f\mapsto g\mapsto a\mapsto (f(a))(g(a)).\,\qedsymbol\]
\end{Beweis}
\newpage

\begin{Satz}
Es gilt $\neg A\land\neg B\Rightarrow\neg (A\lor B)$.
\end{Satz}
\begin{Beweis}
Sei $\Gamma:=[\neg A\land\neg B]$. Es findet sich:
\[
\infer{\vdash \neg A\land\neg B\Rightarrow\neg (A\lor B)}{
  \infer{\Gamma\vdash\neg (A\lor B)}{
    \infer{\Gamma, A\lor B\vdash\bot}{
      \infer{A\lor B\vdash A\lor B}{}
    & \infer{\Gamma,A\vdash\bot}{
        \infer{\Gamma\vdash\neg A}{\infer{\Gamma\vdash\neg A\land\neg B}{}}
      & \infer{A\vdash A}{}}
    & \infer{\Gamma,B\vdash\bot}{
        \infer{\Gamma\vdash\neg B}{\infer{\Gamma\vdash\neg A\land\neg B}{}}
      & \infer{B\vdash B}{}}}}}
\]
Programmterm zu $(A\Rightarrow\bot)\land (B\Rightarrow\bot)\Rightarrow (A\lor B)\Rightarrow\bot$:
\[(A\to 0)\times (B\to 0)\to (A+B)\to 0,\quad (f, g)\mapsto s\mapsto\match s
\begin{cases}
\inl(a)\mapsto f(a),\\
\inr(b)\mapsto g(b).\,\qedsymbol
\end{cases}\]
\end{Beweis}

\begin{Satz}
Es gilt $\neg A\lor\neg B\Rightarrow\neg (A\land B)$.
\end{Satz}
\begin{Beweis}
Es findet sich:
\[
\infer{\vdash\neg A\lor\neg B\Rightarrow\neg (A\land B)}{
  \infer{\neg A\lor\neg B\vdash\neg (A\land B)}{
    \infer{\neg A\lor\neg B, A\land B\vdash\bot}{
      \infer{\neg A\lor\neg B\vdash\neg A\lor\neg B}{}
    & \infer{A\land B,\neg A\vdash\bot}{
        \infer{\neg A\vdash\neg A}{}
      & \infer{A\land B\vdash A}{\infer{A\land B\vdash A\land B}{}}}
    & \infer{A\land B,\neg B\vdash\bot}{
        \infer{\neg B\vdash\neg B}{}
      & \infer{A\land B\vdash B}{\infer{A\land B\vdash A\land B}{}}}}}}
\]
Der Programmterm zu
$(A\Rightarrow\bot)\lor(B\Rightarrow\bot)\Rightarrow (A\land B\Rightarrow\bot)$:
\[(A\to 0)+(B\to 0)\to (A\times B\to 0),\quad
s\mapsto\match s\begin{cases}
\inl(f)\mapsto (a,b)\mapsto f(a),\\
\inr(g)\mapsto (a,b)\mapsto g(b).\,\qedsymbol
\end{cases}\]
\end{Beweis}

\begin{Satz}
Es gilt $A\lor B\Rightarrow (\neg A\Rightarrow B)$.
\end{Satz}
\begin{Beweis}
Es findet sich:
\[
\infer{\vdash A\lor B\Rightarrow (\neg A\Rightarrow B)}{
  \infer{A\lor B\vdash \neg A\Rightarrow B}{
    \infer{A\lor B, \neg A\vdash B}{
      \infer{A\lor B\vdash A\lor B}{}
    & \infer[\infernote{EFQ}]{\neg A, A\vdash B}{
        \infer{\neg A, A\vdash\bot}{
          \infer{\neg A\vdash\neg A}{}
        & \infer{A\vdash A}{}}}
    & \infer{B\vdash B}{}}}}
\]
Programmterm zu $A\lor B\Rightarrow (A\Rightarrow\bot)\Rightarrow B$:
\[A+B\to (A\to 0)\to B,\quad s\mapsto f\mapsto\match s\begin{cases}
\inl(a)\mapsto \operatorname{exfq}(B)(f(a)),\\
\inr(b)\mapsto b
\end{cases}\]
Der Schluss EFQ bzw. die Verfügbarkeit von
$\operatorname{exfq}(B)\colon 0\to B$
gilt gemäß Axiom \ref{EFQ}.\,\qedsymbol
\end{Beweis}

\newpage
\begin{Satz}[Assoziativgesetz]\newlinefirst
Es gilt $A\land (B\land C)\Leftrightarrow (A\land B)\land C$.
\end{Satz}
\begin{Beweis}
Sei $\Gamma:=[A\land (B\land C)]$. Der Beweisbaum:
\[
\infer{\vdash A\land (B\land C)\Rightarrow (A\land B)\land C}{
  \infer{\Gamma\vdash (A\land B)\land C}{
    \infer{\Gamma\vdash A\land B}{
      \infer{\Gamma\vdash A}{
        \infer{\Gamma\vdash A\land (B\land C)}{}}
    & \infer{\Gamma\vdash B}{
        \infer{\Gamma\vdash B\land C}{
          \infer{\Gamma\vdash A\land (B\land C)}{}}}}
  & \infer{\Gamma\vdash C}{
      \infer{\Gamma\vdash B\land C}{
        \infer{\Gamma\vdash A\land (B\land C)}{}}}}}
\]
Der Programmterm ist
\[A\times (B\times C)\to (A\times B)\times C,\quad
(a, (b, c))\mapsto ((a, b), c).\]
Für die Umkehrung verläuft der Beweis analog.\,\qedsymbol
\end{Beweis}

\begin{Satz}\label{conj-premise}
Es gilt $(A\Rightarrow B)\Leftrightarrow (A\Rightarrow A\land B)$.
\end{Satz}
\begin{Beweis}[Beweis 1]
Der Beweisbaum:
\[
\infer{\vdash (A\Rightarrow B)\Leftrightarrow (A\Rightarrow A\land B)}{
  \infer{\vdash (A\Rightarrow B)\Rightarrow A\Rightarrow A\land B}{
    \infer{A\Rightarrow B\vdash A\Rightarrow A\land B}{
      \infer{A\Rightarrow B, A\vdash A\land B}{
        \infer{A\vdash A}{}
      & \infer{A\Rightarrow B, A\vdash B}{
          \infer{A\Rightarrow B\vdash A\Rightarrow B}{}
        & \infer{A\vdash A}{}}}}}
& \infer{\vdash (A\Rightarrow A\land B)\Rightarrow A\Rightarrow B}{
    \infer{A\Rightarrow A\land B\vdash A\Rightarrow B}{
      \infer{A\Rightarrow A\land B, A\vdash B}{
        \infer{A\Rightarrow A\land B, A\vdash A\land B}{
          \infer{A\Rightarrow A\land B\vdash A\Rightarrow A\land B}{}
        & \infer{A\vdash A}{}}}}}}
\]
Die Programmterme sind
\begin{gather*}
(A\to B)\to A\to A\times B,\quad f\mapsto a\mapsto (a, f(a)),\\
(A\to A\times B)\to A\to B,\quad f\mapsto a\mapsto \pi_{\mathrm r}(f(a)).
\end{gather*}
\end{Beweis}
\begin{Beweis}[Beweis 2 (LEM, boolesche Algebra)]
Es gilt die Umformung
\[A\Rightarrow A\land B\equiv \neg A\lor (A\land B)
\equiv (\neg A\lor A)\land (\neg A\land B)
\equiv 1\land (A\Rightarrow B)\equiv A\Rightarrow B.\,\qedsymbol\]
\end{Beweis}

\begin{Satz}\label{non-eq-prop-neg}
Es gilt $\neg (A\Leftrightarrow\neg A)$.
\end{Satz}
\begin{Beweis}
Sei $\Gamma:=[A\Leftrightarrow\neg A]$. Der Beweisbaum:
\[
\begin{array}{@{}l@{\qquad\quad}l}
\infer{\Gamma\vdash\neg A}{
  \infer{\Gamma,A\vdash\bot}{
    \infer{\Gamma,A\vdash\neg A}{
      \infer{\Gamma\vdash A\Rightarrow\neg A}{
        \infer{\Gamma\vdash A\Leftrightarrow\neg A}{}}
    & \infer{A\vdash A}{}}
  & \infer{A\vdash A}{}}}
&
\infer{\vdash\neg (A\Leftrightarrow\neg A)}{
  \infer{\Gamma\vdash\bot}{
    \Gamma\vdash\neg A
  & \infer{\Gamma\vdash A}{
      \infer{\Gamma\vdash\neg A\Rightarrow A}{
        \infer{\Gamma\vdash A\Leftrightarrow\neg A}{}}
    & \Gamma\vdash\neg A}}}
\end{array}
\]
Der Programmterm ist
\begin{gather*}
(A\to A\to 0)\times ((A\to 0)\to A)\to 0,\\
(f,g)\mapsto\text{\strong{let}}\; h\colon A\to 0 = (a\mapsto f(a)(a))
\;\,\text{\strong{in}}\;\, h(g(h)).\,\qedsymbol
\end{gather*}
\end{Beweis}

\newpage
\begin{Satz}\label{xor-char1}\strong{[LEM]}
Es gilt $A\oplus B \Leftrightarrow (\neg A\Leftrightarrow B)$.
\end{Satz}
\begin{Beweis}[Beweis (boolesche Algebra)]
Es findet sich die Umformung
\begin{align*}
\neg A\Leftrightarrow B
&\equiv (\neg A\Rightarrow B)\land (B\Rightarrow\neg A)
\equiv (\neg\neg A\lor B)\land (\neg B\lor\neg A)\\
&\equiv ((A\lor B)\land\neg B)\lor ((A\lor B)\land\neg A)
\equiv ((A\land\neg B)\lor\bot)\lor (\bot\lor (B\land\neg A))\\
&\equiv (A\land\neg B)\lor (B\land\neg A)
\equiv A\oplus B.\,\qedsymbol
\end{align*}
\end{Beweis}

\begin{Satz}\label{xor-char2}\strong{[LEM]}
Es gilt $A\oplus B \Leftrightarrow \neg (A\Leftrightarrow B)$.
\end{Satz}
\begin{Beweis}[Beweis (boolesche Algebra)]
Es findet sich die Umformung
\begin{align*}
\neg (A\Leftrightarrow B) &\equiv
\neg ((A\Rightarrow B)\land (B\Rightarrow A))
\equiv \neg ((\neg A\lor B)\land (\neg B\lor A))\\
&\equiv \neg (\neg A\lor B)\lor \neg (\neg B\lor A)
\equiv (\neg\neg A\land\neg B)\lor (\neg\neg B\land \neg A)\\
&\equiv (A\land\neg B)\lor (B\land\neg A) \equiv A\oplus B.\,\qedsymbol
\end{align*}
\end{Beweis}

\begin{Satz}\label{iff-assoc}\strong{[LEM]}
Es gilt $(A\Leftrightarrow (B\Leftrightarrow C))\Leftrightarrow
((A\Leftrightarrow B)\Leftrightarrow C)$.
\end{Satz}
\begin{Beweis}[Beweis (boolesche Algebra)]
Zur Abkürzung schreiben wir $AB$ anstelle von $A\land B$,
und $\bar A$ anstelle von $\neg A$. Für die linke Seite findet sich die
Umformung
\begin{align*}
A\Leftrightarrow (B\Leftrightarrow C)
&\equiv A\Leftrightarrow (BC \lor \bar B\bar C)
\equiv A(BC\lor\bar B\bar C)\lor \bar A(\overline{BC\lor\bar B\bar C})\\
&\equiv A(BC\lor\bar B\bar C)\lor \bar A(\bar B\lor \bar C)(B\lor C)
\equiv ABC\lor A\bar B\bar C\lor \bar A\bar BC\lor\bar AB\bar C.
\end{align*}
Die rechte Seite formt man analog in dieselbe disjunktive Normalform
um.\,\qedsymbol
\end{Beweis}

\begin{Satz}\label{xor3-char}\strong{[LEM]}
Es gilt $A\oplus (B\oplus C)\Leftrightarrow (A\Leftrightarrow (B\Leftrightarrow C))$.
\end{Satz}
\begin{Beweis}
Es findet sich die Umformung
\[A\oplus (B\oplus C)\stackrel{\text{(1)}}\equiv
A\Leftrightarrow \neg (B\oplus C)
\stackrel{\text{(2)}}\equiv A\Leftrightarrow\neg\neg (B\Leftrightarrow C)
\equiv A\Leftrightarrow (B\Leftrightarrow C).\]
Hierbei gilt (1) laut Satz \ref{xor-char1}, und (2) laut Satz
\ref{xor-char2}.\,\qedsymbol
\end{Beweis}

\begin{Satz}\label{xor-assoc}\strong{[LEM]}
Es gilt $A\oplus (B\oplus C)\Leftrightarrow (A\oplus B)\oplus C$.
\end{Satz}
\begin{Beweis}
Es findet sich die Umformung
\[A\oplus (B\oplus C)\stackrel{\text{(1)}}\equiv A\Leftrightarrow (B\Leftrightarrow C)
\stackrel{\text{(2)}}\equiv (A\Leftrightarrow B)\Leftrightarrow C
\stackrel{\text{(3)}}\equiv (A\oplus B)\oplus C.\]
Hierbei gilt (1), (3) laut Satz \ref{xor3-char},
und (2) laut Satz \ref{iff-assoc}.\,\qedsymbol
\end{Beweis}

\newpage
\section{Prädikatenlogik}\index{Praedikatenlogik@Prädikatenlogik}

\begin{Definition}[Beschränkte Quantifizierung]%
\label{def:bounded}
\begin{align*}
(\forall x{\in}M\colon P(x)) &\defiff \forall x\colon (x\in M\Rightarrow P(x)),\\
(\exists x{\in}M\colon P(x)) &\defiff \exists x\colon (x\in M\land P(x)).
\end{align*}
\end{Definition}

\begin{Satz}[Allgemeine Distributivgesetze]%
\label{general-dl}
Es gilt:
\begin{align*}
A\land (\exists x\colon P(x)) &\iff \exists x\colon (A\land P(x)),\\
A\lor (\forall x\colon P(x)) &\iff \forall x\colon (A\lor P(x)).
\end{align*}
\end{Satz}
\begin{Beweis}[Beweis]
Die Beweisbäume zur ersten Äquivalenz:
\[\begin{array}{c@{\qquad}c}
\infer[\infernote{$\sim$1}]{\exists x\colon (A\land P(x))}{
  \exists x\colon P(x)
& \infer{\exists x\colon (A\land P(x))}{
    \infer{A\land P(a)}{A & \infer[\infernote{1}]{P(a)}{}}}}
&
\infer[\infernote{$\sim$1}]{A\land\exists x\colon P(x)}{
  \exists x\colon (A\land P(x))
& \infer{A\land \exists x\colon P(x)}{
    \infer{A}{\infer[\infernote{1}]{A\land P(a)}{}}
  & \infer{\exists x\colon P(x)}{
      \infer{P(a)}{\infer[\infernote{1}]{A\land P(a)}{}}}}}
\end{array}\]
Die Beweisbäume zur zweiten Äquivalenz:
{\small
\[\begin{array}{c@{\quad}c}
\infer[\infernote{$\sim$1}]{\forall x\colon (A\lor P(x))}{
A{\lor}(\forall x\colon P(x))
& \infer{\forall x\colon (A{\lor}P(x))}{
    \infer{A\lor P(x)}{\infer[\infernote{1}]{A}{}}
  }
& \infer{\forall x\colon (A{\lor}P(x))}{
    \infer{A\lor P(x)}{
      \infer{P(x)}{\infer[\infernote{1}]{\forall x\colon P(x)}{}}}}}
&
\infer[\infernote{$\sim$1}]{A\lor\forall x\colon P(x)}{
  \infer{A\lor P(x)}{\forall x\colon (A{\lor}P(x))}
  & \infer{A{\lor}\forall x\colon P(x)}{
      \infer[\infernote{1}]{A}{}}
  & \infer{A{\lor}\forall x\colon P(x)}{
      \infer{\forall x\colon P(x)}{
        \infer[\infernote{1}]{P(x)}{}}}}
\end{array}\]
}

\noindent
Die linken Bäume zeigen jeweils die Implikation von links nach
rechts, die rechten die Implikation von rechts nach
links.\,\qedsymbol
\end{Beweis}

\begin{Satz}\label{exists-dl}
Es gilt
\[(\exists x\colon P(x)\lor Q(x)) \iff
(\exists x\colon P(x))\lor(\exists x\colon Q(x)).\]
\end{Satz}
\begin{Beweis}[Beweis]
Der Beweisbaum zur Implikation von links nach rechts:
\[
\infer[\infernote{$\sim$1}]{(\exists x\colon P(x))\lor (\exists x\colon Q(x))}{
  \exists x\colon P(x)\lor Q(x)
& \infer[\infernote{$\sim$2}]{(\exists x\colon P(x))\lor (\exists x\colon Q(x))}{
    \infer[\infernote{1}]{P(a)\lor Q(a)}{}
  & \infer{(\exists x\colon P(x)){\lor}(\exists x\colon Q(x))}{\infer{
      \exists x\colon P(x)}{\infer[\infernote{2}]{P(a)}{}}}
  & \infer{(\exists x\colon P(x)){\lor}(\exists x\colon Q(x))}{\infer{
      \exists x\colon Q(x)}{\infer[\infernote{2}]{Q(a)}{}}}}}
\]
Der Beweisbaum zur Implikation von rechts nach links:
\[\infer[\infernote{$\sim$1}]{\exists x\colon P(x)\lor Q(x)}{
  (\exists x\colon P(x))\lor(\exists x\colon Q(x))
  & \infer[\infernote{$\sim$2}]{\exists x\colon P(x)\lor Q(x)}{
      \infer[\infernote{1}]{\exists x\colon P(x)}{}
      &
      \infer{P(a)\lor Q(a)}{
        \infer[\infernote{2}]{P(a)}{}}}
  & \infer[\infernote{$\sim$3}]{\exists x\colon P(x)\lor Q(x)}{
      \infer[\infernote{1}]{\exists x\colon Q(x)}{}
      &
      \infer{P(b)\lor Q(b)}{
        \infer[\infernote{3}]{Q(b)}{}}}
}\]
Man beachte, dass die Zeugen $a,b$ unterschiedlich sein können.\,\qedsymbol
\end{Beweis}

\begin{Satz}\label{exists-asym-dl}
Es gilt:
\[(\exists x\colon P(x)\land Q(x)) \implies (\exists x\colon P(x))\land (\exists x\colon Q(x)).\]
\end{Satz}
\begin{Beweis}[Beweis] Sei $A:\equiv P(a)\land Q(a)$. Es findet sich:
\[
\infer{\exists x\colon P(x)\land Q(x)\vdash (\exists x\colon P(x))\land (\exists x\colon Q(x))}{
  \infer{\exists a\colon A\vdash\exists a\colon A}{}
& \infer{A\vdash (\exists x\colon P(x))\land (\exists x\colon Q(x))}{
    \infer{A\vdash\exists x\colon P(x)}{
      \infer{A\vdash P(a)}{\infer{A\vdash P(a)\land Q(a)}{}}}
  & \infer{A\vdash\exists x\colon Q(x)}{
       \infer{A\vdash Q(a)}{\infer{A\vdash P(a)\land Q(a)}{}}}}}
\]
In Worten: Weil aufgrund der Prämisse ein Zeuge $a$ mit sowohl $P(a)$
als auch $Q(a)$ vorliegt, dürfen wir schließen, dass die
Existenzaussagen $\exists x\colon P(x)$ und $\exists x\colon Q(x)$
erfüllt sind.\,\qedsymbol
\end{Beweis}

\begin{Satz}[Allgemeines de morgansches Gesetz 1]%
\label{dm-g1} Es gilt
\[(\neg\exists x\colon P(x))\iff (\forall x\colon \neg P(x)).\]
\end{Satz}
\begin{Beweis} Unter Spezialisierung von Satz \ref{exists-implies-const}
findet sich die äquivalente Umformung%
\[\neg\exists x\colon P(x)\equiv (\exists x\colon P(x))\Rightarrow 0
\equiv \forall x\colon (P(x)\Rightarrow 0)
\equiv \forall x\colon \neg P(x).\,\qedsymbol\]
\end{Beweis}
\begin{Satz}[Allgemeines de morgansches Gesetz 2]%
\label{dm-g2} Es gilt
\[(\neg\forall x\colon P(x))\iff (\exists x\colon \neg P(x)).\]
\end{Satz}
\begin{Beweis}[Beweis (LEM)]
Nutzung von LEM und Satz \ref{dm-g1} gestattet die äquivalente
Umformung%
\[\neg\forall x\colon P(x)
\equiv\neg\forall x\colon \neg\neg P(x)
\equiv\neg\neg\exists x\colon \neg P(x)
\equiv\exists x\colon \neg P(x).\,\qedsymbol\]
\end{Beweis}

\begin{Satz}\label{exists-implies-const}
Es gilt die Äquivalenz
\[(\forall x\colon (P(x)\Rightarrow A))
\iff ((\exists x\colon P(x))\Rightarrow A).\]
\end{Satz}
\begin{Beweis}[Beweis 1] Mit den Abkürzungen $\Gamma:=[\forall x\colon (P(x)\Rightarrow A)]$
und $\Gamma':=[(\exists x\colon P(x))\Rightarrow A]$
lauten die Beweisbäume:
\[\begin{array}{c@{\qquad}c}\small
\infer{\Gamma\vdash (\exists x\colon P(x))\Rightarrow A}{
  \infer{\Gamma,\exists x\colon P(x)\vdash A}{
    \infer{\exists x\colon P(x)\vdash\exists a\colon P(a)}{}
  & \infer{\Gamma,P(a)\vdash A}{
      \infer{\Gamma\vdash P(a)\Rightarrow A}{
        \infer{\Gamma\vdash\forall x\colon (P(x)\Rightarrow A)}{}}
    & \infer{P(a)\vdash P(a)}{}}}}
&\small
\infer{\Gamma'\vdash\forall x\colon (P(x)\Rightarrow A)}{
  \infer{\Gamma'\vdash P(x)\Rightarrow A}{
    \infer{\Gamma',P(x)\vdash A}{
      \infer{\Gamma'\vdash (\exists x\colon P(x))\Rightarrow A}{}
    & \infer{P(x)\vdash \exists x\colon P(x)}{
        \infer{P(x)\vdash P(x)}{}}}}}
\end{array}
\]
Die linke Seite zeigt die Implikation von links nach rechts,
die rechte die Implikation von rechts nach links.\,\qedsymbol
\end{Beweis}
\begin{Beweis}[Beweis 2 (LEM, boolesche Algebra)]
Unter Nutzung von Satz \ref{general-dl}
und Satz \ref{dm-g1} gelingt die Umformung
\begin{align*}
\forall x\colon (P(x)\Rightarrow A)
&\equiv \forall x\colon (\neg P(x)\lor A)
\equiv (\forall x\colon \neg P(x))\lor A\\
&\equiv \neg (\exists x\colon P(x))\lor A
\equiv (\exists x\colon P(x))\Rightarrow A.\,\qedsymbol
\end{align*}
\end{Beweis}

\begin{Satz}[Kommutativgesetz der Quantoren]\label{exists-cl}
Es gilt die Äquivalenz
\[(\exists x\colon\exists y\colon P(x,y)) \iff (\exists y\colon\exists x\colon P(x,y)).\]
\end{Satz}
\begin{Beweis}[Beweis]
Sei $\Gamma:=[\exists x\colon\exists y\colon P(x,y)]$.
Beweisbaum:
\[
\infer{\vdash(\exists x\colon\exists y\colon P(x,y))
  \Rightarrow (\exists y\colon\exists x\colon P(x,y))
}{
  \infer{\Gamma\vdash\exists y\colon\exists x\colon P(x,y)}{
    \infer{\Gamma\vdash\exists a\colon\exists y\colon P(a,y)}{}
  & \infer{\exists y\colon P(a,y)\vdash\exists y\colon\exists x\colon P(x,y)}{
      \infer{\exists y\colon P(a,y)\vdash\exists b\colon P(a,b)}{}
    & \infer{P(a,b)\vdash\exists y\colon\exists x\colon P(x,y)}{
        \infer{P(a,b)\vdash\exists x\colon P(x,b)}{
          \infer{P(a,b)\vdash P(a,b)}{}}}}}}
\]
Der Beweis der Umkehrung verläuft analog.\,\qedsymbol
\end{Beweis}

\begin{Satz}[Kommutativgesetz der Quantoren]\label{all-cl}
Es gilt die Äquivalenz
\[(\forall x\colon\forall y\colon P(x,y))
\iff (\forall y\colon\forall x\colon P(x,y)).\]
\end{Satz}
\begin{Beweis}[Beweis]
Sei $\Gamma:=[\forall x\colon\forall y\colon P(x,y)]$.
Beweisbaum:
\[
\infer{\vdash (\forall x\colon\forall y\colon P(x,y))
  \Rightarrow (\forall y\colon\forall x\colon P(x,y))
}{
  \infer{\Gamma\vdash\forall y\colon\forall x\colon P(x,y)}{
    \infer{\Gamma\vdash\forall x\colon P(x,y)}{
      \infer{\Gamma\vdash P(x,y)}{
        \infer{\Gamma\vdash\forall y\colon P(x,y)}{
          \infer{\Gamma\vdash\forall x\colon\forall y\colon P(x,y)}{}}}}}}
\]
Der Beweis der Umkehrung verläuft analog.\,\qedsymbol
\end{Beweis}

\begin{Satz}[Allgemeine Distributivgesetze]%
\label{bounded-general-dl}
Es gilt:
\begin{align*}
A\land (\exists x{\in}M\colon P(x)) &\iff (\exists x{\in}M\colon A\land P(x)),\\
A\lor (\forall x{\in}M\colon P(x)) &\iff (\forall x{\in}M\colon A\lor P(x)).
\end{align*}
\end{Satz}

\begin{Beweis}
Nach Def. \ref{def:bounded} und Satz \ref{general-dl} gilt:
\begin{gather*}
A\land \exists x{\in}M\colon P(x)
\equiv A\land \exists x\colon x\in M\land P(x)
\equiv \exists x\colon A\land x\in M\land P(x)\\
\equiv \exists x\colon x\in M\land A\land P(x)
\equiv \exists x{\in}M\colon A\land P(x).
\end{gather*}
Nach Def. \ref{def:bounded} und Satz \ref{general-dl} gilt:
\begin{gather*}
A\lor\forall x{\in}M\colon P(x)
\equiv A\lor\forall x\colon (x\in M\Rightarrow P(x))
\equiv A\lor\forall x\colon x\notin M\lor P(x)\\
\equiv \forall x\colon A\lor x\notin M\lor P(x)
\equiv \forall x\colon (x\in M\Rightarrow A\lor P(x))\\
\equiv \forall x{\in}M\colon A\lor P(x).\,\qedsymbol
\end{gather*}
\end{Beweis}

\begin{Satz}\label{bounded-exists-cl}
Es gilt:
\[(\exists x{\in}A\colon \exists y{\in}B\colon P(x,y))
\iff (\exists y{\in}B\colon \exists x{\in}A\colon P(x,y)).\]
\end{Satz}

\begin{Beweis} Nach Def. \ref{def:bounded}, Satz \ref{general-dl}
und Satz \ref{exists-cl} gilt:
\begin{gather*}
\exists x{\in}A\colon \exists y{\in}B\colon P(x,y)
\equiv \exists x\colon x\in A\land\exists y\colon y\in B\land P(x,y)\\
\equiv \exists x\colon \exists y\colon x\in A\land y\in B\land P(x,y)
\equiv \exists y\colon \exists x\colon y\in B\land x\in A\land P(x,y)\\
\equiv \exists y\colon y\in B\land \exists x\colon x\in A\land P(x,y)
\equiv \exists y{\in}B\colon\exists x{\in}A\colon P(x,y).\,\qedsymbol
\end{gather*}
\end{Beweis}

\begin{Satz}\label{bounded-all-cl}
Es gilt:
\[(\forall x{\in}A\colon \forall y{\in}B\colon P(x,y))
\iff (\forall y{\in}B\colon \forall x{\in}A\colon P(x,y)).\]
\end{Satz}
\begin{Beweis}[Beweis (LEM, boolesche Algebra)]\newlinefirst
Nach Def. \ref{def:bounded}, Satz \ref{general-dl} und
Satz \ref{all-cl} gilt:
\begin{gather*}
\forall x{\in}A\colon \forall y{\in}B\colon P(x,y)
\equiv \forall x\colon x\in A\Rightarrow \forall y\colon y\in B\Rightarrow P(x,y)\\
\equiv \forall x\colon x\notin A\lor \forall y\colon y\notin B\lor P(x,y)
\equiv \forall x\colon \forall y\colon x\notin A\lor y\notin B\lor P(x,y)\\
\equiv \forall y\colon \forall x\colon y\notin B\lor x\notin A\lor P(x,y)
\equiv \forall y\colon y\notin B\lor \forall x\colon x\notin A\lor P(x,y)\\
\equiv \forall y\colon y\in B\Rightarrow \forall x\colon x\in A\Rightarrow P(x,y)
\equiv \forall y{\in}B\colon\forall x{\in}A\colon P(x,y).\,\qedsymbol
\end{gather*}
\end{Beweis}

\begin{Satz}
Für eine Aussage $P$, die nicht von $x$ abhängt, und ein nichtleeres
Diskursuniversum gilt:
\[(\exists x\colon P) \iff P.\]
\end{Satz}
\begin{Beweis}[Beweis]
Nach Satz \ref{general-dl} gilt:
\[\exists x\colon P \equiv \exists x\colon (1\land P)
\equiv (\exists x\colon 1)\land P\equiv 1\land P\equiv P.\]
Im vorletzten Schritt wurde dabei ausgenutzt, dass
für ein nichtleeres Diskursuniversum immer $(\exists x\colon 1)\equiv 1$
gelten muss.\,\qedsymbol
\end{Beweis}

\begin{Satz}
Es gilt
\[(\exists x{\in}M\colon P) \iff (M\ne\emptyset)\land P.\]
\end{Satz}

\begin{Beweis}
Nach Def. \ref{def:bounded} und Satz \ref{general-dl} gilt:
\begin{gather*}
\exists x{\in}M\colon P \equiv \exists x\colon (x\in M\land P)
\equiv (\exists x\colon x\in M)\land P\equiv (M\ne\emptyset)\land P.\,\qedsymbol
\end{gather*}
\end{Beweis}

\begin{Satz}\label{imp-dl}
Es gilt $(A\Rightarrow\forall x\colon P(x))\Leftrightarrow (\forall x\colon A\Rightarrow P(x))$.
\end{Satz}
\begin{Beweis}
Mit $\Gamma:=[A\Rightarrow\forall x\colon P(x)]$
und $\Gamma':=[\forall x\colon A\Rightarrow P(x)]$
lautet der Beweisbaum:
\[
\infer{\vdash (A\Rightarrow\forall x\colon P(x))\Leftrightarrow (\forall x\colon A\Rightarrow P(x))}{
  \infer{\vdash (A\Rightarrow\forall x\colon P(x))\Rightarrow (\forall x\colon A\Rightarrow P(x))}{
    \infer{\Gamma\vdash\forall x\colon A\Rightarrow P(x)}{
      \infer{\Gamma\vdash A\Rightarrow P(x)}{
        \infer{\Gamma, A\vdash P(x)}{
          \infer{\Gamma, A\vdash\forall x\colon P(x)}{
            \infer{\Gamma\vdash A\Rightarrow\forall x\colon P(x)}{}
          & \infer{A\vdash A}{}}}}}}
& \infer{\vdash (\forall x\colon A\Rightarrow P(x))\Rightarrow (A\Rightarrow\forall x\colon P(x))}{
    \infer{\Gamma'\vdash A\Rightarrow\forall x\colon P(x)}{
      \infer{\Gamma', A\vdash\forall x\colon P(x)}{
        \infer{\Gamma', A\vdash P(x)}{
          \infer{\Gamma'\vdash A\Rightarrow P(x)}{
            \infer{\Gamma'\vdash\forall x\colon A\Rightarrow P(x)}{}
          }
        & \infer{A\vdash A}{}}}}}}\;\qedsymbol
\]
\end{Beweis}

\begin{Satz}\label{redundant-eq}
Es gilt $P(x)\Leftrightarrow (\forall y\colon x=y\Rightarrow P(y))$.
\end{Satz}
\begin{Beweis}
Sei $\Gamma:=[\forall y\colon x=y\Rightarrow P(y)]$. Es findet sich:
\[
\begin{array}{@{}l@{\qquad\quad}l}
\infer{P(x)\vdash\forall y\colon x=y\Rightarrow P(y)}{
  \infer{P(x)\vdash x=y\Rightarrow P(y)}{
    \infer{P(x),x=y\vdash P(y)}{
      \infer{P(x)\vdash P(x)}{} & \infer{x=y\vdash x=y}{}}}}
&
\infer{\Gamma\vdash P(x)}{
  \infer{\Gamma\vdash x=x\Rightarrow P(x)}{
    \infer{\Gamma\vdash\forall y\colon x=y\Rightarrow P(y)}{}}
& \infer{\vdash x=x}{}}\;\qedsymbol
\end{array}
\]
\end{Beweis}

\newpage
\begin{Satz}[Leere Wahrheit]\newlinefirst
Die Formeln $\forall x\in\emptyset\colon P(x)$ und $\neg\exists x\in\emptyset\colon P(x)$
sind gültig.
\end{Satz}
\begin{Beweis}
Zur Abkürzung sei $A:=(x\in\emptyset\land P(x))$. Die Beweisbäume:
\[
\begin{array}{@{}l@{\qquad\quad}l}
\infer{\vdash\forall x\in\emptyset\colon P(x)}{
  \infer{\vdash\forall x\colon x\in\emptyset\Rightarrow P(x)}{
    \infer{\vdash x\in\emptyset\Rightarrow P(x)}{
      \infer{x\in\emptyset\vdash P(x)}{
        \infer{x\in\emptyset\vdash\bot}{
          \infer{\vdash x\notin\emptyset}{}
        & \infer{x\in\emptyset\vdash x\in\emptyset}{}}}}}}
&
\infer{\vdash\neg\exists x\in\emptyset\colon P(x)}{
  \infer{\vdash\neg\exists x\colon A}{
    \infer{\exists x\colon A\vdash\bot}{
      \infer{\exists x\colon A\vdash\exists x\colon A}{}
    & \infer{A\vdash\bot}{
        \infer{\vdash x\notin\emptyset}{}
      & \infer{A\vdash x\in\emptyset}{
          \infer{A\vdash x\in\emptyset\land P(x)}{}}}}}}\;\qedsymbol
\end{array}
\]
\end{Beweis}

\section{Modallogik}
\subsection{Modallogische Axiome}

\begin{Axiom}[K]\label{axiom-K}
Es gilt das Schema $\vdash\lnec (A\cond B)\cond (\lnec A\cond\lnec B)$.
\end{Axiom}

\begin{Axiom}[T]\label{axiom-T}
Es gilt das Schema $\vdash\lnec A\cond A$.
\end{Axiom}

\begin{Axiom}[B]
Es gilt das Schema $\vdash A\cond\lnec\lpos A$.
\end{Axiom}

\begin{Axiom}[D]
Es gilt das Schema $\vdash\lnec A\cond\lpos A$.
\end{Axiom}

\begin{Axiom}[4]
Es gilt das Schema $\vdash\lnec A\cond\lnec\lnec A$.
\end{Axiom}

\begin{Axiom}[5]
Es gilt das Schema $\vdash\lpos A\cond\lnec\lpos A$.
\end{Axiom}

\subsection{System K}

\begin{Definition}[System K]\newlinefirst
Das formale System~K ist dadurch festgelegt, dass sämtliche Regeln
und Axiome der klassischen Aussagenlogik gelten, und zusätzlich
das Axiom \ref{axiom-K} (K) und die Schlussregel
\[\dfrac{\vdash A}{\vdash\lnec A}.\qquad\text{(N: Nezessisierungsregel)}\]
\end{Definition}

\begin{Definition}[Möglichkeit]\label{def:possibly}\newlinefirst
Man legt $\lpos A :\equiv \neg\lnec\neg A$ fest, gelesen
»möglicherweise $A$«.
\end{Definition}

\begin{Satz}[Regel K]
Die Schlussregel $\dfrac{\Gamma\vdash\lnec (A\cond B)}{\Gamma\vdash\lnec A\cond\lnec B}$
ist im System~K zulässig. 
\end{Satz}
\begin{Beweis} Wird kurzerhand durch den Beweisbaum
\[
\infer[\infernote{Modus ponens}]{\Gamma\vdash\lnec A\cond\lnec B}{
  \infer[\infernote{Axiom K}]{\vdash\lnec (A\cond B)\cond (\lnec A\cond\lnec B)}{}
& \Gamma\vdash\lnec (A\cond B)}
\]
bestätigt.\,\qedsymbol
\end{Beweis}

\begin{Satz}[Regel N*]
Für $n\ge 0$ gilt im System~K die Schlussregel
\[\dfrac{A_1,\ldots,A_n\vdash B}{\lnec A_1,\ldots,\lnec A_n\vdash\lnec B}.\]
\end{Satz}
\begin{Beweis}
Induktion über $n$. Im Anfang $n=0$ nimmt die Regel schlicht die Form
der Nezessisierungsregel an. Der Induktionsschritt wird durch den Beweisbaum
\[
\infer[\infernote{Modus ponens}]{\lnec A_1,\ldots,\lnec A_n,\lnec A_{n+1}\vdash\lnec B}{
  \infer[\infernote{K}]{\lnec A_1,\ldots,\lnec A_n\vdash\lnec A_{n+1}\cond\lnec B}{
    \infer[\infernote{IV}]{\lnec A_1,\ldots,\lnec A_n\vdash\lnec (A_{n+1}\cond B)}{
      \infer{A_1,\ldots,A_n\vdash A_{n+1}\cond B}{
        A_1,\ldots,A_n,A_{n+1}\vdash B}}}
& \infer{\lnec A_{n+1}\vdash\lnec A_{n+1}}{}}
\]
bestätigt.\,\qedsymbol
\end{Beweis}

\begin{Satz} Mit Axiom \ref{DNE} (DN) gilt $\lnec\neg A\bicond\neg\lpos A$
im System~K.
\end{Satz}
\begin{Beweis}
Mit dem Baum
\[
\infer{\vdash \lnec\neg A\bicond\neg\lpos A}{
  \infer[\infernote{Def. \ref{def:possibly}}]{\vdash\lnec\neg A\cond\neg\lpos A}{
    \infer[\infernote{$B:=\lnec\neg A$}\quad]{\vdash\lnec\neg A\cond\neg\neg\lnec\neg A}{
      \infer{\vdash B\cond\neg\neg B}{}}}
& \infer[\infernote{Def. \ref{def:possibly}}]{\vdash\neg\lpos A\cond\lnec\neg A}{
    \infer[\infernote{$B:=\lnec\neg A$}]{\vdash\neg\neg\lnec\neg A\cond\lnec\neg A}{
      \infer[\infernote{DN}]{\vdash\neg\neg B\cond B}{}}}}
\]
ist die Ableitung gefunden.\,\qedsymbol
\end{Beweis}

\begin{Satz}
Mit Axiom \ref{DNE} (DN) gilt $\lpos\neg A\bicond\neg\lnec A$ im System~K.
\end{Satz}
\begin{Beweis} Der Beweisbaum
\[
\infer{\vdash\lpos\neg A\bicond\neg\lnec A}{
  \infer[\infernote{Def. \ref{def:possibly}}]{\vdash\lpos\neg A\cond\neg\lnec A}{
    \infer[\infernote{Kontraposition}]{\vdash\neg\lnec\neg\neg A\cond\neg\lnec A}{
      \infer[\infernote{K}]{\vdash\lnec A\cond\lnec\neg\neg A}{
        \infer[\infernote{Nezessisierung}]{\vdash\lnec (A\cond\neg\neg A)}{
          \infer{\vdash A\cond\neg\neg A}{}
        }
      }
    }}
& \infer[\infernote{Def. \ref{def:possibly}}]{\vdash\neg\lnec A\cond\lpos\neg A}{
    \infer[\infernote{Kontraposition}]{\vdash\neg\lnec A\cond\neg\lnec\neg\neg A}{
      \infer[\infernote{K}]{\vdash\lnec\neg\neg A\cond\lnec A}{
        \infer[\infernote{Nezessisierung}]{\vdash\lnec(\neg\neg A\cond A)}{
          \infer[\infernote{DN}]{\vdash\neg\neg A\cond A}{}}}}}}
\]
zeigt die Ableitung.\,\qedsymbol
\end{Beweis}

\begin{Satz}
Im System K gilt $\lnec (A\cond B)\cond (\lpos A\cond\lpos B)$.
\end{Satz}
\begin{Beweis} Mit dem Baum
\[
\infer{\vdash\lnec (A\cond B)\cond (\lpos A\cond\lpos B)}{
  \infer[\infernote{Def. \ref{def:possibly}}]{\lnec (A\cond B)\vdash\lpos A\cond\lpos B}{
    \infer[\infernote{Kontraposition}]{\lnec (A\cond B)\vdash\neg\lnec\neg A\cond\neg\lnec\neg B}{
      \infer[\infernote{K}]{\lnec (A\cond B)\vdash\lnec\neg B\cond\lnec\neg A}{
        \infer{\lnec (A\cond B)\vdash\lnec (\neg B\cond\neg A)}{
          \infer{\lnec (A\cond B)\vdash\lnec (A\cond B)}{}
        & \infer[\infernote{K}]{\vdash\lnec (A\cond B)\cond\lnec (\neg B\cond\neg A)}{
            \infer[\infernote{Nezessisierung}]{\vdash\lnec ((A\cond B)\cond(\neg B\cond\neg A))}{
              \infer[\infernote{Kontraposition}]{\vdash (A\cond B)\cond (\neg B\cond\neg A)}{}}}}}}}}
\]
ist die Ableitung gefunden.\,\qedsymbol
\end{Beweis}

\begin{Satz}
Im System K gilt $\lpos (A\land B)\cond\lpos A$
und $\lpos (A\land B)\cond\lpos B$.\\
Mithin gilt $\lpos (A\land B)\cond\lpos A\land\lpos B$.
\end{Satz}
\begin{Beweis} Der Beweisbaum:
\[
\begin{array}{l@{\qquad\quad}l}
\infer[\infernote{Def. \ref{def:possibly}}]{\vdash\lpos (A\land B)\cond\lpos A}{
  \infer[\infernote{Kontraposition}]{\vdash\neg\lnec\neg (A\land B)\cond\neg\lnec\neg A}{
    \infer[\infernote{K}]{\vdash\lnec\neg A\cond\lnec\neg (A\land B)}{
      \infer[\infernote{Nezessisierung}]{\vdash\lnec(\neg A\cond\neg (A\land B))}{
        \infer[\infernote{Kontraposition}]{\vdash\neg A\cond\neg (A\land B)}{
          \infer{\vdash A\land B\cond A}{}}}}}}
&
\infer{\vdash\lpos (A\land B)\cond\lpos A\land\lpos B}{
  \infer{\lpos (A\land B)\vdash\lpos A\land\lpos B}{
    \infer{\lpos (A\land B)\vdash\lpos A}{
      \infer{\vdash\lpos (A\land B)\cond\lpos A}{}}
  & \infer{\lpos (A\land B)\vdash\lpos B}{
      \infer{\vdash\lpos (A\land B)\cond\lpos B}{}}}}
\end{array}
\]
Der Beweis der zweiten Formel verläuft analog.\,\qedsymbol
\end{Beweis}

\subsection{System T}

\begin{Satz}
Mit Axiom \ref{DNE} (DN) ist
die Hinzunahme von Axiom \ref{axiom-T} (T) zum System~K äquivalent
zur Hinzunahme des Schemas $\vdash A\cond\lpos A$.
\end{Satz}
\begin{Beweis} Die Ableitungen
\[
\begin{array}{l@{\qquad}l}
\infer{\vdash A\cond\lpos A}{
  \infer{A\vdash\lpos A}{
    \infer[\infernote{Def. \ref{def:possibly}}]{\vdash\neg\neg A\cond\lpos A}{
      \infer{\vdash\neg\neg A\cond\neg\lnec\neg A}{
        \infer[\infernote{$B:=\neg A$}]{\vdash\lnec\neg A\cond\neg A}{
          \vdash\lnec B\cond B}}}
  & \infer{A\vdash\neg\neg A}{}}}
&
\infer{\vdash\lnec A\cond A}{
  \infer[\infernote{DN}]{\lnec A\vdash A}{
    \infer{\lnec A\vdash\neg\neg A}{
      \infer{\vdash\neg\lpos\neg A\cond\neg\neg A}{
        \infer[\infernote{$B:=\neg A$}]{\vdash\neg A\cond\lpos\neg A}{
          \vdash B\cond\lpos B}}
    & \infer[\infernote{Def. \ref{def:possibly}}]{\lnec A\vdash\neg\lpos\neg A}{
        \infer{\lnec A\vdash\neg\neg\lnec\neg\neg A}{
          \infer[\infernote{N*}]{\lnec A\vdash\lnec\neg\neg A}{
            \infer{A\vdash\neg\neg A}{}}}}}}}
\end{array}
\]
bestätigen das Gesagte.\,\qedsymbol
\end{Beweis}

\newpage
\section{Metatheorie der Logik}
\subsection{Klassische Aussagenlogik}

\begin{Definition}[Ableitbare Formel]\newlinefirst
Eine Theorie sei eine beliebige Menge von Formeln.
Es sei $T$ eine Theorie. Wir nennen eine Formel $\varphi$ ableitbar aus
$T$, kurz $T\vdash\varphi$, wenn es eine endliche Teilmenge
$\Gamma\subseteq T$ gibt, so dass die Sequenz $\Gamma\succ\varphi$
ableitbar ist.
\end{Definition}
\begin{Satz}\label{derivable-iff}
Sei $\Gamma$ endlich. Es gilt $\Gamma\vdash\varphi$ genau dann,
wenn $\Gamma\succ\varphi$ ableitbar ist, kurz
\[(\Gamma\vdash\varphi) \iff (\vdash \Gamma\succ\varphi).\]
\end{Satz}
\begin{Beweis}
Es gelte $\Gamma\vdash\varphi$. Somit existiert
$\Gamma'\subseteq\Gamma$ mit $\Gamma'\succ\varphi$.
Per Abschwächungsregel erhält man $\Gamma\succ\varphi$.

Es sei umgekehrt $\Gamma\succ\varphi$ ableitbar. Man wähle schlicht
$\Gamma\subseteq\Gamma$ als Zeuge für die endliche Teilmenge.
Es folgt $\Gamma\vdash\varphi$.\,\qedsymbol
\end{Beweis}
\begin{Satz}
Es gilt $\vdash\varphi$ genau dann, wenn $\vdash\succ\varphi$.
\end{Satz}
\begin{Beweis}
Korollar zu Satz \ref{derivable-iff} mit $\Gamma=\emptyset$.
\end{Beweis}

\begin{Definition}[Klassische Semantik]\label{def:sat}\newlinefirst
Sei $V$ die Menge der Variablen. Sei $I\colon V\to\{0,1\}$ eine
Belegung mit Wahrheitswerten. Man definiert $I\models\varphi$,
sprich »$I$ erfüllt $\varphi$«, rekursiv als
\begin{align*}
(I\models v) &\iff I(v),\\
(I\models\neg\varphi) &\iff \neg (I\models\varphi),\\
(I\models \varphi\land\psi) &\iff (I\models\varphi)\land (I\models\psi),\\
(I\models \varphi\lor\psi) &\iff (I\models\varphi)\lor (I\models\psi),\\
(I\models \varphi\Rightarrow\psi) &\iff (I\models\varphi)\Rightarrow (I\models\psi),\\
(I\models \varphi\Leftrightarrow\psi) &\iff (I\models\varphi)\Leftrightarrow (I\models\psi),
\end{align*}
wobei der jeweilige Junktor der rechten Seite per Wahrheitstafel
definiert ist.
\end{Definition}
\begin{Definition}\label{def:sat-context}
Man definiert außerdem noch die Abkürzung
\[(I\models\{\varphi_1,\ldots,\varphi_n\})
\defiff (I\models\varphi_1)\land\ldots\land (I\models\varphi_n).\]
\end{Definition}

\begin{Definition}[Gültigkeit in der klassischen Aussagenlogik]%
\label{def:valid}\newlinefirst
Man nennt eine Formel $\varphi$ gültig im Kontext $\Gamma$,
wenn sie erfüllt ist, sofern sämtliche Formeln des Kontextes
erfüllt sind. Kurz
\[(\Gamma\models\varphi) \defiff \forall I\colon
(I\models\Gamma) \Rightarrow (I\models\varphi).\]
\end{Definition}

\newpage
\begin{Satz}[Korrektheit des natürlichen Schließens]\newlinefirst
Aus $\Gamma\vdash\varphi$ folgt $\Gamma\models\varphi$.
\end{Satz}
\begin{Beweis}
Strukturelle Induktion über die Konstruktion von Beweisen.
Zu bestätigen ist zunächst das Axiom der
Grundsequenzen als Induktionsanfang. Des Weiteren ist die
Abschwächungsregel zu verifizieren.

\begin{tabular}{l@{\qquad\quad}l}
\strong{Zu den Grundsequenzen} & \strong{Zur Abschwächungsregel}\\[2pt]
$\infer[\infernote{Def. \ref{def:valid}}]{\Gamma\cup\{\varphi\}\models\varphi}{
  \infer[\infernote{$\sim$1}]{(I\models\Gamma\cup\{\varphi\})\Rightarrow (I\models\varphi)}{
    \infer{I\models\varphi}{
      \infer[\infernote{Def. \ref{def:sat-context}}]{(I\models\Gamma)\land (I\models\varphi)}{
        \infer[\infernote{1}]{I\models\Gamma\cup\{\varphi\}}{}}}}}$
&
$\infer[\infernote{$\sim$1}]{(\Gamma\models\varphi)\Rightarrow (\Gamma\cup\{\psi\}\models\varphi)}{
  \infer[\infernote{$\sim$2, Def. \ref{def:valid}}]{\Gamma\cup\{\psi\}\models\varphi}{
    \infer{I\models\varphi}{
      \infer[\infernote{Def. \ref{def:valid}}]{(I\models\Gamma)\Rightarrow (I\models\varphi)}{
        \infer[\infernote{1}]{\Gamma\models\varphi}{}}
    & \infer[\infernote{Def. \ref{def:sat-context}}]{I\models\Gamma}{
        \infer[\infernote{2}]{I\models\Gamma\cup\{\psi\}}{}}}}}$
\end{tabular}

\noindent
Nun zu den Schlussregeln für die Junktoren.

\begin{tabular}{l}
\strong{Zur Einführung der Konjunktion}\\[2pt]
$\infer[\infernote{Def. \ref{def:valid}}]{\Gamma\models\varphi\land\psi}{
  \infer[\infernote{$\sim$1}]{(I\models\Gamma)\Rightarrow (I\models\varphi\land\psi)}{
    \infer[\infernote{Def. \ref{def:sat}}]{I\models\varphi\land\psi}{
      \infer{(I\models\varphi)\land (I\models\psi)}{
        \infer{I\models\varphi}{
          \infer[\infernote{Def. \ref{def:valid}}]{(I\models\Gamma)\Rightarrow (I\models\varphi)}{
            \Gamma\models\varphi}
        & \infer[\infernote{1}]{I\models\Gamma}{}}
      & \infer{I\models\psi}{
          \infer[\infernote{Def. \ref{def:valid}}]{(I\models\Gamma)\Rightarrow (I\models\psi)}{
            \Gamma\models\psi}
        & \infer[\infernote{1}]{I\models\Gamma}{}}}}}}$
\end{tabular}

\begin{tabular}{l@{\qquad}l}
\strong{Zur Beseitigung der Konjunktion}
& \strong{Zur Einführung der Implikation}\\[2pt]
$\infer[\infernote{Def. \ref{def:valid}}]{\Gamma\models\varphi}{
  \infer[\infernote{$\sim$1}]{(I\models\Gamma)\Rightarrow (I\models\varphi)}{
    \infer{I\models\varphi}{
      \infer[\infernote{Def. \ref{def:sat}}]{(I\models\varphi)\land (I\models\psi)}{
        \infer{I\models\varphi\land\psi}{
          \infer[\infernote{Def. \ref{def:valid}}]{(I\models\Gamma)\Rightarrow (I\models\varphi\land\psi)}{
            \Gamma\models\varphi\land\psi}
        & \infer[\infernote{1}]{I\models\Gamma}{}}}}}}$
&
$\infer[\infernote{Def. \ref{def:valid}}]{\Gamma\models\varphi\Rightarrow\psi}{
  \infer[\infernote{$\sim$1}]{(I\models\Gamma)\Rightarrow (I\models\varphi\Rightarrow\psi)}{
    \infer[\infernote{Def. \ref{def:sat}}]{I\models\varphi\Rightarrow\psi}{
      \infer[\infernote{$\sim$2}]{(I\models\varphi)\Rightarrow (I\models\psi)}{
        \infer{I\models\psi}{
          \infer{(I\models\Gamma\cup\{\varphi\})\Rightarrow (I\models\psi)}{
            \Gamma\cup\{\varphi\}\models\psi}
        & \infer{I\models\Gamma\cup\{\varphi\}}{
            \infer[\infernote{1}]{I\models\Gamma}{}
          & \infer[\infernote{2}]{I\models\varphi}{}}
        }}}}}$
\end{tabular}

\noindent
Die Bestätigung der restlichen Schlussregeln verläuft analog.\,\qedsymbol
\end{Beweis}

\newpage
\subsection{Modallogik}
\begin{Definition}[Kripke-Modell]\newlinefirst
Ein Kripke-Modell ist ein Triplel $M=(W,R,V)$. Es ist $W$ eine nichtleere
Menge von Welten, auch Universum genannt. Es ist $R\subseteq W\times W$
eine beliebige Relation, auch Zugänglichkeitsrelation genannt. Es ist
$V$ eine Bewertungsfunktion, die je Welt $w\in W$ einer jeden logischen
Variable $v$ einen Wahrheitswert $V(w,v)\in\{0,1\}$ zuordnet.
\end{Definition}

\begin{Definition}[Kripke-Semantik der Modallogiken]%
\label{def:modal-sat}\newlinefirst
Sei $M=(W,R,V)$ ein Kripke-Modell. Man definiert $M,w\models\varphi$
rekursiv als
\begin{align*}
(M,w\models v) &\iff V(w,v),\\
(M,w\models\bot) &\iff 0,\\
(M,w\models\top) &\iff 1,\\
(M,w\models\neg\varphi) &\iff \neg (M,w\models\varphi),\\
(M,w\models\varphi\land\psi) &\iff (M,w\models\varphi)\land (M,w\models\psi),\\
(M,w\models\varphi\lor\psi) &\iff (M,w\models\varphi)\lor (M,w\models\psi),\\
(M,w\models\varphi\Rightarrow\psi) &\iff (M,w\models\varphi)\Rightarrow (M,w\models\psi),\\
(M,w\models\varphi\Leftrightarrow\psi) &\iff
  (M,w\models\varphi\Rightarrow\psi)\land (M,w\models\psi\Rightarrow\phi),\\
(M,w\models\lnec\varphi) &\iff\forall w'\in W\colon R(w,w')\Rightarrow (M,w'\models\varphi),\\
(M,w\models\lpos\psi) &\iff\exists w'\in W\colon R(w,w')\land (M,w'\models\varphi).
\end{align*}
\end{Definition}

\begin{Definition}\label{def:modal-sat-context}
Für einen Kontext $\Gamma = \{\varphi_1,\ldots,\varphi_n\}$ definiert man
\[(M,w\models\Gamma)\defiff
(M,w\models\varphi_1)\land\ldots\land (M,w\models\varphi_n).\]
\end{Definition}

\begin{Definition}[Gültigkeit bezüglich Kripke-Semantik]%
\label{def:modal-valid}\newlinefirst
Die $M=(W,R,V)$ seien Kripke-Modelle, wobei der Rahmen $(W,R)$ ggf. eine für
die jeweilige Modallogik spezifische Rahmenbedingung erfüllen soll.
Man definiert die Gültigkeit einer Formel $\varphi$ im Kontext $\Gamma$ als
\[(\Gamma\models\varphi)\defiff\forall M\colon\forall w\in W\colon
(M,w\models\Gamma)\Rightarrow (M,w\models\varphi).\]
\end{Definition}

\begin{Definition}[Rahmen]\newlinefirst
Ein Rahmen ist ein Paar $F=(W,R)$, wobei $W$ nichtleer und $R$ eine
binäre Relation auf $W$ ist. Ein Modell $M$ basiert genau dann auf
$F$, wenn $M=(W,R,\_)$ gilt.
\end{Definition}

\newpage
\begin{Satz}
Ein modallogisches System darf die Abschwächungsregel
\[\dfrac{\Gamma\vdash\psi}{\Gamma,\varphi\vdash\psi}.\]
enthalten.
\end{Satz}
\begin{Beweis}
Wir schreiben $w\models$ als Abkürzung für $M,w\models$. Der Baum
\[
\infer[\infernote{Def. \ref{def:modal-valid}}]{\Gamma\cup\{\varphi\}\models\psi}{
  \infer[\infernote{$\sim$1}]{\forall M\colon\forall w\in W\colon
  (w\models\Gamma\cup\{\varphi\})\Rightarrow (w\models\psi)}{
    \infer[\infernote{$\sim$2}]{(w\models\Gamma\cup\{\varphi\})\Rightarrow (w\models\psi)}{
      \infer{w\models\psi}{
        \infer{(w\models\Gamma)\Rightarrow (w\models\psi)}{
          \infer[\infernote{Def. \ref{def:modal-valid}}]%
          {\forall w\in W\colon (w\models\Gamma)\Rightarrow (w\models\psi)}{
            \Gamma\models\psi}
        & \infer[\infernote{1}]{w\in W}{}}
      & \infer[\infernote{Def. \ref{def:modal-sat-context}}]{w\models\Gamma}{
          \infer[\infernote{2}]{w\models\Gamma\cup\{\varphi\}}{}}}}}}
\]
bestätigt die Aussage\,\qedsymbol
\end{Beweis}

\begin{Satz}
Ein modallogisches System darf die Schlussregeln
\[\dfrac{\Gamma\vdash\varphi\qquad\Gamma\vdash\psi}{\Gamma\vdash\varphi\land\psi},
\qquad\dfrac{\Gamma\vdash\varphi\land\psi}{\Gamma\vdash\varphi},
\qquad\dfrac{\Gamma\vdash\varphi\land\psi}{\Gamma\vdash\psi}\]
enthalten.
\end{Satz}
\begin{Beweis}
Wir schreiben $w\models$ als Abkürzung für $M,w\models$.
Die Bäume
\[
\infer[\infernote{Def. \ref{def:modal-valid}}]{\Gamma\models\varphi\land\psi}{
  \infer[\infernote{$\sim$1}]{\forall M\colon\forall w\in W\colon
  (w\models\Gamma)\Rightarrow (w\models\varphi\land\psi)}{
    \infer[\infernote{$\sim$2}]{(w\models\Gamma)\Rightarrow (w\models\varphi\land\psi)}{
      \infer[\infernote{Def. \ref{def:modal-sat}}]{w\models\varphi\land\psi}{
        \infer{(w\models\varphi)\land (w\models\psi)}{
          \infer{w\models\varphi}{
            \infer{(w\models\Gamma)\Rightarrow (w\models\varphi)}{
              \infer[\infernote{Def. \ref{def:modal-valid}}]%
              {\forall w\in W\colon (w\models\Gamma)\Rightarrow (w\models\varphi)}{
                \Gamma\models\varphi}
            & \infer[\infernote{1}]{w\in W}{}}
          & \infer[\infernote{2}]{w\models\Gamma}{}}
        & \infer{w\models\psi}{\text{analog}}}}}}}
\]
und
\[
\infer[\infernote{Def. \ref{def:modal-valid}}]{\Gamma\models\varphi}{
  \infer[\infernote{$\sim$1}]{\forall M\colon\forall w\in W\colon
  (w\models\Gamma)\Rightarrow (w\models\varphi)}{
    \infer[\infernote{$\sim$2}]{(w\models\Gamma)\Rightarrow (w\models\varphi)}{
      \infer{w\models\varphi}{
        \infer[\infernote{Def. \ref{def:modal-sat}}]%
        {(w\models\varphi)\land (w\models\psi)}{
          \infer{w\models\varphi\land\psi}{
            \infer{(w\models\Gamma)\Rightarrow (w\models\varphi\land\psi)}{
              \infer[\infernote{Def. \ref{def:modal-valid}}]%
              {\forall w\in W\colon (w\models\Gamma)\Rightarrow (w\models\varphi\land\psi)}{
                \Gamma\models\varphi\land\psi}
            & \infer[\infernote{1}]{w\in W}{}
            }
          & \infer[\infernote{2}]{w\models\Gamma}{}}}}}}}
\]
bestätigen die Aussage.\,\qedsymbol
\end{Beweis}

\newpage
\begin{Satz}
Ein modallogisches System darf die Schlussregeln
\[\dfrac{\Gamma\vdash\varphi\Rightarrow\psi\qquad\Gamma\vdash\varphi}{\Gamma\vdash\psi},
\qquad\dfrac{\Gamma,\varphi\vdash\psi}{\Gamma\vdash\varphi\Rightarrow\psi}\]
enthalten.
\end{Satz}
\begin{Beweis}
Wir schreiben $w\models$ als Abkürzung für $M,w\models$.
Die Bäume
\begin{footnotesize}
\[
\infer{\Gamma\models\psi}{
  \infer[\infernote{$\sim$1}]{\forall W\colon\forall w\in W\colon (w\models\Gamma)\Rightarrow (w\models\psi)}{
    \infer[\infernote{$\sim$2}]{(w\models\Gamma)\Rightarrow (w\models\psi)}{
      \infer{w\models \psi}{
        \infer{(w\models \varphi)\Rightarrow (w\models \psi)}{
          \infer{w\models \varphi\Rightarrow\psi}{
            \infer{(w\models\Gamma)\Rightarrow (w\models \varphi\Rightarrow\psi)}{
              \infer{\forall w\in W\colon (w\models\Gamma)\Rightarrow (w\models\varphi\Rightarrow\psi)}{
                \Gamma\models \varphi\Rightarrow\psi}
            & \infer[\infernote{1}]{w\in W}{}}
          & \infer[\infernote{2}]{w\models\Gamma}{}}}
      & \infer{w\models\varphi}{
          \infer{(w\models\Gamma)\Rightarrow (w\models\varphi)}{
            \infer{\forall w\in W\colon (w\models\Gamma)\Rightarrow (w\models\varphi)}{
              \Gamma\models \varphi
            }
          & \infer[\infernote{1}]{w\in W}{}}
        & \infer[\infernote{2}]{w\models\Gamma}{}}}}}}
\]
\end{footnotesize}
und
\[
\infer[\infernote{Def. \ref{def:modal-valid}}]{\Gamma\models \varphi\Rightarrow\psi}{
  \infer[\infernote{$\sim$1}]{\forall W\colon\forall w\in W\colon
  (w\models\Gamma)\Rightarrow (w\models\varphi\Rightarrow\psi)}{
    \infer[\infernote{$\sim$2}]{(w\models\Gamma)\Rightarrow (w\models \varphi\Rightarrow\psi)}{
      \infer[\infernote{Def. \ref{def:modal-sat}}]{(w\models \varphi\Rightarrow\psi)}{
        \infer[\infernote{$\sim$3}]{(w\models\varphi)\Rightarrow (w\models \psi)}{
          \infer{w\models \psi}{
            \infer{(w\models\Gamma\cup\{\varphi\})\Rightarrow (w\models \psi)}{
              \infer[\infernote{Def. \ref{def:modal-valid}}]%
              {\forall w\in W\colon (w\models\Gamma\cup\{\varphi\})\Rightarrow (w\models\psi)}{
                \Gamma\cup\{\varphi\}\models\psi}
            & \infer[\infernote{1}]{w\in W}{}}
          & \infer[\infernote{Def. \ref{def:modal-sat-context}}]%
            {w\models\Gamma\cup\{\varphi\}}{
              \infer[\infernote{2}]{w\models\Gamma}{}
            & \infer[\infernote{3}]{w\models\varphi}{}}}}}}}}
\]
bestätigen die Aussage.\,\qedsymbol
\end{Beweis}

\begin{Satz}
Ein modallogisches System darf die Schlussregel
\[\dfrac{\vdash\varphi}{\vdash\lnec\varphi}\]
enthalten.
\end{Satz}
\begin{Beweis}
Wir schreiben $w\models$ als Abkürzung für $M,w\models$. Der Baum
\[
\infer[\infernote{Def. \ref{def:modal-valid}}]{\models\lnec\varphi}{
  \infer{\forall (W,R,V)\colon\forall w\in W\colon (w\models \lnec\varphi)}{
    \infer[\infernote{Def. \ref{def:modal-sat}}]{w\models\lnec\varphi}{
      \infer[\infernote{$\sim$1}]%
      {\forall w'\in W\colon R(w,w')\Rightarrow (w'\models\varphi)}{
        \infer{R(w,w')\Rightarrow (w'\models\varphi)}{
          \infer{w'\models\varphi}{
            \infer[\infernote{Def. \ref{def:modal-valid}}]%
            {\forall (W,R,V)\colon\forall w'\in W\colon (w'\models\varphi)}{
              \models\varphi}
          & \infer[\infernote{1}]{w'\in W}{}}}}}}
}
\]
bestätigt die Aussage.\,\qedsymbol
\end{Beweis}

\newpage
\begin{Satz}
Bereits im modallogischen System K gilt das Axiomenschema
\[\vdash \lnec(\varphi\Rightarrow\psi)\Rightarrow
(\lnec\varphi\Rightarrow\lnec\psi).\]
\end{Satz}
\begin{Beweis}
Es sei $w\models$ Abkürzung für $M,w\models$.
Per Def. \ref{def:modal-valid} ist
\[\forall (W,R,V)\colon\forall w\in W\colon
(w\models\lnec(\varphi\Rightarrow\psi)\Rightarrow
(\lnec\varphi\Rightarrow\lnec\psi))\]
zu bestätigen. Laut Def. \ref{def:modal-sat} lässt sich die
Implikation zurückführen auf
\[(w\models\lnec(\varphi\Rightarrow\psi))\Rightarrow
(w\models \lnec\varphi)\Rightarrow (w\models \lnec\psi).\]
Zu zeigen ist also, dass $\forall w'\in W\colon R(w,w')\Rightarrow (w'\models\psi)$ aus
\begin{gather*}
\forall w'\in W\colon R(w,w')\Rightarrow (w'\models\varphi\Rightarrow\psi),\\
\forall w'\in W\colon R(w,w')\Rightarrow (w'\models\varphi)
\end{gather*}
folgt. Aus den beiden Annahmen $w'\in W$ und $R(w,w')$ ist also
$w'\models\psi$ zu folgern. Mit den Annahmen ergibt sich
$w'\models\varphi$ und $(w'\models\varphi)\Rightarrow (w'\models\psi)$
aus den Prämissen. Infolge ergibt sich schließlich $w'\models\psi$.\,\qedsymbol
\end{Beweis}

\newpage
\subsection{Intuitionistische Logik}

\begin{Definition}[Intuitionistisches Kripke-Modell]\newlinefirst
Ein intuitionistisches Kripke-Modell ist ein Tripel $M=(W,\le,V)$.
Es ist $W$ eine nichtleere Menge. Es ist $\le$ eine
Quasiordnung (eine Halbordnung ginge auch). Es ist $V$ eine monotone
Funktion, die jeder Welt $w\in W$ und jeder logischen Variable $v$
einen Wahrheitswert $V(w,v)\in\{0,1\}$ zuordnet. Monoton heißt
\[w\le w'\land V(w,v)\implies V(w',v).\]
\end{Definition}

\begin{Definition}[Kripke-Semantik der intuitionistischen Logik]%
\label{def:intu-sat}\newlinefirst
Es sei $M=(W,\le,V)$ intuitionistisches Kripke-Modell. Zu einer Welt
$w\in W$ definiert man $M,w\models\varphi$ rekursiv als
\begin{align*}
(M,w\models v) &\iff V(w,v),\\
(M,w\models\bot) &\iff 0,\\
(M,w\models\top) &\iff 1,\\
(M,w\models\lnot\varphi) &\iff
  \forall w'\in W\colon w\le w'\Rightarrow \lnot (M,w'\models\varphi),\\
(M,w\models\varphi\land\psi) &\iff (M,w\models\varphi)\land (M,w\models\psi),\\
(M,w\models\varphi\lor\psi) &\iff (M,w\models\varphi)\lor (M,w\models\psi),\\
(M,w\models\varphi\Rightarrow\psi) &\iff
  \forall w'\in W\colon w\le w'\land (M,w'\models\varphi)\Rightarrow (M,w'\models\psi),\\
(M,w\models\varphi\Leftrightarrow\psi) &\iff
  (M,w\models\varphi\Rightarrow\psi)\land (M,w\models\psi\Rightarrow\varphi).
\end{align*}
\end{Definition}

\begin{Definition}
Für einen Kontext $\Gamma = \{\varphi_1,\ldots,\varphi_n\}$ definiert
man
\[(M,w\models\Gamma)\defiff (M,w\models\varphi_1)\land\ldots\land (M,w\models\varphi_n).\]
\end{Definition}

\begin{Definition}[Intuitionistische Gültigkeit]%
\label{def:intu-valid}\newlinefirst
Man definiert die Gültigkeit einer Formel $\varphi$ im Kontext $\Gamma$ als
\[(\Gamma\models\varphi)\defiff\forall M\colon\forall w\in W\colon (M,w\models\Gamma)\Rightarrow (M,w\models\varphi).\]
\end{Definition}

\begin{Satz}
Ex falso quodlibet, $\bot\Rightarrow\varphi$, ist ein intuitionistisch
gültiges Schema.
\end{Satz}
\begin{Beweis}
Weil die Metalogik klassisch ist, darf EFQ (ex falso quodlibet) in ihr
angewendet werden. Der Baum
\[
\infer[\infernote{Def. \ref{def:intu-valid}}]{\models\bot\Rightarrow\varphi}{
  \infer{\forall M\colon\forall w\in W\colon (M,w\models\bot\Rightarrow\varphi)}{
    \infer[\infernote{Def. \ref{def:intu-sat}}]{M,w\models\bot\Rightarrow\varphi}{
      \infer{\forall w'\in W\colon w\le w'\Rightarrow (M,w'\models\bot)\Rightarrow (M,w'\models\varphi)}{
        \infer{(M,w'\models\bot)\Rightarrow (M,w'\models\varphi)}{
          \infer[\infernote{Def. \ref{def:intu-sat}}]{(M,w'\models\bot)\Leftrightarrow 0}{}
        & \infer[\infernote{EFQ}]{0\Rightarrow (M,w'\models\varphi)}{}}}}}}
\]
bestätigt insofern die Aussage.\,\qedsymbol
\end{Beweis}

\newpage
\section{Mengenlehre}\index{Mengenlehre}
\subsection{Definitionen}

\begin{Definition}[Gleichheit von Mengen]
\label{def:seteq}\index{Gleichheit!von Mengen}
\[A=B \defiff \forall x\colon (x\in A\iff x\in B).\]
\end{Definition}

\begin{Definition}[Teilmenge]%
\label{def:subseteq}\index{Teilmenge}
\[A\subseteq B \defiff \forall x\colon (x\in A\implies x\in B).\]
\end{Definition}

\begin{Definition}[Beschreibende Angabe]\label{def:filter}
\[a\in\{x\mid P(x)\} \defiff P(a).\]
\end{Definition}

\begin{Definition}[Schnittmenge]%
\label{def:cap}\index{Schnittmenge}
\[A\cap B := \{x\mid x\in A\land x\in B\}.\]
\end{Definition}

\begin{Definition}[Vereinigungsmenge]%
\label{def:cup}\index{Vereinigungsmenge}
\[A\cup B := \{x\mid x\in A\lor x\in B\}.\]
\end{Definition}

\begin{Definition}[Differenzmenge]%
\label{def:set-diff}\index{Differenzmenge}
\[A\setminus B := \{x\mid x\in A\land x\notin B\}.\]
\end{Definition}

\begin{Definition}[Schnittmenge]\label{def:intersection}
\[\bigcap_{i\in I} A_i := \{x\mid \forall i{\in}I\colon x\in A_i\}
= \{x\mid \forall i\colon (i\in I\implies x\in A_i)\}.\]
\end{Definition}

\begin{Definition}[Vereinigungsmenge]\label{def:union}
\[\bigcup_{i\in I} A_i := \{x\mid \exists i{\in}I\colon x\in A_i\}
= \{x\mid \exists i\colon (i\in I\land x\in A_i)\}.\]
\end{Definition}

\begin{Definition}[Kartesisches Produkt]%
\label{def:cart}\index{kartesisches Produkt}
\[A\times B := \{(a,b)\mid a\in A\land b\in B\}
= \{t\mid\exists a\colon\exists b\colon t=(a,b)\land a\in A\land b\in B\}.\]
\end{Definition}

\subsection{Rechenregeln}

\begin{Satz}[Kommutativgesetze]\index{Kommutativgesetz!Mengen, boolesche Algebra}
Es gilt $A\cap B = B\cap A$ und $A\cup B = B\cup A$.
\end{Satz}

\begin{Beweis}
Man unternimmt die Umformung
\[x\in A\cap B
\stackrel{\text{(1)}}{\iff} x\in A\land x\in B
\stackrel{\text{(2)}}{\iff} x\in B\land x\in A
\stackrel{\text{(3)}}{\iff} x\in B\cap A.\]
Hierbei gilt (1), (3) laut Def. \ref{def:cap} in Verbindung mit
Def. \ref{def:filter}. Die Vertauschung (2) darf gemäß Satz \ref{bool-cl}
vorgenommen werden. Man erhält also
\[\forall x\colon (x\in A\cap B \iff x\in B\cap A).\]
Laut Def. \ref{def:seteq} ist diese Aussage äquivalent zu
$A\cap B = B\cap A$.
Bei der Vereinigung verläuft der Beweis analog.\,\qedsymbol
\end{Beweis}

\begin{Satz}[Assoziativgesetze]%
\index{Assoziativgesetz!Mengen, boolesche Algebra}\newlinefirst
Es gilt $A\cap (B\cap C) = (A\cap B)\cap C$
und $A\cup (B\cup C) = (A\cup B)\cup C$.
\end{Satz}

\begin{Beweis}
Es findet sich die Umformung
\begin{gather*}
x\in A\cap (B\cap C)
\stackrel{\text{(1)}}\iff x\in A\land x\in B\cap C
\stackrel{\text{(2)}}\iff x\in A\land (x\in B\land x\in C)\\
\stackrel{\text{(3)}}\iff (x\in A\land x\in B)\land x\in C
\stackrel{\text{(4)}}\iff x\in A\cap B\land x\in C
\stackrel{\text{(5)}}\iff x\in (A\cap B)\cap C.
\end{gather*}
Hierbei gilt (1), (2), (4), (5) laut Def. \ref{def:cap} in Verbindung
mit Def. \ref{def:filter}. Die Umformung (3) gilt, weil Konjunktionen
das Assoziativgesetz allgemein erfüllen. Man erhält also
\[\forall x\colon (x\in A\cap (B\cap C) \iff x\in (A\cap B)\cap C).\]
Per Def. \ref{def:seteq} ist diese Aussage äquivalent zur
Behauptung. Bei der Vereinigung verläuft der Beweis analog.\,\qedsymbol
\end{Beweis}

\begin{Satz}\label{cap-subseteq}
Es gilt $A\cap B\subseteq A$.
\end{Satz}
\begin{Beweis}
Laut Def. \ref{def:subseteq} und Def. \ref{def:cap} in Verbindung
mit Def. \ref{def:filter} darf $A\cap B\subseteq A$ zu
\[\forall x\colon (x\in A\land x\in B\implies x\in A)\]
umgeformt werden. Diese Aussage ist laut Satz \ref{from-conj}
immer erfüllt.
\end{Beweis}

\begin{Satz}\label{subseteq-char}
Es gilt $A\subseteq B\iff A\cap B=A$.
\end{Satz}
\begin{Beweis}
Aufgrund von Satz \ref{cap-subseteq} muss lediglich die Äquivalenz
von $A\subseteq B$ und $A\subseteq A\cap B$ gezeigt werden.
Laut Satz \ref{conj-premise} gilt die Formel
\[x\in A\Rightarrow x\in B\iff x\in A\Rightarrow x\in A\land x\in B.\]
Gemäß der Ersetzungsregel gilt also auch
\[(\forall x\colon x\in A\Rightarrow x\in B)
\iff (\forall x\colon x\in A\Rightarrow x\in A\land x\in B).\]
Per Def. \ref{def:subseteq} und Def. \ref{def:cup} in Verbindung
mit Def. \ref{def:filter} folgt nun die Behauptung.\,\qedsymbol
\end{Beweis}

\begin{Satz}\label{setdiff-dist}
Es gilt
\begin{align*}
(A\cup B)\setminus C &= (A\setminus C)\cup (B\setminus C),\\
(A\cap B)\setminus C &= (A\setminus C)\cap (B\setminus C).
\end{align*}
\end{Satz}
\begin{Beweis}
Es findet sich
\begin{align*}
x\in (A\cup B)\setminus C &\iff x\in A\cup B\land x\notin C
\iff (x\in A\lor x\in B)\land x\notin C\\
&\iff (x\in A\land x\notin C)\lor (x\in B\land x\notin C)\\
&\iff (x\in A\setminus C)\lor (x\in B\setminus C)
\iff x\in (A\setminus C)\cup (B\setminus C)
\end{align*}
und
\begin{align*}
x\in (A\cap B)\setminus C&\iff x\in A\cap B\land x\notin C
\iff x\in A\land x\in B\land x\notin C\\
&\iff (x\in A\land x\notin C)\land (x\in B\land x\notin C)\\
& \iff x\in A\setminus C\land x\in B\setminus C
\iff x\in (A\setminus C)\cap (B\setminus C).\,\qedsymbol
\end{align*}
\end{Beweis}

\begin{Satz}\label{subseteq-diff}
Es gilt $A\subseteq B\implies A\setminus C\subseteq B\setminus C$.
\end{Satz}
\begin{Beweis}[Beweis 1]
Die Aussage ist äquivalent zu
\[\forall x\colon x\in A\Rightarrow x\in B\vdash \forall x\colon x\in A\setminus C\Rightarrow x\in B\setminus C.\]
Zu zeigen verbleibt daher
\[x\in A, x\notin C, \forall x\colon x\in A\Rightarrow x\in B\vdash x\in B\land x\notin C.\]
Es ist $x\notin C$ bereits gegeben. Es folgt $x\in B$ per Modus ponens aus $x\in A$.\,\qedsymbol
\end{Beweis}
\begin{Beweis}[Beweis 2]
Mit $A\subseteq B\Leftrightarrow A\cup B=B$ und Satz \ref{setdiff-dist} findet sich
\[B\setminus C = (A\cup B)\setminus C = (A\setminus C)\cup (B\setminus C),\]
was wiederum zu $A\setminus C\subseteq B\setminus C$ äquivalent ist.\,\qedsymbol
\end{Beweis}

\begin{Satz}\label{eq-iff-all-iff}
Es gilt $a=b\iff \forall x\colon x=a\Leftrightarrow x=b$.
\end{Satz}

\begin{Beweis}
Für von links nach rechts findet sich:
\[
\infer{\vdash a=b\Rightarrow\forall x\colon x=a\Leftrightarrow x=b}{
  \infer{a=b\vdash\forall x\colon x=a\Leftrightarrow x=b}{
    \infer{a=b\vdash x=a\Leftrightarrow x=b}{
      \infer{a=b\vdash x=a\Rightarrow x=b}{
        \infer{a=b,x=a\vdash x=b}{
          \infer{x=a\vdash x=a}{} & \infer{a=b\vdash a=b}{}}}
    & \infer{a=b\vdash x=b\Rightarrow x=a}{
        \infer{a=b,x=b\vdash x=a}{
          \infer{x=b\vdash x=b}{} & \infer{a=b\vdash b=a}{}}}}}}
\]
Für von rechts nach links findet sich:
\[
\infer{\vdash(\forall x\colon x=a\Leftrightarrow x=b)\Rightarrow a=b}{
  \infer{\Gamma\vdash a=b}{
    \infer{\Gamma\vdash a=a\Rightarrow a=b}{
      \infer{\Gamma\vdash a=a\Leftrightarrow a=b}{
        \infer{\Gamma\vdash\forall x\colon x=a\Leftrightarrow x=b}{}}}
  & \infer{\vdash a=a}{}}}
\]
Hierbei steht $\Gamma$ als Abkürzung für
$\Gamma:=[\forall x\colon x=a\Leftrightarrow x=b]$.\,\qedsymbol
\end{Beweis}

\begin{Satz}
Es gilt $a=b\iff\{a\}=\{b\}$.
\end{Satz}

\begin{Beweis}
Man unternimmt die Umformung
\[\{a\}=\{b\}\iff (\forall x\colon x\in\{a\}\Leftrightarrow x\in\{b\})
\iff (\forall x\colon x=a\Leftrightarrow x=b).\]
Laut Satz \ref{eq-iff-all-iff} ist die rechte Aussage
äquivalent zu $a=b$.\;\qedsymbol
\end{Beweis}

\begin{Satz}\label{eq-substitution}
Es gilt $P(x)\iff (\exists y\colon P(y)\land x=y)$.
\end{Satz}
\begin{Beweis}
Für von links nach rechts wähle $y:=x$ als Zeugen:
\[
\infer{P(x)\vdash\exists y\colon P(y)\land x=y}{
  \infer{P(x)\vdash P(x)\land x=x}{
    \infer{P(x)\vdash P(x)}{}
  & \infer{\vdash x=x}{}}}
\]
Für von rechts nach links:
\[
\infer{\exists y\colon P(y)\land x=y\vdash P(x)}{
  \infer{\exists y\colon A\vdash\exists y\colon A}{}
& \infer{A\vdash P(x)}{
    \infer{A\vdash P(y)}{
      \infer{A\vdash P(y)\land x=y}{}}
  & \infer{A\vdash x=y}{
      \infer{A\vdash P(y)\land x=y}{}}}}
\]
Hierbei steht $A$ als Abkürzung für $P(y)\land x=y$.\,\qedsymbol
\end{Beweis}

\begin{Satz}\label{all-cart}
Es gilt
$(\forall t \in A{\times}B\colon P(t))
\Leftrightarrow (\forall a{\in}A\colon\forall b{\in}B\colon P(a,b))$.
\end{Satz}

\begin{Beweis}
Es findet sich die Umformung
\begin{gather*}
(\forall t \in A{\times}B\colon P(t))
\stackrel{\text{(1)}}{\iff} (\forall t\colon t\in A\times B\Rightarrow P(t))\\
\stackrel{\text{(2)}}{\iff}
(\forall t\colon (\exists a\colon\exists b\colon t=(a,b)\land a\in A\land b\in B)\Rightarrow P(t))\\
\stackrel{\text{(3)}}{\iff}
(\forall t\colon\forall a\colon\forall b\colon t=(a,b)\land a\in A\land b\in B\Rightarrow P(t))\\
\stackrel{\text{(4)}}{\iff}
(\forall a\colon\forall b\colon\forall t\colon t=(a,b)\Rightarrow a\in A\Rightarrow b\in B\Rightarrow P(t)).
\end{gather*}
Hierbei gilt (1) laut Def. \ref{def:bounded}, (2) laut Def. \ref{def:cart},
(3) unter doppelter Anwendung von Satz \ref{exists-implies-const}
und (4) unter doppelter Anwendung von Satz \ref{all-cl}, gefolgt
von doppelter Anwendung von Satz \ref{curry-impl}.
Auf der anderen Seite gilt die Umformung
\begin{align*}
(\forall a{\in}A\colon\forall b{\in}B\colon P(a,b))
&\stackrel{\text{(5)}}{\iff}
(\forall a\colon a\in A\Rightarrow\forall b\colon b\in B\Rightarrow P(a,b))\\
&\stackrel{\text{(6)}}{\iff}
(\forall a\colon\forall b\colon a\in A\Rightarrow b\in B\Rightarrow P(a,b)).
\end{align*}
Hierbei gilt (5) per Def. \ref{def:bounded} und (6) per Satz \ref{imp-dl}.
Wir definieren die Abkürzung
\[Q(t) := (a\in A\Rightarrow b\in B\Rightarrow P(t)).\]
Zu zeigen verbleibt also
\[Q(a,b)\iff (\forall t\colon t=(a,b)\Rightarrow Q(t)).\]
Diese Äquivalenz gilt laut Satz \ref{redundant-eq}.\,\qedsymbol
\end{Beweis}

\begin{Satz}\label{exists-cart}
Es gilt $(\exists t\in A{\times}B\colon P(t))
\Leftrightarrow (\exists a{\in}A\colon \exists b{\in}B\colon P(a,b))$.
\end{Satz}

\begin{Beweis}
Es findet sich die Umformung
\begin{gather*}
(\exists t{\in}A{\times}B\colon P(t))
\stackrel{\text{(1)}}{\iff} (\exists t\colon P(t)\land t\in A\times B)\\
\stackrel{\text{(2)}}{\iff}
(\exists t\colon P(t)\land \exists a\colon\exists b\colon t=(a,b)\land a\in A\land b\in B)\\
\stackrel{\text{(3)}}{\iff}
(\exists t\colon \exists a\colon \exists b\colon P(t)\land a\in A\land b\in B\land t=(a,b))\\
\stackrel{\text{(4)}}{\iff}
(\exists a{\in}A\colon \exists b{\in}B\colon \exists t\colon P(t)\land t=(a,b)).
\end{gather*}
Hierbei gilt (1), (4) per Def. \ref{def:bounded}, (2) per Def. \ref{def:cart}
und (3) per Satz. \ref{general-dl}.

Laut Satz \ref{eq-substitution} gilt nun
\[(\exists t\colon P(t)\land t=(a,b))\iff P(a,b).\,\qedsymbol\]
\end{Beweis}

\begin{Satz}\label{cup-cart}
Es gilt:
\[\bigcup_{t\in I\times J} A_t
= \bigcup_{i\in I}\bigcup_{j\in J} A_{ij}.\quad (t=(i,j))\]
\end{Satz}

\begin{Beweis}
Nach Def. \ref{def:union} und Satz \ref{exists-cart} gilt:
\begin{gather*}
x\in \bigcup_{t\in I\times J} A_t
\iff (\exists t\in I{\times J}\colon x\in A_t)
\iff (\exists i{\in}I\colon\exists j{\in}J\colon x\in A_{ij})\\
\iff (\exists i{\in}I\colon x\in \bigcup_{j\in J} A_{ij})
\iff x\in\bigcup_{i\in I}\bigcup_{j\in J} A_{ij}.
\end{gather*}
Nach Def. \ref{def:seteq} folgt die Behauptung.\,\qedsymbol
\end{Beweis}

\begin{Satz}
Es gilt:
\[\bigcup_{i\in I}\bigcup_{j\in J} A_{ij}
= \bigcup_{j\in J}\bigcup_{i\in I} A_{ij}.\]
\end{Satz}

\begin{Beweis}
Nach Def. \ref{def:union} und Satz \ref{bounded-exists-cl} gilt:
\begin{gather*}
x\in\bigcup_{i\in I}\bigcup_{j\in J} A_{ij}
\iff (\exists i{\in}I\colon x\in\bigcup_{j\in J} A_{ij})
\iff (\exists i{\in}I\colon\exists j{\in}J\colon x\in A_{ij})\\
\iff (\exists j{\in}J\colon\exists i{\in}I\colon x\in A_{ij})
\iff (\exists j{\in}J\colon x\in \bigcup_{i\in I}A_{ij})
\iff x\in\bigcup_{j\in J}\bigcup_{i\in I} A_{ij}.
\end{gather*}
Nach Def. \ref{def:seteq} folgt die Behauptung.\,\qedsymbol
\end{Beweis}

\begin{Satz}
Sei $M$ eine beliebige Menge. Die Relation $A\subseteq B$ macht die
Potenzmenge von $M$ zu einer partiell geordneten.
Im Einzelnen gilt%
\begin{align*}
\text{(1)}\quad & A\subseteq A, && \text{(Reflexivität)}\\
\text{(2)}\quad & A\subseteq B\land B\subseteq A \implies A = B, && \text{(Antisymmetrie)}\\
\text{(3)}\quad & A\subseteq B\land B\subseteq C \implies A\subseteq C. && \text{(Transitivität)}
\end{align*}
\end{Satz}
\begin{Beweis}
Jeweils Def. \ref{def:subseteq} nutzen.

Zu (1). Die Aussage $A\subseteq A$ ist 
äquivalent zu $\forall x\colon (x\in A\Rightarrow x\in A)$.
Eine Prämisse impliziert sich im Allgemeinen selbst.

Zu (2). Es findet sich die äquivalente Umformung
\begin{gather*}
A\subseteq B\land B\subseteq A\iff
(\forall x\colon x\in A\Rightarrow x\in B)
\land (\forall x\colon x\in B\Rightarrow x\in A)\\
\iff (\forall x\colon (x\in A\Rightarrow x\in B)\land (x\in B\Rightarrow x\in A))
\iff (\forall x\colon (x\in A\Leftrightarrow x\in B))\\
\iff A = B.
\end{gather*}

Zu (3). Zu zeigen ist $\forall x\colon (x\in A\Rightarrow x\in C)$.
Sei $x\in A$ fest, aber beliebig. Wegen $A\subseteq B$ muss $x\in B$
sein. Wegen $B\subseteq C$ muss infolge $x\in C$ sein.\,\qedsymbol
\end{Beweis}

\begin{Satz}
Die Aussage $A\subseteq B$ ist äquivalent zu $1_A\le 1_B$.
\end{Satz}
\begin{Beweis}
Es gelte $A\subseteq B$. Um $\forall x\colon 1_A(x)\le 1_B(x)$ zu
zeigen, wird eine Fallunterscheidung in drei Fälle vorgenommen.
Sei $x\notin B$. Dann ist $x\notin A$
und daher $1_A(x)=0$ und $1_B(x)=0$, womit $1_A(x)\le 1_B(x)$ gilt.
Sei $x\in A$. Dann ist $x\in B$, und daher $1_A(x)=1$ und $1_B(x)=1$,
womit $1_A(x)\le 1_B(x)$ gilt.
Sei $x\notin A$, aber $x\in B$. Dann ist $1_A(x)=0$ und $1_B(x)=1$,
womit $1_A(x)\le 1_B(x)$ gilt.

Es gelte $\forall x\colon 1_A(x)\le 1_B(x)$. Sei $x\in A$ fest,
aber beliebig. Wegen $1_A(x)=1$ ist $1\le 1_B(x)$. Weil somit
$1_B(x)\ne 0$ ist, verbleibt nur noch $1_B(x)=1$, was
gleichbedeutend mit $x\in B$ ist.\,\qedsymbol
\end{Beweis}

\begin{Definition}[Symmetrische Differenz]\label{def:symm-diff}\newlinefirst
\[A\triangle B := (A\setminus B)\cup (B\setminus A) = \{x\mid (x\in A)\oplus (x\in B)\}.\]
\end{Definition}
\begin{Satz}\label{symm-diff-assoc}\strong{[LEM]}
Es gilt $A\triangle (B\triangle C) = (A\triangle B)\triangle C$.
\end{Satz}
\begin{Beweis}
Mit Def. \ref{def:symm-diff} und Satz \ref{xor-assoc} findet sich
\begin{gather*}
x\in A\triangle (B\triangle C) \stackrel{\text{def}}\iff (x\in A)\oplus ((x\in B)\oplus (x\in C))\\
\iff ((x\in A)\oplus (x\in B))\oplus (x\in C)
\stackrel{\text{def}}\iff x\in (A\triangle B)\triangle C.\,\qedsymbol
\end{gather*}
\end{Beweis}

\newpage
\begin{Satz}\label{symm-diff-empty}
Es gilt $A\triangle A = \emptyset$ und $A\triangle\emptyset = A$.
\end{Satz}
\begin{Beweis}
Mit Def. \ref{def:symm-diff} findet sich
\begin{align*}
A\triangle A &\stackrel{\text{def}}= (A\setminus A)\cup (A\setminus A)
= \emptyset\cup\emptyset = \emptyset,\\
A\triangle\emptyset &\stackrel{\text{def}}=
(A\setminus\emptyset)\cup (\emptyset\setminus A)
= A\cup\emptyset = A.
\end{align*}
\end{Beweis}

\begin{Satz}\strong{[LEM]}\label{symm-diff-involution}
Es gilt $(A\triangle B)\triangle B = A$.
\end{Satz}
\begin{Beweis}
Man darf rechnen
\[(A\triangle B)\triangle B \stackrel{\text{(1)}}= A\triangle (B\triangle B)
\stackrel{\text{(2)}}= A\triangle\emptyset \stackrel{\text{(3)}}= A.\]
Hierbei gilt (1) laut Satz \ref{symm-diff-assoc}, und (2), (3) laut
Satz \ref{symm-diff-empty}.\,\qedsymbol
\end{Beweis}

\begin{Satz}\strong{[LEM]}
Es gilt $A\triangle C = B\triangle C\Rightarrow A = B$.
\end{Satz}
\begin{Beweis}
Es findet sich
\[A \stackrel{\text(1)}= (A\triangle C)\triangle C
\stackrel{\text{(2)}}= (B\triangle C)\triangle C
\stackrel{\text{(3)}}= B.\]
Hierbei gilt (1), (3) laut Satz \ref{symm-diff-involution},
und (2) laut Prämisse.\,\qedsymbol
\end{Beweis}

\newpage
\section{Relationen}\index{Relation}

\subsection{Allgemeine Gesetzmäßigkeiten}

\begin{Definition}[Relation]\newlinefirst
Zu zwei Mengen $X,Y$ bezeichnet man
jede Menge $R\subseteq X\times Y$ als Relation.
\end{Definition}
\begin{Definition}[Bildmenge]\label{def:relation-img}
Zu einer Relation $R$ wird die Menge
\[R(M) := \{y\mid\exists x\in M\colon (x,y)\in R\}\]
als Bildmenge von $M$ unter $R$ bezeichnet.
\end{Definition}

\begin{Satz}
Sei $R$ eine Relation und seien $A,B$ beliebige Mengen.\\
Es gilt $R(A\cup B) = R(A)\cup R(B)$.
\end{Satz}
\begin{Beweis}
Expansion mit Def. \ref{def:relation-img} führt zur Behauptung
\[(\exists x\in A\cup B\colon (x,y)\in R) \iff (\exists x\in A\colon (x,y)\in R)
\lor (\exists x\in B\colon (x,y)\in R).\]
Die linke Seite lässt sich gemäß Def. \ref{def:bounded},
Def. \ref{def:cup} und Satz \ref{exists-dl}
äquivalent in die rechte umformen.\;\qedsymbol
\end{Beweis}

\begin{Satz}
Sei $R$ eine Relation und seien $A,B$ beliebige Mengen.\\
Es gilt $R(A)\setminus R(B)\subseteq R(A\setminus B)$.
\end{Satz}
\begin{Beweis}
Expansion mit Def. \ref{def:relation-img} führt zur Behauptung
\[(\exists x\in A\colon (x,y)\in R)\land (\forall x\in B\colon (x,y)\notin R)
\implies \exists x\in A\setminus B\colon (x,y)\in R.\]
Laut der ersten Prämisse existiert ein $x\in A$ mit $(x,y)\in R$. Die
zweite Prämisse ist äquivalent zur Kontraposition $(x,y)\in R\Rightarrow x\notin B$.
Infolge ist $x\in A\setminus B$. Somit bezeugt $x$ die
Existenzaussage auf der rechten Seite.\,\qedsymbol
\end{Beweis}

\subsection{Äquivalenzrelationen}

\begin{Definition}[Äquivalenzrelation]\newlinefirst
Sei $M$ eine Menge. Man nennt $R\subseteq M\times M$, notiert als
$R(x,y) = (x\sim y)$, eine Äquivalenzrelation auf $M$, wenn für alle
$x,y,z\in M$ erfüllt ist:%
\begin{align*}
& x\sim x, && \text{(Reflexivität)}\\
& x\sim y\implies y\sim x, && \text{(Symmetrie)}\\
& x\sim y\land y\sim z\implies x\sim y. && \text{(Transitivität)}
\end{align*}
\end{Definition}

\begin{Definition}[Äquivalenzklasse]\newlinefirst
Sei $M$ eine Menge und $x\sim y$ eine Äquivalenzrelation für $x,y\in M$.
Die Menge%
\[[a] := \{x\in M\mid x\sim a\}\]
nennt man Äquivalenzklasse zum Repräsentanten $a\in M$.
\end{Definition}

\begin{Definition}[Quotientenmenge]\index{Quotientenmenge}\newlinefirst
Die Menge $M/{\sim} := \{A\mid \exists a\in M\colon A = [a]\}$
aller Äquivalenzklassen heißt Quotientenmenge von $M$ bezüglich der
Äquivalenzlation ${\sim}$.
\end{Definition}

\begin{Definition}[Quotientenabbildung]\newlinefirst
Die Abbildung $\pi\colon M\to M/{\sim}$ mit $\pi(x):=[x]$
heißt Quotientenabbildung.
\end{Definition}

\begin{Satz}[Äquivalenzrelation induziert disjunkte Zerlegung]\newlinefirst
Eine Menge wird durch eine auf ihr definierte Äquivalenzrelation
in paarweise disjunkte Äquivalenzklassen zerlegt.
\end{Satz}
\begin{Beweis}
Es ist zu zeigen, dass zwei unterschiedliche Äquivalenzklassen 
kein Element gemeinsam haben. Wir zeigen die Kontraposition, dass die
Existenz eines $x$ mit $x\in [a]$ und $x\in [b]$ bereits $[a]=[b]$ impliziert.
Laut Prämisse ist $x\sim a$ und $x\sim b$, und wegen der Transitivtät
infolge $a\sim b$, was äquivalent zu $[a]=[b]$ ist.

Zu bestätigen verbleibt noch, dass die Quotientenabbildung eine
surjektive ist. Dies ist wahr, weil $M/{\sim}$ gerade so definiert ist, 
dass direkt $M/{\sim}=\pi(M)$ gilt.\,\qedsymbol
\end{Beweis}

\begin{Satz}[Disjunkte Zerlegung induziert Äquivalenzrelation]%
\label{eq-relation-from-partition}\newlinefirst
Sei $M$ eine Menge. Die Familie $(A_k)$ der $A_k\subseteq M$ sei
eine Zerlegung von $M$ in paarweise disjunkte Mengen. Dann definiert
\[x\sim y\defiff \exists k\colon x\in A_k\land y\in A_k\]
eine Äquivalenzrelation.
\end{Satz}
\begin{Beweis}
Da die $A_k$ die Menge $M$ überdecken, muss für jedes $x\in M$
ein $k$ mit $x\in A_k$ existieren, womit die Reflexivität
$x\sim x$ erfüllt ist.

Die Symmetrie folgt unmittelbar aus der Kommutativität der Konjunktion.

Zur Transitivität. Seien $x,y,z$ fest, aber beliebig. Zudem seien die
Prämissen $x\sim y$ und $y\sim z$ erfüllt. Wir haben daher einen Zeugen
$i$ mit $x\in A_i$ und $y\in A_i$ und einen Zeugen $j$ mit $y\in A_j$
und $z\in A_j$. Infolge ist $y\in A_i\cap A_j$. Wegen
$A_i\cap A_j=\emptyset$ für $i\ne j$ muss $i=j$ sein. Deshalb ist
$i$ ein Zeuge für $\exists i\colon x\in A_i\land z\in A_i$, womit
$x\sim z$ gilt.\,\qedsymbol
\end{Beweis}

\begin{Satz}[Charakterisierung von Äquivalenzklassen]\newlinefirst
Sei $\sim$ eine Äquivalenzrelation auf der Menge $M$. Eine
Teilmenge $A\subseteq M$ ist genau dann eine Äquivalenzklasse, wenn%
\begin{gather*}
\text{(1)}\quad A\ne\emptyset,\\
\text{(2)}\quad x,y\in A\implies x\sim y,\\
\text{(3)}\quad x\in A\land y\in M\land x\sim y\implies y\in A.
\end{gather*}
\end{Satz}
\begin{Beweis}
Sei $A$ eine Äquivalenzklasse. Dann existiert definitionsgemäß
ein $a$, so dass $A=[a]$ gilt. Ergo ist $a\in A$, womit
$A\ne\emptyset$ sein muss. Mit $x,y\in A$ ergibt sich $[x]=[y]$,
was äquivalent zu $x\sim y$ ist. Sei nun $x\in A$ und $y$ irgendein
Element in $M$ mit $x\sim y$. Dies bedeutet $A=[x]=[y]$, womit
$y\in A$ sein muss.

Umgekehrt seien die drei Eigenschaften erfüllt. Zu zeigen ist,
dass ein Zeuge $a$ mit $A=[a]$ existiert. Weil $A$ gemäß (1)
nichtleer ist, muss ein Element $a\in A$ existieren. Für jedes
weitere Element $x\in A$ ergibt sich $x\sim a$ aufgrund (2),
also $x\in [a]$, womit wir $A\subseteq [a]$ haben.
Es verbleibt $[a]\subseteq A$ zu zeigen. Sei also $x\in [a]$.
Wir haben damit die Situation $a\in A$ und $x\sim a$, womit
laut (3) ebenso $x\in A$ sein muss.\,\qedsymbol
\end{Beweis}

\newpage
\begin{Satz}[Universelle Eigenschaft der Quotientenmenge]%
\label{quotient-universal-property}\newlinefirst
Sei $\pi\colon X\to X/{\sim}$ die Quotientenabbildung einer
Äquivalenzrelation. Es sei $f\colon X\to Y$ eine auf den
Äquivalenzklassen konstante Abbildung, das heißt, $f(a)=f(b)$ gelte
für $a\sim b$. Es gibt genau eine Abbildung $\overline f$, die der
Gleichung $f=\overline f\circ\pi$ genügt.
\end{Satz}
\begin{Beweis}
Der Zeuge $\overline f$ sei durch $\overline f(\pi(x)):=f(x)$
beschrieben, womit die Forderung
per Definition erfüllt wird. Es gilt allerdings zu zeigen, dass auf
diese Weise eine wohldefinierte Abbildung konstruiert wurde.
Das heißt, ihr Wert darf nicht vom gewählten Repräsentanten abhängen:
Für je zwei Elemente $a,b\in X$ muss die Implikation
\[\pi(a)=\pi(b) \implies f(a) = f(b)\]
gelten. Dies ist aber gerade die gegebene Konstanz auf den Klassen.

Es verbleibt zu zeigen, dass die Abbildung eindeutig bestimmt
ist. Sei dazu $\overline g$ eine weitere Abbildung mit der Eigenschaft
$f=\overline g\circ\pi$. Infolge gilt die Gleichung
$\overline g\circ\pi = \overline f\circ\pi$. Als surjektive Abbildung
ist $\pi$ laut Satz \ref{right-cancellative} rechtskürzbar, womit sich
$\overline g = \overline f$ ergibt.\,\qedsymbol
\end{Beweis}

\begin{Satz}
Es sei $f\colon X\to Y$ eine Abbildung und $a\sim b$ genau dann,
wenn $f(a)=f(b)$. Hierdurch ist eine Äquivalenzrelation definiert.
Ferner lässt sich auf diese Weise jede Äquivalenzrelation durch
passende Wahl von $f$ definieren.
\end{Satz}
\begin{Beweis}[Beweis 1]
Die Bestätigung der drei Axiome Reflexivität, Symmetrie und
Transitivität verläuft trivial, da es sich bei der Gleichheitsrelation
bereits um eine Äquivalenzrelation handelt.

Sei ${\sim}$ nun gegeben. Gesucht ist ein passendes $f$. Man wähle
$Y:=X/{\sim}$ und für $f$ die Quotientenabbildung, also $f(x):=[x]$.\,\qedsymbol
\end{Beweis}
\begin{Beweis}[Beweis 2]
Für $y,y'\in f(X)$ mit $y\ne y'$ sind die Urbilder $f^{-1}(\{y\})$ und
$f^{-1}(\{y'\})$ gemäß Satz \ref{disjoint-preimg} disjunkt. Die
Abbildung $f$ zerlegt $X$ auf diese Weise in paarweise disjunkte
Teile, wodurch laut Satz \ref{eq-relation-from-partition} eine
Äquivalenzrelation gegeben ist. Es ist dieselbe, wie definiert
wurde, denn genau innerhalb eines Urbilds verbleibt $f$ konstant.

Sei ${\sim}$ nun gegeben. Wir nehmen ein vollständiges
Repräsentantensystem und ordnen jedem Repräsentant $x$ einen
einzigartigen Wert $y$ zu. Es soll nun $f([x])=\{y\}$ gelten,
wodurch $f$ auf ganz $X$ definiert wird. Wir setzen
$Y:=f(X)$.\,\qedsymbol
\end{Beweis}

\begin{Satz}
Sei $f\colon X\to Y$ surjektiv und $a\sim b$ genau dann, wenn
$f(a)=f(b)$. Zwischen $Y$ und der Quotientenmenge $X/{\sim}$
besteht eine Bijektion.
\end{Satz}
\begin{Beweis}[Beweis 1]
Man definiert
\[\varphi\colon Y\to X/{\sim},\quad \varphi(y):=f^{-1}(\{y\}).\]
Zu jedem $y\in Y$ ist $\varphi(y)$ die Klasse all der $x\in X$, für
die $f(x)=y$ gilt. Weil $f$ surjektiv ist, ist sie nichtleer und damit
in $X/{\sim}$. Sei nun $x\in X$ beliebig, womit $[x]$ eine beliebige
Klasse ist. Weil $f$ surjektiv ist, muss ein $y$ mit $f(x)=y$ existieren,
womit $x\in f^{-1}(\{y\})$ gilt, also $[x] = \varphi(y)$. Ergo ist
$\varphi$ surjektiv.

Weiterhin darf man rechnen
\begin{gather*}
\varphi(y) = \varphi(y') \iff f^{-1}(\{y\}) = f^{-1}(\{y\})
\implies f(f^{-1}(\{y\})) = f(f^{-1}(\{y'\}))\\
\stackrel{\text{(1)}}\implies \{y\} = \{y'\} \implies y=y',
\end{gather*}
wobei (1) gemäß Satz \ref{sur-img-of-preimg} gilt. Somit muss
$\varphi$ injektiv sein.\,\qedsymbol
\end{Beweis}

\newpage
\begin{Beweis}[Beweis 2]
Durch $\varphi\colon Y\to X/{\sim}$ mit $\varphi(y):=f^{-1}(\{y\})$ ist
eine Bijektion definiert.

Man betrachte hierzu $\eta(y):=\{y\}$ und $(Ff)(B) = f^{-1}(B).$
Gemäß Satz \ref{preimg-induces-functor} handelt es sich bei $F$ um einen
kontravarianten Funktor. Das heißt, es gilt
$F(\id)=\id$ und $F(f\circ g) = (Fg)\circ (Ff).$ Weil $f$ surjektiv ist,
existiert eine Rechtsinverse $g$ mit $f\circ g = \id.$ Ergo gilt
\[\id = F(\id) = F(f\circ g) = (Fg)\circ (Ff).\]
Somit ist $Fg$ eine Linksinverse von $Ff.$ Folglich ist $Ff$ injektiv,
womit ebenso die Verkettung $\varphi = (Ff)\circ \eta$ injektiv ist,
denn $\eta$ ist ja eine Injektion.

Weiterhin ist die Äquivalenzrelation so definiert, dass die
Quotientenabbildung $\pi$ die Gleichung $\pi=\varphi\circ f$ erfüllt, denn
\begin{gather*}
(\varphi\circ f)(x) = f^{-1}(\{f(x)\}) = \{a\in X\mid f(a)\in \{f(x)\}\}\\
= \{a\in X\mid f(a)=f(b)\} = \{a\in X\mid a\sim x\} = \pi(x).
\end{gather*}
Weil $f$ surjektiv ist, gilt $Y=f(X).$ Infolge gilt
\[\varphi(Y) = \varphi(f(X)) = (\varphi\circ f)(X) = \pi(X) = X/{\sim}.\]
Ergo ist $\varphi$ surjektiv.\,\qedsymbol
\end{Beweis}
\begin{Beweis}[Beweis 3]
Die Abbildung $\varphi\colon Y\to X/{\sim}$ mit $\varphi(y):=f^{-1}(\{y\})$
ist bijektiv, wie die folgende Argumentation bestätigt.
Aufgrund der universellen Eigenschaft der Quotientenmenge, Satz
\ref{quotient-universal-property}, gibt es
genau eine Abbildung $\psi$ mit $f = \psi\circ\pi$. Die
Äquivalenzrelation wurde überdies so definiert, dass
$\pi = \varphi\circ f$ gilt. Daraus ergeben sich die Gleichungen
\[\psi\circ\varphi\circ f = f,\quad \varphi\circ\psi\circ\pi = \pi.\]
Als Surjektionen sind $f$ und $\pi$ rechtskürzbar, womit man
die Gleichungen $\psi\circ\varphi = \id$ und $\varphi\circ\psi = \id$
erhält. Somit ist $\psi$ die Linksinverse $\varphi$, womit
$\varphi$ injektiv sein muss, und auch die Rechtsinverse, womit
$\varphi$ surjektiv sein muss.\,\qedsymbol
\end{Beweis}

\subsection{Ordnungsrelationen}
\begin{Definition}[Halbordnung]\index{Halbordnung}\newlinefirst
Eine Relation $\le$ auf einer Menge $M$ heißt Halbordnung
und die Struktur $(M,\le)$ halbgeordnete Menge,
wenn die folgenden drei Axiome erfüllt sind:
\begin{align*}
&\forall x\in M\colon x\le x, &&\text{(Reflexivität)}\\
&\forall x,y,z\in M\colon x\le y\land y\le z\Rightarrow x\le z, &&\text{(Transitivität)}\\
&\forall x,y\in M\colon x\le y\land y\le x\Rightarrow x=y. &&\text{(Antisymmetrie)}
\end{align*}
\end{Definition}

\newpage
\section{Abbildungen}\index{Abbildung}
\subsection{Definitionen}

\begin{Definition}[Applikation]\label{def:app}
Für eine Abbildung $f$ ist
\[y=f(x)\defiff (x,y)\in G_f.\]
\end{Definition}

\begin{Definition}[Bildmenge]%
\label{def:img}\index{Bildmenge}\newlinefirst
Für eine Abbildung $f\colon X\to Y$ und $A\subseteq X$
wird die Menge
\[f(A) := \{y\mid \exists x\in A\colon y=f(x)\}
= \{y\mid \exists x\colon (x\in A\land y=f(x))\}\]
als Bildmenge von $A$ unter $f$ bezeichnet.
\end{Definition}

\begin{Definition}[Urbildmenge]\label{def:preimg}\index{Urbildmenge}
Für eine Abbildung $f\colon X\to Y$ wird
\[f^{-1}(B) := \{x\mid f(x)\in B\} = \{x\mid \exists y\in B\colon y=f(x)\}\]
als Urbildmenge von $B$ unter $f$ bezeichnet.
\end{Definition}

\begin{Definition}[Injektion]%
\label{def:inj}\index{Injektion}\newlinefirst
Eine Abbildung $f\colon X\to Y$ heißt genau dann injektiv, wenn gilt:%
\[\forall x_1\colon \forall x_2\colon (f(x_1)=f(x_2)\implies x_1=x_2)\]
bzw. äquivalent (Kontraposition)
\[\forall x_1\colon \forall x_2\colon (x_1\ne x_2\implies f(x_1)\ne f(x_2)).\]
\end{Definition}

\begin{Definition}[Surjektion]%
\label{def:sur}\index{Surjektion}\newlinefirst
Eine Abbildung $f\colon X\to Y$ heißt genau dann surjektiv, wenn gilt:%
\[Y\subseteq f(X).\]
\end{Definition}

\begin{Definition}[Verkettung]\label{def:composition}%
\index{Komposition}\index{Verkettung}\newlinefirst
Für Abbildungen $f\colon X\to Y$ und $g\colon Y\to Z$ heißt%
\[(g\circ f)\colon X\to Z,\quad (g\circ f)(x):=g(f(x))\]
Verkettung von $f$ und $g$, sprich »$g$ nach $f$«.
\end{Definition}

\subsection{Grundlagen}
\begin{Satz}[Gleichheit von Abbildungen]%
\label{feq}\index{Gleichheit!von Abbildungen}
Zwei Abbildungen $f\colon X\to Y$ und $g\colon X'\to Y'$ sind genau
dann gleich, kurz $f=g$, wenn $X=X'$ und $Y=Y'$ und%
\[\forall x\colon f(x)=g(x).\]
\end{Satz}

\begin{Beweis}
Nach Definition gilt $f=g$ genau dann, wenn $(G_f,X,Y)=(G_g,X',Y')$,
was äquivalent zu $G_f=G_g\land X=X'\land Y=Y'$ ist. Nach Def.
\ref{def:seteq} gilt%
\[G_f=G_g\iff \forall t\colon (t\in G_f\Leftrightarrow t\in G_g).\]
Nach Satz \ref{eq-iff-all-iff} und Def. \ref{def:app} gilt
\begin{align*}
&(\forall x\colon f(x)=g(x)) \iff (\forall x\colon\forall y\colon (y=f(x)\Leftrightarrow y=g(x)))\\
&\iff (\forall x\colon\forall y\colon((x,y)\in G_f\Leftrightarrow (x,y)\in G_g))
\iff \forall t\colon (t\in G_f\Leftrightarrow t\in G_g).
\end{align*}
Da die Quantifizerung auf $x\in X$, $y\in Y$ und $t\in X\times Y$
beschränkt ist, konnte im letzten Schritt Satz \ref{all-cart}
angewendet werden.\;\qedsymbol
\end{Beweis}

\begin{Satz} Für eine Abbildung $f$ gilt
\[f(x)\in A\cap B\iff f(x)\in A\land f(x)\in B.\]
\end{Satz}
\begin{Beweis}
Es gelte $f(x)\in A\cap B$. Dann existiert laut Definition ein
$y\in A\cap B$ mit $y=f(x)$, womit $y\in A$ und $y\in B$ gilt.
Folglich gilt $f(x)\in A$ und $f(x)\in B$.

Es gelte $f(x)\in A$ und $f(x)\in B$. Dann existiert laut Definition
ein $y\in A$ mit $y=f(x)$ und ein $y'\in B$ mit $y'=f(x)$.
Weil $f$ dem $x$ genau ein Bild zuordnet, muss $y=y'$ gelten.
Folglich gilt $y\in A\cap B$, und somit $f(x)\in A\cap B$.\,\qedsymbol
\end{Beweis}

\begin{Satz} Für eine Abbildung $f$ gilt
\[f(x)\in A\cup B\iff f(x)\in A\lor f(x)\in B.\]
\end{Satz}
\begin{Beweis} Es findet sich die äquivalente Umformung
\begin{align*}
f(x)\in A\cup B &\iff (\exists y\colon y\in A\cup B\land y=f(x))\\
&\iff (\exists y\colon (y\in A\lor y\in B)\land y = f(x))\\
&\iff (\exists y\colon y\in A\land y=f(x)\lor y\in B\land y = f(x))\\
&\iff (\exists y\colon y\in A\land y=f(x))\lor (\exists y\colon y\in B\land y = f(x))\\
&\iff f(x)\in A\lor f(x)\in B.\,\qedsymbol
\end{align*}
\end{Beweis}

\begin{Satz}[Distributivität der Urbildoperation]%
\label{preimg-dl}\index{Distributivgesetz!Urbildoperation}\newlinefirst
Für $f\colon X\to Y$ und beliebige Mengen $M_i$ gilt:%
\begin{align}
f^{-1}(M_1\cap M_2) &= f^{-1}(M_1)\cap f^{-1}(M_2),\\
f^{-1}(M_1\cup M_2) &= f^{-1}(M_1)\cup f^{-1}(M_2),\\
f^{-1}(\bigcap_{i\in I} M_i) &= \bigcap_{i\in I} f^{-1}(M_i),\\
f^{-1}(\bigcup_{i\in I} M_i) &= \bigcup_{i\in I} f^{-1}(M_i).
\end{align}
\end{Satz}

\begin{Beweis}
Nach Def. \ref{def:seteq} expandieren:
\[\forall x\colon [x\in f^{-1}(M_1\cap M_2)\iff x\in f^{-1}(M_1)\cap f^{-1}(M_2)].\]
Nach Def. \ref{def:preimg} und Def. \ref{def:cap}
zusammen mit Def. \ref{def:filter} gilt:
\begin{align*}
& x\in f^{-1}(M_1\cap M_2) \iff f(x)\in M_1\cap M_2
\iff f(x)\in M_1\land f(x)\in M_2\\
&\iff x\in f^{-1}(M_1)\land x\in f^{-1}(M_2)
\iff x\in f^{-1}(M_1)\cap f^{-1}(M_2).
\end{align*}
Für die Vereinigung ist das analog.

Schnitt von beliebig vielen Mengen.
Nach Def. \ref{def:seteq} expandieren:
\[\forall x\colon [x\in f^{-1}(\bigcap_{i\in I}M_i)
\iff x\in \bigcap_{i\in I} f^{-1}(M_i)].\]
Nach Def. \ref{def:preimg} und Def. \ref{def:intersection}
zusammen mit Def. \ref{def:filter} gilt:
\begin{align*}
& x\in f^{-1}(\bigcap_{i\in I} M_i)\iff f(x)\in\bigcap_{i\in I} M_i
\iff \forall i(i\in I\implies f(x)\in M_i)\\
&\iff \forall i(i\in I\implies x\in f^{-1}(M_i))
\iff x\in \bigcap_{i\in I} f^{-1}(M_i).
\end{align*}
Für die Vereinigung ist das analog.\;\qedsymbol
\end{Beweis}

\begin{Satz}[Distributivität der Bildoperation über die Vereinigung]\newlinefirst
Für $f\colon X\to Y$ und Mengen $M_i\subseteq X$ gilt:
\begin{align}
f(M_1\cup M_2) &= f(M_1)\cup f(M_2),\\
f(\bigcup_{i\in I} M_i) &= \bigcup_{i\in I} f(M_i).
\end{align}
\end{Satz}
\begin{Beweis}
Nach Def. \ref{def:seteq} expandieren:
\[\forall y\colon (y\in f(M_1\cup M_2)\iff y\in f(M_1)\cup f(M_2)).\]
Nach Def. \ref{def:img}, Def. \ref{def:cup},
Satz \ref{bool-dl} und Satz \ref{exists-dl} gilt:
\begin{align*}
&y\in f(M_1\cup M_2) \iff (\exists x\colon x\in M_1\cup M_2\land y=f(x))\\
&\iff (\exists x\colon (x\in M_1\lor x\in M_2)\land y=f(x))\\
&\iff (\exists x\colon x\in M_1\land y=f(x)\lor x\in M_2\land y=f(x))\\
&\iff (\exists x\colon x\in M_1\land y=f(x))\lor (\exists x\colon x\in M_2\land y=f(x))\\
&\iff y\in f(M_1)\lor y\in f(M_2) \iff y\in f(M_1)\cup f(M_2).
\end{align*}
Nach Def. \ref{def:seteq} expandieren:
\[\forall y\colon [y\in f(\bigcup_{i\in I} M_i)\iff y\in \bigcup_{i\in I} f(M_i)].\]
Nach Def. \ref{def:img}, Def. \ref{def:union},
Satz \ref{general-dl} und Satz \ref{exists-cl} gilt:
\begin{align*}
& y\in f(\bigcup_{i\in I} M_i)
\iff (\exists x\colon x\in\bigcup_{i\in I} M_i\land y=f(x))\\
&\iff (\exists x\colon (\exists i\colon i\in I\land x\in M_i)\land y=f(x))
\iff (\exists x\colon \exists i\colon i\in I\land x\in M_i\land y=f(x))\\
&\iff (\exists i\exists x\colon i\in I\land x\in M_i\land y=f(x))
\iff (\exists i\colon i\in I\land\exists x(x\in M_i\land y=f(x))\\
&\iff (\exists i\colon i\in I\land y\in f(M_i))
\iff y\in\bigcup_{i\in I} f(M_i).\;\qedsymbol
\end{align*}
\end{Beweis}

\begin{Satz}\label{img-cap-semi-dl}
Es gilt:
\begin{align}
f(M_1\cap M_2) &\subseteq f(M_1)\cap f(M_2),\\
f(\bigcap_{i\in I} M_i) &\subseteq \bigcap_{i\in I} f(M_i).
\end{align}
\end{Satz}

\begin{Beweis}
Nach Def. \ref{def:subseteq} expandieren:
\[\forall y\colon (y\in f(M_1\cap M_2)\implies y\in f(M_1)\cap f(M_2)).\]
Nach Def. \ref{def:img}, Def. \ref{def:cap}
und Satz. \ref{exists-asym-dl} gilt:
\begin{align*}
& y\in f(M_1\cap M_2) \iff (\exists x\colon x\in M_1\cap x\in M_2\land y=f(x))\\
&\iff (\exists x\colon x\in M_1\land x\in M_2\land y=f(x))\\
&\iff (\exists x\colon x\in M_1\land y=f(x)\land x\in M_2\land y=f(x))\\
&\implies (\exists x\colon x\in M_1\land y=f(x))\land (\exists x\colon x\in M_2\land y=f(x))\\
&\iff y\in f(M_1)\land y\in f(M_2)\iff y\in f(M_1)\cap f(M_2).
\end{align*}
Nach Def. \ref{def:subseteq} expandieren:
\[\forall y\colon (y\in f(\bigcap_{i\in I} M_i)\implies y\in \bigcap_{i\in I} f(M_i))\]
Nach Def. \ref{def:img} und Def. \ref{def:intersection} gilt:
\begin{align*}
& y\in f(\bigcap_{i\in I} M_i)\iff (\exists x\colon x\in\bigcap_{i\in I} M_i\land y=f(x))\\
& \iff (\exists x\colon (\forall i\colon i\in I\Rightarrow x\in M_i)\land y=f(x))\\
& \iff (\exists x\colon \forall i\colon i\in I\Rightarrow x\in M_i\land y=f(x))\\
& \implies (\forall i\colon\exists x\colon i\in I\Rightarrow x\in M_i\land y=f(x))\\
& \iff (\forall i\colon i\in I\Rightarrow \exists x\colon x\in M_i\land y=f(x))\\
& \iff (\forall i\colon i\in I\Rightarrow y\in f(M_i))
\iff y\in\bigcap_{i\in I} f(M_i).\;\qedsymbol
\end{align*}
\end{Beweis}

\begin{Satz}\label{disjoint-preimg}
Zwei disjunkte Mengen haben disjunkte Urbilder.
\end{Satz}
\begin{Beweis} Sei $A\cap B=\emptyset$. Gemäß Satz
\ref{preimg-dl} ist
\[f^{-1}(A)\cap f^{-1}(B) = f^{-1}(A\cap B) = f^{-1}(\emptyset)
= \emptyset.\;\qedsymbol\]
\end{Beweis}

\begin{Satz}
Es gilt $M\subseteq N\implies f^{-1}(M)\subseteq f^{-1}(N)$.
\end{Satz}
\begin{Beweis}[Beweis\;1]
Gemäß Satz \ref{subseteq-char} ist $M\subseteq N$ äquivalent zu
$M\cap N=M$. Man wendet die Urbildoperation $f^{-1}$ nun auf beide
Seiten der Gleichung an und erhält mittles Satz \ref{preimg-dl}
dann%
\[f^{-1}(M\cap N) = f^{-1}(M)\cap f^{-1}(N) = f^{-1}(M).\]
Nochmalige Anwendung von Satz \ref{subseteq-char} liefert
das gewünschte Resultat
\[f^{-1}(M)\subseteq f^{-1}(N).\;\qedsymbol\]
\end{Beweis}

\begin{Beweis}[Beweis\;2]
Die Expansion der Aussage bringt
\[(y\in M\Rightarrow y\in N)\implies (f(x)\in M\Rightarrow f(x)\in N).\]
Trivialerweise kann die Prämisse mit $y:=f(x)$ spezialisiert
werden werden.\;\qedsymbol
\end{Beweis}

\begin{Satz}\label{img-subseteq}
Es gilt $M\subseteq N\implies f(M)\subseteq f(N)$.
\end{Satz}
\begin{Beweis}
Gemäß Satz \ref{subseteq-char} ist $M\subseteq N$ äquivalent zu
$M\cap N=M$. Man wendet die Bildoperation nun auf beide Seiten
der Gleichung an und erhält mittels Satz \ref{img-cap-semi-dl}
dann%
\[f(M) = f(M\cap N)\subseteq f(M)\cap f(N).\]
Laut Satz \ref{cap-subseteq} ist folglich $f(M)=f(M)\cap f(N)$.
Nochmalige Anwendung von Satz \ref{subseteq-char} bringt das
gewünschte Resultat $f(M)\subseteq f(N)$.\;\qedsymbol
\end{Beweis}

\begin{Satz}\label{img-as-cup}
Es gilt:
\[f(M) = \bigcup_{x\in M} \{f(x)\}.\]
\end{Satz}

\begin{Beweis}
Nach Def. \ref{def:img} und Def. \ref{def:union} gilt:
\[y\in f(M) \iff (\exists x{\in}M\colon y=f(x))
\iff (\exists x{\in}M\colon y\in \{f(x)\})
\iff y\in\bigcup_{x\in M}\{f(x)\}.\]
Nach Def. \ref{def:seteq} folgt dann die Behauptung.\,\qedsymbol
\end{Beweis}

\begin{Satz}\label{preimg-chain}
Es gilt $(g\circ f)^{-1}(M) = f^{-1}(g^{-1}(M))$.
\end{Satz}
\begin{Beweis}
Nach Def. \ref{def:preimg} und Def. \ref{def:seteq}
expandieren und Def. \ref{def:filter} anwenden:%
\[(g\circ f)(x)\in M \iff f(x)\in\{y\mid g(y)\in M\}.\]
Links Def. \ref{def:composition} anwenden und rechts
nochmals Def. \ref{def:filter}:%
\[g(f(x))\in M \iff g(f(x))\in M.\;\qedsymbol\]
\end{Beweis}

\begin{Satz}\label{img-chain}
Es gilt $(g\circ f)(M) = g(f(M))$.
\end{Satz}
\begin{Beweis}
Nach Def. \ref{def:img} und Def. \ref{def:seteq} expandieren,
dann Def. \ref{def:filter} anwenden:%
\[(\exists x\colon x\in M\land z=(g\circ f)(x))
\iff (\exists y\colon y\in f(M)\land z=g(y)).\]
Die rechte Seite mit Def. \ref{def:img} expandieren und
Def. \ref{def:filter} anwenden. Unter Anwendung von
Satz \ref{general-dl} und Satz \ref{exists-cl} ergibt sich%
\begin{gather*}
(\exists y\colon (\exists x\colon x\in M\land y=f(x))\land z=g(y))\\
\iff (\exists y\colon\exists x\colon x\in M\land y=f(x)\land z=g(y))\\
\iff (\exists x\colon x\in M\land\exists y\colon y=f(x)\land z=g(y))\\
\iff (\exists x\colon x\in M\land z=g(f(x)))\\
\iff (\exists x\colon x\in M\land z=(g\circ f)(x)).\,\qedsymbol
\end{gather*}
\end{Beweis}

\begin{Satz}\label{left-inverse}
Sei $f\colon A\to B$ eine Abbildung und $A\ne\emptyset$. Man nennt
eine Funktion $g\colon B\to A$ mit $g\circ f=\id_A$
Linksinverse\index{Linksinverse}
zu $f$. Die Abbildung $f$ ist genau dann injektiv, wenn eine
Linksinverse zu $f$ existiert.
\end{Satz}
\begin{Beweis}
Sei $f$ injektiv. Man wähle ein $a\in A$, das wegen $A\ne\emptyset$
existieren muss. Man definiert nun $g\colon B\to A$ mit%
\[g(y):=\begin{cases}
x\;\text{wobei}\;y=f(x),\;\text{wenn}\;y\in f(A),\\
a\;\text{wenn}\;y\notin f(A).
\end{cases}\]
Diese Funktion ist eindeutig definiert, weil $f$ injektiv ist.
Gemäß ihrer Definition gilt $g(f(x))=x$, bzw. $g\circ f = \id$.

Sei nun eine Linksinverse $g$ mit $g\circ f=\id$ gegeben. Dann gilt
\[f(a)=f(b) \implies g(f(a))=g(f(b))\]
und
\[g(f(a))=g(f(b))
\iff (g\circ f)(a)=(g\circ f)(a)
\iff \id(a)=\id(b)
\iff a=b.\]
Es ergibt sich
\[f(a)=f(b)\implies a=b.\,\qedsymbol\]
\end{Beweis}

\begin{Satz}\label{preimg-compl}
Es gilt $f^{-1}(A^\comp)=f^{-1}(A)^\comp$ bzw.
$f(x)\in A^\comp\Leftrightarrow \neg f(x)\in A$.
\end{Satz}
\begin{Beweis} Zufolge der Expansion von Def. \ref{def:preimg} ist
\[(\exists y\in A^\comp\colon y=f(x))\iff \neg (\exists y\in A\colon y=f(x))
\iff (\forall y\in A\colon y\ne f(x))\]
zu zeigen. Weil $x$ ein Element des Definitionsbereichs ist, muss
der Funktionswert $f(x)$ in irgendeiner Menge liegen. Die Implikation
von rechts nach links. Weil $f(x)$ nicht in $A$ liegt, muss $f(x)$
in $A^\comp$ liegen. Die Implikaton von links nach rechts.
Gemäß Prämisse liegt $f(x)$ in $A^\comp$. Weil $x$ nur einen
Funktionswert besitzt, kann $f(x)$ nicht in $A$ liegen.\,\qedsymbol
\end{Beweis}

\begin{Satz}\label{preimg-setminus}
Für jede Abbildung $f$ gilt
$f^{-1}(A\setminus B) = f^{-1}(A)\setminus f^{-1}(B)$.
\end{Satz}
\begin{Beweis}[Beweis 1] Ergibt sich sofort gemäß Definition:
\begin{gather*}
f^{-1}(A)\setminus f^{-1}(B) = \{x\mid x\in f^{-1}(A)\land \neg x\in f^{-1}(B)\}\\
= \{x\mid f(x)\in A\land f(x)\notin B\}
= \{x\mid f(x)\in A\setminus B\}
= f^{-1}(A\setminus B).\,\qedsymbol
\end{gather*}
\end{Beweis}
\begin{Beweis}[Beweis 2] Gemäß Satz \ref{preimg-compl} und Satz
\ref{preimg-dl} ist
\begin{gather*}
f^{-1}(A)\setminus f^{-1}(B) = \{x\mid x\in f^{-1}(A)\land \neg x\in f^{-1}(B)\}\\
= \{x\mid x\in f^{-1}(A)\land x\in f^{-1}(B^\comp)\}
= f^{-1}(A)\cap f^{-1}(B^\comp)\\
= f^{-1}(A\cap B^\comp) = f^{-1}(A\setminus B).\,\qedsymbol
\end{gather*}
\end{Beweis}

\begin{Satz}\label{img-preimg}
Sei $f\colon X\to Y$, $A\subseteq X$, $B\subseteq Y$. Es gilt
$A\subseteq f^{-1}(f(A))$ und $f(f^{-1}(B)\subseteq B$.
\end{Satz}
\begin{Beweis}
Gemäß Definition bekommt man
\begin{gather*}
y\in f(f^{-1}(B))
\iff (\exists x\colon x\in f^{-1}(B)\land y=f(x))
\iff (\exists x\colon f(x)\in B\land y=f(x)),\\
x\in f^{-1}(f(A)) \iff f(x)\in f(A)
\iff (\exists a\in A\colon f(x)=f(a)).
\end{gather*}
Leicht zu bestätigen ist nun, dass
\begin{align*}
& \exists x\colon f(x)\in B\land y=f(x)\vdash y\in B, && (\text{Substitution $f(x):=y$ in $f(x)\in B$})\\
& x\in A\vdash\exists a\in A\colon f(x)=f(a).\,\qedsymbol && (\text{Zeuge $a:=x$})
\end{align*}
\end{Beweis}

\begin{Satz}\label{img-preimg-imgset}
Für jede Abbildung $f\colon X\to Y$ gilt $f(f^{-1}(B))=B$,
sofern $B\subseteq f(X)$ ist.
\end{Satz}
\begin{Beweis}
Laut Satz \ref{img-preimg} bleibt zu zeigen
\begin{gather*}
y\in B\implies (\exists x\in X\colon f(x)\in B\land y=f(x)).
\end{gather*}
Setzt man nun $B\subseteq f(X)$ voraus, dann ist $f(x)\in B$
allgemeingültig. Man bekommt
\begin{gather*}
(\exists x\in X\colon f(x)\in B\land y=f(x))
\iff (\exists x\in X\colon y=f(x))\iff y\in f(X).
\end{gather*}
Die Implikation $y\in B\Rightarrow y\in f(X)$ ist nun
wiederum definitionsgemäß äquivalent zu $B\subseteq f(X)$,
was Voraussetzung war.\;\qedsymbol
\end{Beweis}

\begin{Satz}
Sei $f\colon X\to Y$. Es gilt $B\subseteq f(X)$ genau dann,
wenn $\exists A\colon f(A)=B$.
\end{Satz}
\begin{Beweis}
Hat man ein $A$ mit $f(A)=B$, dann ist trivialerweise
$f(A)\subseteq f(X)$, also $B\subseteq f(X)$. Liegt umgekehrt
eine Menge $B\subseteq f(X)$ vor, dann kann man $A:=f^{-1}(B)$
setzen, nach Satz \ref{img-preimg-imgset} gilt dann $f(A)=B$.\,\qedsymbol
\end{Beweis}

\begin{Satz}\label{inj-img-setminus}
Ist $f$ injektiv, dann gilt $f(A\setminus B)=f(A)\setminus f(B)$.
\end{Satz}
\begin{Beweis}
Da $f$ injektiv ist, gibt es nach Satz \ref{left-inverse} eine
Linksinverse $f^{-1}$. Nach Satz \ref{img-chain} ist für eine
beliebige Menge $M$ die Gleichung
\[f^{-1}(f(M)) = (f^{-1}\circ f)(M) = \id(M) = M\]
erfüllt. Unter Heranziehung von Satz \ref{preimg-setminus}
bekommt man
\begin{gather*}
f^{-1}(f(A)\setminus f(B)) = f^{-1}(f(A))\setminus f^{-1}(f(B))
= \id(A)\setminus\id(B) = A\setminus B.
\end{gather*}
Wendet man nun auf beide Seiten der Gleichung $f$ an, dann ergibt
sich nach Satz \ref{img-preimg-imgset} das gesuchte Resultat
$f(A)\setminus f(B) = f(A\setminus B).\;\qedsymbol$
\end{Beweis}

\begin{Satz}
Ist $f\colon X\to Y$ bijektiv und $f^{-1}$ die
Umkehrabbildung von $f$, dann stimmt das Urbild $f^{-1}(M)$ mit der Bildmenge
von $M$ unter der Umkehrabbildung -- zur Unterscheidung $(f^{-1})(M)$
geschrieben -- überein.
\end{Satz}
\begin{Beweis}
Zu zeigen ist $(f^{-1})(M) = f^{-1}(M)$. Es findet sich die Umformung
\begin{align*}
x\in (f^{-1})(M) &\stackrel{\text{(1)}}\iff (\exists y\colon y\in M\land x=f^{-1}(y))
\stackrel{\text{(2)}}\iff (\exists y\colon f(x)\in M)\\
&\iff f(x)\in M\stackrel{\text{(3)}}\iff x\in f^{-1}(M).
\end{align*}
Hierbei gilt (1), (3) laut Definition. Es gilt (2), weil $f$ bijektiv
ist, womit die Gleichung $x=f^{-1}(y)$ äquivalent zu
$f(x)=f(f^{-1}(y))=y$ umgeformt werden darf.\,\qedsymbol
\end{Beweis}
Es genügt nicht, wenn $f$ injektiv ist. Als Gegenbeispiel setze
\[f\colon\{0\}\to\{0,1\},\quad f(x):=x.\]
Hier ist $f^{-1}(\{1\})=\emptyset$. Jedoch ist
$(f^{-1})(\{1\})=\{0\}$.

\begin{Satz}[Rechtskürzbarkeit von Surjektionen]%
\label{right-cancellative}\newlinefirst
Ist $f\colon X\to Y$ eine surjektive Abbildung, dann gilt die Implikation
\[g\circ f = h\circ f \implies g=h.\]
\end{Satz}
\begin{Beweis}
Laut der Prämisse und Satz \ref{feq} ist $g(f(x)) = h(f(x))$
für jedes $x\in X$. Da $f$ surjektiv ist, lässt sich zu
jedem $y\in Y$ ein $x\in X$ finden, so dass $y=f(x)$. Demnach ist
$g(y)=h(y)$ für alle $y\in Y$, denn man kann immer mindestens
ein $x$ finden, so dass sich $y:=f(x)$ substituieren lässt.
Laut Satz \ref{feq} ist daher $g=h$.\;\qedsymbol
\end{Beweis}

\begin{Satz}
Sei $f\colon X\to Y$ eine Abbildung. Man nennt $g\colon Y\to X$ eine
Rechtsinverse\index{Rechtsinverse} von $f$, falls $f\circ g = \id$ gilt.
Besitzt die Abbildung $f$ eine Rechtsinverse, dann ist sie surjektiv.
\end{Satz}
\begin{Beweis}
Per Definition ist die Implikation
\[(\exists g\colon f\circ g = \id)\implies (\forall y\in Y\colon\exists x\in X\colon y=f(x))\]
zu zeigen. Diese erhält man als Ableitung aus
\[(\exists g\colon f\circ g = \id), y\in Y\vdash\exists x\in X\colon y=f(x).\]
Wir haben also $g$ und $y$ zur Verfügung und können damit $g(y)$ konstruieren.
Es ist nun $x:=g(y)$ ein Zeuge der Existenzaussage, denn
\[f(x) = f(g(y)) = (f\circ g)(y) = \id(y) = y.\,\qedsymbol\]
\end{Beweis}

\newpage
\begin{Satz}
Genau dann ist $f\colon X\to Y$ injektiv, wenn
$\forall A\subseteq X\colon A=f^{-1}(f(A))$.
\end{Satz}
\begin{Beweis}
Sei $f$ injektiv. Laut Satz \ref{img-preimg} genügt es,
$f^{-1}(f(A))\subseteq A$ zu zeigen. Laut Definition gilt die Umformung
\[x\in f^{-1}(f(A))\iff f(x)\in f(A)\iff(\exists a\in A\colon f(x)=f(a)).\]
Zu zeigen ist insofern
\[f\;\text{injektiv}, (\exists a\in A\colon f(x)=f(a))\vdash x\in A.\]
Mit dem Zeugen $a\in A$ hat man nun $f(x)=f(a)$, aufgrund der Injektivität
somit $x=a$, womit $x\in A$ gelten muss.

Nun zur Umkehrung. Zu zeigen verbleibt
\[(\forall A\colon \forall x\colon x\in f^{-1}(f(A))\Rightarrow x\in A)
\vdash f\;\text{injektiv}.\]
Man setze $A:=\{a\}$. Nun ist $f(A)=f(\{a\}) = \{f(a)\}$. Es finden sich
die Umformungen
\begin{gather*}
x\in f^{-1}(f(A))\iff f(x)\in f(A)\iff f(x)\in\{f(a)\}\iff f(x)=f(a),\\
x\in A\iff x\in\{a\}\iff x=a.
\end{gather*}
Weil $a$ von nichts abhängt, darf generalisiert werden. Man erhält
\[\forall a\colon\forall x\colon f(x)=f(a)\Rightarrow x=a.\]
Diese Bedingung definiert die Injektivität.\,\qedsymbol
\end{Beweis}

\begin{Satz}
Eine Abbildung $f\colon X\to Y$ ist genau dann injektiv, wenn
$f^{-1}(\{y\})$ zu jedem $y\in Y$ höchstens ein Element enthält.
\end{Satz}
\begin{Beweis}
Dass eine Menge $A\subseteq X$ höchstens ein Element enthält,
lässt sich als $\exists x\in X\colon A\subseteq\{x\}$ formalisieren.
Zu beweisen ist also die Aussage
\[f\;\text{injektiv}\iff\forall y\in Y\colon\exists x\in X\colon f^{-1}(\{y\})\subseteq\{x\}.\]
Für die Implikation von links nach rechts ist
\[f\;\text{injektiv}\vdash\exists x\in X\colon \forall a\colon f(a)=y\Rightarrow a=x\]
zu bestätigen. Wir machen eine Fallunterscheidung. Im Fall $y\notin Y$
ist $f(a)=y$ nicht erfüllbar, womit aber $a=x$ per ex falso quodlibet
gilt. Insofern darf man sich ein beliebiges $x\in X$ aussuchen.
Im Fall $y\in Y$ gibt es ein $b\in X$ mit $y=f(b)$. Aufgrund
der Injektivität zieht $f(a)=f(b)$ nun $a=b$ nach sich. Man wählt
also $x:=b$ als Zeugen.

Für die Implikation von rechts nach links ist
\[(\forall y\in Y\colon\exists x\in X\colon
\forall a\colon f(a)=y\Rightarrow a=x)
\vdash f\;\text{injektiv},\]
zu bestätigen. Das heißt, zu zeigen verbleibt
\[f(a)=f(b), (\forall y\in Y\colon\exists x\in X\colon
\forall u\colon f(u)=y\Rightarrow u=x)
\vdash a=b.\]
Man setze $y:=f(b)$. Nun liegt ein $x$ vor, mit dem die Aussage
unter dem Existenzquantor gilt. Man setze einmal $u:=a$, womit man
mit $f(a)=f(b)$ die Gleichung $a=x$ erhält. Man setze nun $u:=b$.
Weil $f(b)=f(b)$ tautologisch ist, erhält man
die Gleichung $b=x$. Demzufolge besteht in $x$ der Brückenpfeiler,
um transitiv die gewünschte Gleichung $a=b$ zu erhalten.\,\qedsymbol
\end{Beweis}

\newpage
\begin{Satz}\label{sur-img-of-preimg}
Genau dann ist $f\colon X\to Y$ surjektiv,
wenn $\forall B\subseteq Y\colon B=f(f^{-1}(B))$.
\end{Satz}
\begin{Beweis}
Sei $f$ surjektiv. Aufgrund von Satz \ref{img-preimg} genügt es,
$B\subseteq f(f^{-1}(B))$ zu zeigen. Das heißt, zu bestätigen ist
\[y\in B\vdash\exists x\colon f(x)\in B\land y=f(x).\]
Weil $f$ surjektiv und $y\in Y$ ist, gibt es $x\in X$ mit $y=f(x)$.
Weil außerdem $y\in B$ Prämisse ist, gilt zusätzlich $f(x)\in B$.

Zur Umkehrung. Setze $B:=Y$. Dann ist $f^{-1}(B)=X$. Wir haben also
$Y=f(X)$, womit $f$ laut Definition surjektiv sein muss.\,\qedsymbol
\end{Beweis}

\begin{Satz}
Die Abbildung $F\colon\Abb(X,Y)\to\Abb(\mathcal P(X),\mathcal P(Y))$
mit $F(f)(A):=f(A)$ ist ein kovarianter Funktor. Das heißt, es gilt
$F(\id)=\id$ und $F(g\circ f) = F(g)\circ F(f)$.
\end{Satz}
\begin{Beweis}
Man darf rechnen
\[F(g\circ f)(A) = (g\circ f)(A) \stackrel{\text{(1)}}= g(f(A))
= F(g)(F(f)(A)) = (F(g)\circ F(f))(A),\]
wobei (1) gemäß Satz \ref{img-chain} gilt. Und es gilt
\[F(\id_X)(A) = \id_X(A) = \bigcup_{x\in A} \{\id_X(x)\}
= \bigcup_{x\in A}\{x\}
= A = \id_{\mathcal P(X)}(A).\,\qedsymbol\]
\end{Beweis}

\begin{Satz}\label{preimg-induces-functor}
Die Abbildung $F\colon\Abb(X,Y)\to\Abb(\mathcal P(Y),\mathcal P(X))$
mit $F(f)(B):=f^{-1}(B)$ ist ein kontravarianter Funktor. Das
heißt, es gilt $F(\id) = \id$ und $F(g\circ f) = F(f)\circ F(g)$.
\end{Satz}
\begin{Beweis}
Man darf rechnen
\[F(g\circ f)(B) = (g\circ f)^{-1}(B)
\stackrel{\text{(1)}}= f^{-1}(g^{-1}(B))
= F(f)(F(g)(B)) = (F(f)\circ F(g))(B),\]
wobei (1) gemäß Satz \ref{preimg-chain} gilt. Und es gilt
\[F(\id_X)(B) = \id_X^{-1}(B) = \{x\in X\mid x\in B\}
= B = \id_{\mathcal P(X)}(B).\,\qedsymbol\]
\end{Beweis}

\newpage
\subsection{Kardinalzahlen}

\begin{Axiom}[ACC: Abzählbares Auswahlaxiom]\label{acc}%
\index{abzählbares Auswahlaxiom}\index{Auswahlaxiom!abzählbares}
Sei $(A_n)_{n\in\N}$ eine Folge nichtleerer Mengen.
Dann existiert eine Funktion $f\colon\N\to\bigcup_{n\in\N} A_n$
mit $f(n)\in A_n$.
\end{Axiom}

\begin{Definition}[Gleichmächtigkeit]%
\label{def:equipotent}\index{gleichmaechtig@gleichmächtig}
Zwei Mengen $A$, $B$ heißen genau dann gleichmächtig, wenn
eine Bijektion $f\colon A\to B$ existiert.
\end{Definition}

\begin{Satz}
Es sei $M$ eine beliebige Menge. Die Potenzmenge $\mathcal P(M)$ ist zur
Menge $\Abb(M,\{0,1\})$ gleichmächtig.
\end{Satz}

\begin{Beweis}
Für eine Aussage $A$ sei
\[[A] := \begin{cases}
1&\text{wenn $A$ gilt},\\
0&\text{sonst}.
\end{cases}\]
Für eine Menge $A\subseteq M$ betrachte man nun die
Indikatorfunktion\index{Indikatorfunktion}
\[1_A\colon M\to\{0,1\},\quad 1_A(x):=[x\in A].\]
Die Abbildung
\[\varphi\colon\mathcal P(M)\to\Abb(M,\{0,1\}),\quad \varphi(A):=1_A\]
ist eine kanonische Bijektion.

\strong{Zur Injektivität.}
Nach Def. \ref{def:inj} muss gelten:
\[\varphi(A)=\varphi(B)\implies A=B,\quad\text{d.\,h.}\quad
1_A=1_B \implies A=B.\]
Nach Satz \ref{feq} und Def. \ref{def:seteq}
wird die Aussage expandiert zu:
\[(\forall x\colon 1_A(x)=1_B(x))\implies
(\forall x\colon x\in A\iff x\in B).\]
Es gilt aber nun:
\[1_A(x)=1_B(x)\iff [x\in A]=[x\in B] \iff (x\in A\iff x\in B).\]
\end{Beweis}
\strong{Zur Surjektivität.} Wir müssen nach Def. \ref{def:sur}
prüfen, dass $\Abb(M,\{0,1\})\subseteq \varphi(\mathcal P(M))$ gilt.
Expansion nach Def. \ref{def:subseteq} und Def. \ref{def:img} ergibt:
\[\forall f\colon (f\in\Abb(M,\{0,1\})\implies\exists A\in\mathcal P(M)\colon f=\varphi(A)).\]
Um dem Existenzquantor zu genügen, wähle
\[A := f^{-1}(\{1\}) = \{x\in M\mid f(x)\in \{1\}\} = \{x\in M\mid f(x)=1\}.\]
Es gilt $f=1_A$, denn
\[1_A(x) = [x\in A] = [x\in\{x\mid f(x)=1\}] = [f(x)=1] = f(x).\]
Da $\varphi$ eine Bijektion ist, müssen $\mathcal P(M)$ und $\Abb(M,\{0,1\})$
nach Def. \ref{def:equipotent} gleichmächtig
sein.\,\qedsymbol

\newpage
\begin{Satz}\label{countable-union-countable}
Man setze Axiom \ref{acc} (ACC) voraus.
Die Vereinigung von abzählbar vielen abzählbar unendlichen Mengen
ist abzählbar unendlich. Kurz $|\bigcup_{n\in\N} A_n| = |\N|$, wenn
$|A_n|=|\N|$ für jedes $n$.
\end{Satz}

\begin{Beweis}
Sei $B_n$ die Menge der Bijektionen aus $\Abb(\N,A_n)$. 
Nach Axiom \ref{acc} (ACC) kann aus jeder Menge $B_n$
eine Bijektion $f_n\colon\N\to A_n$ ausgewählt werden.
Man betrachte nun
\[\varphi\colon\N\times\N\to\bigcup_{n\in\N} A_n,\quad
\varphi(n,m):=f_n(m).\]
Die Abbildung $\varphi$ ist surjektiv, denn nach
Satz \ref{img-as-cup} und Satz \ref{cup-cart} gilt
\begin{gather*}
\varphi(\N\times\N) = \bigcup_{(n,m)\in\N{\times}\N} \{f_n(m)\}
= \bigcup_{n\in\N}\bigcup_{m\in\N} \{f_n(m)\}\\
= \bigcup_{n\in\N} f_n(\bigcup_{m\in\N} \{m\})
= \bigcup_{n\in\N} f_n(\N) = \bigcup_{n\in\N} A_n.
\end{gather*}
Daher gilt $|\bigcup_{n\in\N} A_n|\le |\N\times \N| = |\N|$.
Für eine beliebige der Bijektionen $f_n\in B_n$ lässt sich die Zielmenge
erweitern, so dass man eine Injektion $f\colon\N\to\bigcup_{n\in\N} A_n$
erhält. Daher ist auch $|\N|\le |\bigcup_{n\in\N} A_n|$. Nach dem
Satz von Cantor-Bernstein gilt also
$|\bigcup_{n\in\N} A_n|=|\N|$.\,\qedsymbol
\end{Beweis}

\begin{Satz}\label{countable-polynomial-ring}
Wenn $R$ abzählbar ist, dann ist auch der Polynomring $R[X]$ abzählbar.
\end{Satz}

\begin{Beweis}
Zu jedem Polynom vom Grad $n\ge 1$ gehört auf kanonische Weise
genau ein Tupel aus $M_n:=R^{n-1}\times R\setminus\{0\}$. Da $R$
abzählbar ist, sind auch $R^{n-1}$ und $R\setminus\{0\}$ abzählbar.
Dann ist auch $M_n$ abzählbar. Nach Satz \ref{countable-union-countable}
gilt
\[|R[X]| = 1+|\bigcup_{n\in\N} M_n| = 1+|\N| = |\N|.\,\qedsymbol\]
\end{Beweis}

\begin{Satz}\index{algebraische Zahlen!Kardinalität}
Es gibt nur abzählbar unendlich viele algebraische Zahlen.
\end{Satz}

\begin{Beweis}[Beweis 1]
Zu zeigen ist $|\A|=|\N|$ mit
\[\A := \{a\in\C\mid \exists p(p\in \Q[X]\setminus\{0\}\land p(a)=0)\}.\]
Dass $\A$ unendlich ist, ist leicht ersichtlich, denn schon jede
rationale Zahl $q$, von denen es unendlich viele gibt, ist Nullstelle
von $p(X):=X-q$ und daher algebraisch.

Ein Polynom vom Grad $n$ kann höchstens $n$ Nullstellen besitzen.
Nach Satz \ref{countable-polynomial-ring} gilt $\Q[X]=|\N|$.
Für $\Q[X]$ lässt sich also eine Abzählung angeben.
Bei dieser Abzählung lässt sich für jedes Polynom $p$ die Liste der
Nullstellen von $p$ einfügen. Streicht man alle Nullstellen, die schon
einmal vorkamen, dann erhält man eine Abzählung der algebraischen
Zahlen. Demnach gilt $|\A|=|\N|$.\,\qedsymbol
\end{Beweis}

\begin{Beweis}[Beweis 2]
Jedem $p=\sum_{k=0}^n a_kX^k$ lässt sich eine Höhe
$h:=n+\sum_{k=0}^n |a_k|$ zuordnen. Zu einer festen Höhe kann es nur
endlich viele Polynome $p\in\Z[X]$ geben, wodurch man eine Abzählung der
Polynome erhält, wenn für $h=1$, $h=2$, $h=3$ usw. jeweils die Liste
der Polynome eingefügt wird. Für jedes Polynom $p$ lässt sich die
Liste der Nullstellen von $p$ einfügen. Streicht man alle Nullstellen,
die schon einmal vorkamen, dann erhält man eine Abzählung der
algebraischen Zahlen.\,\qedsymbol
\end{Beweis}

\begin{Beweis}[Beweis 3]
Für $n\in\N$ sei
\[\begin{split}
A_n := \{x\in\A\mid \;&\text{$x$ ist Nullstelle eines
$p\in\Z[X]\setminus\{0\}$ mit $\deg(p)=n$,}\\
& \text{dessen Koeffizienten $a_k$ alle $|a_k|\le n$
erfüllen}\}.
\end{split}\]
Alle $A_n$ sind endlich und es gilt $\A=\bigcup_{n\in\N} A_n$.
Daher muss $|\A|\le |\N|$ sein.\,\qedsymbol
\end{Beweis}

\begin{Satz}[Satz und Def. Multiplikation von Kardinalzahlen]\newlinefirst
Die Operation $|X|\cdot |Y|:=|X\times Y|$ ist wohldefiniert.
\end{Satz}
\begin{Beweis}
Zu zeigen ist, dass $|X\times Y|=|X'\times Y'|$
aus $|X|=|X'|$ und $|Y|=|Y'|$ folgt. Nach Voraussetzung gibt
es Bijektionen $f_1\colon X\to X'$ und $f_2\colon Y\to Y'$. Gesucht
ist mindestens eine Bijektion $f\colon X\times Y\to X'\times Y'$.
Diese erhält man gemäß folgender Konstruktion:
\[f(x,y) := (f_1(x),f_2(y)).\]
Die Abbildung $f$ ist injektiv, denn
\begin{gather*}
f(x_1,y_1)=f(x_2,y_2) \iff (f_1(x_1),f_2(y_1))=(f_1(x_2),f_2(y_2))\\
\iff f_1(x_1)=f_1(x_2)\land f_2(y_1)=f_2(y_2)\iff x_1=x_2\land y_1=y_2\\
\iff (x_1,y_1)=(x_2,y_2).
\end{gather*}
Für die Surjektivität muss es für jedes $(x',y')$ mindestens
ein $(x,y)$ mit $(x',y')=f(x,y)$ geben. Die Konstruktion ergibt
\[(x',y')=(f_1(x),f_2(y))\iff x'=f_1(x)\land y'=f_2(y).\]
Man findet $x=f_1^{-1}(x')$ und $y=f_2^{-1}(y')$.

Die Umkehrabbildung ist gegeben gemäß
\begin{gather*}
f^{-1}(x',y')=f^{-1}((x',y')):=((f_1^{-1}\circ\pi_1)(x',y'),(f_2^{-1}\circ\pi_2)(x',y'))\\
= (f_1^{-1}(x'),f_2^{-1}(y')).
\end{gather*}
Mit $\pi_k$ ist die Projektion auf die $k$-te Komponente gemeint.\;\qedsymbol
\end{Beweis}

\begin{Satz}[Satz und Def. Addition von Kardinalzahlen]\newlinefirst
Für $X\cap Y=\emptyset$ ist $|X|+|Y|:=|X\cup Y|$ wohldefiniert.
Dies schließt den Spezialfall $|X|+|Y|:=|X\sqcup Y|$ mit
$X\sqcup Y:=(\{0\}\times X)\cup(\{1\}\times Y)$ ein.
\end{Satz}
\begin{Beweis}
Zu zeigen ist, dass $|X\cup Y|=|X'\cup Y'|$
aus $|X|=|X'|$ und $|Y|=|Y'|$ folgt. Nach Voraussetzung gibt
es Bijektionen $f_1\colon X\to X'$ und $f_2\colon Y\to Y'$, wobei
$X\cap Y=\emptyset$ und $X'\cap Y'=\emptyset$ gilt. Gesucht
ist mindestens eine Bijektion $f\colon X\cup Y\to X'\cup Y'$.
Diese erhält man gemäß folgender Konstruktion:
\[f(x):=\begin{cases}
f_1(x)&\text{für}\;x\in X,\\
f_2(x)&\text{für}\;x\in Y.
\end{cases}\]
Die Abbildung $f$ ist injektiv, denn entweder ist $x'\in X'$
und somit
\[x'=f(a)=f(b)\iff x'=f_1(a)=f_1(b)\iff a=b\]
oder $x'\in Y'$ und somit
\[x'=f(a)=f(b)\iff x'=f_2(a)=f_2(b)\iff a=b.\]
Zusammengefasst folgt $f(a)=f(b)\iff a=b$ für alle $a,b\in X\cup Y$.

Für die Surjektivität muss es für jedes $x'$ mindestens ein $x$ mit
$x'=f(x)$ geben. Entweder ist $x'\in X'$, dann ist $x'=f_1(x)$
und daher $x=f_1^{-1}(x')$. Oder es ist $x'\in Y'$, dann ist
$x'=f_2(x)$ und daher $x=f_2^{-1}(x')$.\;\qedsymbol
\end{Beweis}

\newpage
\begin{Satz}[Satz und Def. Potenz von Kardinalzahlen]\newlinefirst
Die Operation $|Y|^{|X|}:=|Y^X|$ ist wohldefiniert.
\end{Satz}
\begin{Beweis}
Zu zeigen ist, dass $|\Abb(X,Y)|=|\Abb(X',Y')|$
aus $|X|=|X'|$ und $|Y|=|Y'|$ folgt. Nach Voraussetzung gibt
es Bijektionen $f_1\colon X\to X'$ und $f_2\colon Y\to Y'$.
Gesucht ist eine Bijektion $F\colon\Abb(X,Y)\to\Abb(X',Y')$.
Diese erhält man gemäß folgender Konstruktion:
\[F(f) := f_2\circ f\circ f_1^{-1}.\]
Die Abbildung $F$ ist injektiv, da
\begin{gather*}
F(f)=F(g) \iff f_2\circ f\circ f_1^{-1} = f_2\circ f\circ f_1^{-1}
\iff f_2\circ f = f_2\circ g\iff f=g,
\end{gather*}
denn Bijektionen sind kürzbar. Für die Surjektivität muss
es für jedes $f'$ mindestens ein $f$ mit $f'=F(f)$ geben.
Das führt auf die Gleichung $f'=f_2\circ f\circ f_1^{-1}$.
Diese lässt sich Umformen zu $f_2^{-1}\circ f'=f\circ f_1^{-1}$.
Wendet man beide Seiten auf $f_1$ an, ergibt sich
$f=f_2^{-1}\circ f'\circ f_1$.\;\qedsymbol
\end{Beweis}

\begin{Definition}[Weniger mächtig]\newlinefirst
Eine Menge $A$ heißt weniger mächtig als eine Menge $B$, kurz
$|A|<|B|$, wenn es eine Injektion $A\to B$ gibt, aber keine
Bijektion $A\to B$.
\end{Definition}

\begin{Satz}[Satz von Cantor]\newlinefirst
Eine Menge ist stets weniger mächtig als ihre Potenzmenge.
Kurz $|A|<|\mathcal P(A)|$. 
\end{Satz}

\begin{Beweis}
Für das Vorhandensein einer Injektion $A\to\mathcal P(A)$ bietet $x\mapsto\{x\}$
einen schlichten Beleg. Nun wird aus der Annahme, es gäbe eine Surjektion
$f\colon A\to 2^A$, ein Widerspruch abgeleitet, womit erst recht
keine Bijektion bestehen kann. Erreicht wird dies nach Cantor durch ein
Diagonalargument. Dazu definiert man die Menge
\[D := \{x\in A\mid x\notin f(x)\}.\]
Weil $D\subseteq A$ ist, hat man $D\in\mathcal P(A)$. Angenommen, $f$ ist surjektiv.
Es liegt somit ein Zeuge $x\in A$ vor, so dass $f(x) = D$. Speziell muss also
$x\in f(x) \Leftrightarrow x\in D$
per Def. \ref{def:seteq} erfüllt sein. Man erhält weiterhin die Umformung
\[x\in D \stackrel{\text{(1)}}\iff x\in A\land x\notin f(x)
\stackrel{\text{(2)}}\iff x\notin f(x),\]
wobei (1) laut Definition von $D$ gilt, und (2), weil $x\in A$
ja vorliegt. Man gelangt zur Aussage
\[x\in f(x) \iff x\notin f(x),\]
die laut Satz \ref{non-eq-prop-neg} aber widersprüchlich ist.\,\qedsymbol
\end{Beweis}

\begin{Satz}
Die Menge der endlichen Teilmengen der natürlichen Zahlen ist
abzählbar.
\end{Satz}

\begin{Beweis}
Zu jeder Teilmenge $A\subseteq\N_0$ gehört genau eine Indikatorfunktion
\[1_A\colon\N_0\to\{0,1\},\quad 1_A(n) := [n\in A].\]
Weil die Indikatorfunktion die natürlichen Zahlen als Definitionsbereich
besitzt, handelt es sich um eine Folge, die wie jede Folge als formale Potenzreihe.
\[p_A(X) = \sum_{n=0}^\infty 1_A(n)X^n = \sum_{n\in A} X^n\]
dargestellt werden kann. Ist die Teilmenge $A$ eine endliche, besitzt
die Indiktorfunktion eine endliche Nichtnullstellenmenge, womit sich
$p_A(X)$ zu einem Polynom reduziert. Wegen $1_A(n)<2$, kann man
$1_A$ als Kodierung der Zahl $p_A(2)$ auffassen, nämlich als Binärdarstellung
der Zahl $p_A(2)$ im Stellenwertsystem zur Basis $X=2$. Deshalb fungiert
die Abbildung
\[f(A) := p_A(2) = \sum_{n\in A} 2^n\]
als kanonische Bijektion zwischen der Menge der endlichen Teilmengen der
natürlichen Zahlen und der natürlichen Zahlen. Sie ist injektiv
aufgrund der Eigenheiten von Stellenwertsystemen. Sie ist surjektiv,
weil $1_A$ laut Prämisse jede beliebige Binärdarstellung sein darf,
und man somit aufgrund der Eigenheiten von Stellenwertsystemen jede
beliebige natürliche Zahl erhält.\,\qedsymbol
\end{Beweis}
