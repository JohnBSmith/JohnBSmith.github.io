
\chapter{Grundlagen}
\section{Aussagenlogik}\index{Aussagenlogik}

\begin{Satz}[bool-dl: Distributivgesetze]%
\label{bool-dl}\index{Distributivgesetz!boolesche Algebra}
Es gilt:
\begin{align}
A\land (B\lor C) &\iff A\land B\lor A\land C,\\
A\lor (B\land C) &\iff (A\lor B)\land (A\lor C).
\end{align}
\end{Satz}

\section{Prädikatenlogik}\index{Praedikatenlogik@Prädikatenlogik}

\begin{Definition}[bounded: beschränkte Quantifizierung]%
\label{def:bounded}
\begin{align}
\forall x{\in}M\;(P(x)) &\defiff \forall x(x\in M\implies P(x)),\\
\exists x{\in}M\;(P(x)) &\defiff \exists x(x\in M\land P(x)).
\end{align}
\end{Definition}

\begin{Satz}[general-dl: allgemeine Distributivgesetze]%
\label{general-dl}
Es gilt:
\begin{align}
A\land \exists x(P(x)) &\iff \exists x(A\land P(x)),\\
A\lor \forall x(P(x)) &\iff \forall x(A\lor P(x)).
\end{align}
\end{Satz}

\begin{Satz}[exists-dl: Distributivgesetz]\label{exists-dl}
Es gilt:
\[\exists x(P(x)\lor Q(x)) \iff \exists x(P(x))\lor\exists x(Q(x)).\]
\end{Satz}

\begin{Satz}[exists-asym-dl: asymmetrisches Distributivgesetz]\label{exists-asym-dl}
Es gilt:
\[\exists x(P(x)\land Q(x)) \implies \exists x(P(x))\land\exists x(Q(x)).\]
\end{Satz}

\begin{Satz}\label{exists-implies-const}
Es gilt:
\[\forall x(P(x)\implies A) \iff \exists x(P(x))\implies A.\]
\end{Satz}

\begin{Satz}[exists-cl: Kommutativgesetz]\label{exists-cl}
Es gilt:
\[\exists x\exists y(P(x,y)) \iff \exists y\exists x(P(x,y)).\]
\end{Satz}

\begin{Satz}[all-cl: Kommutativgesetz]\label{all-cl}
Es gilt:
\[\forall x\forall y(P(x,y)) \iff \forall y\forall x(P(x,y)).\]
\end{Satz}

\begin{Satz}[bounded-general-dl: allgemeine Distributivgesetze]%
\label{bounded-general-dl}
Es gilt:
\begin{align}
A\land \exists x{\in}M\,(P(x)) &\iff \exists x{\in}M\,(A\land P(x)),\\
A\lor \forall x{\in}M\,(P(x)) &\iff \forall x{\in}M\,(A\lor P(x)).
\end{align}
\end{Satz}

\begin{Beweis}
Nach Def. \ref{def:bounded} (bounded)
und Satz \ref{general-dl} (general-dl) gilt:
\begin{align*}
&A\land \exists x{\in}M\,(P(x))
\iff A\land\exists x(x\in M\land P(x))
\iff \exists x(A\land x\in M\land P(x))\\
&\iff \exists x(x\in M\land A\land P(x))
\iff \exists x{\in}M\,(A\land P(x)).
\end{align*}
Nach Def. \ref{def:bounded} (bounded)
und Satz \ref{general-dl} (general-dl) gilt:
\begin{align*}
& A\lor\forall x{\in}M\,(P(x))
\iff A\lor\forall x(x\in M\implies P(x))
\iff A\lor\forall x(x\notin M\lor P(x))\\
&\iff \forall x(A\lor x\notin M\lor P(x))
\iff \forall x(x\in M\implies A\lor P(x))\\
&\iff \forall x{\in}M\,(A\lor P(x)).\;\qedsymbol
\end{align*}
\end{Beweis}

\begin{Satz}\label{bounded-exists-cl}
Es gilt:
\[\exists x{\in}A\;\exists y{\in}B\;(P(x,y))
\iff \exists y{\in}B\;\exists x{\in}A\;(P(x,y)).\]
\end{Satz}

\begin{Beweis} Nach Def. \ref{def:bounded} (bounded),
Satz \ref{general-dl} (general-dl)
und Satz \ref{exists-cl} (exists-cl) gilt:
\begin{gather*}
\exists x{\in}A\;\exists y{\in}B\;(P(x,y))
\iff \exists x(x\in A\land\exists y[y\in B\land P(x,y)])\\
\iff \exists x\exists y[x\in A\land y\in B\land P(x,y)]
\iff \exists y\exists x[y\in B\land x\in A\land P(x,y)]\\
\iff \exists y(y\in B\land \exists x[x\in A\land P(x,y)])
\iff \exists y{\in}B\;\exists x{\in}A\; (P(x,y)).\;\qedsymbol
\end{gather*}
\end{Beweis}

\begin{Satz}\label{bounded-all-cl}
Es gilt:
\[\forall x{\in}A\;\forall y{\in}B\;(P(x,y))
\iff \forall y{\in}B\;\forall x{\in}A\;(P(x,y)).\]
\end{Satz}

\begin{Beweis}
Nach Def. \ref{def:bounded} (bounded),
Satz \ref{general-dl} (general-dl)
und Satz \ref{all-cl} (all-cl) gilt:
\begin{gather*}
\forall x{\in}A\;\forall y{\in}B\;(P(x,y))
\iff \forall x(x\in A\Rightarrow \forall y[y\in B\Rightarrow P(x,y)])\\
\iff \forall x(x\notin A\lor \forall y[y\notin B\lor P(x,y)])
\iff \forall x\forall y[x\notin A\lor y\notin B\lor P(x,y)]\\
\iff \forall y\forall x[y\notin B\lor x\notin A\lor P(x,y)]
\iff \forall y(y\notin B\lor \forall x[x\notin A\lor P(x,y)])\\
\iff \forall y(y\in B\Rightarrow \forall x[x\in A\Rightarrow P(x,y)])
\iff \forall y{\in}B\;\forall x{\in}A\;(P(x,y).\;\qedsymbol
\end{gather*}
\end{Beweis}

\begin{Satz}
Für eine Aussage $P$, die nicht von $x$ abhängt, und ein nichtleeres
Diskursuniversum gilt:
\[\exists x(P) \iff P.\]
\end{Satz}
\begin{Beweis}[Beweis (informal)]
Nach \ref{general-dl} (general-dl) gilt:
\[\exists x(P) \iff \exists x(1\land P)
\iff \exists x(1)\land P\iff 1\land P\iff P.\]
Im vorletzten Schritt wurde dabei ausgenutzt, dass
für ein nichtleeres Diskursuniversum immer $\exists x(1)\iff 1$
gelten muss.\,\qedsymbol
\end{Beweis}

\begin{Satz}
Es gilt
\[\exists x{\in}M\;(P) \iff (M\ne\emptyset)\land P.\]
\end{Satz}

\begin{Beweis}
Nach Def. \ref{def:bounded} (bounded) und Satz
\ref{general-dl} (general-dl) gilt:
\begin{gather*}
\exists x{\in}M\;(P) \iff \exists x(x\in M\land P)
\iff \exists x(x\in M)\land P\iff (M\ne\emptyset)\land P.\;\qedsymbol
\end{gather*}
\end{Beweis}

\newpage
\section{Mengenlehre}\index{Mengenlehre}
\subsection{Definitionen}

\begin{Definition}[seteq: Gleichheit von Mengen]
\label{def:seteq}\index{Gleichheit!von Mengen}
\[A=B \defiff \forall x(x\in A\iff x\in B).\]
\end{Definition}

\begin{Definition}[subseteq: Teilmenge]%
\label{def:subseteq}\index{Teilmenge}
\[A\subseteq B \defiff \forall x(x\in A\implies x\in B).\]
\end{Definition}

\begin{Definition}[filter: beschreibende Angabe]\label{def:filter}
\[a\in\{x\mid P(x)\} \defiff P(a).\]
\end{Definition}

\begin{Definition}[cap: Schnitt]%
\label{def:cap}\index{Schnittmenge}
\[A\cap B := \{x\mid x\in A\land x\in B\}.\]
\end{Definition}

\begin{Definition}[cup: Vereinigung]%
\label{def:cup}\index{Vereinigungsmenge}
\[A\cup B := \{x\mid x\in A\lor x\in B\}.\]
\end{Definition}

\begin{Definition}[intersection: Schnitt]\label{def:intersection}
\[\bigcap_{i\in I} A_i := \{x\mid \forall i{\in}I\,(x\in A_i)\}
= \{x\mid \forall i\,(i\in I\implies x\in A_i)\}.\]
\end{Definition}

\begin{Definition}[union: Vereinigung]\label{def:union}
\[\bigcup_{i\in I} A_i := \{x\mid \exists i{\in}I\,(x\in A_i)\}
= \{x\mid \exists i\,(i\in I\land x\in A_i)\}.\]
\end{Definition}

\begin{Definition}[cart: kartesisches Produkt]%
\label{def:cart}\index{kartesisches Produkt}
\[A\times B := \{(a,b)\mid a\in A\land b\in B\}
= \{t\mid\exists a\exists b(t=(a,b)\land a\in A\land b\in B)\}.\]
\end{Definition}

\subsection{Rechenregeln}

\begin{Satz}[Kommutativgesetze]\index{Kommutativgesetz!Mengen, boolesche Algebra}
Es gilt $A\cap B = B\cap A$ und $A\cup B = B\cup A$.
\end{Satz}

\begin{Beweis}
Nach Def. \ref{def:seteq} (seteq) expandieren:
\[\forall x(x\in A\cap B \iff x\in B\cap A).\]
Nach Def. \ref{def:cap} (cap) und Def. \ref{def:filter} (filter) gilt:
\[x\in A\cap B \iff x\in A\land x\in B \iff x\in B\land x\in A
\iff x\in B\cap A.\]
Für die Vereinigung ist das analog.\,\qedsymbol
\end{Beweis}

\begin{Satz}[Assoziativgesetze]%
\index{Assoziativgesetz!Mengen, boolesche Algebra}
Es gilt $A\cap (B\cap C) = (A\cap B)\cap C$
und $A\cup (B\cup C) = (A\cup B)\cup C$.
\end{Satz}

\begin{Beweis}
Nach Def. \ref{def:seteq} (seteq) expandieren:
\[\forall x[x\in A\cap (B\cap C) \iff x\in (A\cap B)\cap C].\]
Nach Def. \ref{def:cap} (cap) und Def. \ref{def:filter} (filter) gilt:
\begin{align*}
&x\in A\cap (B\cap C) \iff x\in A\land x\in B\cap C
\iff x\in A\land (x\in B\land x\in C)\\
&\iff (x\in A\land x\in B)\land x\in C
\iff x\in A\cap B\land x\in C
\iff x\in (A\cap B)\cap C.
\end{align*}
Für die Vereinigung ist das analog.\,\qedsymbol
\end{Beweis}

\begin{Satz}\label{eq-iff-all-iff}
Es gilt $a=b\iff \forall x(x=a\iff x=b)$.
\end{Satz}

\begin{Beweis}
Die Implikation $a=b\implies\forall x(x=a\iff x=b)$.
Wenn wir $a=b$ voraussetzen, kann $b$ gegen $a$ ersetzt werden
und es ergibt sich
\[\forall x(x=a\iff x=a)\iff\forall x(1)\iff 1.\]
Die andere Implikation bringen wir zunächst in ihre Kontraposition:
\[a\ne b\implies \exists x((x=a)\oplus (x=b)).\]
Auf einer leeren Grundmenge wird der Allquantifizierung
über $a,b$ immer genügt. Besitzt die Grundmenge nur ein Element,
dann muss $a=b$ sein, womit $a\ne b$ falsch ist und die Implikation
somit erfüllt. Wir setzen nun $a\ne b$ voraus. Wählt man nun
$x=a$, dann ist $x\ne b$, womit die Kontravalenz erfüllt wird.\;\qedsymbol
\end{Beweis}

\begin{Satz}
Es gilt $a=b\iff\{a\}=\{b\}$.
\end{Satz}

\begin{Beweis}
Es gilt:
\[\{a\}=\{b\}\iff \{x\mid x=a\}=\{x\mid x=b\}\iff \forall x(x=a\iff x=b).\]
Nach Satz \ref{eq-iff-all-iff} ist das aber äquivalent zu $a=b$.\;\qedsymbol
\end{Beweis}

\begin{Satz}\label{eq-substitution}
Es gilt:
\[\forall x\forall y(x=y\land P(x)\iff P(y))\]
\end{Satz}

\begin{Satz}\label{all-cart}
Es gilt:
\[\forall t{\in}A{\times}B\;(P(t)) \iff \forall a{\in}A\;\forall b{\in}B\;(P(a,b)).\]
\end{Satz}

\begin{Beweis}
Nach Def. \ref{def:cart} (cart) gilt:
\begin{align*}
&\forall t{\in}A{\times}B\;(P(t))\iff \forall t(t\in A\times B\implies P(t))\\
&\iff \forall t(\exists a\exists b[t=(a,b)\land a\in A\land b\in B]\implies P(t))
\end{align*}
Unter doppelter Anwendung von Satz \ref{exists-implies-const} gilt weiter:
\[\iff \forall t\forall a\forall b[t=(a,b)\land a\in A\land b\in B\implies P(t)]\]
Substituiert man $t:=(a,b)$, dann ergibt sich:
\[\implies \forall a\forall b[a\in A\land B\in B\implies P(a,b)]
\iff \forall a{\in}a\;\forall b{\in}B\;(P(a,b)),\]
wobei $P(a,b)$ eine Kurzschreibweise für $P((a,b))$ ist.
Von der Gegenrichtung bilden wir die Kontraposition:
\[\exists t\exists a\exists b[t=(a,b)\land a\in A\land b\in B\land \overline{P(t)}]
\implies \exists a\exists b(a\in a\land b\in B\land \overline{P(a,b)}).\]
Dem $\exists t$ wird aber immer durch $t:=(a,b)$ genügt, so dass sich die
äquivalente Formel
\[\exists a\exists b[a\in A\land b\in B\land \overline{P(a,b)}]
\implies \exists a\exists b(a\in A\land b\in B\land \overline{P(a,b)}).\]
ergibt.\;\qedsymbol
\end{Beweis}

\begin{Satz}\label{exists-cart}
Es gilt:
\[\exists t{\in}A{\times}B\;(P(t))
\iff \exists a{\in}A\;\exists b{\in}B\;(P(a,b)).\]
\end{Satz}

\begin{Beweis}
Nach Def. \ref{def:cart} (cart) gilt:
\begin{gather*}
\exists t{\in}A{\times}B\;(P(t))
\iff \exists t(t\in A\times B\land P(t))\\
\iff \exists t(\exists a\exists b[t=(a,b)\land a\in A\land b\in B]\land P(t))\\
\iff \exists t\exists a\exists b[a\in A\land b\in B\land t=(a,b)\land P(t)]\\
\iff \exists a{\in}A\;\exists b{\in}B\;\exists t[t=(a,b)\land P(t)].
\end{gather*}
Nun gilt aber ganz offensichtlich
\[\exists t[t=(a,b)\land P(t)]\iff P(a,b).\]
Nimmt man $P(a,b)$ an, dann lässt sich $\exists t[t=(a,b)\land P(t)]$
durch Wahl von $t:=(a,b)$ bestätigen. Nimmt man umgekehrt
$\exists t[t=(a,b)\land P(t)]$ an, lässt sich $P(a,b)$ daraus
unter Anwendung von Satz \ref{eq-substitution} ableiten.
Da $\exists t[t=(a,b)\land P(t)]$ gegen $P(a,b)$ ersetzt werden
darf, folgt die Behauptung.\,\qedsymbol
\end{Beweis}

\begin{Satz}\label{cup-cart}
Es gilt:
\[\bigcup_{t\in I\times J} A_t
= \bigcup_{i\in I}\bigcup_{j\in J} A_{ij}.\quad (t=(i,j))\]
\end{Satz}

\begin{Beweis}
Nach Def. \ref{def:union} (union) und Satz \ref{exists-cart} gilt:
\begin{gather*}
x\in \bigcup_{t\in I\times J} A_t
\iff \exists t{\in}I{\times J}\;(x\in A_t)
\iff \exists i{\in}I\;\exists j{\in}J\;(x\in A_{ij})\\
\iff \exists i{\in}I\; (x\in \bigcup_{j\in J} A_{ij})
\iff x\in\bigcup_{i\in I}\bigcup_{j\in J} A_{ij}.
\end{gather*}
Nach Def. \ref{def:seteq} (seteq) folgt die Behauptung.\,\qedsymbol
\end{Beweis}

\begin{Satz}
Es gilt:
\[\bigcup_{i\in I}\bigcup_{j\in J} A_{ij}
= \bigcup_{j\in J}\bigcup_{i\in I} A_{ij}.\]
\end{Satz}

\begin{Beweis}
Nach Def. \ref{def:union} (union) und Satz \ref{bounded-exists-cl} gilt:
\begin{gather*}
x\in\bigcup_{i\in I}\bigcup_{j\in J} A_{ij}
\iff \exists i{\in}I\;(x\in\bigcup_{j\in J} A_{ij})
\iff \exists i{\in}I\;\exists j{\in}J\;(x\in A_{ij})\\
\iff \exists j{\in}J\;\exists i{\in}I\;(x\in A_{ij})
\iff \exists j{\in}J\;(x\in \bigcup_{i\in I}A_{ij})
\iff x\in\bigcup_{j\in J}\bigcup_{i\in I} A_{ij}.
\end{gather*}
Nach Def. \ref{def:seteq} (seteq) folgt die Behauptung.\,\qedsymbol
\end{Beweis}

\newpage
\section{Abbildungen}\index{Abbildungen}
\subsection{Definitionen}

\begin{Definition}[app: Applikation]\label{def:app}
Für eine Abbildung $f$ ist
\[y=f(x)\defiff (x,y)\in G_f.\]
\end{Definition}

\begin{Definition}[img: Bildmenge]\label{def:img}\index{Bildmenge}
Für eine Abbildung $f\colon A\to B$ und $M\subseteq A$
wird die Menge
\[f(M) := \{y\mid \exists x{\in}M\;(y=f(x))\}
= \{y\mid \exists x(x\in M\land y=f(x))\}\]
als Bildmenge von $M$ unter $f$ bezeichnet.
\end{Definition}

\begin{Definition}[preimg: Urbildmenge]\label{def:preimg}\index{Urbildmenge}
Für eine Abbildung $f\colon A\to B$ wird
\[f^{-1}(M) := \{x\mid f(x)\in M\}\]
als Urbildmenge von $M$ unter $f$ bezeichnet.
\end{Definition}

\begin{Definition}[inj: Injektion]\label{def:inj}\index{Injektion}
Eine Abbildung $f\colon A\to B$ heißt genau dann injektiv, wenn gilt:
\[\forall x_1\forall x_2(f(x_1)=f(x_2)\implies x_1=x_2)\]
bzw. äquivalent
\[\forall x_1\forall x_2(x_1\ne x_2\implies f(x_1)\ne f(x_2)).\]
\end{Definition}

\begin{Definition}[sur: Surjektion]\label{def:sur}\index{Surjektion}
Eine Abbildung $f\colon A\to B$ heißt genau dann surjektiv, wenn gilt:
\[B\subseteq f(A).\]
\end{Definition}


\subsection{Grundlagen}
\begin{Satz}[feq: Gleichheit von Abbildungen]%
\label{feq}\index{Gleichheit!von Abbildungen}
Zwei Abbildungen $f\colon A\to B$ und $g\colon C\to D$ sind genau
dann gleich, kurz $f=g$, wenn $A=C$ und $B=D$ und
\[\forall x(f(x)=g(x)).\]
\end{Satz}

\begin{Beweis}
Nach Definition gilt $f=g$ genau dann, wenn $(G_f,A,B)=(G_g,C,D)$,
was äquivalent zu $G_f=G_g\land A=C\land B=D$ ist. Nach Def.
\ref{def:seteq} (seteq) gilt
\[G_f=G_g\iff \forall t(t\in G_f\iff t\in G_g).\]
Nach Satz \ref{eq-iff-all-iff} und Def. \ref{def:app} (app) gilt
\begin{align*}
&\forall x[f(x)=g(x)] \iff \forall x\forall y[y=f(x)\iff y=g(x)]\\
&\iff \forall x\forall y[(x,y)\in G_f\iff (x,y)\in G_g]
\iff \forall t(t\in G_f\iff t\in G_g).
\end{align*}
Da die Quantifizerung auf $x\in A$, $y\in B$ und $t\in A\times B$
beschränkt ist, konnte im letzten Schritt Satz \ref{all-cart}
angewendet werden.\;\qedsymbol
\end{Beweis}

\begin{Satz}[preimg-dl: Distributivität der Urbildoperation]%
\index{Distributivgesetz!Urbildoperation}\mbox{}\\
Für $f\colon A\to B$ und beliebige Mengen $M_i$ gilt:
\begin{align}
f^{-1}(M_1\cap M_2) &= f^{-1}(M_1)\cap f^{-1}(M_2),\\
f^{-1}(M_1\cup M_2) &= f^{-1}(M_1)\cup f^{-1}(M_2),\\
f^{-1}(\bigcap_{i\in I} M_i) &= \bigcap_{i\in I} f^{-1}(M_i),\\
f^{-1}(\bigcup_{i\in I} M_i) &= \bigcup_{i\in I} f^{-1}(M_i).
\end{align}
\end{Satz}

\begin{Beweis}
Nach Def. \ref{def:seteq} (seteq) expandieren:
\[\forall x[x\in f^{-1}(M_1\cap M_2)\iff x\in f^{-1}(M_1)\cap f^{-1}(M_2)].\]
Nach Def. \ref{def:preimg} (preimg) und Def. \ref{def:cap} (cap)
zusammen mit Def. \ref{def:filter} (filter) gilt:
\begin{align*}
& x\in f^{-1}(M_1\cap M_2) \iff f(x)\in M_1\cap M_2
\iff f(x)\in M_1\land f(x)\in M_2\\
&\iff x\in f^{-1}(M_1)\land x\in f^{-1}(M_2)
\iff x\in f^{-1}(M_1)\cap f^{-1}(M_2).
\end{align*}
Für die Vereinigung ist das analog.

Schnitt von beliebig vielen Mengen.
Nach Def. \ref{def:seteq} (seteq) expandieren:
\[\forall x[x\in f^{-1}(\bigcap_{i\in I}M_i)
\iff x\in \bigcap_{i\in I} f^{-1}(M_i)].\]
Nach Def. \ref{def:preimg} (preimg) und Def. \ref{def:intersection}
(intersection) zusammen mit Def. \ref{def:filter} (filter) gilt:
\begin{align*}
& x\in f^{-1}(\bigcap_{i\in I} M_i)\iff f(x)\in\bigcap_{i\in I} M_i
\iff \forall i(i\in I\implies f(x)\in M_i)\\
&\iff \forall i(i\in I\implies x\in f^{-1}(M_i))
\iff x\in \bigcap_{i\in I} f^{-1}(M_i).
\end{align*}
Für die Vereinigung ist das analog.\;\qedsymbol
\end{Beweis}

\begin{Satz}[img-cup-dl: Distributivität der Bildoperation über die Vereinigung]
Für $f\colon A\to B$ und Mengen $M_i\subseteq A$ gilt:
\begin{align}
f(M_1\cup M_2) &= f(M_1)\cup f(M_2),\\
f(\bigcup_{i\in I} M_i) &= \bigcup_{i\in I} f(M_i).
\end{align}
\end{Satz}
\begin{Beweis}
Nach Def. \ref{def:seteq} (seteq) expandieren:
\[\forall y(y\in f(M_1\cup M_2)\iff y\in f(M_1)\cup f(M_2)).\]
Nach Def. \ref{def:img} (img), Def. \ref{def:cup} (cup),
Satz \ref{bool-dl} (bool-dl) und Satz \ref{exists-dl} (exists-dl) gilt:
\begin{align*}
&y\in f(M_1\cup M_2) \iff \exists x[x\in M_1\cup M_2\land y=f(x)]\\
&\iff \exists x[(x\in M_1\lor x\in M_2)\land y=f(x)]\\
&\iff \exists x[x\in M_1\land y=f(x)\lor x\in M_2\land y=f(x)]\\
&\iff \exists x[x\in M_1\land y=f(x)]\lor\exists x[x\in M_2\land y=f(x)]\\
&\iff y\in f(M_1)\lor y\in f(M_2) \iff y\in f(M_1)\cup f(M_2).
\end{align*}
Nach Def. \ref{def:seteq} (seteq) expandieren:
\[\forall y[y\in f(\bigcup_{i\in I} M_i)\iff y\in \bigcup_{i\in I} f(M_i)].\]
Nach Def. \ref{def:img} (img), Def. \ref{def:union} (union),
Satz \ref{general-dl} (general-dl)\\
und Satz \ref{exists-cl} (exists-cl) gilt:
\begin{align*}
& y\in f(\bigcup_{i\in I} M_i)
\iff \exists x(x\in\bigcup_{i\in I} M_i\land y=f(x))\\
&\iff \exists x(\exists i(i\in I\land x\in M_i)\land y=f(x))
\iff \exists x\exists i(i\in I\land x\in M_i\land y=f(x))\\
&\iff \exists i\exists x(i\in I\land x\in M_i\land y=f(x))
\iff \exists i(i\in I\land\exists x(x\in M_i\land y=f(x))\\
&\iff \exists i(i\in I\land y\in f(M_i))
\iff y\in\bigcup_{i\in I} f(M_i).\;\qedsymbol
\end{align*}
\end{Beweis}

\begin{Satz}
Es gilt:
\begin{align}
f(M_1\cap M_2) &\subseteq f(M_1)\cap f(M_2),\\
f(\bigcap_{i\in I} M_i) &\subseteq \bigcap_{i\in I} f(M_i).
\end{align}
\end{Satz}

\begin{Beweis}
Nach Def. \ref{def:subseteq} (subseteq) expandieren:
\[\forall y(y\in f(M_1\cap M_2)\implies y\in f(M_1)\cap f(M_2)).\]
Nach Def. \ref{def:img} (img), Def. \ref{def:cap} (cap)
und Satz. \ref{exists-asym-dl} (exists-asym-dl) gilt:
\begin{align*}
& y\in f(M_1\cap M_2) \iff \exists x(x\in M_1\cap x\in M_2\land y=f(x))\\
&\iff \exists x(x\in M_1\land x\in M_2\land y=f(x))\\
&\iff \exists x(x\in M_1\land y=f(x)\land x\in M_2\land y=f(x))\\
&\implies \exists x(x\in M_1\land y=f(x))\land\exists x(x\in M_2\land y=f(x))\\
&\iff y\in f(M_1)\land y\in f(M_2)\iff y\in f(M_1)\cap f(M_2).
\end{align*}
Nach Def. \ref{def:subseteq} (subseteq) expandieren:
\[\forall y(y\in f(\bigcap_{i\in I} M_i)\implies y\in \bigcap_{i\in I} f(M_i))\]
Nach Def. \ref{def:img} (img) und Def. \ref{def:intersection} (intersection)
gilt:
\begin{align*}
& y\in f(\bigcap_{i\in I} M_i)\iff \exists x[x\in\bigcap_{i\in I} M_i\land y=f(x)]\\
& \iff \exists x[\forall i(i\in I\implies x\in M_i)\land y=f(x)]\\
& \iff \exists x\forall i(i\in I\implies x\in M_i\land y=f(x))\\
& \implies \forall i\exists x[i\in I\implies x\in M_i\land y=f(x)]\\
& \iff \forall i(i\in I\implies \exists x[x\in M_i\land y=f(x)])\\
& \iff \forall i(i\in I\implies y\in f(M_i))
\iff y\in\bigcap_{i\in I} f(M_i).\;\qedsymbol
\end{align*}
\end{Beweis}

\begin{Satz}\label{img-as-cup}
Es gilt:
\[f(M) = \bigcup_{x\in M} \{f(x)\}.\]
\end{Satz}

\begin{Beweis}
Nach Def. \ref{def:img} (img) und Def. \ref{def:union} (union) gilt:
\[y\in f(M) \iff \exists x{\in}M\;(y=f(x))
\iff \exists x{\in}M\;(y\in \{f(x)\})
\iff y\in\bigcup_{x\in M}\{f(x)\}.\]
Nach Def. \ref{def:seteq} (seteq) folgt dann die Behauptung.\,\qedsymbol
\end{Beweis}

\newpage
\subsection{Kardinalzahlen}
\begin{Satz}[acc: abzählbares Auswahlaxiom]\label{acc}%
\index{abzählbares Auswahlaxiom}\index{Auswahlaxiom!abzählbares}
Sei $(A_n)_{n\in\N}$ eine Folge nichtleerer Mengen.
Dann existiert eine Funktion $f\colon\N\to\bigcup_{n\in\N} A_n$
mit $f(n)\in A_n$.
\end{Satz}

\begin{Definition}[equipotent: Gleichmächtigkeit]%
\label{def:equipotent}\index{gleichmaechtig@gleichmächtig}
Zwei Mengen $A$, $B$ heißen genau dann gleichmächtig, wenn
eine Bijektion $f\colon A\to B$ existiert.
\end{Definition}

\begin{Satz}
Sei $M$ eine beliebige Menge. Die Potenzmenge $2^M$ ist zur
Menge $\{0,1\}^M$ gleichmächtig.
\end{Satz}

\begin{Beweis}
Für eine Aussage $A$ sei
\[[A] := \begin{cases}
1&\text{wenn }$A$\text{ gilt},\\
0&\text{sonst}.
\end{cases}\]
Für $A\subseteq M$ betrachte man nun die
Indikatorfunktion\index{Indikatorfunktion}
\[\chi_A\colon M\to\{0,1\},\quad \chi_A(x):=[x\in A].\]
Die Abbildung
\[\varphi\colon 2^M\to \{0,1\}^M,\quad \varphi(A):=\chi_A\]
ist eine kanonische Bijektion.

\strong{Zur Injektivität.}
Nach Def. \ref{def:inj} (inj) muss gelten:
\[\varphi(A)=\varphi(B)\implies A=B,\quad\text{d.\,h.}\quad
\chi_A=\chi_B \implies A=B.\]
Nach Satz \ref{feq} (feq) und Def. \ref{def:seteq} (seteq)
wird die Aussage expandiert zu:
\[\forall x(\chi_A(x)=\chi_B(x))\implies \forall x(x\in A\iff x\in B).\]
Es gilt aber nun:
\[\chi_A(x)=\chi_B(x)\iff [x\in A]=[x\in B] \iff (x\in A\iff x\in B).\]
\end{Beweis}
\strong{Zur Surjektivität.} Wir müssen nach Def. \ref{def:sur} (sur)
prüfen, dass $\{0,1\}^M\subseteq \varphi(2^M)$ gilt.
Expansion nach Def. \ref{def:subseteq} (subseteq) und
Def. \ref{def:img} (img)
ergibt:
\[\forall f(f\in \{0,1\}^M\implies\exists A{\in}2^M[f=\varphi(A)]).\]
Um dem Existenzquantor zu genügen, wähle
\[A := f^{-1}(\{1\}) = \{x\in M\mid f(x)\in \{1\}\} = \{x\in M\mid f(x)=1\}.\]
Es gilt $f=\chi_A$, denn
\[\chi_A(x) = [x\in A] = [x\in\{x\mid f(x)=1\}] = [f(x)=1] = f(x).\]
Da $\varphi$ eine Bijektion ist, müssen $2^M$ und $\{0,1\}^M$
nach Def. \ref{def:equipotent} (equipotent) gleichmächtig
sein.\,\qedsymbol

\newpage
\begin{Satz}\label{countable-union-countable}
Man setze Axiom \ref{acc} (acc) voraus.
Die Vereinigung von abzählbar vielen abzählbar unendlichen Mengen
ist abzählbar unendlich. Kurz $|\bigcup_{n\in\N} A_n| = |\N|$, wenn
$|A_n|=|\N|$ für jedes $n$.
\end{Satz}

\begin{Beweis}
Sei $B_n$ die Menge der Bijektionen aus $\Abb(\N,A_n)$. 
Nach Axiom \ref{acc} (acc) kann aus jeder Menge $B_n$
eine Bijektion $f_n\colon\N\to A_n$ ausgewählt werden.
Man betrachte nun
\[\varphi\colon\N\times\N\to\bigcup_{n\in\N} A_n,\quad
\varphi(n,m):=f_n(m).\]
Die Abbildung $\varphi$ ist surjektiv, denn nach
Satz \ref{img-as-cup} und Satz \ref{cup-cart} gilt
\begin{gather*}
\varphi(\N\times\N) = \bigcup_{(n,m)\in\N{\times}\N} \{f_n(m)\}
= \bigcup_{n\in\N}\bigcup_{m\in\N} \{f_n(m)\}\\
= \bigcup_{n\in\N} f_n(\bigcup_{m\in\N} \{m\})
= \bigcup_{n\in\N} f_n(\N) = \bigcup_{n\in\N} A_n.
\end{gather*}
Daher gilt $|\bigcup_{n\in\N} A_n|\le |\N\times \N| = |\N|$.
Für eine beliebige der Bijektionen $f_n\in B_n$ lässt sich die Zielmenge
erweitern, so dass man eine Injektion $f\colon\N\to\bigcup_{n\in\N} A_n$
erhält. Daher ist auch $|\N|\le |\bigcup_{n\in\N} A_n|$. Nach dem
Satz von Cantor-Bernstein gilt also
$|\bigcup_{n\in\N} A_n|=|\N|$.\,\qedsymbol
\end{Beweis}

\begin{Satz}\label{countable-polynomial-ring}
Wenn $R$ abzählbar ist, dann ist auch der Polynomring $R[X]$ abzählbar.
\end{Satz}

\begin{Beweis}
Zu jedem Polynom vom Grad $n\ge 1$ gehört auf kanonische Weise
genau ein Tupel aus $M_n:=R^{n-1}\times R\setminus\{0\}$. Da $R$
abzählbar ist, sind auch $R^{n-1}$ und $R\setminus\{0\}$ abzählbar.
Dann ist auch $M_n$ abzählbar. Nach Satz \ref{countable-union-countable}
gilt
\[|R[X]| = 1+|\bigcup_{n\in\N} M_n| = 1+|\N| = |\N|.\,\qedsymbol\]
\end{Beweis}

\begin{Satz}\index{algebraische Zahlen!Kardinalität}
Es gibt nur abzählbar unendlich viele algebraische Zahlen.
\end{Satz}

\begin{Beweis}[Beweis 1]
Zu zeigen ist $|\A|=|\N|$ mit
\[\A := \{a\in\C\mid \exists p(p\in \Q[X]\setminus\{0\}\land p(a)=0)\}.\]
Dass $\A$ unendlich ist, ist leicht ersichtlich, denn schon jede
rationale Zahl $q$, von denen es unendlich viele gibt, ist Nullstelle
von $p(X):=X-q$ und daher algebraisch.

Ein Polynom vom Grad $n$ kann höchstens $n$ Nullstellen besitzen.
Nach Satz \ref{countable-polynomial-ring} gilt $\Q[X]=|\N|$.
Für $\Q[X]$ lässt sich also eine Abzählung angeben.
Bei dieser Abzählung lässt sich für jedes Polynom $p$ die Liste der
Nullstellen von $p$ einfügen. Streicht man alle Nullstellen, die schon
einmal vorkamen, dann erhält man eine Abzählung der algebraischen
Zahlen. Demnach gilt $|\A|=|\N|$.\,\qedsymbol
\end{Beweis}

\begin{Beweis}[Beweis 2]
Jedem $p=\sum_{k=0}^n a_kX^k$ lässt sich eine Höhe
$h:=n+\sum_{k=0}^n |a_k|$ zuordnen. Zu einer festen Höhe kann es nur
endlich viele Polynome $p\in\Z[X]$ geben, wodurch man eine Abzählung der
Polynome erhält, wenn für $h=1$, $h=2$, $h=3$ usw. jeweils die Liste
der Polynome eingefügt wird. Für jedes Polynom $p$ lässt sich die
Liste der Nullstellen von $p$ einfügen. Streicht man alle Nullstellen,
die schon einmal vorkamen, dann erhält man eine Abzählung der
algebraischen Zahlen.\,\qedsymbol
\end{Beweis}

\begin{Beweis}[Beweis 3]
Für $n\in\N$ sei
\[\begin{split}
A_n := \{x\in\A\mid \;&\text{$x$ ist Nullstelle eines
$p\in\Z[X]\setminus\{0\}$ mit $\deg(p)=n$,}\\
& \text{dessen Koeffizienten $a_k$ alle $|a_k|\le n$
erfüllen}\}.
\end{split}\]
Alle $A_n$ sind endlich und es gilt $\A=\bigcup_{n\in\N} A_n$.
Daher muss $|\A|\le |\N|$ sein.\,\qedsymbol
\end{Beweis}
