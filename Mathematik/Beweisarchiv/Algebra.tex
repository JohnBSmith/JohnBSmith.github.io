
\chapter{Algebra}

\section{Gruppentheorie}

\subsection{Grundlagen}

\begin{Definition}[Gruppe]
Das Tupel $(G,*)$ bestehend aus einer Menge $G$ und
Abbildung $*\colon G\times G\to\Omega$ heißt Gruppe, wenn die folgenden
Axiome erfüllt sind:
\begin{enumerate}
\item[(G1)] Für alle $a,b\in G$ gilt $a*b\in G$. D.\,h. man darf $G=\Omega$ setzen.
\item[(G2)] Es gilt das Assoziativgesetz: für alle $a,b,c\in G$ gilt $(a*b)*c=a*(b*c)$.
\item[(G3)] Es gibt ein Element $e\in G$, so dass $e*g=g=g*e$ für jedes $g\in G$ gilt.
\item[(G4)] Zu jedem $g\in G$ gibt es ein $g^{-1}\in G$ so dass $g*g^{-1}=e=g^{-1}*g$ gilt.
\end{enumerate}
Das Element $e$ wird neutrales Element der Gruppe genannt.
Das Element $g^{-1}$ wird inverses Element zu $g$ genannt.
Anstelle von $a*b$ schreibt man auch kurz $ab$. Ist $(G,+)$ eine
Gruppe, dann schreibt man immer $a+b$, und $-g$ anstelle von $g^{-1}$.
\end{Definition}

\begin{Korollar}
Das neutrale Element einer Gruppe $G$ ist eindeutig bestimmt.
D.\,h. es gibt keine zwei unterschiedlichen neutralen Elemente. 
\end{Korollar}
\begin{Beweis}
Seien $e,e'$ zwei neutrale Elemente von $G$. Nach Axiom (G3)
gilt dann $e=e'e$, und weiter $e'e=e'$ bei nochmaliger Anwendung
von (G3). Daher ist $e=e'$.\;\qedsymbol
\end{Beweis}

\begin{Korollar}
Sei $G$ eine Gruppe. Zu jedem Element $g\in G$ ist das inverse
Element $g^{-1}$ eindeutig bestimmt. D.\,h. es kann keine zwei
unterschiedlichen inversen Elemente zu $g$ geben.
\end{Korollar}
\begin{Beweis}
Seien $a,b$ zwei inverse Elemente zu $g$. Nach Axiom (G3), Axiom (G2)
und Axiom (G4) gilt
\[a \stackrel{(G3)}= ae \stackrel{(G4)} = a(gb) \stackrel{(G2)}
= (ag)b \stackrel{(G4)}= eb \stackrel{(G3)}= b.\]
Daher ist $a=b$.\;\qedsymbol
\end{Beweis}

\begin{Definition}[Untergruppe]
Sei $(G,*)$ eine Gruppe. Eine Teilmenge $U\subseteq G$ heißt
Untergruppe von $G$, kurz $U\le G$, wenn $U$ bezüglich der selben
Verknüpfung $*$ selbst eine Gruppe $(U,*)$ bildet.
\end{Definition}

\begin{Korollar}
Jede Gruppe $G$ besitzt die Untergruppen $\{e\}\le G$ und $G\le G$,
wobei $e\in G$ das neutrale Element ist. Man spricht von den
trivialen Untergruppen.
\end{Korollar}
\begin{Beweis}
Die Aussage $G\le G$ ist trivial, denn $G\subseteq G$ ist allgemeingültig
und $(G,*)$ bildet nach Voraussetzung eine Gruppe. Zu (G1):
Es gilt $ee=e$. Da es nur diese eine Möglichkeit gibt, sind damit alle
überprüft.
Zu (G2): Das Assoziativgesetz wird auf Elemente der Teilmenge vererbt.
Zu (G3): Das neutrale Element ist in $\{e\}$ enthalten.
Zu (G4): Das neutrale Element ist gemäß $ee=e$ zu sich selbst invers.
Da $e$ das einzige Element von $\{e\}$ ist, sind damit alle
überprüft.\;\qedsymbol
\end{Beweis}
