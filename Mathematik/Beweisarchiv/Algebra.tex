
\chapter{Algebra}

\section{Gruppentheorie}

\subsection{Grundlagen}

\begin{Definition}[Gruppe]
Das Tupel $(G,*)$ bestehend aus einer Menge $G$ und
Abbildung $*\colon G\times G\to\Omega$ heißt Gruppe, wenn die folgenden
Axiome erfüllt sind:
\begin{enumerate}
\item[(G1)] Für alle $a,b\in G$ gilt $a*b\in G$. D.\,h., man darf $G=\Omega$ setzen.
\item[(G2)] Es gilt das Assoziativgesetz: für alle $a,b,c\in G$ gilt $(a*b)*c=a*(b*c)$.
\item[(G3)] Es gibt ein Element $e\in G$, so dass $e*g=g=g*e$ für jedes $g\in G$ gilt.
\item[(G4)] Zu jedem $g\in G$ gibt es ein $g^{-1}\in G$ so dass $g*g^{-1}=e=g^{-1}*g$ gilt.
\end{enumerate}
Das Element $e$ wird neutrales Element der Gruppe genannt.
Das Element $g^{-1}$ wird inverses Element zu $g$ genannt.
Anstelle von $a*b$ schreibt man auch kurz $ab$. Ist $(G,+)$ eine
Gruppe, dann schreibt man immer $a+b$, und $-g$ anstelle von $g^{-1}$.
\end{Definition}

\begin{Korollar}
Das neutrale Element einer Gruppe $G$ ist eindeutig bestimmt.
D.\,h., es gibt keine zwei unterschiedlichen neutralen Elemente. 
\end{Korollar}
\begin{Beweis}
Seien $e,e'$ zwei neutrale Elemente von $G$. Nach Axiom (G3)
gilt dann $e=e'e$, und weiter $e'e=e'$ bei nochmaliger Anwendung
von (G3). Daher ist $e=e'$.\;\qedsymbol
\end{Beweis}

\begin{Korollar}
Sei $G$ eine Gruppe. Zu jedem Element $g\in G$ ist das inverse
Element $g^{-1}$ eindeutig bestimmt. D.\,h., es kann keine zwei
unterschiedlichen inversen Elemente zu $g$ geben.
\end{Korollar}
\begin{Beweis}
Seien $a,b$ zwei inverse Elemente zu $g$. Nach Axiom (G3), Axiom (G2)
und Axiom (G4) gilt
\[a \stackrel{(G3)}= ae \stackrel{(G4)} = a(gb) \stackrel{(G2)}
= (ag)b \stackrel{(G4)}= eb \stackrel{(G3)}= b.\]
Daher ist $a=b$.\;\qedsymbol
\end{Beweis}

\begin{Definition}[Untergruppe]
Sei $(G,*)$ eine Gruppe. Eine Teilmenge $U\subseteq G$ heißt
Untergruppe von $G$, kurz $U\le G$, wenn $U$ bezüglich derselben
Verknüpfung $*$ selbst eine Gruppe $(U,*)$ bildet.
\end{Definition}

\begin{Korollar}
Jede Gruppe $G$ besitzt die Untergruppen $\{e\}\le G$ und $G\le G$,
wobei $e\in G$ das neutrale Element ist. Man spricht von den
trivialen Untergruppen.
\end{Korollar}
\begin{Beweis}
Die Aussage $G\le G$ ist trivial, denn $G\subseteq G$ ist allgemeingültig
und $(G,*)$ bildet nach Voraussetzung eine Gruppe. Zu (G1):
Es gilt $ee=e$. Da es nur diese eine Möglichkeit gibt, sind damit alle
überprüft.
Zu (G2): Das Assoziativgesetz wird auf Elemente der Teilmenge vererbt.
Zu (G3): Das neutrale Element ist in $\{e\}$ enthalten.
Zu (G4): Das neutrale Element ist gemäß $ee=e$ zu sich selbst invers.
Da $e$ das einzige Element von $\{e\}$ ist, sind damit alle
überprüft.\;\qedsymbol
\end{Beweis}

\section{Ringtheorie}

\subsection{Grundlagen}

\begin{Definition}[Ring]
Eine Struktur $(R,+,\cdot)$ heißt genau dann Ring, wenn die folgenden
Axiome erfüllt sind
\begin{enumerate}
\item[1.] $(R,+)$ ist eine kommutative Gruppe.
\item[2.] $(R,\cdot)$ ist eine Halbgruppe.
\item[3.] Für alle $a,b,c\in R$ gilt $a(b+c) = ab+ac$. (Linksdistributivgesetz)
\item[4.] Für alle $a,b,c\in R$ gilt $(a+b)c = ac+bc$. (Rechtsdistributivgesetz)
\end{enumerate}
\end{Definition}
Bemerkung: Das neutrale Element von $(R,+)$ wird als Nullelement
bezeichnet und meist $0$ geschrieben.

\begin{Definition}[Ring mit Eins]
Ein Ring $R$ heißt genau dann Ring mit Eins, wenn $(R,\cdot)$ ein
Monoid ist. Monoid heißt, es gibt ein Element $e\in R$, so dass
$e\cdot a = a$ und $a\cdot e = a$ für alle $a\in R$.
\end{Definition}
Bemerkung: Man bezeichnet $e$ als Einselement des Rings.

\begin{Korollar}
Sei $R$ ein Ring und $0\in R$ das Nullelement.\\
Für jedes $a\in R$ gilt $0\cdot a = 0$ und $a\cdot 0 = 0$.
\end{Korollar}
\begin{Beweis} Man rechnet
\[0a = 0a+0 = 0a+0a-0a = (0+0)a-0a = 0a-0a = 0.\]
\end{Beweis}
Die Rechnung für $a\cdot 0$ ist analog.\;\qedsymbol

\begin{Korollar}\label{Minus-vorziehen}
Sei $R$ ein Ring und $a,b\in R$, dann gilt $(-a)b = -(ab) = a(-b)$.
\end{Korollar}
\begin{Beweis}
Man rechnet
\begin{align*}
(-a)b &= (-a)b+0 = (-a)b+ab-(ab) = ((-a)+a)b-(ab)\\
&= 0b-(ab) = 0-(ab) = -(ab).\;\qedsymbol
\end{align*}
\end{Beweis}

\begin{Korollar}[»Minus mal minus macht plus«]\mbox{}\\*
Sei $R$ ein Ring und $a,b\in R$, dann gilt
$(-a)(-b) = ab$.
\end{Korollar}
Beachtung von $-(-x)=x$ nach zweifacher Anwendung von
Korollar \ref{Minus-vorziehen} bringt
\[(-a)(-b) = -((-a)b) = -(-(ab)) = ab.\;\qedsymbol\]

\subsection{Ringhomomorphismen}

\begin{Definition}[Ringhomomorphismus]
Seien $R,R'$ Ringe. Eine Abbildung $\varphi\colon R\to R'$ heißt
Ringhomomorphismus, falls
\begin{gather*}
\varphi(x+y) = \varphi(x)+\varphi(y),\\
\varphi(xy) = \varphi(x)\varphi(y)
\end{gather*}
für alle $x,y\in R$ gilt. Liegen Ringe mit Eins vor, und gilt
zusätzlich $\varphi(1)=1$, dann spricht man von einem unitären
Ringhomomorphismus.
\end{Definition}

\begin{Korollar}
Bei jedem Ringhomomorphismus $\varphi$ gilt
$\varphi(kx) = k\varphi(x)$ für $k\in\Z$.
\end{Korollar}
\strong{Beweis.} Für $k>0$ ist
\[\varphi(kx) = \varphi(\sum_{i=1}^k x)
= \sum_{i=1}^k \varphi(x) = k\varphi(x).\]
Nun der Fall $k=0$. Man rechnet $f(0) = f(0+0) = f(0)+f(0)$.
Subtraktion von $f(0)$ auf beiden Seiten ergibt $f(0)=0$.
Schließlich bleibt noch $f(-kx)=-kf(x)$ für $k>0$ zu zeigen. Hier
rechnet man zunächst
\[0 = f(0) = f(-x+x) = f(-x)+f(x).\]
Subtraktion von $f(x)$ auf beiden Seiten ergibt $f(-x) = -f(x)$.
Somit gilt
\[f(-kx) = -f(kx) = -kf(x).\;\qedsymbol\]

\section{Polynomringe}

\subsection{Einsetzungshomomorphismus}

\begin{Satz}
Die Abbildung $\Phi\colon\R[X]\to\Abb(\R,\R)$ mit $\Phi(f)(x):=f(x)$
ist injektiv.
\end{Satz}
\begin{Beweis} Sei $f=\sum_{k=0}^n a_k X^k$
und $g=\sum_{k=0}^n b_k X^k$, wobei $n=\max(\deg f,\deg g)$.
Zu zeigen ist
\[(\forall x\in\R\colon \Phi(f)(x)=\Phi(g)(x))\implies f=g,\]
das heißt
\[(\forall x\in\R\colon \sum_k a_k x^k = \sum_k b_k x^k)\implies (\forall k\colon a_k=b_k).\]
Die Umformung der Voraussetzung ergibt $\sum_k (b_k-a_k)x^k = 0$.
D.\,h., jedes der $(b_k-a_k)$ muss verschwinden. Zu zeigen ist also
lediglich
\[(\forall x\in\R\colon \sum_{k=0}^n c_k x^k = 0)\implies (\forall k\colon c_k=0).\]
Wenn $f(x)=0$ für alle $x$ ist, muss auch die Ableitung $D^m f(x)=0$
sein. Es gilt $D^k x^k = k!$, und daher
\[D^n\sum_{k=0}^n c_k x^k = n!\cdot c_n = 0 \implies c_n=0.\]
Demnach ergibt sich dann aber auch
\[D^{n-1}\sum_{k=0}^n c_k x^k = (n-1)!\cdot c_{n-1} = 0\implies c_{n-1}=0\]
usw. Man erhält $c_k=0$ für alle $k$.\;\qedsymbol
\end{Beweis}
