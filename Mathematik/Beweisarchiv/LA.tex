
\chapter{Lineare Algebra}
\section{Matrizen}
\subsection{Definitionen}
\begin{Definition}[Transponierte Matrix]%
\index{transponierte Matrix}
Sei $R$ ein Ring und $A\in R^{m\times n}$ eine Matrix.
Die Matrix $A^T\in R^{n\times m}$ mit $(A^T)_{ij} := A_{ji}$
heißt Transponierte von $A$.
\end{Definition}

\begin{Definition}[Inverse Matrix]%
\label{matrix-inv}\index{inverse Matrix}
Sei $K$ ein Körper und $A\in K^{n\times n}$ eine quadratische
Matrix. Man nennt $A$ invertierbar, wenn es eine Matrix $B$ gibt,
mit $AB=BA=E_n$, wobei $E_n$ die Einheitsmatrix ist. Die Matrix
$A^{-1}:=B$ heißt dann inverse Matrix zu $A$.
\end{Definition}

\subsection{Rechenregeln}

\begin{Korollar}\label{transpose-product}
Sei $R$ ein kommutativer Ring. Für Matrizen $A\in R^{m\times n}$ und
$B\in R^{n\times p}$ gilt
\[(AB)^T = B^T A^T.\]
\end{Korollar}
\begin{Beweis} Es gilt:
\begin{align}
(AB)^T &= \bigg(\sum_{k=1}^n A_{ik}B_{kj}\bigg)^T
= \bigg(\sum_{k=1}^n A_{jk}B_{ki}\bigg)
= \bigg(\sum_{k=1}^n B_{ki} A_{jk}\bigg)\\
&= \bigg(\sum_{k=1}^n (B^T)_{ik} (A^T)_{kj}\bigg)
= B^T A^T.\;\qedsymbol
\end{align}
\end{Beweis}

\begin{Korollar}
Sei $A\in K^{n\times n}$ eine invertierbare Matrix. Dann ist auch
$A^T$ invertierbar und es gilt $(A^{-1})^T = (A^T)^{-1}$.
\end{Korollar}
\begin{Beweis}
Aus $E=A^{-1}A=AA^{-1}$ und Korollar \ref{transpose-product} folgt
\begin{equation}
E = E^T = (A^{-1}A)^T = A^T (A^{-1})^T = (AA^{-1})^T = (A^{-1})^T A^T.
\end{equation}
Dann muss $A^T$ nach Def. \ref{matrix-inv} die inverse Matrix
zu $(A^{-1})^T$ sein.\;\qedsymbol
\end{Beweis}

\section{Eigenwerte}

\begin{Satz}
Gegeben sei eine quadratische Matrix $A\in\R^{n\times n}$.
Dann ist die Matrix $M=A^T A$ symmetrisch und besitzt nur
nichtnegative Eigenwerte, speziell bei $\det(A)\ne 0$ nur positive.
\end{Satz}

\begin{Beweis}
Gemäß Satz \ref{transpose-product} gilt
\begin{equation}
M^T = (A^T A) = A^T (A^T)^T = A^T A = M.
\end{equation}
Ist nun $\lambda$ ein Eigenwert von $M$ und $v$ ein Eigenvektor dazu,
dann gilt $Mv=\lambda v$. Daraus folgt
\begin{equation}
\lambda |v|^2 = \langle\lambda v,v\rangle
= \langle Mv,v\rangle = \langle A^T Av,v\rangle
= \langle Av,Av\rangle = |Av|^2\ge 0.
\end{equation}
Ergo ist $\lambda|v|^2\ge 0$. Unter der Voraussetzung $v\ne 0$ ist
$|v|>0$. Dann muss auch $\lambda\ge 0$ sein. Wenn nun $\det(A)\ne 0$
ist, also $A$ eine reguläre Matrix, dann hat $A$ trivialen Kern,
also $Av=0$ nur im Fall $v=0$. Da $v\ne 0$ vorausgesetzt wurde,
muss auch $Av\ne 0$, und damit $|Av|>0$ sein. Dann ist auch
$\lambda>0$. Alternativ folgt $\lambda>0$ daraus, dass $\det(A)$
das Produkt der Eigenwerte ist.\;\qedsymbol
\end{Beweis}
