
\chapter{Analysis}
\section{Folgen}
\subsection{Konvergenz}

\begin{Definition}[Offene Epsilon-Umgebung]%
\index{Epsilon-Umgebung}\index{offene Epsilon-Umgebung}
Sei $(M,d)$ ein metrischer Raum. Unter der offenen Epsilon-Umgebung
von $a\in M$ versteht man:%
\[U_\varepsilon(a) := \{x\mid d(x,a)<\varepsilon\}.\]
\end{Definition}

\noindent
Man setze zunächst speziell $d(x,a):=|x-a|$ bzw. $d(x,a):=\|x-a\|$.

\begin{Definition}[Konvergente Folge, Grenzwert]%
\label{def:lim}\index{konvergente Folge}\index{Grenzwert}\newlinefirst
Eine Folge $(a_n)$ heißt konvergent gegen einen Grenzwert $a$, wenn es
zu jedem noch so kleinen $\varepsilon$ einen Index $n_0$ gibt, so dass
ab diesem Index sämtliche ihrer Werte in der Umgebung
$U_\varepsilon(a)$ liegen. Formal:
\[\lim_{n\to\infty} a_n = a
\defiff \forall\varepsilon{>}0\colon\exists n_0\colon\forall n{\ge}n_0\colon a_n\in U_\varepsilon(a)\]
bzw.
\[\lim_{n\to\infty} a_n = a
\defiff \forall\varepsilon{>}0\colon\exists n_0\colon\forall n{\ge}n_0\colon \|a_n-a\|<\varepsilon.\]
\end{Definition}

\begin{Definition}[Beschränkte Folge]%
\label{def:bseq}\index{beschreankte Folge@beschränkte Folge}
Eine Folge $(a_n)$ mit $a_n\in\R$ heißt genau dann beschränkt,
wenn es eine reelle Zahl $S$ gibt mit $|a_n|<S$ für alle $n$.

Eine Folge $(a_n)$ von Punkten eines normierten Raums heißt genau
dann beschränkt, wenn es eine reelle Zahl $S$ gibt mit $\|a_n\|<S$
für alle $n$.
\end{Definition}

\begin{Satz}[Grenzwert bei Konvergenz eindeutig bestimmt]\newlinefirst
Eine konvergente Folge von Elementen eines metrischen Raumes
besitzt genau einen Grenzwert.
\end{Satz}

\begin{Beweis}
Sei $(a_n)$ eine konvergente Folge mit $a_n\to g_1$. Sei weiterhin
$g_1\ne g_2$. Es wird nun gezeigt, dass $g_2$ kein Grenzwert von $a_n$
sein kann. Wir müssen also zeigen:
\[\neg\lim_{n\to\infty} a_n=g_2 \iff
\exists\varepsilon{>}0\colon\forall n_0\colon\exists n{\ge}n_0\colon
a_n\notin U_\varepsilon(g_2)\]
mit $a_n\notin U_\varepsilon(g_2)\iff d(a_n,g_2)\ge\varepsilon$.

Um dem Existenzquantor zu genügen, wählt man nun
$\varepsilon = \frac{1}{2}d(g_1,g_2)$.
Nach Def. \ref{metric-space} (Metrischer Raum) gilt 
$d(g_1,g_2)>0$, daher ist auch $\varepsilon>0$. Nach Satz
\ref{construction-disjoint-ep-balls} sind die Umgebungen
$U_\varepsilon(g_1)$ und $U_\varepsilon(g_2)$ disjunkt.
Wegen $a_n\to g_1$ gibt es ein $n_0$ mit $a_n\in U_\varepsilon(g_1)$ für alle
$n\ge n_0$. Dann gibt es für jedes beliebig große $n_0$ aber auch
$n\ge n_0$ mit $a_n\notin U_\varepsilon(g_2)$.\,\qedsymbol
\end{Beweis}

\begin{Satz}[Skaliertes Epsilon]\label{lim-scaled-ep}
Es gilt:
\[\lim_{n\to\infty} a_n=a \iff
\forall\varepsilon{>}0\colon\exists n_0\colon\forall n{\ge}n_0\colon \|a_n-a\|<R\varepsilon,\]
wobei $R>0$ ein fester aber beliebieger Skalierungsfaktor ist.
\end{Satz}

\begin{Beweis}
Betrachte $\varepsilon>0$ und multipliziere auf beiden Seiten
mit $R$. Dabei handelt es sich um eine Äquivalenzumformung.
Setze $\varepsilon':=R\varepsilon$. Demnach gilt:
\[\varepsilon>0 \iff \varepsilon'>0.\]
Nach der Ersetzungsregel düfen wir die Teilformel $\varepsilon>0$
nun ersetzen. Es ergibt sich die äquivalente Formel
\[\lim_{n\to\infty} a_n=a \iff
\forall\varepsilon'{>}0\colon\exists n_0\colon\forall n{\ge}n_0\colon
\|a_n-a\|<\varepsilon'.\]
Das ist aber genau Def. \ref{def:lim}.\,\qedsymbol
\end{Beweis}

\begin{Satz}
Es gilt:
\[\lim_{n\to\infty} a_n = a\implies \lim_{n\to\infty} \|a_n\| = \|a\|.\]
\end{Satz}

\begin{Beweis}
Nach Satz \ref{rev-tineq} (umgekehrte Dreiecksungleichung) gilt:
\[|\|a_n\|-\|a\|| \le \|a_n-a\| < \varepsilon.\]
Dann ist aber erst recht $|\|a_n\|-\|a\||<\varepsilon$.\,\qedsymbol
\end{Beweis}

\begin{Satz}\label{zero-seq-bounded}
Ist $(a_n)$ eine Nullfolge und $(b_n)$ eine beschränkte Folge,
dann ist auch $(a_n b_n)$ eine Nullfolge.
\end{Satz}

\begin{Beweis}
Wenn $(b_n)$ beschränkt ist, dann existiert nach
Def. \ref{def:bseq} eine Schranke $S$ mit
$|b_n|<S$ für alle $n$. Man multipliziert nun auf beiden Seiten
mit $|a_n|$ und erhält
\[|a_n b_n| = |a_n| |b_n| < |a_n| S.\]
Wenn $a_n\to 0$, dann muss für jedes $\varepsilon$
ein $n_0$ existieren mit $|a_n|<\varepsilon$ für $n\ge n_0$.
Multipliziert man auf beiden Seiten mit $S$, und ergibt sich
\[|a_n b_n-0| = |a_n b_n| < |a_n| S < S\varepsilon.\]
Nach Satz \ref{lim-scaled-ep} gilt dann
aber $a_n b_n\to 0$.\,\qedsymbol
\end{Beweis}

\begin{Satz}
Sind $(a_n)$ und $(b_n)$ Nullfolgen,
dann ist auch $(a_n b_n)$ eine Nullfolge.
\end{Satz}

\begin{Beweis}[Beweis 1]
Wenn $(b_n)$ eine Nullfolge ist, dann ist $(b_n)$ auch beschränkt.
Nach Satz \ref{zero-seq-bounded} gilt dann die Behauptung.
\end{Beweis}

\begin{Beweis}[Beweis 2]
Sei $\varepsilon>0$ beliebig.
Es gibt ein $n_0$, so dass
$|a_n|<\varepsilon$ und $|b_n|<\varepsilon$ für $n\ge n_0$.
Demnach ist
\[|a_n b_n| = |a_n| |b_n|< |a_n|\varepsilon <\varepsilon^2.\]
Wegen $\varepsilon>0\iff\varepsilon'>0$ mit
$\varepsilon'=\varepsilon^2$ gilt
\[\forall\varepsilon'{>}0\colon\exists n_0\colon\forall n{\ge}n_0\colon
|a_n b_n|<\varepsilon'.\]
Nach Def. \ref{def:lim} gilt somit die Behauptung.\,\qedsymbol
\end{Beweis}

\newpage
\begin{Satz}[Grenzwertsatz zur Addition]%
\label{lim-add}\index{Grenzwertsaetze@Grenzwertsätze}
Seien $(a_n)$, $(b_n)$ Folgen von Vektoren eines normierten Raumes.
Es gilt:
\[\lim_{n\to\infty} a_n = a\land \lim_{n\to\infty} b_n
= b \implies \lim_{n\to\infty} a_n+b_n = a+b.\]
\end{Satz}

\begin{Beweis}
Dann gibt es ein $n_0$, so dass für $n\ge n_0$ sowohl
$\|a_n-a\|<\varepsilon$ als auch $\|b_n-b\|<\varepsilon$.
Addition der beiden Ungleichungen führt zu
\[\|a_n-a\| + \|b_n-b\| < 2\varepsilon.\]
Laut der Dreiecksungleichung, das ist Axiom (N3) in Def.
\ref{def:normed-space} (normierter Raum), gilt nun aber die Abschätzung
\[\|(a_n+b_n)-(a+b)\| = \|(a_n-a)+(b_n-b)\| \le \|a_n-a\|+\|b_n-b\|.\]
Somit gilt erst recht
\[\|(a_n+b_n)-(a+b)\| < 2\varepsilon.\]
Nach Satz \ref{lim-scaled-ep} folgt die Behauptung.\,\qedsymbol
\end{Beweis}

\begin{Satz}[Grenzwertsatz zur Skalarmultiplikation]\label{lim-smult}
Sei $(a_n)$ eine Folge von Vektoren eines normierten Raumes
und sei $r\in\R$ oder $r\in\C$. Es gilt:
\[\lim_{n\to\infty} a_n = a\implies \lim_{n\to\infty} ra_n\to ra.\]
\end{Satz}

\begin{Beweis}
Sei $\varepsilon>0$ fest aber beliebig. Es gibt nun ein $n_0$, so
dass $\|a_n-a\|<\varepsilon$ für $n\ge n_0$.
Multipliziert man auf beiden Seiten
mit $|r|$ und zieht Axiom (N2) aus
Def. \ref{def:normed-space} (normierter Raum)
heran, dann ergibt sich
\[\|ra_n-ra\| = |r|\,\|a_n-a\|<|r|\varepsilon.\]
Nach Satz \ref{lim-scaled-ep} folgt die Behauptung.\,\qedsymbol
\end{Beweis}

\begin{Satz}[Grenzwertsatz zum Produkt]\newlinefirst
Seien $(a_n)$ und $(b_n)$ Folgen
reeller Zahlen. Es gilt:
\[\lim_{n\to\infty} a_n=a\land\lim_{n\to\infty} b_n=b\implies
\lim_{n\to\infty} a_n b_n = ab.\]
\end{Satz}

\begin{Beweis}
Nach Voraussetzung sind $a_n-a$ und $b_n-b$ Nullfolgen.
Da das Produkt von Nullfolgen wieder eine Nullfolge ist, gilt
\[(a_n-a)(b_n-b) = a_n b_n-a_n b-ab_n+ab\to 0.\]
Da nach Satz \ref{lim-smult} aber $a_n b\to ab$ und $ab_n\to ab$,
ergibt sich nach Satz \ref{lim-add} nun
\[(a_n-a)(b_n-b)+a_n b+ab_n = a_n b_n+ab\to 2ab.\]
Addiert man nun noch die konstante Folge $-2ab$
und wendet nochmals Satz \ref{lim-add} an, dann ergibt sich
die Behauptung
\[a_n b_n\to ab.\,\qedsymbol\]
\end{Beweis}

\begin{Satz}
Ist $(a_n)$ konvergent und $(b_n)$ divergent, dann ist
$(a_n+b_n)$ divergent.
\end{Satz}
\begin{Beweis}
Per Negationseinführung verbleibt das Ziel, aus der Annahme,
$(a_n+b_n)$ wäre konvergent, einen Widerspruch abzuleiten.
Mit den Grenzwertsätzen erhält man nun
\[\lim_{n\to\infty} (a_n + b_n) - \lim_{n\to\infty} a_n
= \lim_{n\to\infty} (a_n + b_n - a_n) = \lim_{n\to\infty} b_n,\]
womit $(b_n)$ konvergent sein würde, was jedoch im Widerspruch zur
zweiten Prämisse steht.\,\qedsymbol
\end{Beweis}

\begin{Satz}
Sei $(a_n)$ frei von Nullstellen. Ist $(a_n)$ konvergent und $(b_n)$
divergent, dann ist $(a_n b_n)$ divergent.
\end{Satz}
\begin{Beweis}
Per Negationseinführung verbleibt das Ziel, aus der Annahme,
$(a_n b_n)$ wäre konvergent, einen Widerspruch abzuleiten.
Mit den Grenzwertsätzen erhält man nun
\[\lim_{n\to\infty} (a_n^{-1})\lim_{n\to\infty} (a_n b_n)
= \lim_{n\to\infty} (a_n^{-1} a_n b_n) = \lim_{n\to\infty} b_n,\]
womit $(b_n)$ konvergent sein würde, was jedoch im Widerspruch
zur zweiten Prämisse steht.\,\qedsymbol
\end{Beweis}

\begin{Satz}
Seien $(a_n),(b_n)$ Folgen reeller Zahlen mit $a_n\to a$ und $b_n\to b$,
wobei $b_n>0$ und $b>0$. Dann gilt $(b_n)^{a_n}\to b^a$.
\end{Satz}
\begin{Beweis}
Aufgrund der Stetigkeit von $\exp$ und $\ln$ findet sich
\[\lim_{n\to\infty} (b_n)^{a_n} = \lim_{n\to\infty} \exp(\ln(b_n)a_n)
= \exp(\ln(\lim_{n\to\infty} b_n) \lim_{n\to\infty} a_n)
= \exp(\ln(b)a) = b^a.\,\qedsymbol\]
\end{Beweis}

\newpage
\begin{Satz}\label{cont-seqcont}%
\index{folgenstetig}\index{stetig!folgenstetig}
Sei $M$ ein metrischer Raum und $X$ ein topologischer Raum.
Eine Abbildung $f\colon M\to X$ ist genau dann stetig, wenn
sie folgenstetig ist.
\end{Satz}

\begin{Satz}[Satz zur Fixpunktgleichung]\index{Fixpunktgleichung}
Sei $M$ ein metrischer Raum und sei $f\colon M\to M$.
Sei $x_{n+1}:=f(x_n)$ eine Fixpunktiteration. Wenn die Folge
$(x_n)$ zu einem Startwert $x_0$ gegen ein $x\in M$ konvergiert, und
wenn $f$ eine stetige Abbildung ist, dann muss der Grenzwert $x$ die
Fixpunktgleichung $x=f(x)$ erfüllen.
\end{Satz}

\begin{Beweis}
Wenn $x_n\to x$, dann gilt trivialerweise auch $x_{n+1}\to x$.
Weil $f$ stetig ist, ist $f$ nach Satz \ref{cont-seqcont}
auch folgenstetig. Daher gilt $\lim f(a_n) = f(\lim a_n)$ für jede
konvergente Folge $(a_n)$. Somit gilt:%
\[x = \lim_{n\to\infty} x_{n+1} = \lim_{n\to\infty} f(x_n)
= f(\lim_{n\to\infty} x_n) = f(x).\;\qedsymbol\]
\end{Beweis}

\subsection{Wachstum und Landau-Symbole}
\begin{Definition}\label{Landau-O}
Seien $f,g\colon D\to\R$ mit $D=\N$ oder $D=\R$. Man sagt, die
Funktion $f$ wächst nicht wesentlich schneller als $g$, kurz
$f\in\mathcal O(g)$, genau dann, wenn%
\[\exists c{>}0\colon\,\exists x_0\colon\,\forall x{>}x_0\colon\, |f(x)|\le c|g(x)|.\]
\end{Definition}

\begin{Satz}
Ist $r\in\R$ mit $r\ne 0$ eine Konstante, dann gilt
$\mathcal O(rg)=\mathcal O(g)$.
\end{Satz}
\begin{Beweis}
Nach Def. \ref{Landau-O} ist
\[f\in\mathcal O(rg) \iff 
\exists c{>}0\colon\,\exists x_0\colon\,\forall x{>}x_0\colon\,|f(x)|\le c|rg(x)|.\]
Man hat nun
\[|f(x)|\le c|rg(x)| = c\cdot |r|\cdot |g(x)|.\]
Wegen $r\ne 0$ ist $|r|>0$ und daher auch $c>0\iff c|r|>0$. Sei
$c':=r|c|$. Also gilt $c>0\iff c'>0$. Nach der Ersetzungsregel
darf $c>0$ gegen $c'>0$ ersetzt werden und man erhält die
äquivalente Bedingung%
\[\exists c'{>}0\colon\,\exists x_0\colon\,
\forall x{>}x_0\colon\,|f(x)|\le c'|g(x)|.\]
Nach Def. \ref{Landau-O} ist das gerade $f\in\mathcal O(g)$.\;\qedsymbol
\end{Beweis}

\begin{Satz}
Sind $f_1,f_2\in\mathcal O(g)$, ist auch $f_1+f_2\in\mathcal O(g)$.
\end{Satz}
\begin{Beweis}
Als Prämissen liegen Zeugen $c'>0,x_0'$ und $c''>0,x_0''$ für%
\begin{gather*}
\forall x>x_0'\colon |f_1(x)|\le c'|g(x)|,\\
\forall x>x_0''\colon |f_2(x)|\le c''|g(x)|
\end{gather*}
vor. Mit der Dreiecksungleichung findet sich
\[|f_1(x)+f_2(x)|\le |f_1(x)|+|f_2(x)| \le c'|g(x)|+c''|g(x)| = (c'+c'')|g(x)|\]
für $x>\max(x_0',x_0'')$. Ergo sind $x_0:=\max(x_0',x_0'')$ und $c:=c'+c''$
Zeugen für%
\[\exists c>0\colon\exists x_0\colon\forall x>x_0\colon
  |f_1(x)+f_2(x)|\le c|g(x)|.\,\qedsymbol\]
\end{Beweis}

\newpage
\section{Stetige Funktionen}

\begin{Definition}[Grenzwert einer Funktion]\label{fn-lim}
Sei $f\colon D\to\R$ mit $D\subseteq\R$ und sei $p$ ein
Häufungspunkt von $D$. Die Funktion $f$ heißt konvergent
gegen $L$ für $x\to p$, wenn%
\[\forall \varepsilon{>}0\colon\,\exists \delta{>}0\colon\,\forall x{\in}D\colon\,
(0<|x-x_0|<\delta\implies |f(x)-L|<\varepsilon).\]
Bei Konvergenz schreibt man $L=\lim\limits_{x\to p} f(x)$ und nennt $L$ den Grenzwert.
\end{Definition}

\begin{Definition}[Stetig an einer Stelle]%
\label{cont}\index{stetig}\newlinefirst
Eine Funktion $f\colon D\to\R$ mit $D\subseteq\R$ heißt stetig an der
Stelle $x_0\in D$, wenn%
\[\forall \varepsilon{>}0\colon\,\exists \delta{>}0\colon\,\forall x{\in}D\colon\,
(|x-x_0|<\delta\implies |f(x)-f(x_0)|<\varepsilon).\]
\end{Definition}

\begin{Definition}[Dehnungsbeschränkte Funktion]%
\index{Lipschitz-stetig}\index{stetig!Lipschitz-stetig}\newlinefirst
Eine Funktion $f\colon D\to\R$ mit $D\subseteq\R$ heißt
dehnungsbeschränkt, auch Lipschitz"=stetig genannt, wenn eine
Konstante $L$ existiert, so dass%
\[|f(b)-f(a)|\le L|b-a|\]
für alle $a,b\in D$.
\end{Definition}

\begin{Definition}[Dehnungsbeschränkt an einer Stelle]%
\label{Lipschitz-cont-at}\newlinefirst
Eine Funktion $f\colon D\to\R$ mit $D\subseteq\R$ heißt
dehnungsbeschränkt an der Stelle $x_0\in D$, wenn eine Konstante $L$
existiert, so dass%
\[|f(x_0)-(a)|\le L|x_0-a|\]
für alle $a\in D$.
\end{Definition}

\begin{Satz}
Eine Funktion ist genau dann dehnungsbeschränkt, wenn sie an jeder
Stelle dehnungsbeschränkt ist und die Menge der optimalen
Lipschitz"=Konstanten dabei beschränkt.
\end{Satz}
\begin{Beweis}
Eine dehnungsbeschränkte Funktion ist trivialerweise an jeder Stelle
dehnungsbeschränkt. Ist $f\colon D\to\R$ an der Stelle $b$
dehnungsbeschränkt, dann existiert eine Lipschitz"=Konstante $L_b$ mit%
\[\forall a\in D\colon |f(b)-f(a)|\le L_b |b-a|.\]
Nach Voraussetzung ist $L=\sup_{b\in D} L_b$ endlich. Alle $L_b$ können
nun zu $L$ abgeschwächt werden und es ergibt sich%
\[\forall b\in D\colon\forall a\in D\colon |f(b)-f(a)|\le L|b-a|.\;\qedsymbol\]
\end{Beweis}


\begin{Definition}[Lokal dehnungsbeschränkt]\newlinefirst
Eine Funktion $f\colon D\to\R$ mit $D\subseteq\R$ heißt lokal
dehnungsbeschränkt in der Nähe einer Stelle $x_0\in D$, wenn es eine
Epsilon"=Umgebung $U_\varepsilon(x_0)$ gibt, so dass die Einschränkung
von $f$ auf diese Umgebung dehnungsbeschränkt ist. Die Funktion heißt
lokal dehnungsbeschränkt, wenn sie in der Nähe jeder Stelle
dehnungsbeschränkt ist.
\end{Definition}

\begin{Satz}\label{diff-nh-Lipschitz-cont-at}
Ist die Funktion $f\colon D\to\R$ an der Stelle $x_0$ differenzierbar,
dann gibt es ein $\delta>0$, so dass die Einschränkung von $f$
auf $U_\delta(x_0)$ an der Stelle $x_0$ dehnungsbeschränkt ist.
\end{Satz}

\begin{Beweis}
Def. \ref{fn-lim} wird in Def. \ref{diff} (diff) eingesetzt.
Es ergibt sich:%
\[0<|x-x_0|<\delta\implies
\left|\frac{f(x)-f(x_0)}{x-x_0}-f'(x_0)\right|<\varepsilon.\]
Nach der umgekehrten Dreiecksungleichung \ref{rev-tineq} gilt%
\[\left|\frac{f(x)-f(x_0)}{x-x_0}\right|-|f'(x_0)| \le
\left|\frac{f(x)-f(x_0)}{x-x_0}-f'(x_0)\right|
< \varepsilon.\]
Daraus ergibt sich
\[|f(x)-f(x_0)| < (|f'(x_0)|+\varepsilon)\cdot |x-x_0|\]
und somit erst recht
\[|f(x)-f(x_0)| \le (|f'(x_0)|+\varepsilon)\cdot |x-x_0|,\]
wobei jetzt auch $x=x_0$ erlaubt ist. Demnach wird Def.
\ref{Lipschitz-cont-at} erfüllt:%
\[\exists \delta{>}0\colon\,\forall x\in U_\delta(x_0)\colon\,
|f(x)-f(x_0)| \le (|f'(x_0)|+\varepsilon)\cdot |x-x_0|.\;\qedsymbol\]
\end{Beweis}

\begin{Satz}\label{diff-bounded-Lipschitz-cont}
Eine differenzierbare Funktion ist genau dann dehnungsbeschränkt,
wenn ihre Ableitung beschränkt ist.
\end{Satz}
\begin{Beweis}
Wenn $f\colon I\to\R$ dehnungsbeschränkt ist, dann gibt es $L$ mit%
\[\left|\frac{f(b)-f(a)}{b-a}\right|\le L\]
für alle $a,b\in D$ mit $a\ne b$. Daraus folgt%
\[|f'(a)| = \left|\lim_{b\to a} \frac{f(b)-f(a)}{b-a}\right|
= \lim_{b\to a} \left|\frac{f(b)-f(a)}{b-a}\right|
\le L.\]
Demnach ist die Ableitung beschränkt.

Sei nun umgekehrt die Ableitung beschränkt. Für $a,b\in I$ mit $a\ne b$
gibt es nach dem Mittelwertsatz ein $x_0\in(a,b)$, so dass%
\[|f'(x_0)| = \left|\frac{f(b)-f(a)}{b-a}\right|.\]
Da die Ableitung beschränkt ist gibt es ein Supremum
$L = \sup_{x\in I} |f'(x)|$. Demnach ist $|f'(x)|\le L$ für alle $x$.
Es ergibt sich%
\[\left|\frac{f(b)-f(a)}{b-a}\right|\le L|b-a| \implies |f(b)-f(a)|\le L|b-a|.\]
Nun darf auch $a=b$ gewählt werden.\;\qedsymbol
\end{Beweis}

\begin{Satz}\label{diff-compact-Lipschitz-cont}
Eine auf einem kompakten Intervall $[a,b]$ definierte stetig
differenzierbare Funktion ist dehnungsbeschränkt.
\end{Satz}
\begin{Beweis}
Sei $f\colon [a,b]\to\R$ stetig differenzierbar. Dann ist $f'(x)$ stetig.
Nach dem Satz vom Minimum und Maximum ist $|f'(x)|$ beschränkt. Nach
Satz \ref{diff-bounded-Lipschitz-cont} muss $f$ dehnungsbeschränkt
sein.\;\qedsymbol
\end{Beweis}

\begin{Satz}
Eine stetig differenzierbare Funktion ist lokal dehnungsbeschränkt.
\end{Satz}
\begin{Beweis}
Sei $f\colon D\to\R$ stetig differenzierbar. Sei $[a,b]\in D$. Sei
$x_0\in [a,b]$. Die Einschränkung von $f$ auf $[a,b]$ ist
dehnungsbeschränkt nach Satz \ref{diff-compact-Lipschitz-cont}.
Dann ist auch die Einschränkung von $f$ auf
$U_\varepsilon(x_0)\subseteq [a,b]$ dehnungsbeschränkt.\;\qedsymbol
\end{Beweis}

\begin{Satz}
Es gibt differenzierbare Funktionen, die nicht überall lokal
dehnungsbeschränkt sind.
\end{Satz}
\begin{Beweis}
Aus Satz \ref{diff-bounded-Lipschitz-cont} ergibt sich als
Kontraposition, dass eine Funktion mit unbeschränkter Ableitung
nicht dehnungsbeschränkt sein kann.

Ist $f\colon D\to\R$ an jeder Stelle differenzierbar und ist $f'$
in jeder noch so kleinen Umgebung der Stelle $x_0$ unbeschränkt, dann
kann $f$ also in der Nähe dieser Stelle auch nicht lokal
dehnungsbeschränkt sein.

Ein Beispiel für eine solche Funktion ist
$f\colon{}[0,\infty)\to\R$ mit%
\[f(0):=0\quad \text{und}\quad f(x):=x^{3/2}\cos\Big(\tfrac{1}{x}\Big).\]
Einerseits gilt
\[f'(0) = \lim_{h\to 0}\frac{f(0+h)-f(0)}{h} = \lim_{h\to 0}\frac{f(h)}{h}
= \lim_{h\to 0} (h^{1/2}\cos\Big(\tfrac{1}{h}\Big)) = 0.\]
Die Funktion ist also an der Stelle $x=0$ differenzierbar.
Andererseits gilt nach den Ableitungsregeln%
\[f'(x) = \frac{3}{2}\sqrt{x}\cos\Big(\tfrac{1}{x}\Big)
+ \frac{1}{\sqrt{x}}\sin\Big(\tfrac{1}{x}\Big)\]
für $x>0$. Der Term $\tfrac{1}{\sqrt{x}}$ erwirkt für $x\to 0$ immer
größere Maxima von $|f'(x)|$. Daher kann $f$ in der Nähe von $x=0$ nicht
lokal dehnungsbeschränkt sein.\;\qedsymbol
\end{Beweis}

\begin{Satz}
Sei $f\colon\R\to\R$ differenzierbar und $f(x)$ konvergent
für $x\to\infty$. Ist außerdem $f'$  dehnungsbeschränkt,
zieht dies $f'(x)\to 0$ für $x\to\infty$ nach sich.
\end{Satz}
\begin{Beweis}
Gemäß dem cauchyschen Konvergenzkriterium gibt es zu jedem
$\varepsilon>0$ eine Stelle $x_0$, so dass%
\[|f(b)-f(a)| < \varepsilon\]
für alle $a,b$ mit $x_0 < a \le b$. Nun ist $f'$ aufgrund
der Dehnungsbeschränktheit erst recht stetig, womit%
\[\bigg|\int_a^b f'(x)\,\mathrm dx\bigg| = |f(b)-f(a)|\]
laut dem Fundamentalsatz gilt. Gezeigt wird nun, dass $|f'(a)|$
beschränkt ist. Sei dazu $L$ die Lipschitz"=Konstante. Ohne
Beschränkung der Allgemeinheit sei $f'(a)>0$. Fallen darf $f'$ maximal
mit dem Anstieg $-L$. Geschieht dies linear bis zur Nullstelle $b$,
ergibt sich ein rechtwinkliges Dreieck mit dem Flächeninhalt%
\[\frac{1}{2L} f'(a)^2 = \int_a^b f'(x)\,\mathrm dx < \varepsilon.\]
Demnach ist $f'(a) < \sqrt{2L\varepsilon}$. Weil dies für alle $a>x_0$
gilt, muss $f'$ jede Beschränkung unterbieten, womit
der Beweis der Behauptung erbracht ist.\;\qedsymbol
\end{Beweis}

\noindent
Die Diskussion des Gegenbeispiels $f(0):=0$, $f(x):=\sin(x^2)/x$ macht
ersichtlich, dass die Aussage ohne Dehnungsbeschränktheit nicht einmal
für glatte Funktionen gilt.

\newpage
\begin{Satz}
Eine Bijektion $f\colon\R\to [0,1]$ kann nicht stetig sein.
\end{Satz}
\begin{Beweis}
Angenommen, $f$ ist stetig. Weil die Werte 0 und 1 Bilder von $f$
sind, existieren $a,b$ mit $f(a)=0$ und $f(b)=1$. Zudem gilt
$[0,1]\subseteq f([a,b])$ laut dem Zwischenwertsatz. Infolge findet sich
\[\R = f^{-1}([0,1]) \stackrel{\text{(1)}}\subseteq f^{-1}(f([a,b]))
\stackrel{\text{(2)}}= (f^{-1}\circ f)([a,b]) = \id([a,b]) = [a,b],\]
wobei (1) laut Satz \ref{img-subseteq} und (2) laut Satz
\ref{img-chain} gilt. Die Aussage $\R\subseteq [a,b]$ ist aber
absurd. Widerspruch zur Annahme.\,\qedsymbol
\end{Beweis}

\begin{Satz}
Die Funktion $f\colon\R\to\R$ mit $f(x):=ax+b$ ist stetig.
\end{Satz}
\begin{Beweis}
Zunächst unternimmt man die Termumformung
\[|f(x)-f(x_0)| = |ax-ax_0| = |a|\cdot |x-x_0|.\]
Laut Definition der Stetigkeit ist ein $\delta>0$ ausfindig zu machen,
mit dem die Ableitung von
\[\varepsilon>0, |x-x_0|<\delta\vdash |a|\cdot |x-x_0|<\varepsilon\]
gelingt. Im Fall $a=0$ reduziert sich die Sukzedenz zu
$0<\varepsilon$, was ja gegeben ist. Im Fall $a\ne 0$ gelangt man nun
durch äquivalente Umformung der Sukzedenz zu
$|x-x_0| < \frac{\varepsilon}{|a|}$.
Es darf also $\delta:=\frac{\varepsilon}{|a|}$ gesetzt werden.
Da die Umformung eine äquivalente war, darf sie rückgängig gemacht
werden, was zur gewünschten Aussage führt.\,\qedsymbol
\end{Beweis}

\begin{Satz}
Sei $a$ eine Konstante. Ist eine Funktion $f\colon\R\to\R$ an einer
Stelle stetig, dann ist auch $x\mapsto af(x)$ an dieser Stelle stetig.
\end{Satz}
\begin{Beweis}
Man unternimmt die Termumformung
\[|af(x)-af(x_0)| = |a|\cdot |f(x)-f(x_0)|.\]
Laut Definition ist somit ein $\delta>0$ zu finden, mit dem sich
\[f\;\text{stetig in}\;x_0,\varepsilon>0,|x-x_0|<\delta\vdash |a|\cdot |f(x)-f(x_0)|<\varepsilon\]
bestätigt. Im Fall $a=0$ ist dies trivial. Sei also $a\ne 0$.
Aus $\varepsilon>0$ folgt nun $\tfrac{\varepsilon}{|a|}>0$. Laut Definition gilt zudem
\[\forall\varepsilon'>0\colon\exists\delta>0\colon\forall x\colon |x-x_0|<\delta\to |f(x)-f(x_0)|<\varepsilon'.\]
Mit der Spezialisierung $\varepsilon':=\tfrac{\varepsilon}{|a|}$ erhält
man nun einen gewünschten Zeugen $\delta$. Mit $|x-x_0|<\delta$ folgt
daraufhin $|f(x)-f(x_0)|<\tfrac{\varepsilon}{|a|}$ per Modus ponens.
Äquivalente Umformung führt schließlich zur gewünschten Aussage
$|a|\cdot |f(x)-f(x_0)|<\varepsilon$.\,\qedsymbol
\end{Beweis}

\begin{Satz}
Sind $f,g$ stetig in einer Stelle, so auch $x\mapsto f(x)+g(x)$.
\end{Satz}
\begin{Beweis}
Es ist ein $\delta>0$ zu finden, mit dem
\[f,g\;\text{stetig in}\;x_0,\varepsilon>0,\|x-x_0\|<\delta\vdash
\|(f(x)+g(x))-(f(x_0)+g(x_0))\|<\varepsilon\]
bestätigt werden kann. Gegeben ist
\begin{gather*}
\forall\varepsilon'\;>0\colon\exists\delta'\;>0\colon\forall x\colon
  \|x-x_0\|<\delta'\;\to \|f(x)-f(x_0)\|<\varepsilon',\\
\forall\varepsilon''>0\colon\exists\delta''>0\colon\forall x\colon
  \|x-x_0\|<\delta''\to \|g(x)-g(x_0)\|<\varepsilon''.
\end{gather*}
Man spezialisiere $\varepsilon':=\frac{\varepsilon}{2}$ und
$\varepsilon'':=\frac{\varepsilon}{2}$. Man wähle nun
$\delta:=\min(\delta',\delta'')$. Dann impliziert $\|x-x_0\|<\delta$
sowohl $\|x-x_0\|<\delta'$ als auch $\|x-x_0\|<\delta''$. Per Modus ponens
erhält man also $\|f(x)-f(x_0)\|<\varepsilon'$ und $\|g(x)-g(x_0)\|<\varepsilon''$.
Mit diesen findet sich
\begin{align*}
\|(f(x)+g(x))-(f(x_0)+g(x_0))\| &= \|f(x)-f(x_0)+g(x)-g(x_0)\|\\
&\stackrel{\text{(1)}}\le \|f(x)-f(x_0)\|+\|g(x)-g(x_0)\|
< \varepsilon'+\varepsilon'' = \varepsilon,
\end{align*}
wobei für (1) die Dreiecksungleichung zur Anwendung kam.\,\qedsymbol
\end{Beweis}

\newpage
\section{Differentialrechnung}

\subsection{Ableitungsregeln}

\begin{Definition}[Differenzierbarkeit, Ableitung]%
\label{diff}\index{differenzierbar}\index{Ableitung}
Eine Funktion $f\colon D\to\R$ heißt differenzieraber an der Stelle
$x_0\in D$, wenn der Grenzwert%
\[f'(x_0) = \lim_{x\to x_0}\frac{f(x)-f(x_0)}{x-x_0}
= \lim_{h\to 0}\frac{f(x_0+h)-f(x_0)}{h}\]
existiert. Man nennt $f'(x_0)$ die Ableitung von $f$ an der Stelle
$x_0$.
\end{Definition}

\begin{Satz}\label{diff-add-sub-mul}\index{Produktregel}
Sei $I$ ein Intervall und $f,g\colon I\to\R$. Sind $f,g$
differenzierbar an der Stelle $x\in I$, dann ist auch%
\begin{align*}
f+g&\;\text{dort differenzierbar mit}\;(f+g)'(x)=f'(x)+g'(x),\\
f-g&\;\text{dort differenzierbar mit}\;(f-g)'(x)=f'(x)-g'(x),\\
fg&\;\text{dort differenzierbar mit}\;(fg)'(x)=f'(x)g(x)+f(x)g'(x).
\end{align*}
\end{Satz}

\begin{Beweis} Es gilt
\begin{gather*}
(f+g)'(x)
= \lim_{h\to 0}\frac{(f+g)(x+h)-(f+g)(x)}{h}\\
= \lim_{h\to 0}\frac{(f(x+h)+g(x+h))-(f(x)+g(x))}{h}\\
= \lim_{h\to 0}\bigg(\frac{f(x+h)-f(x)}{h}+\frac{g(x+h)-g(x)}{h}\bigg)\\
= \lim_{h\to 0}\frac{f(x+h)-f(x)}{h}+\lim_{h\to 0}\frac{g(x+h)-g(x)}{h}
= f'(x)+g'(x).
\end{gather*}
Da die Grenzwerte auf der rechten Seite nach Voraussetzung existieren,
muss auch der Grenzwert der Summe existieren.
Die Rechnung für die Subtraktion ist analog.

Bei der Multiplikation wird ein Nullsummentrick angewendet:
\begin{gather*}
g(x)f'(x)+f(x)g'(x)
= g(x)\lim_{h\to 0}\frac{f(x+h)-f(x)}{h}
+ f(x)\lim_{h\to 0}\frac{g(x+h)-g(x)}{h}\\
= \lim_{h\to 0}\bigg[g(x+h)\frac{f(x+h)-f(x)}{h}\bigg]
+ \lim_{h\to 0}\bigg[f(x)\frac{g(x+h)-g(x)}{h}\bigg]\\
= \lim_{h\to 0}\frac{f(x+h)g(x+h)-f(x)g(x+h)}{h}
+ \lim_{h\to 0}\frac{f(x)g(x+h)-f(x)g(x)}{h}\\
= \lim_{h\to 0}\frac{f(x+h)g(x+h)-f(x)g(x+h)+f(x)g(x+h)-f(x)g(x)}{h}\\
= \lim_{h\to 0}\frac{f(x+h)g(x+h)-f(x)g(x)}{h}
= \lim_{h\to 0}\frac{(fg)(x+h)-(fg)(x)}{h}
= (fg)'(x).
\end{gather*}
Hierbei wurde $\lim_{h\to 0}g(x+h)=g(x)$ benutzt, was richtig ist,
weil $g$ an der Stelle $x$ differenzierbar ist und dort somit ganz
sicher stetig.\;\qedsymbol
\end{Beweis}

\newpage
\begin{Satz}
Sei $I$ ein Intervall. Sind $f,g\colon I\to\R$ an der Stelle
$x$ differenzierbar und ist $g(x)\ne 0$, dann
ist auch $f/g$ differenzierbar und es gilt
\[\bigg(\frac{f}{g}\bigg)'(x) = \frac{f'(x)g(x)-f(x)g'(x)}{g(x)^2}.\]
\end{Satz}
\begin{Beweis}
Nach der Produktregel in Satz \ref{diff-add-sub-mul} gilt
\[0 = 1' = \bigg(g\cdot\frac{1}{g}\bigg)'
= g'\cdot\frac{1}{g}+g\cdot \bigg(\frac{1}{g}\bigg)'.\]
Umstellen bringt $(1/g)'(x)=-g'(x)/g(x)^2$. Nochmalige Anwendung der
Produktregel bringt
\begin{align*}
\bigg(\frac{f}{g}\bigg)'(x)
&= \bigg(f\cdot\frac{1}{g}\bigg)'(x)
= f'(x)\cdot\frac{1}{g(x)}+f(x)\bigg(\frac{1}{g}\bigg)'(x)\\
&= \frac{f'(x)}{g(x)}-\frac{f(x)g'(x)}{g(x)^2}
= \frac{f'(x)g(x)-f(x)g'(x)}{g(x)^2}.\;\qedsymbol
\end{align*}
\end{Beweis}

\begin{Satz}\label{diff-power}
Für $f\colon\R\to\R$, $f(x):=x^n$ mit $n\in\N$ gilt
$f'(x)=nx^{n-1}$.
\end{Satz}
\begin{Beweis}[Beweis 1]
Heranziehung des binomischen Lehrsatzes bringt
\begin{align*}
f'(x) &= \lim_{h\to 0}\frac{(x+h)^n-x^n}{h}
= \lim_{h\to 0}\frac{\sum_{k=0}^n\binom{n}{k}x^{n-k} h^k-x^n}{h}\\
&= \lim_{h\to 0}\bigg(nx^{n-1}+\sum_{k=2}^n\binom{n}{k}x^{n-k}h^{k-1}\bigg)
= nx^{n-1}.\;\qedsymbol
\end{align*}
\end{Beweis}
\begin{Beweis}[Beweis 2]
Induktiv. Der Induktionsanfang $\tfrac{\mathrm d}{\mathrm dx}x=1$ ist klar.
Mittels der Produktregel in Satz \ref{diff-add-sub-mul}
bewältigt man den Induktionsschritt 
\begin{align*}
\tfrac{\mathrm d}{\mathrm dx} x^n = \tfrac{\mathrm d}{\mathrm dx} (x\cdot x^{n-1})
= x^{n-1}+x\tfrac{\mathrm d}{\mathrm dx}x^{n-1}
= x^{n-1}+(n-1)x^{n-1} = nx^{n-1}.\;\qedsymbol
\end{align*}
\end{Beweis}

\begin{Satz}
Für $f\colon\R{\setminus}\{0\}\to\R$, $f(x):=x^n$ mit $n\in\Z$
gilt $f'(x)=nx^{n-1}$.
\end{Satz}
\begin{Beweis}
Der Fall $n=0$ ist trivial und $n\ge 1$ wurde schon in Satz
\ref{diff-power} gezeigt. Sei nun $a\in\Z_{\ge 1}$ und $n=-a$. Nach der
Produktregel (Satz \ref{diff-add-sub-mul}) und Satz
\ref{diff-power} gilt
\[0 = \tfrac{\mathrm d}{\mathrm dx} 1
= \tfrac{\mathrm d}{\mathrm dx} (x^a x^{-a})
= x^{-a}\tfrac{\mathrm d}{\mathrm dx} x^a+x^a\tfrac{\mathrm d}{\mathrm dx} x^{-a}
= x^{-a}ax^{a-1}+x^a\tfrac{\mathrm d}{\mathrm dx} x^{-a}.\]
Dividiert man nun durch $x^a$ und formt um, dann ergibt sich
\[\tfrac{\mathrm d}{\mathrm dx} x^{-a} = -ax^{-a-1},
\;\text{das heißt,}\;
\tfrac{\mathrm d}{\mathrm dx} x^n = nx^{n-1}.\;\qedsymbol\]
\end{Beweis}

\newpage
\subsection{Glatte Funktionen}

\begin{Satz}
Sei $f\colon\R\to\R$ eine Funktion mit der Eigenschaft
$f(x)=0$ für $x\le 0$ und $f(x)>0$ für $x>0$. Es gibt glatte Funktionen
mit dieser Eigenschaft, jedoch keine analytischen.
\end{Satz}

\begin{Beweis}
Wegen $f(x)=0$ für $x\le 0$ muss die linksseitige $n$-te Ableitung
an der Stelle $x=0$ immer verschwinden. Wenn die $n$-te Ableitung
stetig sein soll, muss auch die rechtsseitige Ableitung bei $x=0$
verschwinden. Da die Funktion glatt sein soll, muss das für jede
Ableitung gelten. Daher verschwindet die Taylorreihe an der Stelle
$x=0$. Da aber $f(x)>0$ für $x>0$, gibt es keine noch so kleine
Umgebung mit Übereinstimmung von $f$ und ihrer Taylorreihe.
Daher kann $f$ an der Stelle $x=0$ nicht analytisch sein.

Eine glatte Funktion lässt sich jedoch konstruieren:
\[f(x):=\begin{cases}
\ee^{-1/x}&\text{wenn}\;x>0,\\
0&\text{wenn}\;x\le 0.
\end{cases}\]
Ist nämlich $g(x)$ an einer Stelle glatt, dann ist
es nach Kettenregel, Produktregel und Summenregel auch $\ee^{g(x)}$.
Die $n$-te Ableitung lässt sich immer in der Form%
\[\sum\nolimits_k e^{g(x)}{r_k(x)}
= e^{g(x)}\sum\nolimits_k r_k(x) = e^{g(x)}r(x)\]
darstellen, wobei die $r_k(x)$ bzw. $r(x)$ in diesem Fall rationale
Funktionen mit Polstelle bei $x=0$ sind. Da aber $e^{-1/x}$ für
$x\to 0$ schneller fällt als jede rationale Funktion steigen kann,
muss die rechtsseitige Ableitung an der Stelle $x=0$ immer
verschwinden.\;\qedsymbol
\end{Beweis}

\subsection{Richtungsableitung}

\begin{Definition}[Richtungsableitung]\index{Richtungsableitung}
Sei $U\subseteq\R^n$ offen, $x\in U$ eine Stelle und $v\in\R^n$
ein Vektor. Man betrachte für ein kleines $\varepsilon>0$
die Parametergerade
\[\gamma\colon(-\varepsilon,\varepsilon)\to U,\quad \gamma(t):=x+tv.\]
Für eine Funktion $f\colon U\to\R$ ist die Zahl
\[D_v f(x) := (f\circ\gamma)'(0) = \lim_{h\to 0}\frac{f(x+hv)-f(x)}{h},\]
falls sie existiert, die Richtungsableitung von $f$ an der Stelle $x$ in
Richtung $v$.
\end{Definition}
\begin{Satz}
Die Funktionen $f,g$ seien an der Stelle $x$ in Richtung $v$
differenzierbar. Sei $c$ eine reelle Zahl. Dann sind auch
$f+g$, $f-g$, $cf$, $fg$ differenzierbar und es gelten die
den üblichen Ableitungsregeln analogen Regeln
\begin{align*}
D_v(f+g)(x) &= D_v f(x)+D_v g(x),\\
D_v(f-g)(x) &= D_v f(x)+D_v g(x),\\
D_v(cf)(x) &= cD_v f(x),\\
D_v(fg)(x) &= g(x)D_v f(x) + f(x)D_v g(x).
\end{align*}
\end{Satz}
\begin{Beweis}
Die Ableitungsregeln werden über die Definition
auf die Ableitungsregeln für gewöhnliche reelle Funktionen
zurückgeführt. So ist
\begin{align*}
D_v(f+g)(x) &= ((f+g)\circ\gamma)'(0)
= ((f\circ\gamma)+(g\circ\gamma))'(0)\\
&= (f\circ\gamma)'(0)+(g\circ\gamma)'(0)
= D_v f(x) + D_v g(x).
\end{align*}
Der Beweis der restlichen Regeln ist analog.\,\qedsymbol
\end{Beweis}

\begin{Satz}[Kettenregel]\newlinefirst
Sei $g\colon\R\to\R$ differenzierbar und
$f$ differenzierbar an der Stelle $x$ in Richtung $v$. Dann ist
auch $g\circ f$ entsprechend differenzierbar, und es gilt
\[D_v(g\circ f)(x) = (g'\circ f)(x)\cdot D_v f(x).\]
\end{Satz}
\begin{Beweis}
Die Regel ist gemäß der Definition auf die gewöhnliche Kettenregel
zurückführbar. Man bekommt
\begin{align*}
D_v(g\circ f)(x) = (g\circ f\circ\gamma)'(0)
= g'(f(\gamma(0)))\cdot (f\circ\gamma)'(0)
= g'(f(x))\cdot D_v f(x).\;\qedsymbol
\end{align*}
\end{Beweis}

\begin{Definition}[Partielle Ableitung]\index{partielle Ableitung}%
\index{Ableitung!partielle}\newlinefirst
Sei $(\mathbf e_1,\ldots,\mathbf e_n)$ die Standardbasis. 
Die partielle Ableitung $\partial_k f(x)$ ist definiert als
die Richtungsableitung $D_v f(x)$ bezüglich $v=\mathbf e_k$.
\end{Definition}

\begin{Satz}
Zur jeder gewöhnlichen Ableitungsregel besitzt die
Richtungsableitung eine analoge Regel.
\end{Satz}
\strong{Vorbereitung.}
Sei $f=(f_1,\ldots,f_n)$ ein Tupel von Funktionen aus einem
Funktionenraum und sei entsprechend
$f(x):=(f_1(x),\ldots, f_n(x))$. Sei $p$ eine beliebige mehrstellige
Operation. Sei $\eta_p(f)(x)$ die punktweise Anwendung von $p$.
Ein Beispiel ist die Addition $p(y_1,y_2):=y_1+y_2$. Dann ist
$\eta_p(f_1,f_2)(x)=f_1(x)+f_2(x)$. Sei%
\[F(T)(f) := (T f_1,\ldots ,T f_n)\]
die komponentenweise Anwendung eines Operators $T$.
Sei $C_\gamma$ der durch $C_\gamma f := f\circ\gamma$ definierte
Kompositionsoperator. Allgemein gilt%
\[C_\gamma\circ\eta_p = \eta_p\circ F(C_\gamma).\]

\begin{Beweis}
Prämisse ist, dass der gewöhnliche Ableitungsoperator $D$ die Regel
\[D(\eta_p(f))(x) = (D\circ\eta_p)(f)(x) = R(f(x),F(D)(f)(x))\]
erfüllt. Für die Richtungsableitung von $\eta_p(f)$ gilt dann
\begin{gather*}
D_v(\eta_p(f))(x) = (\eta_p(f)\circ\gamma)'(0)
= (D\circ C_\gamma\circ\eta_p)(f)(0)
= (D\circ\eta_p\circ F(C_\gamma))(f)(0)\\
= (D\circ\eta_p)(F(C_\gamma)(f))(0)
= R(F(C_\gamma)(f)(0),F(D)(F(C_\gamma)(f))(0))\\
= R(f(x),F(D\circ C_\gamma)(f)(0))
= R(f(x),F(D_v)(f)(x)).\;\qedsymbol
\end{gather*}
\end{Beweis}

\noindent
Beispiele sind
\begin{gather*}
\begin{array}{ll}
p(y_1,y_2) = y_1+y_2, & R((y_1,y_2),(y_1',y_2')) = y_1'+y_2',\\[4pt]
p(y_1,y_2) = y_1 y_2, & R((y_1,y_2),(y_1',y_2')) = y_1'y_2 + y_1y_2',\\[4pt]
p(y) = cy, & R(y,y') = cy',\\[4pt]
p(y) = g(y), & R(y,y') = g'(y)y'.
\end{array}
\end{gather*}

\newpage
\section{Fixpunkt-Iterationen}%
\index{Fixpunkt-Iteration}

\begin{Definition}[Kontraktion]\index{Kontraktion}
Sei $(M,d)$ ein vollständiger metrischer Raum. Eine Abbildung
$\varphi\colon M\to M$ heißt Kontraktion, wenn sie
dehnungsbeschränkt mit Lipschitz"=Konstante $L<1$ ist, das heißt,
\[d(\varphi(x),\varphi(y))<L\,d(x,y)\]
für alle $x,y\in M$.
\end{Definition}

\begin{Satz}[Fixpunktsatz von Banach]\label{Banach-fixed-point-theorem}%
\index{Fixpunktsatz von Banach}\index{Banach!Fixpunktsatz von}
Sei $(M,d)$ ein nichtleerer vollständiger metrischer Raum
und sei $\varphi\colon M\to M$ eine Kontraktion. Es gibt genau
einen Fixpunkt $x\in M$ mit $x=\varphi(x)$ und die Folge
$(x_n)\colon\N\to M$ mit $x_{n+1}=\varphi(x_n)$ konvergiert
gegen den Fixpunkt, unabhängig vom Startwert $x_0$.
\end{Satz}

\begin{Satz}[Hinreichendes Konvergenzkriterium]\label{diff-fixed-point-iter}
Sei $M=[a,b]$. Ist $\varphi\colon M\to M$ differenzierbar und gibt es
eine Zahl $r$ mit $|\varphi'(x)|<r<1$ für alle $x\in M$, dann
hat $\varphi$ genau einen Fixpunkt und die Folge $(x_n)$ mit $x_{n+1}=\varphi(x_n)$
konvergiert für jeden Startwert $x_0\in M$ gegen diesen Fixpunkt.
\end{Satz}
\begin{Beweis}
Nach Satz \ref{diff-bounded-Lipschitz-cont} ist eine differenzierbare
Funktion $\varphi$ mit beschränkter Ableitung auch dehnungsbeschränkt,
und $L=\sup_{x\in M}|\varphi'(x)|$ eine Lipschitz"=Konstante.
Wegen $|\varphi'(x)|<r$ muss $L\le r$ sein, und somit $L<1$.
Das heißt, $\varphi$ ist eine Kontraktion. Die Konvergenz der Folge
$(x_n)$ ist gemäß Satz \ref{Banach-fixed-point-theorem}
gewährleistet.\;\qedsymbol
\end{Beweis}

\begin{Satz}[Hinreichendes Konvergenzkriterium zum Newton-Verfahren]%
\index{Newton-Verfahren}\newlinefirst
Sei $f\colon [a,b]\to\R$ zweimal stetig differenzierbar und
$f'(x)\ne 0$ für alle $x$. Sei%
\[\varphi\colon [a,b]\to [a,b],\quad \varphi(x):=x-\frac{f(x)}{f'(x)}.\]
Man beachte $\varphi([a,b])\subseteq [a,b]$. Gilt für alle $x$ die Ungleichung%
\[|\varphi'(x)| = \bigg|\frac{f(x)f''(x)}{f'(x)^2}\bigg| < 1,\]
dann besitzt $f$ genau eine Nullstelle und die Folge $(x_n)$ mit
$x_{n+1}=\varphi(x_n)$ konvergiert gegen diese Nullstelle.
\end{Satz}

\begin{Beweis}
Gemäß den Ableitungsregeln ist $\varphi$ stetig differenzierbar
und es gilt%
\[\varphi'(x) = 1-\frac{f'(x)f'(x)-f(x)f''(x)}{f'(x)^2}
= \frac{f(x)f''(x)}{f'(x)^2}.\]
Da $|\varphi'(x)|$ stetig ist, gibt es nach dem Satz vom Minimum
und Maximum ein Maximum $M$ und nach Voraussetzung ist $M<1$.
Man setze nun $r:=(M+1)/2$. Dann ist $|\varphi'(x)|<r<1$.
Gemäß Satz \ref{diff-fixed-point-iter} konvergiert die Iteration
$(x_n)$ gegen den einzigen Fixpunkt von $\varphi$. Wegen $f'(x)\ne 0$
gilt dabei%
\[x = \varphi(x) = x-\frac{f(x)}{f'(x)} \iff \frac{f(x)}{f'(x)}=0\iff f(x)=0.\]
Der Fixpunkt von $\varphi$ ist also die einzige Nullstelle von $f$.\;\qedsymbol
\end{Beweis}

\newpage
\section{Ungleichungen}

\begin{Satz}[Lagrangesche Schranke]\index{lagrangesche Schranke}%
\index{Schranke!lagrangesche}\newlinefirst
Sei $a_k\in\C$ und $a_n\ne 0$. Für jede Lösung $z\in\C$ der Gleichung
$\sum_{k=0}^n a_k z^k = 0$ gilt%
\[|z|\le\max(1,\sum_{k=0}^{n-1}\bigg|\frac{a_k}{a_n}\bigg|).\]
\end{Satz}
\begin{Beweis}
Sei $z$ eine Lösung. Für $|z|\le 1$ ist die Aussage offenkundig
erfüllt. Sei also $|z|>1$, womit insbesondere $|z|\ne 0$ gilt.
Abspaltung des Leitmonoms von der Summe bringt die Gleichung in die
Form $-a_n z^n = \sum_{k=0}^{n-1} a_k z^k$.
Man findet nun%
\[|a_n| |z|^n = |a_n z^n| = \bigg|\sum_{k=0}^{n-1} a_k z^k\bigg|
\stackrel{\text{(1)}}\le\sum_{k=0}^{n-1} |a_k| |z|^k
\stackrel{\text{(2)}}\le\sum_{k=0}^{n-1} |a_k| |z|^{n-1}.\]
Dividiert man diese Ungleichung auf beiden Seiten durch $|a_n|$
und $|z|^{n-1}$, ergibt sich%
\[|z|\le \sum_{k=0}^{n-1}\frac{|a_k|}{|a_n|}.\]
Die Abschätzung (1) gilt gemäß der Dreiecksungleichung. Zur
Abschätzung (2) überlegt man sich $|z|^k\le |z|^{n-1}$ wegen
$|z|>1$.\,\qedsymbol
\end{Beweis}

\begin{Satz}[Cauchysche Schranke]\index{cauchysche Schranke}%
\index{Schranke!cauchysche}\newlinefirst
Sei $a_k\in\C$ und $a_n\ne 0$. Für jede Lösung $z\in\C$ der Gleichung
$\sum_{k=0}^n a_k z^k = 0$ gilt%
\[|z| \le 1 + \max(\bigg|\frac{a_0}{a_n}\bigg|,\ldots,\bigg|\frac{a_{n-1}}{a_n}\bigg|).\]
\end{Satz}
\begin{Beweis}
Sei $z$ eine Lösung. Für $|z|\le 1$ ist die Aussage offenkundig
erfüllt. Sei also $|z|>1$. Abspaltung des Leitmonoms von der Summe
bringt die Gleichung in die Form $-a_n z^n = \sum_{k=0}^{n-1} a_k z^k$.
Man findet
\[|a_n| |z|^n = |a_n z^n| = \bigg|\sum_{k=0}^{n-1} a_k z^k\bigg|
\stackrel{\text{(1)}} \le \sum_{k=0}^{n-1} |a_k| |z|^k
\stackrel{\text{(2)}} \le \sum_{k=0}^{n-1} m |z|^k
= m\frac{|z|^n-1}{|z|-1} \le \frac{m|z|^n}{|z|-1},\]
wobei $m=\max(|a_0|,\ldots,|a_{n-1}|)$. Kürzen von $|z|^n$ und
anschließende Umformung nach $|z|$ resultiert in
\[|z| \le 1 + \frac{m}{|a_n|} \stackrel{\text(3)}=
1 + \max(\frac{|a_0|}{|a_n|},\ldots,\frac{|a_{n-1}|}{|a_n|}).\]
Die Abschätzung (1) gilt gemäß der Dreiecksungleichung. Die
Abschätzung (2) gilt aufgrund $|a_k|\le m$ für jedes $k$. Gleichung
(3) gilt, weil die Maximumsfunktion positiv homogen ist.\,\qedsymbol
\end{Beweis}

\newpage
\section{Monotonie}
\begin{Definition}[Monotone reelle Funktion]\label{def:monotonic}\newlinefirst
Eine Funktion $f\colon X\to \R$ mit $X\subseteq\R$ heißt
\begin{align*}
\text{monoton steigend}&\defiff\forall a,b\in X\colon a\le b\Rightarrow f(a)\le f(b),\\
\text{monoton fallend}&\defiff\forall a,b\in X\colon a\le b\Rightarrow f(b)\le f(a),\\
\text{streng monoton steigend}&\defiff\forall a,b\in X\colon a<b\Rightarrow f(a)<f(b),\\
\text{streng monoton fallend}&\defiff\forall a,b\in X\colon a<b\Rightarrow f(b)<f(a).
\end{align*}
\end{Definition}

\begin{Satz}
Jede streng monotone Funktion ist injektiv.
\end{Satz}
\begin{Beweis}
Laut Def. \ref{def:inj} ist $a=b$ unter der Annahme $f(a)=f(b)$ zu
bestätigen. Es sei $f$ streng monoton steigend. Wäre $a<b$, dann wäre $f(a)<f(b)$,
was der Annahme widerspricht. Wäre $b<a$, dann wäre $f(b)<f(a)$, was der
Annahme ebenso widerspricht. Aufgrund der Trichotomie verbleibt nur
noch $a=b$. Für streng monoton fallende Funktionen verläuft die
Argumentation analog.\,\qedsymbol
\end{Beweis}

\begin{Satz}
Ist $f$ streng monoton steigend und $g$ wenigstens monoton steigend,
dann ist $f+g$ streng monoton steigend.
\end{Satz}
\begin{Beweis}
Laut Def. \ref{def:monotonic} ist
\[a<b\implies f(a)+g(a) < f(b)+g(b)\]
zu bestätigen. Mit den mit Def. \ref{def:monotonic} entfalteten Prämissen
impliziert $a<b$ zunächst $f(a)<f(b)$ und $g(a)\le g(b)$. Weiterhin gilt
\begin{align*}
f(a) < f(b) &\implies f(a)+g(a) < f(b)+g(a),\\
g(a)\le g(b) &\implies f(b)+g(a)\le f(b)+g(b).
\end{align*}
Per Transitivgesetz erhält man nun das gewünschte Resultat.\,\qedsymbol
\end{Beweis}

\begin{Satz}
Die Potenzfunktion $f\colon\R_{\ge 0}\to\R, f(x):=x^n$ ist für $n\in\Z_{\ge 1}$
streng monoton steigend.
\end{Satz}
\begin{Beweis}
Induktion über $n$. Im Anfang $n=1$ ist die Forderung von Def.
\ref{def:monotonic} trivial erfüllt. Im Schritt ist
\[0\le a,a<b,a^n<b^n\vdash a^{n+1}<b^{n+1}\]
zu bestätigen. Mit $a\ge 0$ ist auch $a^n\ge 0$.
Multipliziert man also $a<b$ auf beiden Seiten mit $a^n$,
erhält man $a^{n+1}\le a^n b$. Multipliziert man $a^n<b^n$ auf beiden
Seiten mit $b>0$, erhält man $a^n b < b^{n+1}$. Per Transitivgesetz
findet sich nun $a^{n+1}<b^{n+1}$.\,\qedsymbol
\end{Beweis}

\begin{Satz}
Sei $I$ ein offenes Intervall und $f\colon I\to\R$. Ist $f$ differenzierbar
mit positiver Ableitung an jeder Stelle, dann ist $f$ streng monoton steigend.
\end{Satz}
\begin{Beweis}
Seien $a,b\in I$ beliebig. Laut Def. \ref{def:monotonic} ist $f(a)<f(b)$
aus $a<b$ zu folgern. Laut dem Mittelwertsatz der Differentialrechnung
existiert eine Stelle $x\in (a,b)$ mit
\[f(b)-f(a) = f'(x)(b-a).\]
Gemäß der Prämisse ist nun $f'(x)>0$. Wegen $a<b$ ist außerdem $b-a>0$.
Ergo ist $f'(x)(b-a)>0$ und somit $f(b)-f(a)>0$, also $f(a)<f(b)$.\,\qedsymbol
\end{Beweis}

\newpage
\section{Spezielle Funktionen}
\subsection{Gammafunktion}

\begin{Definition}[Gammafunktion]
Für $x>0$ definiert man
\[\Gamma(x) := \int_0^\infty t^{x-1}\ee^{-t}\,\mathrm dt.\]
\end{Definition}

\begin{Satz}
Die Gammafunktion ist wohldefiniert.
\end{Satz}
\begin{Beweis}
Es ist zu bestätigen, dass das Integral, durch das $\Gamma(x)$ definiert
wurde, für $x>0$ existiert. Zu beachten ist hierbei, dass es
für $x<1$ ebenfalls bezüglich der unteren Grenze ein uneigentliches ist.
Wir trennen es daher bezüglich der Stelle $t=1$ in zwei Summanden auf.
Man hat also
\[\Gamma(x) = I_1 + I_2 = \int_0^1 t^{x-1}\ee^{-t}\,\mathrm dt +
\int_1^\infty t^{x-1}\ee^{-t}\,\mathrm dt.\]
Wir haben nun
\[\lim_{t\searrow 0} \frac{t^{x-1}\ee^{-t}}{t^{x-1}}
= \lim_{t\searrow 0}\ee^{-t} = 1.\]
Somit besitzt $I_1$ laut dem
Grenzwertkriterium für uneigentliche Integrale dasselbe
Konvergenzverhalten wie $\int_0^1 t^{x-1}\,\mathrm dt$, das für
$x>0$ konvergiert. Außerdem haben wir
\[\lim_{t\to\infty}\frac{t^{x-1}\ee^{-t}}{1/t^2}
= \lim_{t\to\infty} t^{x+1}\ee^{-t} = 0.\]
Laut dem Grenzwertkriterium muss $I_2$ also für alle $x\in\R$
konvergieren, da das Integral $\int_1^\infty \frac{1}{t^2}\,\mathrm dt$
konvergent ist.\,\qedsymbol
\end{Beweis}

\begin{Satz}\label{Gamma-funceq}
Die Gammafunktion genügt der Funktionalgleichung
$\Gamma(x+1)=x\Gamma(x)$.
\end{Satz}
\begin{Beweis}
Man rechnet die partielle Integration
\[\Gamma(x+1) = \int_0^\infty t^x\,\ee^{-t}\,\mathrm dt =
[-t^x\ee^{-t}]_{t=0}^\infty + x\int_0^\infty t^{x-1}\ee^{-t}\,\mathrm dt
= 0 + x\Gamma(x).\,\qedsymbol\]
\end{Beweis}

\begin{Satz}
Es gilt $n! = \Gamma(n+1) \stackrel{\text{def}}= \int_0^\infty t^n\,\ee^{-t}\,\mathrm dt$.
\end{Satz}
\begin{Beweis}
Induktion über $n$. Im Anfang $n=0$ bestätigt man $\Gamma(1)=1=0!$
mühelos. Im Induktionsschritt rechnet man
\[\Gamma(n+1) \stackrel{\text{(1)}}= n\Gamma(n)
\stackrel{\mathrm{IV}}= n(n-1)! \stackrel{\text{(2)}}= n!,\]
wobei (1) gemäß Satz \ref{Gamma-funceq} und (2) laut Definition
\ref{def:factorial} gilt.\,\qedsymbol
\end{Beweis}

\begin{Satz}
Es gilt $n! = \int_0^1 \ln\big(\frac{1}{x}\big)^n\,\mathrm dx$.
\end{Satz}
\begin{Beweis}
Per Substitution $\ln(1/x)=t$ erhält man
\[
\lim_{a\to 0}\int_a^1 \ln\big(\tfrac{1}{x}\big)^n\,\mathrm dx
= \lim_{a\to 0}\int_{\ln(1/a)}^0 -t^n\ee^{-t}\,\mathrm dt
= \int_\infty^0 -t^n\ee^{-t}\,\mathrm dt = \int_0^\infty t^n\ee^{-t}\,\mathrm dt
= n!.\,\qedsymbol
\]
\end{Beweis}

\newpage
\section{Maßtheorie}

\subsection{Grundlagen}

\begin{Definition}[Sigma-Algebra]\label{def:sigma-algebra}\newlinefirst
Ein Mengensystem $\setsystem A\subseteq\powerset(X)$ heißt $\sigma$"=Algebra auf
$X$, wenn gilt,
\begin{enumerate}\setlength{\parskip}{0pt}
\item $\emptyset\in\setsystem A$,
\item aus $A\in\setsystem A$ folgt $A^\comp\in\setsystem A$ mit $A^\comp := X\setminus A$,
\item aus $(\forall n\in\N\colon A_n\in\setsystem A)$ folgt
$\bigcup_{n\in\N} A_n\in\setsystem A$.
\end{enumerate}
\end{Definition}

\begin{Satz}
Sei $\setsystem A$ eine $\sigma$"=Algebra auf $X$. Gilt $\forall n\in\N\colon A_n\in\setsystem A$,
so gilt auch $\bigcap_{n\in\N} A_n\in\setsystem A$. Gilt $A,B\in\setsystem A$,
so gilt auch $A\cup B\in\setsystem A$, $A\cap B\in\setsystem A$,
$A\setminus B\in\setsystem A$.
\end{Satz}
\begin{Beweis}
Sei $n\in\N$ und $A_n\in\setsystem A$. Laut Axiom 2 gilt $(A_n)^\comp\in\setsystem A$, laut Axiom 3 also
\[\textstyle (\bigcap_{n\in\N} A_n)^\comp = \bigcup_{n\in\N} (A_n)^\comp\in\setsystem A.\]
Laut Axiom 1 folgt somit $\bigcap_{n\in\N} A_n = (\bigcap_{n\in\N} A_n)^{\comp\comp}\in\setsystem A$.

Sei nun $A_0:=A$, $A_1:=B$ und $A_n:=\emptyset$ für $n\ge 2$. Dann gilt
\[\textstyle A\cup B = \bigcup_{n\in\N} A_n\in\setsystem A.\]
Sei nun $A_0:=A$, $A_1:=B$ und $A_n:=X$ für $n\ge 2$. Dann gilt
\[\textstyle A\cap B = \bigcap_{n\in\N} A_n\in\setsystem A.\]
Die letzte Aussage $A\setminus B\in\setsystem A$ folgt hiermit schließlich aus
$A\setminus B = A\cap B^\comp$.\,\qedsymbol
\end{Beweis}

\begin{Satz}\label{sigma-algebras-intersection}
Ist $\setsystem A_i$ für jedes $i\in I\ne\emptyset$ eine $\sigma$"=Algebra,
so ist auch $\setsystem A := \bigcap_{i\in I}\setsystem A_i$ eine.
\end{Satz}
\begin{Beweis}
Via Nachweis der drei Axiome aus Def. \ref{def:sigma-algebra}.

Zum ersten Axiom. Laut Voraussetzung ist $\emptyset\in A_i$ für jedes $i\in I$,
und somit laut Def. \ref{def:intersection} auch $\emptyset\in\setsystem A$.

Zum zweiten Axiom. Es gelte $A\in\setsystem A$. Laut Def.
\ref{def:intersection} muss somit $A\in\setsystem A_i$ für jedes $i\in I$.
Ergo gilt $A^\comp\in\setsystem A_i$ für jedes $i\in I$ laut der
Voraussetzung. Mit nochmaliger Anwendung von Def. \ref{def:intersection}
erhält man folglich $A^\comp\in\setsystem A$.

Zum dritten Axiom. Es gelte $A_n\in\setsystem A$ für jedes $n\in\N$.
Laut Def. \ref{def:intersection} gilt daher $A_n\in\setsystem A_i$
und laut Voraussetzung somit $\bigcup_{n\in\N}\in\setsystem A_i$ für
jedes $i\in I$. Mit nochmaliger Anwendung von Def. \ref{def:intersection}
folgt also $\bigcup_{n\in\N}\in\setsystem A$.\,\qedsymbol
\end{Beweis}

\begin{Satz}
Es sei $\varphi_X(\setsystem A)$ die Aussage »$\setsystem A$ ist eine
$\sigma$"=Algebra auf $X$«. Zu
\[H\colon\powerset(\powerset(X))\to\powerset(\powerset(X)),\quad
H(\setsystem B) := \bigcap\{\setsystem A\mid
\setsystem B\subseteq\setsystem A\land\varphi_X(\setsystem A)\}\]
gilt $\varphi_X(H(\setsystem B))$ für jedes $\setsystem B\subseteq\powerset(X)$,
und $H$ ist ein Hüllenoperator.
\end{Satz}
\begin{Beweis}
Man erhält $\varphi_X(H(\setsystem B))$ kurzum als Spezialisierung
aus Satz \ref{sigma-algebras-intersection}.

Man kann $H(\setsystem B)$ als $H(\setsystem B) =
\bigcap\{\setsystem A\in S\mid\setsystem B\subseteq\setsystem A\}$
mit $S := \{\setsystem A\mid\varphi_X(\setsystem A)\}$
abstrakt fassen. Wegen $\varphi_X(\setsystem P(X))$ gilt
$\setsystem P(X)\in S$. Es sei $T\subseteq S$ mit $T\ne\emptyset$.
Vermittels Satz \ref{sigma-algebras-intersection} ergibt sich
$\bigcap T\in S$. Also ist $S$ ein Hüllensystem auf $\powerset(X)$,
und $H$ damit ein Hüllenoperator.\,\qedsymbol
\end{Beweis}
