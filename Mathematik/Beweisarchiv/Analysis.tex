
\chapter{Analysis}
\section{Folgen}
\subsection{Konvergenz}

\begin{Definition}[open-ep-ball: offene Epsilon-Umgebung]%
\index{Epsilon-Umgebung}\index{offene Epsilon-Umgebung}
Sei $(M,d)$ ein metrischer Raum. Unter der offenen Epsilon-Umgebung
von $a\in M$ versteht man:%
\[U_\varepsilon(a) := \{x\mid d(x,a)<\varepsilon\}\]
Setze zunächst speziell $d(x,a):=|x-a|$ bzw. $d(x,a):=\|x-a\|$.
\end{Definition}

\begin{Definition}[lim: konvergente Folge, Grenzwert]%
\label{def:lim}\index{konvergente Folge}\index{Grenzwert}
\[\lim_{n\to\infty} a_n = a
\defiff \forall\varepsilon{>}0\;\exists n_0\;\forall n{\ge}n_0\;(a_n\in U_\varepsilon(a))\]
bzw.
\[\lim_{n\to\infty} a_n = a
\defiff \forall\varepsilon{>}0\;\exists n_0\;\forall n{\ge}n_0\;(\|a_n-a\|<\varepsilon).\]
\end{Definition}

\begin{Definition}[bseq: beschränkte Folge]%
\label{def:bseq}\index{beschreankte Folge@beschränkte Folge}
Eine Folge $(a_n)$ mit $a_n\in\R$ heißt genau dann beschränkt,
wenn es eine reelle Zahl $S$ gibt mit $|a_n|<S$ für alle $n$.

Eine Folge $(a_n)$ von Punkten eines normierten Raums heißt genau
dann beschränkt, wenn es eine reelle Zahl $S$ gibt mit $\|a_n\|<S$
für alle $n$.
\end{Definition}

\begin{Satz}[Grenzwert bei Konvergenz eindeutig bestimmt]\mbox{}\\
Eine konvergente Folge von Elementen eines metrischen Raumes
besitzt genau einen Grenzwert.
\end{Satz}

\begin{Beweis}
Sei $(a_n)$ eine konvergente Folge mit $a_n\to g_1$. Sei weiterhin
$g_1\ne g_2$. Es wird nun gezeigt, dass $g_2$ kein Grenzwert von $a_n$
sein kann. Wir müssen also zeigen:
\[\neg\lim_{n\to\infty} a_n=g_2 \iff
\exists\varepsilon{>}0\;\forall n_0\;\exists n{\ge}n_0\;
(a_n\notin U_\varepsilon(g_2))\]
mit $a_n\notin U_\varepsilon(g_2)\iff d(a_n,g_2)\ge\varepsilon$.

Um dem Existenzquantor zu genügen, wählt man nun
$\varepsilon = \frac{1}{2}d(g_1,g_2)$.
Nach Def. \ref{metric-space} (metric-space) gilt 
$d(g_1,g_2)>0$, daher ist auch $\varepsilon>0$. Nach Satz
\ref{construction-disjoint-ep-balls} sind die Umgebungen
$U_\varepsilon(g_1)$ und $U_\varepsilon(g_2)$ disjunkt.
Wegen $a_n\to g_1$ gibt es ein $n_0$ mit $a_n\in U_\varepsilon(g_1)$ für alle
$n\ge n_0$. Dann gibt es für jedes beliebig große $n_0$ aber auch
$n\ge n_0$ mit $a_n\notin U_\varepsilon(g_2)$.\,\qedsymbol
\end{Beweis}

\begin{Satz}[lim-scaled-ep: skaliertes Epsilon]\label{lim-scaled-ep}
Es gilt:
\[\lim_{n\to\infty} a_n=a \iff
\forall\varepsilon{>}0\;\exists n_0\;\forall n{\ge}n_0\;(\|a_n-a\|<R\varepsilon),\]
wobei $R>0$ ein fester aber beliebieger Skalierungsfaktor ist.
\end{Satz}

\begin{Beweis}
Betrachte $\varepsilon>0$ und multipliziere auf beiden Seiten
mit $R$. Dabei handelt es sich um eine Äquivalenzumformung.
Setze $\varepsilon':=R\varepsilon$. Demnach gilt:
\[\varepsilon>0 \iff \varepsilon'>0.\]
Nach der Ersetzungsregel düfen wir die Teilformel $\varepsilon>0$
nun ersetzen. Es ergibt sich die äquivalente Formel
\[\lim_{n\to\infty} a_n=a \iff
\forall\varepsilon'{>}0\;\exists n_0\;\forall n{\ge}n_0\;
(\|a_n-a\|<\varepsilon').\]
Das ist aber genau Def. \ref{def:lim} (lim).\,\qedsymbol
\end{Beweis}

\begin{Satz}
Es gilt:
\[\lim_{n\to\infty} a_n = a\implies \lim_{n\to\infty} \|a_n\| = \|a\|.\]
\end{Satz}

\begin{Beweis}
Nach Satz \ref{rev-tineq} (umgekehrte Dreiecksungleichung) gilt:
\[|\|a_n\|-\|a\|| \le \|a_n-a\| < \varepsilon.\]
Dann ist aber rest recht $|\|a_n\|-\|a\||<\varepsilon$.\,\qedsymbol
\end{Beweis}

\begin{Satz}\label{zero-seq-bounded}
Ist $(a_n)$ eine Nullfolge und $(b_n)$ eine beschränkte Folge,
dann ist auch $(a_n b_n)$ eine Nullfolge.
\end{Satz}

\begin{Beweis}
Wenn $(b_n)$ beschränkt ist, dann existiert nach
Def. \ref{def:bseq} (bseq) eine Schranke $S$ mit
$|b_n|<S$ für alle $n$. Man multipliziert nun auf beiden Seiten
mit $|a_n|$ und erhält
\[|a_n b_n| = |a_n| |b_n| < |a_n| S.\]
Wenn $a_n\to 0$, dann muss für jedes $\varepsilon$
ein $n_0$ existieren mit $|a_n|<\varepsilon$ für $n\ge n_0$.
Multipliziert man auf beiden Seiten mit $S$, und ergibt sich
\[|a_n b_n-0| = |a_n b_n| < |a_n| S < S\varepsilon.\]
Nach Satz \ref{lim-scaled-ep} (lim-scaled-ep) gilt dann
aber $a_n b_n\to 0$.\,\qedsymbol
\end{Beweis}

\begin{Satz}
Sind $(a_n)$ und $(b_n)$ Nullfolgen,
dann ist auch $(a_n b_n)$ eine Nullfolge.
\end{Satz}

\begin{Beweis}[Beweis 1]
Wenn $(b_n)$ eine Nullfolge ist, dann ist $(b_n)$ auch beschränkt.
Nach Satz \ref{zero-seq-bounded} gilt dann die Behauptung.
\end{Beweis}

\begin{Beweis}[Beweis 2]
Sei $\varepsilon>0$ beliebig.
Es gibt ein $n_0$, so dass
$|a_n|<\varepsilon$ und $|b_n|<\varepsilon$ für $n\ge n_0$.
Demnach ist
\[|a_n b_n| = |a_n| |b_n|< |a_n|\varepsilon <\varepsilon^2.\]
Wegen $\varepsilon>0\iff\varepsilon'>0$ mit
$\varepsilon'=\varepsilon^2$ gilt
\[\forall\varepsilon'{>}0\;\exists n_0\;\forall n{\ge}n_0\;
(|a_n b_n|<\varepsilon').\]
Nach Def. \ref{def:lim} (lim) gilt somit die Behauptung.\,\qedsymbol
\end{Beweis}

\begin{Satz}[Grenzwertsatz zur Addition]%
\label{lim-add}\index{Grenzwertsaetze@Grenzwertsätze}
Seien $(a_n)$, $(b_n)$ Folgen von Vektoren eines normierten Raumes.
Es gilt:
\[\lim_{n\to\infty} a_n = a\land \lim_{n\to\infty} b_n
= b \implies \lim_{n\to\infty} a_n+b_n = a+b.\]
\end{Satz}

\begin{Beweis}
Dann gibt es ein $n_0$, so dass für $n\ge n_0$ sowohl
$\|a_n-a\|<\varepsilon$ als auch $\|b_n-b\|<\varepsilon$.
Addition der beiden Ungleichungen ergibt
\[\|a_n-a\| + \|b_n-b\| < 2\varepsilon.\]
Nach der Dreiecksungleichung, das ist Axiom (N3) in Def.
\ref{def:normed-space} (normed-space), gilt nun aber die Abschätzung
\[\|(a_n+b_n)-(a+b)\| = \|(a_n-a)+(b_n-b)\| \le \|a_n-a\|+\|b_n-b\|.\]
Somit gilt erst recht
\[\|(a_n+b_n)-(a+b)\| < 2\varepsilon.\]
Nach Satz \ref{lim-scaled-ep} (lim-scaled-ep)
folgt die Behauptung.\,\qedsymbol
\end{Beweis}

\begin{Satz}[Grenzwertsatz zur Skalarmultiplikation]\label{lim-smult}
Sei $(a_n)$ eine Folge von Vektoren eines normierten Raumes
und sei $r\in\R$ oder $r\in\C$. Es gilt:
\[\lim_{n\to\infty} a_n = a\implies \lim_{n\to\infty} ra_n\to ra.\]
\end{Satz}

\begin{Beweis}
Sei $\varepsilon>0$ fest aber beliebig. Es gibt nun ein $n_0$, so
dass $\|a_n-a\|<\varepsilon$ für $n\ge n_0$.
Multipliziert man auf beiden Seiten
mit $|r|$ und zieht Def. \ref{def:normed-space} (normed-space)
Axiom (N2) heran, dann ergibt sich
\[\|ra_n-ra\| = |r|\,\|a_n-a\|<|r|\varepsilon.\]
Nach Satz \ref{lim-scaled-ep} (lim-scaled-ep)
folgt die Behauptung.\,\qedsymbol
\end{Beweis}

\begin{Satz}[Grenzwertsatz zum Produkt]\mbox{}\\
Seien $(a_n)$ und $(b_n)$ Folgen
reeller Zahlen. Es gilt:
\[\lim_{n\to\infty} a_n=a\land\lim_{n\to\infty} b_n=b\implies
\lim_{n\to\infty} a_n b_n = ab.\]
\end{Satz}

\begin{Beweis}
Nach Voraussetzung sind $a_n-a$ und $b_n-b$ Nullfolgen.
Da das Produkt von Nullfolgen wieder eine Nullfolge ist, gilt
\[(a_n-a)(b_n-b) = a_n b_n-a_n b-ab_n+ab\to 0.\]
Da nach Satz \ref{lim-smult} aber $a_n b\to ab$ und $ab_n\to ab$,
ergibt sich nach Satz \ref{lim-add} nun
\[(a_n-a)(b_n-b)+a_n b+ab_n = a_n b_n+ab\to 2ab.\]
Addiert man nun noch die konstante Folge $-2ab$
und wendet nochmals Satz \ref{lim-add} an, dann ergibt sich
die Behauptung
\[a_n b_n\to ab.\,\qedsymbol\]
\end{Beweis}

\newpage
\begin{Satz}\label{cont-seqcont}%
\index{folgenstetig}\index{stetig!folgenstetig}
Sei $M$ ein metrischer Raum und $X$ ein topologischer Raum.
Eine Abbildung $f\colon M\to X$ ist genau dann stetig, wenn
sie folgenstetig ist.
\end{Satz}

\begin{Satz}[Satz zur Fixpunktgleichung]\index{Fixpunktgleichung}
Sei $M$ ein metrischer Raum und sei $f\colon M\to M$.
Sei $x_{n+1}:=f(x_n)$ eine Fixpunktiteration. Wenn die Folge
$(x_n)$ zu einem Startwert $x_0$ konvergiert mit $x_n\to x$, und
wenn $f$ eine stetige Abbildung ist, dann muss der Grenzwert $x$ die
Fixpunktgleichung $x=f(x)$ erfüllen.
\end{Satz}

\begin{Beweis}
Wenn $x_n\to x$, dann gilt trivialerweise auch $x_{n+1}\to x$.
Weil $f$ stetig ist, ist $f$ nach Satz \ref{cont-seqcont}
auch folgenstetig. Daher gilt $\lim f(a_n) = f(\lim a_n)$ für jede
konvergente Folge $(a_n)$. Somit gilt:
\[x=\lim_{n\to\infty} x_{n+1} = \lim_{n\to\infty} f(x_n)
= f(\lim_{n\to\infty} x_n) = f(x).\;\qedsymbol\]
\end{Beweis}

\section{Differentialrechnung}
\begin{Satz}
Sei $f\colon\R\to\R$ eine Funktion mit der Eigenschaft
$f(x)=0$ für $x\le 0$ und $f(x)>0$ für $x>0$. Es gibt glatte Funktionen
mit dieser Eigenschaft, jedoch keine analytische.
\end{Satz}

\begin{Beweis}
Wegen $f(x)=0$ für $x\le 0$ muss die linksseitige $n$-te Ableitung
an der Stelle $x=0$ immer verschwinden. Wenn die $n$-te Ableitung
stetig sein soll, muss auch die rechtsseitige Ableitung bei $x=0$
verschwinden. Da die Funktion glatt sein soll, muss das für jede
Ableitung gelten. Daher verschwindet die Taylorreihe an der Stelle
$x=0$. Da aber $f(x)>0$ für $x>0$, gibt es keine noch so kleine
Umgebung mit Übereinstimmung von $f$ und ihrer Taylorreihe.
Daher kann $f$ an der Stelle $x=0$ nicht analytisch sein.

Eine glatte Funktion lässt sich jedoch konstruieren:
\[f(x):=\begin{cases}
\ee^{-1/x}&\text{wenn}\;x>0,\\
0&\text{wenn}\;x\le 0.
\end{cases}\]
Ist nämlich $g(x)$ an einer Stelle glatt, dann ist
es nach Kettenregel, Produktregel und Summenregel auch $\ee^{g(x)}$.
Die $n$-te Ableitung lässt sich immer in der Form
\[\sum\nolimits_k e^{g(x)}{r_k(x)}
= e^{g(x)}\sum\nolimits_k r_k(x) = e^{g(x)}r(x)\]
darstellen, wobei die $r_k(x)$ bzw. $r(x)$ in diesem Fall rationale
Funktionen mit Polstelle bei $x=0$ sind. Da aber $e^{-1/x}$ für
$x\to 0$ schneller fällt als jede rationale Funktion steigen kann,
muss die rechtsseitige Ableitung an der Stelle $x=0$ immer
verschwinden.\;\qedsymbol
\end{Beweis}
