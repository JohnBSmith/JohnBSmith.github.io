\documentclass[a4paper,10pt,fleqn,twoside]{scrbook}
\usepackage[utf8]{inputenc}
\usepackage[T1]{fontenc}
\usepackage{arev}

\usepackage{ngerman}
\usepackage{amsmath}
\usepackage{amssymb}
\usepackage{amsthm}
\usepackage{thmtools}

\usepackage[all]{xy}
\usepackage{enumitem}

\usepackage{color}
\definecolor{c1}{RGB}{0,40,80}
\definecolor{gray1}{RGB}{80,80,80}
\usepackage[colorlinks=true,linkcolor=c1]{hyperref}
\usepackage{geometry}
\geometry{a4paper,left=38mm,right=22mm,top=24mm,bottom=40mm}
\setlength{\columnsep}{4mm}
\numberwithin{equation}{section}
\setcounter{tocdepth}{2}

\newcommand{\N}{\mathbb N}
\newcommand{\Z}{\mathbb Z}
\newcommand{\R}{\mathbb R}
\newcommand{\C}{\mathbb C}
\newcommand{\id}{\operatorname{id}}
\renewcommand{\Re}{\operatorname{Re}}
\renewcommand{\Im}{\operatorname{Im}}
\newcommand{\ui}{\mathrm i}
\newcommand{\ee}{\mathrm e}
\newcommand{\sur}{\operatorname{sur}}
\newcommand{\defiff}{\;:\Longleftrightarrow\;}
\newcommand{\strong}[1]{{\sf\bfseries #1}}

\newcommand{\emdef}[1]{\emph{#1}}
\newcommand{\pstrut}[1]{\rule{0pt}{\dimexpr 8pt+#1}}
\newcommand{\bitem}{\item[\scriptsize\color{gray1}$\blacksquare$]}
\renewcommand{\qedsymbol}{\ensuremath{\square}}

\declaretheoremstyle[%
  within=chapter,%
  spaceabove=\topsep,%
  spacebelow=\topsep%,
  headfont=\bfseries,%
  headpunct={.},%
  % postheadspace=,%
  notefont=\bfseries,%
  notebraces={(}{)},%
]{baplain}

\declaretheorem[style=baplain,name=Definition]{Definition}
\declaretheorem[style=baplain,name=Satz]{Satz}
\newenvironment{Beweis}{\par\noindent\strong{Beweis.}}{\par}

\title{Beweisarchiv}
\date{Mai 2018}

\begin{document}

\maketitle
\tableofcontents

\chapter{Grundlagen}
\section{Mengenlehre}
\subsection{Definitionen}

\begin{Definition}[seteq: Gleichheit von Mengen]\label{def:seteq}
\[A=B \defiff \forall x(x\in A\iff x\in B).\]
\end{Definition}

\begin{Definition}[subseteq: Teilmenge]\label{def:subseteq}
\[A\subseteq B \defiff \forall x(x\in A\implies x\in B).\]
\end{Definition}

\begin{Definition}[filter: beschreibende Angabe]\label{def:filter}
\[a\in\{x\mid P(x)\} \defiff P(a).\]
\end{Definition}

\begin{Definition}[cap: Schnitt]\label{def:cap}
\[A\cap B = \{x\mid x\in A\land x\in B\}.\]
\end{Definition}

\begin{Definition}[cup: Vereinigung]\label{def:cup}
\[A\cup B = \{x\mid x\in A\lor x\in B\}.\]
\end{Definition}

\begin{Definition}[intersection: Schnitt]\label{def:intersection}
\[\bigcap_{i\in I} A_i \iff \{x\mid \forall i{\in}I\,(x\in A_i)\}.\]
\end{Definition}

\begin{Definition}[union: Vereinigung]\label{def:union}
\[\bigcup_{i\in I} A_i \iff \{x\mid \exists i{\in}I\,(x\in A_i)\}.\]
\end{Definition}


\subsection{Rechenregeln}

\begin{Satz}[Kommutativgesetze]
Es gilt $A\cap B = B\cap A$ und $A\cup B = B\cup A$.
\end{Satz}

\begin{Beweis}
Nach Def. \ref{def:seteq} (seteq) expandieren:
\[\forall x(x\in A\cap B \iff x\in B\cap A).\]
Nach Def. \ref{def:cap} (cap) und Def. \ref{def:filter} (filter) gilt:
\[x\in A\cap B \iff x\in A\land x\in B \iff x\in B\land x\in A
\iff x\in B\cap A.\]
Für die Vereinigung ist das analog.\,\qedsymbol
\end{Beweis}

\begin{Satz}[Assoziativgesetze]
Es gilt $A\cap (B\cap C) = (A\cap B)\cap C$
und $A\cup (B\cup C) = (A\cup B)\cup C$.
\end{Satz}

\begin{Beweis}
Nach Def. \ref{def:seteq} (seteq) expandieren:
\[\forall x[x\in A\cap (B\cap C) \iff x\in (A\cap B)\cap C].\]
Nach Def. \ref{def:cap} (cap) und Def. \ref{def:filter} (filter) gilt:
\begin{align*}
&x\in A\cap (B\cap C) \iff x\in A\land x\in B\cap C
\iff x\in A\land (x\in B\land x\in C)\\
&\iff (x\in A\land x\in B)\land x\in C
\iff x\in A\cap B\land x\in C
\iff x\in (A\cap B)\cap C.
\end{align*}
Für die Vereinigung ist das analog.\,\qedsymbol
\end{Beweis}



\chapter{Analysis}
\section{Folgen}
\subsection{Konvergenz}

\begin{Definition}[openepball: offene Epsilon-Umgebung]
Sei $(M,d)$ ein metrischer Raum. Unter der offenen Epsilon-Umgebung
von $a\in M$ versteht man:
\[U_\varepsilon(a) := \{x\mid d(x,a)<\varepsilon\}\]
Setze zunächst speziell $d(x,a):=|x-a|$ bzw. $d(x,a):=\|x-a\|$.
\end{Definition}

\begin{Definition}[lim: konvergente Folge, Grenzwert]\label{def:lim}
\[\lim_{n\to\infty} a_n = a
\defiff \forall\varepsilon{>}0\;\exists n_0\;\forall n{>}n_0\;(a_n\in U_\varepsilon(a))\]
bzw.
\[\lim_{n\to\infty} a_n = a
\defiff \forall\varepsilon{>}0\;\exists n_0\;\forall n{>}n_0\;(|a_n-a|<\varepsilon).\]
\end{Definition}

\begin{Satz}
Es gilt:
\[\lim_{n\to\infty} a_n=a \iff
\forall\varepsilon{>}0\;\exists n_0\;\forall n{>}n_0\;(|a_n-a|<R\varepsilon),\]
wobei $R>0$ ein fester aber beliebieger Skalierungsfaktor ist.
\end{Satz}

\begin{Beweis}
Betrachte $\varepsilon>0$ und multipliziere auf beiden Seiten
mit $R$. Dabei handelt es sich um eine Äquivalenzumformung.
Setze $\varepsilon':=R\varepsilon$. Demnach gilt:
\[\varepsilon>0 \iff \varepsilon'>0.\]
Nach der Ersetzungsregel düfen wir die Teilformel $\varepsilon>0$
nun ersetzen. Es ergibt sich die äquivalente Formel
\[\lim_{n\to\infty} a_n=a \iff
\forall\varepsilon'{>}0\;\exists n_0\;\forall n{>}n_0\;(|a_n-a|<\varepsilon').\]
Das ist aber genau Def. \ref{def:lim} (lim).\,\qedsymbol
\end{Beweis}

\chapter{Topologie}
\section{Grundbegriffe}
\subsection{Definitionen}

\begin{Definition}[nhfilter: Umgebungsfilter]\label{def:nhfilter}
\[\underline U(x) := \{U{\subseteq}X\mid
\exists O(O\in T\land x\in O\land O\subseteq U)\}.\]
\end{Definition}

\begin{Definition}[int: Offener Kern]\label{def:int}
\[\operatorname{int}(M) := \{x\in M\mid M\in \underline U(x)\}\]
\end{Definition}

\begin{Satz}
Der offene Kern von $M$ ist die Vereinigung der offenen Teilmengen
von $M$. Kurz:%
\[\operatorname{int}(M) = \bigcup_{O\in 2^M\cap T} O.\]
\end{Satz}

\begin{Beweis}
Nach Def. \ref{def:seteq} (seteq) und Def. \ref{def:int} (int)
expandieren:
\[\forall x[x\in M\land M\in\underline U(x)
\iff x\in\bigcup_{O\in 2^M\cap T} O].\]
Den äußeren Allquantor brauchen wir nicht weiter mitschreiben, da alle
freien Variablen automatisch allquantifiziert werden.
Nach Def. \ref{def:nhfilter} (nhfilter) weiter expandieren, wobei die
Bedingung $U\subseteq X$ als tautologisch entfallen kann,
weil $X$ die Grundmenge ist. Auf der rechten Seite wird nach Def.
\ref{def:union} (union) expandiert. Es ergibt sich:
\[x\in M\land \exists O(O\in T\land x\in O\land O\subseteq M)
\iff \exists O(O\subseteq M\land O\in T\land x\in O).\]
Wegen $A\land\exists x(P(x))\iff \exists x(A\land P(x))$ ergibt
sich auf der linken Seite:
\[\exists O(x\in M\land O\in T\land x\in O\land O\subseteq M).\]
Wenn aber $O\subseteq M$ erfüllt sein muss, gilt
$x\in O\implies x\in M$. Demnach kann $x\in M$ entfallen.
Auf beiden Seiten steht dann die gleiche Bedingung.\,\qedsymbol
\end{Beweis}



\vfill
\texttt{Dieses Heft steht unter der Creative"=Commons"=Lizenz CC0.}
\end{document}



