\documentclass[a4paper,10pt,fleqn,twoside]{scrbook}
\usepackage[utf8]{inputenc}
\usepackage[T1]{fontenc}
\usepackage[ngerman]{babel}
\usepackage{microtype}

\usepackage{arev}
% Patch punctuation to be upright
\DeclareMathSymbol{.}{\mathpunct}{operators}{`.}
\DeclareMathSymbol{,}{\mathpunct}{operators}{`,}

\usepackage{amsmath}
\usepackage{amssymb}
\usepackage{amsthm}
\usepackage{mdframed}
\usepackage{mathtools}
\usepackage{booktabs}
% \usepackage{thmtools}

\usepackage[all]{xy}
\usepackage{enumitem}
\usepackage{graphicx}
\usepackage{proof}

\usepackage{color}
\definecolor{c1}{RGB}{0,40,80}
\definecolor{gray1}{RGB}{80,80,80}
\usepackage[colorlinks=true,linkcolor=c1]{hyperref}

\usepackage{geometry}
% \geometry{a4paper,left=38mm,right=22mm,top=24mm,bottom=40mm}
\geometry{a4paper,left=36mm,right=24mm,top=28mm,bottom=34mm}

\numberwithin{equation}{chapter}
\setcounter{tocdepth}{2}

\newcommand{\N}{\mathbb N}
\newcommand{\Z}{\mathbb Z}
\newcommand{\Q}{\mathbb Q}
\newcommand{\R}{\mathbb R}
\newcommand{\C}{\mathbb C}
\newcommand{\A}{\mathbb A}
\newcommand{\id}{\operatorname{id}}
\renewcommand{\Re}{\operatorname{Re}}
\renewcommand{\Im}{\operatorname{Im}}
\newcommand{\ui}{\mathrm i}
\newcommand{\ee}{\mathrm e}
\newcommand{\sur}{\operatorname{sur}}
\newcommand{\Abb}{\operatorname{Abb}}
\newcommand{\sgn}{\operatorname{sgn}}
\newcommand{\defiff}{\;:\Longleftrightarrow\;}
\newcommand{\strong}[1]{{\normalfont\sffamily\bfseries #1}}
\newcommand{\comp}{\mathrm c}
\newcommand{\bv}[1]{\mathbf{#1}}

\newcommand{\emdef}[1]{\emph{#1}}
\newcommand{\pstrut}[1]{\rule{0pt}{\dimexpr 8pt+#1}}
\newcommand{\bitem}{\item[\scriptsize\color{gray1}$\blacksquare$]}
\renewcommand{\qedsymbol}{\ensuremath{\square}}
\newcommand{\match}{\text{\strong{match}}\;}
\newcommand{\inl}{\operatorname{inl}}
\newcommand{\inr}{\operatorname{inr}}
\newcommand{\infernote}[1]{\!\!\text{\scriptsize #1}}
\newcommand{\stackrelrm}[2]{\stackrel{\mathrm{#1}}{#2}}
\newcommand{\newlinefirst}{\mbox{}\\*}

\newtheoremstyle{rmbox}%
  {0pt}% space above
  {0pt}% space below
  {}% bodyfont
  {}% indent
  {\bfseries}% head font
  {\;}% punctuation between head and body
  {0pt}% space after theorem head
  {\thmname{#1}\;\thmnumber{#2}\thmnote{\;(#3)}.}

\theoremstyle{rmbox}
\newtheorem{Definition}{Definition}[chapter]
\newtheorem{Satz}{Satz}[chapter]
\newtheorem{Axiom}[Satz]{Axiom}
\newtheorem{Lemma}[Satz]{Lemma}

\definecolor{greenblue}{rgb}{0.0,0.32,0.4}
\definecolor{grayblue}{rgb}{0.2,0.2,0.4}

\surroundwithmdframed[topline=false,rightline=false,bottomline=false,%
  linecolor=greenblue, linewidth=4pt, innerleftmargin=6pt,%
  innertopmargin=2pt, innerbottommargin=4pt,%
  innerrightmargin=0pt%
]{Definition}

\newcommand{\framedtheorem}[1]{%
\surroundwithmdframed[topline=false,rightline=false,bottomline=false,%
  linecolor=grayblue, linewidth=4pt, innerleftmargin=6pt,%
  innertopmargin=2pt, innerbottommargin=4pt,%
  innerrightmargin=0pt%
]{#1}}

\framedtheorem{Axiom}
\framedtheorem{Satz}
\framedtheorem{Lemma}

\newenvironment{Beweis}[1][Beweis]%
  {\par\noindent\strong{#1.}\;\ignorespaces}%
  {\par\addvspace{\topsep}}

\usepackage{makeidx}
\makeindex

\title{Beweisarchiv}
\date{November 2022}

\begin{document}
\setlength{\baselineskip}{13pt}

\maketitle

% \vfill
\texttt{Dieses Buch steht unter der Lizenz Creative Commons CC0.}

\tableofcontents


\chapter{Grundlagen}
\section{Arithmetik}
\subsection{Zahlenbereiche}
Natürliche Zahlen ab null:
\begin{equation}
\N_0 := \{0,1,2,3,4,\ldots\}.
\end{equation}
Natürliche Zahlen ab eins:
\begin{equation}
\N_1 := \{1,2,3,4,5,\ldots\}.
\end{equation}
Natürliche Zahlen:
\begin{equation}
\begin{split}
&\text{$\N$, wenn es keine Rolle spielt,}\\
&\text{ob $\N:=\N_0$ oder $\N:=\N_1$}.
\end{split}
\end{equation}
Ganze Zahlen:
\begin{equation}
\Z := \{\ldots -3,-2,-1,0,1,2,3,\ldots\}.
\end{equation}
Rationale Zahlen:
\begin{equation}
\Q := \{\tfrac{z}{n}\mid z\in\Z,n\in\N_0\}.
\end{equation}
Reelle Zahlen:
\begin{equation}
\R := \overline{\Q}\enspace\text{bezüglich}\; d(x,y)=|x-y|.
\end{equation}
Positive reelle Zahlen:
\begin{equation}
\R^+ := \{x\in\R\mid x>0\}.
\end{equation}
Nichtnegative reelle Zahlen:
\begin{equation}
\R_0^+ := \{x\in\R\mid x\ge 0\}.
\end{equation}
Negative reelle Zahlen:
\begin{equation}
\R^- := \{x\in\R\mid x<0\}.
\end{equation}
Nichtpositive reelle Zahlen:
\begin{equation}
\R_0^- := \{x\in\R\mid x\le 0\}.
\end{equation}
Komplexe Zahlen:
\begin{equation}
\C := \{a+b\ui\mid a,b\in\R\}.
\end{equation}
Quaternionen:
\begin{equation}
\mathbb H := \{a+b\ui+c\uj+d\uk\mid a,b,c,d\in\R\}.
\end{equation}
Algebraische Zahlen:
\begin{equation}
\mathbb A := \{a\in\C\mid \exists P{\in}\Q[X]\colon P(a)=0\}.
\end{equation}
Irrationale Zahlen:
\begin{equation}
\R\setminus\Q = \{\sqrt{2},\sqrt{3},\pi,\ee,\ldots\}.
\end{equation}
Transzendente Zahlen:
\begin{equation}
\R\setminus\mathbb A = \{\pi,\ee,\ldots\}.
\end{equation}
Es gelten die folgenden Teilmengenbeziehungen:
\begin{equation}
\N\subset\Z\subset\Q\subset\R\subset\C\subset\mathbb H.
\end{equation}
Es gilt die folgende Abstufung der Mächtigkeit:
\begin{equation}
|\N| = |\Z| = |\Q| = |\mathbb A| < |\R| = |\C|.
\end{equation}

\newpage
\subsection{Intervalle}
Abgeschlossene Intervalle:
\begin{equation}
[a,b] := \{x\in\R\mid a\le x\le b\}.
\end{equation}
Offene Intervalle:
\begin{equation}
(a,b) := \{x\in\R\mid a<x<b\}.
\end{equation}
Halboffene Intervalle:
\begin{align}
(a,b] &:= \{x\in\R\mid a<x\le b\},\\
[a,b) &:= \{x\in\R\mid a\le x<b\}.
\end{align}
Unbeschränkte Intervalle:
\begin{align}
[a,\infty) &:= \{x\in\R\mid a\le x\},\\
(a,\infty) &:= \{x\in\R\mid a<x\},\\
(-\infty,b] &:= \{x\in\R\mid x\le b\},\\
(-\infty,b) &:= \{x\in\R\mid x<b\}.
\end{align}

\subsection{Summen}
\begin{Definition} Für eine Folge $(a_n)$:
\begin{align}
\sum_{k=m}^{m-1} a_k &:= 0,\qquad(\text{leere Summe})\\
\sum_{k=m}^n a_k &:= \sum_{k=m}^{n-1} a_k.\qquad(n\ge m)
\end{align}
\end{Definition}
\noindent
Für eine Konstante $c$ gilt:
\begin{equation}
\sum_{k=m}^n c = (n-m+1)\,c.
\end{equation}
Der Summierungsoperator ist linear:
\begin{align}
\sum_{k=m}^n (a_k+b_k) &= \sum_{k=m}^n a_k + \sum_{k=m}^n b_k,\\
\sum_{k=m}^n ca_k &= c\sum_{k=m}^n a_k.
\end{align}
Indexverschiebung ist möglich:
\begin{equation}
\sum_{k=m}^n a_k = \sum_{k=m-j}^{n-j} a_{k+j} = \sum_{k=m+j}^{n+j} a_{k-j}.
\end{equation}
Aufspaltung ist möglich:
\begin{equation}
\sum_{k=m}^n a_k = \sum_{k=m}^p a_k + \sum_{k=p+1}^n a_k.
\end{equation}
Vertauschung der Reihenfolge bei Doppelsummen:
\begin{equation}
\sum_{i=p}^m \sum_{j=q}^n a_{ij} = \sum_{j=q}^n \sum_{i=p}^m a_{ij}.
\end{equation}

\subsection{Binomischer Lehrsatz}\index{binomischer Lehrsatz}
Sei $R$ ein unitärer Ring. 
Für $a,b\in R$ mit $ab=ba$ gilt:%
\begin{equation}
(a+b)^n = \sum_{k=0}^n \binom{n}{k} a^{n-k} b^k
\end{equation}
und
\begin{equation}
(a-b)^n = \sum_{k=0}^n \binom{n}{k} (-1)^k a^{n-k} b^k.
\end{equation}
Die ersten Formeln sind:\index{binomische Formeln}
\begin{gather}
(a+b)^2 = a^2+2ab+b^2,\\
(a-b)^2 = a^2-2ab+b^2,\\
(a+b)^3 = a^3+3a^2 b+3ab^2+b^3,\\
(a-b)^3 = a^3-3a^2 b+3ab^2-b^3,\\
(a+b)^4 = a^4+4a^3 b+6a^2 b^2+4ab^3+b^4,\\
(a-b)^4 = a^4-4a^3 b+6a^2 b^2-4ab^3+b^4.
\end{gather}
\subsection{Potenzgesetze}
\begin{Definition}
Für $a\in\R, a>0$ und $x\in\C$:
\begin{equation}
a^x := \exp(\ln(a)\,x).
\end{equation}
\end{Definition}
\noindent
Für $a\in\R, a>0$ und $x,y\in\C$ gilt:
\begin{gather}
a^{x+y} = a^x a^y,\quad a^{x-y} = \frac{a^x}{a^y},
\quad a^{-x} = \frac{1}{a^x}.
\end{gather}

\section{Komplexe Zahlen}\index{komplexe Zahl}
\subsection{Rechenoperationen}

\begin{gather}
\frac{z_1}{z_2}
= \frac{z_1\overline z_2}{z_2\overline z_2}
= \frac{z_1\overline z_2}{|z_2|^2},\\
\frac{1}{z} = \frac{\overline z}{z\overline z}
= \frac{\overline z}{|z|^2}.
\end{gather}

\subsection{Betrag}\index{Betrag!einer komplexen Zahl}
Für alle $z_1,z_2\in\C$ gilt:
\begin{gather}
|z_1z_2| = |z_1|\,|z_2|,\\
z_2\ne 0\implies \Big|\frac{z_1}{z_2}\Big|
= \frac{|z_1|}{|z_2|},\\
z\,\overline z = |z|^2.
\end{gather}

\subsection{Konjugation}\index{Konjugation!einer komplexen Zahl}
Für alle $z_1,z_2\in\C$ gilt:
\begin{gather}
\overline{z_1+z_2} = \overline z_1+\overline z_2,\qquad
\overline{z_1-z_2} = \overline z_1-\overline z_2,\\
\overline{z_1 z_2} = \overline z_1\,\overline z_2,\qquad
z_2\ne 0 \implies \overline{\Big(\frac{z_1}{z_2}\Big)}
= \frac{\overline z_1}{\overline z_2},\\
\overline{\overline z}=z,\qquad
|\overline{z}| = |z|,\qquad
z\,\overline z = |z|^2,\\
\real(z) = \frac{z+\overline z}{2},\qquad
\imag(z) = \frac{z-\overline z}{2\ui},\\
\overline{\cos(z)} = \cos(\overline z),\qquad
\overline{\sin(z)} = \sin(\overline z),\\
\overline{\exp(z)} = \exp(\overline z).
\end{gather}

\begin{table*}[t]
\caption{Rechenoperationen}
\bgroup
\def\arraystretch{1.4}
\begin{tabular}{|l|r|l|l|}
\hline
  \thbf{Name}
& \thbf{Operation}
& \thbf{Polarform}
& \thbf{kartesische Form}\\
\hline
  Identität
& $z$ & $=r\ee^{\ui\varphi}$
& $= a+b\ui$\\
\hline
  Addition
& $z_1+z_2$ &
& $= (a_1+a_2)+(b_1+b_2)\ui$\\
\hline
  Subtraktion
& $z_1-z_2$ &
& $= (a_1-a_2)+(b_1-b_2)\ui$\\
\hline
  Multiplikation
& $z_1 z_2$
& $= r_1 r_2 \ee^{\ui(\varphi_1+\varphi_2)}$
& $= (a_1 a_2 - b_1 b_2)+(a_1 b_2+a_2 b_1)\ui$\\
\hline
  Division
& $\displaystyle\frac{z_1}{z_2}$
& $\displaystyle =\frac{r_1}{r_2}\ee^{\ui(\varphi_1-\varphi_2)}$
& $\displaystyle =\frac{a_1 a_2 + b_1 b_2}{a_2^2+b_2^2}
   + \frac{a_2 b_1 - a_1 b_2}{a_2^2+b_2^2}\ui$\\
\hline
  Kehrwert
& $\displaystyle\frac{1}{z}$
& $\displaystyle =\frac{1}{r}\ee^{-\ui\varphi}$
& $\displaystyle =\frac{a}{a^2+b^2}-\frac{b}{a^2+b^2}\ui$\\
\hline
  Realteil
& $\real(z)$
& $=\cos\varphi$
& $=a$\\
\hline
  Imaginärteil
& $\imag(z)$
& $=\sin\varphi$
& $=b$\\
\hline
  Konjugation
& $\overline{z}$
& $=r\ee^{-\varphi\ui}$
& $=a-b\ui$\\
\hline
Betrag
& $|z|$
& $=r$
& $=\sqrt{a^2+b^2}$\\
\hline
  Argument
& $\arg(z)$
& $=\varphi$
& $\displaystyle = s(b)\arccos\Big(\frac{a}{r}\Big)$\\
\hline
\end{tabular}
\egroup\\
\\
$s(b):=\begin{cases}
+1 & \text{if}\;b\ge 0,\\
-1 & \text{if}\;b<0
\end{cases}$
\end{table*}

\section{Logik}
\subsection{Aussagenlogik}\index{Aussagenlogik}
\subsubsection{Boolesche Algebra}\index{boolesche Algebra}
\begin{table*}[t]
\caption{Boolesche Algebra}
\begin{tabular}{l|l|l}
\thbf{Disjunktion} & \thbf{Konjunktion} &\\
  $A\lor A \Leftrightarrow A$
& $A\land A \Leftrightarrow A$
& Idempotenzgesetze\\
  $A\lor 0 \Leftrightarrow A$
& $A\land 1 \Leftrightarrow A$
& Neutralitätsgesetze\\
  $A\lor 1 \Leftrightarrow 1$
& $A\land 0 = 0$
& Extremalgesetze\\
  $A\lor \overline A \Leftrightarrow 1$
& $A\land \overline A \Leftrightarrow 0$
& Komplementärgesetze\\
\noalign{\vspace{1em}}
  $A\lor B \Leftrightarrow B\lor A$
& $A\land B \Leftrightarrow B\land A$
& Kommutativgesetze\\
  $(A\lor B)\lor C \Leftrightarrow A\lor (B\lor C)$
& $(A\land B)\land C \Leftrightarrow A\land (B\land C)$
& Assoziativgesetze\\
  $\overline{A\lor B} \Leftrightarrow \overline A\land\overline B$
& $\overline{A\land B} \Leftrightarrow \overline A\lor\overline B$
& De Morgansche Regeln\\
  $A\lor (A\land B) \Leftrightarrow A$
& $A\land (A\lor B) \Leftrightarrow A$
& Absorptionsgesetze\\
\end{tabular}
\end{table*}

\noindent
\strong{Distributivgesetze}:
\begin{gather}
A\lor (B\land C) \iff (A\lor B)\land (A\lor C),\\
A\land (B\lor C) \iff (A\land B)\lor (A\land C).
\end{gather}

\subsubsection{Zweistellige Funktionen}
Es gibt 16 zweistellige boolesche\\
Funktionen.

\begin{tabular}{r|l}
\textbf{\texttt{AB}} & \thbf{Wert}\\
\texttt{00} & \texttt{a}\\
\texttt{01} & \texttt{b}\\
\texttt{10} & \texttt{c}\\
\texttt{11} & \texttt{d}
\end{tabular}

\begin{tabular}{r|l|l|l}
\thbf{Nr.}& \textbf{\texttt{dcba}} & \thbf{Fkt.} & \thbf{Name}\\
 0 & \texttt{0000} & 0 & Kontradiktion\\
 1 & \texttt{0001} & $\overline{A\lor B}$ & NOR\\
 2 & \texttt{0010} & $\overline{B\Rightarrow A}$\\
 3 & \texttt{0011} & $\overline A$\\
 4 & \texttt{0100} & $\overline{A\Rightarrow B}$\\
 5 & \texttt{0101} & $\overline{B}$\\
 6 & \texttt{0110} & $A\oplus B$ & Kontravalenz\index{Kontravalenz}\\
 7 & \texttt{0111} & $\overline{A\land B}$ & NAND\\
 8 & \texttt{1000} & $A\land B$ & Konjunktion\index{Konjunktion}\\
 9 & \texttt{1001} & $A\Leftrightarrow B$ & Äquivalenz\\
10 & \texttt{1010} & $B$ & Projektion\\
11 & \texttt{1011} & $A\Rightarrow B$ & Implikation\\
12 & \texttt{1100} & $A$ & Projektion\\
13 & \texttt{1101} & $B\Rightarrow A$ & Implikation\\
14 & \texttt{1110} & $A\lor B$ & Disjunktion\index{Disjunktion}\\
15 & \texttt{1111} & $1$ & Tautologie
\end{tabular}

\subsubsection{Darstellung mit Negation, Konjunktion und Disjunktion}
\begin{gather}
A\Rightarrow B \iff \overline A\lor B,\\
(A\Leftrightarrow B) \iff
  (\overline A\land\overline B)\lor(A\land B),\\
A\oplus B \iff (\overline A\land B)\lor(A\land\overline B).
\end{gather}

\subsubsection{Tautologien}
Modus ponens:
\begin{equation}
(A\Rightarrow B)\land A\implies B.
\end{equation}
Modus tollens:
\begin{equation}
(A\Rightarrow B)\land\overline B\implies\overline A.
\end{equation}
Modus tollendo ponens:
\begin{equation}
(A\lor B)\land\overline A \implies B.
\end{equation}
Modus ponendo tollens:
\begin{equation}
\overline{A\land B}\land A\implies\overline B.
\end{equation}
Kontraposition:\index{Kontraposition}
\begin{equation}
A\Rightarrow B \iff \overline B\Rightarrow \overline A.
\end{equation}
Beweis durch Widerspruch:\index{Widerspruch}
\begin{equation}
(\overline A\Rightarrow B\land\overline B)\implies A.
\end{equation}
Zerlegung einer Äquivalenz:
\begin{equation}
(A\Leftrightarrow B) \iff (A\Rightarrow B)\land(B\Rightarrow A).
\end{equation}
Kettenschluss:
\begin{equation}
(A\Rightarrow B)\land(B\Rightarrow C)\implies (A\Rightarrow C).
\end{equation}
Ringschluss:
\begin{equation}
\begin{split}
&(A\Rightarrow B)\land (B\Rightarrow C)\land(C\Rightarrow A)\\
&\implies (A\Leftrightarrow B)\land(A\Leftrightarrow C)\land(B\Leftrightarrow C).
\end{split}
\end{equation}
Ringschluss, allgemein:
\begin{equation}
\begin{split}
& (A_1{\Rightarrow }A_2)\land\ldots\land(A_{n-1}{\Rightarrow}A_n)
\land(A_n{\Rightarrow}A_1)\\
& \implies \forall i,j\,[A_i\Leftrightarrow A_j].
\end{split}
\end{equation}
Ersetzungsregel:

Für jede Funktion $P\colon\{0,1\}\to\{0,1\}$ gilt:
\begin{equation}
P(A)\land (A\Leftrightarrow B)\implies P(B).
\end{equation}
Regel zur Implikation:
\begin{equation}
A\land B\Rightarrow C \iff A\Rightarrow (B\Rightarrow C).
\end{equation}
Vollständige Fallunterscheidung:
\begin{gather}
(A\Rightarrow C)\land (B\Rightarrow C)\implies (A\oplus B\Rightarrow C),\\
(A\Rightarrow C)\land (B\Rightarrow C)\iff (A\lor B\Rightarrow C).
\end{gather}
Vollständige Fallunterscheidung, allgemein:
\begin{gather}
\textstyle \forall k[A_k\Rightarrow C]
\implies (\bigoplus_{k=1}^n A_k\Rightarrow C),\\
\forall k[A_k\Rightarrow C]
\iff (\exists k[A_k]\Rightarrow C).
\end{gather}

\newpage
\subsection{Prädikatenlogik}
\subsubsection{Rechenregeln}
Verneinung (De Morgansche Regeln):
\begin{gather}
\overline{\forall x[P(x)]}\iff \exists x[\overline{P(x)}],\\
\overline{\exists x[P(x)]}\iff \forall x[\overline{P(x)}].
\end{gather}
Verallgemeinerte Distributivgesetze:
\begin{gather}
P\lor\forall x[Q(x)] \iff \forall x[P\lor Q(x)],\\
P\land\exists x[Q(x)] \iff \exists x[P\land Q(x)].
\end{gather}
Verallgemeinerte Idempotenzgesetze:
\begin{gather}
\begin{split}
\exists x{\in}M\,[P] & \iff
(M\ne\{\})\land P\\
& \iff\begin{cases}
P & \text{wenn}\; M\ne\{\},\\
0 & \text{wenn}\; M=\{\}.
\end{cases}
\end{split}\\
\begin{split}
\forall x{\in}M\,[P]& \iff
(M=\{\})\lor P\\
&\iff\begin{cases}
P & \text{wenn}\; M\ne\{\},\\
1 & \text{wenn}\; M=\{\}.
\end{cases}
\end{split}
\end{gather}
\newpage\noindent
Äquivalenzen:
\begin{gather}
\hspace{-2em}\forall x\forall y[P(x,y)] \iff \forall y\forall x[P(x,y)],\\
\hspace{-2em}\exists x\exists y[P(x,y)] \iff \exists y\exists x[P(x,y)],\\
\hspace{-2em}\forall x[P(x)\land Q(x)] \iff \forall x[P(x)]\land\forall x[Q(x)],\\
\hspace{-2em}\exists x[P(x)\lor Q(x)] \iff \exists x[P(x)]\lor\exists x[Q(x)],\\
\hspace{-2em}\forall x[P(x)\Rightarrow Q] \iff \exists x[P(x)]\Rightarrow Q,\\
\hspace{-2em}\forall x[P\Rightarrow Q(x)] \iff P\Rightarrow\forall x[Q(x)],\\
\hspace{-2em}\exists x[P(x)\Rightarrow Q(x)]
  \iff\forall x[P(x)]\Rightarrow\exists x[Q(x)].
\end{gather}
% \newpage\noindent
Implikationen:
\begin{gather}
\hspace{-2em}\exists x\forall y[P(x,y)]\implies \forall y\exists x[P(x,y)],\\
\hspace{-2em}\forall x[P(x)]\lor\forall x[Q(x)]\implies\forall x[P(x)\lor Q(x)],\\
\hspace{-2em}\exists x[P(x)\land Q(x)]\implies
  \exists x[P(x)]\land \exists x[Q(x)],\\
\hspace{-2em}\forall x[P(x)\Rightarrow Q(x)]\implies
  (\forall x[P(x)]\Rightarrow\forall x[Q(x)]),\\
\hspace{-2em}\forall x[P(x)\Leftrightarrow Q(x)]\implies
  (\forall x[P(x)]\Leftrightarrow\forall x[Q(x)]).
\end{gather}

\subsubsection{Endliche Mengen}
Sei $M=\{x_1,\ldots,x_n\}$. Es gilt:
\begin{gather}
\forall x{\in}M\,[P(x)]\iff P(x_1)\land\ldots\land P(x_n),\\
\exists x{\in}M\,[P(x)]\iff P(x_1)\lor\ldots\lor P(x_n).
\end{gather}

\subsubsection{Beschränkte Quantifizierung}
\begin{gather}
\begin{split}
& \forall x{\in}M\,[P(x)]\;:\Longleftrightarrow\;\forall x[x\notin M\lor P(x)]\\
& \quad\iff\forall x[x\in M\Rightarrow P(x)],
\end{split}\\
\exists x{\in}M\,[P(x)]\;:\Longleftrightarrow\;\exists x[x\in M\land P(x)],\\
\forall x{\in}M{\setminus}N\,[P(x)]\iff \forall x[x\notin N\Rightarrow P(x)].
\end{gather}

\subsubsection[Quantifizierung über Produktmengen]{\\
Quantifizierung über Produktmengen}
\begin{gather}
\forall(x,y)\,[P(x,y)]\iff \forall x\forall y[P(x,y)],\\
\exists(x,y)\,[P(x,y)]\iff \exists x\exists y[P(x,y)].
\end{gather}
Analog gilt
\begin{gather}
\forall(x,y,z)\,\iff \forall x\forall y\forall z,\\
\exists(x,y,z)\,\iff \exists x\exists y\exists z
\end{gather}
usw.

\subsubsection{Alternative Darstellung}
Sei $P\colon G\to\{0,1\}$ und $M\subseteq G$.
Mit $P(M)$ ist die Bildmenge von $P$ bezüglich $M$ gemeint.
Es gilt
\begin{equation}
\begin{split}
&\forall x{\in}M\,[P(x)] \iff P(M)=\{1\}\\
& \iff M\subseteq\{x{\in}G\mid P(x)\}
\end{split}
\end{equation}
und
\begin{equation}
\begin{split}
& \exists x{\in}M\,[P(x)] \iff \{1\}\subseteq P(M)\\
& \iff M\cap\{x{\in}G\mid P(x)\}\ne\{\}.
\end{split}
\end{equation}

\subsubsection{Eindeutigkeit}
Quantor für eindeutige Existenz:
\begin{equation}
\begin{split}
&\exists!x\,[P(x)]\\
&:\Longleftrightarrow\; \exists x\,[P(x)\land \forall y\,[P(y)\Rightarrow x=y]]\\
&\iff \exists x\,[P(x)]\land \forall x\forall y[P(x)\land P(y)\Rightarrow x=y].
\end{split}
\end{equation}

\newpage
\section{Mengenlehre}
\subsection{Definitionen}
Aufzählende Notation:
\begin{equation}
a\in\{x_1,\ldots,x_n\} :\Leftrightarrow a=x_1\lor\ldots\lor a=x_n.
\end{equation}
Beschreibende Notation:
\begin{gather}
a\in\{x\mid P(x)\}\;:\Longleftrightarrow\; P(a),\\
\{x\in M\mid P(x)\} := \{x\mid x\in M\land P(x)\},\\
\{f(x)\mid P(x)\} := \{y\mid y=f(x)\land P(x)\}.
\end{gather}
Teilmengenrelation:
\begin{equation}
A\subseteq B\;:\Longleftrightarrow\; \forall x\,[x\in A\implies x\in B].
\end{equation}
Gleichheit:
\begin{equation}
A=B\;:\Longleftrightarrow\; \forall x\,[x\in A\iff x\in B].
\end{equation}
Vereinigungsmenge:
\begin{equation}
A\cup B:=\{x\mid x\in A\lor x\in B\}.
\end{equation}
Schnittmenge:
\begin{equation}
A\cap B:=\{x\mid x\in A\land x\in B\}.
\end{equation}
Differenzmenge:
\begin{equation}
A\setminus B:=\{x\mid x\in A\land x\not\in B\}.
\end{equation}
Symmetrische Differenz:
\begin{equation}
A\triangle B:=\{x\mid x\in A\oplus x\in B\}.
\end{equation}
Komplementärmenge:
\begin{equation}
A^\comp := G\setminus A.\qquad (\text{$G$: Grundmenge})
\end{equation}
Vereinigung über indizierte Mengen:
\begin{equation}
\bigcup_{i\in I} A_i := \{x\mid\exists i{\in}I\,[x\in A_i]\}.
\end{equation}
Schnitt über indizierte Mengen:
\begin{equation}
\bigcap_{i\in I} A_i := \{x\mid\forall i{\in}I\,[x\in A_i]\}.
\end{equation}


\subsection{Boolesche Algebra}
\begin{table*}[t]
\caption{Boolesche Algebra}
\begin{tabular}{l|l|l}
\thbf{Vereinigung} & \thbf{Schnitt} &\\
  $A\cup A = A$
& $A\cap A = A$
& Idempotenzgesetze\\
  $A\cup \{\} = A$
& $A\cap G = A$
& Neutralitätsgesetze\\
  $A\cup G = G$
& $A\cap \{\} = \{\}$
& Extremalgesetze\\
  $A\cup \overline A = G$
& $A\cap \overline A = \{\}$
& Komplementärgesetze\\
\noalign{\vspace{1em}}
  $A\cup B = B\cup A$
& $A\cap B = B\cap A$
& Kommutativgesetze\\
  $(A\cup B)\cup C = A\cup (B\cup C)$
& $(A\cap B)\cap C = A\cap (B\cap C)$
& Assoziativgesetze\\
  $\overline{A\cup B} = \overline A\cap\overline B$
& $\overline{A\cap B} = \overline A\cup\overline B$
& De Morgansche Regeln\\
  $A\cup (A\cap B) = A$
& $A\cap (A\cup B) = A$
& Absorptionsgesetze\\
\end{tabular}\\
\\
$G$: Grundmenge
\end{table*}

\noindent
\strong{Distributivgesetze}:
\begin{gather}
M\cup (A\cap B) = (M\cup A)\cap (M\cup B),\\
M\cap (A\cup B) = (M\cap A)\cup (M\cap B).
\end{gather}

\subsection{Teilmengenrelation}
Zerlegung der Gleichheit:
\begin{equation}
A=B \iff A\subseteq B \land B\subseteq A.
\end{equation}
Umschreibung der Teilmengenrelation:
\begin{equation}
\begin{split}
A\subseteq B &\iff A\cap B=A\\
& \iff A\cup B=B\\
& \iff A\setminus B=\{\}.
\end{split}
\end{equation}
Kontraposition:
\begin{equation}
A\subseteq B = B^\comp\subseteq A^\comp.
\end{equation}

\subsection{Natürliche Zahlen}
\subsubsection{Von-Neumann-Modell}
Mengentheoretisches Modell der natürlichen Zahlen:
\begin{equation}
\begin{split}
& 0:=\{\},\quad 1:=\{0\},\quad 2:=\{0,1\},\\
& 3:=\{0,1,2\},\quad \text{usw.}
\end{split}
\end{equation}
Nachfolgerfunktion:
\begin{equation}
x' := x\cup\{x\}.
\end{equation}
\subsubsection{Vollständige Induktion}
Ist $A(n)$ mit $n\in\N$
eine Aussageform, so gilt:
\begin{equation}
\begin{split}
& A(n_0)\land \forall n\ge n_0\,[A(n)\Rightarrow A(n+1)]\\
& \implies \forall n\ge n_0\,[A(n)].
\end{split}
\end{equation}
Die Aussage $A(n_0)$ ist der \emph{Induktionsanfang}.

Die Implikation
\begin{equation}
A(n)\Rightarrow A(n+1)
\end{equation}
heißt \emph{Induktionsschritt}. Beim Induktionsschritt muss
$A(n+1)$ gezeigt werden, wobei $A(n)$ als gültig vorausgesetzt werden
darf.

\newpage
\subsection{ZFC-Axiome}

Axiom der Bestimmtheit:
\begin{equation}
\forall A\forall B\,[A=B\iff\forall x\,[x\in A\Leftrightarrow x\in B]].
\end{equation}
Axiom der leeren Menge:
\begin{equation}
\exists M\forall x\,[x\notin M].
\end{equation}
Axiom der Paarung:
\begin{equation}
\forall x\forall y\exists M\forall a\,[a\in M\iff x=a\lor y=a].
\end{equation}
Axiom der Vereinigung:
\begin{equation}
\forall S\exists M\forall x\,[x\in M\iff\exists A{\in}S\,[x\in A]].
\end{equation}
Axiom der Aussonderung:
\begin{equation}
\forall A\exists M\forall x\,[x\in M\iff x\in A\land\varphi(x)].
\end{equation}
Axiom des Unendlichen:
\begin{equation}
\exists M\,[\{\}\in M\land\forall x{\in}M\,[x\cup\{x\}\in M]].
\end{equation}
Axiom der Potenzmenge:
\begin{equation}
\forall A\exists M\forall T\,[T\in M\iff T\subseteq A].
\end{equation}
Axiom der Ersetzung:
\begin{equation}
\begin{split}
&\forall a{\in}A\;\exists^{=1} b\,[\varphi(a,b)]\\
&\implies\exists B\,\forall b\,[b\in B\iff\exists a{\in}A\,[\varphi(a,b)]].
\end{split}
\end{equation}
Axiom der Fundierung:
\begin{equation}
\forall A\,[A\ne\{\}\implies\exists x{\in}A\,[x\cap A=\{\}]].
\end{equation}
Auswahlaxiom:
\begin{equation}
\begin{split}
&\forall x,y{\in}A\,[x\ne y\implies x\cap y=\{\}]\\
&\quad\land\forall x{\in}A\,[x\ne\{\}]\\
&\implies\exists M\;\forall x{\in}A\;\exists^{=1}u{\in}x\,[u\in M].
\end{split}
\end{equation}

\newpage
\subsection{Kardinalität}
\begin{Definition}
Zwei Mengen $M,N$ heißen \emdef{gleichmächtig}, notiert als
$|M|=|N|$, wenn es eine bijektive Abbildung $f\colon M\to N$ gibt.

Eine Menge $M$ heißt \emdef{weniger mächtig oder gleichmächtig},
notiert als $|M|\le|N|$, wenn es eine injektive Abbildung
$f\colon M\to N$ gibt. Äquivalent dazu ist, dass es eine
surjektive Abbildung $g\colon N\to M$ gibt.

Eine Menge heißt \emdef{abzählbar unendlich}, wenn sie gleichmächtig
zu den natürlichen Zahlen ist.
\end{Definition}
Gleichmächtigkeit ist eine Äquivalenzrelation.
\begin{Definition}
Die Äquivalenzklassen
\begin{equation}
|M| := \{N\mid\;{\scriptstyle |M|=|N|}\}
\end{equation}
heißen \emdef{Kardinalzahlen}.
\end{Definition}

\strong{Satz von Cantor-Bernstein.}

Aus $|M|\le |N|$ und $|N|\le |M|$ folgt $|M|=|N|$.

\subsubsection{Potenzmengen}

\strong{Satz von Cantor.}
Für jede Menge gilt $|M|<|2^M|$.

Ist $M$ endlich, dann gilt $|M|=2^{|M|}$.


\section{Funktionen}
\subsection{Surjektionen}\index{surjektiv}
\begin{Definition}
Eine Funktion $f\colon A\to B$ heißt \emdef{surjektiv},\\
wenn $f(A)=B$ ist. Damit ist gemeint, dass jedes Element
der Zielmenge wenigstens einmal der Funktionswert von einem
Element der Definitionsmenge ist.
\end{Definition}

\subsection{Injektionen}\index{injektiv}
\begin{Definition}
Eine Funktion $f\colon A\to B$ heißt \emdef{injektiv},\\
wenn
\begin{equation}
\forall x_1,x_2\in A\,[f(x_1)=f(x_2)\implies x_1=x_2]
\end{equation}
gilt.
\end{Definition}

\subsection{Bijektionen}\index{bijektiv}
\begin{Definition}
Eine Funktion $f\colon A\to B$ heißt \emdef{bijektiv},
wenn sie injektiv und surjektiv ist.

Eine Funktion $f\colon A\to B$ ist genau dann bijektiv, wenn es
ein $g$ mit
\begin{equation}
g\circ f = \id_A\quad\text{und}\quad f\circ g = \id_B
\end{equation}
gibt. Wenn $f$ bijektiv ist, so gibt es $g$ genau einmal und
$g$ wird die \emph{Umkehrfunktion}\index{Umkehrfunktion}
oder \emph{Inverse}
von $f$ genannt und als $f^{-1}$ notiert.
\end{Definition}

\subsection{Komposition}\index{Komposition}
\begin{Definition} Für zwei Funktionen $f\colon A\to B$
und $g\colon B\to C$ ist die \emdef{Komposition}
($g$ nach $f$)
durch
\begin{equation}\label{eq:composition}
g\circ f\colon A\to C,\quad (g\circ f)(x) := g(f(x))
\end{equation}
definiert.
\end{Definition}
Für die Komposition gilt das Assozativgesetz:
\begin{equation}
(f\circ g)\circ h = f\circ(g\circ h).
\end{equation}

Die Komposition von Injektionen ist eine Injektion.

Die Komposition von Surjektionen ist eine Surjektion.

Die Komposition von Bijektionen ist eine Bijektion.

Sind $f,g$ Bijektionen, so gilt
\begin{equation}
(g\circ f)^{-1} = f^{-1}\circ g^{-1}.
\end{equation}

Ist $g\circ f$ injektiv, so ist $f$ injektiv.

Ist $g\circ f$ surjektiv, so ist $g$ surjektiv.

Ist $g\circ f$ bijektiv, so ist $f$ injektiv und $g$ surjektiv.

\begin{Definition}
Für eine Funktion $\varphi\colon A\to A$ wird
\begin{equation}
\varphi^0:=\operatorname{id}_A,\quad \varphi^{n+1}:=\varphi^n\circ\varphi
\end{equation}
\emdef{Iteration}\index{Iteration} von $\varphi$ genannt.
\end{Definition}

\subsection{Einschränkung}\index{Einschränkung}
\begin{Definition} Sei $f\colon A\to B$ und $M\subseteq A$.
Die Funktion $g(x)=f(x)$ mit $g\colon M\to B$ wird \emdef{Einschränkung}
von $f$ genannt und mit $f|_M$ notiert.
\end{Definition}
Sei $f\colon A\to B$ und $M\subseteq A$.
Mit der Inklusionsabbildung $i(x):=x$ mit $i\colon M\to A$ gilt:
\begin{equation}
f|_M = f\circ i.
\end{equation}
Es gilt
\begin{equation}
g\circ (f|_M) = (g\circ f)|_M.
\end{equation}
%\newpage
\subsection{Bild}\index{Bild}
\begin{Definition} Ist $f\colon A\to B$ und $M\subseteq A$, so wird
\begin{equation}
f(M) := \{f(x)\mid x\in M\}
\end{equation}
das \emdef{Bild} von $M$ unter $f$ genannt.
\end{Definition}
Es gilt
\begin{align}
&f(M\cup N) = f(M)\cup f(N),\\
&f(M\cap N) = f(M)\cap f(N),\\
&f\Big(\bigcup_{i\in I}M_i\Big) = \bigcup_{i\in I} f(M_i),\\
&I\ne\emptyset\implies f\Big(\bigcap_{i\in I} M_i\Big) = \bigcap_{i\in I} f(M_i),\\
&M\subseteq N\implies f(M)\subseteq f(N),\\
&f(\emptyset) = \emptyset,\\
&(g\circ f)(M) = g(f(M)).
\end{align}

\subsection{Urbild}\index{Urbild}
\begin{Definition}
Ist $f\colon A\to B$, so wird
\begin{equation}
f^{-1}(M) := \{x\in A\mid f(x)\in M\}.
\end{equation}
das \emdef{Urbild} von $M$ unter $f$ genannt.
\end{Definition}
Es gilt
\begin{align}
& f^{-1}(M\cup N) = f^{-1}(M)\cup f^{-1}(N),\\
& f^{-1}(M\cap N) = f^{-1}(M)\cap f^{-1}(N),\\
& f^{-1}\Big(\bigcup_{i\in I}M_i\Big) = \bigcup_{i\in I} f^{-1}(M_i),\\
& I\ne\emptyset\implies f^{-1}\Big(\bigcap_{i\in I} M_i\Big) = \bigcap_{i\in I}f^{-1}(M_i),\\
& M\subseteq N\implies f^{-1}(M)\subseteq f^{-1}(N),\\
& f^{-1}(\emptyset) = \emptyset,\\
& f^{-1}(B) = A,\\
& f^{-1}(M\setminus N) = f^{-1}(M)\setminus f^{-1}(N),\\
& f^{-1}(B\setminus M) = B\setminus f^{-1}(M),\\
& (g\circ f)^{-1}(M) = f^{-1}(g^{-1}(M)),\\
& (f|_M)^{-1}(N) = M\cap f^{-1}(N).
\end{align}

\newpage
\section{Mathematische Strukturen}\label{sec:Strukturen}
\subsubsection*{Axiome}

\noindent\bsf{E:} Abgeschlossenheit.
\ibox{Die Verknüpfung führt nicht aus der Menge heraus.}

\noindent\bsf{A:} Assoziativgesetz.
\ibox{$\forall a,b,c\bright (a*b)*c = a*(b*c)\bleft$.}

\noindent\bsf{N:} Existenz des neutralen Elements.
\ibox{$\exists e\forall a\bright e*a=a*e=a\bleft$.}

\noindent\bsf{I:} Existenz der inversen Elemente.
\ibox{$\forall a\exists b\bright a*b=b*a=e\bleft$.}

\noindent\bsf{K:} Kommutativgesetz.
\ibox{$\forall a,b\bright a*b=b*a\bleft.$}

\noindent
\bsf{I*:} Existenz der multiplikativ inversen Elemente.
\ibox{$\forall a{\ne}0\;\exists b\bright a*b=b*a=1\bleft$.}

\noindent\bsf{Dl:} Linksdistributivgestz.
\ibox{$\forall a,x,y\bright a*(x+y) = a*x+a*y\bleft$.}

\noindent\bsf{Dr:} Rechtsdistributivgesetz.
\ibox{$\forall a,x,y\bright (x+y)*a = x*a+y*a\bleft$.}

\noindent\bsf{D:} Distributivgesetze.
\ibox{Dl und Dr.}

\noindent\bsf{T:} Nullteilerfreiheit.
\ibox{$\forall a,b\bright a\ne 0\land b\ne 0\implies a*b\ne 0\bleft$}
\ibox{bzw. die Kontraposition}
\ibox{$\forall a,b\bright a*b=0\implies a=0\lor b=0\bleft$.}

\noindent\bsf{U:} Unterscheibarkeit von Null- und Einselement.
\ibox{Die neutralen Elemente bezüglich Addition und}
\ibox{Multiplikation sind unterschiedlich.}

\subsubsection*{Strukturen}
Strukturen mit einer inneren Verknüpfung:\\
\begin{tabular}{l|l}
\bsf{EA} & Halbgruppe\\
\bsf{EAN} & Monoid\\
\bsf{EANI} & Gruppe\\
\bsf{EANIK} & abelsche Gruppe
\end{tabular}

\noindent
Strukturen mit zwei inneren Verknüpfungen:\\
\begin{tabular}{l|l}
\bsf{EANIK, EA, D}\dotfill & Ring\\
\bsf{EANIK, EAK, D}\dotfill & kommutativer Ring\\
\bsf{EANIK, EAN, D}\dotfill & unitärer Ring\\
\bsf{EANIK, EANK, DTU} & Integritätsring\\
\bsf{EANIK, EANI*K, DTU} & Körper
\end{tabular}

\newpage
\subsubsection*{Axiome für Relationen}

\noindent\bsf{R:} Reflexivität.
\ibox{$\forall a\,(a R a)$.}

\noindent\bsf{S:} Symmetrie.
\ibox{$\forall a,b\,(aRb\iff bRa)$.}

\noindent\bsf{T:} Transitivität.
\ibox{$\forall a,b,c\,(aRb\land bRc\implies aRc)$.}

\noindent\bsf{An:} Antisymmetrie.
\ibox{$\forall a,b\,(aRb\land bRa\implies a=b)$.}

\noindent\bsf{L:} Linearität.
\ibox{$\forall a,b\,(aRb\lor bRa)$.}

\noindent\bsf{Ri:} Irrreflexivität.
\ibox{$\forall a\,(\neg aRa)$.}

\noindent\bsf{A:} Asymmetrie.
\ibox{$\forall a,b\,(aRb\implies \neg bRa)$.}

\noindent\bsf{Min:} Existenz der Minimalelemente.
\ibox{$\forall T{\subseteq}M, T{\ne}\emptyset\;\exists x{\in}T\;\forall y{\in}T{\setminus}\{x\}\,(x<y)$.}

\subsubsection*{Relationen}
\begin{tabular}{l|l}
\bsf{RST}\dotfill & Äquivalenzrelation\\
\bsf{RAnT}\dotfill & Halbordnung\\
\bsf{RAnTL}\dotfill & Totalordnung\\
\bsf{RiAT}\dotfill & strenge Halbordnung\\
\bsf{RiATL}\dotfill & strenge Totalordnung\\
\bsf{RiATLMin} & Wohlordnung
\end{tabular}


\chapter{Analysis}
\section{Ableitungen}
\subsection{Differentialquotient}
Sei $U\subseteq\mathbb R$ ein offenes Intervall
und sei $f\colon U\to\mathbb R$. Die Funktion $f$ heißt
differenzierbar an der Stelle $x_0\in U$, falls der Grenzwert
\begin{equation}
\begin{split}
&\lim_{x\to x_0} \frac{f(x)-f(x_0)}{x-x_0}
= \lim_{h\to 0}\frac{f(x+h)-f(x)}{h}
\end{split}
\end{equation}
existiert. Dieser Grenzwert heißt
Differentialquotient oder Ableitung
von $f$ an der Stelle $x_0$. Notation:
\begin{equation}
f'(x_0),\,\qquad (Df)(x_0),\qquad \frac{\mathrm df(x)}{\mathrm dx}\Big|_{x=x_0}.
\end{equation}



\chapter{Topologie}
\section{Grundbegriffe}
\subsection{Definitionen}

\begin{Definition}[nhfilter: Umgebungsfilter]%
\label{def:nhfilter}\index{Umgebungsfilter}
\[\underline U(x) := \{U{\subseteq}X\mid
\exists O(O\in T\land x\in O\land O\subseteq U)\}.\]
\end{Definition}

\begin{Definition}[int: offener Kern]
\label{def:int}\index{offener Kern}
\[\operatorname{int}(M) := \{x\in M\mid M\in \underline U(x)\}\]
\end{Definition}

\begin{Satz}
Der offene Kern von $M$ ist die Vereinigung der offenen Teilmengen
von $M$. Kurz:%
\[\operatorname{int}(M) = \bigcup_{O\in 2^M\cap T} O.\]
\end{Satz}

\begin{Beweis}
Nach Def. \ref{def:seteq} (seteq) und Def. \ref{def:int} (int)
expandieren:
\[\forall x[x\in M\land M\in\underline U(x)
\iff x\in\bigcup_{O\in 2^M\cap T} O].\]
Den äußeren Allquantor brauchen wir nicht weiter mitschreiben, da alle
freien Variablen automatisch allquantifiziert werden.
Nach Def. \ref{def:nhfilter} (nhfilter) weiter expandieren, wobei die
Bedingung $U\subseteq X$ als tautologisch entfallen kann,
weil $X$ die Grundmenge ist. Auf der rechten Seite wird nach Def.
\ref{def:union} (union) expandiert. Es ergibt sich:
\[x\in M\land \exists O(O\in T\land x\in O\land O\subseteq M)
\iff \exists O(O\subseteq M\land O\in T\land x\in O).\]
Wegen $A\land\exists x(P(x))\iff \exists x(A\land P(x))$ ergibt
sich auf der linken Seite:
\[\exists O(x\in M\land O\in T\land x\in O\land O\subseteq M).\]
Wenn aber $O\subseteq M$ erfüllt sein muss, gilt
$x\in O\implies x\in M$. Demnach kann $x\in M$ entfallen.
Auf beiden Seiten steht dann die gleiche Bedingung.\,\qedsymbol
\end{Beweis}

\section{Metrische Räume}
\subsection{Metrischer Räume}
\begin{Definition}[metric-space: metrischer Raum]%
\index{metrischer Raum}\label{metric-space}
Man bezeichet $(M,d)$ mit $d\colon M^2\to\R$ genau dann als
metrischen Raum, wenn die folgenden Axiome erfüllt sind:
\begin{align*}
\text{(M1)}\quad & d(x,y)=0\iff x=y, &&\text{(Gleichheit abstandsloser Punkte)}\\
\text{(M2)}\quad & d(x,y)=d(y,x), &&\text{(Symmetrie)}\\
\text{(M3)}\quad & d(x,y)\le d(x,z)+d(z,y). &&\text{(Dreiecksungleichung)}
\end{align*}
\end{Definition}

\begin{Definition}[open-ep-ball: offene Epsilon-Umgebung]\mbox{}\\
Für einen metrischen Raum $(M,d)$ und $p\in M$:
\[U_\varepsilon(p) := \{x\mid d(p,x)<\varepsilon\}.\]
\end{Definition}
Bemerkung: Unter einer Epsilon-Umgebung ohne weitere Attribute
versteht man immer eine offene Epsilon-Umgebung.

\begin{Satz}[Konstruktion disjunkter Epsilon-Umgebungen]%
\label{construction-disjoint-ep-balls}
Sei $(M,d)$ ein metrischer Raum und $p,q\in M$ mit $p\ne q$.
Betrachte die Streckenzerlegung $d(p,q)=A+B$. Für $a\le A$ und
$b\le B$ sind die Epsilon-Umgebungen $U_a(p)$ und $U_b(q)$ disjunkt.
\end{Satz}

\begin{Beweis}
Angenommen $U_a(p)$ und $U_b(q)$ wären nicht disjunkt, dann gäbe
es mindestens ein $x$ mit $x\in U_a(p)$ und $x\in U_b(q)$, d.\,h.
$d(p,x)<a$ und $d(q,x)<b$. Addition der beiden Ungleichungen
bringt
\[d(p,x)+d(q,x)<a+b\le d(p,q).\]
Gemäß der Dreiecksungleichung Def. \ref{metric-space} Axiom (M3) gilt
nun aber
\[d(p,q)\le d(p,x)+d(q,x)\]
für alle $x$. Sei $c:=d(p,x)+d(q,x)$. Wir erhalten damit nun
$c<a+b\le c$ und somit den Widerspruch $c<c$.\,\qedsymbol
\end{Beweis}

\begin{Korollar}[Unterschiedliche Punkte eines metrischen Raumes
besitzen disjunkte Epsilon-Umgebungen]
Sei $(M,d)$ ein metrischer Raum und $p,q\in M$.
Wenn $p\ne q$ ist, dann gibt es disjunkte offene
Epsilon-Umgebungen $U_a(p)$ und $U_b(q)$.
\end{Korollar}

\begin{Beweis}
Folgt trivial aus Satz \ref{construction-disjoint-ep-balls}.
Wähle speziell z.\,B. $a=b=d(p,q)/2$.\,\qedsymbol
\end{Beweis}

\subsection{Normierte Räume}
\begin{Definition}[normed-space: normierter Raum]%
\label{def:normed-space}\index{normierter Raum}\index{Dreiecksungleichung}
Sei $V$ ein Vektorraum über dem Körper der rellen oder komplexen
Zahlen. Sei $N(x)=\|x\|$ eine Abbildung, die jedem $x\in V$ eine
reelle Zahl zuordnet. Man nennt $(V,N)$ genau dann einen
normierten Raum, wenn die folgenden Axiome erfüllt sind:
\begin{align*}
\text{(N1)}\quad &\|x\|=0\iff x=0,&&\text{(Definitheit)}\\
\text{(N2)}\quad &\|\lambda x\|=|\lambda|\|x\|,&&\text{(betragsmäßige Homogenität)}\\
\text{(N3)}\quad &\|x+y\| \le \|x\|+\|y\|.&&\text{(Dreiecksungleichung)}
\end{align*}
\end{Definition}

\begin{Satz}[umgekehrte Dreiecksungleichung]%
\label{rev-tineq}\index{umgekehrte Dreiecksungleichung}%
\index{Dreiecksungleichung!umgekehrte}
In jedem normierten Raum gilt
\[|\|x\|-\|y\|| \le \|x-y\|.\]
\end{Satz}
\begin{Beweis}
Auf beiden Seiten von Def. \ref{def:normed-space} (normed-space)
Axiom (N3) wird $\|y\|$ subtrahiert.
Es ergibt sich
\[\|x+y\| - \|y\| \le \|x\|.\]
Substitution $x:=x-y$ bringt nun
\[\|x\| - \|y\| \le \|x-y\|.\]
Vertauscht man nun $x$ und $y$, dann ergibt sich
\[\|y\|-\|x\| \le \|y-x\| \iff -(\|x\|-\|y\|)\le \|x-y\|.\]
Wir haben nun $a\le b$ und $-a\le b$,
wobei $a:=\|x\|-\|y\|$ und $b:=\|x-y\|$ ist. Multipliziert
man die letzte Ungleichung mit $-1$, dann ergibt sich $a\ge -b$.
Somit ist $-b\le a\le b$, kurz $|a|\le b$.\,\qedsymbol
\end{Beweis}

\subsection{Homöomorphien}
\begin{Satz}[Verallgemeinerung des Zwischenwertsatzes]%
\label{intermediate-value-general}\mbox{}\\*
Ist $f\colon X\to Y$ eine stetige Abbildung zwischen topologischen
Räumen und $A\subseteq X$ ein zusammenhängender Teilraum,
dann ist auch $f(A)$ zusammenhängend.
\end{Satz}

\begin{Satz}
Eine injektive Abbildung $f\colon\R_{\ge 0}\to\R$ kann nicht stetig sein.
\end{Satz}
\begin{Beweis}
Da $f$ injektiv ist, ist die Rechnung
\[f(\R_{>0}) = f(\R_{\ge 0}\setminus\{0\})
= f(\R_{\ge 0})\setminus f(\{0\}) = \R\setminus\{f(0)\}\]
gültig gemäß Satz \ref{inj-img-setminus}. Da $\R_{>0}$ zusammenhängend
ist, $\R\setminus\{f(0)\}$ aber nicht, kann $f$ laut Satz
\ref{intermediate-value-general} nicht stetig sein.\;\qedsymbol
\end{Beweis}


\chapter{Lineare Algebra}
\section{Grundbegriffe}
\subsection{Norm}\index{Norm}
\begin{Definition}
Eine Abbildung $v\mapsto\|v\|$ von einem
Vektorraum $V$ über dem Körper $K$ in die nichtnegativen reellen
Zahlen heißt \emdef{Norm}, wenn für alle $v,w\in V$ und $a\in K$
die drei Axiome%
\begin{gather}
\|v\|=0 \implies v=0,\\
\|av\| = |a|\,\|v\|,\\
\|v+w\| \le \|v\|+\|w\|
\end{gather}
erfüllt sind.
\end{Definition}

Eigenschaften:
\begin{gather}
\|v\|=0\iff v=0,\\
\|-v\|=\|v\|,\\
\|v\|\ge 0.
\end{gather}
Dreiecksungleichung nach unten:
\begin{equation}
|\|v\|-\|w\||\le \|v-w\|.
\end{equation}

\subsection{Skalarprodukt}\index{Skalarprodukt}

\subsubsection{Axiome}
{\definition}
Eine Abbildung heißt \emdef{Skalarprodukt},
wenn folgende Axiome erfüllt sind.

Axiome für $v,w$ aus einem reellen Vektorraum und $\lambda$ ein Skalar:
\begin{gather}
\langle v,w\rangle = \langle w,v\rangle,\\
\langle v,\lambda w\rangle = \lambda\langle v,w\rangle,\\
\langle v,w_1+w_2\rangle = \langle v,w_1\rangle +\langle v,w_2\rangle,\\
\langle v,v\rangle\ge 0,\\
\langle v,v\rangle=0 \iff v=0.
\end{gather}
Axiome für $v,w$ aus einem komplexen Vektorraum und $\lambda$ ein Skalar:
\begin{gather}
\langle v,w\rangle = \overline{\langle w,v\rangle},\\
\langle \lambda v,w\rangle = \overline{\lambda}\langle v,w\rangle,\\
\langle v,\lambda w\rangle = \lambda\langle v,w\rangle,\\
\langle v,w_1+w_2\rangle = \langle v,w_1\rangle +\langle v,w_2\rangle,\\
\langle v,v\rangle\ge 0,\\
\langle v,v\rangle=0 \iff v=0.
\end{gather}

\subsubsection{Eigenschaften}
Das reelle Skalarprodukt ist eine symmetrische bilineare Abbildung.

\subsubsection{Winkel und Längen}
{\definition}
Der \emdef{Winkel} $\varphi$ zwischen $v$ und $w$
ist definiert durch die Beziehung:
\begin{equation}
\langle v,w\rangle = \|v\|\,\|w\|\,\cos\varphi.
\end{equation}
{\definition}
\emdef{Orthogonal}:\index{Orthogonal}
\begin{equation}
v\perp w \;:\Longleftrightarrow\; \langle v,w\rangle=0.
\end{equation}
Ein Skalarprodukt $\langle v,w\rangle$ induziert die Norm
\begin{equation}
\|v\| := \sqrt{\langle v,v\rangle}.
\end{equation}

\subsubsection{Orthonormalbasis}\label{sec:ONB}
\index{Orthogonalsystem}\index{Orthogonalbasis}
\index{Orthonormalsystem}\index{Orthonormalbasis}
Sei $B=(b_k)_{k=1}^n$ eine Basis eines endlichdimensionalen
Vektorraumes über den reellen oder komplexen Zahlen.

{\definition}
Gilt $\langle b_i,b_j\rangle=0$
für alle $i,j$ mit $i\ne j$, so wird $B$ \emdef{Orthogonalbasis}
genannt. Ist $B$ nicht unbedingt eine Basis, so spricht man von einem 
\emdef{Orthogonalsystem}.

{\definition}
Ist $B$ eine Orthogonalbasis und gilt
zusätzlich $\langle b_k,b_k\rangle=1$ für alle $k$, so wird
$B$ \emdef{Orthonormalbasis} (ONB) genannt. Ist $B$ nicht unbedingt
eine Basis,  so spricht man von einem \emdef{Orthonormalsystem}.

Sei $v=\sum_k v_kb_k$ und $w=\sum_k w_kb_k$.
Mit $\sum_k$ ist immer $\sum_{k=1}^n$ gemeint.

Ist $B$ eine Orthonormalbasis, so gilt:
\begin{equation}
\langle v,w\rangle = \sum_k \overline{v_k}\,w_k.
\end{equation}
Ist $B$ nur eine Orthogonalbasis, so gilt:
\begin{equation}
\langle v,w\rangle = \sum_k \langle b_k,b_k\rangle \overline{v_k}\,w_k
\end{equation}
Allgemein gilt:
\begin{equation}
\langle v,w\rangle = \sum_{i,j} g_{ij} \overline{v_i}\,w_j
\end{equation}
mit $g_{ij}=\langle b_i,b_j\rangle$. In reellen Vektorräumen
ist die komplexe Konjugation wirkungslos und kann somit entfallen.

Ist $B$ eine Orthogonalbasis und $v=\sum_k v_k b_k$, so gilt:
\begin{equation}
v_k = \frac{\langle b_k,v\rangle}{\langle b_k,b_k\rangle}.
\end{equation}
Ist $B$ eine Orthonormalbasis, so gilt speziell:
\begin{equation}
v_k = \langle b_k,v\rangle.
\end{equation}


\subsubsection{Orthogonale Projektion}
Orthogonale Projektion von $v$ auf $w$:
\begin{equation}
P[w](v) := \frac{\langle v,w\rangle}{\langle w,w\rangle}\,w.
\end{equation}
\subsubsection{Gram-Schmidt-Verfahren}
Für linear unabhängige Vektoren $v_1,\ldots,v_n$
wird durch%
\begin{equation}
w_k := v_k - \sum_{i=1}^{k-1} P[w_i](v_k)
\end{equation}
ein Orthogonalsystem $w_1,\ldots,w_n$ berechnet.

Speziell für zwei nicht kollineare Vektoren $v_1,v_2$ gilt
\begin{gather}
w_1=v_1,\\
w_2=v_2-P[w_1](v_2).
\end{gather}
\section{Koordinatenvektoren}
\subsection{Koordinatenraum}
Addition von $a,b\in K^n$:
\begin{equation}\label{eq:Koordinatenraum-Addition}
\begin{bmatrix}
a_1\\
\vdots\\
a_n
\end{bmatrix}
+\begin{bmatrix}
b_1\\
\vdots\\
b_n
\end{bmatrix}
:= \begin{bmatrix}
a_1+b_1\\
\vdots\\
a_n+b_n
\end{bmatrix}.
\end{equation}
Subtraktion:
\begin{equation}
\begin{bmatrix}
a_1\\
\vdots\\
a_n
\end{bmatrix}
-\begin{bmatrix}
b_1\\
\vdots\\
b_n
\end{bmatrix}
:= \begin{bmatrix}
a_1-b_1\\
\vdots\\
a_n-b_n
\end{bmatrix}.
\end{equation}
Skalarmultiplikation von $\lambda\in K$ mit $a\in K^n$:
\begin{align}\label{eq:Koordinatenraum-Skalarmultiplikation}
\lambda\begin{bmatrix}
a_1\\
\vdots\\
a_n
\end{bmatrix}
:= \begin{bmatrix}
\lambda a_1\\
\vdots\\
\lambda a_n
\end{bmatrix}.
\end{align}
Ist $K$ ein Körper, so bildet die Menge
\begin{equation}
K^n = \{(a_1,\ldots,a_n)\mid \forall k\colon a_k\in K\}
\end{equation}
bezüglich der Addition \eqref{eq:Koordinatenraum-Addition}
und der Multiplikation \eqref{eq:Koordinatenraum-Skalarmultiplikation}
einen Vektorraum, der \emdef{Koordinatenraum} genannt wird.
Das Tupel $E_n=(e_1,\ldots,e_n)$ mit
\begin{equation}\label{eq:kanonische-Basis}
\begin{split}
e_1 &:= (1,0,0,0,\ldots, 0),\\
e_2 &:= (0,1,0,0,\ldots, 0),\\
e_3 &:= (0,0,1,0,\ldots, 0),\\
\ldots\\
e_n &:= (0,0,0,0,\ldots, 1)
\end{split}
\end{equation}
bildet eine geordnete Basis von $K^n$, die \emdef{kanonische Basis}
genannt wird. Es gilt
\begin{equation}
a = (a_1,\ldots,a_n) = a_1 e_1+\ldots+a_n e_n.
\end{equation}

\subsection{Kanonisches Skalarprodukt}
\begin{Definition}
Für $a,b\in\R^n$:
\begin{equation}
\langle a,b\rangle := \sum_{k=1}^n a_k b_k.
\end{equation}
Für $a,b\in\C^n$:
\begin{equation}
\langle a,b\rangle := \sum_{k=1}^n \overline{a_k}\,b_k
\end{equation}
\end{Definition}
\noindent
Die kanonische Basis \eqref{eq:kanonische-Basis} ist eine
Orthonormalbasis bezüglich diesem Skalarprodukt, s. \ref{sec:ONB}.
Das Skalarprodukt induziert die Norm
\begin{equation}
|a| := \sqrt{\langle a,a\rangle} = \sqrt{\textstyle \sum_{k=1}^n |a_k|^2},
\end{equation}
die \emdef{Vektorbetrag} genannt wird.

Jedem Koordinatenvektor $a\ne 0$ lässt sich ein Einheitsvektor
$\hat a:=\frac{a}{|a|}$ zuordnen, der in Richtung von $a$ zeigt
und die Eigenschaft $|\hat a|=1$ besitzt.


\section{Matrizen}\index{Matrix}
\subsection{Quadratische Matrizen}%
\index{Matrix!quadratische}\index{quadratische Matrix}
\subsubsection{Matrizenring}%
\index{Ring!Matrizenring}\index{Matrizenring}
Mit $K^{n\times n}$ wird die Menge quadratischen Matrizen
\begin{equation}
(a_{ij}) = \begin{bmatrix}
a_{11} & \ldots & a_{1n}\\
\ldots & \ddots & \ldots\\
a_{n1} & \ldots & a_{nn}
\end{bmatrix}
\end{equation}
mit Einträgen $a_{ij}$ aus dem Körper $K$ bezeichnet.

Die Menge $K^{n\times n}$ bildet bezüglich Addition
und Multiplikation von Matrizen einen Ring (s. \ref{sec:Strukturen}).

Das neutrale Element der Multiplikation
ist die Einheitsmatrix
\begin{equation}
E_n = (\delta_{ij}),\quad
\delta_{ij}:=\begin{cases}
1 & \text{wenn}\;i=j,\\
0 & \text{sonst}.
\end{cases}
\end{equation}
Das sind
\begin{equation}
E_2 = \begin{bmatrix}
1 & 0\\
0 & 1
\end{bmatrix},\quad
E_3 = \begin{bmatrix}
1 & 0 & 0\\
0 & 1 & 0\\
0 & 0 & 1
\end{bmatrix},
\quad\text{usw.}
\end{equation}

\subsubsection{Symmetrische Matrizen}
Eine quadratiche Matrix $A=(a_{ij})$ heißt
symmetrisch\index{symmetrische Matrix},
falls gilt $a_{ij}=a_{ji}$ bzw. $A^T=A$.

Jede reelle symmetrische Matrix besitzt ausschließlich reelle
Eigenwerte und die algebraischen Vielfachheiten stimmen mit den
geometrischen Vielfachheiten überein.

Jede reelle symmetrische Matrix $A$ ist diagonalisierbar, d.\,h. es gibt
eine invertierbare Matrix $T$ und eine Diagonalmatrix $D$, so dass
$A=TDT^{-1}$ gilt.

Sei $V$ ein $K$-Vektorraum und $(b_k)_{k=1}^n$ eine Basis von $V$.
Für jede symmetrische Bilinearform\index{symmetrische Bilinearform}
$f\colon V^2\to K$ ist die
Darstellungsmatrix
\begin{equation}
A = (f(b_i,b_j))
\end{equation}
symmetrisch. Ist $A\in K^{n\times n}$ eine symmetrische Matrix, so
ist
\begin{equation}\label{eq:symmBf}
f(x,y) = x^T A y.
\end{equation}
eine symmetrische Bilinearform für  $x,y\in K^n$.
Ist $K=\R$ und $A$ positiv definit, so ist
\eqref{eq:symmBf} ein Skalarprodukt auf $\R^n$.

\subsubsection{Reguläre Matrizen}\index{inverse Matrix}
Eine quadratische Matrix $A\in K^{n\times n}$ heißt \emdef{regulär}
oder \emdef{invertierbar}, wenn es eine inverse Matrix $A^{-1}$ gibt,
so dass
\begin{equation}
A^{-1}A = E_n \quad (\iff AA^{-1} = E_n)
\end{equation}
gilt, wobei mit $E_n$ die Einheitsmatrix gemeint ist. Jede
reguläre Matrix besitzt genau eine inverse Matrix. Eine Matrix $A$
ist genau dann regulär, wenn $\det(A)\ne 0$ gilt. Die Menge
der regulären Matrizen bildet bezüglich Matrizenmultiplikation
eine Gruppe, die
\emdef{allgemeine lineare Gruppe}\index{allgemeine lineare Gruppe}
\begin{equation}
\operatorname{GL}(n,K) := \{A\in K^{n\times n}\mid\det(A)\ne 0\}.
\end{equation}
Ist $V$ ein Vektorraum über dem Körper $K$, so bilden die
Automorphismen bezüglich Verkettung eine Gruppe, die
\emph{Automorphismengruppe}
\begin{equation}
\operatorname{GL}(V) = \operatorname{Aut}(V).
\end{equation}
Ein \emdef{Endomorphismus}\index{Endomorphismus!auf einem Vektorraum}
ist eine lineare Abbildung, welche eine Selbstabbildung ist.
Ein \emdef{Automorphismus}\index{Automorphismus!auf einem Vektorraum}
ist eine bijektiver Endomorphismus.

Wählt man auf $V$ eine Basis
$B$, so ist die Zuordnung der Darstellungsmatrix
\begin{equation}
M_B^B\colon \operatorname{Aut}(V)\to\operatorname{GL}(\dim V,K)
\end{equation}
eine Gruppenisomorphismus.

Eine quadratische Matrix, die nicht regulär ist, heißt
\emdef{singulär}. Endomorphismen, die nicht bijektiv sind, lassen
die Dimension ihrer Definitionsmenge schrumpfen:
\begin{equation}
f{\in}\operatorname{End}(V){\setminus}\operatorname{Aut}(V)
\Longleftrightarrow \dim f(V)<\dim V.
\end{equation}
Für Matrizen $A\in K^{n\times n}$ bedeutet das, dass sie nicht
den vollen Rang besitzen:
\begin{equation}
\det A=0\iff \operatorname{rk}(A) < n = \dim K^n.
\end{equation}
Inversionsformel:
\begin{equation}
\begin{bmatrix}
a & b\\
c & d
\end{bmatrix}^{-1}
= \frac{1}{ad-bc}\begin{bmatrix}
d & -b\\
-c & a
\end{bmatrix}.
\end{equation}
\begin{Definition}
Wird in der Matrix $A$ die Zeile $i$ und die Spalte $j$ entfernt,
so entsteht eine neue Matrix $[A]_{ij}$, die
\emdef{Streichungsmatrix}\index{Streichungsmatrix}
von $A$ genannt wird.
\end{Definition}
Laplacescher Entwicklungssatz:
\begin{align}
\det A = \sum_{i=1}^n (-1)^{i+j}a_{ij}\det([A]_{ij}),\\
\det A = \sum_{j=1}^n (-1)^{i+j}a_{ij}\det([A]_{ij}).
\end{align}

\subsection{Determinanten}\index{Determinante}
Für Matrizen $A,B\in K^{n\times n}$ und $r\in K$ gilt:
\begin{gather}
\det(AB) = \det(A)\det(B),\\
\det(A^T) = \det(A),\\
\det(rA) = r^n\det(A),\\
\det(A^{-1}) = \det(A)^{-1}.
\end{gather}
Für eine Diagonalmatrix $D=\diag(d_1,\ldots,d_n)$ gilt:
\begin{gather}
\det(D) = \prod_{k=1}^n d_k.
\end{gather}
Eine linke Dreiecksmatrix ist eine Matrix der Form
$(a_{ij})$ mit $a_{ij}=0$ für $i<j$. Eine rechte
Dreiecksmatrix ist die Transponierte einer linken
Dreiecksmatrix.

Für eine linke oder rechte Dreiecksmatrix $A=(a_{ij})$ gilt:
\begin{gather}
\det(A) = \prod_{k=1}^n a_{kk}.
\end{gather}

\subsection{Eigenwerte}
\strong{Eigenwertproblem:}\index{Eigenwert}
Für eine gegebene quadratische Matrix $A$ bestimme
\begin{equation}
\{(\lambda,v)\mid Av = \lambda v,\,v\ne 0\}.
\end{equation}
Das homogene lineare Gleichungssystem
\begin{equation}
Av=\lambda v \iff (A-\lambda E_n)v=0
\end{equation}
besitzt Lösungen $v\ne 0$ gdw.
\begin{equation}
p(\lambda)=\det(A-\lambda E_n)=0.
\end{equation}
Bei $p(\lambda)$ handelt es sich um ein normiertes Polynom
vom Grad $n$, das \emdef{charakeristisches Polynom}
\index{charakteristisches Polynom}
genannt wird.

\strong{Eigenraum:}\index{Eigenraum}
\begin{equation}
\Eig(A,\lambda):=\{v\mid Av=\lambda v\}.
\end{equation}
Die Dimension $\dim\Eig(A,\lambda)$ wird
\emdef{geometrische Vielfachheit}\index{geometrische Vielfachheit}
von $\lambda$ genannt.

\strong{Spektrum:}\index{Spektrum}
\begin{equation}
\sigma(A) := \{\lambda\mid \exists v\ne 0\colon Av=\lambda v\}.
\end{equation}

\section{Lineare Gleichungssysteme}
\index{lineares Gleichungssytem}
Ein lineares Gleichungssystem mit $m$ Gleichungen und $n$ Unbekannten
hat die Form:
\begin{equation}\label{eq:LGS}
\begin{split}
a_{11} x_1 + a_{12} x_2 + \ldots + a_{1n} x_n &= b_1,\\
a_{21} x_1 + a_{22} x_2 + \ldots + a_{2n} x_n &= b_2,\\
&\;\;\vdots\\
a_{m1} x_1 + a_{m2} x_2 + \ldots + a_{mn} x_n &= b_n.
\end{split}
\end{equation}
Das System lässt sich durch
\begin{equation}
A:=\begin{bmatrix}
a_{11} & a_{12} & \ldots & a_{1n}\\
a_{21} & a_{22} & \ldots & a_{2n}\\
\vdots & \vdots & \ddots & \vdots\\
a_{m1} & a_{m1} & \ldots & a_{mn}
\end{bmatrix}
\end{equation}
und
\begin{equation}
x:=\begin{bmatrix}
x_1 \\ x_2 \\ \vdots \\ x_n
\end{bmatrix},\quad
b:=\begin{bmatrix}
b_1 \\ b_2 \\ \vdots \\ b_n
\end{bmatrix}
\end{equation}
zusammenfassen.

Äquivalente Matrixform von \eqref{eq:LGS}:
\begin{equation}
Ax=b.
\end{equation}
Erweiterte Koeffizientenmatrix:\index{erweiterte Koeffizientenmatrix}
\begin{equation}
(A\,|\,b) := \left[\begin{array}{ccc|c}
a_{11} & \ldots & a_{1n} & b_1\\
\vdots & \ddots & \vdots & \vdots\\
a_{m1} & \ldots & a_{mn} & b_n
\end{array}\right].
\end{equation}
Lösungskriterium:
\begin{equation}
\exists x[Ax=b] \iff \rg(A)=\rg(A\,|\,b).
\end{equation}
Eindeutige Lösung (bei $n$ Unbekannten):
\begin{equation}
\exists! x[Ax=b] \iff\rg(A)=\rg(A\,|\,b)=n.
\end{equation}
Im Fall $m=n$ gilt:
\begin{equation}
\begin{split}
&\exists! x[Ax=b] \iff A\in\operatorname{GL}(n,K)\\
&\iff \rg(A)=n \iff \det(A)\ne 0.
\end{split}
\end{equation}

\newpage
\section{Multilineare Algebra}
\subsection{Äußeres Produkt}
Sei $V$ ein Vektorraum und sei $v_k\in V$ für alle $k$.

Sind $a=\sum_{k=1}^n a_k v_k$
und $b=\sum_{k=1}^n b_k v_k$ beliebige
Linearkombinationen, so gilt
\begin{equation}
\begin{split}
a\wedge b &= \sum_{i,j} a_i b_j\,v_i\wedge v_j\\
&= \sum_{1\le i<j\le n} (a_i b_j-a_j b_i)\,v_i\wedge v_j
\end{split}
\end{equation}
und
\begin{equation}
\begin{split}
a\wedge b &= a\otimes b-b\otimes a\\
&= \sum_{i,j} (a_i b_j-a_j b_i)\,v_i\otimes v_j\\
&= \sum_{i,j} a_i b_j (v_i\otimes v_j-v_j\otimes v_i).
\end{split}
\end{equation}
\subsubsection{Alternator}\index{Alternator}
Für $a_k\in V$ ist
$\operatorname{Alt}_p\colon T^p(V)\to A^p(V)\subseteq T^p(V)$
mit
\begin{equation}
\begin{split}
& \operatorname{Alt}_p (a_1\otimes\ldots\otimes a_p)\\
&:= \frac{1}{p!}\sum_{\sigma\in S_{\scriptstyle p}}
\sgn(\sigma)\,(a_{\sigma(1)}\otimes\ldots\otimes a_{\sigma(p)}).
\end{split}
\end{equation}
Es ist $\Lambda^p(V)$ isomorph zu $A^p(V)$ und man setzt:
\begin{equation}
a_1\wedge\ldots\wedge a_p
= p!\operatorname{Alt}_p(a_1\otimes\ldots\otimes a_p).
\end{equation}
Speziell gilt
\begin{equation}
\operatorname{Alt}_2 (a\otimes b) := \frac{1}{2}(a\otimes b-b\otimes a).
\end{equation}
und
\begin{equation}
a\wedge b = 2\operatorname{Alt}_2(a\otimes b).
\end{equation}

\subsubsection{Äußere Algebra}\index{aussere Algebra@äußere Algebra}
Darstellung als Quotientenraum:
\begin{equation}
\Lambda^2(V) = T^2(V)/\{v\otimes v\mid v\in V\}.
\end{equation}
Dimension: Ist $\dim(V)=n$, so gilt
\begin{equation}
\dim(\Lambda^k(V)) = \binom{n}{k}.
\end{equation}

\clearpage
\section{Analytische Geometrie}
\subsection{Geraden}\index{Gerade}
\subsubsection{Parameterdarstellung}
\index{Parameterdarstellung!einer Geraden}

\strong{Punktrichtungsform:}\index{Punktrichtungsform}
\begin{equation}
p(t) = p_0+t\underline v,
\end{equation}
$p_0$: Stützpunkt, $\underline v$: Richtungsvektor.
Die Gerade ist dann die Menge $g=\{p(t)\mid t\in\R\}$.

Der Vektor $\underline v$ repräsentiert außerdem die Geschwindigkeit,
mit der diese Parameterdarstellung durchlaufen wird:
$p'(t)=\underline v$.

\strong{Gerade durch zwei Punkte:}
Sind zwei Punkte $p_1,p_2$ mit $p_1\ne p_2$ gegeben, so ist
durch die beiden Punkte eine Gerade gegeben. Für diese Gerade ist
\begin{equation}
p(t) = p_1+t(p_2-p_1)
\end{equation}
eine Punktrichtungsform\index{Punktrichtungsform}.
Durch Umformung ergibt sich die \strong{Zweipunkteform:}
\begin{equation}\label{eq:Zweipunkteform}
p(t) = (1-t)p_1+tp_2.
\end{equation}
Bei \eqref{eq:Zweipunkteform} handelt es sich um eine
Affinkombination. Gilt $t\in[0,1]$, so ist \eqref{eq:Zweipunkteform}
eine Konvexkombination: eine Parameterdarstellung für die Strecke
von $p_1$ nach $p_2$.

\subsubsection{Parameterfreie Darstellung}
\strong{Hesse-Form:}
\begin{equation}\label{eq:Hesse-Form}
g = \{p\mid\langle \uv n,p-p_0\rangle = 0\},
\end{equation}
$p_0$: Stützpunkt, $\uv n$: Normalenvektor.

Die Hesse-Form ist nur in der Ebene möglich.
Form \eqref{eq:Hesse-Form} hat in Koordinaten
die Form
\begin{equation}
\begin{split}
g &= \{(x,y)\mid n_x(x-x_0)+n_y(y-y_0)=0\}\\
&= \{(x,y)\mid n_x x+n_y y = n_x x_0+n_y y_0\}.
\end{split}
\end{equation}

\strong{Hesse-Normalform:} \eqref{eq:Hesse-Form} mit $|\uv n|=1$.


Sei $\uv v\wedge\uv w$ das äußere Produkt.

\strong{Plückerform:}
\begin{equation}
g = \{p\mid (p-p_0)\wedge \underline v=0\}.
\end{equation}
Die Größe $\underline m = p_0\wedge\underline v$ heißt Moment.
Beim Tupel $(\underline v:\underline m)$ handelt es sich um
Plückerkoordinaten für die Gerade.

In der Ebene gilt speziell:
\begin{equation}\label{eq:Gerade-Ebene}
g = \{(x,y)\mid (x-x_0)\Delta y = (y-y_0)\Delta x\}
\end{equation}
mit $\underline v=(\Delta x,\Delta y)$.

Sei $a:=\Delta y$ und $b:=-\Delta x$ und $c:=ax_0+by_0$.
Aus \eqref{eq:Gerade-Ebene} ergibt sich:
\begin{equation}
g = \{(x,y)\mid ax+by=c\}.
\end{equation}
Im Raum ergibt sich ein Gleichungssystem:
\begin{equation}
g = \{\begin{pmatrix}x\\ y\\ z\end{pmatrix}
\mid
\begin{vmatrix}
(x-x_0)\Delta y = (y-y_0)\Delta x\\
(y-y_0)\Delta z = (z-z_0)\Delta y\\
(x-x_0)\Delta z = (z-z_0)\Delta x
\end{vmatrix}\}
\end{equation}
mit $\underline v=(\Delta x,\Delta y,\Delta z)$.

\subsubsection{Abstand Punkt zu Gerade}
Sei $p(t):=p_0+t\underline v$ die Punktrichtungsform einer Geraden und
sei $q$ ein weiterer Punkt. Bei $\underline d(t):=p(t)-q$ handelt
es sich um den Abstandsvektor in Abhängigkeit von $t$.

Ansatz: Es gibt genau ein $t$, so dass gilt:
\begin{equation}
\langle\underline d,\underline v\rangle=0.
\end{equation}
Lösung:
\begin{equation}
t = \frac
  {\langle\underline v,q{-}p_0\rangle}
  {\langle\underline v,\underline v\rangle}.
\end{equation}

\subsection{Ebenen}\index{Ebene}
\subsubsection{Parameterdarstellung}
\index{Parameterdarstellung!einer Ebene}
Seien $\uv u, \uv v$ zwei nicht kollineare Vektoren.

Punktrichtungsform:
\begin{equation}\label{eq:Ebene-Punktrichtungsform}
p(s,t) = p_0+s\uv u+t\uv v.
\end{equation}

\subsubsection{Parameterfreie Darstellung}
Seien $\uv v, \uv w$ zwei nicht kollineare Vektoren.
Durch
\begin{equation}
E = \{p\mid (p-p_0)\wedge\uv v\wedge\uv w=0\}.
\end{equation}
wird eine Ebene beschrieben.

\strong{Hesse-Form:}
\begin{equation}
E = \{p\mid \langle\uv n,p-p_0\rangle=0\},
\end{equation}
$p_0$: Stützpunkt, $\uv n$: Normalenvektor. Die Hesse-Form einer
Ebene ist nur im dreidimensionalen Raum möglich.
Den Normalenvektor bekommt man aus \eqref{eq:Ebene-Punktrichtungsform}
mit $\uv n = \uv u\times\uv v$.

\subsubsection{Abstand Punkt zu Ebene}
Sei $p(s,t):=p_0+s\uv u+t\uv v$ die Punktrichtungsform einer Ebene
und sei $q$ ein weiterer Punkt. Bei $\uv d(s,t):=p-q$ handelt es sich um
den Abstandsvektor in Abhängigkeit von $(s,t)$.

Ansatz: Es gibt genau ein Tupel $(s,t)$, so dass gilt:
\begin{equation}
\langle\uv d,\uv u\rangle=0\enspace\text{und}\enspace
\langle\uv d,\uv v\rangle=0.
\end{equation}
Lösung: Es ergibt sich ein LGS:
\begin{equation}
\begin{bmatrix}
\langle\uv u,\uv v\rangle & \langle\uv v,\uv v\rangle\\
\langle\uv v,\uv v\rangle & \langle\uv u,\uv v\rangle
\end{bmatrix}
\begin{bmatrix}
s\\ t
\end{bmatrix}
= \begin{bmatrix}
\langle\uv v,q{-}p_0\rangle\\
\langle\uv u,q{-}p_0\rangle
\end{bmatrix}.
\end{equation}
Bemerkung: Die Systemmatrix $g_{ij}$ ist der metrische Tensor für die
Basis $B=(\uv u,\uv v)$. Die Lösung des LGS ist:
\begin{gather}
s = \frac
  {\langle g_{12}\uv v-g_{12}\uv u, q{-}p_0\rangle}
  {g_{11}^2-g_{12}^2},\\
t = \frac
  {\langle g_{12}\uv u-g_{12}\uv v, q{-}p_0\rangle}
  {g_{11}^2-g_{12}^2}.
\end{gather}



\chapter{Algebra}
\section{Gruppentheorie}
\subsection{Grundbegriffe}
\begin{definition}[Gruppenhomomorphismus]
Sind $(G,*)$ und $(H,\bullet)$ zwei Gruppen, so
heißt $\varphi\colon G\to H$ \emdef{Gruppenhomomorphismus}%
\index{Gruppenhomomorphismus}, wenn
\begin{gather}
\forall g_1,g_2\in G\colon
  \varphi(g_1*g_2) = \varphi(g_1)\bullet\varphi(g_2)
\end{gather}
gilt. Ein \emdef{Gruppenisomorphismus}\index{Isomorphismus!zwischen Gruppen}
ist ein bijektiver Gruppenhomomorphismus, da die Umkehrabbildung
auch wieder ein Gruppenhomomorphismus ist.
\end{definition}
\begin{definition}[Direktes Produkt]
\emdef{Direktes Produkt}\index{direktes Produkt}:
\begin{gather}
G\times H := \{(g,h)\mid g\in G, h\in H\},\\
(g_1,h_1)*(g_2,h_2) := (g_1*g_2, h_1*h_2).
\end{gather}
\end{definition}
\noindent
\strong{Satz von Lagrange.} Für Gruppen $G,H$ gilt:
\begin{equation}
H\le G\implies |G| = |G/H|\cdot |H|.
\end{equation}

\subsection{Gruppenaktionen}\label{Gruppenaktion}
\begin{definition}[Gruppenaktion]
Eine Funktion $f\colon G\times X\to X$ heißt
\emdef{Gruppenaktion}\index{Gruppenaktion}, wenn
\begin{gather}
\hspace{-1em}\forall g_1,g_2{\in}G, x{\in}X\colon f(g_1,f(g_2,x)) = f(g_1 g_2,x),\\
\hspace{-1em}\forall x\in X\colon f(e,x) = x
\end{gather}
gilt, wobei mit $e$ das neutrale Element von $G$ gemeint ist.
Anstelle von $f(g,x)$ wird üblicherweise kurz $gx$ (oder
$g+x$ bei einer Gruppe $(G,+)$) geschrieben.

Anstelle von \emdef{Linksaktionen} kommen auch \emdef{Rechtsaktionen}
vor, die sich von Linksaktionen in der Reihenfolge unterscheiden.
Eine Rechtsaktion $f\colon X\times G\to X$ genügt den Regeln
\begin{gather}
\hspace{-1em}\forall g_1,g_2{\in}G, x{\in}X\colon f(f(x,g_1),g_2) = f(x,g_1 g_2),\\
\hspace{-1em}\forall x\in X\colon f(x,e) = x.
\end{gather}
\end{definition}

\begin{definition}[Orbit, Stabilisator]
Für ein $x\in X$ wird
\begin{equation}\label{eq:Orbit}
Gx := \{gx\mid g\in G\}
\end{equation}
\emdef{Bahn}\index{Bahn} oder
\emdef{Orbit}\index{Orbit!unter einer Gruppenaktion} genannt.
Die Menge
\begin{equation}
G_x := \{g\in G\mid gx=x\}
\end{equation}
wird \emdef{Fixgruppe}\index{Fixgruppe}
oder \emdef{Stabilisator}\index{Stabilisator} genannt.
Die Menge
\begin{equation}
X^g := \{x\in X\mid gx=x\}
\end{equation}
heißt \emdef{Fixpunktmenge}.
\end{definition}

\noindent
Fixgruppen sind immer Untergruppen:
\begin{equation}
\forall x\colon G_x\le G.
\end{equation}
Bahnen sind Äquivalenzklassen, die Quotientenmenge
\begin{equation}
X/G := \{Gx\mid x\in X\}
\end{equation}
wird \emdef{Bahnenraum}\index{Bahnenraum} genannt.

\strong{Bahnformel.}\index{Bahnformel}
Ist $G$ eine endliche Gruppe, so gilt:
\begin{equation}
|G| = |Gx|\cdot |G_x|.
\end{equation}
\strong{Lemma von Burnside.}\index{Lemma von Burnside}
Für eine endliche Gruppe $G$ gilt:%
\begin{equation}
|X/G| = \frac{1}{|G|}\sum_{g\in G}|X^g|.
\end{equation}

\section{Ringe}\index{Ring}
Ist $(R,+,*)$ ein Ring, so gilt für alle $a,b\in R$:
\begin{align}
0*a &= a*0 = 0,\\
(-a)*b &= a*(-b) = -(a*b),\\
(-a)*(-b) &= -(a*b).
\end{align}
\begin{definition}[Ringhomomorphismus]
Sind $(R,+,*)$ und $(R',+',*')$ Ringe, so wird
$\varphi\colon R\to R'$ als \emdef{Ringhomomorphismus}
bezeichnet, wenn
\begin{align}
\varphi(a+b) &= \varphi(a)+'\varphi(b),\\
\varphi(a*b) &= \varphi(a)*'\varphi(b),
\end{align}
für alle $a,b\in R$ gilt und $\varphi(1)=1$ ist.
\end{definition}

\subsection{Polynome}\index{Polynom}
\begin{definition}[Polynom, Polynomring, Koeffizienten]
Sei $R$ ein kommutativer unitärer Ring.
Mit $R[X]$ bezeichnen wir die Menge der unendlichen Folgen
\begin{equation}
(a_k) = (a_0,a_1,\ldots,a_n,0,0,0,\ldots)
\end{equation}
mit $a_k\in R$, bei denen ab einem Index alle Folgenglieder null sind.

Für zwei Folgen aus $R[X]$ wird nun die Addition
\begin{equation}
(a_k) + (b_k) := (a_k+b_k)
\end{equation}
und die Multiplikation
\begin{equation}\label{eq:Faltung}
(a_i)*(b_j) = \bigg(\sum_{i=0}^k a_i b_{k-i}\bigg)
\end{equation}
erklärt. In der Form \eqref{eq:Faltung} wird die Operation auch
\emdef{Faltung}\index{Faltung!von zwei Folgen}
der Folgen $(a_i)$ und $(b_j)$ genannt.

Die Menge $R[X]$ bildet mit der Addition und Multiplikation
einen kommutativen unitären Ring, den \emdef{Polynomring}
mit Koeffizienten in $R$. Ein Element von $R[X]$ wird
\emdef{Polynom} genannt.

Man definiert nun
\begin{equation}
X:=(0,1,0,0,0,\ldots),
\end{equation}
womit sich jedes Polynom in der Form
\begin{equation}\textstyle
(a_k) = \sum_{k=0}^n a_k X^k
\end{equation}
schreiben lässt. Die $a_k$ nennt man \emdef{Koeffizienten}
des Polynoms.
\end{definition}

\noindent
Die Addition bekommt nun die Form
\begin{equation}
\sum_{k=0}^m a_k X^k + \sum_{k=0}^n b_k X^k
:= \sum_{k=0}^p (a_k+b_k)X^k.
\end{equation}
mit $p=\max(m,n)$. Die Multiplikation lässt sich nun in der Form
\begin{equation}
\bigg(\sum_{i=0}^m a_i X^i\bigg)\bigg(\sum_{j=0}^n b_j X^j\bigg)
:= \sum_{k=0}^{m+n}\bigg(\sum_{i=0}^k a_i b_{k-i}\bigg) X^k.
\end{equation}
schreiben. Die Multiplikation von Polynomen ist das gewöhnlichen
Ausmultiplizieren der Polynome, wobei $X^i X^j=X^{i+j}$ gilt.

Die $X^k$ können als Vektorraumbasis betrachtet
werden und dienen dabei dazu, die $a_k$ auseinanderzuhalten.
Zwei Polynome $\sum_{k=0}^m a_k X^k$ und $\sum_{k=0}^n b_k X^k$
sind genau dann gleich, wenn $a_k=b_k$ für alle $k\le\max(m,n)$ gilt.

Da $R[X]$ wieder ein kommutativer unitärer Ring ist,
ist auch $R[X][Y]$ ein Polynomring. Man definiert
\begin{equation}
R[X,Y] := R[X][Y].
\end{equation}
Polynome aus $R[X,Y]$ lassen sich in der Form
\begin{equation}
\sum_{j=0}^n \bigg(\sum_{i=0}^m a_{ij}X^i\bigg)Y^j
= \sum_{i=0}^m\sum_{j=0}^n a_{ij} X^i Y^j
\end{equation}
mit $a_{ij}\in R$ schreiben.

Allgemein ist die Menge
\begin{equation}
R[X_1,\ldots,X_q] := X[X_1,\ldots,X_{q-1}][X_q]
\end{equation}
ein kommutativer unitärer Ring. Die Polynome lassen sich in der Form
\begin{equation}
\sum_{k\in\N_0^q} a_k X^k\quad (a_k\in R)
\end{equation}
mit
\[k=(k_1,\ldots,k_q)\quad\text{und}\quad X^k:=\prod_{i=1}^q X_i^{k_i}\]
schreiben.

\begin{definition}[Grad]
Für ein Polynom $f=\sum_{k=0}^n a_k X^k$ mit $a_n\ne 0$ wird
$n$ als \emdef{Grad} von $f$ bezeichnet. Man schreibt $n=\deg f$.

Für ein Monom $a_{ij} X^i Y^j$ mit $a_{ij}\ne 0$ heißt $i+j$
\emdef{Totalgrad}. Der \emdef {Grad} eines Polynoms
\begin{equation}
\textstyle\sum_{i=1}^m\sum_{j=1}^n a_{ij} X^i Y^j
\end{equation}
ist der maximale Totalgrad aller Monome mit $a_{ij}\ne 0$.
Für Polynome in mehr als zwei Variablen ist die Definition analog.
\end{definition}

\strong{Regeln.}\\
Für zwei Polynome $f,g\in R[X_1,\ldots,X_q]$ gilt:
\begin{align}
\deg(f+g)&\le \max(\deg f,\deg g),\\
\deg(fg)&\le (\deg f)(\deg g).
\end{align}
Für zwei Polynome $f,g$ mit $\deg f\ne\deg g$ gilt:
\begin{equation}
\deg(f+g) = \max(\deg f,\deg g).
\end{equation}
Ist $R$ ein Integritätsring, so gilt für $f,g\in R[X_1,\ldots,X_q]$:%
\begin{equation}
\deg(fg) = (\deg f)(\deg g).
\end{equation}
Jeder Körper, z.\,B. $\R$ oder $\C$ ist ein Integritätsring.
Auch die ganzen Zahlen $\Z$ bilden einen Integritätsring.
Ein Polynomring ist genau dann ein Integritätsring, wenn die
Koeffizienten aus einem Integritätsring entstammen.

\begin{definition}[Einsetzungshomomorphismus]
Seien $R,R'$ kommutative unitäre Ringe. Sei $R'$ eine Ringerweiterung
von $R$ und sei $r\in R'$. Die Abbildung $\varphi_r\colon R[X]\to R'$
mit
\begin{equation}\textstyle
\varphi_r(p) = p(r) := \sum_{k=0}^n a_k r^k 
\end{equation}
für jedes Polynom
\[\textstyle p = \sum_{k=0}^n a_k X^k\]
ist ein Ringhomomorphismus. Man bezeichnet $p(r)$ als \emdef{Einsetzung}
von $r$ in $p$ und $\varphi_r$ als
\emdef{Einsetzungshomomorphismus}\index{Einsetzungshomomorphimus}.
\end{definition}

Man kann auch $R'=R$ und $r=X$ setzen, dann gilt $p=p(X)$.
Ein Polynom stimmt also mit der Einsetzung seiner eigenen formalen
Variablen überein.

\begin{definition}[Polynomfunktion]
Für ein festes $p\in R[X]$ wird die Funktion
\begin{equation}
f\colon R'\to R',\quad f(x):=p(x)
\end{equation}
als \emdef{Polynomfunktion} bezeichnet.
\end{definition}

\noindent
In einigen Ringen können unterschiedliche Polynome zur selben
Polynomfunktion führen. Handelt es sich bei $R$ jedoch um einen
unendlichen Körper, z.\,B. $R=\R$ oder $R=\C$, dann gibt es zu jeder
Polynomfunktion nur ein einziges Polynom.

\section{Körper}
\begin{definition}[Körperhomomorphismus]
Sind $(K,+,\bullet)$ und $(K',+',\bullet')$ Körper, so
wird $\varphi\colon K\to K'$ als \emph{Körperhomomorphismus}
bezeichnet, wenn
\begin{align}
\varphi(a+b) &= \varphi(a)+'\varphi(b),\\
\varphi(a\bullet b) &= \varphi(a)\bullet'\varphi(b)
\end{align}
für alle $a,b\in K$ gilt und $\varphi(1)=1$ ist.
\end{definition}




\chapter{Kombinatorik}
\section{Kombinatorische Funktionen}
\subsection{Faktorielle}\index{Faktorielle}
\subsubsection{Fakultät}\index{Fakultät}
\strong{Definition.} Für $n\in\mathbb Z, n\ge 0$:
\begin{equation}
n! := \prod_{k=1}^n k.
\end{equation}
Rekursionsgleichung:
\begin{equation}
(n+1)! = n!\,(n+1)
\end{equation}
Die Gammafunktion ist eine Verallgemeinerung der Fakultät:
\begin{equation}
n! = \Gamma(n+1).
\end{equation}

\subsubsection{Fallende Faktorielle}
\strong{Definition.} Für $a\in\mathbb C$ und $k\ge 0$:
\begin{equation}\label{eq:FF}
a^{\underline k} := \prod_{j=0}^{k-1} (a-j).
\end{equation}
Für $n\ge k$ und $k\ge 0$ gilt:
\begin{equation}
n^{\underline k} = \frac{n!}{(n-k)!}.
\end{equation}
Für $a\in\mathbb C\setminus\{-1,-2,\ldots\}$
und $k\in\mathbb C$ gilt:
\begin{equation}
a^{\underline k} = \frac{\Gamma(a+1)}{\Gamma(a-k+1)}.
\end{equation}

\subsubsection{Steigende Faktorielle}
\strong{Definition.} Für $a\in\mathbb C$ und $k\ge 0$:
\begin{equation}
a^{\overline k} := \prod_{j=0}^{k-1} (a+j).
\end{equation}
Für $n\ge 1$ und $n+k\ge 1$ gilt:
\begin{equation}
n^{\overline k} = \frac{(n+k-1)!}{(n-1)!}.
\end{equation}

\subsection{Binomialkoeffizienten}\index{Binomialkoeffizient}
\strong{Definition.} Für $a\in\mathbb C$
und $k\in\mathbb Z$:
\begin{equation}
\binom{a}{k} := \begin{cases}
\frac{a^{\underline k}}{k!} & \text{wenn}\;k>0,\\
1 & \text{wenn}\;k=0,\\
0 & \text{wenn}\;k<0.
\end{cases}
\end{equation}
Für $a\in\mathbb C\setminus\{-1,-2,\ldots\}$ und $b\in\mathbb C$:
\begin{equation}\label{eq:bc-allg}
\binom{a}{b} := \frac{\Gamma(a+1)}{\Gamma(b+1)\Gamma(a-b+1)}.
\end{equation}
Es gilt die Symmetriebeziehung
\begin{equation}
\binom{a}{b} = \binom{a}{a-b}
\end{equation}
und die Rekursionsgleichung
\begin{equation}
\binom{a+1}{b+1} = \binom{a}{b+1}+\binom{a}{b}.
\end{equation}
Für $a\in\mathbb C$ und $k\in\mathbb Z$ gilt:
\begin{equation}
\binom{-a}{k} = (-1)^k \binom{a+k-1}{k}.
\end{equation}

\section{Formale Potenzreihen}
\subsection{Binomische Reihe}
\strong{Definition.} Für $a\in\mathbb C$:
\begin{equation}
(1+X)^a := \sum_{k=0}^\infty \binom{a}{k} X^k
\end{equation}
Es gilt:
\begin{equation}
(1+X)^{a+b} = (1+X)^a (1+X)^b 
\end{equation}
und
\begin{equation}
(1+X)^{ab} = ((1+X)^a)^b.
\end{equation}


\chapter{Wahrscheinlichkeitsrechnung}

\section{Diskrete Verteilungen}

\subsection{Diskreter Wahrscheinlichkeitsraum}

\begin{definition}[Ergebnis, Ereignis, Ergebnismenge, Ereignisraum,
unmögliches Ereignis, sicheres Ereignis]\mbox{}\newline%
\index{Ergebnismenge}\index{Ereignisraum}%
\index{sicheres Ereignis}\index{unmögliches Ereignis}
Eine abzählbare \emdef{Ergebnismenge} $\Omega$ ist eine endliche
(oder abzählbar unendliche) Menge, die als Grundmenge verwendet wird.
Ein Element von $\Omega$ heißt \emdef{Ergebnis} oder
\emdef{Elementarereignis}.

Die Potenzmenge $2^\Omega$ heißt \emdef{Ereignisraum}, die
Elemente heißen \emdef{Ereignisse}.
Man nennt die leere Menge $\emptyset$ das \emdef{unmögliche} und $\Omega$
das \emdef{sichere} Ereignis.
\end{definition}

\begin{definition}[Diskreter Wahrscheinlichkeitsraum,\\
Wahrscheinlichkeitsmaß]\mbox{}\newline%
\index{Wahrscheinlichkeitsraum!diskreter}\index{Wahrscheinlichkeitsmaß!diskretes}%
\index{Verteilung!diskrete Wahrscheinlichkeitsverteilung}
Ein Paar $(\Omega,P)$ heißt \emdef{diskreter Wahrscheinlichkeitsraum}, wenn
$\Omega$ eine abzählbare Ergebnismenge ist und%
\begin{equation}
P(A):=\sum_{\omega\in A} P(\{\omega\}),\quad P\colon 2^\Omega\to [0,1]
\end{equation}
die Eigenschaft
\begin{equation}
\sum_{\omega\in\Omega} P(\{\omega\})=1
\end{equation}
besitzt. Die Abbildung $P$ heißt (das von den
Einzelwahrscheinlichkeiten induzierte) \emdef{Wahrscheinlichkeitsmaß}.
Man spricht auch von einer \emdef{Verteilung} auf $\Omega$.
\end{definition}


\subsection{Axiome von Kolmogorow}
\begin{definition}[Wahrscheinlichkeitsmaß\\
(Axiome von Kolmogorow)]\mbox{}\newline%
\index{Wahrscheinlichkeitsmaß!Axiome von Kolmogorow}%
\index{Axiome von Kolmogorow}
Gegeben ist ein Messraum $(\Omega,\Sigma)$. Man nennt $P$ ein
\emdef{Wahrscheinlichkeitsmaß}, wenn gilt:

1. $P$ ist eine Funktion $P\colon\Sigma\to [0,1]$.

2. $P(\Omega)=1$.

3. Ist $I$ eine abzählbare Indexmenge und sind die $A_i$
für $i\in I$ paarweise disjunkte Ereignisse, so gilt
\begin{equation}
P\Big(\bigcup_{i\in I} A_i\Big) = \sum_{i\in I}P(A_i).
\end{equation}
\end{definition}

\noindent
Bei einem diskreten Wahrscheinlichkeitsraum $(\Omega,P)$ mit
$\Sigma=2^\Omega$ sind die Axiome erfüllt.

\subsection{Rechenregeln}
Aus den Axiomen von Kolmogorow folgen folgende
Rechenregeln für ein Wahrscheinlichkeitsmaß $P$:
\begin{gather}
P(\emptyset) = 0,\\
P(\Omega) = 1,\\
P(A\cup B) = P(A)+P(B)-P(A\cap B).
\end{gather}
Man nennt $A^\comp:=\Omega\setminus A$ das
\emdef{komplementäre Ereignis}\index{komplementäres Ereignis}
zu $A$. Es gilt:
\begin{gather}
A\cup A^\comp = \Omega,\\
A\cap A^\comp = \emptyset,\\
P(A\cup A^\comp) = P(A)+P(A^\comp) = 1.
\end{gather}
\strong{Mehrstufige Experimente.}
Ein zweistufiges Zufallsexperiment mit einem ersten Ergebnis aus
$\Omega_1$ und einem zweiten aus $\Omega_2$ lässt sich als
Zufallsexperiment modellieren, bei dem die Ergebnismenge das
kartesische Produkt $\Omega=\Omega_1\times\Omega_2$ ist. Bei einem
$n$-stufigen Experiment gilt
\begin{equation}
\Omega = \Omega_1\times\ldots\times\Omega_n.
\end{equation}

\strong{Erste Pfadregel.}
Sei $a\in\Omega_1$, $b\in\Omega_2$, $A=\{a\}\times\Omega_2$
und $B=\Omega_1\times\{b\}$. Es gilt%
\begin{equation}
P(\{(a,b)\}) = P(A\cap B) = P(A)\,P(B\mid A).
\end{equation}
Das Ereignis $\{(a,b)\}$ tritt ein, wenn zuerst
der Pfad $A$ eingetreten ist, und dann auch der Pfad $B$.
Die Wahrscheinlichkeit ist das Produkt der Pfadwahrscheinlichkeiten
$P(A)$ und $P(B\mid A)$.

\strong{Zweite Pfadregel.}
Sind $a,b\in\Omega$ zwei unterschiedliche
Ergebnisse, dann gilt%
\begin{equation}
P(\{a\}\cup\{b\}) = P(\{a\})+P(\{b\}).
\end{equation}
Wenn die Teilexperimente eines mehrstufigen Experiments
stochastisch unabhängig sind, dann gilt nach der ersten Pfadregel
die Formel%
\begin{equation}
P(\{(a_1,\ldots,a_n)\}) = \prod_{k=1}^n P(A_k),
\end{equation}
wobei $A_k$ der Pfad zu $a_k$ ist.
Für den Fall, dass die einzelnen
Experimente alle Laplace-Experimente sind, gilt speziell%
\begin{equation}
P(\{(a_1,\ldots,a_n)\}) = \frac{1}{|\Omega|} = \prod_{k=1}^n \frac{1}{|\Omega_k|}
\end{equation}
mit $\Omega=\Omega_1\times\ldots\times\Omega_n$ und $(a_1,\ldots,a_n)\in\Omega$.

Führt man immer wieder dasselbe Laplace-Experiment aus, gilt
mit $t\in\Omega$ und $\Omega=\Omega_1^n$ die Regel%
\begin{equation}
P(t) = \frac{1}{|\Omega|} = \frac{1}{|\Omega_1|^n}.
\end{equation}
Würfelt man z.\,B. $n$-mal hintereinander, dann gibt es $6^n$ Pfade
und für jeden Pfad ergibt sich eine Wahrscheinlichkeit von $(1/6)^n$.

\subsection{Bedingte Wahrscheinlichkeit}
\begin{definition}[Bedingte Wahrscheinlichkeit]\mbox{}\newline%
\index{bedingte Wahrscheinlichkeit}
Für zwei Ereignisse $A,B$ mit $P(B)>0$ nennt man%
\begin{equation}
P(A\mid B) := \frac{P(A\cap B)}{P(B)}
\end{equation}
die \emdef{bedingte Wahrscheinlichkeit} von $A$, vorausgesetzt $B$.
\end{definition}
Bei
\begin{equation}
P'(A) := P(A\mid B),\quad P'\colon 2^B\to [0,1]
\end{equation}
handelt es sich wieder um ein Wahrscheinlichkeitsmaß.

\strong{Satz von Bayes.} Für $P(A)>0$ und $P(B)>0$ gilt%
\begin{equation}
P(A\mid B) = \frac{P(B\mid A)\, P(A)}{P(B)}.
\end{equation}

\strong{Gesetz der totalen Wahrscheinlichkeit.} Bilden die $B_i$
eine Zerlegung des Wahrscheinlichkeitsraums, dann gilt%
\begin{equation}
P(A) = \sum_{i\in I} P(A\mid B_i)P(B_i).
\end{equation}
Das Gesetz kann als eine Form zweiten in Verbindung mit
der ersten Pfadregel betrachtet werden.


\newpage
\subsection{Unabhängige Ereignisse}
\begin{definition}[Stochastische Unabhängigkeit]\mbox{}\newline%
\index{stochastisch unabhängig}
Zwei Ereignisse $A,B$ heißen \emdef{stochastisch unabhängig}, wenn%
\begin{equation}
P(A\cap B) = P(A)\, P(B)
\end{equation}
gilt.
\end{definition}

\subsection{Gleichverteilung}
\begin{definition}[Gleichverteilung (Laplace-Verteilung)]%
\mbox{}\newline\index{Gleichverteilung}\index{Laplace-Verteilung}
Sei $\Omega$ eine endliche Ergebnismenge. Mann nennt $P$ eine
\emdef{Gleichverteilung} oder \emdef{Laplace-Verteilung}, wenn%
\begin{equation}
P(\{\omega\}) = \frac{1}{|\Omega|}
\end{equation}
für alle Ergebnisse $\omega\in\Omega$ gilt.
\end{definition}

\noindent
Für eine Gleichverteilung gilt
\begin{equation}
P(A) = \frac{|A|}{|\Omega|}.
\end{equation}

\subsection{Zufallsvariablen}
\begin{definition}[Zufallsvariable]\mbox{}\newline
Sei $(\Omega,P)$ ein diskreter Wahrscheinlichkeitsraum. Jede Funktion%
\begin{equation}
X\colon\Omega\to\R
\end{equation}
heißt \emdef{Zufallsvariable}. Die Funktionswerte $x=X(\omega)$ heißen
\emdef{Realisationen} der Zufallsvariable.
\end{definition}

\noindent
Eine Zufallsvariable $X$ ordent dem Raum $(\Omega,P)$
einen neuen Wahrscheinlichkeitsraum $(\R,P_X)$ zu, wobei%
\begin{equation}
P_X\colon 2^{X(\Omega)}\to [0,1],\; P_X(A):=P(X^{-1}(A))
\end{equation}
definiert wird. Mit
\begin{equation}
X^{-1}(A) := \{\omega\in\Omega\mid X(\omega)\in A\}
\end{equation}
ist das Urbild von $A$ gemeint.
Die folgenden Kurzschreibweisen haben sich
eingebürgert:
\begin{align}
P(X\in A) &:= P(\{\omega\mid X(\omega)\in A\}),\\
P(X=x) &:= P(\{\omega\mid X(\omega)=x\}),\\
P(X\le x) &:= P(\{\omega\mid (X\omega)\le x\}).
\end{align}

\begin{definition}[Verteilungsfunktion]\mbox{}\newline
Für eine Zufallsvariable $X$ wird
\begin{equation}
F(x):=P(X\le x),\quad F\colon\R\to [0,1]
\end{equation}
\emdef{Verteilungsfunktion} von $X$ genannt.
\end{definition}

\noindent
\strong{Eigenschaften von Verteilungsfunktionen.}\\
Für eine Verteilungsfunktion $F$ gilt:
\begin{gather}
\bulletbs F\text{ ist monoton wachsend},\\
\bulletbs F\text{ ist rechtsseitig stetig},\\
\bulletbs \lim\limits_{x\to -\infty} F(x) = 0,\\
\bulletbs \lim\limits_{x\to\infty} F(x)=1,\\
\bulletbs P(a<X\le b) = F(b)-F(a).
\end{gather}

\subsection{Erwartungswert}

\begin{definition}[Erwartungswert]\mbox{}\\*
Sei $\Omega$ endlich. Zu einer Zufallsgröße
$X\colon\Omega\to\R$ ist%
\begin{equation}
\E(X) := \sum_{\omega\in\Omega} X(\omega)P(\{\omega\})
\end{equation}
der \emdef{Erwartungswert}.
\end{definition}
Es gilt die praktische Formel
\begin{equation}
\E(X) = \!\!\!\sum_{x\in X(\Omega)}\!\!\! xP(X=x)
= \!\!\!\sum_{x\in X(\Omega)}\!\!\! xP(X^{-1}(x)).
\end{equation}
Wenn $(x_i)$ eine Abzählung von $X(\Omega)$ ist, schreibt man
die Formel alternativ in der Form
\begin{equation}
\E(X) = \sum_{i=1}^n x_i p_i = \sum_{i=1}^n x_i P(X=x_i).
\end{equation}
Elementare Eigenschaften sind
\begin{gather}
\E(X+Y) = \E(X)+\E(Y),\\
\E(aX) = a\E(x),\\
\E(1_A) = P(A),\\
X\le Y \implies \E(X) \le \E(Y).
\end{gather}

\subsection{Zufallszahlen}

\begin{definition}[Zufallszahlengenerator]\mbox{}\\*
Sei $(X_k)$ eine Folge von unabhängigen und
identisch verteilten Zufallsgrößen. Eine Folge
$(x_k)$ von Realisierungen $x_k=X_k(\omega_k)$ wird
Zufallszahlengenerator (kurz RNG, engl. \emph{random number generator})
genannt.
\end{definition}

\noindent
Bemerkung: Die $x_k$ werden durch Auswürfeln oder algorithmisch
ermittelt, wobei die $\omega_k$ unbekannt bleiben und auch
mathematisch keine Rolle spielen.

\minisection\strong{Inversionsmethode.}
Die uniforme Verteilung ist definiert durch die Verteilungsfunktion
\begin{equation}
U\colon\R\to [0,1],\quad U(x):=\begin{cases}
0\;\text{wenn}\;x<0,\\
x\;\text{wenn}\;x\in [0,1],\\
1\;\text{wenn}\;x>1.
\end{cases}
\end{equation}
Hat man nur einen Generator $(u_k)$ zur Verfügung, der uniform
verteilte Zufallszahlen erzeugt, möchte aber Zufallszahlen $x_k$
mit Verteilungsfunktion $F$ erzeugen, dann lassen sich diese
gemäß
\begin{equation}
x_k = F^{-1}(u_k)
\end{equation}
ermitteln. Ist $F$ stetig und streng monoton steigend, dann
ist $F^{-1}$ die Umkehrfunktion von $F$, andernfalls setzt man
\begin{equation}
F^{-1}(u) := \inf\{x\in\R\mid F(x)\ge u\}.
\end{equation}

\minisection\strong{Gesetz der totalen Wahrscheinlichkeit.}
Für eine Zufallsgröße $X$ mit Verteilungsfunktion $F$ gilt
\begin{equation}
P(A) = \int_{-\infty}^\infty P(A\mid X=x)\,\mathrm dF(x).
\end{equation}


\chapter{Zahlentheorie}

\section{Kongruenzen}

\begin{Definition}[Kongruenz]\index{Kongruenz}
Zwei ganze Zahlen $a,b$ heißen kongruent modulo $m$, wenn ihre Differenz
$(b-a)$ durch $m$ teilbar ist:%
\[a\equiv b\pmod{m}\defiff (\exists k\in\Z)(b-a=km).\]
\end{Definition}
Anstelle von »$(\mathrm{mod}\;m)$« schreibt man beim Rechnen meist
kürzer »$(m)$«.

\begin{Satz}
Die Kongruenz ist eine Äquivalenzrelation, d.\,h. es gilt
\begin{align*}
&a\equiv a\pmod{m},&&\text{(Reflexivität)}\\
&a\equiv b\implies b\equiv a\pmod{m},&&\text{(Symmetrie)}\\
&a\equiv b\land b\equiv c\implies a\equiv c\pmod{m}.&&\text{(Transitivität)}
\end{align*}
\end{Satz}
\strong{Beweis.} Für die Reflexivität ist ein $k$ mit $0=a-a=km$
zu finden. Setze $k=0$.

Bei der Symmetrie gibt es nach Voraussetzung
ein $k$ mit $b-a=km$. Dann ist $a-b=-km$. Setze $k'=-k$.
Es gibt also $k'$ mit $a-b=k'm$, somit gilt $b\equiv a$.

Bei der Transitivität gibt es nach Voraussetzung $k$ mit
$b-a=km$ und $l$ mit $b-c=lm$. Das heißt, es gilt
\[b = a+km = c+lm\implies c-a = km-lm = (k-l)m.\]
Setze $k'=k-l$. Es gibt also $k'$ mit $c-a=k'm$.
Somit gilt $a\equiv c$.\;\qedsymbol

\begin{Satz}\label{Kongruenz-add-sub}
Sind $a,b,c$ ganze Zahlen, dann gilt
\begin{align*}
a\equiv b\pmod{m}&\iff a+c\equiv b+c\pmod{m},\\
a\equiv b\pmod{m}&\iff a-c\equiv b-c\pmod{m}.
\end{align*}
\end{Satz}
\strong{Beweis.}
Unter Beachtung von $(b+c)-(a+c)=b-a$ findet man
\begin{gather*}
a\equiv b\pmod{m}
\iff (\exists k\in\Z)(b-a=km)\\
\iff (\exists k\in\Z)((b+c)-(a+c)=km)\\
\iff a+c\equiv b+c\pmod{m}.
\end{gather*}
Für die Subtraktion von $c$ ist die Überlegung analog.\;\qedsymbol

\newpage
\begin{Satz}\label{Kongruenz-mul}
Sind $a,b,c$ ganze Zahlen, dann gilt
\[a\equiv b\pmod{m} \implies ac\equiv bc\pmod{m}.\]
\end{Satz}
\strong{Beweis.}
Unter der Voraussetzung $a\equiv b\pmod{m}$ gibt es ein
$k$ mit $b-a=km$. Es gilt
\[b-a=km\iff (b-a)c=kcm \iff bc-ac=k'm\]
mit $k':=kc$. Man hat also
\[(\exists k'\in\Z)(bc-ac=k'm)\iff ac\equiv bc\pmod{m}.\;\qedsymbol\]

\begin{Satz}
Gilt $a\equiv a'\pmod{m}$ und
$b\equiv b'\pmod{m}$, dann gilt auch
\begin{align*}
a+b&\equiv a'+b'\pmod{m},\\
a-b&\equiv a'-b'\pmod{m},\\
ab&\equiv a'b'\pmod{m}.
\end{align*}
\end{Satz}
\strong{Beweis.} Man findet
\begin{equation}
\left.\begin{aligned}
a\equiv a'&\implies a+b\equiv a'+b\\
b\equiv b'&\implies a'+b\equiv a'+b'
\end{aligned}\right\}
\implies a+b\equiv a'+b\equiv a'+b'\pmod{m}.
\end{equation}
Für die Subtraktion ist die Überlegung analog. Für die Multiplikation
ebenfalls:%
\begin{equation}
\left.\begin{aligned}
a\equiv a'&\implies ab\equiv a'b\\
b\equiv b'&\implies a'b\equiv a'b'
\end{aligned}\right\}
\implies ab\equiv a'b\equiv a'b'\pmod{m}.\;\qedsymbol
\end{equation}

\begin{Satz}
Addition des Moduls führt auf eine kongruente Zahl:%
\[a\equiv a+m\equiv a-m\pmod{m}.\]
\end{Satz}
\strong{Beweis.}
Es gilt
\[a\equiv a+m\pmod{m}\iff (\exists k\in\Z)(km=(a+m)-a=m).\]
Setze $k=1$. Bei
\[a\equiv a-m\pmod{m}\iff (\exists k\in\Z)(km=(a-m)-a=-m)\]
setze $k=-1$.\;\qedsymbol

\section{Der Restklassenring}

Wir könnten nun beginnen, mit der Kongruenzenrechnung interessante
Probleme zu lösen. Zunächst möchte ich aber erläutern, wie die
Kongruenzenrechnung mit dem Restklassenring zusammenhängt. Unter
diesem Blickwinkel bekommen wir ein tieferes Verständnis und können
Mittel der Ringtheorie und Gruppentheorie anwenden.

Zu einer ganzen Zahl $a$ ist die Restklasse modulo $m$ definiert als 
\[[a]_m := \{x\mid x\equiv a\pmod m\}.\]
Eine alternative Schreibweise für $[a]_m$ ist $a+m\Z$. Weil die
Kongruenz eine Äquivalenzrelation ist, handelt es sich bei den
Restklassen um Äquivalenzklassen. Wir betrachten nun die
Quotientenmenge
\[\Z/m\Z := \{[a]_m\mid a\in\Z\}.\]
Nun können wir die Addition und Multiplikation von Restklassen
definieren.
\begin{Satz}
Auf $\Z/m\Z$ sind die beiden Operationen
\begin{align*}
[a]_m + [b]_m &:= [a+b]_m,\\
[a]_m\cdot [b]_m &:= [ab]_m
\end{align*}
wohldefiniert.
\end{Satz}
\strong{Beweis.} Zu zeigen ist, dass $a+b\equiv x+y$ gilt, sofern
$a\equiv x$ und $b\equiv y$ ist. Gemäß Satz \ref{Kongruenz-add-sub} gilt
\begin{align*}
a\equiv x &\iff a+b\equiv x+b,\\
b\equiv y &\iff x+b\equiv x+y.
\end{align*}
Aus den beiden Prämssen erhalten wir demzufolge
$a+b\equiv x+b\equiv x+y$.
Die Argumentation zur Multiplikation ist analog, wobei
Satz \ref{Kongruenz-mul} zur Anwendung kommt.\,\qedsymbol

Die Struktur $(\Z/m\Z,+,\cdot)$ nennt man den \emph{Restklassenring}
zum Modul $m$.

\begin{Lemma}\label{Surjektion-strukturerhaltend}
Sei $M$ eine Struktur mit einer Verknüpfung, von
Magma bis kommutative Gruppe. Ist
$\varphi\colon M\to M'$ eine strukturerhaltende Surjektion,
dergestalt dass $\varphi(ab)=\varphi(a)\varphi(b)$, dann ist
$M'$ von derselben Struktur und $\varphi$ ein Homomorphismus.
\end{Lemma}
\strong{Beweis.} Die Verknüpfung auf $M$ sei abgeschlossen. Weil
$\varphi$ surjektiv ist, gibt es zu $a',b'\in M'$ immer
$a,b\in M$ mit $a'=\varphi(a)$ und $b'=\varphi(b)$. Somit gilt
\[a'b' = \varphi(a)\varphi(b) = \varphi(ab)\in M'.\]
Die Verknüpfung auf $M$ erfülle das Assoziativgesetz. Dann gilt
\[(a'b')c' = \varphi(ab)\varphi(c) = \varphi(abc)
= \varphi(a)\varphi(bc) = a'(b'c').\]
Die Verknüpfung auf $M$ habe ein neutrales Element $e$. Dann gilt
\[\varphi(a)=\varphi(ea) = \varphi(e)\varphi(a),\quad
\varphi(a)=\varphi(ae)=\varphi(a)\varphi(e).\]
Demzufolge besitzt $M'$ mit $e':=\varphi(e)$ ein neutrales Element.

Zur Verknüpfung auf $M$ gebe es zu jedem Element ein inverses. Dann gilt
\[\varphi(e) = \varphi(aa^{-1}) = \varphi(a)\varphi(a^{-1}),\quad
\varphi(e) = \varphi(a^{-1}a) = \varphi(a^{-1})\varphi(a).\]
Demzufolge ist $\varphi(a)^{-1}=\varphi(a^{-1})$.

Die Verknüpfung auf $M$ sei kommutativ. Dann gilt
\[a'b' = \varphi(a)\varphi(b) = \varphi(ab) = \varphi(ba) = \varphi(b)\varphi(a) = b'a'.\]
Somit ist die Verknüpfung auf $M'$ kommutativ.\,\qedsymbol

\begin{Satz} Jeder Restklassenring ist ein kommutativer unitärer Ring.
\end{Satz}
\strong{Beweis.} Bereits bewiesen wurde, dass die ganzen Zahlen einen
kommutativen unitären Ring bilden. Wir betrachten nun die
Quotientenabbildung
\[\pi\colon\Z \to \Z/m\Z,\quad \pi(a):=[a]_m.\]
Die Operationen wurden so definiert dass
$\pi(a+b)=\pi(a)+\pi(b)$ und $\pi(ab)=\pi(a)\pi(b)$ gilt.
Gemäß Lemma \ref{Surjektion-strukturerhaltend} ist $(\Z/m\Z,+)$ also
eine kommutative Gruppe und $(\Z/m\Z,\cdot)$ ein kommutatives
Monoid. Es verbleibt noch das Distributivgesetz zu prüfen. Man rechnet
\begin{align*}
a'(b'+c') &= \pi(a)(\pi(b)+\pi(c)) = \pi(a)(\pi(b+c))
= \pi(a(b+c)) = \pi(ab+ac)\\
&= \pi(ab)+\pi(ac) = \pi(a)\pi(b)+\pi(a)\pi(c) = a'b'+a'c'.
\end{align*}
Damit ist der Satz gezeigt, und ferner ist gezeigt dass $\pi$ ein
Eins"=erhaltender Ringhomomorphismus ist.\,\qedsymbol


\setlength{\baselineskip}{12pt}
\printindex
\end{document}


