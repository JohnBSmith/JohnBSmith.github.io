\documentclass[a4paper,10pt,fleqn,twoside]{scrbook}
\usepackage[utf8]{inputenc}
\usepackage[T1]{fontenc}
\usepackage{arev}

\usepackage{ngerman}
\usepackage{amsmath}
\usepackage{amssymb}
\usepackage{amsthm}
\usepackage{mdframed}
% \usepackage{thmtools}

\usepackage[all]{xy}
\usepackage{enumitem}
\usepackage{graphicx}

\usepackage{color}
\definecolor{c1}{RGB}{0,40,80}
\definecolor{gray1}{RGB}{80,80,80}
\usepackage[colorlinks=true,linkcolor=c1]{hyperref}
\usepackage{geometry}
\geometry{a4paper,left=38mm,right=22mm,top=24mm,bottom=40mm}
\setlength{\columnsep}{4mm}
\numberwithin{equation}{chapter}
\setcounter{tocdepth}{2}

\newcommand{\N}{\mathbb N}
\newcommand{\Z}{\mathbb Z}
\newcommand{\Q}{\mathbb Q}
\newcommand{\R}{\mathbb R}
\newcommand{\C}{\mathbb C}
\newcommand{\A}{\mathbb A}
\newcommand{\id}{\operatorname{id}}
\renewcommand{\Re}{\operatorname{Re}}
\renewcommand{\Im}{\operatorname{Im}}
\newcommand{\ui}{\mathrm i}
\newcommand{\ee}{\mathrm e}
\newcommand{\sur}{\operatorname{sur}}
\newcommand{\Abb}{\operatorname{Abb}}
\newcommand{\sgn}{\operatorname{sgn}}
\newcommand{\defiff}{\;:\Longleftrightarrow\;}
\newcommand{\strong}[1]{{\normalfont\sffamily\bfseries #1}}
\newcommand{\comp}{\mathrm c}

\newcommand{\emdef}[1]{\emph{#1}}
\newcommand{\pstrut}[1]{\rule{0pt}{\dimexpr 8pt+#1}}
\newcommand{\bitem}{\item[\scriptsize\color{gray1}$\blacksquare$]}
\renewcommand{\qedsymbol}{\ensuremath{\square}}

\newtheoremstyle{rmbox}%
  {0pt}% space above
  {0pt}% space below
  {}% bodyfont
  {}% indent
  {\bfseries}% head font
  {\;}% punctuation between head and body
  {0pt}% space after theorem head
  {\thmname{#1}\;\thmnumber{#2}\thmnote{\;(#3)}.}

\theoremstyle{rmbox}
\newtheorem{Definition}{Definition}[chapter]
\newtheorem{Satz}{Satz}[chapter]
\newtheorem{Lemma}[Satz]{Lemma}
\newtheorem{Korollar}[Satz]{Korollar}

\definecolor{greenblue}{rgb}{0.0,0.32,0.4}
\definecolor{grayblue}{rgb}{0.2,0.2,0.4}

\surroundwithmdframed[topline=false,rightline=false,bottomline=false,%
  linecolor=greenblue, linewidth=4pt, innerleftmargin=7pt,%
  innertopmargin=2pt, innerbottommargin=4pt,%
  innerrightmargin=0pt%
]{Definition}

\newcommand{\framedtheorem}[1]{%
\surroundwithmdframed[topline=false,rightline=false,bottomline=false,%
  linecolor=grayblue, linewidth=4pt, innerleftmargin=7pt,%
  innertopmargin=2pt, innerbottommargin=4pt,%
  innerrightmargin=0pt%
]{#1}}

\framedtheorem{Satz}
\framedtheorem{Lemma}
\framedtheorem{Korollar}

\newenvironment{Beweis}[1][Beweis]%
  {\par\noindent\strong{#1.}\;\ignorespaces}%
  {\par\addvspace{\topsep}}

\usepackage{makeidx}
\makeindex

\title{Beweisarchiv}
\date{Mai 2021}

\begin{document}

\maketitle

% \vfill
\texttt{Dieses Heft steht unter der Lizenz Creative Commons CC0.}

\tableofcontents


\chapter{Grundlagen}
\section{Arithmetik}
\subsection{Zahlenbereiche}
Natürliche Zahlen ab null:
\begin{equation}
\N_0 := \{0,1,2,3,4,\ldots\}.
\end{equation}
Natürliche Zahlen ab eins:
\begin{equation}
\N_1 := \{1,2,3,4,5,\ldots\}.
\end{equation}
Natürliche Zahlen:
\begin{equation}
\begin{split}
&\text{$\N$, wenn es keine Rolle spielt,}\\
&\text{ob $\N:=\N_0$ oder $\N:=\N_1$}.
\end{split}
\end{equation}
Ganze Zahlen:
\begin{equation}
\Z := \{\ldots,-3,-2,-1,0,1,2,3,\ldots\}.
\end{equation}
Rationale Zahlen:
\begin{equation}
\Q := \{\tfrac{z}{n}\mid z\in\Z,n\in\N_0\}.
\end{equation}
Reelle Zahlen:
\begin{equation}
\R := \overline{\Q}\enspace\text{bezüglich}\; d(x,y)=|x-y|.
\end{equation}
Positive reelle Zahlen:
\begin{equation}
\R^+ := \{x\in\R\mid x>0\}.
\end{equation}
Nichtnegative reelle Zahlen:
\begin{equation}
\R_0^+ := \{x\in\R\mid x\ge 0\}.
\end{equation}
Negative reelle Zahlen:
\begin{equation}
\R^- := \{x\in\R\mid x<0\}.
\end{equation}
Nichtpositive reelle Zahlen:
\begin{equation}
\R_0^- := \{x\in\R\mid x\le 0\}.
\end{equation}
Komplexe Zahlen:
\begin{equation}
\C := \{a+b\ui\mid a,b\in\R\}.
\end{equation}
Quaternionen:
\begin{equation}
\mathbb H := \{a+b\ui+c\uj+d\uk\mid a,b,c,d\in\R\}.
\end{equation}
Algebraische Zahlen:
\begin{equation}
\mathbb A := \{a\in\C\mid \exists P{\in}\Q[X]\colon P(a)=0\}.
\end{equation}
Irrationale Zahlen:
\begin{equation}
\R\setminus\Q = \{\sqrt{2},\sqrt{3},\pi,\ee,\ldots\}.
\end{equation}
Transzendente Zahlen:
\begin{equation}
\R\setminus\mathbb A = \{\pi,\ee,\ldots\}.
\end{equation}
Es gelten die folgenden Teilmengenbeziehungen:
\begin{equation}
\N\subset\Z\subset\Q\subset\R\subset\C\subset\mathbb H.
\end{equation}
Es gilt die folgende Abstufung der Mächtigkeit:
\begin{equation}
|\N| = |\Z| = |\Q| = |\mathbb A| < |\R| = |\C|.
\end{equation}

\newpage
\subsection{Intervalle}
Abgeschlossene Intervalle:
\begin{equation}
[a,b] := \{x\in\R\mid a\le x\le b\}.
\end{equation}
Offene Intervalle:
\begin{equation}
(a,b) := \{x\in\R\mid a<x<b\}.
\end{equation}
Halboffene Intervalle:
\begin{align}
(a,b] &:= \{x\in\R\mid a<x\le b\},\\
[a,b) &:= \{x\in\R\mid a\le x<b\}.
\end{align}
Unbeschränkte Intervalle:
\begin{align}
[a,\infty) &:= \{x\in\R\mid a\le x\},\\
(a,\infty) &:= \{x\in\R\mid a<x\},\\
(-\infty,b] &:= \{x\in\R\mid x\le b\},\\
(-\infty,b) &:= \{x\in\R\mid x<b\}.
\end{align}

\subsection{Summen}
\begin{definition}[Summe]
Für eine Folge $(a_n)$:
\begin{align}
\sum_{k=m}^{m-1} a_k &:= 0,\qquad(\text{leere Summe})\\
\sum_{k=m}^n a_k &:= a_n+\sum_{k=m}^{n-1} a_k.\qquad(n\ge m)
\end{align}
\end{definition}
\noindent
Für eine Konstante $c$ gilt:
\begin{equation}
\sum_{k=m}^n c = (n-m+1)\,c.
\end{equation}
Der Summierungsoperator ist linear:
\begin{align}
\sum_{k=m}^n (a_k+b_k) &= \sum_{k=m}^n a_k + \sum_{k=m}^n b_k,\\
\sum_{k=m}^n ca_k &= c\sum_{k=m}^n a_k.
\end{align}
Indexverschiebung ist möglich:
\begin{equation}
\sum_{k=m}^n a_k = \sum_{k=m-j}^{n-j} a_{k+j} = \sum_{k=m+j}^{n+j} a_{k-j}.
\end{equation}
Aufspaltung ist möglich:
\begin{equation}
\sum_{k=m}^n a_k = \sum_{k=m}^p a_k + \sum_{k=p+1}^n a_k.
\end{equation}
Vertauschung der Reihenfolge bei Doppelsummen:
\begin{equation}
\sum_{i=p}^m \sum_{j=q}^n a_{ij} = \sum_{j=q}^n \sum_{i=p}^m a_{ij}.
\end{equation}

\subsection{Produkte}
\begin{definition}[Produkt]
Für eine Folge $(a_n)$:
\begin{align}
\prod_{k=m}^{m-1} a_k &:= 1,\qquad(\text{leeres Produkt})\\
\prod_{k=m}^n a_k &:= a_n\prod_{k=m}^{n-1} a_k.\qquad(n\ge m)
\end{align}
\end{definition}

\noindent
Für eine Konstante $c$ gilt:
\begin{equation}
\prod_{k=m}^n c = c^{n-m+1}.
\end{equation}
Unter Voraussetzung des Kommutativgesetzes gilt
\begin{align}
\label{eq:product-product}
\prod_{k=m}^n (a_k b_k) &= \bigg(\prod_{k=m}^n a_k\bigg)\bigg(\prod_{k=m}^n b_k\bigg),\\
\label{eq:product-power}
\prod_{k=m}^n a_k^c &= \bigg(\prod_{k=m}^n a_k\bigg)\Big.^c.\qquad (c\in\N_0)
\end{align}

Formel \eqref{eq:product-power} gilt auch für $a_k\in\R^+$ und $c\in\C$.

Formel \eqref{eq:product-product} ist ein Spezialfall von
\begin{equation}
\prod_{i=p}^m \prod_{j=q}^n a_{ij} = \prod_{j=q}^n \prod_{i=p}^m a_{ij}.
\end{equation}
Indexverschiebung ist möglich:
\begin{equation}
\prod_{k=m}^n a_k = \prod_{k=m-j}^{n-j} a_{k+j} = \prod_{k=m+j}^{n+j} a_{k-j}.
\end{equation}
Aufspaltung ist möglich:
\begin{equation}
\prod_{k=m}^n a_k = \bigg(\prod_{k=m}^p a_k\bigg)\bigg(\prod_{k=p+1}^n a_k\bigg).
\end{equation}
Für $a_k\in\R^+$ gilt
\begin{equation}
\prod_{k=m}^n a_k = \exp\bigg(\sum_{k=m}^n \ln(a_k)\bigg).
\end{equation}

\subsection{Binomischer Lehrsatz}\index{binomischer Lehrsatz}
Sei $R$ ein unitärer Ring, z.\,B. $R=\R$ oder $R=\C$.\\
Für $a,b\in R$ mit $ab=ba$ gilt:%
\begin{equation}
(a+b)^n = \sum_{k=0}^n \binom{n}{k} a^{n-k} b^k
\end{equation}
und
\begin{equation}
(a-b)^n = \sum_{k=0}^n \binom{n}{k} (-1)^k a^{n-k} b^k.
\end{equation}
Die ersten Formeln sind:\index{binomische Formeln}
\begin{gather}
(a+b)^2 = a^2+2ab+b^2,\\
(a-b)^2 = a^2-2ab+b^2,\\
(a+b)^3 = a^3+3a^2 b+3ab^2+b^3,\\
(a-b)^3 = a^3-3a^2 b+3ab^2-b^3,\\
(a+b)^4 = a^4+4a^3 b+6a^2 b^2+4ab^3+b^4,\\
(a-b)^4 = a^4-4a^3 b+6a^2 b^2-4ab^3+b^4.
\end{gather}
\subsection{Potenzgesetze}
\begin{definition}[Potenz]
Für $a$ aus einem Monoid und $n\in\Z,n\ge 1$:
\begin{align}
a^0 &:= 1,\\
a^n &:= a^{n-1}\cdot a.
\end{align}

Für $a\in\R, a>0$ und $x\in\C$:
\begin{equation}
a^x := \exp(\ln(a)\,x).
\end{equation}
\end{definition}
\noindent
Für $a\in\R, a>0$ und $x,y\in\C$ gilt:
\begin{gather}
a^{x+y} = a^x a^y,\quad a^{x-y} = \frac{a^x}{a^y},
\quad a^{-x} = \frac{1}{a^x}.
\end{gather}

\section{Gleichungen}
\begin{definition}[Bestimmungsgleichung]
Sind $f,g$ auf der Grundmenge $G$ definierte Funktionen, so nennt
man
\begin{equation}
f(x) = g(x)\\
\end{equation}
eine \emdef{Bestimmungsgleichung}\index{Bestimmungsgleichung},
wenn die Lösungemenge
\begin{equation}
L = \{x\in G\mid f(x)=g(x)\}
\end{equation}
gesucht ist.
\end{definition}
Bei den $x\in G$ kann es sich auch um Tupel $x=(x_1,x_2)$ oder
$x=(x_1,x_2,x_3)$ usw. handeln. Man spricht in diesem Fall
von einer Gleichung \emdef{in mehreren Variablen}.

Handelt es sich bei den Funktionswerten von $f,g$ um Tupel,
dann spricht man von einem
\emdef{Gleichungssystem}\index{Gleichungssystem}.

\subsection{Äquivalenzumformungen}

Äquivalenzumformungen lassen die Lösungsmenge einer Gleichung
unverändert. Seien $A(x),B(x)$ zwei Aussageformen bzw. zwei
Gleichungen. Aus
\begin{equation}
\forall x{\in}G\,[A(x)\Longleftrightarrow B(x)]
\end{equation}
folgt
\begin{equation}
\{x\in G\mid A(x)\} = \{x\in G\mid B(x)\}.
\end{equation}
Aus
\begin{equation}
\forall x{\in} G\,[A(x)\Longrightarrow B(x)]
\end{equation}
folgt jedoch nur noch
\begin{equation}
\{x\in G\mid A(x)\}\subseteq\{x\in G\mid B(x)\}.
\end{equation}
Seien $f,g,h$ Funktionen mit Definitionsmenge $G$ und
Zielmenge $Z=\R$ oder $Z=\C$.

Für alle $x$ gilt:
\begin{align}
f(x)=g(x) &\Longleftrightarrow f(x){+}h(x)=g(x){+}h(x),\\
f(x)=g(x) &\Longleftrightarrow f(x){-}h(x)=g(x){-}h(x).
\end{align}
Besitzt $h(x)$ keine Nullstellen, dann gilt für alle $x$:
\begin{align}
f(x)=g(x) &\iff f(x)h(x)=g(x)h(x),\\
f(x)=g(x) &\iff \frac{f(x)}{h(x)}=\frac{f(x)}{h(x)}.
\end{align}
Besitzt $h(x)$ aber Nullstellen, dann gilt immerhin noch für alle $x$:
\begin{equation}
f(x)=g(x) \implies f(x)h(x)=g(x)h(x).
\end{equation}
Sei $f,g\colon G\to Z$. Sei $\varphi_x\colon Z\to Z'$ eine Injektion
für jedes $x\in G$. Es gilt
\begin{equation}
f(x)=g(x) \iff \varphi_x(f(x))=\varphi_x(g(x))
\end{equation}
für alle $x\in G$.

Bei einer Kette von Äquivalenzumformungen wird links das
Äquivalenzzeichen geschrieben, in der Mitte die Gleichung
und rechts hinter einem senkrechten Strich die Operation
$\varphi_x(w)$, welche als nächstes auf beide Seiten der Gleichung
angwendet werden soll.

Beispiel:
\begin{equation*}\setlength{\arraycolsep}{2pt}
\begin{array}{rrl@{\qquad}l}
& 2x+4 &= 2x^2-8x+2 &\mid w/2\\[2pt]
\Longleftrightarrow& x+2 &= x^2-4x+1 &\mid w-2\\[2pt]
\Longleftrightarrow& x &= x^2-4x-1 &\mid w-x\\[2pt]
\Longleftrightarrow& 0 &= x^2-7x-1.
\end{array}
\end{equation*}
Am Anfang befinden sich eventuell Bedingungen für $x$.
Bei Fallunterscheidungen wird eine Verschärfung der Bedingungen
vorgenommen, so dass es zur Verkleinerung der Grundmenge kommt.
Nach einer Fallunterscheidung ergeben sich unter Umständen neue
Injektionen.

\subsection{Quadratische Gleichungen}
\begin{definition}[Quadratische Gleichung]
Eine Gleichung der Form $ax^2+bx+c=0$ mit $a\ne 0$ heißt
\emdef{quadratische Gleichung}.
\end{definition}

Wegen $a\ne 0$ lässt sich die Gleichung durch $a$ dividieren
und es ensteht die äquivalente Normalform $x^2+px+q=0$
mit $p:=b/a$ und $q:=c/a$.

\strong{Lösung.}
Seien nun die $a,b,c$ reelle Zahlen. Die Zahl
\begin{equation}
D = p^2-4q
\end{equation}
heißt \emdef{Diskriminante}. Für $D>0$ gibt es zwei reelle Lösungen:
\begin{align}
x_1 &= \frac{-p-\sqrt{D}}{2} = \frac{-b-\sqrt{b^2-4ac}}{2a},\\
x_2 &= \frac{-p+\sqrt{D}}{2} = \frac{-b+\sqrt{b^2-4ac}}{2a}.
\end{align}
Für $D=0$ fallen beiden Lösungen zu einer \emdef{doppelten Lösung}
zusammen:
\begin{equation}
x_1 = x_2 = -\frac{p}{2} = -\frac{b}{2a}.
\end{equation}
Für $D<0$ gibt es keine reelle Lösung. Aber es gibt zwei komplexe
Lösungen, die zueinander konjugiert sind:
\begin{equation}
x_1 = \frac{-p-\ui\sqrt{|D|}}{2},\quad
x_2 = \frac{-p+\ui\sqrt{|D|}}{2}.
\end{equation}
In jedem Fall gelten die Formeln von Vieta:
\begin{equation}
p = -(x_1+x_2),\qquad q = x_1 x_2.
\end{equation}

\section{Komplexe Zahlen}\index{komplexe Zahl}
\subsection{Rechenoperationen}

\begin{gather}
\frac{z_1}{z_2}
= \frac{z_1\overline z_2}{z_2\overline z_2}
= \frac{z_1\overline z_2}{|z_2|^2},\\
\frac{1}{z} = \frac{\overline z}{z\overline z}
= \frac{\overline z}{|z|^2}.
\end{gather}

\subsection{Betrag}\index{Betrag!einer komplexen Zahl}
Für alle $z_1,z_2\in\C$ gilt:
\begin{gather}
|z_1z_2| = |z_1|\,|z_2|,\\
z_2\ne 0\implies \Big|\frac{z_1}{z_2}\Big|
= \frac{|z_1|}{|z_2|},\\
z\,\overline z = |z|^2.
\end{gather}

\subsection{Konjugation}\index{Konjugation!einer komplexen Zahl}
Für alle $z_1,z_2\in\C$ gilt:
\begin{gather}
\overline{z_1+z_2} = \overline z_1+\overline z_2,\qquad
\overline{z_1-z_2} = \overline z_1-\overline z_2,\\
\overline{z_1 z_2} = \overline z_1\,\overline z_2,\qquad
z_2\ne 0 \implies \overline{\Big(\frac{z_1}{z_2}\Big)}
= \frac{\overline z_1}{\overline z_2},\\
\overline{\overline z}=z,\qquad
|\overline{z}| = |z|,\qquad
z\,\overline z = |z|^2,\\
\real(z) = \frac{z+\overline z}{2},\qquad
\imag(z) = \frac{z-\overline z}{2\ui},\\
\overline{\cos(z)} = \cos(\overline z),\qquad
\overline{\sin(z)} = \sin(\overline z),\\
\overline{\exp(z)} = \exp(\overline z).
\end{gather}

\begin{table*}[t]
\caption{Rechnen mit komplexen Zahlen}
\bgroup
\def\arraystretch{1.4}
\begin{tabular}{|l|r|l|l|}
\hline
  \thbf{Name}
& \thbf{Operation}
& \thbf{Polarform}
& \thbf{kartesische Form}\\
\hline
  Identität
& $z$ & $=r\ee^{\ui\varphi}$
& $= a+b\ui$\\
\hline
  Addition
& $z_1+z_2$ &
& $= (a_1+a_2)+(b_1+b_2)\ui$\\
\hline
  Subtraktion
& $z_1-z_2$ &
& $= (a_1-a_2)+(b_1-b_2)\ui$\\
\hline
  Multiplikation
& $z_1 z_2$
& $= r_1 r_2 \ee^{\ui(\varphi_1+\varphi_2)}$
& $= (a_1 a_2 - b_1 b_2)+(a_1 b_2+a_2 b_1)\ui$\\
\hline
  Division
& $\displaystyle\frac{z_1}{z_2}$
& $\displaystyle =\frac{r_1}{r_2}\ee^{\ui(\varphi_1-\varphi_2)}$
& $\displaystyle =\frac{a_1 a_2 + b_1 b_2}{a_2^2+b_2^2}
   + \frac{a_2 b_1 - a_1 b_2}{a_2^2+b_2^2}\ui$\\
\hline
  Kehrwert
& $\displaystyle\frac{1}{z}$
& $\displaystyle =\frac{1}{r}\ee^{-\ui\varphi}$
& $\displaystyle =\frac{a}{a^2+b^2}-\frac{b}{a^2+b^2}\ui$\\
\hline
  Realteil
& $\real(z)$
& $=\cos\varphi$
& $=a$\\
\hline
  Imaginärteil
& $\imag(z)$
& $=\sin\varphi$
& $=b$\\
\hline
  Konjugation
& $\overline{z}$
& $=r\ee^{-\varphi\ui}$
& $=a-b\ui$\\
\hline
Betrag
& $|z|$
& $=r$
& $=\sqrt{a^2+b^2}$\\
\hline
  Argument
& $\arg(z)$
& $=\varphi$
& $\displaystyle = s(b)\arccos\Big(\frac{a}{r}\Big)$\\
\hline
\end{tabular}
\egroup\\
\\
$s(b):=\begin{cases}
+1 & \text{if}\;b\ge 0,\\
-1 & \text{if}\;b<0
\end{cases}$
\end{table*}

\section{Logik}
\subsection{Aussagenlogik}\index{Aussagenlogik}
\subsubsection{Boolesche Algebra}\index{boolesche Algebra}
\begin{table*}[t]
\caption{Boolesche Algebra}
\begin{tabular}{l|l|l}
\thbf{Disjunktion} & \thbf{Konjunktion} &\\
  $A\lor A \Leftrightarrow A$
& $A\land A \Leftrightarrow A$
& Idempotenzgesetze\\
  $A\lor 0 \Leftrightarrow A$
& $A\land 1 \Leftrightarrow A$
& Neutralitätsgesetze\\
  $A\lor 1 \Leftrightarrow 1$
& $A\land 0 = 0$
& Extremalgesetze\\
  $A\lor \overline A \Leftrightarrow 1$
& $A\land \overline A \Leftrightarrow 0$
& Komplementärgesetze\\
\noalign{\vspace{1em}}
  $A\lor B \Leftrightarrow B\lor A$
& $A\land B \Leftrightarrow B\land A$
& Kommutativgesetze\\
  $(A\lor B)\lor C \Leftrightarrow A\lor (B\lor C)$
& $(A\land B)\land C \Leftrightarrow A\land (B\land C)$
& Assoziativgesetze\\
  $\overline{A\lor B} \Leftrightarrow \overline A\land\overline B$
& $\overline{A\land B} \Leftrightarrow \overline A\lor\overline B$
& De Morgansche Regeln\\
  $A\lor (A\land B) \Leftrightarrow A$
& $A\land (A\lor B) \Leftrightarrow A$
& Absorptionsgesetze\\
\end{tabular}
\end{table*}

\noindent
\strong{Distributivgesetze}:
\begin{gather}
A\lor (B\land C) \iff (A\lor B)\land (A\lor C),\\
A\land (B\lor C) \iff (A\land B)\lor (A\land C).
\end{gather}

\subsubsection{Zweistellige Funktionen}
Es gibt 16 zweistellige boolesche\\
Funktionen.

\begin{tabular}{r|l}
\textbf{\texttt{AB}} & \thbf{Wert}\\
\texttt{00} & \texttt{a}\\
\texttt{01} & \texttt{b}\\
\texttt{10} & \texttt{c}\\
\texttt{11} & \texttt{d}
\end{tabular}

\begin{tabular}{r|l|l|l}
\thbf{Nr.}& \textbf{\texttt{dcba}} & \thbf{Fkt.} & \thbf{Name}\\
 0 & \texttt{0000} & 0 & Kontradiktion\\
 1 & \texttt{0001} & $\overline{A\lor B}$ & NOR\\
 2 & \texttt{0010} & $\overline{B\Rightarrow A}$\\
 3 & \texttt{0011} & $\overline A$\\
 4 & \texttt{0100} & $\overline{A\Rightarrow B}$\\
 5 & \texttt{0101} & $\overline{B}$\\
 6 & \texttt{0110} & $A\oplus B$ & Kontravalenz\index{Kontravalenz}\\
 7 & \texttt{0111} & $\overline{A\land B}$ & NAND\\
 8 & \texttt{1000} & $A\land B$ & Konjunktion\index{Konjunktion}\\
 9 & \texttt{1001} & $A\Leftrightarrow B$ & Äquivalenz\\
10 & \texttt{1010} & $B$ & Projektion\\
11 & \texttt{1011} & $A\Rightarrow B$ & Implikation\\
12 & \texttt{1100} & $A$ & Projektion\\
13 & \texttt{1101} & $B\Rightarrow A$ & Implikation\\
14 & \texttt{1110} & $A\lor B$ & Disjunktion\index{Disjunktion}\\
15 & \texttt{1111} & $1$ & Tautologie
\end{tabular}

\subsubsection[Darstellung mit Negation, Konjunktion und Disjunktion]%
{Darstellung mit Negation,\newline Konjunktion und Disjunktion}
\begin{gather}\label{eq:implication-definition}
A\Rightarrow B \iff \overline A\lor B,\\
(A\Leftrightarrow B) \iff
  (\overline A\land\overline B)\lor(A\land B),\\
A\oplus B \iff (\overline A\land B)\lor(A\land\overline B).
\end{gather}

\subsubsection{Tautologien}
Modus ponens:
\begin{equation}\label{eq:modus-ponens}
(A\Rightarrow B)\land A\implies B.
\end{equation}
Modus tollens:
\begin{equation}
(A\Rightarrow B)\land\overline B\implies\overline A.
\end{equation}
Modus tollendo ponens:
\begin{equation}
(A\lor B)\land\overline A \implies B.
\end{equation}
Modus ponendo tollens:
\begin{equation}
\overline{A\land B}\land A\implies\overline B.
\end{equation}
Kontraposition:\index{Kontraposition}
\begin{equation}
A\Rightarrow B \iff \overline B\Rightarrow \overline A.
\end{equation}
Beweis durch Widerspruch:\index{Widerspruch}
\begin{equation}
(\overline A\Rightarrow B\land\overline B)\implies A.
\end{equation}
Zerlegung einer Äquivalenz:
\begin{equation}
(A\Leftrightarrow B) \iff (A\Rightarrow B)\land(B\Rightarrow A).
\end{equation}
Kettenschluss:
\begin{equation}
(A\Rightarrow B)\land(B\Rightarrow C)\implies (A\Rightarrow C).
\end{equation}
Ringschluss:
\begin{equation}
\begin{split}
&(A\Rightarrow B)\land (B\Rightarrow C)\land(C\Rightarrow A)\\
&\implies (A\Leftrightarrow B)\land(A\Leftrightarrow C)\land(B\Leftrightarrow C).
\end{split}
\end{equation}
Ringschluss, allgemein:
\begin{equation}
\begin{split}
& (A_1{\Rightarrow }A_2)\land\ldots\land(A_{n-1}{\Rightarrow}A_n)
\land(A_n{\Rightarrow}A_1)\\
& \implies \forall i,j\,[A_i\Leftrightarrow A_j].
\end{split}
\end{equation}
Ersetzungsregel:

Für jede Funktion $P\colon\{0,1\}\to\{0,1\}$ gilt:
\begin{equation}
P(A)\land (A\Leftrightarrow B)\implies P(B).
\end{equation}
Regel zur Implikation:
\begin{equation}
A\land B\Rightarrow C \iff A\Rightarrow (B\Rightarrow C).
\end{equation}
Vollständige Fallunterscheidung:
\begin{gather}
(A\Rightarrow C)\land (B\Rightarrow C)\implies (A\oplus B\Rightarrow C),\\
(A\Rightarrow C)\land (B\Rightarrow C)\iff (A\lor B\Rightarrow C).
\end{gather}
Vollständige Fallunterscheidung, allgemein:
\begin{gather}
\textstyle \forall k[A_k\Rightarrow C]
\implies (\bigoplus_{k=1}^n A_k\Rightarrow C),\\
\forall k[A_k\Rightarrow C]
\iff (\exists k[A_k]\Rightarrow C).
\end{gather}

\subsubsection{Schlussregeln}
\strong{Ersetzungsregel.} Sei $p(\varphi)$ eine aussagenlogische
Formel in expliziter Abhängigkeit von der Formelvariablen $\varphi$.
Es gilt
\begin{equation}
\{p(\varphi),\varphi\leftrightarrow\psi\}\vdash p(\psi).
\end{equation}
\strong{Beispiel.} Betrachte $\varphi\land A\rightarrow B$ mit
$\varphi:=(A\rightarrow B)$, was expandiert wird zu
\[(A\rightarrow B)\land A\rightarrow B.\qquad\text{(s. \eqref{eq:modus-ponens})}\]
Nun gilt nach \eqref{eq:implication-definition} aber
\[A\rightarrow B\leftrightarrow \overline A\lor B.\]
Daher lässt sich folgern:
\[(\overline A\lor B)\land A\rightarrow B.\]

\subsubsection{Metatheoreme}
\strong{Korrektheit der Aussagenlogik.}

Für die Aussagenlogik gilt:
\begin{equation}
(\Gamma\vdash\psi)\implies (\Gamma\models\psi).
\end{equation}

\noindent
\strong{Vollständigkeit der Aussagenlogik.}

Für die Aussagenlogik gilt:
\begin{equation}
(\Gamma\models\psi)\implies (\Gamma\vdash\psi).
\end{equation}

\noindent
\strong{Deduktionstheorem (syntaktisch).}

Für die Aussagenlogik gilt:
\begin{equation}
(\Gamma\cup\{\varphi\}\vdash\psi)\iff (\Gamma\vdash\varphi\rightarrow\psi).
\end{equation}

Infolge gilt auch:
\begin{equation}
\begin{split}
&(\{\varphi_1,\ldots,\varphi_n\}\vdash\psi)\\
&\iff (\vdash \varphi_1\land\ldots\land\varphi_n\rightarrow\psi).
\end{split}
\end{equation}

\noindent
\strong{Deduktionstheorem (semantisch).}

Für die Aussagenlogik gilt:
\begin{equation}
(\Gamma\cup\{\varphi\}\models\psi)\iff (\Gamma\models\varphi\rightarrow\psi).
\end{equation}

Infolge gilt auch:
\begin{equation}
\begin{split}
&(\{\varphi_1,\ldots,\varphi_n\}\models\psi)\\
&\iff (\models \varphi_1\land\ldots\land\varphi_n\rightarrow\psi).
\end{split}
\end{equation}

\noindent
\strong{Einsetzungsregel.}

Sei $v$ eine metasprachliche Variable, die für eine beliebige
objektsprachliche Variable steht. Dann gilt:
\begin{equation}
(\models\varphi) \implies (\models\varphi[v:=\psi]).
\end{equation}
D.\,h. wenn in der tautologischen Formel $\varphi$ jedes auftreten
der Variable $v$ gegen die Formel $\psi$ ersetzt wird, ergibt
sich wieder eine tautologische Formel.

%\newpage
\subsection{Prädikatenlogik}
\subsubsection{Rechenregeln}
Verneinung (De Morgansche Regeln):
\begin{gather}
\overline{\forall x[P(x)]}\iff \exists x[\overline{P(x)}],\\
\overline{\exists x[P(x)]}\iff \forall x[\overline{P(x)}].
\end{gather}
Verallgemeinerte Distributivgesetze:
\begin{gather}
P\lor\forall x[Q(x)] \iff \forall x[P\lor Q(x)],\\
P\land\exists x[Q(x)] \iff \exists x[P\land Q(x)].
\end{gather}
Verallgemeinerte Idempotenzgesetze:
\begin{gather}
\begin{split}
\exists x{\in}M\,[P] & \iff
(M\ne\{\})\land P\\
& \iff\begin{cases}
P & \text{wenn}\; M\ne\{\},\\
0 & \text{wenn}\; M=\{\}.
\end{cases}
\end{split}\\
\begin{split}
\forall x{\in}M\,[P]& \iff
(M=\{\})\lor P\\
&\iff\begin{cases}
P & \text{wenn}\; M\ne\{\},\\
1 & \text{wenn}\; M=\{\}.
\end{cases}
\end{split}
\end{gather}
%\newpage\noindent
Äquivalenzen:
\begin{gather}
\hspace{-2em}\forall x\forall y[P(x,y)] \iff \forall y\forall x[P(x,y)],\\
\hspace{-2em}\exists x\exists y[P(x,y)] \iff \exists y\exists x[P(x,y)],\\
\hspace{-2em}\forall x[P(x)\land Q(x)] \iff \forall x[P(x)]\land\forall x[Q(x)],\\
\hspace{-2em}\exists x[P(x)\lor Q(x)] \iff \exists x[P(x)]\lor\exists x[Q(x)],\\
\hspace{-2em}\forall x[P(x)\Rightarrow Q] \iff \exists x[P(x)]\Rightarrow Q,\\
\hspace{-2em}\forall x[P\Rightarrow Q(x)] \iff P\Rightarrow\forall x[Q(x)],\\
\hspace{-2em}\exists x[P(x)\Rightarrow Q(x)]
  \iff\forall x[P(x)]\Rightarrow\exists x[Q(x)].
\end{gather}
% \newpage\noindent
Implikationen:
\begin{gather}
\hspace{-2em}\exists x\forall y[P(x,y)]\implies \forall y\exists x[P(x,y)],\\
\hspace{-2em}\forall x[P(x)]\lor\forall x[Q(x)]\implies\forall x[P(x)\lor Q(x)],\\
\hspace{-2em}\exists x[P(x)\land Q(x)]\implies
  \exists x[P(x)]\land \exists x[Q(x)],\\
\hspace{-2em}\forall x[P(x)\Rightarrow Q(x)]\implies
  (\forall x[P(x)]\Rightarrow\forall x[Q(x)]),\\
\hspace{-2em}\forall x[P(x)\Leftrightarrow Q(x)]\implies
  (\forall x[P(x)]\Leftrightarrow\forall x[Q(x)]).
\end{gather}

\newpage
\subsubsection{Endliche Mengen}
Sei $M=\{x_1,\ldots,x_n\}$. Es gilt:
\begin{gather}
\forall x{\in}M\,[P(x)]\iff P(x_1)\land\ldots\land P(x_n),\\
\exists x{\in}M\,[P(x)]\iff P(x_1)\lor\ldots\lor P(x_n).
\end{gather}

\subsubsection{Beschränkte Quantifizierung}
\begin{gather}
\begin{split}
& \forall x{\in}M\,[P(x)]\defiff\forall x[x\notin M\lor P(x)]\\
& \quad\iff\forall x[x\in M\Rightarrow P(x)],
\end{split}\\
\exists x{\in}M\,[P(x)]\defiff\exists x[x\in M\land P(x)],\\
\forall x{\in}M{\setminus}N\,[P(x)]\iff \forall x[x\notin N\Rightarrow P(x)].
\end{gather}

\subsubsection[Quantifizierung über Produktmengen]%
{Quantifizierung über\newline Produktmengen}
\begin{gather}
\forall(x,y)\,[P(x,y)]\iff \forall x\forall y[P(x,y)],\\
\exists(x,y)\,[P(x,y)]\iff \exists x\exists y[P(x,y)].
\end{gather}
Analog gilt
\begin{gather}
\forall(x,y,z)\,\iff \forall x\forall y\forall z,\\
\exists(x,y,z)\,\iff \exists x\exists y\exists z
\end{gather}
usw.

\subsubsection{Alternative Darstellung}
Sei $P\colon G\to\{0,1\}$ und $M\subseteq G$.
Mit $P(M)$ ist die Bildmenge von $P$ bezüglich $M$ gemeint.
Es gilt
\begin{equation}
\begin{split}
&\forall x{\in}M\,[P(x)] \iff P(M)=\{1\}\\
& \iff M\subseteq\{x{\in}G\mid P(x)\}
\end{split}
\end{equation}
und
\begin{equation}
\begin{split}
& \exists x{\in}M\,[P(x)] \iff \{1\}\subseteq P(M)\\
& \iff M\cap\{x{\in}G\mid P(x)\}\ne\{\}.
\end{split}
\end{equation}

\subsubsection{Eindeutigkeit}
Quantor für eindeutige Existenz:
\begin{equation}
\begin{split}
&\exists!x\,[P(x)]\\
&:\Longleftrightarrow\; \exists x\,[P(x)\land \forall y\,[P(y)\Rightarrow x=y]]\\
&\iff \exists x\,[P(x)]\land \forall x\forall y[P(x)\land P(y)\Rightarrow x=y].
\end{split}
\end{equation}

\newpage
\section{Mengenlehre}
\subsection{Definitionen}
Aufzählende Notation:
\begin{equation}
\hspace{-1em} a\in\{x_1,\ldots,x_n\} :\Leftrightarrow a=x_1\lor\ldots\lor a=x_n.
\end{equation}
Beschreibende Notation:
\begin{gather}
a\in\{x\mid P(x)\}\defiff P(a),\\
\{x\in M\mid P(x)\} := \{x\mid x\in M\land P(x)\},\\
\hspace{-1em}\{f(x)\mid P(x)\} := \{y\mid \exists x(y=f(x)\land P(x))\}.
\end{gather}
Teilmengenrelation:
\begin{equation}
A\subseteq B\defiff \forall x\,(x\in A\implies x\in B).
\end{equation}
Gleichheit:
\begin{equation}
A=B\defiff \forall x\,(x\in A\iff x\in B).
\end{equation}
Vereinigungsmenge:
\begin{equation}
A\cup B:=\{x\mid x\in A\lor x\in B\}.
\end{equation}
Schnittmenge:
\begin{equation}
A\cap B:=\{x\mid x\in A\land x\in B\}.
\end{equation}
Differenzmenge:
\begin{equation}
A\setminus B:=\{x\mid x\in A\land x\not\in B\}.
\end{equation}
Symmetrische Differenz:
\begin{equation}
A\triangle B:=\{x\mid x\in A\oplus x\in B\}.
\end{equation}
Komplementärmenge:
\begin{equation}
A^\comp := G\setminus A.\qquad (\text{$G$: Grundmenge})
\end{equation}
Vereinigung über indizierte Mengen:
\begin{equation}
\bigcup_{i\in I} A_i := \{x\mid\exists i{\in}I\,(x\in A_i)\}.
\end{equation}
Schnitt über indizierte Mengen:
\begin{equation}
\bigcap_{i\in I} A_i := \{x\mid\forall i{\in}I\,(x\in A_i)\}.
\end{equation}


\subsection{Boolesche Algebra}
\begin{table*}[t]
\caption{Boolesche Algebra}
\begin{tabular}{l|l|l}
\thbf{Vereinigung} & \thbf{Schnitt} &\\
  $A\cup A = A$
& $A\cap A = A$
& Idempotenzgesetze\\
  $A\cup \{\} = A$
& $A\cap G = A$
& Neutralitätsgesetze\\
  $A\cup G = G$
& $A\cap \{\} = \{\}$
& Extremalgesetze\\
  $A\cup \overline A = G$
& $A\cap \overline A = \{\}$
& Komplementärgesetze\\
\noalign{\vspace{1em}}
  $A\cup B = B\cup A$
& $A\cap B = B\cap A$
& Kommutativgesetze\\
  $(A\cup B)\cup C = A\cup (B\cup C)$
& $(A\cap B)\cap C = A\cap (B\cap C)$
& Assoziativgesetze\\
  $\overline{A\cup B} = \overline A\cap\overline B$
& $\overline{A\cap B} = \overline A\cup\overline B$
& De Morgansche Regeln\\
  $A\cup (A\cap B) = A$
& $A\cap (A\cup B) = A$
& Absorptionsgesetze\\
\end{tabular}\\
\\
$G$: Grundmenge
\end{table*}

\noindent
\strong{Distributivgesetze}:
\begin{gather}
M\cup (A\cap B) = (M\cup A)\cap (M\cup B),\\
M\cap (A\cup B) = (M\cap A)\cup (M\cap B).
\end{gather}

\subsection{Teilmengenrelation}
Zerlegung der Gleichheit:
\begin{equation}
A=B \iff A\subseteq B \land B\subseteq A.
\end{equation}
Umschreibung der Teilmengenrelation:
\begin{equation}
\begin{split}
A\subseteq B &\iff A\cap B=A\\
& \iff A\cup B=B\\
& \iff A\setminus B=\{\}.
\end{split}
\end{equation}
Kontraposition:
\begin{equation}
A\subseteq B = B^\comp\subseteq A^\comp.
\end{equation}

\subsection{Natürliche Zahlen}
\subsubsection{Von-Neumann-Modell}
Mengentheoretisches Modell der natürlichen Zahlen:
\begin{equation}
\begin{split}
& 0:=\{\},\quad 1:=\{0\},\quad 2:=\{0,1\},\\
& 3:=\{0,1,2\},\quad \text{usw.}
\end{split}
\end{equation}
Nachfolgerfunktion:
\begin{equation}
x' := x\cup\{x\}.
\end{equation}
\subsubsection{Vollständige Induktion}
Ist $A(n)$ mit $n\in\N$
eine Aussageform, so gilt:
\begin{equation}
\begin{split}
& A(n_0)\land \forall n\ge n_0\,[A(n)\Rightarrow A(n+1)]\\
& \implies \forall n\ge n_0\,[A(n)].
\end{split}
\end{equation}
Die Aussage $A(n_0)$ ist der \emph{Induktionsanfang}.

Die Implikation
\begin{equation}
A(n)\Rightarrow A(n+1)
\end{equation}
heißt \emph{Induktionsschritt}. Beim Induktionsschritt muss
$A(n+1)$ gezeigt werden, wobei $A(n)$ als gültig vorausgesetzt werden
darf.

% \newpage
\subsection{ZFC-Axiome}

Axiom der Bestimmtheit:
\begin{equation}
\forall A\forall B\,[A=B\iff\forall x\,[x\in A\Leftrightarrow x\in B]].
\end{equation}
Axiom der leeren Menge:
\begin{equation}
\exists M\forall x\,[x\notin M].
\end{equation}
Axiom der Paarung:
\begin{equation}
\forall x\forall y\exists M\forall a\,[a\in M\iff x=a\lor y=a].
\end{equation}
Axiom der Vereinigung:
\begin{equation}
\forall S\exists M\forall x\,[x\in M\iff\exists A{\in}S\,[x\in A]].
\end{equation}
Axiom der Aussonderung:
\begin{equation}
\forall A\exists M\forall x\,[x\in M\iff x\in A\land\varphi(x)].
\end{equation}
Axiom des Unendlichen:
\begin{equation}
\exists M\,[\{\}\in M\land\forall x{\in}M\,[x\cup\{x\}\in M]].
\end{equation}
Axiom der Potenzmenge:
\begin{equation}
\forall A\exists M\forall T\,[T\in M\iff T\subseteq A].
\end{equation}
Axiom der Ersetzung:
\begin{equation}
\begin{split}
&\forall a{\in}A\;\exists^{=1} b\,[\varphi(a,b)]\\
&\implies\exists B\,\forall b\,[b\in B\iff\exists a{\in}A\,[\varphi(a,b)]].
\end{split}
\end{equation}
Axiom der Fundierung:
\begin{equation}
\forall A\,[A\ne\{\}\implies\exists x{\in}A\,[x\cap A=\{\}]].
\end{equation}
Auswahlaxiom:
\begin{equation}
\begin{split}
&\forall x,y{\in}A\,[x\ne y\implies x\cap y=\{\}]\\
&\quad\land\forall x{\in}A\,[x\ne\{\}]\\
&\implies\exists M\;\forall x{\in}A\;\exists^{=1}u{\in}x\,[u\in M].
\end{split}
\end{equation}

\newpage
\subsection{Kardinalität}
\begin{definition}[Gleichmächtigkeit]
Zwei Mengen $M,N$ heißen \emdef{gleichmächtig}, notiert als
$|M|=|N|$, wenn es eine bijektive Abbildung $f\colon M\to N$ gibt.

Eine Menge $M$ heißt \emdef{weniger mächtig oder gleichmächtig},
notiert als $|M|\le|N|$, wenn es eine injektive Abbildung
$f\colon M\to N$ gibt. Äquivalent dazu ist, dass es eine
surjektive Abbildung $g\colon N\to M$ gibt.

Eine Menge heißt \emdef{abzählbar unendlich}, wenn sie gleichmächtig
zu den natürlichen Zahlen ist.
\end{definition}
Gleichmächtigkeit ist eine Äquivalenzrelation.
\begin{definition}[Kardinalzahl]
Die Äquivalenzklassen
\begin{equation}
|M| := \{N\mid\;{\scriptstyle |M|=|N|}\}
\end{equation}
heißen \emdef{Kardinalzahlen}.
\end{definition}

\strong{Satz von Cantor-Bernstein.}

Aus $|M|\le |N|$ und $|N|\le |M|$ folgt $|M|=|N|$.

\subsubsection{Potenzmengen}

\strong{Satz von Cantor.}
Für jede Menge gilt $|M|<|2^M|$.

Ist $M$ endlich, dann gilt $|M|=2^{|M|}$.


\section{Funktionen}
\subsection{Injektionen}\index{injektiv}
\begin{definition}[Injektion]
Eine Funktion $f\colon A\to B$ heißt \emdef{injektiv},
wenn
\begin{equation}
\forall x_1,x_2\in A\,[f(x_1)=f(x_2)\implies x_1=x_2]
\end{equation}
gilt.
\end{definition}

\begin{definition}[Linksinverse]
Sei $f\colon A\to B$. Eine Funktion $g\colon B\to A$ mit
\begin{equation}
g\circ f = \id_A
\end{equation}
heißt \emdef{Linksinverse} von $f$.
\end{definition}
Eine Funktion ist genau dann injektiv, wenn sie eine Linksinverse
besitzt. Zu einer Injektion kann es aber mehrere unterschiedliche
Linksinverse geben.

\subsection{Surjektionen}\index{surjektiv}
\begin{definition}[Surjektion]
Eine Funktion $f\colon A\to B$ heißt \emdef{surjektiv},\\
wenn $f(A)=B$ ist. Damit ist gemeint, dass jedes Element
der Zielmenge wenigstens einmal der Funktionswert von einem
Element der Definitionsmenge ist.
\end{definition}

\newpage
\begin{definition}[Rechtsinverse]
Sei $f\colon A\to B$. Eine Funktion $g\colon B\to A$ mit
\begin{equation}
f\circ g = \id_B
\end{equation}
heißt \emdef{Rechtsinverse} von $f$.
\end{definition}
Eine Funktion ist genau dann surjektiv, wenn sie eine Rechtsinverse
besitzt. Zu einer Surjektion kann es aber mehrere unterschiedliche
Rechtsinverse geben.

\subsection{Bijektionen}\index{bijektiv}
\begin{definition}[Bijektion]
Eine Funktion $f\colon A\to B$ heißt \emdef{bijektiv},
wenn sie injektiv und surjektiv ist.

Eine Funktion $f\colon A\to B$ ist genau dann bijektiv, wenn es
ein $g$ mit
\begin{equation}
g\circ f = \id_A\quad\text{und}\quad f\circ g = \id_B
\end{equation}
gibt. Wenn $f$ bijektiv ist, so gibt es $g$ genau einmal und
$g$ wird die \emph{Umkehrfunktion}\index{Umkehrfunktion}
oder \emph{Inverse}
von $f$ genannt und als $f^{-1}$ notiert.
\end{definition}

\subsection{Komposition}\index{Komposition}
\begin{definition}[Komposition]
Für zwei Funktionen $f\colon A\to B$
und $g\colon B\to C$ ist die \emdef{Komposition}
($g$ nach $f$)
durch
\begin{equation}\label{eq:composition}
g\circ f\colon A\to C,\quad (g\circ f)(x) := g(f(x))
\end{equation}
definiert.
\end{definition}
Für die Komposition gilt das Assozativgesetz:
\begin{equation}
(f\circ g)\circ h = f\circ(g\circ h).
\end{equation}

Die Komposition von Injektionen ist eine Injektion.

Die Komposition von Surjektionen ist eine Surjektion.

Die Komposition von Bijektionen ist eine Bijektion.

Sind $f,g$ Bijektionen, so gilt
\begin{equation}
(g\circ f)^{-1} = f^{-1}\circ g^{-1}.
\end{equation}

Ist $g\circ f$ injektiv, so ist $f$ injektiv.

Ist $g\circ f$ surjektiv, so ist $g$ surjektiv.

Ist $g\circ f$ bijektiv, so ist $f$ injektiv und $g$ surjektiv.

\begin{definition}[Iteration]
Für eine Funktion $\varphi\colon A\to A$ wird
\begin{equation}
\varphi^0:=\operatorname{id}_A,\quad \varphi^{n+1}:=\varphi^n\circ\varphi
\end{equation}
\emdef{Iteration}\index{Iteration} von $\varphi$ genannt.
\end{definition}

\newpage
\subsection{Einschränkung}\index{Einschränkung}
\begin{definition}[Einschränkung]
Sei $f\colon A\to B$ und $M\subseteq A$.
Die Funktion $g(x)=f(x)$ mit $g\colon M\to B$ wird \emdef{Einschränkung}
von $f$ genannt und mit $f|_M$ notiert.
\end{definition}
Sei $f\colon A\to B$ und $M\subseteq A$.
Mit der Inklusionsabbildung $i(x):=x$ mit $i\colon M\to A$ gilt:
\begin{equation}
f|_M = f\circ i.
\end{equation}
Es gilt
\begin{equation}
g\circ (f|_M) = (g\circ f)|_M.
\end{equation}

\subsection{Bild}\index{Bild}
\begin{definition}[Bild]
Ist $f\colon A\to B$ und $M\subseteq A$, so wird
\begin{equation}
f(M) := \{f(x)\mid x\in M\}
\end{equation}
das \emdef{Bild} von $M$ unter $f$ genannt.
\end{definition}
Es gilt
\begin{align}
&f(M\cup N) = f(M)\cup f(N),\\
&f(M\cap N) \subseteq f(M)\cap f(N),\\
&f\Big(\bigcup_{i\in I}M_i\Big) = \bigcup_{i\in I} f(M_i),\\
&I\ne\emptyset\implies f\Big(\bigcap_{i\in I} M_i\Big) \subseteq \bigcap_{i\in I} f(M_i),\\
&M\subseteq N\implies f(M)\subseteq f(N),\\
&f(\emptyset) = \emptyset,\\
&(g\circ f)(M) = g(f(M)),\\
&f(M) = \bigcup_{x\in M} f(\{x\}).
\end{align}

\subsection{Urbild}\index{Urbild}
\begin{definition}[Urbild]
Ist $f\colon A\to B$, so wird
\begin{equation}
f^{-1}(M) := \{x\in A\mid f(x)\in M\}.
\end{equation}
das \emdef{Urbild} von $M$ unter $f$ genannt.
\end{definition}
Es gilt
\begin{align}
& f^{-1}(M\cup N) = f^{-1}(M)\cup f^{-1}(N),\\
& f^{-1}(M\cap N) = f^{-1}(M)\cap f^{-1}(N),\\
& f^{-1}\Big(\bigcup_{i\in I}M_i\Big) = \bigcup_{i\in I} f^{-1}(M_i),\\
& I\ne\emptyset\implies f^{-1}\Big(\bigcap_{i\in I} M_i\Big) = \bigcap_{i\in I}f^{-1}(M_i),\\
& M\subseteq N\implies f^{-1}(M)\subseteq f^{-1}(N),\\
& f^{-1}(\emptyset) = \emptyset,\\
& f^{-1}(B) = A,\\
& f^{-1}(M\setminus N) = f^{-1}(M)\setminus f^{-1}(N),\\
& f^{-1}(B\setminus M) = B\setminus f^{-1}(M),\\
& (g\circ f)^{-1}(M) = f^{-1}(g^{-1}(M)),\\
& (f|_M)^{-1}(N) = M\cap f^{-1}(N).
\end{align}

\newpage
\phantom{x}

\newpage
\section{Formale Systeme}
\subsection{Formale Sprachen}
\begin{definition}[Formale Sprache]
Eine \emdef{formale Sprache} $L$ ist eine Teilmenge der kleenschen
Hülle über einer Menge $\Sigma$, kurz $L\subseteq\Sigma^*$.
Die Menge $\Sigma$ wird \emdef{Alphabet} genannt,
ihre Elemente heißen \emdef{Symbole}.

Die kleensche Hülle $\Sigma^*$ besteht aus allen möglichen
Konkatenationen von Symbolen aus $\Sigma$. Die Konkatenationen
von $\Sigma^*$ heißen \emdef{Wörter}. Die leere Konkatenation ist
zulässig und wird mit $\varepsilon$ notiert. Die Elemente von $L$ heißen
\emdef{wohlgeformte Wörter} oder \emdef{wohlgeformte Formeln},
engl. \emdef{well formed formulas}, kurz \emdef{wff}.
\end{definition}

Ein Wort $a$ ist ein Tupel
\begin{equation}
a = (a_1,\ldots, a_m).\qquad (a_k\in\Sigma)
\end{equation}
Sind $a,b$ zwei Wörter, dann ist mit $ab$ deren Konkatenation
gemeint:
\begin{equation}
ab := (a_1,\ldots,a_m,b_1,\ldots b_n).
\end{equation}
Es gilt $\varepsilon a=a$ und $a\varepsilon=a$.
Bei $\varepsilon$ handelt es sich um das leere Tupel.

\begin{definition}[Konkatenation von Sprachen]
\emdef{Konkatenation} von $L_1$ und $L_2$:
\begin{equation}
L_1\circ L_2 := \{ab\mid a\in L_1, b\in L_2\}.
\end{equation}
\end{definition}

\begin{definition}[Potenz einer Sprache]
\emdef{Potenzen} von $L$:
\begin{align}
L^0 &:= \{\varepsilon\},\\
L^n &:= L^{n-1}\circ L.
\end{align}
\end{definition}

\begin{definition}[Kleensche Hülle einer Sprache]
\emdef{Kleensche Hülle} von $L$:
\begin{equation}
L^* := \bigcup_{k\in\N_0} L^k.
\end{equation}

\emdef{Positive Hülle} von $L$:
\begin{equation}
L^+ := \bigcup_{k\in\N_1} L^k.
\end{equation}
\end{definition}

% \newpage
\subsection{Formale Grammatiken}
\begin{definition}[Formale Grammatik]
Eine \emdef{formale Grammatik} ist ein Tupel $(N,\Sigma,P,S)$,
wobei $N$ die \emdef{Nonterminalsymbolen}\index{Nonterminalsymbol},
$\Sigma$ die \emdef{Terminalsymbolen}\index{Terminalsymbol},
$P$ die \emdef{Produktionsregeln}\index{Produktionsregel} sind
und $S$ ein \emdef{Startsymbol}\index{Startsymbol} ist.
Die Mengen $N,\Sigma,P$ müssen endlich sein. Die Mengen $N$ und
$\Sigma$ müssen disjunkt sein. Bei $\Sigma$ handelt es sich um
ein Alphabet. Das Startsymbol ist ein Element $S\in N$.

Bei $P$ handelt es sich um eine Relation
\begin{equation}\label{eq:einfache-Produktionsregeln}
P\subseteq N\times (N\cup\Sigma)^*
\end{equation}
oder allgemeiner
\begin{equation}
P\subseteq (N\cup\Sigma)^*\setminus\Sigma^*\times (N\cup\Sigma)^*.
\end{equation}
Produktionsregeln werden in der Form $n\to w$ notiert und drücken aus,
dass in jedem Wort das Nonterminalsymbol $n$ durch das Wort $w$ ersetzt
werden darf. Allgemeiner bedeutet $t\to w$, dass ein Teilwort $t$
durch $w$ ersetzt werden darf.

Die Produktionsregeln werden ausgehend vom Startsymbol immer weiter
angewendet bis keine Nonterminalsymbole mehr vorhanden sind.
Die Menge aller möglichen Produktionen bildet
eine formale Sprache $L\subseteq\Sigma^*$.
\end{definition}

Für Produktionsregeln der Form \eqref{eq:einfache-Produktionsregeln}
wurde eine Kurznotation geschaffen, die EBNF:

\begin{tabular}{l|l}
\verb|Symbol| & Nonterminalsymbol\\
\verb|"Symbol"| & Terminalsymbol\\
\verb|w1, w2| & $w_1w_2$ (Konkatenation)\\
\verb/n = w1 | w2./ & $n\to w_1,\; n\to w_2$\\
\verb|n = {w}.| & $n\to \varepsilon,\; n\to wn$\\
\verb|n = [w].| & $n\to w,\; n\to wn$
\end{tabular}

\subsection{Formale Systeme}
\begin{definition}[Formales System]
Ein \emdef{formales System} ist ein Tupel $(\Sigma,L,A,R)$, wobei
$\Sigma$ ein Alphabet, $L$ eine formale Sprache über
dem Alphabet, $A$ eine Menge von Axiomen und $R$ eine Menge von
Ableitungsrelationen ist. Die Menge der \emdef{Axiome} ist eine
beliebige Teilmenge von $L$. 
Eine \emdef{Ableitungsrelation} ist eine zwei oder mehrstellige
Relation über $L$, die
\begin{equation}
a_1,\ldots,a_n\vdash b
\end{equation}
geschrieben wird. Eine wohlgeformte Formel wird $\emdef{Satz}$
genannt, wenn sie ein Axiom ist oder über eine Kette von
Ableitungen aus den Axiomen folgt.
\end{definition}

\subsection{Semantik}
\begin{definition}[Interpretation (Aussagenlogik)]
Eine \emph{Interpretation}\index{Interpretation}
$I\colon V\to\{0,1\}$ ist eine Abbildung,
welche jeder logischen Variablen einen Wahrheitswert zuordnet.

Eine \emph{Interpretation} $I\colon F\to\{0,1\}$ erweitert den
Definitionsbereich einer Interpretation wie folgt auf die
Menge aller wohlgeformten Formeln:
\begin{gather}
I(\varphi\land\psi) = (I(\varphi)\land I(\psi)),\\
I(\varphi\lor\psi) = (I(\varphi)\lor I(\psi)),\\
I(\varphi\rightarrow\psi) = (I(\varphi)\rightarrow I(\psi)),\\
I(\varphi\leftrightarrow\psi) = (I(\varphi)\leftrightarrow I(\psi)),\\
I(\neg\varphi) = (\neg I(\varphi)).
\end{gather}
Die rechten Seiten werden hierbei entsprechend den Wertetabellen
ausgewertet.
\end{definition}

\begin{definition}[Modellrelation]
Sei $\Gamma=\{\varphi_1,\ldots,\varphi_n\}$ eine endliche Menge
von Formeln und sei $\psi$ eine Formel. Die Formelmenge $\Gamma$
\emph{modelliert} $\psi$, wenn jede Interpretation, die alle
Formeln in $\Gamma$ erfüllt, auch $\psi$ erfüllt. Kurz:
\begin{equation}
(\Gamma\models\psi) \defiff \forall I[\forall\varphi{\in}\Gamma(I(\varphi))\Rightarrow I(\psi)].
\end{equation}
\end{definition}

\newpage
\section{Mathematische Strukturen}\label{sec:Strukturen}
\subsubsection*{Axiome}

\noindent\bsf{E:} Abgeschlossenheit.
\ibox{Die Verknüpfung führt nicht aus der Menge heraus.}

\noindent\bsf{A:} Assoziativgesetz.
\ibox{$\forall a,b,c\bright (a*b)*c = a*(b*c)\bleft$.}

\noindent\bsf{N:} Existenz des neutralen Elements.
\ibox{$\exists e\forall a\bright e*a=a*e=a\bleft$.}

\noindent\bsf{I:} Existenz der inversen Elemente.
\ibox{$\forall a\exists b\bright a*b=b*a=e\bleft$.}

\noindent\bsf{K:} Kommutativgesetz.
\ibox{$\forall a,b\bright a*b=b*a\bleft.$}

\noindent
\bsf{I*:} Existenz der multiplikativ inversen Elemente.
\ibox{$\forall a{\ne}0\;\exists b\bright a*b=b*a=1\bleft$.}

\noindent\bsf{Dl:} Linksdistributivgestz.
\ibox{$\forall a,x,y\bright a*(x+y) = a*x+a*y\bleft$.}

\noindent\bsf{Dr:} Rechtsdistributivgesetz.
\ibox{$\forall a,x,y\bright (x+y)*a = x*a+y*a\bleft$.}

\noindent\bsf{D:} Distributivgesetze.
\ibox{Dl und Dr.}

\noindent\bsf{T:} Nullteilerfreiheit.
\ibox{$\forall a,b\bright a\ne 0\land b\ne 0\implies a*b\ne 0\bleft$}
\ibox{bzw. die Kontraposition}
\ibox{$\forall a,b\bright a*b=0\implies a=0\lor b=0\bleft$.}

\noindent\bsf{U:} Unterscheibarkeit von Null- und Einselement.
\ibox{Die neutralen Elemente bezüglich Addition und}
\ibox{Multiplikation sind unterschiedlich.}

\subsubsection*{Strukturen}
Strukturen mit einer inneren Verknüpfung:\\
\begin{tabular}{l|l}
\bsf{EA} & Halbgruppe\\
\bsf{EAN} & Monoid\\
\bsf{EANI} & Gruppe\\
\bsf{EANIK} & abelsche Gruppe
\end{tabular}

\noindent
Strukturen mit zwei inneren Verknüpfungen:\\
\begin{tabular}{l|l}
\bsf{EANIK, EA, D}\dotfill & Ring\\
\bsf{EANIK, EAK, D}\dotfill & kommutativer Ring\\
\bsf{EANIK, EAN, D}\dotfill & unitärer Ring\\
\bsf{EANIK, EANK, DTU} & Integritätsring\\
\bsf{EANIK, EANI*K, DTU} & Körper
\end{tabular}

\newpage
\subsubsection*{Axiome für Relationen}

\noindent\bsf{R:} Reflexivität.
\ibox{$\forall a\,(a R a)$.}

\noindent\bsf{S:} Symmetrie.
\ibox{$\forall a,b\,(aRb\iff bRa)$.}

\noindent\bsf{T:} Transitivität.
\ibox{$\forall a,b,c\,(aRb\land bRc\implies aRc)$.}

\noindent\bsf{An:} Antisymmetrie.
\ibox{$\forall a,b\,(aRb\land bRa\implies a=b)$.}

\noindent\bsf{L:} Linearität.
\ibox{$\forall a,b\,(aRb\lor bRa)$.}

\noindent\bsf{Ri:} Irrreflexivität.
\ibox{$\forall a\,(\neg aRa)$.}

\noindent\bsf{A:} Asymmetrie.
\ibox{$\forall a,b\,(aRb\implies \neg bRa)$.}

\noindent\bsf{Min:} Existenz der Minimalelemente.
\ibox{$\forall T{\subseteq}M, T{\ne}\emptyset\;\exists x{\in}T\;\forall y{\in}T{\setminus}\{x\}\,(x<y)$.}

\subsubsection*{Relationen}
\begin{tabular}{l|l}
\bsf{RST}\dotfill & Äquivalenzrelation\\
\bsf{RAnT}\dotfill & Halbordnung\\
\bsf{RAnTL}\dotfill & Totalordnung\\
\bsf{RiAT}\dotfill & strenge Halbordnung\\
\bsf{RiATL}\dotfill & strenge Totalordnung\\
\bsf{RiATLMin} & Wohlordnung
\end{tabular}


\chapter{Analysis}
\section{Folgen}
\subsection{Konvergenz}

\begin{Definition}[open-ep-ball: offene Epsilon-Umgebung]%
\index{Epsilon-Umgebung}\index{offene Epsilon-Umgebung}
Sei $(M,d)$ ein metrischer Raum. Unter der offenen Epsilon-Umgebung
von $a\in M$ versteht man:%
\[U_\varepsilon(a) := \{x\mid d(x,a)<\varepsilon\}.\]
\end{Definition}

\noindent
Man setze zunächst speziell $d(x,a):=|x-a|$ bzw. $d(x,a):=\|x-a\|$.

\begin{Definition}[lim: konvergente Folge, Grenzwert]%
\label{def:lim}\index{konvergente Folge}\index{Grenzwert}
Eine Folge $(a_n)$ heißt konvergent gegen einen Grenzwert $a$, wenn es
zu jedem noch so kleinen $\varepsilon$ einen Index $n_0$ gibt, so dass
ab diesem Index sämtliche ihrer Werte in der Umgebung
$U_\varepsilon(a)$ liegen. Formal:
\[\lim_{n\to\infty} a_n = a
\defiff \forall\varepsilon{>}0\colon\exists n_0\colon\forall n{\ge}n_0\colon a_n\in U_\varepsilon(a)\]
bzw.
\[\lim_{n\to\infty} a_n = a
\defiff \forall\varepsilon{>}0\colon\exists n_0\colon\forall n{\ge}n_0\colon \|a_n-a\|<\varepsilon.\]
\end{Definition}

\begin{Definition}[bseq: beschränkte Folge]%
\label{def:bseq}\index{beschreankte Folge@beschränkte Folge}
Eine Folge $(a_n)$ mit $a_n\in\R$ heißt genau dann beschränkt,
wenn es eine reelle Zahl $S$ gibt mit $|a_n|<S$ für alle $n$.

Eine Folge $(a_n)$ von Punkten eines normierten Raums heißt genau
dann beschränkt, wenn es eine reelle Zahl $S$ gibt mit $\|a_n\|<S$
für alle $n$.
\end{Definition}

\begin{Satz}[Grenzwert bei Konvergenz eindeutig bestimmt]\newlinefirst
Eine konvergente Folge von Elementen eines metrischen Raumes
besitzt genau einen Grenzwert.
\end{Satz}

\begin{Beweis}
Sei $(a_n)$ eine konvergente Folge mit $a_n\to g_1$. Sei weiterhin
$g_1\ne g_2$. Es wird nun gezeigt, dass $g_2$ kein Grenzwert von $a_n$
sein kann. Wir müssen also zeigen:
\[\neg\lim_{n\to\infty} a_n=g_2 \iff
\exists\varepsilon{>}0\colon\forall n_0\colon\exists n{\ge}n_0\colon
a_n\notin U_\varepsilon(g_2)\]
mit $a_n\notin U_\varepsilon(g_2)\iff d(a_n,g_2)\ge\varepsilon$.

Um dem Existenzquantor zu genügen, wählt man nun
$\varepsilon = \frac{1}{2}d(g_1,g_2)$.
Nach Def. \ref{metric-space} (metric-space) gilt 
$d(g_1,g_2)>0$, daher ist auch $\varepsilon>0$. Nach Satz
\ref{construction-disjoint-ep-balls} sind die Umgebungen
$U_\varepsilon(g_1)$ und $U_\varepsilon(g_2)$ disjunkt.
Wegen $a_n\to g_1$ gibt es ein $n_0$ mit $a_n\in U_\varepsilon(g_1)$ für alle
$n\ge n_0$. Dann gibt es für jedes beliebig große $n_0$ aber auch
$n\ge n_0$ mit $a_n\notin U_\varepsilon(g_2)$.\,\qedsymbol
\end{Beweis}

\begin{Satz}[lim-scaled-ep: skaliertes Epsilon]\label{lim-scaled-ep}
Es gilt:
\[\lim_{n\to\infty} a_n=a \iff
\forall\varepsilon{>}0\colon\exists n_0\colon\forall n{\ge}n_0\colon \|a_n-a\|<R\varepsilon,\]
wobei $R>0$ ein fester aber beliebieger Skalierungsfaktor ist.
\end{Satz}

\begin{Beweis}
Betrachte $\varepsilon>0$ und multipliziere auf beiden Seiten
mit $R$. Dabei handelt es sich um eine Äquivalenzumformung.
Setze $\varepsilon':=R\varepsilon$. Demnach gilt:
\[\varepsilon>0 \iff \varepsilon'>0.\]
Nach der Ersetzungsregel düfen wir die Teilformel $\varepsilon>0$
nun ersetzen. Es ergibt sich die äquivalente Formel
\[\lim_{n\to\infty} a_n=a \iff
\forall\varepsilon'{>}0\colon\exists n_0\colon\forall n{\ge}n_0\colon
\|a_n-a\|<\varepsilon'.\]
Das ist aber genau Def. \ref{def:lim} (lim).\,\qedsymbol
\end{Beweis}

\begin{Satz}
Es gilt:
\[\lim_{n\to\infty} a_n = a\implies \lim_{n\to\infty} \|a_n\| = \|a\|.\]
\end{Satz}

\begin{Beweis}
Nach Satz \ref{rev-tineq} (umgekehrte Dreiecksungleichung) gilt:
\[|\|a_n\|-\|a\|| \le \|a_n-a\| < \varepsilon.\]
Dann ist aber erst recht $|\|a_n\|-\|a\||<\varepsilon$.\,\qedsymbol
\end{Beweis}

\begin{Satz}\label{zero-seq-bounded}
Ist $(a_n)$ eine Nullfolge und $(b_n)$ eine beschränkte Folge,
dann ist auch $(a_n b_n)$ eine Nullfolge.
\end{Satz}

\begin{Beweis}
Wenn $(b_n)$ beschränkt ist, dann existiert nach
Def. \ref{def:bseq} (bseq) eine Schranke $S$ mit
$|b_n|<S$ für alle $n$. Man multipliziert nun auf beiden Seiten
mit $|a_n|$ und erhält
\[|a_n b_n| = |a_n| |b_n| < |a_n| S.\]
Wenn $a_n\to 0$, dann muss für jedes $\varepsilon$
ein $n_0$ existieren mit $|a_n|<\varepsilon$ für $n\ge n_0$.
Multipliziert man auf beiden Seiten mit $S$, und ergibt sich
\[|a_n b_n-0| = |a_n b_n| < |a_n| S < S\varepsilon.\]
Nach Satz \ref{lim-scaled-ep} (lim-scaled-ep) gilt dann
aber $a_n b_n\to 0$.\,\qedsymbol
\end{Beweis}

\begin{Satz}
Sind $(a_n)$ und $(b_n)$ Nullfolgen,
dann ist auch $(a_n b_n)$ eine Nullfolge.
\end{Satz}

\begin{Beweis}[Beweis 1]
Wenn $(b_n)$ eine Nullfolge ist, dann ist $(b_n)$ auch beschränkt.
Nach Satz \ref{zero-seq-bounded} gilt dann die Behauptung.
\end{Beweis}

\begin{Beweis}[Beweis 2]
Sei $\varepsilon>0$ beliebig.
Es gibt ein $n_0$, so dass
$|a_n|<\varepsilon$ und $|b_n|<\varepsilon$ für $n\ge n_0$.
Demnach ist
\[|a_n b_n| = |a_n| |b_n|< |a_n|\varepsilon <\varepsilon^2.\]
Wegen $\varepsilon>0\iff\varepsilon'>0$ mit
$\varepsilon'=\varepsilon^2$ gilt
\[\forall\varepsilon'{>}0\colon\exists n_0\colon\forall n{\ge}n_0\colon
|a_n b_n|<\varepsilon'.\]
Nach Def. \ref{def:lim} (lim) gilt somit die Behauptung.\,\qedsymbol
\end{Beweis}

\newpage
\begin{Satz}[Grenzwertsatz zur Addition]%
\label{lim-add}\index{Grenzwertsaetze@Grenzwertsätze}
Seien $(a_n)$, $(b_n)$ Folgen von Vektoren eines normierten Raumes.
Es gilt:
\[\lim_{n\to\infty} a_n = a\land \lim_{n\to\infty} b_n
= b \implies \lim_{n\to\infty} a_n+b_n = a+b.\]
\end{Satz}

\begin{Beweis}
Dann gibt es ein $n_0$, so dass für $n\ge n_0$ sowohl
$\|a_n-a\|<\varepsilon$ als auch $\|b_n-b\|<\varepsilon$.
Addition der beiden Ungleichungen führt zu
\[\|a_n-a\| + \|b_n-b\| < 2\varepsilon.\]
Laut der Dreiecksungleichung, das ist Axiom (N3) in Def.
\ref{def:normed-space} (normed-space), gilt nun aber die Abschätzung
\[\|(a_n+b_n)-(a+b)\| = \|(a_n-a)+(b_n-b)\| \le \|a_n-a\|+\|b_n-b\|.\]
Somit gilt erst recht
\[\|(a_n+b_n)-(a+b)\| < 2\varepsilon.\]
Nach Satz \ref{lim-scaled-ep} (lim-scaled-ep)
folgt die Behauptung.\,\qedsymbol
\end{Beweis}

\begin{Satz}[Grenzwertsatz zur Skalarmultiplikation]\label{lim-smult}
Sei $(a_n)$ eine Folge von Vektoren eines normierten Raumes
und sei $r\in\R$ oder $r\in\C$. Es gilt:
\[\lim_{n\to\infty} a_n = a\implies \lim_{n\to\infty} ra_n\to ra.\]
\end{Satz}

\begin{Beweis}
Sei $\varepsilon>0$ fest aber beliebig. Es gibt nun ein $n_0$, so
dass $\|a_n-a\|<\varepsilon$ für $n\ge n_0$.
Multipliziert man auf beiden Seiten
mit $|r|$ und zieht Def. \ref{def:normed-space} (normed-space)
Axiom (N2) heran, dann ergibt sich
\[\|ra_n-ra\| = |r|\,\|a_n-a\|<|r|\varepsilon.\]
Nach Satz \ref{lim-scaled-ep} (lim-scaled-ep)
folgt die Behauptung.\,\qedsymbol
\end{Beweis}

\begin{Satz}[Grenzwertsatz zum Produkt]\newlinefirst
Seien $(a_n)$ und $(b_n)$ Folgen
reeller Zahlen. Es gilt:
\[\lim_{n\to\infty} a_n=a\land\lim_{n\to\infty} b_n=b\implies
\lim_{n\to\infty} a_n b_n = ab.\]
\end{Satz}

\begin{Beweis}
Nach Voraussetzung sind $a_n-a$ und $b_n-b$ Nullfolgen.
Da das Produkt von Nullfolgen wieder eine Nullfolge ist, gilt
\[(a_n-a)(b_n-b) = a_n b_n-a_n b-ab_n+ab\to 0.\]
Da nach Satz \ref{lim-smult} aber $a_n b\to ab$ und $ab_n\to ab$,
ergibt sich nach Satz \ref{lim-add} nun
\[(a_n-a)(b_n-b)+a_n b+ab_n = a_n b_n+ab\to 2ab.\]
Addiert man nun noch die konstante Folge $-2ab$
und wendet nochmals Satz \ref{lim-add} an, dann ergibt sich
die Behauptung
\[a_n b_n\to ab.\,\qedsymbol\]
\end{Beweis}

\newpage
\begin{Satz}\label{cont-seqcont}%
\index{folgenstetig}\index{stetig!folgenstetig}
Sei $M$ ein metrischer Raum und $X$ ein topologischer Raum.
Eine Abbildung $f\colon M\to X$ ist genau dann stetig, wenn
sie folgenstetig ist.
\end{Satz}

\begin{Satz}[Satz zur Fixpunktgleichung]\index{Fixpunktgleichung}
Sei $M$ ein metrischer Raum und sei $f\colon M\to M$.
Sei $x_{n+1}:=f(x_n)$ eine Fixpunktiteration. Wenn die Folge
$(x_n)$ zu einem Startwert $x_0$ gegen ein $x\in M$ konvergiert, und
wenn $f$ eine stetige Abbildung ist, dann muss der Grenzwert $x$ die
Fixpunktgleichung $x=f(x)$ erfüllen.
\end{Satz}

\begin{Beweis}
Wenn $x_n\to x$, dann gilt trivialerweise auch $x_{n+1}\to x$.
Weil $f$ stetig ist, ist $f$ nach Satz \ref{cont-seqcont}
auch folgenstetig. Daher gilt $\lim f(a_n) = f(\lim a_n)$ für jede
konvergente Folge $(a_n)$. Somit gilt:
\[x=\lim_{n\to\infty} x_{n+1} = \lim_{n\to\infty} f(x_n)
= f(\lim_{n\to\infty} x_n) = f(x).\;\qedsymbol\]
\end{Beweis}

\subsection{Wachstum und Landau-Symbole}
\begin{Definition}\label{Landau-O}
Seien $f,g\colon D\to\R$ mit $D=\N$ oder $D=\R$. Man sagt, die
Funktion $f$ wächst nicht wesentlich schneller als $g$, kurz
$f\in\mathcal O(g)$, genau dann, wenn
\[\exists c{>}0\colon\,\exists x_0\colon\,\forall x{>}x_0\colon\, |f(x)|\le c|g(x)|.\]
\end{Definition}

\begin{Korollar}
Ist $r\in\R$ mit $r\ne 0$ eine Konstante, dann gilt
$\mathcal O(rg)=\mathcal O(g)$.
\end{Korollar}
\begin{Beweis}
Nach Def. \ref{Landau-O} ist
\[f\in\mathcal O(rg) \iff 
\exists c{>}0\colon\,\exists x_0\colon\,\forall x{>}x_0\colon\,|f(x)|\le c|rg(x)|.\]
Man hat nun
\[|f(x)|\le c|rg(x)| = c\cdot |r|\cdot |g(x)|.\]
Wegen $r\ne 0$ ist $|r|>0$ und daher auch $c>0\iff c|r|>0$. Sei
$c':=r|c|$. Also gilt $c>0\iff c'>0$. Nach der Ersetzungsregel
darf $c>0$ gegen $c'>0$ ersetzt werden und man erhält die
äquivalente Bedingung
\[\exists c'{>}0\colon\,\exists x_0\colon\,
\forall x{>}x_0\colon\,|f(x)|\le c'|g(x)|.\]
Nach Def. \ref{Landau-O} ist das gerade $f\in\mathcal O(g)$.\;\qedsymbol
\end{Beweis}

\begin{Korollar}
Sind $f_1,f_2\in\mathcal O(g)$, ist auch $f_1+f_2\in\mathcal O(g)$.
\end{Korollar}
\begin{Beweis}
Als Prämissen liegen Zeugen $c'>0,x_0'$ und $c''>0,x_0''$ für
\begin{gather*}
\forall x>x_0'\colon |f_1(x)|\le c'|g(x)|,\\
\forall x>x_0''\colon |f_2(x)|\le c''|g(x)|
\end{gather*}
vor. Mit der Dreiecksungleichung findet sich
\[|f_1(x)+f_2(x)|\le |f_1(x)|+|f_2(x)| \le c'|g(x)|+c''|g(x)| = (c'+c'')|g(x)|\]
für $x>\max(x_0',x_0'')$. Ergo sind $x_0:=\max(x_0',x_0'')$ und $c:=c'+c''$
Zeugen für
\[\exists c>0\colon\exists x_0\colon\forall x>x_0\colon |f_1(x)+f_2(x)|\le c|g(x)|.\,\qedsymbol\]
\end{Beweis}

\newpage
\section{Stetige Funktionen}

\begin{Definition}[Grenzwert einer Funktion]\label{fn-lim}
Sei $f\colon D\to\R$ mit $D\subseteq\R$ und sei $p$ ein
Häufungspunkt von $D$. Die Funktion $f$ heißt konvergent
gegen $L$ für $x\to p$, wenn%
\[\forall \varepsilon{>}0\colon\,\exists \delta{>}0\colon\,\forall x{\in}D\colon\,
(0<|x-x_0|<\delta\implies |f(x)-L|<\varepsilon).\]
Bei Konvergenz schreibt man $L=\lim\limits_{x\to p} f(x)$ und nennt $L$ den Grenzwert.
\end{Definition}

\begin{Definition}[cont: stetig]\label{cont}
Eine Funktion $f\colon D\to\R$ mit $D\subseteq\R$ heißt stetig an der
Stelle $x_0\in D$, wenn
\[\forall \varepsilon{>}0\colon\,\exists \delta{>}0\colon\,\forall x{\in}D\colon\,
(|x-x_0|<\delta\implies |f(x)-f(x_0)|<\varepsilon).\]
\end{Definition}

\begin{Definition}[Lipschitz-stetig]\newlinefirst
Eine Funktion $f\colon D\to\R$ mit $D\subseteq\R$ heißt
Lipschitz"=stetig, wenn eine Konstante $L$ existiert, so dass
\[|f(b)-f(a)|\le L|b-a|\]
für alle $a,b\in D$.
\end{Definition}

\begin{Definition}[Lipschitz-stetig an einer Stelle]%
\label{Lipschitz-cont-at}\newlinefirst
Eine Funktion $f\colon D\to\R$ mit $D\subseteq\R$ heißt
Lipschitz"=stetig an der Stelle $x_0\in D$, wenn eine Konstante $L$
existiert, so dass
\[|f(x_0)-(a)|\le L|x_0-a|\]
für alle $a\in D$.
\end{Definition}

\begin{Korollar}
Eine Funktion ist genau dann Lipschitz"=stetig, wenn sie an jeder
Stelle Lipschitz"=stetig ist und die Menge der optimalen
Lipschitz"=Konstanten dabei beschränkt.
\end{Korollar}
\begin{Beweis}
Eine Lipschitz"=stetige Funktion ist trivialerweise an jeder Stelle
Lipschitz"=stetig. Ist $f\colon D\to\R$ an der Stelle $b$ Lipschitz"=stetig,
dann existiert eine Lipschitz"=Konstante $L_b$ mit%
\[\forall a\in D\colon |f(b)-f(a)|\le L_b |b-a|.\]
Nach Voraussetzung ist $L=\sup_{b\in D} L_b$ endlich. Alle $L_b$ können
nun zu $L$ abgeschwächt werden und es ergibt sich%
\[\forall b\in D\colon\forall a\in D\colon |f(b)-f(a)|\le L|b-a|.\;\qedsymbol\]
\end{Beweis}


\begin{Definition}[lokal Lipschitz-stetig]\newlinefirst
Eine Funktion $f\colon D\to\R$ mit $D\subseteq\R$ heißt lokal
Lipschitz"=stetig in der Nähe einer Stelle $x_0\in D$, wenn es eine
Epsilon"=Umgebung $U_\varepsilon(x_0)$ gibt, so dass die Einschränkung
von $f$ auf diese Umgebung Lipschitz"=stetig ist. Die Funktion heißt
lokal Lipschitz"=stetig, wenn sie in der Nähe jeder Stelle
Lipschitz"=stetig ist.
\end{Definition}

\begin{Satz}\label{diff-nh-Lipschitz-cont-at}
Ist die Funktion $f\colon D\to\R$ an der Stelle $x_0$ differenzierbar,
dann gibt es ein $\delta>0$, so dass die Einschränkung von $f$
auf $U_\delta(x_0)$ an der Stelle $x_0$ Lipschitz"=stetig ist.
\end{Satz}

\begin{Beweis}
Def. \ref{fn-lim} wird in Def. \ref{diff} (diff) eingesetzt.
Es ergibt sich:
\[0<|x-x_0|<\delta\implies
\left|\frac{f(x)-f(x_0)}{x-x_0}-f'(x_0)\right|<\varepsilon.\]
Nach der umgekehrten Dreiecksungleichung \ref{rev-tineq} gilt
\[\left|\frac{f(x)-f(x_0)}{x-x_0}\right|-|f'(x_0)| \le
\left|\frac{f(x)-f(x_0)}{x-x_0}-f'(x_0)\right|
< \varepsilon.\]
Daraus ergibt sich
\[|f(x)-f(x_0)| < (|f'(x_0)|+\varepsilon)\cdot |x-x_0|\]
und somit erst recht
\[|f(x)-f(x_0)| \le (|f'(x_0)|+\varepsilon)\cdot |x-x_0|,\]
wobei jetzt auch $x=x_0$ erlaubt ist. Demnach wird Def.
\ref{Lipschitz-cont-at} erfüllt:
\[\exists \delta{>}0\colon\,\forall x\in U_\delta(x_0)\colon\,
|f(x)-f(x_0)| \le (|f'(x_0)|+\varepsilon)\cdot |x-x_0|.\;\qedsymbol\]
\end{Beweis}

\begin{Satz}\label{diff-bounded-Lipschitz-cont}
Eine differenzierbare Funktion ist genau dann Lipschitz"=stetig,
wenn ihre Ableitung beschränkt ist.
\end{Satz}
\begin{Beweis}
Wenn $f\colon I\to\R$ Lipschitz"=stetig ist, dann gibt es $L$ mit
\[\left|\frac{f(b)-f(a)}{b-a}\right|\le L\]
für alle $a,b\in D$ mit $a\ne b$. Daraus folgt
\[|f'(a)| = \left|\lim_{b\to a} \frac{f(b)-f(a)}{b-a}\right|
= \lim_{b\to a} \left|\frac{f(b)-f(a)}{b-a}\right|
\le L.\]
Demnach ist die Ableitung beschränkt.

Sei nun umgekehrt die Ableitung beschränkt. Für $a,b\in I$ mit $a\ne b$
gibt es nach dem Mittelwertsatz ein $x_0\in(a,b)$, so dass
\[|f'(x_0)| = \left|\frac{f(b)-f(a)}{b-a}\right|.\]
Da die Ableitung beschränkt ist gibt es ein Supremum
$L = \sup_{x\in I} |f'(x)|$. Demnach ist $|f'(x)|\le L$ für alle $x$.
Es ergibt sich
\[\left|\frac{f(b)-f(a)}{b-a}\right|\le L|b-a| \implies |f(b)-f(a)|\le L|b-a|.\]
Nun darf auch $a=b$ gewählt werden.\;\qedsymbol
\end{Beweis}

\begin{Satz}\label{diff-compact-Lipschitz-cont}
Eine auf einem kompakten Intervall $[a,b]$ definierte stetig
differenzierbare Funktion ist Lipschitz"=stetig.
\end{Satz}
\begin{Beweis}
Sei $f\colon [a,b]\to\R$ stetig differenzierbar. Dann ist $f'(x)$ stetig.
Nach dem Satz vom Minimum und Maximum ist $|f'(x)|$ beschränkt. Nach
Satz \ref{diff-bounded-Lipschitz-cont} muss $f$ Lipschitz"=stetig
sein.\;\qedsymbol
\end{Beweis}

\begin{Korollar}
Eine stetig differenzierbare Funktion ist lokal Lipschitz"=stetig.
\end{Korollar}
\begin{Beweis}
Sei $f\colon D\to\R$ stetig differenzierbar. Sei $[a,b]\in D$. Sei
$x_0\in [a,b]$. Die Einschränkung von $f$ auf $[a,b]$ ist
Lipschitz"=stetig nach Satz \ref{diff-compact-Lipschitz-cont}.
Dann ist auch die Einschränkung von $f$ auf
$U_\varepsilon(x_0)\subseteq [a,b]$ Lipschitz"=stetig.\;\qedsymbol
\end{Beweis}

\begin{Satz}
Es gibt differenzierbare Funktionen, die nicht überall lokal
Lipschitz"=stetig sind.
\end{Satz}
\begin{Beweis}
Aus Satz \ref{diff-bounded-Lipschitz-cont} ergibt sich also
Kontraposition, dass eine Funktion mit unbeschränkter Ableitung
nicht Lipschitz"=stetig sein kann.

Ist $f\colon D\to\R$ an jeder Stelle differenzierbar und ist $f'$
in jeder noch so kleinen Umgebung der Stelle $x_0$ unbeschränkt, dann
kann $f$ also in der Nähe dieser Stelle auch nicht lokal
Lipschitz"=stetig sein.

Ein Beispiel für eine solche Funktion ist $f\colon{}[0,\infty)\to\R$
mit
\[f(0):=0\quad \text{und}\quad f(x):=x^{3/2}\cos\Big(\tfrac{1}{x}\Big).\]
Einerseits gilt
\[f'(0) = \lim_{h\to 0}\frac{f(0+h)-f(0)}{h} = \lim_{h\to 0}\frac{f(h)}{h}
= \lim_{h\to 0} (h^{1/2}\cos\Big(\tfrac{1}{h}\Big)) = 0.\]
Die Funktion ist also an der Stelle $x=0$ differenzierbar.
Andererseits gilt nach den Ableitungsregeln%
\[f'(x) = \frac{3}{2}\sqrt{x}\cos\Big(\tfrac{1}{x}\Big)+\frac{1}{\sqrt{x}}\sin\Big(\tfrac{1}{x}\Big).\]
für $x>0$. Der Term $\tfrac{1}{\sqrt{x}}$ erwirkt für $x\to 0$ immer
größere Maxima von $|f'(x)|$. Daher kann $f$ in der Nähe von $x=0$ nicht
lokal Lipschitz"=stetig sein.\;\qedsymbol
\end{Beweis}

\begin{Satz}
Sei $f\colon\R\to\R$ differenzierbar und $f(x)$ konvergent
für $x\to\infty$. Ist außerdem $f'$  Lipschitz-stetig,
zieht dies $f'(x)\to 0$ für $x\to\infty$ nach sich.
\end{Satz}
\begin{Beweis}
Gemäß dem cauchyschen Konvergenzkriterium gibt es zu jedem
$\varepsilon>0$ eine Stelle $x_0$, so dass
\begin{equation}
|f(b)-f(a)| < \varepsilon
\end{equation}
für alle $a,b$ mit $x_0 < a \le b$. Nun ist $f'$ aufgrund
der Lipschitz-Stetigkeit erst recht stetig, womit
\begin{equation}
\bigg|\int_a^b f'(x)\,\mathrm dx\bigg| = |f(b)-f(a)|
\end{equation}
laut dem Fundamentalsatz gilt. Gezeigt wird nun, dass $|f'(a)|$
beschränkt ist. Sei dazu $L$ die Lipschitz-Konstante. Ohne
Beschränkung der Allgemeinheit sei $f'(a)>0$. Fallen darf $f'$ maximal
mit dem Anstieg $-L$. Geschieht dies linear bis zur Nullstelle $b$,
ergibt sich ein rechtwinkliges Dreieck mit dem Flächeninhalt
\begin{equation}
\frac{1}{2L} f'(a)^2 = \int_a^b f'(x)\,\mathrm dx < \varepsilon.
\end{equation}
Demnach ist $f'(a) < \sqrt{2L\varepsilon}$. Weil dies für alle $a>x_0$
gilt, muss $f'$ jede Beschränkung unterbieten, womit
der Beweis der Behauptung erbracht ist.\;\qedsymbol
\end{Beweis}

\noindent
Die Diskussion Gegenbeispiels $f(0):=0$, $f(x):=\sin(x^2)/x$ macht
ersichtlich, dass die Aussage ohne Lipschitz-Stetigkeit nicht einmal
für glatte Funktionen gilt.

\newpage
\section{Differentialrechnung}

\subsection{Ableitungsregeln}

\begin{Definition}[diff: differenzierbar, Ableitung]%
\label{diff}\index{differenzierbar}\index{Ableitung}
Eine Funktion $f\colon D\to\R$ heißt differenzieraber an der Stelle
$x_0\in D$, wenn der Grenzwert%
\[f'(x_0) = \lim_{x\to x_0}\frac{f(x)-f(x_0)}{x-x_0}
= \lim_{h\to 0}\frac{f(x_0+h)-f(x_0)}{h}\]
existiert. Man nennt $f'(x_0)$ die Ableitung von $f$ an der Stelle
$x_0$.
\end{Definition}

\begin{Satz}\index{Produktregel}
Sei $I$ ein Intervall und $f,g\colon I\to\R$. Sind $f,g$
differenzierbar an der Stelle $x\in I$, dann ist auch%
\begin{align}
f+g&\;\text{dort differenzierbar mit}\;(f+g)'(x)=f'(x)+g'(x),\\
f-g&\;\text{dort differenzierbar mit}\;(f-g)'(x)=f'(x)-g'(x),\\
\label{eq:diff-mul}
fg&\;\text{dort differenzierbar mit}\;(fg)'(x)=f'(x)g(x)+f(x)g'(x).
\end{align}
\end{Satz}

\begin{Beweis} Es gilt
\begin{gather}
(f+g)'(x)
= \lim_{h\to 0}\frac{(f+g)(x+h)-(f+g)(x)}{h}\\
= \lim_{h\to 0}\frac{(f(x+h)+g(x+h))-(f(x)+g(x))}{h}\\
= \lim_{h\to 0}\bigg(\frac{f(x+h)-f(x)}{h}+\frac{g(x+h)-g(x)}{h}\bigg)\\
= \lim_{h\to 0}\frac{f(x+h)-f(x)}{h}+\lim_{h\to 0}\frac{g(x+h)-g(x)}{h}
= f'(x)+g'(x).
\end{gather}
Da die Grenzwerte auf der rechten Seite nach Voraussetzung existieren,
muss auch der Grenzwert der Summe existieren.
Die Rechnung für die Subtraktion ist analog.

Bei der Multiplikation wird ein Nullsummentrick angewendet:
\begin{gather}
g(x)f'(x)+f(x)g'(x)
= g(x)\lim_{h\to 0}\frac{f(x+h)-f(x)}{h}
+ f(x)\lim_{h\to 0}\frac{g(x+h)-g(x)}{h}\\
= \lim_{h\to 0}\bigg[g(x+h)\frac{f(x+h)-f(x)}{h}\bigg]
+ \lim_{h\to 0}\bigg[f(x)\frac{g(x+h)-g(x)}{h}\bigg]\\
= \lim_{h\to 0}\frac{f(x+h)g(x+h)-f(x)g(x+h)}{h}
+ \lim_{h\to 0}\frac{f(x)g(x+h)-f(x)g(x)}{h}\\
= \lim_{h\to 0}\frac{f(x+h)g(x+h)-f(x)g(x+h)+f(x)g(x+h)-f(x)g(x)}{h}\\
= \lim_{h\to 0}\frac{f(x+h)g(x+h)-f(x)g(x)}{h}
= \lim_{h\to 0}\frac{(fg)(x+h)-(fg)(x)}{h}
= (fg)'(x).
\end{gather}
Hierbei wurde $\lim_{h\to 0}g(x+h)=g(x)$ benutzt, was richtig ist,
weil $g$ an der Stelle $x$ differenzierbar ist und dort somit ganz
sicher stetig.\;\qedsymbol
\end{Beweis}

\newpage
\begin{Satz}
Sei $I$ ein Intervall. Sind $f,g\colon I\to\R$ an der Stelle
$x$ differenzierbar und ist $g(x)\ne 0$, dann
ist auch $f/g$ differenzierbar und es gilt
\begin{equation}
\bigg(\frac{f}{g}\bigg)'(x) = \frac{f'(x)g(x)-f(x)g'(x)}{g(x)^2}.
\end{equation}
\end{Satz}
\begin{Beweis}
Nach der Produktregel \eqref{eq:diff-mul} gilt
\begin{equation}
0 = 1' = \bigg(g\cdot\frac{1}{g}\bigg)'
= g'\cdot\frac{1}{g}+g\cdot \bigg(\frac{1}{g}\bigg)'.
\end{equation}
Umstellen bringt $(1/g)'(x)=-g'(x)/g(x)^2$. Nochmalige Anwendung der
Produktregel \eqref{eq:diff-mul} bringt
\begin{align}
\bigg(\frac{f}{g}\bigg)'(x)
&= \bigg(f\cdot\frac{1}{g}\bigg)'(x)
= f'(x)\cdot\frac{1}{g(x)}+f(x)\bigg(\frac{1}{g}\bigg)'(x)\\
&= \frac{f'(x)}{g(x)}-\frac{f(x)g'(x)}{g(x)^2}
= \frac{f'(x)g(x)-f(x)g'(x)}{g(x)^2}.\;\qedsymbol
\end{align}
\end{Beweis}

\begin{Satz}\label{diff-power}
Für $f\colon\R\to\R$, $f(x):=x^n$ mit $n\in\N$ gilt
$f'(x)=nx^{n-1}$.
\end{Satz}
\begin{Beweis}[Beweis 1]
Heranziehung des binomischen Lehrsatzes bringt
\begin{align}
f'(x) &= \lim_{h\to 0}\frac{(x+h)^n-x^n}{h}
= \lim_{h\to 0}\frac{\sum_{k=0}^n\binom{n}{k}x^{n-k} h^k-x^n}{h}\\
&= \lim_{h\to 0}\bigg(nx^{n-1}+\sum_{k=2}^n\binom{n}{k}x^{n-k}h^{k-1}\bigg)
= nx^{n-1}.\;\qedsymbol
\end{align}
\end{Beweis}
\begin{Beweis}[Beweis 2]
Induktiv. Der Induktionsanfang $\tfrac{\mathrm d}{\mathrm dx}x=1$ ist klar.
Induktionsschritt mittels Produktregel \eqref{eq:diff-mul}:
\begin{align}
\tfrac{\mathrm d}{\mathrm dx} x^n = \tfrac{\mathrm d}{\mathrm dx} (x\cdot x^{n-1})
= x^{n-1}+x\tfrac{\mathrm d}{\mathrm dx}x^{n-1}
= x^{n-1}+(n-1)x^{n-1} = nx^{n-1}.\;\qedsymbol
\end{align}
\end{Beweis}

\begin{Satz}
Für $f\colon\R{\setminus}\{0\}\to\R$, $f(x):=x^n$ mit $n\in\Z$
gilt $f'(x)=nx^{n-1}$.
\end{Satz}
\begin{Beweis}
Der Fall $n=0$ ist trivial und $n\ge 1$ wurde schon in Satz
\ref{diff-power} gezeigt. Sei nun $a\in\N$ und $n=-a$. Nach der
Produktregel \eqref{eq:diff-mul} und Satz \ref{diff-power} gilt
\begin{equation}
0 = \tfrac{\mathrm d}{\mathrm dx} 1
= \tfrac{\mathrm d}{\mathrm dx} (x^a x^{-a})
= x^{-a}\tfrac{\mathrm d}{\mathrm dx} x^a+x^a\tfrac{\mathrm d}{\mathrm dx} x^{-a}
= x^{-a}ax^{a-1}+x^a\tfrac{\mathrm d}{\mathrm dx} x^{-a}.
\end{equation}
Dividiert man nun durch $x^a$ und formt um, dann ergibt sich
\begin{equation}
\tfrac{\mathrm d}{\mathrm dx} x^{-a} = -ax^{-a-1}
\implies \tfrac{\mathrm d}{\mathrm dx} x^n = nx^{n-1}.\;\qedsymbol
\end{equation}
\end{Beweis}

\newpage
\subsection{Glatte Funktionen}

\begin{Satz}
Sei $f\colon\R\to\R$ eine Funktion mit der Eigenschaft
$f(x)=0$ für $x\le 0$ und $f(x)>0$ für $x>0$. Es gibt glatte Funktionen
mit dieser Eigenschaft, jedoch keine analytischen.
\end{Satz}

\begin{Beweis}
Wegen $f(x)=0$ für $x\le 0$ muss die linksseitige $n$-te Ableitung
an der Stelle $x=0$ immer verschwinden. Wenn die $n$-te Ableitung
stetig sein soll, muss auch die rechtsseitige Ableitung bei $x=0$
verschwinden. Da die Funktion glatt sein soll, muss das für jede
Ableitung gelten. Daher verschwindet die Taylorreihe an der Stelle
$x=0$. Da aber $f(x)>0$ für $x>0$, gibt es keine noch so kleine
Umgebung mit Übereinstimmung von $f$ und ihrer Taylorreihe.
Daher kann $f$ an der Stelle $x=0$ nicht analytisch sein.

Eine glatte Funktion lässt sich jedoch konstruieren:
\[f(x):=\begin{cases}
\ee^{-1/x}&\text{wenn}\;x>0,\\
0&\text{wenn}\;x\le 0.
\end{cases}\]
Ist nämlich $g(x)$ an einer Stelle glatt, dann ist
es nach Kettenregel, Produktregel und Summenregel auch $\ee^{g(x)}$.
Die $n$-te Ableitung lässt sich immer in der Form%
\[\sum\nolimits_k e^{g(x)}{r_k(x)}
= e^{g(x)}\sum\nolimits_k r_k(x) = e^{g(x)}r(x)\]
darstellen, wobei die $r_k(x)$ bzw. $r(x)$ in diesem Fall rationale
Funktionen mit Polstelle bei $x=0$ sind. Da aber $e^{-1/x}$ für
$x\to 0$ schneller fällt als jede rationale Funktion steigen kann,
muss die rechtsseitige Ableitung an der Stelle $x=0$ immer
verschwinden.\;\qedsymbol
\end{Beweis}

\subsection{Richtungsableitung}

\begin{Definition}[Richtungsableitung]
Sei $U\subseteq\R^n$ offen, $x\in U$ eine Stelle und $v\in\R^n$
ein Vektor. Man betrachte für ein kleines $\varepsilon>0$
die Parametergerade
\[\gamma\colon(-\varepsilon,\varepsilon)\to U,\quad \gamma(t):=x+tv.\]
Für eine Funktion $f\colon U\to\R$ ist die Zahl
\[D_v f(x) := (f\circ\gamma)'(0) = \lim_{h\to 0}\frac{f(x+hv)-f(x)}{h},\]
falls sie existiert, die Richtungsableitung von $f$ an der Stelle $x$ in
Richtung $v$.
\end{Definition}
\begin{Korollar}
Die Funktionen $f,g$ seien an der Stelle $x$ in Richtung $v$
differenzierbar. Sei $c$ eine reelle Zahl. Dann sind auch
$f+g$, $f-g$, $cf$, $fg$ differenzierbar und es gelten die
den üblichen Ableitungsregeln analogen Regeln
\begin{align*}
D_v(f+g)(x) &= D_v f(x)+D_v g(x),\\
D_v(f-g)(x) &= D_v f(x)+D_v g(x),\\
D_v(cf)(x) &= cD_v f(x),\\
D_v(fg)(x) &= g(x)D_v f(x) + f(x)D_v g(x).
\end{align*}
\end{Korollar}
\begin{Beweis}
Die Ableitungsregeln werden über die Definition
auf die Ableitungsregeln für gewöhnliche reelle Funktionen
zurückgeführt. So ist
\begin{align*}
D_v(f+g)(x) &= ((f+g)\circ\gamma)'(0)
= ((f\circ\gamma)+(g\circ\gamma))'(0)\\
&= (f\circ\gamma)'(0)+(g\circ\gamma)'(0)
= D_v f(x) + D_v g(x).
\end{align*}
Der Beweis der restlichen Regeln ist analog.\,\qedsymbol
\end{Beweis}

\begin{Korollar}[Kettenregel]\newlinefirst
Sei $g\colon\R\to\R$ differenzierbar und
$f$ differenzierbar an der Stelle $x$ in Richtung $v$. Dann ist
auch $g\circ f$ entsprechend differenzierbar, und es gilt
\[D_v(g\circ f)(x) = (g'\circ f)(x)\cdot D_v f(x).\]
\end{Korollar}
\begin{Beweis}
Die Regel ist gemäß der Definition auf die gewöhnliche Kettenregel
zurückführbar. Man bekommt
\begin{align*}
D_v(g\circ f)(x) = (g\circ f\circ\gamma)'(0)
= g'(f(\gamma(0)))\cdot (f\circ\gamma)'(0)
= g'(f(x))\cdot D_v f(x).\;\qedsymbol
\end{align*}
\end{Beweis}

\begin{Definition}[Partielle Ableitung]\newlinefirst
Sei $(\mathbf e_1,\ldots,\mathbf e_n)$ die Standardbasis. 
Die partielle Ableitung $\partial_k f(x)$ ist definiert als
die Richtungsableitung $D_v f(x)$ bezüglich $v=\mathbf e_k$.
\end{Definition}

\begin{Korollar}
Zur jeder gewöhnlichen Ableitungsregel besitzt die
Richtungsableitung eine analoge Regel.
\end{Korollar}
\strong{Vorbereitung.}
Sei $f=(f_1,\ldots,f_n)$ ein Tupel von Funktionen aus einem
Funktionenraum und sei entsprechend
$f(x):=(f_1(x),\ldots, f_n(x))$. Sei $p$ eine beliebige mehrstellige
Operation. Sei $\eta_p(f)(x)$ die punktweise Anwendung von $p$.
Ein Beispiel ist die Addition $p(y_1,y_2):=y_1+y_2$. Dann ist
$\eta_p(f_1,f_2)(x)=f_1(x)+f_2(x)$. Sei%
\[F(T)(f) := (T f_1,\ldots ,T f_n)\]
die komponentenweise Anwendung eines Operators $T$.
Sei $C_\gamma$ der durch $C_\gamma f := f\circ\gamma$ definierte
Kompositionsoperator. Allgemein gilt%
\[C_\gamma\circ\eta_p = \eta_p\circ F(C_\gamma).\]

\begin{Beweis}
Prämisse ist, dass der gewöhnliche Ableitungsoperator $D$ die Regel
\[D(\eta_p(f))(x) = (D\circ\eta_p)(f)(x) = R(f(x),F(D)(f)(x))\]
erfüllt. Für die Richtungsableitung von $\eta_p(f)$ gilt dann
\begin{gather*}
D_v(\eta_p(f))(x) = (\eta_p(f)\circ\gamma)'(0)
= (D\circ C_\gamma\circ\eta_p)(f)(0)
= (D\circ\eta_p\circ F(C_\gamma))(f)(0)\\
= (D\circ\eta_p)(F(C_\gamma)(f))(0)
= R(F(C_\gamma)(f)(0),F(D)(F(C_\gamma)(f))(0))\\
= R(f(x),F(D\circ C_\gamma)(f)(0))
= R(f(x),F(D_v)(f)(x)).\;\qedsymbol
\end{gather*}
\end{Beweis}

\noindent
Beispiele sind
\begin{gather*}
\begin{array}{ll}
p(y_1,y_2) = y_1+y_2, & R((y_1,y_2),(y_1',y_2')) = y_1'+y_2',\\[4pt]
p(y_1,y_2) = y_1 y_2, & R((y_1,y_2),(y_1',y_2')) = y_1'y_2 + y_1y_2',\\[4pt]
p(y) = cy, & R(y,y') = cy',\\[4pt]
p(y) = g(y), & R(y,y') = g'(y)y'.
\end{array}
\end{gather*}

\newpage
\section{Fixpunkt-Iterationen}%
\index{Fixpunkt-Iteration}

\begin{Definition}[Kontraktion]\index{Kontraktion}
Sei $(M,d)$ ein vollständiger metrischer Raum. Eine Abbildung
$\varphi\colon M\to M$ heißt Kontraktion, wenn sie
Lipschitz"=stetig mit Lipschitz"=Konstante $L<1$ ist, d.\,h.
\[d(\varphi(x),\varphi(y))<L\,d(x,y)\]
für alle $x,y\in M$.
\end{Definition}

\begin{Satz}[Fixpunktsatz von Banach]\label{Banach-fixed-point-theorem}%
\index{Fixpunktsatz von Banach}\index{Banach!Fixpunktsatz von}
Sei $(M,d)$ ein nichtleerer vollständiger metrischer Raum
und sei $\varphi\colon M\to M$ eine Kontraktion. Es gibt genau
einen Fixpunkt $x\in M$ mit $x=\varphi(x)$ und die Folge
$(x_n)\colon\N\to M$ mit $x_{n+1}=\varphi(x_n)$ konvergiert
gegen den Fixpunkt, unabhängig vom Startwert $x_0$.
\end{Satz}

\begin{Satz}[Hinreichendes Konvergenzkriterium]\label{diff-fixed-point-iter}
Sei $M=[a,b]$. Ist $\varphi\colon M\to M$ differenzierbar und gibt es
eine Zahl $r$ mit $|\varphi'(x)|<r<1$ für alle $x\in M$, dann
hat $\varphi$ genau einen Fixpunkt und die Folge $(x_n)$ mit $x_{n+1}=\varphi(x_n)$
konvergiert für jeden Startwert $x_0\in M$ gegen diesen Fixpunkt.
\end{Satz}
\begin{Beweis}
Nach Satz \ref{diff-bounded-Lipschitz-cont} ist eine differenzierbare
Funktion $\varphi$ mit beschränkter Ableitung auch Lipschitz"=stetig,
und $L=\sup_{x\in M}|\varphi'(x)|$ eine Lipschitz"=Konstante.
Wegen $|\varphi'(x)|<r$ muss $L\le r$ sein, und somit $L<1$.
D.\,h., $\varphi$ ist eine Kontraktion. Die Konvergenz der Folge
$(x_n)$ ist gemäß Satz \ref{Banach-fixed-point-theorem}
gewährleistet.\;\qedsymbol
\end{Beweis}

\begin{Satz}[Hinreichendes Konvergenzkriterium zum Newton-Verfahren]%
\index{Newton-Verfahren}\newlinefirst
Sei $f\colon [a,b]\to\R$ zweimal stetig differenzierbar und
$f'(x)\ne 0$ für alle $x$. Sei%
\[\varphi\colon [a,b]\to [a,b],\quad \varphi(x):=x-\frac{f(x)}{f'(x)}.\]
Man beachte $\varphi([a,b])\subseteq [a,b]$. Gilt für alle $x$ die Ungleichung%
\[|\varphi'(x)| = \bigg|\frac{f(x)f''(x)}{f'(x)^2}\bigg| < 1,\]
dann besitzt $f$ genau eine Nullstelle und die Folge $(x_n)$ mit
$x_{n+1}=\varphi(x_n)$ konvergiert gegen diese Nullstelle.
\end{Satz}

\begin{Beweis}
Gemäß den Ableitungsregeln ist $\varphi$ stetig differenzierbar
und es gilt%
\[\varphi'(x) = 1-\frac{f'(x)f'(x)-f(x)f''(x)}{f'(x)^2}
= \frac{f(x)f''(x)}{f'(x)^2}.\]
Da $|\varphi'(x)|$ stetig ist, gibt es nach dem Satz vom Minimum
und Maximum ein Maximum $M$ und nach Voraussetzung ist $M<1$.
Man setze nun $r:=(M+1)/2$. Dann ist $|\varphi'(x)|<r<1$.
Gemäß Satz \ref{diff-fixed-point-iter} konvergiert die Iteration
$(x_n)$ gegen den einzigen Fixpunkt von $\varphi$. Wegen $f'(x)\ne 0$
gilt dabei%
\[x = \varphi(x) = x-\frac{f(x)}{f'(x)} \iff \frac{f(x)}{f'(x)}=0\iff f(x)=0.\]
Der Fixpunkt von $\varphi$ ist also die einzige Nullstelle von $f$.\;\qedsymbol
\end{Beweis}



\chapter{Topologie}
\section{Grundbegriffe}
\subsection{Definitionen}

\begin{Definition}[Topologischer Raum]
Sei $X$ eine Menge und $T$ eine Menge von Teilmengen von $X$.
Man nennt das System $T$ eine Topologie und $(X,T)$ einen topologischen
Raum, falls die folgenden drei Axiome erfüllt sind:
\begin{enumerate}
\item Es gilt $\emptyset\in T$ und $X\in T$.
\item Sind $A,B\in T$, dann ist auch $A\cap B\in T$.
\item Sind die $A_i\in T$, dann ist auch $\bigcup_I A_i\in T$, wobei $I$ unendlich sein darf.
\end{enumerate}
Die Elemente der Topologie nennt man offene Mengen.
\end{Definition}

\begin{Definition}[Abgeschlossene Menge]
Sei $X$ ein topologischer Raum. Eine Menge $M\subseteq X$
nennt man abgeschlossen, wenn das Komplement $X\setminus M$ offen ist.
\end{Definition}

\begin{Definition}[nh-filter: Umgebungsfilter]%
\label{def:nh-filter}\index{Umgebungsfilter}
Zu einem Punkt $x\in X$ ist
\[\underline U(x) := \{U{\subseteq}X\mid
\exists O\colon O\in T\land x\in O\land O\subseteq U\}\]
der Umgebungsfilter. Eine Menge $U\in\underline U(x)$ heißt
Umgebung von $x$.
\end{Definition}

\begin{Definition}[int: Inneres]\mbox{}\\*
\label{def:int}\index{offener Kern}%
Das Innere von $M$, auch offener Kern genannt, ist
\[\operatorname{int}(M) := \{x\in M\mid M\in \underline U(x)\}.\]
\end{Definition}

\begin{Definition}[ext: Äußeres]\mbox{}\\*
Sei $X$ ein topologischer Raum und $M\subseteq X$. Das Äußere von $M$ ist
\[\operatorname{ext}(M) := \operatorname{int}(X\setminus M) = \operatorname{int}(M^\comp).\]
\end{Definition}

\begin{Definition}[Abgeschlossene Hülle]\mbox{}\\*
Sei $X$ ein topologischer Raum und $M\subseteq X$. Die abgeschlossene Hülle von $M$ ist%
\[\overline M := X\setminus\operatorname{ext}(M) = \operatorname{int}(M^\comp)^\comp.\]
\end{Definition}

\begin{Definition}[Rand]
Der Rand einer Menge $M$ ist
\[\partial M := \overline M\setminus\operatorname{int}(M)
= \overline M\cap\operatorname{int}(M)^\comp.\]
\end{Definition}

\begin{Definition}[Teilraumtopologie]\mbox{}\\*
Sei $(X,T)$ ein topologischer Raum und $M\subseteq X$. Man bezeichnet
\[T|M := \{A\cap M\mid A\in T\}.\]
als Teilraumtopologie und $(M,T|M)$ als Teilraum.
\end{Definition}

\begin{Definition}[Diskrete Topologie]\mbox{}\\*
Man sagt, ein topologischer Raum $(X,T)$ habe die diskrete Topologie $T$,
wenn $T$ die Potenzmenge von $X$ ist -- wenn also jede Teilmenge
von $X$ offen ist.
\end{Definition}

\subsection{Stetige Abbildungen}

\begin{Definition}[Stetige Abbildung]\mbox{}\\*
Seien $X$ und $Y$ topologischen Räume. Eine Abbildung $f\colon X\to Y$
heißt stetig, wenn unter ihr das Urbild einer offenen Menge stets
wieder offen ist.
\end{Definition}

\begin{Korollar}
Sei $M\subseteq X$. Ist $f\colon X\to Y$ stetig, so ist
die Einschränkung $f|_M$ stetig.
\end{Korollar}
\begin{Beweis}
Vorgelegt ist ein offenes $B\subseteq Y$. Laut Prämisse ist $A=f^{-1}(B)$
ebenfalls offen. Per Definition der Einschränkung gilt
\[(f|_M)^{-1}(B) = M\cap f^{-1}(B) = M\cap A.\]
Laut Definition ist $M\cap A$ ein Element der Teilraumtopologie $T(X)|M$.\,\qedsymbol
\end{Beweis}

\begin{Korollar}
Auf jedem topologischen Raum ist die identische Abbildung stetig.
\end{Korollar}
\begin{Beweis}
Ihre Urbildoperation ist ebenfalls eine identische Abbildung. Ergo
verbleiben offene Mengen unverändert, also offen.\,\qedsymbol
\end{Beweis}

\begin{Korollar}
Die Verkettung stetiger Abbildungen ist stetig.
\end{Korollar}
\begin{Beweis}
Seien $f\colon X\to Y$ und $g\colon Y\to Z$ stetig.
Gemäß Satz \ref{preimg-chain} gilt $(g\circ f)^{-1}(M)=f^{-1}(g^{-1}(M))$.
Ist $M$ offen, so auch $g^{-1}(M)$ und infolge $f^{-1}(g^{-1}(M))$.
Ergo ist $g\circ f$ stetig.\,\qedsymbol
\end{Beweis}

\begin{Definition}[Homöomorphismus]\mbox{}\\*
Seien $X$ und $Y$ topologische Räume. Eine bijektive Abbildung
$f\colon X\to Y$ heißt Homöomorphismus, wenn sowohl $f$ als auch
$f^{-1}$ stetig sind.
\end{Definition}

\newpage
\subsection{Elementares}

\begin{Korollar}\label{partition-int-bd-ext}
Sei $X$ ein topologischer Raum. Für jede Menge $M\subseteq X$ ist%
\[X = \operatorname{int}(M)\cup\partial M\cup\operatorname{ext}(M)\]
eine disjunkte Zerlegung.
\end{Korollar}
\begin{Beweis} Sei $A:=\operatorname{int}(M)$ und $B:=\operatorname{ext}(M)$. Dann ist
\begin{gather*}
\operatorname{int}(M)\cup\partial M\cup\operatorname{ext}(M)
= A\cup B^\comp\cap A^\comp\cup B
= A\cup A^\comp\cup B = X\cup B = X.
\end{gather*}
Nun verbleibt zu prüfen, dass die Mengen paarweise disjunkt sind. Wir haben%
\begin{align*}
\operatorname{int}(M)\cap\partial M &= A\cap B^\comp\cap A^\comp = \emptyset,\\
\operatorname{ext}(M)\cap\partial M &= B\cap B^\comp\cap A^\comp = \emptyset.
\end{align*}
Wegen $M\cap M^\comp=\emptyset$ ist erst recht $A\cap B=\emptyset$,
denn $A\subseteq M$ und $B\subseteq M^\comp$.\,\qedsymbol
\end{Beweis}

\begin{Korollar}
Für jede Menge $A\subseteq X$ gilt $\operatorname{int}(A)^\comp\cup A = X$.
\end{Korollar}
\begin{Beweis}
Setze $B:=\operatorname{int}(A)$. Gemäß Definition gilt $B\subseteq A$,
was äquivalent zu $A\cap B = B$ ist. Damit ergibt sich
\[B^\comp\cup A = (A\cap B)^\comp\cup A = A^\comp\cup B^\comp\cup A
= X\cup B^\comp = X.\,\qedsymbol\]
\end{Beweis}

\begin{Satz}
Das Innere von $M$ ist die Vereinigung der offenen Teilmengen
von $M$, kurz%
\[\operatorname{int}(M) = \bigcup_{O\in 2^M\cap T} O.\]
\end{Satz}

\begin{Beweis}
Nach Def. \ref{def:seteq} (seteq) und Def. \ref{def:int} (int)
expandieren:%
\[\forall x\colon [x\in M\land M\in\underline U(x)
\iff x\in\bigcup_{O\in 2^M\cap T} O].\]
Den äußeren Allquantor brauchen wir nicht weiter mitschreiben, da alle
freien Variablen automatisch allquantifiziert werden.
Nach Def. \ref{def:nh-filter} (nh-filter) weiter expandieren, wobei die
Bedingung $U\subseteq X$ als tautologisch entfallen kann,
weil $X$ die Grundmenge ist. Auf der rechten Seite wird nach Def.
\ref{def:union} (union) expandiert. Es ergibt sich:
\[x\in M\land (\exists O\colon O\in T\land x\in O\land O\subseteq M)
\iff (\exists O\colon O\subseteq M\land O\in T\land x\in O).\]
Wegen $A\land(\exists x\colon P(x))\iff (\exists x\colon A\land P(x))$ ergibt
sich auf der linken Seite:
\[\exists O\colon x\in M\land O\in T\land x\in O\land O\subseteq M.\]
Wenn aber $O\subseteq M$ erfüllt sein muss, gilt
$x\in O\implies x\in M$. Demnach kann $x\in M$ entfallen.
Auf beiden Seiten steht dann die gleiche Bedingung.\,\qedsymbol
\end{Beweis}

\begin{Satz}\label{boundary-point-char}
Ein Punkt $p$ liegt genau dann auf dem Rand einer Menge $M$, wenn
jede Umgebung von $p$ mindestens einen Punkt aus $M$ und
einen Punkt aus dem Komplement von $M$ enthält.
\end{Satz}

\newpage
\subsection{Zusammenhang}

\begin{Definition}[Zusammenhängender Raum]\mbox{}\\*
Ein topologischer Raum $(X,T)$ heißt zusammenhängend, wenn er sich
nicht in zwei disjunkte nichtleere offene Mengen zerlegen lässt. Gemeint ist
\[\forall A,B\in T\colon A\ne\emptyset\land B\ne\emptyset\land A\cap B = \emptyset \implies A\cup B\ne X.\]
\end{Definition}
\strong{Bemerkung.} Ein Raum $(X,T)$ ist demnach unzusammenhängend, wenn
Zeugen $A,B$ für die Aussage
\[\exists A,B\in T\colon A\ne\emptyset\land B\ne\emptyset\land A\cap B = \emptyset \land A\cup B = X\]
gefunden sind.

\begin{Satz}\label{connected-iff-cont-is-const}
Sei $X$ ein topologischer Raum und $\{0,1\}$ der topologische
Raum mit der diskreten Topologie. Es ist $X$ genau dann
zusammenhängend, wenn jede stetige Abbildung $f\colon X\to\{0,1\}$
konstant sein muss.
\end{Satz}
\begin{Beweis}
Sei $f$ stetig und nicht"=konstant. Man betrachte die Fasern
$A:=f^{-1}(\{0\})$ und $B:=f^{-1}(\{1\})$. Weil $f$, wie gerade
gefordert, beide Werte annehmen muss, gilt $A\ne\emptyset$ und
$B\ne\emptyset$. Zudem sind $A,B$ offen, weil sie die Urbilder
offener Mengen sind unter stetigem $f$ sind. Urbilder disjunkter
Mengen sind immer disjunkt, womit $A\cap B=\emptyset$ gilt. Zudem gilt
\[A\cup B = f^{-1}(\{0\})\cup f^{-1}(\{1\}) = f^{-1}(\{0\}\cup \{1\}) = X.\]
Somit ist ein Gegenbeispiel konstruiert, so dass $X$ unzusammenhängend
sein muss.

Sei $X$ unzusammenhängend. Es existieren somit Zeugen $A,B\in T$
mit $A\ne\emptyset$, $B\ne\emptyset$, $A\cup B=\emptyset$ und
$A\cup B = X$. Sei $f$ definiert durch
$f(x):=0$ für alle $x\in A$ und $f(x):=1$ für alle $x\in B$.
Nun ist $f$ keine konstante Abbildung, da sie auf nichtleeren $A,B$
unterschiedliche Werte annimmt. Wohl aber ist $f$ stetig,
wie im Folgenden noch durchgerechnet wird. Die diskrete Topologie
von $\{0,1\}$ ist ihre Potenzmenge. Es bestätigt sich
\begin{gather*}
f^{-1}(\emptyset) = \emptyset\in T,\\
f^{-1}(\{0\}) = A\in T,\\
f^{-1}(\{1\}) = B\in T,\\
f^{-1}(\{0,1\}) = f^{-1}(\{0\}\cup\{1\}) = f^{-1}(\{0\})\cup f^{-1}(\{1\})
= A\cup B = X\in T.\,\qedsymbol\\
\end{gather*}
\end{Beweis}

\begin{Korollar}
Die Vereinigung zweier zusammenhängender nichtdisjunkter Räume
ist ein zusammenhängender Raum.
\end{Korollar}
\begin{Beweis}
Seien $X,Y$ zusammenhängend und sei $X\cap Y\ne\emptyset$.
Laut Prämisse existiert mindestens ein $p\in X\cap Y$. Sei
$f\colon X\cup Y\to\{0,1\}$ stetig. Nun ist $f|_X$ stetig und
konstant mit $f(x)=f(p)$ für alle $x\in X$, da $X$ zusammenhängend
ist. Entsprechend ist $f|_Y$ stetig und konstant mit $f(y)=f(p)$
für alle $y\in Y$. Ergo ist $f$ auf ganz $X\cup Y$ konstant $f(p)$.
Laut Satz \ref{connected-iff-cont-is-const} muss $X\cup Y$
also zusammenhängend sein.\,\qedsymbol
\end{Beweis}

\newpage
\begin{Korollar}
Ein topologischer Raum $X$ ist genau dann zusammenhängend,
wenn mit Ausnahme von $\emptyset$ und $X$ keine Teilmenge
von $X$ sowohl offen als auch abgeschlossen ist.
\end{Korollar}
\begin{Beweis}
Es existiere außer $\emptyset,X$ kein offenes $A$ mit offenem
$X\setminus A$. Sei $f\colon X\to\{0,1\}$ stetig. Angenommen, $f$
wäre nicht konstant. Dann gäbe es die beiden nichtleeren offenen Mengen
$A:=f^{-1}(\{0\})$ und $B:=f^{-1}(\{1\})$. Weil außerdem $X=A\cup B$
eine disjunkte Zerlegung wäre, gälte $B=X\setminus A$. Dies steht
im Widerspurch zur Prämisse, womit $f$ konstant sein muss.
Ergo ist $X$ laut Satz \ref{connected-iff-cont-is-const} ein
zusammenhängender Raum.

Es existiere nun offnes, von $\emptyset,X$ verschiedenes $A$ mit
offenem $B:=X\setminus A$, womit neben $A\ne\emptyset$ auch
$B\ne\emptyset$ gilt. Wegen $A\cap B=\emptyset$ und $A\cup B=X$
gilt $X$ als in zwei nichtleere disjunkte offene Mengen zerlegt.
Ergo ist $X$ unzusammenhängend.\,\qedsymbol
\end{Beweis}

\begin{Korollar}
Ist $X$ zusammenhängend und $f\colon X\to Y$ stetig, dann
ist der Teilraum $f(X)$ von $Y$ ebenfalls zusammenhängend.
\end{Korollar}
\begin{Beweis}
Sei für die Kontraposition $f(X)$ unzusammenhängend. Hiermit existiert
eine disjunkte Zerlegung in offene nichtleere Mengen $A,B$ mit
$A\cup B=f(X)$. Aufgrund der elementaren Eigenschaften der
Urbildoperation hat man mit
\[X = f^{-1}(A\cup B) = f^{-1}(A)\cup f^{-1}(B),\qquad
f^{-1}(A)\cap f^{-1}(B) = \emptyset\]
eine disjunkte Zerlegung von $X$. Wegen $A\subseteq f(X)$ und
$B\subseteq f(X)$ gilt hierbei $f^{-1}(A)\ne\emptyset$ und
$f^{-1}(B)\ne\emptyset$. Weil $f$ stetig ist, sind $f^{-1}(A)$
und $f^{-1}(B)$ offen. Somit ist für $X$ eine Zerlegung
in zwei nichtleere disjunkte offene Mengen bezeugt. Ergo ist
$X$ ebenfalls unzusammenhängend.\,\qedsymbol
\end{Beweis}


\newpage
\section{Metrische Räume}
\subsection{Metrische Räume}
\begin{Definition}[metric-space: metrischer Raum]%
\index{metrischer Raum}\label{metric-space}
Man bezeichet $(M,d)$ mit $d\colon M^2\to\R$ genau dann als
metrischen Raum, wenn die folgenden Axiome erfüllt sind:
\begin{align*}
\text{(M1)}\quad & d(x,y)=0\iff x=y, &&\text{(Gleichheit abstandsloser Punkte)}\\
\text{(M2)}\quad & d(x,y)=d(y,x), &&\text{(Symmetrie)}\\
\text{(M3)}\quad & d(x,y)\le d(x,z)+d(z,y). &&\text{(Dreiecksungleichung)}
\end{align*}
\end{Definition}

\begin{Definition}[open-ep-ball: offene Epsilon-Umgebung]\mbox{}\\
Für einen metrischen Raum $(M,d)$ und $p\in M$:
\[U_\varepsilon(p) := \{x\mid d(p,x)<\varepsilon\}.\]
\end{Definition}
Bemerkung: Unter einer Epsilon-Umgebung ohne weitere Attribute
versteht man immer eine offene Epsilon-Umgebung.

\begin{Satz}[Konstruktion disjunkter Epsilon-Umgebungen]%
\label{construction-disjoint-ep-balls}
Sei $(M,d)$ ein metrischer Raum und $p,q\in M$ mit $p\ne q$.
Betrachte die Streckenzerlegung $d(p,q)=A+B$. Für $a\le A$ und
$b\le B$ sind die Epsilon-Umgebungen $U_a(p)$ und $U_b(q)$ disjunkt.
\end{Satz}

\begin{Beweis}
Angenommen $U_a(p)$ und $U_b(q)$ wären nicht disjunkt, dann gäbe
es mindestens ein $x$ mit $x\in U_a(p)$ und $x\in U_b(q)$, d.\,h.
$d(p,x)<a$ und $d(q,x)<b$. Addition der beiden Ungleichungen
bringt
\[d(p,x)+d(q,x)<a+b\le d(p,q).\]
Gemäß der Dreiecksungleichung Def. \ref{metric-space} Axiom (M3) gilt
nun aber
\[d(p,q)\le d(p,x)+d(q,x)\]
für alle $x$. Sei $c:=d(p,x)+d(q,x)$. Wir erhalten damit nun
$c<a+b\le c$ und somit den Widerspruch $c<c$.\,\qedsymbol
\end{Beweis}

\begin{Korollar}[Unterschiedliche Punkte eines metrischen Raumes
besitzen disjunkte Epsilon-Umgebungen]
Sei $(M,d)$ ein metrischer Raum und $p,q\in M$.
Wenn $p\ne q$ ist, dann gibt es disjunkte offene
Epsilon-Umgebungen $U_a(p)$ und $U_b(q)$.
\end{Korollar}

\begin{Beweis}
Folgt trivial aus Satz \ref{construction-disjoint-ep-balls}.
Wähle speziell z.\,B. $a=b=d(p,q)/2$.\,\qedsymbol
\end{Beweis}

\subsection{Normierte Räume}
\begin{Definition}[normed-space: normierter Raum]%
\label{def:normed-space}\index{normierter Raum}\index{Dreiecksungleichung}
Sei $V$ ein Vektorraum über dem Körper der rellen oder komplexen
Zahlen. Sei $N(x)=\|x\|$ eine Abbildung, die jedem $x\in V$ eine
reelle Zahl zuordnet. Man nennt $(V,N)$ genau dann einen
normierten Raum, wenn die folgenden Axiome erfüllt sind:
\begin{align*}
\text{(N1)}\quad &\|x\|=0\iff x=0,&&\text{(Definitheit)}\\
\text{(N2)}\quad &\|\lambda x\|=|\lambda|\|x\|,&&\text{(betragsmäßige Homogenität)}\\
\text{(N3)}\quad &\|x+y\| \le \|x\|+\|y\|.&&\text{(Dreiecksungleichung)}
\end{align*}
\end{Definition}

\begin{Satz}[umgekehrte Dreiecksungleichung]%
\label{rev-tineq}\index{umgekehrte Dreiecksungleichung}%
\index{Dreiecksungleichung!umgekehrte}
In jedem normierten Raum gilt
\[|\|x\|-\|y\|| \le \|x-y\|.\]
\end{Satz}
\begin{Beweis}
Auf beiden Seiten von Def. \ref{def:normed-space} (normed-space)
Axiom (N3) wird $\|y\|$ subtrahiert.
Es ergibt sich
\[\|x+y\| - \|y\| \le \|x\|.\]
Substitution $x:=x-y$ bringt nun
\[\|x\| - \|y\| \le \|x-y\|.\]
Vertauscht man nun $x$ und $y$, dann ergibt sich
\[\|y\|-\|x\| \le \|y-x\| \iff -(\|x\|-\|y\|)\le \|x-y\|.\]
Wir haben nun $a\le b$ und $-a\le b$,
wobei $a:=\|x\|-\|y\|$ und $b:=\|x-y\|$ ist. Multipliziert
man die letzte Ungleichung mit $-1$, dann ergibt sich $a\ge -b$.
Somit ist $-b\le a\le b$, kurz $|a|\le b$.\,\qedsymbol
\end{Beweis}

\subsection{Homöomorphien}
\begin{Satz}[Verallgemeinerung des Zwischenwertsatzes]%
\label{intermediate-value-general}\mbox{}\\*
Ist $f\colon X\to Y$ eine stetige Abbildung zwischen topologischen
Räumen und $A\subseteq X$ ein zusammenhängender Teilraum,
dann ist auch $f(A)$ zusammenhängend.
\end{Satz}

\begin{Satz}
Eine injektive Abbildung $f\colon\R_{\ge 0}\to\R$ kann nicht stetig sein.
\end{Satz}
\begin{Beweis}
Da $f$ injektiv ist, ist die Rechnung
\[f(\R_{>0}) = f(\R_{\ge 0}\setminus\{0\})
= f(\R_{\ge 0})\setminus f(\{0\}) = \R\setminus\{f(0)\}\]
gültig gemäß Satz \ref{inj-img-setminus}. Da $\R_{>0}$ zusammenhängend
ist, $\R\setminus\{f(0)\}$ aber nicht, kann $f$ laut Satz
\ref{intermediate-value-general} nicht stetig sein.\,\qedsymbol
\end{Beweis}

\section{Übungen}

\begin{Satz}
Sei $M\subseteq\R^n$. Das Innere von $M$ besitze den Punkt $p$,
das Äußere den Punkt $q$. Dann schneidet das Bild jedes Weges von $p$
nach $q$ den Rand von $M$.
\end{Satz}
\begin{Beweis}[Beweis 1]
Sei $\gamma\colon [0,1]\to\R^n$ so ein Weg mit
$\gamma(0)=p$ und $\gamma(1)=q$. Angenommen, das Bild schneidet
den Rand nicht. Das heißt, $\gamma([0,1])\cap\partial M = \emptyset$,
oder äquivalent $\gamma^{-1}(\partial M)=\emptyset$.
Allgemein ist
\[\R^n = \operatorname{int}(M) \cup \partial M \cup \operatorname{ext}(M)\]
laut Korollar \ref{partition-int-bd-ext} eine disjunkte Zerlegung.
Gemäß Satz \ref{preimg-dl} (preimg-dl) gilt
\[[0,1] = \gamma^{-1}(\R^n) = \gamma^{-1}(\operatorname{int}(M))
\cup \gamma^{-1}(\partial M)\cup \gamma^{-1}(\operatorname{ext}(M)),\]
was auch eine disjunkte Zerlegung ist, weil je zwei disjunkte Mengen
gemäß Korollar \ref{disjoint-preimg} disjunkte Urbilder haben.
Weil $\gamma$ gemäß Definition stetig ist,
sind die Urbilder $\gamma^{-1}(\operatorname{int}(M))$ und
$\gamma^{-1}(\operatorname{ext}(M))$ offen im Raum $[0,1]$. Sie sind
nichtleer, weil sie jeweils laut Prämisse mindestens einen Punkt
enthalten. Damit ist $[0,1]$ eine Zerlegung in disjunkte nichtleere
offene Mengen, gemäß Definition also ein unzusammhängender Raum. Das
steht im Widerspruch zur Erkenntnis, dass alle Intervalle
zusammenhängend sind.\,\qedsymbol
\end{Beweis}

\begin{Beweis}[Beweis 2]
Sei $\gamma\colon [0,1]\to\R^n$ ein solcher Weg mit $\gamma(0)=p$ und
$\gamma(1)=q$. Wir nehmen nun eine Bisektion vor. Sei $a_0:=0$ und
$b_0:=1$. Sei $m:=\tfrac{1}{2}(a_{k}+b_{k})$, also der Mittelwert. Liegt
$\gamma(m)$ im Inneren, dann ist $[a_{k+1},b_{k+1}]=[m,b_k]$ das nächste
Intervall. Liegt $\gamma(m)$ im Äußeren, dann $[a_{k+1},b_{k+1}]=[a_k,m]$.
Liegt $\gamma(m)$ auf dem Rand, ist ein Schnittpunkt gefunden und
das Verfahren bricht ab. Betrachten wir daher
den Fall, dass das Verfahren nicht abbricht. Als
Intervallschachtelung konvergieren die Folgen $a_k,b_k$ gegen
denselben Grenzwert $a$. Weil $\gamma$ stetig ist, konvergiert
$\gamma(a_k)\to \gamma(a)$ für $a_k\to a$ und $\gamma(b_k)\to\gamma(a)$
für $b_k\to a$. Demnach sind in jeder Umgebung von $\gamma(a)$ sowohl
Punkte aus dem Inneren als auch Punkte aus dem Äußeren. Gemäß
Satz \ref{boundary-point-char} muss $\gamma(a)$ infolge
auf dem Rand liegen.\,\qedsymbol
\end{Beweis}


\chapter{Lineare Algebra}
\section{Grundbegriffe}
\subsection{Norm}\index{Norm}
\begin{definition}[Norm]\mbox{}\newline
Eine Abbildung $v\mapsto\|v\|$ von einem
Vektorraum $V$ über dem Körper $K$ in die nichtnegativen reellen
Zahlen heißt \emdef{Norm}, wenn für alle $v,w\in V$ und $a\in K$
die drei Axiome%
\begin{gather}
\|v\|=0 \implies v=0,\\
\|av\| = |a|\,\|v\|,\\
\|v+w\| \le \|v\|+\|w\|
\end{gather}
erfüllt sind.
\end{definition}

\noindent
Eigenschaften:
\begin{gather}
\|v\|=0\iff v=0,\\
\|-v\|=\|v\|,\\
\|v\|\ge 0.
\end{gather}
Dreiecksungleichung nach unten:
\begin{equation}
|\|v\|-\|w\||\le \|v-w\|.
\end{equation}

\subsection{Skalarprodukt}\index{Skalarprodukt}

Ein Vektorraum über dem Körper $\R$ heißt \emdef{reeller Vektorraum},
einer über dem Körper $\C$ heißt \emdef{komplexer Vektorraum}.

\subsubsection{Definition}
Sei $V$ ein reeller Vektorraum. Eine Abbildung $f\colon V^2\to\R$
mit $f(x,y)=\langle x,y\rangle$ heißt \emdef{Skalarprodukt}, wenn
folgende Axiome erfüllt sind. Für $v,w\in V$ und $\lambda\in\R$ gilt:
\begin{gather}
\langle v,w\rangle = \langle w,v\rangle,\\
\langle v,\lambda w\rangle = \lambda\langle v,w\rangle,\\
\langle v,w_1+w_2\rangle = \langle v,w_1\rangle +\langle v,w_2\rangle,\\
\langle v,v\rangle\ge 0,\\
\langle v,v\rangle=0 \iff v=0.
\end{gather}
Sei $V$ ein komplexer Vektorraum und $f\colon V^2\to\C$.\\
Für $v,w\in V$ und $\lambda\in\R$ gilt:
\begin{gather}
\langle v,w\rangle = \overline{\langle w,v\rangle},\\
\langle \lambda v,w\rangle = \overline{\lambda}\langle v,w\rangle,\\
\langle v,\lambda w\rangle = \lambda\langle v,w\rangle,\\
\langle v,w_1+w_2\rangle = \langle v,w_1\rangle +\langle v,w_2\rangle,\\
\langle v,v\rangle\ge 0,\\
\langle v,v\rangle=0 \iff v=0.
\end{gather}

\subsubsection{Eigenschaften}
Das reelle Skalarprodukt ist eine symmetrische bilineare Abbildung.

\subsubsection{Winkel und Längen}
\begin{definition}[Winkel, orgthogonale Vektoren]\mbox{}\newline
Der \emdef{Winkel} $\varphi$ zwischen $v$ und $w$
ist definiert durch die Beziehung:
\begin{equation}
\langle v,w\rangle = \|v\|\,\|w\|\,\cos\varphi.
\end{equation}
\emdef{Orthogonal}:\index{Orthogonal}
\begin{equation}
v\perp w \;:\Longleftrightarrow\; \langle v,w\rangle=0.
\end{equation}
\end{definition}

Ein Skalarprodukt $\langle v,w\rangle$ induziert die Norm
\begin{equation}
\|v\| := \sqrt{\langle v,v\rangle}.
\end{equation}

\subsubsection{Orthonormalbasis}\label{sec:ONB}
\index{Orthogonalsystem}\index{Orthogonalbasis}
\index{Orthonormalsystem}\index{Orthonormalbasis}
Sei $B=(b_k)_{k=1}^n$ eine Basis eines endlichdimensionalen
Vektorraumes über den reellen oder komplexen Zahlen.

\begin{definition}[Orthogonalbasis]\mbox{}\newline
Gilt $\langle b_i,b_j\rangle=0$
für alle $i,j$ mit $i\ne j$, so wird $B$ \emdef{Orthogonalbasis}
genannt. Ist $B$ nicht unbedingt eine Basis, so spricht man von einem 
\emdef{Orthogonalsystem}.
\end{definition}

\begin{definition}[Orthonormalbasis]\mbox{}\newline
Ist $B$ eine Orthogonalbasis und gilt
zusätzlich $\langle b_k,b_k\rangle=1$ für alle $k$, so wird
$B$ \emdef{Orthonormalbasis} (ONB) genannt. Ist $B$ nicht unbedingt
eine Basis,  so spricht man von einem \emdef{Orthonormalsystem}.
\end{definition}

Sei $v=\sum_k v_kb_k$ und $w=\sum_k w_kb_k$.
Mit $\sum_k$ ist immer $\sum_{k=1}^n$ gemeint.

Ist $B$ eine Orthonormalbasis, so gilt:
\begin{equation}
\langle v,w\rangle = \sum_k \overline{v_k}\,w_k.
\end{equation}
Ist $B$ nur eine Orthogonalbasis, so gilt:
\begin{equation}
\langle v,w\rangle = \sum_k \langle b_k,b_k\rangle \overline{v_k}\,w_k
\end{equation}
Allgemein gilt:
\begin{equation}
\langle v,w\rangle = \sum_{i,j} g_{ij} \overline{v_i}\,w_j
\end{equation}
mit $g_{ij}=\langle b_i,b_j\rangle$. In reellen Vektorräumen
ist die komplexe Konjugation wirkungslos und kann somit entfallen.

Ist $B$ eine Orthogonalbasis und $v=\sum_k v_k b_k$, so gilt:
\begin{equation}
v_k = \frac{\langle b_k,v\rangle}{\langle b_k,b_k\rangle}.
\end{equation}
Ist $B$ eine Orthonormalbasis, so gilt speziell:
\begin{equation}
v_k = \langle b_k,v\rangle.
\end{equation}


\subsubsection{Orthogonale Projektion}
Orthogonale Projektion von $v$ auf $w$:
\begin{equation}
P[w](v) := \frac{\langle v,w\rangle}{\langle w,w\rangle}\,w.
\end{equation}
\subsubsection{Gram-Schmidt-Verfahren}
Für linear unabhängige Vektoren $v_1,\ldots,v_n$
wird durch%
\begin{equation}
w_k := v_k - \sum_{i=1}^{k-1} P[w_i](v_k)
\end{equation}
ein Orthogonalsystem $w_1,\ldots,w_n$ berechnet.

Speziell für zwei nicht kollineare Vektoren $v_1,v_2$ gilt
\begin{gather}
w_1=v_1,\\
w_2=v_2-P[w_1](v_2).
\end{gather}

\newpage
\subsubsection{Musikalische Isomorphismen}
\begin{definition}[Musikalische Isomorphismen]%
\index{musikalische Isomorphismen}\mbox{}\newline
Sei $V$ ein eindlichdimensionaler Vektorraum
mit Skalarprodukt und $V^*$ sein Dualraum.
Die lineare Abbildung%
\begin{equation}
\Phi\colon V\to V^*,\quad \Phi(u)(v):=\langle u,v\rangle
\end{equation}
ist ein kanonischer Isomorphismus.%
\index{kanonischer Isomorphismus!musikalische Isomorphismen}
Man nennt $u^\flat:=\Phi(u)$
und $\omega^\sharp:=\Phi^{-1}(\omega)$ die \emdef{musikalischen
Isomorphismen}.
\end{definition}

\section{Koordinatenvektoren}
\subsection{Koordinatenraum}
Addition von $a,b\in K^n$:
\begin{equation}\label{eq:Koordinatenraum-Addition}
\begin{bmatrix}
a_1\\
\vdots\\
a_n
\end{bmatrix}
+\begin{bmatrix}
b_1\\
\vdots\\
b_n
\end{bmatrix}
:= \begin{bmatrix}
a_1+b_1\\
\vdots\\
a_n+b_n
\end{bmatrix}.
\end{equation}
Subtraktion:
\begin{equation}
\begin{bmatrix}
a_1\\
\vdots\\
a_n
\end{bmatrix}
-\begin{bmatrix}
b_1\\
\vdots\\
b_n
\end{bmatrix}
:= \begin{bmatrix}
a_1-b_1\\
\vdots\\
a_n-b_n
\end{bmatrix}.
\end{equation}
Skalarmultiplikation von $\lambda\in K$ mit $a\in K^n$:
\begin{align}\label{eq:Koordinatenraum-Skalarmultiplikation}
\lambda\begin{bmatrix}
a_1\\
\vdots\\
a_n
\end{bmatrix}
:= \begin{bmatrix}
\lambda a_1\\
\vdots\\
\lambda a_n
\end{bmatrix}.
\end{align}
Ist $K$ ein Körper, so bildet die Menge
\begin{equation}
K^n = \{(a_1,\ldots,a_n)\mid \forall k\colon a_k\in K\}
\end{equation}
bezüglich der Addition \eqref{eq:Koordinatenraum-Addition}
und der Multiplikation \eqref{eq:Koordinatenraum-Skalarmultiplikation}
einen Vektorraum, der \emdef{Koordinatenraum} genannt wird.
Das Tupel $E_n=(e_1,\ldots,e_n)$ mit
\begin{equation}\label{eq:kanonische-Basis}
\begin{split}
e_1 &:= (1,0,0,0,\ldots, 0),\\
e_2 &:= (0,1,0,0,\ldots, 0),\\
e_3 &:= (0,0,1,0,\ldots, 0),\\
\ldots\\
e_n &:= (0,0,0,0,\ldots, 1)
\end{split}
\end{equation}
bildet eine geordnete Basis von $K^n$, die \emdef{kanonische Basis}
genannt wird. Es gilt
\begin{equation}
a = (a_1,\ldots,a_n) = a_1 e_1+\ldots+a_n e_n.
\end{equation}

\newpage
\subsection{Kanonisches Skalarprodukt}
\begin{definition}[Kanonisches Skalarprodukt]\mbox{}\newline
Für $a,b\in\R^n$:
\begin{equation}
\langle a,b\rangle := \sum_{k=1}^n a_k b_k.
\end{equation}
Für $a,b\in\C^n$:
\begin{equation}
\langle a,b\rangle := \sum_{k=1}^n \overline{a_k}\,b_k.
\end{equation}
\end{definition}
\noindent
Die kanonische Basis \eqref{eq:kanonische-Basis} ist eine
Orthonormalbasis bezüglich diesem Skalarprodukt, s. \ref{sec:ONB}.
Das Skalarprodukt induziert die Norm
\begin{equation}
|a| := \sqrt{\langle a,a\rangle} = \sqrt{\textstyle \sum_{k=1}^n |a_k|^2},
\end{equation}
die \emdef{Vektorbetrag}\index{Vektorbetrag} genannt wird.

Jedem Koordinatenvektor $a\ne 0$ lässt sich ein Einheitsvektor\index{Einheitsvektor}
$\hat a:=\frac{a}{|a|}$ zuordnen, der in Richtung von $a$ zeigt
und die Eigenschaft $|\hat a|=1$ besitzt.

Es gilt
\begin{align}
a\perp b &\iff \langle a,b\rangle=0,\\
a\uparrow\uparrow b &\iff \langle a,b\rangle = |a|\,|b|,\\
a\uparrow\downarrow b &\iff \langle a,b\rangle = -|a|\,|b|.
\end{align}
Allgemein gilt
\begin{equation}
\langle a,b\rangle = |a|\,|b|\cos\varphi.\qquad(\varphi=\angle (a,b))
\end{equation}

\subsection{Vektorprodukt}
Für $a,b\in\R^3$:\index{Vektorprodukt}
\begin{equation}
a\times b = \begin{bmatrix}
a_x\\ a_y\\ a_z
\end{bmatrix}\times\begin{bmatrix}
b_x\\ b_y\\ b_z
\end{bmatrix}
= \begin{vmatrix}
e_x & a_x & b_x\\
e_y & a_y & b_y\\
e_z & a_z & b_z
\end{vmatrix}
= \begin{bmatrix}
a_y b_z - a_z b_y\\
a_z b_x - a_x b_z\\
a_x b_y - a_y b_x
\end{bmatrix}.
\end{equation}
Rechenregeln für $a,b,c\in\R^3$ und $r\in\R$:
\begin{gather}
a\times (b+c) = a\times b+a\times c,\\
(a+b)\times c = a\times c+b\times c,\\
(ra)\times b = r(a\times b) = a\times(rb),\\
a\times b = -b\times a,\\
a\times a = 0.
\end{gather}
Für den Betrag gilt:
\begin{equation}
|a\times b| = |a|\,|b|\,\sin\varphi.\qquad(\varphi=\angle (a,b))
\end{equation}
Beziehung zur Determinante:
\begin{equation}
\langle a,b\times c\rangle = \det(a,b,c).
\end{equation}
Jacobi-Identität:%
\index{Identität!Jacobi-Identität}\index{Jacobi-Identität}
\begin{equation}
a\times(b\times c) = b\times (a\times c) - c\times (a\times b). 
\end{equation}
Graßmann-Identität:%
\index{Identität!Graßmann-Identität}\index{Graßmann-Identität}
\begin{equation}
a\times(b\times c) = b\langle a,c\rangle - c\langle a,b\rangle.
\end{equation}
Cauchy-Binet-Identität:
\index{Identität!Cauchy-Binet-Identität}\index{Cauchy-Binet-Identität}
\begin{equation}
\langle a\times b, c\times d\rangle
= \langle a,c\rangle\langle b,d\rangle
- \langle b,c\rangle\langle a,d\rangle.
\end{equation}
Lagrange-Identität:%
\index{Identität!Lagrange-Identität}\index{Lagrange-Identität}
\begin{equation}
|a\times b|^2 = |a|^2 |b|^2 - \langle a,b\rangle^2.
\end{equation}

\newpage
\section{Matrizen}\index{Matrix}
\subsection{Quadratische Matrizen}%
\index{Matrix!quadratische}\index{quadratische Matrix}
\subsubsection{Matrizenring}%
\index{Ring!Matrizenring}\index{Matrizenring}
Mit $K^{n\times n}$ wird die Menge quadratischen Matrizen
\begin{equation}
(a_{ij}) = \begin{bmatrix}
a_{11} & \ldots & a_{1n}\\
\ldots & \ddots & \ldots\\
a_{n1} & \ldots & a_{nn}
\end{bmatrix}
\end{equation}
mit Einträgen $a_{ij}$ aus dem Körper $K$ bezeichnet.

Die Menge $K^{n\times n}$ bildet bezüglich Addition
und Multiplikation von Matrizen einen Ring (s. \ref{sec:Strukturen}).

Das neutrale Element der Multiplikation
ist die Einheitsmatrix
\begin{equation}
E_n = (\delta_{ij}),\quad
\delta_{ij}:=\begin{cases}
1 & \text{wenn}\;i=j,\\
0 & \text{sonst}.
\end{cases}
\end{equation}
Das sind
\begin{equation}
E_2 = \begin{bmatrix}
1 & 0\\
0 & 1
\end{bmatrix},\quad
E_3 = \begin{bmatrix}
1 & 0 & 0\\
0 & 1 & 0\\
0 & 0 & 1
\end{bmatrix},
\quad\text{usw.}
\end{equation}

\subsubsection{Symmetrische Matrizen}
Eine quadratiche Matrix $A=(a_{ij})$ heißt
symmetrisch\index{symmetrische Matrix}\index{Matrix!symmetrische},
falls gilt $a_{ij}=a_{ji}$ bzw. $A^T=A$.

Jede reelle symmetrische Matrix besitzt ausschließlich reelle
Eigenwerte und die algebraischen Vielfachheiten stimmen mit den
geometrischen Vielfachheiten überein.

Jede reelle symmetrische Matrix $A$ ist diagonalisierbar, d.\,h. es gibt
eine invertierbare Matrix $T$ und eine Diagonalmatrix $D$, so dass
$A=TDT^{-1}$ gilt.

Sei $V$ ein $K$-Vektorraum und $(b_k)_{k=1}^n$ eine Basis von $V$.
Für jede symmetrische Bilinearform\index{symmetrische Bilinearform}
$f\colon V^2\to K$ ist die
Darstellungsmatrix
\begin{equation}
A = (f(b_i,b_j))
\end{equation}
symmetrisch. Ist $A\in K^{n\times n}$ eine symmetrische Matrix, so
ist
\begin{equation}\label{eq:symmBf}
f(x,y) = x^T A y.
\end{equation}
eine symmetrische Bilinearform für  $x,y\in K^n$.
Ist $K=\R$ und $A$ positiv definit, so ist
\eqref{eq:symmBf} ein Skalarprodukt auf $\R^n$.

\subsubsection{Reguläre Matrizen}\index{inverse Matrix}
\begin{definition}[Reguläre Matrix, singuläre Matrix]\mbox{}\newline
Eine quadratische Matrix $A\in K^{n\times n}$ heißt
\emdef{regulär}\index{reguläre Matrix}\index{Matrix!reguläre}
oder \emdef{invertierbar}, wenn es eine inverse Matrix $A^{-1}$ gibt,
so dass
\begin{equation}
A^{-1}A = E_n \quad (\iff AA^{-1} = E_n)
\end{equation}
gilt, wobei mit $E_n$ die Einheitsmatrix gemeint ist.
Eine quadratische Matrix die nicht regulär ist, heißt
\emdef{singulär}\index{singuläre Matrix}\index{Matrix!singuläre}.
\end{definition}

\strong{Kriterien.} Für eine Matrix $A\in K^{n\times n}$ gilt:
\begin{gather}
A\text{ ist regulär}\iff \exists B (BA=E_n)\\
\iff \det(A)\ne 0 \iff \operatorname{rk}(A)=n\\
\iff 0\text{ ist kein Eigenwert von }A\\
\iff \ker(A)=\{0\}.
\end{gather}

\noindent
\strong{Eigenschaften.}
Jede reguläre Matrix besitzt genau eine inverse Matrix. 
Die Menge der regulären Matrizen bildet bezüglich Matrizenmultiplikation
eine Gruppe, die
\emdef{allgemeine lineare Gruppe}\index{allgemeine lineare Gruppe}
\begin{equation}
\operatorname{GL}(n,K) := \{A\in K^{n\times n}\mid\det(A)\ne 0\}.
\end{equation}
Ist $V$ ein Vektorraum über dem Körper $K$, so bilden die
Automorphismen bezüglich Verkettung eine Gruppe, die
\emph{Automorphismengruppe}
\begin{equation}
\operatorname{GL}(V) = \operatorname{Aut}(V).
\end{equation}
Ein \emdef{Endomorphismus}\index{Endomorphismus!auf einem Vektorraum}
ist eine lineare Abbildung, welche eine Selbstabbildung ist.
Ein \emdef{Automorphismus}\index{Automorphismus!auf einem Vektorraum}
ist eine bijektiver Endomorphismus.

Wählt man auf $V$ eine Basis
$B$, so ist die Zuordnung der Darstellungsmatrix
\begin{equation}
M_B^B\colon \operatorname{Aut}(V)\to\operatorname{GL}(\dim V,K)
\end{equation}
eine Gruppenisomorphismus.

Endomorphismen die nicht bijektiv sind, bzw. singuläre Matrizen,
lassen die Dimension ihrer Definitionsmenge schrumpfen:
\begin{equation}
f{\in}\operatorname{End}(V){\setminus}\operatorname{Aut}(V)
\Longleftrightarrow \dim f(V)<\dim V.
\end{equation}
Für Matrizen $A\in K^{n\times n}$ bedeutet das, dass sie nicht
den vollen Rang besitzen:
\begin{equation}
\det A=0\iff \operatorname{rk}(A) < n = \dim K^n.
\end{equation}
\strong{Bestimmung der inversen Matrix.}

Für eine Matrix $A\in K^{2\times 2}$ gilt:
\begin{equation}
A^{-1} = \begin{bmatrix}
a & b\\
c & d
\end{bmatrix}^{-1}
= \frac{1}{ad-bc}\begin{bmatrix}
d & -b\\
-c & a
\end{bmatrix},
\end{equation}
wenn $\det A\ne 0$ mit $\det A = ad-bc$.

\begin{definition}[Streichungsmatrix]\mbox{}\newline
Wird in der Matrix $A$ die Zeile $i$ und die Spalte $j$ entfernt,
so entsteht eine neue Matrix $[A]_{ij}$, die
\emdef{Streichungsmatrix}\index{Streichungsmatrix}
von $A$ genannt wird.
\end{definition}
Laplacescher Entwicklungssatz:
\begin{align}
\det A = \sum_{i=1}^n (-1)^{i+j}a_{ij}\det([A]_{ij}),\\
\det A = \sum_{j=1}^n (-1)^{i+j}a_{ij}\det([A]_{ij}).
\end{align}

\subsubsection{Determinanten}\index{Determinante}
Für Matrizen $A,B\in K^{n\times n}$ und $r\in K$ gilt:
\begin{gather}
\det(AB) = \det(A)\det(B),\\
\det(A^T) = \det(A),\\
\det(rA) = r^n\det(A),\\
\det(A^{-1}) = \det(A)^{-1}.
\end{gather}
Für eine Diagonalmatrix $D=\diag(d_1,\ldots,d_n)$ gilt:
\begin{gather}
\det(D) = \prod_{k=1}^n d_k.
\end{gather}
Eine linke Dreiecksmatrix ist eine Matrix der Form
$(a_{ij})$ mit $a_{ij}=0$ für $i<j$. Eine rechte
Dreiecksmatrix ist die Transponierte einer linken
Dreiecksmatrix.

Für eine linke oder rechte Dreiecksmatrix $A=(a_{ij})$ gilt:
\begin{gather}
\det(A) = \prod_{k=1}^n a_{kk}.
\end{gather}

\newpage
\subsubsection{Eigenwerte}
\strong{Eigenwertproblem:}\index{Eigenwert}
Für eine gegebene quadratische Matrix $A$ bestimme
\begin{equation}
\{(\lambda,v)\mid Av = \lambda v,\,v\ne 0\}.
\end{equation}
Das homogene lineare Gleichungssystem
\begin{equation}
Av=\lambda v \iff (A-\lambda E_n)v=0
\end{equation}
besitzt Lösungen $v\ne 0$ gdw.
\begin{equation}
p(\lambda):=\det(A-\lambda E_n)=0.
\end{equation}
Bei $p(\lambda)$ handelt es sich um ein normiertes Polynom
vom Grad $n$, das \emdef{charakeristisches Polynom}
\index{charakteristisches Polynom}
genannt wird.

\strong{Eigenraum:}\index{Eigenraum}
\begin{equation}
\Eig(A,\lambda):=\{v\mid Av=\lambda v\}.
\end{equation}
Die Dimension $\dim\Eig(A,\lambda)$ wird
\emdef{geometrische Vielfachheit}\index{geometrische Vielfachheit}
von $\lambda$ genannt.

\strong{Spektrum:}\index{Spektrum}
\begin{equation}
\sigma(A) := \{\lambda\mid \exists v\ne 0\colon Av=\lambda v\}.
\end{equation}

\subsubsection{Nilpotente Matrizen}
\begin{definition}[Nilpotente Matrix]\mbox{}\newline
Eine quadratische Matrix $A\in K^{n\times n}$ heißt \emdef{nilpotent},
wenn es eine Zahl $k\in\N, k\ge 1$ gibt, so dass gilt:
\begin{equation}
A^k=0.
\end{equation}
Die erste solche Zahl heißt \emdef{Nilpotenzgrad} der Matrix $A$.

Eine äquivalente Bedingung ist:
\begin{equation}
p_A(\lambda):=\det(\lambda E-A)=\lambda^n.
\end{equation}
\end{definition}

\noindent
\strong{Eigenschaften.}
Sei $A$ eine nilpotente Matrix. Es gilt:
\begin{itemize}[itemsep=0pt, leftmargin=3em]
\bitem $A$ besitzt nur den Eigenwert $\lambda=0$.
\bitem $\det(A)=\tr(A)=0$.
\bitem $E-A$ ist invertierbar.
\end{itemize}

\subsection{Matrixfunktionen}
\subsubsection{Matrixexponential}
\begin{definition}[Matrixexponential]\mbox{}\\*
Für eine beliebige Matrix  $X\in \C^{n\times n}$ konvergiert%
\begin{equation}
\exp(X) := \sum_{k=0}^\infty \frac{X^k}{k!}.
\end{equation}
\end{definition}
Für jede Matrix $X$ und $a,b\in\C$ gilt
\begin{gather}
\exp(-X) = \exp(X)^{-1},\\
\exp(X^H) = \exp(X)^H,\\
\exp((a+b)X) = \exp(aX)\exp(bX),\\
\exp(\diag(d_1,\ldots,d_n)) = \diag(\ee^{d_1},\ldots,\ee^{d_n}),\\
\det(\exp(X)) = \ee^{\tr(X)}.
\end{gather}
Für $XY=XY$ gilt
\begin{equation}
\exp(X+Y)=\exp(X)\exp(Y).
\end{equation}
Das Exponential einer Matrix ist immer invertierbar und
jede Matrix aus $\operatorname{GL}(n,\C)$ kann als Matrixexponential
dargestellt werden. D.\,h.
$\exp\colon \C^{n\times n}\to\operatorname{GL}(n,\C)$
ist surjektiv.

\newpage
\subsubsection{Allgemein}

Matrizen bilden bezüglich Matrizenmultiplikation zusammen
mit der Frobeniusnorm oder einer Operatornorm eine assoziative
Banachalgebra mit Einselement.
Man betrachte nun die formale Potenzreihe%
\begin{equation}
f(X) := \sum_{k=0}^\infty a_k X^k,\quad a_k\in \C.
\end{equation}
Besitzt die Einsetzung $f(z)$ für $z\in\C$ den Konvergenzradius
$r>0$ und ist $A$ ein Element einer Banachalgebra
mit Einselement mit $\|A\|<r$,
dann ist $f(A)$ absolut konvergent. Gemäß%
\begin{equation}
f\colon \{\|A\|<r\mid A\in\C^{n\times n}\}\to\C^{n\times n}
\end{equation}
ist daher eine Matrixfunktion definiert.

Ist $A$ diagonalisierbar mit $A=TDT^{-1}$, dann gilt
\begin{equation}
f(A) = Tf(D)T^{-1},
\end{equation}
wobei $f(D)$ gemäß
\begin{equation}
f(\diag(d_1,\ldots,d_n)) = \diag(f(d_1),\ldots,f(d_n))
\end{equation}
berechnet wird.

\strong{Sylvesters Formel.} Allgemein gilt%
\begin{equation}\label{eq:Sylvesters-Formel}
f(A) = \sum_{i=1}^s A_i \sum_{k=0}^{m_i-1} \frac{f^{(k)}(\lambda_i)}{k!}(A-\lambda_i E)^k
\end{equation}
mit
\begin{equation}
A_i := \prod_{j=1,\,j\ne i}^s \frac{1}{\lambda_i-\lambda_j}(A-\lambda_j E).
\end{equation}
Hierbei ist $s$ die Anzahl der unterschiedlichen Eigenwerte und $m_i$
die algebraische Vielfachheit von $\lambda_i$.

Bei einer diagonalisierbaren Matrix vereinfacht sich die Formel zu
\begin{equation}
f(A) = \sum_{i=1}^n A_i f(\lambda_i).
\end{equation}
Speziell für $2\times 2$-Matrizen gilt
\begin{equation}
f(A) = pA+qE
\end{equation}
mit
\begin{align}
p &= \frac{f(\lambda_2)-f(\lambda_1)}{\lambda_2-\lambda_1},\\
q &= f(\lambda_1)-p\lambda_1 = f(\lambda_2)-p\lambda_2.
\end{align}
Im Fall $\lambda_1=\lambda_2$ ist $p=f'(\lambda_1)$.

\strong{Als Cauchy-Integral.} Sei $f\colon U\to\C$ holomorph
und $G\subseteq U$ abgeschlossen und einfach zusammenhängend.
Liegen alle Eigenwerte von $A$ im Inneren von $G$, dann gilt%
\begin{equation}\label{eq:Matrixfunktion-Cauchy}
f(A) = \frac{1}{2\pi\ui}\int_{\partial G} f(z)(zE-A)^{-1}\,\mathrm dz.
\end{equation}
Die Formeln \eqref{eq:Sylvesters-Formel} und
\eqref{eq:Matrixfunktion-Cauchy} liefern außerdem
zwei miteinander verträgliche Verallgemeinerungen der Definition
der Matrixfunktion.

\newpage
\section{Lineare Gleichungssysteme}
\index{lineares Gleichungssytem}
Ein lineares Gleichungssystem mit $m$ Gleichungen und $n$ Unbekannten
hat die Form:
\begin{equation}\label{eq:LGS}
\begin{split}
a_{11} x_1 + a_{12} x_2 + \ldots + a_{1n} x_n &= b_1,\\
a_{21} x_1 + a_{22} x_2 + \ldots + a_{2n} x_n &= b_2,\\
&\;\;\vdots\\
a_{m1} x_1 + a_{m2} x_2 + \ldots + a_{mn} x_n &= b_n.
\end{split}
\end{equation}
Das System lässt sich durch
\begin{equation}
A:=\begin{bmatrix}
a_{11} & a_{12} & \ldots & a_{1n}\\
a_{21} & a_{22} & \ldots & a_{2n}\\
\vdots & \vdots & \ddots & \vdots\\
a_{m1} & a_{m1} & \ldots & a_{mn}
\end{bmatrix}
\end{equation}
und
\begin{equation}
x:=\begin{bmatrix}
x_1 \\ x_2 \\ \vdots \\ x_n
\end{bmatrix},\quad
b:=\begin{bmatrix}
b_1 \\ b_2 \\ \vdots \\ b_n
\end{bmatrix}
\end{equation}
zusammenfassen.

Äquivalente Matrixform von \eqref{eq:LGS}:
\begin{equation}
Ax=b.
\end{equation}
Erweiterte Koeffizientenmatrix:\index{erweiterte Koeffizientenmatrix}
\begin{equation}
(A\,|\,b) := \left[\begin{array}{ccc|c}
a_{11} & \ldots & a_{1n} & b_1\\
\vdots & \ddots & \vdots & \vdots\\
a_{m1} & \ldots & a_{mn} & b_n
\end{array}\right].
\end{equation}
Lösungskriterium:
\begin{equation}
\exists x[Ax=b] \iff \rg(A)=\rg(A\,|\,b).
\end{equation}
Eindeutige Lösung (bei $n$ Unbekannten):
\begin{equation}
\exists! x[Ax=b] \iff\rg(A)=\rg(A\,|\,b)=n.
\end{equation}
Im Fall $m=n$ gilt:
\begin{equation}
\begin{split}
&\exists! x[Ax=b] \iff A\in\operatorname{GL}(n,K)\\
&\iff \rg(A)=n \iff \det(A)\ne 0.
\end{split}
\end{equation}

% \newpage
\section{Multilineare Algebra}
\subsection{Äußeres Produkt}
Sei $V$ ein Vektorraum und sei $v_k\in V$ für alle $k$.

Sind $a=\sum_{k=1}^n a_k v_k$
und $b=\sum_{k=1}^n b_k v_k$ beliebige
Linearkombinationen, so gilt
\begin{equation}
\begin{split}
a\wedge b &= \sum_{i,j} a_i b_j\,v_i\wedge v_j\\
&= \sum_{1\le i<j\le n} (a_i b_j-a_j b_i)\,v_i\wedge v_j
\end{split}
\end{equation}
und
\begin{equation}
\begin{split}
a\wedge b &= a\otimes b-b\otimes a\\
&= \sum_{i,j} (a_i b_j-a_j b_i)\,v_i\otimes v_j\\
&= \sum_{i,j} a_i b_j (v_i\otimes v_j-v_j\otimes v_i).
\end{split}
\end{equation}
\subsubsection{Alternator}\index{Alternator}
Für $a_k\in V$ ist
$\operatorname{Alt}_p\colon T^p(V)\to T^p(V)$
mit
\begin{equation}
\begin{split}
& \operatorname{Alt}_p (a_1\otimes\ldots\otimes a_p)\\
&:= \frac{1}{p!}\sum_{\sigma\in S_{\scriptstyle p}}
\sgn(\sigma)\,(a_{\sigma(1)}\otimes\ldots\otimes a_{\sigma(p)}).
\end{split}
\end{equation}
Mit $A^p(V)$ wird die Bildmenge des Alternators bezeichnet.
Der Raum $\Lambda^p(V)$ wird kanonisch mit $A^p(V)$ identifiziert, indem
\begin{equation}
a_1\wedge\ldots\wedge a_p
= p!\operatorname{Alt}_p(a_1\otimes\ldots\otimes a_p)
\end{equation}
gesetzt wird. Hierdurch wird ein kanonischer Isomorphismus%
\index{kanonischer Isomorphismus!Alternator} zwischen
den Algebren $\Lambda(V)$ und $A(V)$ induziert.

Speziell gilt
\begin{equation}
\operatorname{Alt}_2 (a\otimes b) := \frac{1}{2}(a\otimes b-b\otimes a).
\end{equation}
und
\begin{equation}
a\wedge b = 2\operatorname{Alt}_2(a\otimes b).
\end{equation}

\subsubsection{Äußere Algebra}\index{aussere Algebra@äußere Algebra}
Darstellung als Quotientenraum:
\begin{equation}
\Lambda^2(V) = T^2(V)/\{v\otimes v\mid v\in V\}.
\end{equation}
Dimension: Ist $\dim(V)=n$, so gilt
\begin{equation}
\dim(\Lambda^k(V)) = \binom{n}{k}.
\end{equation}

\clearpage
\section{Analytische Geometrie}
\subsection{Geraden}\index{Gerade}
\subsubsection{Parameterdarstellung}
\index{Parameterdarstellung!einer Geraden}

\strong{Punktrichtungsform:}\index{Punktrichtungsform}
\begin{equation}
p(t) = p_0+t\underline v,
\end{equation}
$p_0$: Stützpunkt, $\underline v$: Richtungsvektor.
Die Gerade ist dann die Menge $g=\{p(t)\mid t\in\R\}$.

Der Vektor $\underline v$ repräsentiert außerdem die Geschwindigkeit,
mit der diese Parameterdarstellung durchlaufen wird:
$p'(t)=\underline v$.

\strong{Gerade durch zwei Punkte:}
Sind zwei Punkte $p_1,p_2$ mit $p_1\ne p_2$ gegeben, so ist
durch die beiden Punkte eine Gerade gegeben. Für diese Gerade ist
\begin{equation}
p(t) = p_1+t(p_2-p_1)
\end{equation}
eine Punktrichtungsform\index{Punktrichtungsform}.
Durch Umformung ergibt sich die \strong{Zweipunkteform:}
\begin{equation}\label{eq:Zweipunkteform}
p(t) = (1-t)p_1+tp_2.
\end{equation}
Bei \eqref{eq:Zweipunkteform} handelt es sich um eine
Affinkombination. Gilt $t\in[0,1]$, so ist \eqref{eq:Zweipunkteform}
eine Konvexkombination: eine Parameterdarstellung für die Strecke
von $p_1$ nach $p_2$.

\subsubsection{Parameterfreie Darstellung}
\strong{Hesse-Form:}
\begin{equation}\label{eq:Hesse-Form}
g = \{p\mid\langle \uv n,p-p_0\rangle = 0\},
\end{equation}
$p_0$: Stützpunkt, $\uv n$: Normalenvektor.

Die Hesse-Form ist nur in der Ebene möglich.
Form \eqref{eq:Hesse-Form} hat in Koordinaten
die Form
\begin{equation}
\begin{split}
g &= \{(x,y)\mid n_x(x-x_0)+n_y(y-y_0)=0\}\\
&= \{(x,y)\mid n_x x+n_y y = n_x x_0+n_y y_0\}.
\end{split}
\end{equation}

\strong{Hesse-Normalform:} \eqref{eq:Hesse-Form} mit $|\uv n|=1$.


Sei $\uv v\wedge\uv w$ das äußere Produkt.

\strong{Plückerform:}
\begin{equation}
g = \{p\mid (p-p_0)\wedge \underline v=0\}.
\end{equation}
Die Größe $\underline m = p_0\wedge\underline v$ heißt Moment.
Beim Tupel $(\underline v:\underline m)$ handelt es sich um
Plückerkoordinaten für die Gerade.

In der Ebene gilt speziell:
\begin{equation}\label{eq:Gerade-Ebene}
g = \{(x,y)\mid (x-x_0)\Delta y = (y-y_0)\Delta x\}
\end{equation}
mit $\underline v=(\Delta x,\Delta y)$.

Sei $a:=\Delta y$ und $b:=-\Delta x$ und $c:=ax_0+by_0$.
Aus \eqref{eq:Gerade-Ebene} ergibt sich:
\begin{equation}
g = \{(x,y)\mid ax+by=c\}.
\end{equation}
Im Raum ergibt sich ein Gleichungssystem:
\begin{equation}
g = \{\begin{pmatrix}x\\ y\\ z\end{pmatrix}
\mid
\begin{vmatrix}
(x-x_0)\Delta y = (y-y_0)\Delta x\\
(y-y_0)\Delta z = (z-z_0)\Delta y\\
(x-x_0)\Delta z = (z-z_0)\Delta x
\end{vmatrix}\}
\end{equation}
mit $\underline v=(\Delta x,\Delta y,\Delta z)$.

\subsubsection{Abstand Punkt zu Gerade}
Sei $p(t):=p_0+t\underline v$ die Punktrichtungsform einer Geraden und
sei $q$ ein weiterer Punkt. Bei $\underline d(t):=p(t)-q$ handelt
es sich um den Abstandsvektor in Abhängigkeit von $t$.

Ansatz: Es gibt genau ein $t$, so dass gilt:
\begin{equation}
\langle\underline d,\underline v\rangle=0.
\end{equation}
Lösung:
\begin{equation}
t = \frac
  {\langle\underline v,q{-}p_0\rangle}
  {\langle\underline v,\underline v\rangle}.
\end{equation}

\subsection{Ebenen}\index{Ebene}
\subsubsection{Parameterdarstellung}
\index{Parameterdarstellung!einer Ebene}
Seien $\uv u, \uv v$ zwei nicht kollineare Vektoren.

Punktrichtungsform:
\begin{equation}\label{eq:Ebene-Punktrichtungsform}
p(s,t) = p_0+s\uv u+t\uv v.
\end{equation}

\subsubsection{Parameterfreie Darstellung}
Seien $\uv v, \uv w$ zwei nicht kollineare Vektoren.
Durch
\begin{equation}
E = \{p\mid (p-p_0)\wedge\uv v\wedge\uv w=0\}.
\end{equation}
wird eine Ebene beschrieben.

\strong{Hesse-Form:}
\begin{equation}
E = \{p\mid \langle\uv n,p-p_0\rangle=0\},
\end{equation}
$p_0$: Stützpunkt, $\uv n$: Normalenvektor. Die Hesse-Form einer
Ebene ist nur im dreidimensionalen Raum möglich.
Den Normalenvektor bekommt man aus \eqref{eq:Ebene-Punktrichtungsform}
mit $\uv n = \uv u\times\uv v$.

Es gilt:
\begin{equation}
\langle\uv n,p-p_0\rangle\iff \langle\uv n,p\rangle = \langle\uv n,p_0\rangle.
\end{equation}
Über den Zusammenhang $\uv n=(a,b,c)$, $p=(x,y,z)$ und $d=\langle\uv n,p_0\rangle$
ergibt sich die

\strong{Koordinatenform:}
\begin{equation}
E = \{(x,y,z)\mid ax+by+cz = d\}.
\end{equation}

\subsubsection{Abstand Punkt zu Ebene}
Sei $p(s,t):=p_0+s\uv u+t\uv v$ die Punktrichtungsform einer Ebene
und sei $q$ ein weiterer Punkt. Bei $\uv d(s,t):=p-q$ handelt es sich um
den Abstandsvektor in Abhängigkeit von $(s,t)$.

Ansatz: Es gibt genau ein Tupel $(s,t)$, so dass gilt:
\begin{equation}
\langle\uv d,\uv u\rangle=0\enspace\text{und}\enspace
\langle\uv d,\uv v\rangle=0.
\end{equation}
Lösung: Es ergibt sich ein LGS:
\begin{equation}
\begin{bmatrix}
\langle\uv u,\uv v\rangle & \langle\uv v,\uv v\rangle\\
\langle\uv v,\uv v\rangle & \langle\uv u,\uv v\rangle
\end{bmatrix}
\begin{bmatrix}
s\\ t
\end{bmatrix}
= \begin{bmatrix}
\langle\uv v,q{-}p_0\rangle\\
\langle\uv u,q{-}p_0\rangle
\end{bmatrix}.
\end{equation}
Bemerkung: Die Systemmatrix $g_{ij}$ ist der metrische Tensor für die
Basis $B=(\uv u,\uv v)$. Die Lösung des LGS ist:
\begin{gather}
s = \frac
  {\langle g_{12}\uv v-g_{12}\uv u, q{-}p_0\rangle}
  {g_{11}^2-g_{12}^2},\\
t = \frac
  {\langle g_{12}\uv u-g_{12}\uv v, q{-}p_0\rangle}
  {g_{11}^2-g_{12}^2}.
\end{gather}



\chapter{Algebra}

\section{Gruppentheorie}

\subsection{Grundlagen}

\begin{Definition}[Gruppe]
Das Tupel $(G,*)$ bestehend aus einer Menge $G$ und
Abbildung $*\colon G\times G\to\Omega$ heißt Gruppe, wenn die folgenden
Axiome erfüllt sind:
\begin{enumerate}
\item[(G1)] Für alle $a,b\in G$ gilt $a*b\in G$. D.\,h., man darf $G=\Omega$ setzen.
\item[(G2)] Es gilt das Assoziativgesetz: für alle $a,b,c\in G$ gilt $(a*b)*c=a*(b*c)$.
\item[(G3)] Es gibt ein Element $e\in G$, so dass $e*g=g=g*e$ für jedes $g\in G$ gilt.
\item[(G4)] Zu jedem $g\in G$ gibt es ein $g^{-1}\in G$ so dass $g*g^{-1}=e=g^{-1}*g$ gilt.
\end{enumerate}
Das Element $e$ wird neutrales Element der Gruppe genannt.
Das Element $g^{-1}$ wird inverses Element zu $g$ genannt.
Anstelle von $a*b$ schreibt man auch kurz $ab$. Ist $(G,+)$ eine
Gruppe, dann schreibt man immer $a+b$, und $-g$ anstelle von $g^{-1}$.
\end{Definition}

\begin{Korollar}
Das neutrale Element einer Gruppe $G$ ist eindeutig bestimmt.
D.\,h., es gibt keine zwei unterschiedlichen neutralen Elemente. 
\end{Korollar}
\begin{Beweis}
Seien $e,e'$ zwei neutrale Elemente von $G$. Nach Axiom (G3)
gilt dann $e=e'e$, und weiter $e'e=e'$ bei nochmaliger Anwendung
von (G3). Daher ist $e=e'$.\;\qedsymbol
\end{Beweis}

\begin{Korollar}
Sei $G$ eine Gruppe. Zu jedem Element $g\in G$ ist das inverse
Element $g^{-1}$ eindeutig bestimmt. D.\,h., es kann keine zwei
unterschiedlichen inversen Elemente zu $g$ geben.
\end{Korollar}
\begin{Beweis}
Seien $a,b$ zwei inverse Elemente zu $g$. Nach Axiom (G3), Axiom (G2)
und Axiom (G4) gilt
\[a \stackrel{(G3)}= ae \stackrel{(G4)} = a(gb) \stackrel{(G2)}
= (ag)b \stackrel{(G4)}= eb \stackrel{(G3)}= b.\]
Daher ist $a=b$.\;\qedsymbol
\end{Beweis}

\begin{Definition}[Untergruppe]
Sei $(G,*)$ eine Gruppe. Eine Teilmenge $U\subseteq G$ heißt
Untergruppe von $G$, kurz $U\le G$, wenn $U$ bezüglich derselben
Verknüpfung $*$ selbst eine Gruppe $(U,*)$ bildet.
\end{Definition}

\begin{Korollar}
Jede Gruppe $G$ besitzt die Untergruppen $\{e\}\le G$ und $G\le G$,
wobei $e\in G$ das neutrale Element ist. Man spricht von den
trivialen Untergruppen.
\end{Korollar}
\begin{Beweis}
Die Aussage $G\le G$ ist trivial, denn $G\subseteq G$ ist allgemeingültig
und $(G,*)$ bildet nach Voraussetzung eine Gruppe. Zu (G1):
Es gilt $ee=e$. Da es nur diese eine Möglichkeit gibt, sind damit alle
überprüft.
Zu (G2): Das Assoziativgesetz wird auf Elemente der Teilmenge vererbt.
Zu (G3): Das neutrale Element ist in $\{e\}$ enthalten.
Zu (G4): Das neutrale Element ist gemäß $ee=e$ zu sich selbst invers.
Da $e$ das einzige Element von $\{e\}$ ist, sind damit alle
überprüft.\;\qedsymbol
\end{Beweis}

\section{Ringtheorie}

\subsection{Grundlagen}

\begin{Definition}[Ring]
Eine Struktur $(R,+,\cdot)$ heißt genau dann Ring, wenn die folgenden
Axiome erfüllt sind
\begin{enumerate}
\item[1.] $(R,+)$ ist eine kommutative Gruppe.
\item[2.] $(R,\cdot)$ ist eine Halbgruppe.
\item[3.] Für alle $a,b,c\in R$ gilt $a(b+c) = ab+ac$. (Linksdistributivgesetz)
\item[4.] Für alle $a,b,c\in R$ gilt $(a+b)c = ac+bc$. (Rechtsdistributivgesetz)
\end{enumerate}
\end{Definition}
Bemerkung: Das neutrale Element von $(R,+)$ wird als Nullelement
bezeichnet und meist $0$ geschrieben.

\begin{Definition}[Ring mit Eins]
Ein Ring $R$ heißt genau dann Ring mit Eins, wenn $(R,\cdot)$ ein
Monoid ist. Monoid heißt, es gibt ein Element $e\in R$, so dass
$e\cdot a = a$ und $a\cdot e = a$ für alle $a\in R$.
\end{Definition}
Bemerkung: Man bezeichnet $e$ als Einselement des Rings.

\begin{Korollar}
Sei $R$ ein Ring und $0\in R$ das Nullelement.\\
Für jedes $a\in R$ gilt $0\cdot a = 0$ und $a\cdot 0 = 0$.
\end{Korollar}
\begin{Beweis} Man rechnet
\[0a = 0a+0 = 0a+0a-0a = (0+0)a-0a = 0a-0a = 0.\]
\end{Beweis}
Die Rechnung für $a\cdot 0$ ist analog.\;\qedsymbol

\begin{Korollar}\label{Minus-vorziehen}
Sei $R$ ein Ring und $a,b\in R$, dann gilt $(-a)b = -(ab) = a(-b)$.
\end{Korollar}
\begin{Beweis}
Man rechnet
\begin{align*}
(-a)b &= (-a)b+0 = (-a)b+ab-(ab) = ((-a)+a)b-(ab)\\
&= 0b-(ab) = 0-(ab) = -(ab).\;\qedsymbol
\end{align*}
\end{Beweis}

\begin{Korollar}[»Minus mal minus macht plus«]\mbox{}\\*
Sei $R$ ein Ring und $a,b\in R$, dann gilt
$(-a)(-b) = ab$.
\end{Korollar}
Beachtung von $-(-x)=x$ nach zweifacher Anwendung von
Korollar \ref{Minus-vorziehen} bringt
\[(-a)(-b) = -((-a)b) = -(-(ab)) = ab.\;\qedsymbol\]

\subsection{Ringhomomorphismen}

\begin{Definition}[Ringhomomorphismus]
Seien $R,R'$ Ringe. Eine Abbildung $\varphi\colon R\to R'$ heißt
Ringhomomorphismus, falls
\begin{gather*}
\varphi(x+y) = \varphi(x)+\varphi(y),\\
\varphi(xy) = \varphi(x)\varphi(y)
\end{gather*}
für alle $x,y\in R$ gilt. Liegen Ringe mit Eins vor, und gilt
zusätzlich $\varphi(1)=1$, dann spricht man von einem unitären
Ringhomomorphismus.
\end{Definition}

\begin{Korollar}
Bei jedem Ringhomomorphismus $\varphi$ gilt
$\varphi(kx) = k\varphi(x)$ für $k\in\Z$.
\end{Korollar}
\strong{Beweis.} Für $k>0$ ist
\[\varphi(kx) = \varphi(\sum_{i=1}^k x)
= \sum_{i=1}^k \varphi(x) = k\varphi(x).\]
Nun der Fall $k=0$. Man rechnet $f(0) = f(0+0) = f(0)+f(0)$.
Subtraktion von $f(0)$ auf beiden Seiten ergibt $f(0)=0$.
Schließlich bleibt noch $f(-kx)=-kf(x)$ für $k>0$ zu zeigen. Hier
rechnet man zunächst
\[0 = f(0) = f(-x+x) = f(-x)+f(x).\]
Subtraktion von $f(x)$ auf beiden Seiten ergibt $f(-x) = -f(x)$.
Somit gilt
\[f(-kx) = -f(kx) = -kf(x).\;\qedsymbol\]

\section{Polynomringe}

\subsection{Einsetzungshomomorphismus}

\begin{Satz}
Die Abbildung $\Phi\colon\R[X]\to\Abb(\R,\R)$ mit $\Phi(f)(x):=f(x)$
ist injektiv.
\end{Satz}
\begin{Beweis} Sei $f=\sum_{k=0}^n a_k X^k$
und $g=\sum_{k=0}^n b_k X^k$, wobei $n=\max(\deg f,\deg g)$.
Zu zeigen ist
\[(\forall x\in\R\colon \Phi(f)(x)=\Phi(g)(x))\implies f=g,\]
das heißt
\[(\forall x\in\R\colon \sum_k a_k x^k = \sum_k b_k x^k)\implies (\forall k\colon a_k=b_k).\]
Die Umformung der Voraussetzung ergibt $\sum_k (b_k-a_k)x^k = 0$.
D.\,h., jedes der $(b_k-a_k)$ muss verschwinden. Zu zeigen ist also
lediglich
\[(\forall x\in\R\colon \sum_{k=0}^n c_k x^k = 0)\implies (\forall k\colon c_k=0).\]
Wenn $f(x)=0$ für alle $x$ ist, muss auch die Ableitung $D^m f(x)=0$
sein. Es gilt $D^k x^k = k!$, und daher
\[D^n\sum_{k=0}^n c_k x^k = n!\cdot c_n = 0 \implies c_n=0.\]
Demnach ergibt sich dann aber auch
\[D^{n-1}\sum_{k=0}^n c_k x^k = (n-1)!\cdot c_{n-1} = 0\implies c_{n-1}=0\]
usw. Man erhält $c_k=0$ für alle $k$.\;\qedsymbol
\end{Beweis}


\chapter{Wahrscheinlichkeitsrechnung}

\section{Diskrete Wahrscheinlichkeitsräume}

\begin{Definition}[Diskreter Wahrscheinlichkeitsraum]\mbox{}\\*
Sei $\Omega$ eine höchstens abzählbare Menge. Das
Paar $(\Omega,P)$ nennt man diskreten Wahrscheinlichkeitsraum,
wenn
\[P\colon 2^\Omega\to [0,1],\quad P(A):=\sum_{\omega\in A} P(\{\omega\})\]
die Eigenschaft $\sum_{\omega\in\Omega} P(\{\omega\}) = 1$ besitzt.
\end{Definition}
Bemerkung: Man schreibt auch $P(\omega):=P(\{\omega\})$.

\begin{Definition}[Reelle Zufallsgröße]\mbox{}\\*
Sei $(\Omega,P)$ ein diskreter Wahrscheinlichkeitsraum.
Eine Funktion $X\colon\Omega\to\R$ nennt man Zufallsgröße.
Die Verteilung von $X$ ist definiert gemäß $P_X(A):=P(X^{-1}(A))$.
\end{Definition}

\begin{Definition}[Erwartungswert]%
\label{def:expected-value}\mbox{}\\*
Sei $(\omega_k)$ eine beliebige Abzählung von $\Omega$.
Ist die Reihe $\sum_{k=0}^{|\Omega|} X(\omega_k)P(\{\omega_k\})$
absolut konvergent, dann nennt man
\[E(X) := \sum_{\omega\in\Omega} X(\omega)P(\{\omega\})\]
den Erwartungswert von $X$.
\end{Definition}

\begin{Satz} Es gilt
\[E(X) = \sum_{x\in X(\Omega)} xP(X^{-1}(x))
= \sum_{x\in X(\Omega)} xP(X=x).\]
\end{Satz}
\strong{Beweis.} Zunächst gilt
\begin{gather*}
\sum_{\substack{\omega\in\Omega\\ X(\omega)=x}}P(\omega)
= P(\bigcup_{\substack{\omega\in\Omega\\ X(\omega)=x}} \{\omega\})
= P(\{\omega\in\Omega\mid X(\omega)=x\})
= P(X^{-1}(x)).
\end{gather*}
Da die Reihe zu $E(X)$ nach Def. \ref{def:expected-value}
absolut konvergent ist, darf sie beliebig umgeordnet werden und
man bekommt
\begin{gather*}
E(X) = \sum_{\omega\in\Omega}X(\omega)P(\omega)
= \sum_{x\in X(\Omega)}\sum_{\substack{\omega\in\Omega\\ X(\omega)=x}} xP(\omega)
= \sum_{x\in X(\Omega)} x\sum_{\substack{\omega\in\Omega\\ X(\omega)=x}}P(\omega)\\
= \sum_{x\in X(\Omega)} xP(X^{-1}(x)).\;\qedsymbol
\end{gather*}

\newpage
\begin{Korollar} Der Erwartungswertoperator ist ein lineares Funktional,
d.\,h. es gilt $E(aX)=aE(X)$ und $E(X+Y)=E(X)+E(Y)$. 
\end{Korollar}
\strong{Beweis.} Aufgrund der Konvergenz der Reihen gilt
\[E(aX) = \sum_{\omega\in\Omega}aX(\omega)P(\omega)
= a\sum_{\omega\in\Omega}X(\omega)P(\omega) = aE(X)\]
und
\begin{align*}
E(X+Y) &= \sum_{\omega\in\Omega} (X(\omega)+Y(\omega))P(\omega)
= \sum_{\omega\in\Omega} (X(\omega)P(\omega)+Y(\omega)P(\omega))\\
&= \sum_{\omega\in\Omega} X(\omega)P(\omega)
+ \sum_{\omega\in\Omega} Y(\omega)P(\omega) = E(X)+E(Y).\;\qedsymbol
\end{align*}

\begin{Korollar} Ist $X\le Y$, dann ist auch $E(X)\le E(Y)$.
\end{Korollar}
\strong{Beweis.} Gemäß $P(\omega)\ge 0$ ist
\begin{gather*}
X\le Y\iff X(\omega)\le Y(\omega)\iff 0\le Y(\omega)-X(\omega)
\iff 0\le (Y(\omega)-X(\omega))P(\omega).
\end{gather*}
Somit hat man
\begin{gather*}
X\le Y\implies 0\le E(Y-X) = \sum_{\omega\in\Omega} (Y(\omega)-X(\omega))P(\omega),
\end{gather*}
und gemäß Linearität daher
\begin{gather*}
X\le Y\implies 0\le E(Y-X) = E(Y)-E(X) \iff E(X)\le E(Y).\;\qedsymbol
\end{gather*}

\begin{Definition}[Unabhängige Ereignisse]\mbox{}\\*
Zwei Ereignisse $A,B$ heißen unabhängig, falls $P(A\cap B)=P(A)P(B)$.
\end{Definition}

\begin{Definition}[Unabhängige Zufallsgrößen]\mbox{}\\*
Zwei Zufallsgrößen $X,Y\colon\Omega\to\R$ heißen unabhängig, wenn
die Ereignisse $\{X\in A\}$ und $\{X\in B\}$
für alle Mengen $A,B\subseteq\R$ unabhängig sind.
\end{Definition}

\begin{Satz}
Zwei Zufallsgrößen $X,Y\colon\Omega\to\R$ sind genau dann unabhängig,
wenn für alle $x\in X(\Omega)$ und $y\in Y(\Omega)$ gilt:
\[P(X=x,Y=y)= P(X=x)P(Y=y).\]
\end{Satz}
\strong{Beweis.} Sind $X,Y$ unabhängig, dann ist
\begin{align*}
P(X=x,Y=y) &= P(\{X\in\{x\}\}\cap\{Y\in\{y\}\})
= P(\{X\in\{x\}\})P(\{Y\in\{y\}\})\\
&= P(X=x)P(Y=y).
\end{align*}
Umgekehrt gelte nun $P(X=x,Y=y)=P(X=x)P(Y=y)$, dann ist
\begin{gather*}
P(\{X\in A\}\cap\{Y\in B\})
= P(\bigcup_{x\in A}\{X=x\}\cap\bigcup_{y\in B}\{Y=y\})\\
= P(\bigcup_{x\in A}\bigcup_{y\in B}(\{X=x\}\cap\{Y=y\}))
= \sum_{x\in A}\sum_{y\in B}P(\{X=x\}\cap\{Y=y\})\\
= \sum_{x\in A}\sum_{y\in B}P(X=x)P(Y=y)
= \sum_{x\in A}P(X=x)\sum_{y\in B}P(Y=y)\\
= P(\bigcup_{x\in A}\{X=x\})P(\bigcup_{y\in B}\{Y=y\})
= P(X\in A)P(Y\in B).\;\qedsymbol
\end{gather*}

\begin{Definition}[Bedingter Erwartungswert]%
\label{def:cond-expected-value}\mbox{}\\*
\[E(X\mid A) = \frac{E(1_A X)}{P(A)} = \frac{1}{P(A)}\sum_{\omega\in A} X(\omega)P(\{\omega\}).\]
\end{Definition}

\begin{Satz} Es gilt
\[E(X\mid A) = \frac{1}{P(A)}\sum_x xP(\{X=x\}\cap A)
= \sum_x xP(X=x\mid A),\]
wobei sich die Summe über alle $x\in X(\Omega)$ erstreckt.
\end{Satz}
\strong{Beweis.} Man kann rechnen
\begin{align*}
E(1_A X) &= \sum_{\omega\in\Omega} 1_A(\omega) X(\omega) P(\{\omega\})
= \sum_x\sum_{\omega\in X^{-1}(x)} 1_A(\omega) X(\omega) P(\{\omega\})\\
&= \sum_x x\sum_{\omega\in X^{-1}(x)} 1_A(\omega) P(\{\omega\})
= \sum_x x\sum_{\omega\in X^{-1}(x)\cap A} P(\{\omega\})\\
&= \sum_x x P(X^{-1}(x)\cap A),
\end{align*}
wobei $X^{-1}(x) = \{X=x\}$.\;\qedsymbol

\begin{Korollar}\label{prob-as-expected-value}
Es gilt $P(A) = E(1_A)$, wobei $1_A$ die Indikatorfunktion ist.
\end{Korollar}
\strong{Beweis.} Gemäß Definition des Erwartungswertes ist
\[E(1_A) = \sum_{\omega\in\Omega} 1_A(\omega)P(\{\omega\})
= \sum_{\omega\in A}P(\{\omega\}) = P(A).\;\qedsymbol\]

\begin{Korollar}
Es gilt $P(A\mid B) = E(1_A\mid B)$, wobei $1_A$ die Indikatorfunktion ist.
\end{Korollar}
\strong{Beweis.} Gemäß Definition \ref{def:cond-expected-value}
und Korollar \ref{prob-as-expected-value} ist
\[E(1_A\mid B) = \frac{E(1_A 1_B)}{P(B)} = \frac{E(1_{A\cap B})}{P(B)}
= \frac{P(A\cap B)}{P(B)} = P(A\mid B).\qedsymbol\]

\section{Allgemeine Wahrscheinlichkeitsräume}

\begin{Satz}\label{rv-transform-pdf}
Sei $g\colon\R\to\R$ eine streng monotone Funktion. Seien
$X,Y$ Zufallsgrößen mit Dichten $f_X,f_Y$. Ist $Y=g(X)$,
dann gilt
\[f_Y(y) = \frac{f_X(g^{-1}(y))}{|g'(g^{-1}(y))|}.\]
\end{Satz}
\strong{Beweis.} Sei $g$ streng monoton steigend. Dann kann man rechnen
\[F_Y(y) = P(Y\le y) = P(g(X)\le y) = P(X\le g^{-1}(y)) = F_X(g^{-1}(y)).\]
Gemäß der Kettenregel findet man
\[f_Y(y) = \frac{\mathrm d}{\mathrm dy}F_Y(y)
= f_X(g^{-1}(y))\frac{\mathrm d}{\mathrm dy}g^{-1}(y) = \frac{f_X(g^{-1}(y))}{g'(g^{-1}(y))}.\]
Sei $g$ nun streng monoton fallend. Dann kann man rechnen
\[F_Y(y) = P(g(X)\le y) = P(X\ge g^{-1}(y)) = 1 - P(X < g^{-1}(y))
= 1 - F_X(g^{-1}(y)).\]
Entsprechend findet man
\[f_Y(y) = -\frac{f_X(g^{-1}(y))}{g'(g^{-1}(y))}.\]
Nun ist $g'$ in beiden Fällen frei von Nullstellen. Demnach ist
$\sgn(g'(x))$ konstant für alle $x$ und wir haben allgemein
\[f_Y(y) = \sgn(g'(x))\frac{f_X(x)}{g'(x)} = \frac{f_X(x)}{|g'(x)|}\]
mit $x=g^{-1}(y)$.\;\qedsymbol

\begin{Satz}[»LOTUS: Law of the unconscious statistican«]\mbox{}\\*
Ist $g\colon\R\to\R$ streng monoton, dann muss gelten
\[E(g(X)) = \int_{-\infty}^\infty g(x)f_X(x)\,\mathrm dx.\]
\end{Satz}
\strong{Beweis.} Sei $Y=g(X)$. Mit Satz \ref{rv-transform-pdf}
und Substitution $y=g(x)$ kann man rechnen
\begin{align*}
E(g(X)) &= E(Y) = \int_{-\infty}^\infty yf_Y(y)\,\mathrm dy
= \int_{-\infty}^\infty y\frac{f_X(g^{-1}(y))}{|g'(g^{-1}(y))|}\,\mathrm dy\\
&= \int_{g^{-1}(-\infty)}^{g^{-1}(\infty)} g(x)
\frac{f_X(x)}{|g'(x)|}g'(x)\,\mathrm dx\\
&= \sgn(g')\int_{-\sgn(g')\infty}^{\sgn(g')\infty}g(x) f_X(x)\mathrm dx
= \int_{-\infty}^\infty g(x) f_X(x)\mathrm dx.\;\qedsymbol
\end{align*}



\printindex
\end{document}


