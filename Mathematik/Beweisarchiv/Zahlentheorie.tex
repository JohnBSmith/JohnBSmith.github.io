
\chapter{Zahlentheorie}
\section{Kongruenzen und Teilbarkeit}

\begin{Definition}[Teiler] Für $a,m\in\Z$ ist die Relation »$m$ teilt
$a$« definiert als
\[m\mid a\defiff \exists k\in\Z\colon a = km.\]
\end{Definition}

\begin{Definition}[Kongruenz]\label{def:cong}
Für $a,b,m\in\Z$ ist die Relation »$a$ ist kongruent zu $b$ modulo
$m$« definiert als
\[a\equiv b \pmod m \defiff m\mid (a-b).\]
\end{Definition}

\begin{Korollar}
Es gilt $m\mid a$ genau dann, wenn $a\equiv 0\pmod m$.
\end{Korollar}
\begin{Beweis}
Spezialisierung von Def. \ref{def:cong} mit $b:=0$.\,\qedsymbol
\end{Beweis}

\begin{Definition}[Teilermenge] Die Teilermenge
einer ganzen Zahl $a$ ist definiert durch
\[m\in T(a)\defiff m\mid a.\]
\end{Definition}
\strong{Bemerkung.} Wir setzen $T_{\ge 1}(a):=T(a)\cap\Z_{\ge 1}$ und
$T_{\ge 0}(a):=T(a)\cap\Z_{\ge 0}$.

\begin{Korollar}\label{divisor-divisor-subset} Es gilt
\[a\mid b \iff T(a)\subseteq T(b).\]
\end{Korollar}
\begin{Beweis}
Zur Implikation von rechts nach links.
Die Aussage $T(a)\subseteq T(b)$ ist laut ihrer Definition
äquivalent zu $\forall m\colon m\mid a\Rightarrow m\mid b$.
Die Anwendung dieser Prämsse auf den Ansatz $a\mid a$
liefert sofort $a\mid b$.

Zur Implikation von links nach rechts. Expandiert man die Aussage
durch Einsetzen der Definitionen, ist
\[(\exists k\colon b=ka)\implies
(\forall m\colon (\exists x\colon a=xm)\implies (\exists y\colon b=ym)).\]
zu zeigen. Wir haben einen Zeugen $k$ für $b=ka$ gegeben. 
Gemäß Prämisse liegt ein Zeuge $x$ für $a=xm$ vor, wonach $ka=kxm$
gilt, also $b=kxm$. Ergo ist $y:=kx$ ein Zeuge für
$\exists y\colon b=ym$.\,\qedsymbol
\end{Beweis}

\begin{Korollar}
Die Teilerrelation ist transitiv, das heißt,
$a\mid b$ und $b\mid c$ impliziert $a\mid c$.
\end{Korollar}
\begin{Beweis}
Unter Nutzung von Korollar \ref{divisor-divisor-subset}
nimmt die Aussage die Gestalt
\[T(a)\subseteq T(b)\land T(b)\subseteq T(c)\implies T(a)\subseteq T(c)\]
an. Diese Aussage ist nun aber offenkundig, da die Relation »ist
Teilmenge von« eine transitive ist.\,\qedsymbol
\end{Beweis}

\begin{Korollar}\label{divides-weaken}
Gilt $m\mid a$ und $b\in\Z$, so gilt auch $m\mid (ab)$.
\end{Korollar}
\begin{Beweis}
Die Prämisse liefert einen Zeugen $x$ mit $a=xm$, womit $ab=bxm$
gilt. Ergo existiert mit $y:=bx$ ein Zeuge für
$\exists y\colon ab=ym$.\,\qedsymbol
\end{Beweis}

\begin{Korollar}
Die Kongruenzrelation ist eine Äquivalenzrelation. Das heißt, es gilt
\begin{align*}
& a\equiv a\pmod m, && \text{(Reflexivität)}\\
& a\equiv b \implies b\equiv a\pmod m, && \text{(Symmetrie)}\\
& a\equiv b\land b\equiv c\implies a\equiv c\pmod m. && \text{(Transitivität)}
\end{align*}
\end{Korollar}
\begin{Beweis}
Zur Reflexivität. Es gilt die äquivalente Umformung
\[a\equiv a\pmod m \iff m\mid (a-a) \iff m\mid 0\iff (\exists k\colon 0 = km).\]
Die letzte Aussage ist durch $k=0$ erfüllt.

Zur Symmetrie. Die Prämisse liefert einen Zeugen $x$ mit $a-b = xm$,
womit $b-a = -xm$ gilt. Ergo ist $y:=-x$ ein Zeuge für die
Existenzaussage $\exists y\colon b-a = ym$.

Zur Reflexivität. Die Prämissen liefern Zeugen $x$ mit $a-b=xm$
und $y$ mit $b-c=ym$. Infolge gilt
\[a-c = (a-b) + (b-c) = xm + ym = (x+y)m.\]
Ergo ist $z:=x+y$ ein Zeuge für $\exists z\colon a-c = zm$.\,\qedsymbol
\end{Beweis}

\begin{Korollar}\label{cong-shift}
Für ganze Zahlen $a,b,c$ gilt
\begin{gather*}
a\equiv b\pmod m \iff a+c\equiv b+c \pmod m,\\
a\equiv b\pmod m \iff a-c\equiv b-c \pmod m.
\end{gather*}
\end{Korollar}
\begin{Beweis}
Unter Nutzung der Definition findet sich die äquivalente Umformung
\begin{align*}
a+c\equiv b+c\pmod m &\iff m\mid ((a+c)-(b+c))
\iff m\mid (a-b)\\
&\iff a\equiv b\pmod m.
\end{align*}
Der Beweis der zweiten Aussage verläuft analog.\,\qedsymbol
\end{Beweis}

\begin{Korollar}\label{cong-scale}
Für ganze Zahlen $a,b,c$ gilt
\[a\equiv b\pmod m\implies ac=bc\pmod m.\]
\end{Korollar}
\begin{Beweis} Die Prämisse liefert einen Zeugen $x$ mit
$a-b = xm$, womit $ac-bc = cxm$ gilt. Ergo existiert mit $y:=cx$ ein Zeuge
für $\exists y\colon ac-bc = ym$.\,\qedsymbol
\end{Beweis}

\begin{Korollar}\label{cong-add-sub-mul}
Gilt $a\equiv a'\pmod m$ und $b\equiv b'\pmod m$,
so gilt auch
\begin{gather*}
a+b\equiv a'+b'\pmod m,\\
a-b\equiv a'-b'\pmod m,\\
ab\equiv a'b'\pmod m.
\end{gather*}
\end{Korollar}
\begin{Beweis}
Laut Prämisse liegen Zeugen $x$ mit $a=a'+xm$ und $y$ mit $b=b'+ym$ vor.
Infolge gilt
\[a+b = a'+xm+b'+ym = a'+b'+(x+y)m.\]
Ergo existiert mit $z:=x+y$ ein Zeuge für
$\exists z\colon (a+b)=(a'+b')+zm$. Der Beweis der zweiten Regel
verläuft analog. Bei der Multiplikation gilt
\[ab = (a'+xm)(b'+ym) = a'b' + a'ym+b'xm + xym^2
= a'b' + (a'y+b'x+xym)m.\]
Ergo existiert mit $z:=a'y+b'x+xym$ ein Zeuge für
$\exists z\colon ab = a'b'+zm$.\,\qedsymbol
\end{Beweis}

\begin{Korollar}\label{cong-pow}
Gilt $a\equiv a'\pmod m$ und $n\in\Z_{\ge 0}$, so gilt auch
$a^n\equiv (a')^n\pmod m$.
\end{Korollar}
\begin{Beweis}
Induktion über $n$ mit Induktionsanfang bei $n=0$. Mit $a^0=1$ und
$(a')^0=1$ wird die Behauptung in diesem Fall zu $1\equiv 1$, die
aufgrund der Reflexivität gilt. Induktionsschritt.
Wendet man Korollar \ref{cong-add-sub-mul} auf die Prämisse
$a\equiv a'$ und die Induktionsvoraussetzung $a^n\equiv (a')^n$
an, findet sich $aa^n \equiv a'(a')^n$, also
$a^{n+1}\equiv (a')^{n+1}$.\,\qedsymbol
\end{Beweis}

\begin{Korollar}\label{cong-sum}
Gilt $a_k\equiv a_k'\pmod m$ für alle $k$, so gilt auch\\
$\sum_{k=0}^{n-1} a_k\equiv \sum_{k=0}^{n-1} a_k'\pmod m$.
\end{Korollar}
\begin{Beweis}
Induktion über $n$. Für $n=0$ wird die Behauptung zu $0\equiv 0$, die
aufgrund der Reflexivität gilt. Induktionsschritt. Wendet man Korollar
\ref{cong-add-sub-mul} auf die
Prämisse $a_n\equiv a_n'$ und die Induktionsvoraussetzung
$\sum_{k=0}^{n-1} a_k\equiv \sum_{k=0}^{n-1} a_k'$ an, folgt
\[\sum_{k=0}^{n-1} a_k = a_n + \sum_{k=0}^{n-1} a_k\equiv a_n' + \sum_{k=0}^{n-1} a_k'
= \sum_{k=0}^{n-1} a_k'.\,\qedsymbol\]
\end{Beweis}

\begin{Korollar}
Sei $p\in\Z[X]$, also ein Polynom mit ganzzahligen Koeffizienten.
Gilt $x\equiv x'\pmod m$, so gilt auch $p(x)\equiv p(x')\pmod m$.
\end{Korollar}
\begin{Beweis}
Laut Korollar \ref{cong-pow} gilt $x^k\equiv (x')^k$ für jedes
$k\ge 0$. Im weiteren Fortgang gilt $a_k x^k\equiv a_k (x')^k$
wegen Korollar \ref{cong-shift}. Mit Korollar \ref{cong-sum} erhält man
schließlich
\[p(x) = \sum_{k=0}^n a_k x^k\equiv\sum_{k=0}^n a_k (x')^k = p(x').\,\qedsymbol\]
\end{Beweis}

\begin{Satz}
Für jede ganze Zahl $n$ gilt $2\mid n \iff 2\mid n^2$.
\end{Satz}
\begin{Beweis}
Die Implikation von links nach rechts gilt gemäß
Korollar \ref{divides-weaken} mit $a:=n$ und $b:=n$.
Zur Implikation von rechts nach links. Wir zeigen die Kontraposition
\[\neg(2\mid n) \implies \neg(2\mid n^2).\]
Laut Prämisse ist $n$ ungerade, also von der Form $n=2k+1$ mit $k\in\Z$.
Nun gilt
\[n^2 = (2k+1)^2 = 4k^2 + 4k + 1 = 2(2k^2 + 2k) + 1,\]
wonach $n^2$ ebenfalls ungerade sein muss.\,\qedsymbol
\end{Beweis}

\section{Primzahlen}

\begin{Definition}[Teilerfunktion]\newlinefirst
Für eine positive ganze Zahl $n$ definiert man
\[\sigma_k(n) := \sum_{d\mid n} d^k,\]
wobei mit $d\mid n$ die positiven Teiler $d\in T_{\ge 1}(n)$ gemeint
sind.
\end{Definition}

\begin{Definition}[Primzahl]\newlinefirst
Eine positive ganze Zahl $n$ wird Primzahl genannt, wenn sie
zwei unterschiedliche positive Teiler besitzt, womit $\sigma_0(n)=2$
gemeint ist.
\end{Definition}

\begin{Korollar}
Eine Zahl $n$ ist genau dann eine Primzahl, wenn $n\ge 2$ ist und
ihre einzigen beiden positiven Teiler 1 und $n$ selbst sind.
\end{Korollar}
\begin{Beweis}
Jede ganze Zahl besitzt 1 und sich selbst als Teiler. Somit muss
$\sigma_0(n)=2$ für $n\ge 2$ äquivalent zu $T_{\ge 1}(n)=\{1,n\}$
sein.\,\qedsymbol
\end{Beweis}

\begin{Satz}[Satz des Euklid]
Es gibt unendlich viele Primzahlen.
\end{Satz}
\begin{Beweis}[Klassischer Beweis]
Sei $M=\{p_1,\ldots,p_n\}$ eine endliche Menge von Primzahlen.
Es wird gezeigt, dass eine weitere Primzahl $p\notin M$ existiert.
Man bilde dazu das Produkt $m=\prod_{k=1}^n p_k$. Nun ist $m+1$ entweder
prim oder nicht. Falls $m+1$ prim ist, ist mit $p=m+1$ eine weitere
Primzahl gefunden. Sei also $m+1$ nicht prim, womit mindestens
ein Primfaktor $p$ enthalten ist. Angenommen, es wäre $p=p_k$ für eines
der $k$. Dann gälte $p_k\mid m+1$. Es gilt gemäß Konstruktion von $m$
aber auch $p_k\mid m$. Ergo wäre $p_k$ ebenso ein Teiler der
Differenz $(m+1)-m = 1$. Das ist absurd, weil keine Primzahl
ein Teiler der Zahl~1 ist.\,\qedsymbol
\end{Beweis}

