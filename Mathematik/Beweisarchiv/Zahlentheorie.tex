
\chapter{Zahlentheorie}

\begin{Definition}[Teiler] Für $a,m\in\Z$ ist die Relation »$m$ teilt
$a$« definiert als
\[m\mid a\defiff \exists k\in\Z\colon a = km.\]
\end{Definition}

\begin{Definition}[Kongruenz]
Für $a,b,m\in\Z$ ist die Relation »$a$ ist kongruent zu $b$ modulo
$m$« definiert als
\[a\equiv b \pmod m \defiff m\mid (b-a).\]
\end{Definition}

\begin{Definition}[Teilermenge] Die Teilermenge
einer ganzen Zahl $a$ ist definiert durch
\[m\in T(a)\defiff m\mid a.\]
\end{Definition}
\strong{Bemerkung.} Wir setzen $T_{>0}(a):=T(a)\cap\Z_{>0}$ und
$T_{\ge 0}(a):=T(a)\cap\Z_{\ge 0}$.

\begin{Korollar}\label{divisor-divisor-subset} Es gilt
\[a\mid b \iff T(a)\subseteq T(b).\]
\end{Korollar}
\begin{Beweis}
Zur Implikation von rechts nach links.
Die Aussage $T(a)\subseteq T(b)$ ist laut ihrer Definition
äquivalent zu $\forall m\colon m\mid a\Rightarrow m\mid b$.
Die Anwendung dieser Prämsse auf den Ansatz $a\mid a$
liefert sofort $a\mid b$.

Zur Implikation von links nach rechts. Expandiert man die Aussage
durch Einsetzen der Definitionen, ist
\[(\exists k\colon b=ka)\implies
(\forall m\colon (\exists x\colon a=xm)\implies (\exists y\colon b=ym)).\]
zu zeigen. Wir haben einen Zeugen $k$ für $b=ka$ gegeben. 
Gemäß Prämisse liegt ein Zeuge $x$ für $a=xm$ vor, wonach $ka=kxm$
gilt, also $b=kxm$. Ergo ist $y:=kx$ ein Zeuge für
$\exists y\colon b=ym$.\,\qedsymbol
\end{Beweis}

\begin{Korollar}
Die Teilerrelation ist transitiv, das heißt,
$a\mid b$ und $b\mid c$ impliziert $a\mid c$.
\end{Korollar}
\begin{Beweis}
Unter Nutzung von Korollar \ref{divisor-divisor-subset}
nimmt die Aussage die Gestalt
\[T(a)\subseteq T(b)\land T(b)\subseteq T(c)\implies T(a)\subseteq T(c)\]
an. Diese Aussage ist nun aber offenkunding, da die Relation »ist
Teilmenge von« eine transitive ist.\,\qedsymbol
\end{Beweis}
