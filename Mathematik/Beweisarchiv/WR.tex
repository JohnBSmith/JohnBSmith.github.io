
\chapter{Wahrscheinlichkeitsrechnung}

\section{Diskrete Wahrscheinlichkeitsräume}

\begin{Definition}[Diskreter Wahrscheinlichkeitsraum]%
\label{def:discrete-prob-space}\mbox{}\\*
Sei $\Omega$ eine höchstens abzählbare Menge. Das
Paar $(\Omega,P)$ nennt man diskreten Wahrscheinlichkeitsraum,
wenn
\[P\colon 2^\Omega\to [0,1],\quad P(A):=\sum_{\omega\in A} P(\{\omega\})\]
die Eigenschaft $P(\Omega) = 1$ besitzt.
\end{Definition}
Bemerkung: Man schreibt auch $P(\omega):=P(\{\omega\})$.

\begin{Definition}[Reelle Zufallsgröße]%
\index{Zufallsgröße}\mbox{}\\*
Sei $(\Omega,P)$ ein diskreter Wahrscheinlichkeitsraum.
Eine Funktion $X\colon\Omega\to\R$ nennt man Zufallsgröße.
Die Verteilung von $X$ ist definiert gemäß $P_X(A):=P(X^{-1}(A))$.
\end{Definition}

\begin{Definition}[Bedingte Wahrscheinlichkeit]%
\index{bedingte Wahrscheinlichkeit}\mbox{}\\*
Seien $A,B\subseteq\Omega$ und sei $B\ne\emptyset$. Dann nennt man
\[P(A\mid B) := \frac{P(A\cap B)}{P(B)}\]
die bedingte Wahrscheinlichkeit von $A$, gegeben $B$.
\end{Definition}

\begin{Lemma}\label{indicator-disjoint}
Seien $A,B$ disjunkt. Seien die $A_i$ disjunkt. Dann gilt
\begin{gather*}
1_{A\cup B} = 1_A + 1_B,\\
1\{\bigcup_{i\in I} A_i\} = \sum_{i\in I} 1\{A_i\}.
\end{gather*}
\end{Lemma}
\begin{Beweis}
Es gilt
\begin{align*}
1_{A\cup B}(\omega) &= [\omega\in A\cup B]
= [\omega\in A\lor\omega\in B]
\stackrel{\text{(P)}}= [\omega\in A\oplus\omega\in B]\\
&= [\omega\in A] + [\omega\in B] = 1_A(\omega) + 1_B(\omega).
\end{align*}
Die allgemeine Rechnung ist
\[1\{\bigcup_{i\in I} A_i\}(\omega) = [\omega\in\bigcup_{i\in I} A_i]
= [\exists i\in I\colon \omega\in A_i]
\stackrel{\text{(P)}}= \sum_{i\in I} [\omega\in A_i]
= \sum_{i\in I} 1\{A_i\}(\omega).\]
Gleichung (P) gilt hierbei laut Prämisse.\,\qedsymbol
\end{Beweis}

\newpage
\begin{Korollar}[Additivität des Wahrscheinlichkeitsmaßes]%
\label{prob-additivity}\mbox{}\\*
Seien die $A_i$ paarweise disjunkte Mengen. Dann gilt
\[P(\bigcup_{i\in I} A_i) = \sum_{i\in I} P(A_i).\]
Für zwei disjunkte Menge $A,B$ gilt speziell
\[P(A\cup B) = P(A) + P(B).\]
\end{Korollar}

\begin{Beweis}[Beweis 1]
Weil $A\cap B=\emptyset$ gilt, ist $[\omega\in A\cup B]
= [\omega\in A]+[\omega\in B]$. Deshalb gilt
\begin{gather*}
P(A\cup B) = \sum_{\omega\in A\cup B} P(\{\omega\})
= \sum_{\omega\in\Omega} [\omega\in A\cup B] P(\{\omega\})\\
= \sum_{\omega\in\Omega} ([\omega\in A]+[\omega\in B]) P(\{\omega\})
= \sum_{\omega\in\Omega} [\omega\in A]P(\{\omega\})
+ \sum_{\omega\in\Omega} [\omega\in B]P(\{\omega\})\\
= \sum_{\omega\in A} P(\{\omega\}) + \sum_{\omega\in B} P(\{\omega\})
= P(A)+B(B).
\end{gather*}
Nun allgemein. Weil die $A_i$ disjunkt sind, gilt
$[\omega\in\bigcup_{i\in I} A_i] = \sum_{i\in I} [\omega\in A_i]$. Deshalb gilt
\begin{gather*}
P(\bigcup_{i\in I} A_i)
= \sum_{\omega\in\Omega}[\omega\in\bigcup_{i\in I} A_i]P(\{\omega\})
= \sum_{\omega\in\Omega}\sum_{i\in I}[\omega\in A_i]P(\{\omega\})\\
= \sum_{i\in I}\sum_{\omega\in\Omega}[\omega\in A_i]P(\{\omega\})
= \sum_{i\in I}\sum_{\omega\in A_i}P(\{\omega\})
= \sum_{i\in I}P(A_i).\,\qedsymbol
\end{gather*}
\end{Beweis}
\begin{Beweis}[Beweis 2]
Laut Korollar \ref{prob-as-expected-value},
Lemma \ref{indicator-disjoint}
und Korollar \ref{expected-value-op-linear} gilt
\[P(A\cup B) = E(1_{A\cup B}) = E(1_A+1_B) = E(1_A) + E(1_B)
= P(A) + P(B).\]
Die allgemeine Rechnung ist
\[P(\bigcup_{i\in I} A_i) = E(1\{\bigcup_{i\in I} A_i\})
= E(\sum_{i\in I} 1\{A_i\})
= \sum_{i\in I} E(1\{A_i\}) = \sum_{i\in I} P(A_i).\]
\end{Beweis}

\begin{Korollar}
Sei $(\Omega,P)$ ein diskreter Wahrscheinlichkeitsraum in Form von Def.
\ref{def:discrete-prob-space}. Der Raum ist in Form des Tripels
$(\Omega,2^\Omega,P)$ ein Wahrscheinlichkeitsraum, denn das Maß
$P$ erfüllt die drei kolmogorowschen Axiome.
\end{Korollar}
\begin{Beweis}
Die ersten beiden Axiome, $(\forall A\colon P(A)\ge 0)$ und
$P(\Omega)=1$, gelten per Definition. Das dritte Axiom, die Additivität,
gilt laut Korollar \ref{prob-additivity}. Dass es sich bei $2^\Omega$
um eine sigma-Algebra handelt, ist kaum einer ausdrücklichen
Erwähnung wert.\,\qedsymbol
\end{Beweis}

\begin{Satz}[Gesetz der totalen Wahrscheinlichkeit]%
\index{Gesetz der totalen Wahrsch.}
Sei $Z$ eine Zerlegung der Ergebnismenge
$\Omega$ in paarweise disjunkte nichtleere Mengen $B\in Z$. Dann gilt
\[P(A) = \sum_{B\in Z} P(A\mid B)P(B).\]
\end{Satz}

\begin{Beweis}
Es gilt
\begin{align*}
P(A) &= P(A\cap\Omega) = P(A\cap\bigcup_{B\in Z} B)
= P(\bigcup_{B\in Z} (A\cap B))\\
&= \sum_{B\in Z} P(A\cap B)
= \sum_{B\in Z} P(A\mid B)P(B).\,\qedsymbol
\end{align*}
\end{Beweis}

\begin{Korollar}[Gesetz der totalen Wahrscheinlichkeit für Zufallsgrößen]\mbox{}\\*
Seien $X,Y\colon\Omega\to\Omega'$ Zufallsgrößen. Dann gilt%
\[P(Y\in A) = \sum_{x\in X(\Omega)} P(Y\in A\mid X=x)P(X=x),\]
speziell
\[P(Y=y) = \sum_{x\in X(\Omega)} P(Y=y\mid X=x)P(X=x).\]
\end{Korollar}

\begin{Beweis}
Sei $Z=X(\Omega)$. Zunächst ist
\[\Omega = X^{-1}(Z) = X^{-1}(\bigcup_{x\in Z} \{x\})
= \bigcup_{x\in Z} X^{-1}(x)\]
gemäß Satz \ref{preimg-dl} (preimg-dl) und Korollar
\ref{disjoint-preimg} eine Vereinigung nichtleerer
paarweise disjunkter Mengen. Laut dem Gesetz der totalen
Wahrscheinlichkeit gilt daher%
\begin{align*}
P(Y\in A) &= P(Y^{-1}(A)) = P(Y^{-1}(A)\cap\Omega)
= \sum_{x\in Z} P(Y^{-1}(A)\mid X^{-1}(x))P(X^{-1}(x))\\
&= \sum_{x\in Z} P(Y\in A\mid X=x)P(X=x).\,\qedsymbol
\end{align*}
\end{Beweis}

\begin{Definition}[Erwartungswert]%
\index{Erwartungswert}\label{def:expected-value}\mbox{}\\*
Sei $(\omega_k)$ eine beliebige Abzählung von $\Omega$.
Ist die Reihe $\sum_{k=0}^{|\Omega|} X(\omega_k)P(\{\omega_k\})$
absolut konvergent, dann nennt man%
\[E(X) := \sum_{\omega\in\Omega} X(\omega)P(\{\omega\})\]
den Erwartungswert von $X$.
\end{Definition}

\begin{Satz} Es gilt
\[E(X) = \sum_{x\in X(\Omega)} xP(X^{-1}(x))
= \sum_{x\in X(\Omega)} xP(X=x).\]
\end{Satz}
\strong{Beweis.} Zunächst gilt
\begin{gather*}
\sum_{\substack{\omega\in\Omega\\ X(\omega)=x}}P(\omega)
= P(\bigcup_{\substack{\omega\in\Omega\\ X(\omega)=x}} \{\omega\})
= P(\{\omega\in\Omega\mid X(\omega)=x\})
= P(X^{-1}(x)).
\end{gather*}
Da die Reihe zu $E(X)$ nach Def. \ref{def:expected-value}
absolut konvergent ist, darf sie beliebig umgeordnet werden und
man bekommt
\begin{gather*}
E(X) = \sum_{\omega\in\Omega}X(\omega)P(\omega)
= \sum_{x\in X(\Omega)}\sum_{\substack{\omega\in\Omega\\ X(\omega)=x}} xP(\omega)
= \sum_{x\in X(\Omega)} x\sum_{\substack{\omega\in\Omega\\ X(\omega)=x}}P(\omega)\\
= \sum_{x\in X(\Omega)} xP(X^{-1}(x)).\;\qedsymbol
\end{gather*}

\newpage
\begin{Korollar}\label{expected-value-op-linear}
Der Erwartungswertoperator ist ein lineares Funktional,
das heißt, es gilt $E(aX)=aE(X)$ und $E(X+Y)=E(X)+E(Y)$. 
\end{Korollar}
\strong{Beweis.} Aufgrund der Konvergenz der Reihen gilt
\[E(aX) = \sum_{\omega\in\Omega}aX(\omega)P(\omega)
= a\sum_{\omega\in\Omega}X(\omega)P(\omega) = aE(X)\]
und
\begin{align*}
E(X+Y) &= \sum_{\omega\in\Omega} (X(\omega)+Y(\omega))P(\omega)
= \sum_{\omega\in\Omega} (X(\omega)P(\omega)+Y(\omega)P(\omega))\\
&= \sum_{\omega\in\Omega} X(\omega)P(\omega)
+ \sum_{\omega\in\Omega} Y(\omega)P(\omega) = E(X)+E(Y).\;\qedsymbol
\end{align*}

\begin{Korollar} Ist $X\le Y$, dann ist auch $E(X)\le E(Y)$.
\end{Korollar}
\strong{Beweis.} Gemäß $P(\omega)\ge 0$ ist
\begin{gather*}
X\le Y\iff X(\omega)\le Y(\omega)\iff 0\le Y(\omega)-X(\omega)
\iff 0\le (Y(\omega)-X(\omega))P(\omega).
\end{gather*}
Somit hat man
\begin{gather*}
X\le Y\implies 0\le E(Y-X) = \sum_{\omega\in\Omega} (Y(\omega)-X(\omega))P(\omega),
\end{gather*}
und gemäß Linearität daher
\begin{gather*}
X\le Y\implies 0\le E(Y-X) = E(Y)-E(X) \iff E(X)\le E(Y).\;\qedsymbol
\end{gather*}

\begin{Definition}[Unabhängige Ereignisse]%
\index{unabhängige Ereignisse}\mbox{}\\*
Zwei Ereignisse $A,B$ heißen unabhängig, falls $P(A\cap B)=P(A)P(B)$.
\end{Definition}

\begin{Definition}[Unabhängige Zufallsgrößen]%
\index{unabhängige Zufallsgrößen}\mbox{}\\*
Zwei Zufallsgrößen $X,Y\colon\Omega\to\R$ heißen unabhängig, wenn
die Ereignisse $\{X\in A\}$ und $\{X\in B\}$
für alle Mengen $A,B\subseteq\R$ unabhängig sind.
\end{Definition}

\begin{Satz}
Zwei Zufallsgrößen $X,Y\colon\Omega\to\R$ sind genau dann unabhängig,
wenn für alle $x\in X(\Omega)$ und $y\in Y(\Omega)$ gilt:
\[P(X=x,Y=y)= P(X=x)P(Y=y).\]
\end{Satz}
\strong{Beweis.} Sind $X,Y$ unabhängig, dann ist
\begin{align*}
P(X=x,Y=y) &= P(\{X\in\{x\}\}\cap\{Y\in\{y\}\})
= P(\{X\in\{x\}\})P(\{Y\in\{y\}\})\\
&= P(X=x)P(Y=y).
\end{align*}
Umgekehrt gelte nun $P(X=x,Y=y)=P(X=x)P(Y=y)$, dann ist
\begin{gather*}
P(\{X\in A\}\cap\{Y\in B\})
= P(\bigcup_{x\in A}\{X=x\}\cap\bigcup_{y\in B}\{Y=y\})\\
= P(\bigcup_{x\in A}\bigcup_{y\in B}(\{X=x\}\cap\{Y=y\}))
= \sum_{x\in A}\sum_{y\in B}P(\{X=x\}\cap\{Y=y\})\\
= \sum_{x\in A}\sum_{y\in B}P(X=x)P(Y=y)
= \sum_{x\in A}P(X=x)\sum_{y\in B}P(Y=y)\\
= P(\bigcup_{x\in A}\{X=x\})P(\bigcup_{y\in B}\{Y=y\})
= P(X\in A)P(Y\in B).\;\qedsymbol
\end{gather*}

\begin{Definition}[Bedingter Erwartungswert]%
\index{bedingter Erwartungswert}\index{Erwartungswert!bedingter}%
\label{def:cond-expected-value}\mbox{}\\*
\[E(X\mid A) = \frac{E(1_A X)}{P(A)} = \frac{1}{P(A)}\sum_{\omega\in A} X(\omega)P(\{\omega\}).\]
\end{Definition}

\begin{Satz} Es gilt
\[E(X\mid A) = \frac{1}{P(A)}\sum_x xP(\{X=x\}\cap A)
= \sum_x xP(X=x\mid A),\]
wobei sich die Summe über alle $x\in X(\Omega)$ erstreckt.
\end{Satz}
\strong{Beweis.} Man kann rechnen
\begin{align*}
E(1_A X) &= \sum_{\omega\in\Omega} 1_A(\omega) X(\omega) P(\{\omega\})
= \sum_x\sum_{\omega\in X^{-1}(x)} 1_A(\omega) X(\omega) P(\{\omega\})\\
&= \sum_x x\sum_{\omega\in X^{-1}(x)} 1_A(\omega) P(\{\omega\})
= \sum_x x\sum_{\omega\in X^{-1}(x)\cap A} P(\{\omega\})\\
&= \sum_x x P(X^{-1}(x)\cap A),
\end{align*}
wobei $X^{-1}(x) = \{X=x\}$.\;\qedsymbol

\begin{Korollar}\label{prob-as-expected-value}
Es gilt $P(A) = E(1_A)$, wobei $1_A$ die Indikatorfunktion ist.
\end{Korollar}
\strong{Beweis.} Gemäß Definition des Erwartungswertes ist
\[E(1_A) = \sum_{\omega\in\Omega} 1_A(\omega)P(\{\omega\})
= \sum_{\omega\in A}P(\{\omega\}) = P(A).\;\qedsymbol\]

\begin{Korollar}
Es gilt $P(A\mid B) = E(1_A\mid B)$, wobei $1_A$ die Indikatorfunktion ist.
\end{Korollar}
\strong{Beweis.} Gemäß Definition \ref{def:cond-expected-value}
und Korollar \ref{prob-as-expected-value} ist
\[E(1_A\mid B) = \frac{E(1_A 1_B)}{P(B)} = \frac{E(1_{A\cap B})}{P(B)}
= \frac{P(A\cap B)}{P(B)} = P(A\mid B).\qedsymbol\]

\section{Allgemeine Wahrscheinlichkeitsräume}

\begin{Satz}\label{rv-transform-pdf}
Sei $g\colon\R\to\R$ eine streng monotone Funktion. Seien
$X,Y$ Zufallsgrößen mit Dichten $f_X,f_Y$. Ist $Y=g(X)$,
dann gilt
\[f_Y(y) = \frac{f_X(g^{-1}(y))}{|g'(g^{-1}(y))|}.\]
\end{Satz}
\strong{Beweis.} Sei $g$ streng monoton steigend. Dann kann man rechnen
\[F_Y(y) = P(Y\le y) = P(g(X)\le y) = P(X\le g^{-1}(y)) = F_X(g^{-1}(y)).\]
Gemäß der Kettenregel findet man
\[f_Y(y) = \frac{\mathrm d}{\mathrm dy}F_Y(y)
= f_X(g^{-1}(y))\frac{\mathrm d}{\mathrm dy}g^{-1}(y) = \frac{f_X(g^{-1}(y))}{g'(g^{-1}(y))}.\]
Sei $g$ nun streng monoton fallend. Dann kann man rechnen
\[F_Y(y) = P(g(X)\le y) = P(X\ge g^{-1}(y)) = 1 - P(X < g^{-1}(y))
= 1 - F_X(g^{-1}(y)).\]
Entsprechend findet man
\[f_Y(y) = -\frac{f_X(g^{-1}(y))}{g'(g^{-1}(y))}.\]
Nun ist $g'$ in beiden Fällen frei von Nullstellen. Demnach ist
$\sgn(g'(x))$ konstant für alle $x$ und wir haben allgemein
\[f_Y(y) = \sgn(g'(x))\frac{f_X(x)}{g'(x)} = \frac{f_X(x)}{|g'(x)|}\]
mit $x=g^{-1}(y)$.\;\qedsymbol

\begin{Satz}[»LOTUS: Law of the unconscious statistican«]\mbox{}\\*
Ist $g\colon\R\to\R$ streng monoton, dann muss gelten
\[E(g(X)) = \int_{-\infty}^\infty g(x)f_X(x)\,\mathrm dx.\]
\end{Satz}
\strong{Beweis.} Sei $Y=g(X)$. Mit Satz \ref{rv-transform-pdf}
und Substitution $y=g(x)$ kann man rechnen
\begin{align*}
E(g(X)) &= E(Y) = \int_{-\infty}^\infty yf_Y(y)\,\mathrm dy
= \int_{-\infty}^\infty y\frac{f_X(g^{-1}(y))}{|g'(g^{-1}(y))|}\,\mathrm dy\\
&= \int_{g^{-1}(-\infty)}^{g^{-1}(\infty)} g(x)
\frac{f_X(x)}{|g'(x)|}g'(x)\,\mathrm dx\\
&= \sgn(g')\int_{-\sgn(g')\infty}^{\sgn(g')\infty}g(x) f_X(x)\mathrm dx
= \int_{-\infty}^\infty g(x) f_X(x)\mathrm dx.\;\qedsymbol
\end{align*}

\section{Stochastische Prozesse}%
\index{stochastischer Prozess}

\subsection{Markow-Prozesse mit endlichem Zustandsraum}

\begin{Definition}[Markow-Prozess]\index{Markow-Prozess}
Sei $X(t)$ ein stochastischer Prozess mit $t\ge 0$ und $X(t)\in S$,
wobei $S$ ein endlicher Zustandsraum ist. Der Prozess heißt
Markow"=Prozess, wenn die Markow"=Eigenschaft
\[P(X(s+t)=j\mid \{X(s)=i\}\cap\bigcap_{0\le u<s}\{X(u)=x_u\}) = P(X(s+t)=j\mid X(s)=i)\]
erfüllt ist. Das heißt, für die bedingte Wahrscheinlichkeit, dass
$X$ in der Zukunft den Zustand $j$ einnimmt, spielt nur der aktuelle
Zustand $i$ eine Rolle, nicht aber Zustände der Vergangenheit.
\end{Definition}

\noindent
\strong{Bemerkung.} Eine geläufige Notation
ist $P(A\mid B,C):=P(A\mid B\cap C).$ Es gilt
\[P(A\mid B\cap C) = P_B(A\mid C) = P_C(A\mid B),\]
wobei $P_B(M):=P(M\mid B)$ und $P_C(M):=P(M\mid C)$.

\begin{Definition}[Homogener Markow-Prozess]%
\index{homogener Markow-Prozess}
Ein Markow-Prozess $X(t)$ heißt homogen, wenn die Gleichung
\[p_{ij}(t) := P(X(s+t)=j\mid X(s)=i) = P(X(t)=j\mid X(0)=i)\]
für jedes $s$ erfüllt ist. Man nennt $P=(P_{ij})$ die Übergangsmatrix.
\end{Definition}

\begin{Satz}[Kolmogorow-Chapman-Gleichung]%
\index{Kolmogorow-Chapman-Gleichung}\mbox{}\\*
Für die Übergangsmatrix eines homogenen Markow"=Prozesses gilt
\[P(s+t) = P(s)P(t).\]
\end{Satz}

\begin{Beweis}
Wir schreiben kurz $P_i(X(t)=j):=P(X(t)=j\mid X(0)=i)$. Gemäß dem
Gesetz der totalen Wahrscheinlichkeit gilt
\[p_{ij}(s+t) = P_i(X(s+t)=j)
= \sum_{k\in S} P_i(X(s)=k) P_i(X(s+t)=j\mid X(s)=k).\]
Aufgrund der Markow-Eigenschaft und der Homogenität gilt nun
\begin{gather*}
P_i(X(s+t)=j\mid X(s)=k) = P(X(s+t)=j\mid X(s)=k, X(0)=i)\\
= P(X(s+t)=j\mid X(s)=k) = P(X(t)=j\mid X(0)=k) = P_k(X(t)=j) = p_{kj}(t).
\end{gather*}
Man erhält die Matrizenmultiplikation
\[p_{ij}(s+t) = \sum_{k\in S} p_{ik}(s) p_{kj}(t),\;\,
\text{kurz}\;\; P(s+t)=P(s)P(t).\,\qedsymbol\]
\end{Beweis}

