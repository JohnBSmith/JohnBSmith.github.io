
\chapter{Kombinatorik}

\section{Endliche Mengen}

\subsection{Indiktatorfunktion}

\begin{Definition}[Iverson-Klammer]\mbox{}\\
Für eine Aussage $A$ der klassischen Aussagenlogik definiert man
\[[A] := \begin{cases}
1 &\text{wenn}\;A,\\
0 &\text{sonst}.
\end{cases}\]
\end{Definition}

\begin{Korollar}\label{iverson-basic-rules}
Es gilt
\begin{gather*}
[A\land B] = [A][B],\\
[A\lor B] = [A]+[B]-[A][B],\\
[\neg A] = 1-[A],\\
[A\to B] = 1-[A](1-[B]).
\end{gather*}
\end{Korollar}
\begin{Beweis} Trivial mittels Wertetabelle.\,\qedsymbol
\end{Beweis}

\begin{Korollar}\label{indicator-set-op}
Für die Indikatorfunktion $1_M(x):=[x\in M]$ gilt
\begin{gather*}
1_{A\cap B} = 1_A 1_B,\\
1_{A\cup B} = 1_A + 1_B - 1_{A\cap B}.
\end{gather*}
\end{Korollar}
\begin{Beweis}
Gemäß Korollar \ref{iverson-basic-rules} darf gelten die
Rechnungen
\begin{align*}
1_{A\cap B}(x) = [x\in A\cap B]
= [x\in A\land x\in B] = [x\in A][x\in B] = 1_A(x)1_B(x)
\end{align*}
und
\begin{align*}
1_{A\cup B}(x) &= [x\in A\cup B] = [x\in A\lor x\in B]
= [x\in A] + [x\in B] - [x\in A][x\in B]\\
&= 1_A(x) + 1_B(x) - 1_{A\cap B}(x).\,\qedsymbol
\end{align*}
\end{Beweis}

\begin{Satz}
Für endliche Mengen $A,B$ gilt $|A\cup B| = |A|+|B|-|A\cap B|$.
\end{Satz}
\begin{Beweis}
Gemäß Korollar \ref{indicator-set-op} darf man rechnen
\begin{align*}
|A\cup B| &= \sum_{x\in G} 1_{A\cup B}(x)
= \sum_{x\in G} (1_A(x) + 1_B(x) - 1_{A\cap B}(x))\\
&= \sum_{x\in G} 1_A(x) + \sum_{x\in G} 1_B(x) - \sum_{x\in G} 1_{A\cap B}(x)
= |A| + |B| - |A\cap B|.\,\qedsymbol
\end{align*}
\end{Beweis}