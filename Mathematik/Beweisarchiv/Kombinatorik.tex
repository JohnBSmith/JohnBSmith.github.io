
\chapter{Kombinatorik}

\section{Endliche Mengen}

\subsection{Indikatorfunktion}

\begin{Definition}[Iverson-Klammer]\newlinefirst
Für eine Aussage $A$ der klassischen Aussagenlogik definiert man
\[[A] := \begin{cases}
1 &\text{wenn}\;A,\\
0 &\text{sonst}.
\end{cases}\]
\end{Definition}

\begin{Satz}\label{iverson-basic-rules}
Es gilt
\begin{gather*}
[A\land B] = [A][B],\\
[A\lor B] = [A]+[B]-[A][B],\\
[\neg A] = 1-[A],\\
[A\to B] = 1-[A](1-[B]).
\end{gather*}
\end{Satz}
\begin{Beweis} Trivial mittels Wertetabelle.\,\qedsymbol
\end{Beweis}

\begin{Satz}\label{indicator-set-op}
Für die Indikatorfunktion $1_M(x):=[x\in M]$ gilt
\begin{gather*}
1_{A\cap B} = 1_A 1_B,\\
1_{A\cup B} = 1_A + 1_B - 1_{A\cap B}.
\end{gather*}
\end{Satz}
\begin{Beweis}
Gemäß Satz \ref{iverson-basic-rules} gelten die
Rechnungen
\begin{align*}
1_{A\cap B}(x) = [x\in A\cap B]
= [x\in A\land x\in B] = [x\in A][x\in B] = 1_A(x)1_B(x)
\end{align*}
und
\begin{align*}
1_{A\cup B}(x) &= [x\in A\cup B] = [x\in A\lor x\in B]
= [x\in A] + [x\in B] - [x\in A][x\in B]\\
&= 1_A(x) + 1_B(x) - 1_{A\cap B}(x).\,\qedsymbol
\end{align*}
\end{Beweis}

\begin{Satz}
Für endliche Mengen $A,B$ gilt $|A\cup B| = |A|+|B|-|A\cap B|$.
\end{Satz}
\begin{Beweis}
Gemäß Satz \ref{indicator-set-op} darf man rechnen
\begin{align*}
|A\cup B| &= \sum_{x\in G} 1_{A\cup B}(x)
= \sum_{x\in G} (1_A(x) + 1_B(x) - 1_{A\cap B}(x))\\
&= \sum_{x\in G} 1_A(x) + \sum_{x\in G} 1_B(x) - \sum_{x\in G} 1_{A\cap B}(x)
= |A| + |B| - |A\cap B|.\,\qedsymbol
\end{align*}
\end{Beweis}

\begin{Satz}
Für endliche Mengen $A,B$ mit $A\subseteq B$ gilt $|A|\le |B|$.
\end{Satz}
\begin{Beweis}
Mithilfe der Indiktorfunktion findet sich
\begin{gather*}
A\subseteq B \iff (\forall x\colon 1_A(x)\le 1_B(x))
\iff (\forall x\colon 0\le 1_B(x)-1_A(x))\\
\implies 0\le \sum_{x\in B}(1_B(x)-1_A(x))
= \sum_{x\in B} 1_B(x) - \sum_{x\in B} 1_A(x) = |B| - |A|\\
\implies |A|\le |B|.\,\qedsymbol
\end{gather*}
\end{Beweis}

\newpage
\subsection{Endliche Abbildungen}

\begin{Satz}[Anzahl der Abbildungen]\newlinefirst
Seien $X,Y$ endliche Mengen mit $|X| = k$ und $|Y|=n$. Die Menge
der Abbildungen $X\to Y$ enthält $n^k$ Elemente.
\end{Satz}
\begin{Beweis}
Induktion über $k$. Im Anfang $k=0$ ist $X=\emptyset$. Es gibt genau
eine Abbildung $\emptyset\to Y$, nämlich die leere Abbildung.
Gleichermaßen ist $n^0=1$.

Zum Induktionsschritt. Induktionsvoraussetzung sei die Gültigkeit
für $k-1$. Es  sei $|X|=k$ und $|Y|=n$. Gesucht ist die Anzahl
der Möglichkeiten zur Festlegung der Abbildung $f\colon X\to Y$.
Sei $x\in X$ fest. Für die Festlegung $f(x)=y$ bestehen nun genau $n$
Möglichkeiten, nämlich so viele, wie es Elemente $y\in Y$ gibt.
Für die Festlegung der übrigen Werte betrachtet man $f$ als Abbildung%
\[f\colon X\setminus\{x\}\to Y,\]
von denen es laut Voraussetzung $n^{k-1}$ gibt. Wir haben also
$n$ mal $n^{k-1}$ Möglichkeiten, das sind $n^k$.\,\qedsymbol
\end{Beweis}

\begin{Satz}[Anzahl der Bijektionen]\newlinefirst
Seien $X,Y$ endliche Mengen, wobei $|X|=|Y|=n$ gelte.
Die Menge der Bijektionen $X\to Y$ enthält $n!$ Elemente.
\end{Satz}
\begin{Beweis}
Induktion über $n$. Im Anfang $n=0$ ist $X=\emptyset$
und $Y=\emptyset$. Es existiert genau eine Bijektion
$\emptyset\to\emptyset$, nämlich die leere Abbildung.
Bei der Fakultät gilt ebenfalls $0! = 1$ laut
Def. \ref{def:factorial}.

Zum Induktionsschritt. Induktionsvoraussetzung sei die Gültigkeit für
$n-1$. Es sei $|X|=n$. Gesucht ist die Anzahl der Möglichkeiten zur
Festlegung der Bijektion $f\colon X\to Y$. Sei $x\in X$ fest. Für
die Festlegung $f(x)=y$ bestehen genau $n$ Möglichkeiten, nämlich
so viele, wie es Elemente $y\in Y$ gibt. Bei der Festlegung der übrigen
Werte entfällt $y$ aufgrund der Injektivität von $f$. Für die Festlegung
betrachtet man $f$ daher als Bijektion%
\[f\colon X\setminus\{x\}\to Y\setminus\{y\},\]
von denen es laut Voraussetzung $(n-1)!$ gibt. Wir haben also
$n$ mal $(n-1)!$ Möglichkeiten, was gemäß Def. \ref{def:factorial}
gleich $n!$ ist.\,\qedsymbol
\end{Beweis}

\begin{Satz}[Anzahl der Injektionen]\newlinefirst
Seien $X,Y$ endliche Mengen, wobei $|X|=k$ und $|Y|=n$ gelte.
Die Menge der Injektionen $X\to Y$ enthält $n^{\underline k}$
Elemente.
\end{Satz}
\begin{Beweis}
Induktion über $k$. Im Anfang $k=0$ ist $X=\emptyset$. Es gibt
genau eine Injektion $\emptyset\to Y$, nämlich die leere Abbildung.
Gleichermaßen gilt $n^{\underline 0} = 1$.

Zum Schritt. Voraussetzung sei die Gültigkeit
für $k-1$. Es sei $|X|=k$ und $|Y|=n$. Gesucht ist die Anzahl
der Möglichkeiten zur Festlegung der Injektion $f\colon X\to Y$.
Sei $x\in X$ fest. Für die Festlegung $f(x)=y$ bestehen genau
$n$ Möglichkeiten, nämlich so viele, wie es Elemente $y\in Y$ gibt.
Bei der Festlegung der übrigen entfällt $y$ aufgrund der Injektivität
von $f$. Für die Festlegung betrachtet man $f$ daher als Injektion%
\[f\colon X\setminus\{x\}\to Y\setminus\{y\},\]
von denen es laut Voraussetzung $(n-1)^{\underline{k-1}}$ gibt. Es sind
also $n$ mal $(n-1)^{\underline{k-1}}$ Möglichkeiten, was gleich
$n^{\underline k}$ ist.\,\qedsymbol
\end{Beweis}

\newpage
\begin{Satz}\label{bijection-from-k-subsets-to-orbits}
Seien $X,Y$ endliche Mengen und sei $|X|=k$. Sei $C_k(Y)$
die Menge der $k$-elementigen Teilmengen von $Y$. Für zwei
Injektionen $X\to Y$ sei ferner die Äquivalenzrelation
\[f\sim g \defiff \exists\pi\in S_k\colon f=g\circ \pi\]
definiert, wobei mit den $\pi\in S_k$ Permutationen gemeint sind.
Zwischen der Quotientenmenge $\operatorname{Inj}(X, Y)/S_k$
und $C_k(Y)$ besteht eine kanonische Bijektion.
\end{Satz}
\begin{Beweis}
Wir definieren diese Bijektion als
\[\varphi\colon \operatorname{Inj}(X, Y)/S_k\to C_k(Y),
\quad \varphi([f]) := f(X),\]
wobei $[f]=f\circ S_k$ die Äquivalenzklasse des Repräsentanten $f$
bezeichne. Sie ist wohldefiniert, denn für $f\sim g$ gilt
\[f(X) = (g\circ\pi)(X) = g(\pi(X)) = g(X).\]
Für die Injektivität von $\varphi$ ist zu zeigen, dass $f(X) = g(X)$
die Aussage $[f]=[g]$ impliziert, also die Existenz einer Permutation
$\pi$ mit $f=g\circ\pi$. Weil $g$ injektiv ist, existiert eine
Linksinverse $g^{-1}$, so dass wir $\pi:=g^{-1}\circ f$ wählen
können. Es verbleibt somit die Gleichung $f=g\circ g^{-1}\circ f$ zu
zeigen. Zwar ist $g^{-1}$ im Allgemeinen keine Rechtsinverse, ihre
Einschränkung auf $g(X)$ aber schon. Wegen $f(X)=g(X)$ hebt sich
$g\circ g^{-1}$ daher wie gewünscht auf $f(X)$ weg.

Zur Surjektivität von $\varphi$. Hier ist zu zeigen, dass es zu jeder
Menge $B\in C_k(Y)$ eine Injektion $f$ mit $f(X)=B$ gibt. Betrachten
wir sie als Bijektion $f\colon X\to B$. Eine solche besteht,
weil $X$ und $B$ gleichmächtig sind.\,\qedsymbol
\end{Beweis}

\begin{Satz}[Anzahl der Kombinationen]\newlinefirst
Sei $Y$ eine $|Y|=n$ Elemente enthaltende endliche Menge und $C_k(Y)$
die Menge der $k$-elementigen Teilmengen von $Y$.
Es gilt $|C_k(Y)| = \binom{n}{k}$.
\end{Satz}
\begin{Beweis}[Beweis 1]
Sei $X$ eine Menge mit $|X|=k$. Es gilt
\[|C_k(Y)|
\stackrel{\text{(1)}}= |\operatorname{Inj}(X,Y)/S_k|
\stackrel{\text{(2)}}= \frac{|\operatorname{Inj}(X,Y)|}{|S_k|}
= \frac{n^{\underline k}}{k!} = \binom{n}{k}.\]
Gleichung (1) gilt hierbei laut Satz
\ref{bijection-from-k-subsets-to-orbits}.
Die Einsicht von (2) erhält man mit der folgenden Überlegung.
Für jede Gruppe $G$ gilt die Bahnformel $|G| = |f\circ G|\cdot |G_f|$.
Ist nun die Fixgruppe $G_f$ trivial, ist $|G_f|=1$ und
infolge $|f\circ G|=|G|$. Dies ist bei der symmetrischen Gruppe
$G=S_k$ der Fall. Aus diesem Grund enthält jede Bahn $f\circ S_k$
gleich viele Elemente, $|S_k|$ an der Zahl. Weil die Bahnen außerdem
paarweise disjunkt sind, erhält man die Faktorisierung
\[|\operatorname{Inj}(X,Y)| = |S_k|\cdot |\operatorname{Inj}(X,Y)/S_k|.\,\qedsymbol\]
\end{Beweis}
\begin{Beweis}[Beweis 2]
Induktion über $(n, k)$. Im Anfang ist $k=0$ oder $k=n$. Der abstruse
Fall $k=0$ sucht nach Teilmengen ohne Elemente. Es existiert genau eine
solche Menge, nämlich die leere Menge, womit $C_0(Y)=1$ ist. Der Fall
$k=n$ sucht nach Teilmengen, die so viele Elemente haben wie $Y$.
Dies kann nur $Y$ selbst sein, womit $C_n(Y)=1$ gilt.
Gleichermaßen gilt $\binom{n}{0}=1$ und $\binom{n}{n}=1$.

Induktionsvoraussetzung sei die Gültigkeit für $(n-1, k-1)$ und
$(n-1, k)$. Man nimmt nun ein Element $y$ aus $Y$ heraus,
womit darin $n-1$ verbleiben. Entscheidet man sich,
$y$ zur Teilmenge hinzuzufügen, verbleiben noch $k-1$ Elemente
auszuwählen. Entscheidet man sich dagegen, verbleibt die Teilmenge
unverändert, womit nach wie vor $k$ Elemente auszuwählen sind.
Die Anzahl der Möglichkeiten ist somit
\[|C_k(Y)| = |C_{k-1}(Y\setminus\{y\})| + |C_k(Y\setminus\{y\})|
\stackrel{\mathrm{IV}}=\binom{n-1}{k-1} + \binom{n-1}{k}
= \binom{n}{k}.\,\qedsymbol\]
\end{Beweis}

\begin{Satz}[Gitterweg-Interpretation]\newlinefirst
Ein Gitterweg\index{Gitterweg} auf dem Gitter $\Z\times\Z$ heißt
monoton, wenn von $(x,y)$ aus lediglich der Schritt nach $(x+1,y)$
oder der Schritt nach $(x,y+1)$ gewährt ist. Die Anzahl der monotonen
Gitterwege von $(0,0)$ zu $(x,y)$ beträgt
\[\frac{(x+y)!}{x!y!} = \binom{x+y}{x} = \binom{x+y}{y}.\]
\end{Satz}
\begin{Beweis}[Beweis 1]
Alle Gitterwege besitzen dieselbe Länge $x+y$. Die Knoten des jeweiligen
Wegs nummerieren wir der Reihe nach mit Ausnahme des letzten. Nun
ist von den $x+y$ Nummern eine Teilmenge von $y$ Nummern auszuwählen,
an denen ein Schritt nach oben stattfinden soll. Dafür gibt es
$\binom{x+y}{y}$ Möglichkeiten.\,\qedsymbol
\end{Beweis}

\begin{Beweis}[Beweis 2]
Es bezeichne $f(x,y)$ die Anzahl der Wege von $(0,0)$
zu $(x,y)$. Zum Erreichen eines Randpunktes besteht immer nur eine
einzige Möglichkeit, womit $f(x,0)=1$ und $f(0,y)=1$ gilt. Der nicht
auf dem Rand befindliche Punkt $(x,y)$ kann nun von $(x-1,y)$ oder von
$(x,y-1)$ aus erreicht werden, womit
\[f(x,y) = f(x-1,y) + f(x,y-1)\]
gelten muss. Man sieht nun, dass diese Rekursion ein gedrehtes
pascalsches Dreieck erzeugt. Wir setzen daher $f(x,y) = C(x+y,x)$ und
führen die Koordinatentransformation $x+y=n$ und $x=k$ aus. Die
Rekurrenz nimmt damit die Form
\begin{gather*}
C(x+y,x) = C(x-1+y,x-1) + C(x+y-1,x)\\
\iff C(n,k) = C(n-1,k-1) + C(n-1,k).
\end{gather*}
an. Die Randbedingungen führen zu $C(n,n)=1$ und $C(n,0)=1$. Durch diese
Rekurrenz ist eindeutig der Binomialkoeffizient $C(n,k)=\binom{n}{k}$
bestimmt, womit
\[f(x,y) = C(x+y,x) = \binom{x+y}{x}\]
gelten muss.\,\qedsymbol
\end{Beweis}

\begin{Satz}[Rekursionsformel der Potenzmengenabbildung]\newlinefirst
Für $x\notin M$ gilt $\mathcal P(M\cup\{x\}) = \mathcal P(M)\cup\{A\cup\{x\}\mid A\in\mathcal P(M)\}$.
\end{Satz}
\begin{Beweis}
Die Gleichung ist äquivalent zu
\[T\subseteq M\cup\{x\}\iff T\subseteq M\lor\exists A\subseteq M\colon T=A\cup\{x\}.\]
Nehmen wir die rechte Seite an. Im Fall $T\subseteq M$ gilt erst
recht $T\subseteq M\cup\{x\}$. Im anderen Fall liegt ein $A\subseteq M$
vor, womit $A\cup\{x\}\subseteq M\cup\{x\}$ gilt. Wegen $T=A\cup\{x\}$
gilt also ebenfalls $T\subseteq M\cup\{x\}$.

Nehmen wir die linke Seite an. Mit $T\subseteq M\cup\{x\}$ und
$x\notin M$ folgt per Satz \ref{subseteq-diff}
\[T\setminus\{x\}\subseteq (M\cup\{x\})\setminus\{x\} = M,
\;\text{also}\; T\setminus\{x\}\subseteq M.\]
Im Fall $x\notin T$ ist
$T=T\setminus\{x\}$, womit man $T\subseteq M$
erhält. Im Fall $x\in T$ wird $A:=T\setminus\{x\}$
als Zeuge der Existenzaussage gewählt.\,\qedsymbol
\end{Beweis}

\newpage
\section{Endliche Summen}

\subsection{Allgemeine Regeln}

\begin{Definition}[Summe]
Sei $(G,+,0)$ eine kommutative Gruppe und $a_k\in G$. Die Summe ist
rekursiv definiert als
\[\sum_{k=m}^{m-1} a_k := 0,\quad \sum_{k=m}^n a_k
:= a_n + \sum_{k=m}^{n-1} a_k.\]
\end{Definition}

\begin{Satz}\label{sum-add}
Es gilt
\[\sum_{k=m}^n (a_k + b_k) = \sum_{k=m}^n a_k + \sum_{k=m}^n b_k.\]
\end{Satz}
\begin{Beweis} Induktion über $n$. Im Anfang $n=m-1$ haben
beide Seiten der Gleichung den Wert null. Induktionsschritt:
\begin{align*}
\sum_{k=m}^n (a_k+b_k) &= a_n + b_n + \sum_{k=m}^{n-1} (a_k+b_k)
\stackrel{\mathrm{IV}}= a_n + b_n + \sum_{k=m}^{n-1} a_k + \sum_{k=m}^{n-1} b_k\\
&= \sum_{k=m}^n a_k + \sum_{k=m}^n b_k.\,\qedsymbol
\end{align*}
\end{Beweis}

\begin{Satz}\label{sum-scale}
Sei $R$ ein Ring und $c,a_k\in R$. Sei $c$ eine
Konstante. Es gilt
\[\sum_{k=m}^n ca_k = c\sum_{k=m}^n a_k.\]
\end{Satz}
\begin{Beweis} Induktion über $n$. Im Anfang $n=m-1$ haben beide
Seiten der Gleichung den Wert null. Induktionsschritt:
\[\sum_{k=m}^n ca_k = ca_n + \sum_{k=m}^{n-1} ca_k
\stackrel{\mathrm{IV}}= ca_n + c\sum_{k=m}^{n-1} a_k
= c(a_n + \sum_{k=m}^{n-1} a_k) = c\sum_{k=m}^n a_k.\,\qedsymbol\]
\end{Beweis}

\begin{Satz}[Aufteilung einer Summe]\label{sum-split}
Für $m\le p\le n$ gilt
\[\sum_{k=m}^n a_k = \sum_{k=m}^{p-1} a_k + \sum_{k=p}^n a_k.\]
\end{Satz}
\begin{Beweis} Induktion über $n$. Im Induktionsanfang ist $n=p$
und folglich:
\[\sum_{k=m}^p a_k = \sum_{k=m}^{p-1} a_k + p_k
= \sum_{k=m}^{p-1} a_k + \sum_{k=p}^p a_k.\]
Induktionsschritt:
\[\sum_{k=m}^n a_k = a_n + \sum_{k=m}^{n-1} a_k
\stackrel{\mathrm{IV}}= a_n + \sum_{k=m}^{p-1} a_k + \sum_{k=p}^{n-1} a_k
= \sum_{k=m}^{p-1} a_k + \sum_{k=p}^n a_k.\,\qedsymbol\]
\end{Beweis}

\begin{Satz}[Indexshift]\label{sum-indexshift}\newlinefirst
Für die Indexverschiebung der Distanz $d\in\Z$ gilt
\[\textstyle\sum_{k=m}^n a_k = \sum_{k=m+d}^{n+d} a_{k-d}.\]
\end{Satz}
\begin{Beweis}[Beweis 1]
Induktion über $n$. Im Anfang $n = m-1$ haben beide Seiten
der Gleichung den Wert null. Induktionsschritt:
\[\sum_{k=m}^n a_k = a_n+\sum_{k=m}^{n-1}a_k \stackrel{\mathrm{IV}}=
a_{(n+d)-d}+\sum_{k=m+d}^{n+d-1}a_{k-d}
= \sum_{k=m+d}^{n+d}a_{k-d}.\,\qedsymbol\]
\end{Beweis}
\begin{Beweis}[Beweis 2]
Mit der Substitution $k=k'-d$ findet sich die Umformung
\[\sum_{k=m}^n a_k \stackrel{\text{(1)}}= \sum_{m\le k\le n} a_k
\stackrel{\text{(2)}}= \sum_{m\le k'-d\le n} a_{k'-d}
\stackrel{\text{(3)}}= \sum_{m+d\le k'\le n+d} a_{k'-d}
\stackrel{\text{(4)}}= \sum_{k'=m+d}^{n+d} a_{k'-d},\]
wobei (1), (4) gemäß Satz \ref{sum-set-is-range} gelten
und (2), (3) eine andere Schreibweise für die Substitutionsregel
\ref{sum-set-subs} ist.\,\qedsymbol
\end{Beweis}
\strong{Bemerkung.} Der zweite Beweis ist eigentlich zirkulär,
weil der Beweis der Substitutionsregel über den Beweis von
Satz \ref{sum-set-well-defined} in transitiver Abhängigkeit zum
generalisierten Kommutativgesetz \ref{sum-perm-index} steht, dessen
Beweis einen Indexshift enthält.

\begin{Satz} Es gilt
\[\sum_{i=m}^n \sum_{j=m'}^{n'} a_{ij} = \sum_{j=m'}^{n'}\sum_{i=m}^n a_{ij}.\]
\end{Satz}
\begin{Beweis}
Induktion über $n$ und $n'$. Im Anfang bei $n=m-1$ und $n'=m-1$
haben beide Seiten der Gleichung den Wert null. Induktionsschritt für $n$:
\[\sum_{i=m}^n\sum_{j=m'}^{n'} a_{ij}
= \!\!\sum_{j=m'}^{n'} a_{nj}
+ \!\sum_{i=m}^{n-1}\sum_{j=m'}^{n'} a_{ij}
\stackrel{\mathrm{IV}}=
\!\sum_{j=m'}^{n'} a_{nj}
+ \!\sum_{j=m'}^{n'}\sum_{i=m}^{n-1} a_{ij}
= \!\!\sum_{j=m'}^{n'} (a_{nj}+\sum_{i=m}^{n-1} a_{ij})
= \!\!\sum_{j=m'}^{n'} \sum_{i=m}^n a_{ij}.\]
Induktionsschritt für $n'$:
\[\sum_{i=m}^n\sum_{j=m'}^{n'} a_{ij}
= \!\!\sum_{i=m}^n (a_{in'}+\!\!\sum_{j=m'}^{n'-1}a_{ij})
= \!\!\sum_{i=m}^n a_{in'}+\!\!\sum_{i=m}^n\sum_{j=m'}^{n'-1}a_{ij}
\stackrel{\mathrm{IV}}=
\!\sum_{i=m}^n a_{in'}+\sum_{j=m'}^{n'-1}\sum_{i=m}^n a_{ij}
= \!\!\sum_{j=m'}^{n'}\sum_{i=m}^n a_{ij}.\]
Weil immer ein Schritt nach rechts oder ein Schritt nach oben durchführbar ist,
werden alle Punkte $(n,n')$ im Gitter $\Z_{\ge m-1}\times\Z_{\ge m'-1}$ erreicht.\,\qedsymbol
\end{Beweis}

\begin{Satz}[Umkehrung der Reihenfolge]\label{sum-rev}\newlinefirst
Es gilt $\sum_{k=0}^n a_k = \sum_{k=0}^n a_{n-k}$.
\end{Satz}
\begin{Beweis}
Induktion über $n$. Im Anfang $n=-1$ haben beide Seiten der Gleichung
den Wert null. Der Induktionsschritt ist
\begin{align*}
\sum_{k=0}^n a_{n-k} &= a_{n-n} + \sum_{k=0}^{n-1} a_{n-k}
\stackrel{\mathrm{IV}}= a_0+\sum_{k=0}^{n-1} a_{n-(n-1-k)}a_k\\
&= a_0+\sum_{k=0}^{n-1} a_{k-1}
\stackrel{\text{(1)}}= \sum_{k=0}^0 a_k+\sum_{k=1}^n a_k
\stackrel{\text{(2)}}= \sum_{k=0}^n a_k,
\end{align*}
wobei (1) gemäß Indexshift \ref{sum-indexshift} und
(2) gemäß Aufteilung \ref{sum-split} gilt.\,\qedsymbol
\end{Beweis}

\newpage
\begin{Satz}[Generalisiertes Kommutativgesetz]%
\label{sum-perm-index}\newlinefirst
Sei $M=\{k\in\Z\mid m\le k\le n\}$. Für jede Permutation $\pi\colon M\to M$ gilt
\[\sum_{k=m}^n a_k = \sum_{k=m}^n a_{\pi(k)}.\]
\end{Satz}
\begin{Beweis} Induktiv. Sei ohne Beschränkung der Allgemeinheit $m=1$.
Im Induktionsanfang $n=0$ und $n=1$ ist die Gleichung offenkundig erfüllt.

Induktionsschritt. Induktionsvoraussetzung sei die Gültigkeit für $M$.
Zu zeigen ist die Gültigkeit für $M\cup\{n+1\}$.

Sei $t$ ein fester Parameter mit $1\le t\le n+1$.
Im Fall $\pi(t) = n+1$ geht man wie folgt vor.
Man setze $\sigma(k):=\pi(k)$ für $1\le k\le t-1$. Man setze
$\sigma(k):=\pi(k+1)$ für $t\le k\le n$. Weil $n+1$ kein Wert von
$\sigma$ ist, muss $\sigma$ eine Permutation $\sigma\colon M\to M$ sein.
Ergo gilt
\begin{align*}
\sum_{k=1}^{n+1} a_{\pi(k)} &= \sum_{k=1}^{t-1} a_{\pi(k)}
+ a_{\pi(t)} + \sum_{k=t+1}^{n+1} a_{\pi(k)}
= a_{\pi(t)} + \sum_{k=1}^{t-1} a_{\pi(k)}
+ \sum_{k=t}^n a_{\pi(k+1)}\\
&= a_{n+1} + \sum_{k=1}^{t-1} a_{\sigma(k)}
+ \sum_{k=t}^n a_{\sigma(k)}
= a_{n+1} + \sum_{k=1}^n a_{\sigma(k)}\\
&\stackrel{\mathrm{IV}}= a_{n+1} + \sum_{k=1}^n a_k
= \sum_{k=1}^{n+1} a_k.
\end{align*}
Man beachte, dass in den beiden Randfällen $t=1$ und $t=n+1$ die
jeweilige Randsumme den Wert null hat und somit verschwindet.\,\qedsymbol
\end{Beweis}

\begin{Definition}\label{def:sum-set}
Für eine endliche Menge $M$ definiert man
\[\sum_{k\in M} a_k := \sum_{i=m}^n a_{f(i)},\]
wobei $f\colon \{m,\ldots,n\}\to M$ eine frei wählbare Bijektion ist.
\end{Definition}

\begin{Satz}\label{sum-set-well-defined}
Der Wert Summe auf der rechten Seite von Def. \ref{def:sum-set}
ist unabhängig von der gewählten Bijektion.
\end{Satz}
\begin{Beweis} Seien $f,g$ zwei solche Bijektionen. Dann existiert
$\pi$ mit $f=g\circ\pi$, womit%
\[\sum_{i=m}^n a_{f(i)} = \sum_{i=m}^n a_{g(\pi(i))} =
\sum_{i=m}^n a_{g(i)}\]
laut Satz \ref{sum-perm-index} gilt.\,\qedsymbol
\end{Beweis}

\begin{Satz}\label{sum-set-is-range}
Für $M = \{k\in\Z\mid m\le k\le n\}$ gilt
\[\sum_{m\le k\le n} a_k := \sum_{k\in M} a_k = \sum_{k=m}^n a_k.\]
\end{Satz}
\begin{Beweis} Es gilt
$\sum_{k\in M} a_k = \sum_{k=m}^n a_{\id(k)} = \sum_{k=m}^n a_k$.\,\qedsymbol
\end{Beweis}

\begin{Satz}[Substitutionsregel]\label{sum-set-subs}
Ist $\varphi\colon M'\to M$ eine Bijektion, gilt
\[\sum_{k\in M} a_k = \sum_{k'\in M'} a_{\varphi(k')}.\]
\end{Satz}
\begin{Beweis} Zur Bijektion $f\colon\{1,\ldots,|M|\}\to M$ existiert die Bijektion
$g$ mit $f = \varphi\circ g$.

Infolge gilt
\[\sum_{k\in M} a_k = \sum_{i=1}^{|M|} a_{f(i)}
= \sum_{i=1}^{|M|} a_{\varphi(g(i))}
= \sum_{k'\in M'} a_{\varphi(k')}.\,\qedsymbol\]
\end{Beweis}

\begin{Satz} Es gilt
$\sum\limits_{k\in M} ca_k = c\sum\limits_{k\in M} a_k$ und
$\sum\limits_{k\in M} (a_k + b_k)
= \sum\limits_{k\in M} a_k + \sum\limits_{k\in M} b_k$.
\end{Satz}
\begin{Beweis}
Laut Definition gilt
\begin{gather*}
\sum_{k\in M} ca_k = \sum_{i=1}^{|M|} ca_{f(i)}
= c\sum_{i=1}^{|M|} a_{f(i)} = c\sum_{k\in M} a_k,\\
\sum_{k\in M} (a_k+b_k) = \sum_{i=1}^{|M|} (a_{f(i)}+b_{f(i)}) =
\sum_{i=1}^{|M|} a_{f(i)} + \sum_{i=1}^{|M|} b_{f(i)}
= \sum_{k\in M} a_k + \sum_{k\in M} b_k.\,\qedsymbol
\end{gather*}
\end{Beweis}

\begin{Satz} Es gilt
\[\sum_{k\in M}\sum_{l\in N} a_{kl} = \sum_{l\in N}\sum_{k\in M} a_{kl}.\]
\end{Satz}
\begin{Beweis}
Laut Definition gilt
\[\sum_{k\in M}\sum_{l\in N} a_{kl}
= \sum_{i=1}^{|M|}\sum_{j=1}^{|N|} a_{f(i),g(j)}
= \sum_{j=1}^{|N|}\sum_{i=1}^{|M|} a_{f(i),g(j)}
= \sum_{l\in N}\sum_{k\in M} a_{k,l}.\]
\end{Beweis}

\begin{Satz}\label{sum-set-disjoint}
Für $M\cap N=\emptyset$ gilt
\[\sum_{k\in M\cup N} a_k = \sum_{k\in M} a_k + \sum_{k\in N} a_k.\]
\end{Satz}
\begin{Beweis}
Sei $m:=|M|$ und $n:=|N|$. Laut Prämisse existiert eine Bijektion
$f\colon \{1,\ldots, m+n\}\to M\cup N$ mit $f(i)\in M$ für
$1\le i\le m$ und $f(i)\in N$ für $m+1\le i\le m+n$. Das macht
\[\sum_{k\in M\cup N} a_k = \sum_{i=1}^{m+n} a_{f(i)}
= \sum_{i=1}^m a_{f(i)} + \sum_{i=m+1}^{m+n} a_{f(i)}
= \sum_{k\in M} a_k + \sum_{k\in N} a_k.\,\qedsymbol\]
\end{Beweis}

\begin{Satz}\label{sum-partition}
Für eine disjunkte Zerlegung $M = \bigcup_{i\in I} M_i$ gilt
\[\sum_{k\in M} a_k = \sum_{i\in I}\sum_{k\in M_i} a_k.\]
\end{Satz}
\begin{Beweis}
Induktion über $I$. Im Anfang $I=\emptyset$ haben beide Seiten
den Wert null. Induktionsvoraussetzung sei die Gültigkeit
für $I$. Zu zeigen ist die Gültigkeit für $I\cup\{n\}$ mit $n\notin I$.
Der Induktionsschritt ist
\[\sum_{k\in M_n\cup M} a_k
= \sum_{k\in M_n} a_k + \sum_{k\in M} a_k
\stackrel{\mathrm{IV}}= \sum_{k\in M_n} a_k +
\sum_{i\in I}\sum_{k\in M_i} a_k
= \sum_{i\in I\cup\{n\}}\sum_{k\in M_i} a_k.\,\qedsymbol\]
\end{Beweis}

\begin{Satz}
Es gilt
\[\sum_{t\in M\times N} a_t = \sum_{k\in M}\sum_{l\in N} a_{(k,l)}.\]
\end{Satz}
\begin{Beweis}
Es ist $M = \bigcup_{k\in M} \{k\}$ und weiter $M\times N =
\bigcup_{k\in M} (\{k\}\times N)$ eine disjunkte Zerlegung. Hiermit
findet sich die Umformung
\[\sum_{t\in M\times N} a_t
\stackrel{\text{(1)}}= \sum_{k\in M}\;\sum_{t\in \{k\}\times N} a_t
\stackrel{\text{(2)}}= \sum_{k\in M}\sum_{l\in N} a_{(k,l)},\]
wobei (1) laut Satz \ref{sum-partition} gilt und (2) per Substitutionsregel
\ref{sum-set-subs} mit der Bijektion $\varphi\colon N\to\{k\}\times N$
mit $\varphi(l):=(k,l)$ und $t=\varphi(l)$.
\end{Beweis}

\begin{Satz} Mit der Indikatorfunktion $1_A\colon M\to\{0,1\}$
für $A\subseteq M$ gilt
\[\sum_{k\in M} 1_A(k)a_k = \sum_{k\in A} a_k.\]
\end{Satz}
\begin{Beweis}
Mit disjunkter Zerlegung $M=A\cup (M\setminus A)$
und Satz \ref{sum-set-disjoint} gilt
\[\sum_{k\in M} 1_A(k)a_k = \sum_{k\in A} \underbrace{1_A(k)}_{1} a_k
+ \sum_{k\in M\setminus A}\underbrace{1_A(k)}_{0} a_k
= \sum_{k\in A} a_k.\,\qedsymbol\]
\end{Beweis}

\begin{Satz} Allgemein gilt
\[\sum_{k\in A\cup B} a_k = \sum_{k\in A} a_k + \sum_{k\in B} a_k
- \sum_{k\in A\cap B} a_k.\]
\end{Satz}
\begin{Beweis} Sei $G=A\cup B$ die Grundmenge. Gemäß Satz
\ref{indicator-set-op} darf man rechnen
\begin{align*}
\sum_{k\in G} a_k &= \sum_{k\in G} 1_{A\cup B}(k)a_k
= \sum_{k\in G} 1_A(k)a_k + \sum_{k\in G} 1_B(k)a_k
- \sum_{k\in G} 1_{A\cap B}(k)a_k\\
&= \sum_{k\in A} a_k + \sum_{k\in B} a_k - \sum_{k\in A\cap B} a_k.\,\qedsymbol
\end{align*}
\end{Beweis}

\begin{Definition}[Differenzenfolge]\index{Differenzenfolge}
Zu einer Folge $(a_k)$ definiert man
\[(\Delta a)_k := a_{k+1} - a_k.\]
\end{Definition}

\begin{Satz}[Teleskopsumme]\label{sum-tele} Es gilt
\[\sum_{k=m}^{n-1} (\Delta a)_k = \sum_{k=m}^{n-1} (a_{k+1} - a_k) = a_n - a_m.\]
\end{Satz}
\begin{Beweis}[Beweis 1] Induktion über $n$. Im Anfang $n=m$ haben beide Seiten
der Gleichung den Wert null. Induktionsschritt:
\[\sum_{k=m}^n (a_{k+1} - a_k) = (a_{n+1} - a_n) + \!\!\sum_{k=m}^{n-1} (a_{k+1} - a_n)
\stackrel{\mathrm{IV}}= a_{n+1} - a_n + a_n - a_m = a_{n+1} - a_m.\,\qedsymbol\]
\end{Beweis}
\begin{Beweis}[Beweis 2]
Per Indexshift \ref{sum-indexshift}
gilt $\sum\limits_{k=m}^{n-1} a_{k+1} = \!\!\sum\limits_{k=m+1}^n\!\! a_k
= a_n - a_m + \sum\limits_{k=m}^{n-1} a_k$.
Somit ist%
\[\sum_{k=m}^{n-1} (a_{k+1} - a_k) = \sum_{k=m}^{n-1} a_{k+1} - \sum_{k=m}^{n-1} a_k
= a_n - a_m + \sum_{k=m}^{n-1} a_k - \sum_{k=m}^{n-1} a_k = a_n - a_m.\,\qedsymbol\]
\end{Beweis}

\begin{Satz}
Zum Beweis einer Formel
\[\sum_{k=m}^{n-1} a_k = s_n\]
genügt es, $s_m=0$ und $(\Delta s)_n = a_n$ zu zeigen.
\end{Satz}
\begin{Beweis}[Beweis 1]
Induktion über $n$. Im Anfang $n=m$ haben beide Seiten der Gleichung laut
der Prämisse den Wert null. Induktionsschritt:
\[\sum_{k=m}^n a_k = a_n + \sum_{k=m}^{n-1} a_k
\stackrel{\mathrm{IV}}= a_n + s_n
= (\Delta s)_n + s_n = s_{n+1} - s_n + s_n = s_{n+1}.\,\qedsymbol\]
\end{Beweis}
\begin{Beweis}[Beweis 2]
Spezialisierung von Satz \ref{sum-tele}.\,\qedsymbol
\end{Beweis}

\begin{Satz} Der Differenzoperator ist linear. Das heißt,
für alle Folgen $(a_n), (b_n)$ und jede Konstante $c$ gilt
\begin{align*}
& \Delta(a+b) = \Delta a + \Delta b, && ((a+b)_n := a_n + b_n)\\
& \Delta(ca) = c\Delta a. && ((ca)_n := ca_n)
\end{align*}
\end{Satz}
\begin{Beweis} Man findet
\begin{align*}
(\Delta(a+b))_n &= (a+b)_{n+1} - (a+b)_n
= (a_{n+1}+b_{n+1}) - (a_n+b_n)\\
&= a_{n+1}-a_n + b_{n+1}-b_n = (\Delta a)_n + (\Delta b)_n
= (\Delta a + \Delta b)_n
\end{align*}
und
\[(\Delta(ca))_n = (ca)_{n+1} - (ca)_n = ca_{n+1} - ca_n
= c(a_{n+1} - a_n) = c(\Delta a)_n = (c\Delta a)_n.\,\qedsymbol\]
\end{Beweis}

\begin{Definition}[Shiftoperator] Man definiert
\[(Ta)_n := a_{n+1}.\]
\end{Definition}

\begin{Satz}[Iterierter Differenzoperator]\newlinefirst
Für jede Folge $(a_n)$ und $m\in\Z_{\ge 0}$ gilt
\[(\Delta^m a)_n = (-1)^m\sum_{k=0}^m\binom{m}{k} (-1)^k a_{n+k}.\]
\end{Satz}
\begin{Beweis} Es gilt $\Delta = T-\id$. Weil $T$ und $\id$ kommutieren,
ist der binomische Lehrsatz anwendbar. Es ergibt sich
\[\Delta^m = (T-\id)^m = \sum_{k=0}^m\binom{m}{k} (-1)^{m-k} T^k\id^{m-k}
= (-1)^m \sum_{k=0}^m\binom{m}{k} (-1)^k T^k.\,\qedsymbol\]
\end{Beweis}

\begin{Satz}\label{delta-deg}
Sei $f$ eine Polynomfunktion. Dann ist $\Delta_h f$ eine
Polynomfunktion mit niedrigerem Grad.
\end{Satz}
\begin{Beweis} Für $f(x)=\sum_{n=0}^m a_n x^n$ gilt
\begin{gather*}
\Delta_h f(x) = f(x+h) - f(x) = \sum_{n=0}^m a_n (x+h)^n - \sum_{n=0}^m a_n x^n
= \sum_{n=0}^m a_n ((x+h)^n - x^n)\\
= \sum_{n=0}^m a_n (x^n + \sum_{k=0}^{n-1}\binom{n}{k}x^k h^{n-k} - x^n)
= \sum_{n=0}^m a_n \sum_{k=0}^{n-1}\binom{n}{k}x^k h^{n-k}.
\end{gather*}
In der Summe treten nur Monome bis $x^{m-1}$ auf.\,\qedsymbol
\end{Beweis}

\begin{Satz} Sei $f$ ein Polynom vom Grad $N$. Für $n,a\in\Z$ und $n\ge a$ gilt
\[f(n) = \sum_{k=0}^N \frac{(\Delta^k f)(a)}{k!}(n-a)^{\underline k}
= \sum_{k=0}^N \binom{n-a}{k}(\Delta^k f)(a).\]
\end{Satz}
\begin{Beweis}
Es gilt $T=\Delta+\id$. Für jede nichtnegative ganze Zahl $m$ gilt
\[T^m = (\Delta+\id)^m = \sum_{k=0}^m\binom{m}{k}\Delta^k\]
mit dem binomischen Lehrsatz, da $\Delta$ und $\id$ kommutieren. Das macht
\[f(a + m) = \sum_{k=0}^m\binom{m}{k}(\Delta^k f)(a).\]
Man substituiere nun $n = a+m$. Für $n\ge a$ gilt dann
\[f(n) = \sum_{k=0}^{n-a}\binom{n-a}{k}(\Delta^k f)(a)
= \sum_{k=0}^N\binom{n-a}{k}(\Delta^k f)(a).\]
Der Indexbereich der Summierung durfte auf bis $k=N$ geändert werden, weil
$\Delta^k f = 0$ für $k>N$ laut Satz \ref{delta-deg} gilt. Dass nun
Summanden mit $k>n-a$ auftreten können, ist nicht weiter schlimm, weil
in diesem Fall $\binom{n-a}{k}=0$ ist.\,\qedsymbol
\end{Beweis}

\begin{Satz}
Es sei $(\theta f)(x) := xf'(x)$, zuweilen geschrieben als $xD$ oder
$x\tfrac{\mathrm d}{\mathrm dx}$. Es sei $f$ an der Stelle $x$ definiert
und dort $m$-mal differenzierbar. Für die $m$-te Iteration von $\theta$
gilt dann die Formel%
\[(\theta^m f)(x) = \sum_{k=0}^m\begin{Bmatrix}m\\ k\end{Bmatrix}x^k f^{(k)}(x),\]
wobei die $\big\{\!\begin{smallmatrix}m\\ k\end{smallmatrix}\!\big\}$ die
Stirling"=Zahlen zweiter Art bezeichnen.
\end{Satz}
\begin{Beweis} Induktion über $m$. Im Anfang $m=0$ muss
$\theta^0$ per Definition der Iteration die identische Abbildung
sein. Dies bestätigt sich mit
$\big\{\!\begin{smallmatrix}0\\ 0\end{smallmatrix}\!\big\}=1$
und $f^{(0)}(x)=f(x)$.

Zum Schritt. Für $m\ge 1$ gilt die Rekurrenz
\[\begin{Bmatrix}m\\ k\end{Bmatrix} = k\begin{Bmatrix}m-1\\ k\end{Bmatrix} + \begin{Bmatrix}m-1\\ k-1\end{Bmatrix}.\]
Mit dieser gelingt die Umformung
\begin{align*}
\theta^m f(x) &= \theta \theta^{m-1} f(x) \stackrel{\mathrm{IV}}=
x\tfrac{\mathrm d}{\mathrm dx}\sum_{k=0}^{m-1}\begin{Bmatrix}m-1\\ k\end{Bmatrix}x^k f^{(k)}(x)\\
&= \sum_{k=0}^{m-1}\begin{Bmatrix}{m-1}\\ k\end{Bmatrix}kx^k f^{(k)}(x)
+ \sum_{k=0}^{m-1}\begin{Bmatrix}m-1\\ k\end{Bmatrix}x^{k+1} f^{(k+1)}(x)\\
&= \sum_{k=1}^m\begin{Bmatrix}{m-1}\\ k\end{Bmatrix}kx^k f^{(k)}(x)
+ \sum_{k=1}^m\begin{Bmatrix}m-1\\ k-1\end{Bmatrix}x^k f^{(k)}(x)\\
&= \sum_{k=1}^m\begin{Bmatrix}m\\ k\end{Bmatrix}x^k f^{(k)}(x)
= \sum_{k=0}^m\begin{Bmatrix}m\\ k\end{Bmatrix}x^k f^{(k)}(x).\,\qedsymbol
\end{align*}
\end{Beweis}

\newpage
\begin{Satz}
Es sei $f\colon\R_{\ge 0}\to\R$ stetig und
$(\theta_a^{-1} f)(x):=\int_a^x\frac{f(t)}{t}\,\mathrm dt$,
wobei dieses Integral zunächst im eigentlichen Sinne verstanden sein
soll, also $a>0$ sein soll. Für die $m$-te Iteration mit
$m\in\Z_{\ge 1}$ gilt%
\[(\theta_a^{-m}f)(x) = \frac{1}{(m-1)!}\int_a^x\ln\left(\frac{x}{t}\right)^{m-1}
\frac{f(t)}{t}\,\mathrm dt.\]
\end{Satz}
\strong{Beweis.}
Mit der Substitution $t=\ee^u$ findet sich%
\[\theta_a^{-1} f(x) = \int_a^x\frac{f(t)}{t}\,\mathrm dt
= \int_{\ln a}^{\ln x} f(\ee^u)\,\mathrm du.\]
Bei der Iteration heben sich nun jeweils der Logarithmus der Grenze und
das Exponentieren des Arguments gegenseitig auf. Man gelangt daher
hinsichtlich $(I_c f)(x):=\int_c^x f(t)\,\mathrm dt$ mit
der cauchyschen Formel für iterierte Integration zu%
\[\theta_a^{-m} f(e^x) = I_{\ln a}^m (u\mapsto f(\ee^u))(x)
= \frac{1}{(m-1)!}\int_{\ln a}^x (x-u)^{m-1}f(\ee^u)\,\mathrm du.\]
Durch Resubstitution $\ee^u=t$ findet sich somit%
\[\theta_a^{-m} f(x) = \frac{1}{(m-1)!}\int_a^x(\ln x-\ln t)^{m-1}
\frac{f(t)}{t}\,\mathrm dt.\,\qedsymbol\]

\newpage
\subsection{Klassische Partialsummen}

\begin{Satz}[Partialsummen der konstanten Folge]%
\label{sum-const}\newlinefirst
Es gilt $\displaystyle\sum_{k=m}^n 1 = n-m+1$.
\end{Satz}
\begin{Beweis}
Induktion über $n$. Im Anfang $n=m-1$ haben beide Seiten
der Gleichung den Wert null. Induktionsschritt:
\[\sum_{k=m}^n 1 = 1 + \sum_{k=m}^{n-1}
\stackrel{\mathrm{IV}}= 1 + n-1-m+1 = n-m+1.\,\qedsymbol\]
\end{Beweis}

\begin{Satz}[Partialsummen der arithmetischen Folge]%
\index{arithmetische Folge}\newlinefirst
Es gilt $\displaystyle\sum_{k=0}^n k = \frac{n}{2}(n+1)$.
\end{Satz}
\begin{Beweis}[Beweis 1]
Induktion über $n$. Im Anfang $n=-1$ haben beide Seiten der Gleichung
den Wert null. Induktionsschritt:
\[\sum_{k=0}^n k = n + \sum_{k=0}^{n-1} k
\stackrel{\mathrm{IV}}= n + \frac{n-1}{2}(n-1+1)
= \frac{n}{2}(2 + n-1) = \frac{n}{2}(n+1).\,\qedsymbol\]
\end{Beweis}
\begin{Beweis}[Beweis 2]
Klassischer Beweis. Man findet die Umformung
\[2\!\sum_{k=0}^n k = \!\sum_{k=0}^n k + \!\sum_{k=0}^n k
\stackrel{\text{(1)}}= \!\sum_{k=0}^n k + \!\sum_{k=0}^n (n-k)
\stackrel{\text{(2)}}= \!\sum_{k=0}^n (k+n-k)
= \!\sum_{k=0}^n n \stackrel{\text{(3)}}= n\!\sum_{k=0}^n 1
\stackrel{\text{(4)}}= n(n+1),\]
wobei (1), (2), (3), (4) gemäß Satz
\ref{sum-rev}, \ref{sum-add}, \ref{sum-scale}, \ref{sum-const}
gelten.\,\qedsymbol
\end{Beweis}

\begin{Satz}[Partialsummen der geometrischen Folge]%
\label{sum-geom-seq}\index{geometrische Folge}\newlinefirst
Für $m\ge 0$ und $z\in\C\setminus\{1\}$ gilt
$\displaystyle\sum_{k=m}^{n-1} z^k = \frac{z^n-z^m}{z-1}$.
\end{Satz}
\begin{Beweis}
Induktion über $n$. Im Anfang $n=m-1$ haben beiden Seiten der Gleichung
den Wert null. Induktionsschritt:
\[\sum_{k=m}^n z^k = z^n + \sum_{k=m}^{n-1} z^k
\stackrel{\mathrm{IV}}= z^n + \frac{z^n-z^m}{z-1}
= \frac{(z-1)z^n+z^n-z^m}{z-1}
= \frac{z^{n+1}-z^m}{z-1}.\,\qedsymbol\]
\end{Beweis}

\begin{Satz}
Für $m\ge 0$ und $z\in\C\setminus\{1\}$ gilt
\[\sum_{k=m}^{n-1} kz^k
= \frac{(nz^n-mz^m)(z-1) - (z^n-z^m)z}{(z-1)^2}.\]
\end{Satz}
\begin{Beweis}
Die Gleichung von Satz \ref{sum-geom-seq} für $m\ge 1$ auf beiden
Seiten nach $z$ ableiten und anschließend beide Seiten mit $z$
multiplizieren. Den Fall $m=0$ und in diesem den Summand zu $k=0$
explizit betrachten, sonst aber auf dieselbe Weise vorgehen.\,\qedsymbol
\end{Beweis}

\newpage
\begin{Satz} Es gilt
\[\sum_{k=1}^n (-1)^k k = (-1)^n\left\lfloor\frac{n+1}{2}\right\rfloor.\]
\end{Satz}
\begin{Beweis}
Induktion über $n$. Im Anfang $n=0$ haben beiden Seiten den Wert null.

Induktionsschritt:
\[\sum_{k=1}^n (-1)^k k = (-1)^n n + \sum_{k=1}^{n-1} (-1)^k k
\stackrel{\mathrm{IV}}= (-1)^n n + (-1)^{n-1}\left\lfloor\frac{n}{2}\right\rfloor
= (-1)^n (n-\left\lfloor\frac{n}{2}\right\rfloor).\]
Zu zeigen verbleibt die Gleichung
\[n-\left\lfloor\frac{n}{2}\right\rfloor = \left\lfloor\frac{n+1}{2}\right\rfloor
\iff n = \left\lfloor\frac{n}{2}\right\rfloor + 
\left\lfloor\frac{n+1}{2}\right\rfloor.\]
Wir nehmen die Fallunterscheidung zwischen geraden und ungeraden
Zahlen vor, um Satz \ref{floor-add-int} und \ref{floor-is-zero}
nutzen zu können. Im geraden Fall $n=2k$ bestätigt sich
\[\left\lfloor\frac{2k}{2}\right\rfloor +  \left\lfloor\frac{2k+1}{2}\right\rfloor
= \lfloor k\rfloor + \left\lfloor k + \frac{1}{2}\right\rfloor = k + k = 2k.\]
Im ungeraden Fall $n=2k+1$ bestätigt sich
\[\left\lfloor\frac{2k+1}{2}\right\rfloor + \left\lfloor\frac{2k+1+1}{2}\right\rfloor
= \left\lfloor k + \frac{1}{2}\right\rfloor + \left\lfloor k+1\right\rfloor
= k + k + 1 = 2k + 1.\,\qedsymbol\]
\end{Beweis}

\section{Funktionen}

\subsection{Floor und Ceil}

\begin{Definition}[Floorfunktion]%
\label{def:floor}\index{Floorfunktion}
Für $x\in\R$ definiert man
\[y = \lfloor x\rfloor\defiff y\in\Z\land 0\le x-y < 1.\]
\end{Definition}

\begin{Definition}[Ceilfunktion]%
\label{def:ceil}\index{Ceilfunktion}
Für $x\in\R$ definiert man
\[y = \lceil x\rceil\defiff y\in\Z\land 0\le y-x < 1.\]
\end{Definition}

\begin{Satz}\label{floor-add-int}
Für jede ganze Zahl $k$ gilt $\lfloor k + x\rfloor = k + \lfloor x\rfloor$.
\end{Satz}
\begin{Beweis} Aufgrund der Prämisse $k\in\Z$ ist $y\in\Z$ äquivalent
zu $y-k\in\Z$. Unter dieser Gegebenheit findet sich mit
Def. \ref{def:floor} die äquivalente Umformung
\begin{align*}
y = \lfloor k+x\rfloor &\iff y\in\Z\land 0\le (k+x)-y < 1\iff y-k\in\Z\land 0\le x-(y-k) < 1\\
&\iff y - k = \lfloor x\rfloor \iff y = k + \lfloor x\rfloor.\,\qedsymbol
\end{align*}
\end{Beweis}

\begin{Satz}\label{floor-is-zero}
Für $0\le x < 1$ gilt $\lfloor x\rfloor = 0$.
\end{Satz}
\begin{Beweis}
Dies folgt unmittelbar aus Def. \ref{def:floor}.\,\qedsymbol
\end{Beweis}

\newpage
\begin{Definition}[Maximum, Minimum]\label{def:max-min}\newlinefirst
Für eine Menge $M\subseteq\R$ definiert man
\[\begin{array}{@{}l@{\,}l@{}}
y = \max M &\defiff y\in M\land\forall k\in M\colon k\le y,\\[2pt]
y = \min M &\defiff y\in M\land\forall k\in M\colon y\le k.
\end{array}\]
\end{Definition}

\begin{Satz}
Es gilt
\begin{align*}
\lfloor x\rfloor &= \max\{k\in\Z\mid k\le x\},\\
\lceil x\rceil &= \min\{k\in\Z\mid x\le k\}.
\end{align*}
\end{Satz}
\begin{Beweis}
Mit bezüglich $y=\lfloor x\rfloor$ entfalteten Def. \ref{def:floor},
\ref{def:max-min} lautet die Aussage
\[y\in\Z\land 0\le x-y<1\iff
y\in\Z\land y\le x\land (\forall k\in\Z\colon k\le x\Rightarrow k\le y).\]
Für die Implikation von links nach rechts ist im Wesentlichen
\[k\in\Z, y\in\Z, k\le x, x < y+1\vdash k\le y\]
zu zeigen. Man erhält zunächst $k < y+1$ per Transitivgesetz.
Wegen $k,y\in\Z$ folgt daraus $k\le y$. Für die Implikation von rechts
nach links ist im Wesentlichen
\[y\in\Z,(\forall k\in\Z\colon k\le x\Rightarrow k\le y)\vdash x < y + 1\]
zu zeigen. Angenommen, es wäre $y+1\le x$. Die Allaussage wird mit
$k:=y+1$ spezialisiert. Per Modus ponens erhält man den Widerspruch
$y+1\le y$. Ergo muss die Annahme falsch sein, was äquivalent zu
$x<y+1$ ist. Der Beweis zu Ceil verläuft analog.\,\qedsymbol
\end{Beweis}

\begin{Satz}\label{floor-div-floor}
Für $x\in\R$ und $n\in\Z_{\ge 1}$ gilt
\[\left\lfloor\frac{x}{n}\right\rfloor
= \left\lfloor\frac{\lfloor x\rfloor}{n}\right\rfloor.\]
\end{Satz}
\begin{Beweis}
Sei $a:=x-\lfloor x\rfloor$. Per Def. \ref{def:floor} gilt $0 \le a < 1$.
Laut dem Lemma zur euklidischen Division gilt außerdem
$\lfloor x\rfloor = qn+r$
mit $q=\lfloor\frac{\lfloor x\rfloor}{n}\rfloor$
und $0\le r\le n - 1$. Infolge gilt $0\le a+r<n$, also
$0\le\frac{a+r}{n}<1$ und somit $\lfloor\frac{a+r}{n}\rfloor = 0$
laut Satz \ref{floor-is-zero}.
Es findet sich%
\[\left\lfloor\frac{x}{n}\right\rfloor
= \left\lfloor\frac{a+\lfloor x\rfloor}{n}\right\rfloor
= \left\lfloor\frac{a+qn+r}{n}\right\rfloor
\stackrel{\text{(1)}}= q + \left\lfloor\frac{a+r}{n}\right\rfloor
\stackrel{\text{(2)}}= q =
\left\lfloor\frac{\lfloor x\rfloor}{n}\right\rfloor,\]
wobei (1) laut Satz \ref{floor-add-int} gilt, und (2)
wie soeben ausgeführt.\,\qedsymbol
\end{Beweis}

\begin{Satz}
Für $x\in\R$ und $m,n\in\Z$ mit $n\ge 1$ gilt
\[\left\lfloor\frac{m+x}{n}\right\rfloor
= \left\lfloor\frac{m+\lfloor x\rfloor}{n}\right\rfloor.\]
\end{Satz}
\begin{Beweis}
Laut Satz \ref{floor-add-int} ist $m+\lfloor x\rfloor = \lfloor m+x\rfloor$.
Die Aussage folgt nun als Korollar aus Satz \ref{floor-div-floor}.\,\qedsymbol
\end{Beweis}

\newpage
\subsection{Faktorielle}

\begin{Definition}[Fakultät]%
\label{def:factorial}\index{Fakultaet@Fakultät}\newlinefirst
Für eine nichtnegative ganze Zahl $n$ definiert man $n!$ rekursiv durch
\[0! := 1,\quad (n+1)! := (n+1)n!.\]
\end{Definition}

\begin{Definition}[Fallende Faktorielle]\label{def:falling-factorial}%
\index{Faktorielle!fallende}\index{fallende Faktorielle}\newlinefirst
Für $k\in\Z_{\ge 0}$ und $n\in\Z$ (oder allgemeiner $n\in\C$)
definiert man $n^{\underline k}$ rekursiv durch
\[n^{\underline 0} := 1,\quad n^{\underline {k+1}}:=n(n-1)^{\underline k}.\]
\end{Definition}

\begin{Definition}[Steigende Faktorielle]\label{def:raising-factorial}%
\index{Faktorielle!steigende}\index{steigende Faktorielle}\newlinefirst
Für $k\in\Z_{\ge 0}$ und $n\in\Z$ (oder allgemeiner $n\in\C$)
definiert man $n^{\overline k}$ rekursiv durch
\[n^{\overline 0} := 1,\quad n^{\overline {k+1}}:=n(n+1)^{\overline k}.\]
\end{Definition}

\begin{Satz}\label{relation-ff-factorial}
Für $n,k\in\Z_{\ge 0}$ und $k\le n$ gilt
\[n^{\underline k} = \frac{n!}{(n-k)!}.\]
\end{Satz}
\begin{Beweis}
Induktion über $k$. Im Anfang $k=0$ resultieren beide Seiten der
Gleichung im gleichen Wert~1. Der Induktionsschritt ist
\[n^{\underline k} = n(n-1)^{\underline{k-1}}
\stackrel{\text{IV}}= n\frac{(n-1)!}{((n-1)-(k-1))!}
= \frac{n(n-1)!}{(n-k)!}
= \frac{n!}{(n-1)!}.\,\qedsymbol\]
\end{Beweis}

\begin{Satz}
Für ganze Zahlen $n,k$ mit $n\ge 1$ und $k\ge 1-n$ gilt
\[n^{\overline k} = \frac{(n+k-1)!}{(n-1)!}.\]
\end{Satz}
\begin{Beweis}
Induktion über $k$. Im Anfang $k=0$ resultieren beide Seiten der
Gleichung im gleichen Wert~1. Der Induktionsschritt ist
\begin{align*}
n^{\overline k} &= n(n+1)^{\overline{k-1}}
\stackrel{\text{IV}}= n\frac{(n+1+k-1-1)!}{(n+1-1)!}
= \frac{n(n+k-1)!}{n!}\\
&= \frac{n(n+k-1)!}{n(n-1)!}
= \frac{(n+k-1)!}{(n-1)!}.\,\qedsymbol
\end{align*}
\end{Beweis}

\begin{Satz}
Für jedes $n\in\Z_{\ge 0}$ gilt $n!\le n^n$.
\end{Satz}
\begin{Beweis}
Induktion über $n$. Im Induktionsanfang $n=0$ hat man $0! = 1$ und $0^0=1$.
Zum Induktionsschritt unternimmt man zunächst die äquivalente Umformung
\[(n+1)!\le (n+1)^{n+1} \iff (n+1)n!\le (n+1)(n+1)^n
\iff n!\le (n+1)^n.\]
Diese Ungleichung bestätigt die Rechnung
\[(n+1)^n = \sum_{k=0}^n\binom{n}{k}n^k =
n^n+\sum_{k=0}^{n-1}\binom{n}{k}n^k\ge n^n
\stackrel{\mathrm{IV}}\ge n!.\,\qedsymbol\]
\end{Beweis}

\newpage
\subsection{Binomialkoeffizient}

\begin{Definition}[Binomialkoeffizient]%
\label{def:binom}\index{Binomialkoeffizient}\newlinefirst
Für $k\in\Z_{\ge 0}$ und $n\in\Z$ (oder allgemeiner $n\in\C$)
definiert man
\[\binom{n}{k} := \frac{n^{\underline k}}{k!}.\]
\end{Definition}

\begin{Satz}
Für $n\in\Z$ mit $k\le n$ gilt
\[\binom{n}{k} = \frac{n!}{k!(n-k)!}\]
\end{Satz}
\begin{Beweis}
Folgt direkt aus Def. \ref{def:binom} und Satz
\ref{relation-ff-factorial}.\,\qedsymbol
\end{Beweis}

\begin{Satz}
Für $k\ge 1$ und $n\in\Z$ (oder allgemeiner $n\in\C$) gilt
\[\binom{n}{k} = \frac{n}{k}\binom{n-1}{k-1}.\]
\end{Satz}
\begin{Beweis}
Es findet sich die Umformung
\[\binom{n}{k} = \frac{n^{\underline k}}{k!}
= \frac{n(n-1)^{\underline {k-1}}}{k(k-1)!}
= \frac{n}{k}\binom{n-1}{k-1}.\,\qedsymbol\]
\end{Beweis}

\begin{Satz}
Für $k\ge 1$ und $n\in\Z$ (oder allgemeiner $n\in\C$) gilt
\[\binom{n}{k} = \binom{n-1}{k-1} + \binom{n-1}{k}.\]
\end{Satz}
\begin{Beweis}
Es findet sich die Umformung
\begin{align*}
\binom{n-1}{k-1} + \binom{n-1}{k}
&= \frac{(n-1)^{\underline{k-1}}}{(k-1)!} + \frac{(n-1)^{\underline k}}{k!}
= \frac{k(n-1)^{\underline{k-1}}}{k!} + \frac{(n-1)^{\underline{k-1}}(n-k)}{k!}\\
&= \frac{(n-1)^{\underline{k-1}}}{k!}(k + n -k)
= \frac{n(n-1)^{\underline{k-1}}}{k!}
= \frac{n^{\underline k}}{k!} = \binom{n}{k}.\,\qedsymbol
\end{align*}
\end{Beweis}
