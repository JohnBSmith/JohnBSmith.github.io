
\chapter{Topologie}
\section{Grundbegriffe}
\subsection{Definitionen}

\begin{Definition}[nhfilter: Umgebungsfilter]%
\label{def:nhfilter}\index{Umgebungsfilter}
\[\underline U(x) := \{U{\subseteq}X\mid
\exists O(O\in T\land x\in O\land O\subseteq U)\}.\]
\end{Definition}

\begin{Definition}[int: offener Kern]
\label{def:int}\index{offener Kern}
\[\operatorname{int}(M) := \{x\in M\mid M\in \underline U(x)\}\]
\end{Definition}

\begin{Satz}
Der offene Kern von $M$ ist die Vereinigung der offenen Teilmengen
von $M$. Kurz:%
\[\operatorname{int}(M) = \bigcup_{O\in 2^M\cap T} O.\]
\end{Satz}

\begin{Beweis}
Nach Def. \ref{def:seteq} (seteq) und Def. \ref{def:int} (int)
expandieren:
\[\forall x[x\in M\land M\in\underline U(x)
\iff x\in\bigcup_{O\in 2^M\cap T} O].\]
Den äußeren Allquantor brauchen wir nicht weiter mitschreiben, da alle
freien Variablen automatisch allquantifiziert werden.
Nach Def. \ref{def:nhfilter} (nhfilter) weiter expandieren, wobei die
Bedingung $U\subseteq X$ als tautologisch entfallen kann,
weil $X$ die Grundmenge ist. Auf der rechten Seite wird nach Def.
\ref{def:union} (union) expandiert. Es ergibt sich:
\[x\in M\land \exists O(O\in T\land x\in O\land O\subseteq M)
\iff \exists O(O\subseteq M\land O\in T\land x\in O).\]
Wegen $A\land\exists x(P(x))\iff \exists x(A\land P(x))$ ergibt
sich auf der linken Seite:
\[\exists O(x\in M\land O\in T\land x\in O\land O\subseteq M).\]
Wenn aber $O\subseteq M$ erfüllt sein muss, gilt
$x\in O\implies x\in M$. Demnach kann $x\in M$ entfallen.
Auf beiden Seiten steht dann die gleiche Bedingung.\,\qedsymbol
\end{Beweis}

\section{Metrische Räume}
\subsection{Metrischer Räume}
\begin{Definition}[metric-space: metrischer Raum]%
\index{metrischer Raum}\label{metric-space}
Man bezeichet $(M,d)$ mit $d\colon M^2\to\R$ genau dann als
metrischen Raum, wenn die folgenden Axiome erfüllt sind:
\begin{align*}
\text{(M1)}\quad & d(x,y)=0\iff x=y, &&\text{(Gleichheit abstandsloser Punkte)}\\
\text{(M2)}\quad & d(x,y)=d(y,x), &&\text{(Symmetrie)}\\
\text{(M3)}\quad & d(x,y)\le d(x,z)+d(z,y). &&\text{(Dreiecksungleichung)}
\end{align*}
\end{Definition}

\begin{Definition}[open-ep-ball: offene Epsilon-Umgebung]\mbox{}\\
Für einen metrischen Raum $(M,d)$ und $p\in M$:
\[U_\varepsilon(p) := \{x\mid d(p,x)<\varepsilon\}.\]
\end{Definition}
Bemerkung: Unter einer Epsilon-Umgebung ohne weitere Attribute
versteht man immer eine offene Epsilon-Umgebung.

\begin{Satz}[Konstruktion disjunkter Epsilon-Umgebungen]%
\label{construction-disjoint-ep-balls}
Sei $(M,d)$ ein metrischer Raum und $p,q\in M$ mit $p\ne q$.
Betrachte die Streckenzerlegung $d(p,q)=A+B$. Für $a\le A$ und
$b\le B$ sind die Epsilon-Umgebungen $U_a(p)$ und $U_b(q)$ disjunkt.
\end{Satz}

\begin{Beweis}
Angenommen $U_a(p)$ und $U_b(q)$ wären nicht disjunkt, dann gäbe
es mindestens ein $x$ mit $x\in U_a(p)$ und $x\in U_b(q)$, d.\,h.
$d(p,x)<a$ und $d(q,x)<b$. Addition der beiden Ungleichungen
bringt
\[d(p,x)+d(q,x)<a+b\le d(p,q).\]
Gemäß der Dreiecksungleichung Def. \ref{metric-space} Axiom (M3) gilt
nun aber
\[d(p,q)\le d(p,x)+d(q,x)\]
für alle $x$. Sei $c:=d(p,x)+d(q,x)$. Wir erhalten damit nun
$c<a+b\le c$ und somit den Widerspruch $c<c$.\,\qedsymbol
\end{Beweis}

\begin{Korollar}[Unterschiedliche Punkte eines metrischen Raumes
besitzen disjunkte Epsilon-Umgebungen]
Sei $(M,d)$ ein metrischer Raum und $p,q\in M$.
Wenn $p\ne q$ ist, dann gibt es disjunkte offene
Epsilon-Umgebungen $U_a(p)$ und $U_b(q)$.
\end{Korollar}

\begin{Beweis}
Folgt trivial aus Satz \ref{construction-disjoint-ep-balls}.
Wähle speziell z.\,B. $a=b=d(p,q)/2$.\,\qedsymbol
\end{Beweis}

\subsection{Normierte Räume}
\begin{Definition}[normed-space: normierter Raum]%
\label{def:normed-space}\index{normierter Raum}\index{Dreiecksungleichung}
Sei $V$ ein Vektorraum über dem Körper der rellen oder komplexen
Zahlen. Sei $N(x)=\|x\|$ eine Abbildung, die jedem $x\in V$ eine
reelle Zahl zuordnet. Man nennt $(V,N)$ genau dann einen
normierten Raum, wenn die folgenden Axiome erfüllt sind:
\begin{align*}
\text{(N1)}\quad &\|x\|=0\iff x=0,&&\text{(Definitheit)}\\
\text{(N2)}\quad &\|\lambda x\|=|\lambda|\|x\|,&&\text{(betragsmäßige Homogenität)}\\
\text{(N3)}\quad &\|x+y\| \le \|x\|+\|y\|.&&\text{(Dreiecksungleichung)}
\end{align*}
\end{Definition}

\begin{Satz}[umgekehrte Dreiecksungleichung]%
\label{rev-tineq}\index{umgekehrte Dreiecksungleichung}%
\index{Dreiecksungleichung!umgekehrte}
In jedem normierten Raum gilt
\[|\|x\|-\|y\|| \le \|x-y\|.\]
\end{Satz}
\begin{Beweis}
Auf beiden Seiten von Def. \ref{def:normed-space} (normed-space)
Axiom (N3) wird $\|y\|$ subtrahiert.
Es ergibt sich
\[\|x+y\| - \|y\| \le \|x\|.\]
Substitution $x:=x-y$ bringt nun
\[\|x\| - \|y\| \le \|x-y\|.\]
Vertauscht man nun $x$ und $y$, dann ergibt sich
\[\|y\|-\|x\| \le \|y-x\| \iff -(\|x\|-\|y\|)\le \|x-y\|.\]
Wir haben nun $a\le b$ und $-a\le b$,
wobei $a:=\|x\|-\|y\|$ und $b:=\|x-y\|$ ist. Multipliziert
man die letzte Ungleichung mit $-1$, dann ergibt sich $a\ge -b$.
Somit ist $-b\le a\le b$, kurz $|a|\le b$.\,\qedsymbol
\end{Beweis}

\subsection{Homöomorphien}
\begin{Satz}[Verallgemeinerung des Zwischenwertsatzes]%
\label{intermediate-value-general}\mbox{}\\*
Ist $f\colon X\to Y$ eine stetige Abbildung zwischen topologischen
Räumen und $A\subseteq X$ ein zusammenhängender Teilraum,
dann ist auch $f(A)$ zusammenhängend.
\end{Satz}

\begin{Satz}
Eine injektive Abbildung $f\colon\R_{\ge 0}\to\R$ kann nicht stetig sein.
\end{Satz}
\begin{Beweis}
Da $f$ injektiv ist, ist die Rechnung
\[f(\R_{>0}) = f(\R_{\ge 0}\setminus\{0\})
= f(\R_{\ge 0})\setminus f(\{0\}) = \R\setminus\{f(0)\}\]
gültig gemäß Satz \ref{inj-img-setminus}. Da $\R_{>0}$ zusammenhängend
ist, $\R\setminus\{f(0)\}$ aber nicht, kann $f$ laut Satz
\ref{intermediate-value-general} nicht stetig sein.\;\qedsymbol
\end{Beweis}
