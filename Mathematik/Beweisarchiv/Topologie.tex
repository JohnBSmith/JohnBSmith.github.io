
\chapter{Topologie}
\section{Grundbegriffe}
\subsection{Definitionen}

\begin{Definition}[Topologischer Raum]
Sei $X$ eine Menge und $T$ eine Menge von Teilmengen von $X$.
Man nennt das System $T$ eine Topologie und $(X,T)$ einen topologischen
Raum, falls die folgenden drei Axiome erfüllt sind:
\begin{enumerate}
\item Es gilt $\emptyset\in T$ und $X\in T$.
\item Sind $A,B\in T$, dann ist auch $A\cap B\in T$.
\item Sind die $A_i\in T$, dann ist auch $\bigcup_I A_i\in T$, wobei $I$ unendlich sein darf.
\end{enumerate}
Die Elemente der Topologie nennt man offene Mengen.
\end{Definition}

\begin{Definition}[Abgeschlossene Menge]
Sei $X$ ein topologischer Raum. Eine Menge $M\subseteq X$
nennt man abgeschlossen, wenn das Komplement $X\setminus M$ offen ist.
\end{Definition}

\begin{Definition}[nh-filter: Umgebungsfilter]%
\label{def:nh-filter}\index{Umgebungsfilter}
Zu einem Punkt $x\in X$ ist
\[\underline U(x) := \{U{\subseteq}X\mid
\exists O\colon O\in T\land x\in O\land O\subseteq U\}\]
der Umgebungsfilter. Eine Menge $U\in\underline U(x)$ heißt
Umgebung von $x$.
\end{Definition}

\begin{Definition}[int: Inneres]\newlinefirst
\label{def:int}\index{offener Kern}%
Das Innere von $M$, auch offener Kern genannt, ist
\[\operatorname{int}(M) := \{x\in M\mid M\in \underline U(x)\}.\]
\end{Definition}

\begin{Definition}[ext: Äußeres]\newlinefirst
Sei $X$ ein topologischer Raum und $M\subseteq X$. Das Äußere von $M$ ist
\[\operatorname{ext}(M) := \operatorname{int}(X\setminus M) = \operatorname{int}(M^\comp).\]
\end{Definition}

\begin{Definition}[Abgeschlossene Hülle]\newlinefirst
Sei $X$ ein topologischer Raum und $M\subseteq X$. Die abgeschlossene Hülle von $M$ ist%
\[\overline M := X\setminus\operatorname{ext}(M) = \operatorname{int}(M^\comp)^\comp.\]
\end{Definition}

\begin{Definition}[Rand]
Der Rand einer Menge $M$ ist
\[\partial M := \overline M\setminus\operatorname{int}(M)
= \overline M\cap\operatorname{int}(M)^\comp.\]
\end{Definition}

\begin{Definition}[Teilraumtopologie]\newlinefirst
Sei $(X,T)$ ein topologischer Raum und $M\subseteq X$. Man bezeichnet
\[T|M := \{A\cap M\mid A\in T\}.\]
als Teilraumtopologie und $(M,T|M)$ als Teilraum.
\end{Definition}

\begin{Definition}[Diskrete Topologie]\newlinefirst
Man sagt, ein topologischer Raum $(X,T)$ habe die diskrete Topologie $T$,
wenn $T$ die Potenzmenge von $X$ ist -- wenn also jede Teilmenge
von $X$ offen ist.
\end{Definition}

\subsection{Stetige Abbildungen}

\begin{Definition}[Stetige Abbildung]\newlinefirst
Seien $X$ und $Y$ topologischen Räume. Eine Abbildung $f\colon X\to Y$
heißt stetig, wenn unter ihr das Urbild einer offenen Menge stets
wieder offen ist.
\end{Definition}

\begin{Satz}
Sei $M\subseteq X$. Ist $f\colon X\to Y$ stetig, so ist
die Einschränkung $f|_M$ stetig.
\end{Satz}
\begin{Beweis}
Vorgelegt ist ein offenes $B\subseteq Y$. Laut Prämisse ist $A=f^{-1}(B)$
ebenfalls offen. Per Definition der Einschränkung gilt
\[(f|_M)^{-1}(B) = M\cap f^{-1}(B) = M\cap A.\]
Laut Definition ist $M\cap A$ ein Element der Teilraumtopologie $T(X)|M$.\,\qedsymbol
\end{Beweis}

\begin{Satz}
Auf jedem topologischen Raum ist die identische Abbildung stetig.
\end{Satz}
\begin{Beweis}
Ihre Urbildoperation ist ebenfalls eine identische Abbildung. Ergo
verbleiben offene Mengen unverändert, also offen.\,\qedsymbol
\end{Beweis}

\begin{Satz}
Die Verkettung stetiger Abbildungen ist stetig.
\end{Satz}
\begin{Beweis}
Seien $f\colon X\to Y$ und $g\colon Y\to Z$ stetig.
Gemäß Satz \ref{preimg-chain} gilt $(g\circ f)^{-1}(M)=f^{-1}(g^{-1}(M))$.
Ist $M$ offen, so auch $g^{-1}(M)$ und infolge $f^{-1}(g^{-1}(M))$.
Ergo ist $g\circ f$ stetig.\,\qedsymbol
\end{Beweis}

\begin{Definition}[Homöomorphismus]\newlinefirst
Seien $X$ und $Y$ topologische Räume. Eine bijektive Abbildung
$f\colon X\to Y$ heißt Homöomorphismus, wenn sowohl $f$ als auch
$f^{-1}$ stetig sind.
\end{Definition}

\newpage
\subsection{Elementares}

\begin{Satz}\label{partition-int-bd-ext}
Sei $X$ ein topologischer Raum. Für jede Menge $M\subseteq X$ ist%
\[X = \operatorname{int}(M)\cup\partial M\cup\operatorname{ext}(M)\]
eine disjunkte Zerlegung.
\end{Satz}
\begin{Beweis} Sei $A:=\operatorname{int}(M)$ und $B:=\operatorname{ext}(M)$. Dann ist
\begin{gather*}
\operatorname{int}(M)\cup\partial M\cup\operatorname{ext}(M)
= A\cup B^\comp\cap A^\comp\cup B
= A\cup A^\comp\cup B = X\cup B = X.
\end{gather*}
Nun verbleibt zu prüfen, dass die Mengen paarweise disjunkt sind. Wir haben%
\begin{align*}
\operatorname{int}(M)\cap\partial M &= A\cap B^\comp\cap A^\comp = \emptyset,\\
\operatorname{ext}(M)\cap\partial M &= B\cap B^\comp\cap A^\comp = \emptyset.
\end{align*}
Wegen $M\cap M^\comp=\emptyset$ ist erst recht $A\cap B=\emptyset$,
denn $A\subseteq M$ und $B\subseteq M^\comp$.\,\qedsymbol
\end{Beweis}

\begin{Satz}
Für jede Menge $A\subseteq X$ gilt $\operatorname{int}(A)^\comp\cup A = X$.
\end{Satz}
\begin{Beweis}
Setze $B:=\operatorname{int}(A)$. Gemäß Definition gilt $B\subseteq A$,
was äquivalent zu $A\cap B = B$ ist. Damit ergibt sich
\[B^\comp\cup A = (A\cap B)^\comp\cup A = A^\comp\cup B^\comp\cup A
= X\cup B^\comp = X.\,\qedsymbol\]
\end{Beweis}

\begin{Satz}
Das Innere von $M$ ist die Vereinigung der offenen Teilmengen
von $M$, kurz%
\[\operatorname{int}(M) = \bigcup_{O\in 2^M\cap T} O.\]
\end{Satz}

\begin{Beweis}
Nach Def. \ref{def:seteq} (seteq) und Def. \ref{def:int} (int)
expandieren:%
\[\forall x\colon [x\in M\land M\in\underline U(x)
\iff x\in\bigcup_{O\in 2^M\cap T} O].\]
Den äußeren Allquantor brauchen wir nicht weiter mitschreiben, da alle
freien Variablen automatisch allquantifiziert werden.
Nach Def. \ref{def:nh-filter} (nh-filter) weiter expandieren, wobei die
Bedingung $U\subseteq X$ als tautologisch entfallen kann,
weil $X$ die Grundmenge ist. Auf der rechten Seite wird nach Def.
\ref{def:union} (union) expandiert. Es ergibt sich:
\[x\in M\land (\exists O\colon O\in T\land x\in O\land O\subseteq M)
\iff (\exists O\colon O\subseteq M\land O\in T\land x\in O).\]
Wegen $A\land(\exists x\colon P(x))\iff (\exists x\colon A\land P(x))$ ergibt
sich auf der linken Seite:
\[\exists O\colon x\in M\land O\in T\land x\in O\land O\subseteq M.\]
Wenn aber $O\subseteq M$ erfüllt sein muss, gilt
$x\in O\implies x\in M$. Demnach kann $x\in M$ entfallen.
Auf beiden Seiten steht dann die gleiche Bedingung.\,\qedsymbol
\end{Beweis}

\begin{Satz}\label{boundary-point-char}
Ein Punkt $p$ liegt genau dann auf dem Rand einer Menge $M$, wenn
jede Umgebung von $p$ mindestens einen Punkt aus $M$ und
einen Punkt aus dem Komplement von $M$ enthält.
\end{Satz}

\newpage
\subsection{Zusammenhang}

\begin{Definition}[Zusammenhängender Raum]\newlinefirst
Ein topologischer Raum $(X,T)$ heißt zusammenhängend, wenn er sich
nicht in zwei disjunkte nichtleere offene Mengen zerlegen lässt. Gemeint ist
\[\forall A,B\in T\colon A\ne\emptyset\land B\ne\emptyset\land A\cap B = \emptyset \implies A\cup B\ne X.\]
\end{Definition}
\strong{Bemerkung.} Ein Raum $(X,T)$ ist demnach unzusammenhängend, wenn
Zeugen $A,B$ für die Aussage
\[\exists A,B\in T\colon A\ne\emptyset\land B\ne\emptyset\land A\cap B = \emptyset \land A\cup B = X\]
gefunden sind.

\begin{Satz}\label{connected-iff-cont-is-const}
Sei $X$ ein topologischer Raum und $\{0,1\}$ der topologische
Raum mit der diskreten Topologie. Es ist $X$ genau dann
zusammenhängend, wenn jede stetige Abbildung $f\colon X\to\{0,1\}$
konstant sein muss.
\end{Satz}
\begin{Beweis}
Sei $f$ stetig und nicht"=konstant. Man betrachte die Fasern
$A:=f^{-1}(\{0\})$ und $B:=f^{-1}(\{1\})$. Weil $f$, wie gerade
gefordert, beide Werte annehmen muss, gilt $A\ne\emptyset$ und
$B\ne\emptyset$. Zudem sind $A,B$ offen, weil sie die Urbilder
offener Mengen sind unter stetigem $f$ sind. Urbilder disjunkter
Mengen sind immer disjunkt, womit $A\cap B=\emptyset$ gilt. Zudem gilt
\[A\cup B = f^{-1}(\{0\})\cup f^{-1}(\{1\}) = f^{-1}(\{0\}\cup \{1\}) = X.\]
Somit ist ein Gegenbeispiel konstruiert, so dass $X$ unzusammenhängend
sein muss.

Sei $X$ unzusammenhängend. Es existieren somit Zeugen $A,B\in T$
mit $A\ne\emptyset$, $B\ne\emptyset$, $A\cup B=\emptyset$ und
$A\cup B = X$. Sei $f$ definiert durch
$f(x):=0$ für alle $x\in A$ und $f(x):=1$ für alle $x\in B$.
Nun ist $f$ keine konstante Abbildung, da sie auf nichtleeren $A,B$
unterschiedliche Werte annimmt. Wohl aber ist $f$ stetig,
wie im Folgenden noch durchgerechnet wird. Die diskrete Topologie
von $\{0,1\}$ ist ihre Potenzmenge. Es bestätigt sich
\begin{gather*}
f^{-1}(\emptyset) = \emptyset\in T,\\
f^{-1}(\{0\}) = A\in T,\\
f^{-1}(\{1\}) = B\in T,\\
f^{-1}(\{0,1\}) = f^{-1}(\{0\}\cup\{1\}) = f^{-1}(\{0\})\cup f^{-1}(\{1\})
= A\cup B = X\in T.\,\qedsymbol\\
\end{gather*}
\end{Beweis}

\begin{Satz}
Die Vereinigung zweier zusammenhängender nichtdisjunkter Räume
ist ein zusammenhängender Raum.
\end{Satz}
\begin{Beweis}
Seien $X,Y$ zusammenhängend und sei $X\cap Y\ne\emptyset$.
Laut Prämisse existiert mindestens ein $p\in X\cap Y$. Sei
$f\colon X\cup Y\to\{0,1\}$ stetig. Nun ist $f|_X$ stetig und
konstant mit $f(x)=f(p)$ für alle $x\in X$, da $X$ zusammenhängend
ist. Entsprechend ist $f|_Y$ stetig und konstant mit $f(y)=f(p)$
für alle $y\in Y$. Ergo ist $f$ auf ganz $X\cup Y$ konstant $f(p)$.
Laut Satz \ref{connected-iff-cont-is-const} muss $X\cup Y$
also zusammenhängend sein.\,\qedsymbol
\end{Beweis}

\newpage
\begin{Satz}
Ein topologischer Raum $X$ ist genau dann zusammenhängend,
wenn mit Ausnahme von $\emptyset$ und $X$ keine Teilmenge
von $X$ sowohl offen als auch abgeschlossen ist.
\end{Satz}
\begin{Beweis}
Es existiere außer $\emptyset,X$ kein offenes $A$ mit offenem
$X\setminus A$. Sei $f\colon X\to\{0,1\}$ stetig. Angenommen, $f$
wäre nicht konstant. Dann gäbe es die beiden nichtleeren offenen Mengen
$A:=f^{-1}(\{0\})$ und $B:=f^{-1}(\{1\})$. Weil außerdem $X=A\cup B$
eine disjunkte Zerlegung wäre, gälte $B=X\setminus A$. Dies steht
im Widerspurch zur Prämisse, womit $f$ konstant sein muss.
Ergo ist $X$ laut Satz \ref{connected-iff-cont-is-const} ein
zusammenhängender Raum.

Es existiere nun offnes, von $\emptyset,X$ verschiedenes $A$ mit
offenem $B:=X\setminus A$, womit neben $A\ne\emptyset$ auch
$B\ne\emptyset$ gilt. Wegen $A\cap B=\emptyset$ und $A\cup B=X$
gilt $X$ als in zwei nichtleere disjunkte offene Mengen zerlegt.
Ergo ist $X$ unzusammenhängend.\,\qedsymbol
\end{Beweis}

\begin{Satz}
Ist $X$ zusammenhängend und $f\colon X\to Y$ stetig, dann
ist der Teilraum $f(X)$ von $Y$ ebenfalls zusammenhängend.
\end{Satz}
\begin{Beweis}
Sei für die Kontraposition $f(X)$ unzusammenhängend. Hiermit existiert
eine disjunkte Zerlegung in offene nichtleere Mengen $A,B$ mit
$A\cup B=f(X)$. Aufgrund der elementaren Eigenschaften der
Urbildoperation hat man mit
\[X = f^{-1}(A\cup B) = f^{-1}(A)\cup f^{-1}(B),\qquad
f^{-1}(A)\cap f^{-1}(B) = \emptyset\]
eine disjunkte Zerlegung von $X$. Wegen $A\subseteq f(X)$ und
$B\subseteq f(X)$ gilt hierbei $f^{-1}(A)\ne\emptyset$ und
$f^{-1}(B)\ne\emptyset$. Weil $f$ stetig ist, sind $f^{-1}(A)$
und $f^{-1}(B)$ offen. Somit ist für $X$ eine Zerlegung
in zwei nichtleere disjunkte offene Mengen bezeugt. Ergo ist
$X$ ebenfalls unzusammenhängend.\,\qedsymbol
\end{Beweis}


\newpage
\section{Metrische Räume}
\subsection{Metrische Räume}
\begin{Definition}[metric-space: metrischer Raum]%
\index{metrischer Raum}\label{metric-space}
Man bezeichet $(M,d)$ mit $d\colon M^2\to\R$ genau dann als
metrischen Raum, wenn die folgenden Axiome erfüllt sind:
\begin{align*}
\text{(M1)}\quad & d(x,y)=0\iff x=y, &&\text{(Gleichheit abstandsloser Punkte)}\\
\text{(M2)}\quad & d(x,y)=d(y,x), &&\text{(Symmetrie)}\\
\text{(M3)}\quad & d(x,y)\le d(x,z)+d(z,y). &&\text{(Dreiecksungleichung)}
\end{align*}
\end{Definition}

\begin{Definition}[open-ep-ball: offene Epsilon-Umgebung]\newlinefirst
Für einen metrischen Raum $(M,d)$ und $p\in M$:
\[U_\varepsilon(p) := \{x\mid d(p,x)<\varepsilon\}.\]
\end{Definition}
Bemerkung: Unter einer Epsilon-Umgebung ohne weitere Attribute
versteht man immer eine offene Epsilon-Umgebung.

\begin{Satz}[Konstruktion disjunkter Epsilon-Umgebungen]%
\label{construction-disjoint-ep-balls}
Sei $(M,d)$ ein metrischer Raum und $p,q\in M$ mit $p\ne q$.
Betrachte die Streckenzerlegung $d(p,q)=A+B$. Für $a\le A$ und
$b\le B$ sind die Epsilon-Umgebungen $U_a(p)$ und $U_b(q)$ disjunkt.
\end{Satz}

\begin{Beweis}
Angenommen $U_a(p)$ und $U_b(q)$ wären nicht disjunkt, dann gäbe
es mindestens ein $x$ mit $x\in U_a(p)$ und $x\in U_b(q)$, d.\,h.
$d(p,x)<a$ und $d(q,x)<b$. Addition der beiden Ungleichungen
bringt
\[d(p,x)+d(q,x)<a+b\le d(p,q).\]
Gemäß der Dreiecksungleichung Def. \ref{metric-space} Axiom (M3) gilt
nun aber
\[d(p,q)\le d(p,x)+d(q,x)\]
für alle $x$. Sei $c:=d(p,x)+d(q,x)$. Wir erhalten damit nun
$c<a+b\le c$ und somit den Widerspruch $c<c$.\,\qedsymbol
\end{Beweis}

\begin{Satz}[Unterschiedliche Punkte eines metrischen Raumes
besitzen disjunkte Epsilon-Umgebungen]
Sei $(M,d)$ ein metrischer Raum und $p,q\in M$.
Wenn $p\ne q$ ist, dann gibt es disjunkte offene
Epsilon-Umgebungen $U_a(p)$ und $U_b(q)$.
\end{Satz}

\begin{Beweis}
Folgt trivial aus Satz \ref{construction-disjoint-ep-balls}.
Wähle speziell z.\,B. $a=b=d(p,q)/2$.\,\qedsymbol
\end{Beweis}

\begin{Satz}\label{metric-transform}
Sei $d$ eine Metrik. Ist $f\colon\R_{\ge 0}\to\R$ eine streng monoton
steigende und subadditive Funktion mit $f(0)=0$, so ist $f\circ d$
ebenfalls eine Metrik.
\end{Satz}
\begin{Beweis}
Aufgrund der strengen Monotonie ist $f(r)=0\Leftrightarrow r = 0$,
womit $d(x,y)=0\Leftrightarrow f(d(x,y))=0$ gilt. Zur
Dreiecksungleichung findet sich
\[d(x,y) \le d(x,z) + d(z,y)
\stackrel{\text{(1)}}\implies f(d(x,y))\le f(d(x,z) + d(z,y))
\stackrel{\text{(2)}}\le f(d(x,z)) + f(d(z,y)),\]
wobei (1) mit der Monotonie und (2) mit der Subadditivität
gilt. Man erhält
\[(f\circ d)(x,y)\le (f\circ d)(x,z) + (f\circ d)(z,y).\,\qedsymbol\]
\end{Beweis}

\begin{Satz}
Zu einer Metrik $d$ ist
\[d_b(x,y) := \frac{d(x,y)}{1+d(x,y)}\]
ebenfalls eine Metrik, die man auch als beschränkte Metrik bezeichnet.
\end{Satz}
\begin{Beweis}
Es gilt $d_b = f\circ d$ bezüglich $f(r) = \frac{r}{1+r}$.
Nun wird Satz \ref{metric-transform} zur Anwendung gebracht, dessen
Voraussetzungen zu prüfen sind. Der Umstand
$f(0)=0$ ist trivial. Es ist $f$ streng monoton steigend, denn
die Ableitung $f'(r)$ ist stetig, hat positive Werte und besitzt keine
Nullstellen. Nun wird die Subadditivität $f(x+y)\le f(x)+f(y)$ gezeigt.
Wegen $x\ge 0$ und $y\ge 0$ gilt $1+x>0$ und $1+y>0$ und
$1+x+y>0$, womit sich die äquivalente Umformung
\[\frac{x+y}{1+x+y} \le \frac{x}{1+x} + \frac{y}{1+y}
\Leftrightarrow (x+y)(1+x)(1+y) \le ((1+y)x + (1+x)y)(1+x+y)\]
findet. Multipliziert man beide Seiten aus und kürzt, verbleibt
\[0\le xy^2 + x^2 y + 2xy = (x+y+2)xy\]
zu zeigen. Diese ist erfüllt wegen $x+y+2>0$ zuzüglich $x\ge 0$ und
$y\ge 0$.\,\qedsymbol
\end{Beweis}

\newpage
\subsection{Normierte Räume}
\begin{Definition}[normed-space: normierter Raum]%
\label{def:normed-space}\index{normierter Raum}\index{Dreiecksungleichung}
Sei $V$ ein Vektorraum über dem Körper der rellen oder komplexen
Zahlen. Sei $N(x)=\|x\|$ eine Abbildung, die jedem $x\in V$ eine
reelle Zahl zuordnet. Man nennt $(V,N)$ genau dann einen
normierten Raum, wenn die folgenden Axiome erfüllt sind:
\begin{align*}
\text{(N1)}\quad &\|x\|=0\iff x=0,&&\text{(Definitheit)}\\
\text{(N2)}\quad &\|\lambda x\|=|\lambda|\|x\|,&&\text{(betragsmäßige Homogenität)}\\
\text{(N3)}\quad &\|x+y\| \le \|x\|+\|y\|.&&\text{(Dreiecksungleichung)}
\end{align*}
\end{Definition}

\begin{Satz}[umgekehrte Dreiecksungleichung]%
\label{rev-tineq}\index{umgekehrte Dreiecksungleichung}%
\index{Dreiecksungleichung!umgekehrte}
In jedem normierten Raum gilt
\[|\|x\|-\|y\|| \le \|x-y\|.\]
\end{Satz}
\begin{Beweis}
Auf beiden Seiten von Def. \ref{def:normed-space} (normed-space)
Axiom (N3) wird $\|y\|$ subtrahiert.
Es ergibt sich
\[\|x+y\| - \|y\| \le \|x\|.\]
Substitution $x:=x-y$ bringt nun
\[\|x\| - \|y\| \le \|x-y\|.\]
Vertauscht man nun $x$ und $y$, dann ergibt sich
\[\|y\|-\|x\| \le \|y-x\| \iff -(\|x\|-\|y\|)\le \|x-y\|.\]
Wir haben nun $a\le b$ und $-a\le b$,
wobei $a:=\|x\|-\|y\|$ und $b:=\|x-y\|$ ist. Multipliziert
man die letzte Ungleichung mit $-1$, dann ergibt sich $a\ge -b$.
Somit ist $-b\le a\le b$, kurz $|a|\le b$.\,\qedsymbol
\end{Beweis}

\subsection{Homöomorphien}
\begin{Satz}[Verallgemeinerung des Zwischenwertsatzes]%
\label{intermediate-value-general}\newlinefirst
Ist $f\colon X\to Y$ eine stetige Abbildung zwischen topologischen
Räumen und $A\subseteq X$ ein zusammenhängender Teilraum,
dann ist auch $f(A)$ zusammenhängend.
\end{Satz}

\begin{Satz}
Eine injektive Abbildung $f\colon\R_{\ge 0}\to\R$ kann nicht stetig sein.
\end{Satz}
\begin{Beweis}
Da $f$ injektiv ist, ist die Rechnung
\[f(\R_{>0}) = f(\R_{\ge 0}\setminus\{0\})
= f(\R_{\ge 0})\setminus f(\{0\}) = \R\setminus\{f(0)\}\]
gültig gemäß Satz \ref{inj-img-setminus}. Da $\R_{>0}$ zusammenhängend
ist, $\R\setminus\{f(0)\}$ aber nicht, kann $f$ laut Satz
\ref{intermediate-value-general} nicht stetig sein.\,\qedsymbol
\end{Beweis}

\newpage
\section{Übungen}

\begin{Satz}
Sei $M\subseteq\R^n$. Das Innere von $M$ besitze den Punkt $p$,
das Äußere den Punkt $q$. Dann schneidet das Bild jedes Weges von $p$
nach $q$ den Rand von $M$.
\end{Satz}
\begin{Beweis}[Beweis 1]
Sei $\gamma\colon [0,1]\to\R^n$ so ein Weg mit
$\gamma(0)=p$ und $\gamma(1)=q$. Angenommen, das Bild schneidet
den Rand nicht. Das heißt, $\gamma([0,1])\cap\partial M = \emptyset$,
oder äquivalent $\gamma^{-1}(\partial M)=\emptyset$.
Allgemein ist
\[\R^n = \operatorname{int}(M) \cup \partial M \cup \operatorname{ext}(M)\]
laut Satz \ref{partition-int-bd-ext} eine disjunkte Zerlegung.
Gemäß Satz \ref{preimg-dl} (preimg-dl) gilt
\[[0,1] = \gamma^{-1}(\R^n) = \gamma^{-1}(\operatorname{int}(M))
\cup \gamma^{-1}(\partial M)\cup \gamma^{-1}(\operatorname{ext}(M)),\]
was auch eine disjunkte Zerlegung ist, weil je zwei disjunkte Mengen
gemäß Satz \ref{disjoint-preimg} disjunkte Urbilder haben.
Weil $\gamma$ gemäß Definition stetig ist,
sind die Urbilder $\gamma^{-1}(\operatorname{int}(M))$ und
$\gamma^{-1}(\operatorname{ext}(M))$ offen im Raum $[0,1]$. Sie sind
nichtleer, weil sie jeweils laut Prämisse mindestens einen Punkt
enthalten. Damit ist $[0,1]$ eine Zerlegung in disjunkte nichtleere
offene Mengen, gemäß Definition also ein unzusammhängender Raum. Das
steht im Widerspruch zur Erkenntnis, dass alle Intervalle
zusammenhängend sind.\,\qedsymbol
\end{Beweis}

\begin{Beweis}[Beweis 2]
Sei $\gamma\colon [0,1]\to\R^n$ ein solcher Weg mit $\gamma(0)=p$ und
$\gamma(1)=q$. Wir nehmen nun eine Bisektion vor. Sei $a_0:=0$ und
$b_0:=1$. Sei $m:=\tfrac{1}{2}(a_{k}+b_{k})$, also der Mittelwert. Liegt
$\gamma(m)$ im Inneren, dann ist $[a_{k+1},b_{k+1}]=[m,b_k]$ das nächste
Intervall. Liegt $\gamma(m)$ im Äußeren, dann $[a_{k+1},b_{k+1}]=[a_k,m]$.
Liegt $\gamma(m)$ auf dem Rand, ist ein Schnittpunkt gefunden und
das Verfahren bricht ab. Betrachten wir daher
den Fall, dass das Verfahren nicht abbricht. Als
Intervallschachtelung konvergieren die Folgen $a_k,b_k$ gegen
denselben Grenzwert $a$. Weil $\gamma$ stetig ist, konvergiert
$\gamma(a_k)\to \gamma(a)$ für $a_k\to a$ und $\gamma(b_k)\to\gamma(a)$
für $b_k\to a$. Demnach sind in jeder Umgebung von $\gamma(a)$ sowohl
Punkte aus dem Inneren als auch Punkte aus dem Äußeren. Gemäß
Satz \ref{boundary-point-char} muss $\gamma(a)$ infolge
auf dem Rand liegen.\,\qedsymbol
\end{Beweis}
