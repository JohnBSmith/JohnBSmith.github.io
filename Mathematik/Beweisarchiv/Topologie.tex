
\chapter{Topologie}
\section{Grundbegriffe}
\subsection{Definitionen}

\begin{Definition}[nhfilter: Umgebungsfilter]\label{def:nhfilter}
\[\underline U(x) := \{U{\subseteq}X\mid
\exists O(O\in T\land x\in O\land O\subseteq U)\}.\]
\end{Definition}

\begin{Definition}[int: Offener Kern]\label{def:int}
\[\operatorname{int}(M) := \{x\in M\mid M\in \underline U(x)\}\]
\end{Definition}

\begin{Satz}
Der offene Kern von $M$ ist die Vereinigung der offenen Teilmengen
von $M$. Kurz:%
\[\operatorname{int}(M) = \bigcup_{O\in 2^M\cap T} O.\]
\end{Satz}

\begin{Beweis}
Nach Def. \ref{def:seteq} (seteq) und Def. \ref{def:int} (int)
expandieren:
\[\forall x[x\in M\land M\in\underline U(x)
\iff x\in\bigcup_{O\in 2^M\cap T} O].\]
Den äußeren Allquantor brauchen wir nicht weiter mitschreiben, da alle
freien Variablen automatisch allquantifiziert werden.
Nach Def. \ref{def:nhfilter} (nhfilter) weiter expandieren, wobei die
Bedingung $U\subseteq X$ als tautologisch entfallen kann,
weil $X$ die Grundmenge ist. Auf der rechten Seite wird nach Def.
\ref{def:union} (union) expandiert. Es ergibt sich:
\[x\in M\land \exists O(O\in T\land x\in O\land O\subseteq M)
\iff \exists O(O\subseteq M\land O\in T\land x\in O).\]
Wegen $A\land\exists x(P(x))\iff \exists x(A\land P(x))$ ergibt
sich auf der linken Seite:
\[\exists O(x\in M\land O\in T\land x\in O\land O\subseteq M).\]
Wenn aber $O\subseteq M$ erfüllt sein muss, gilt
$x\in O\implies x\in M$. Demnach kann $x\in M$ entfallen.
Auf beiden Seiten steht dann die gleiche Bedingung.\,\qedsymbol
\end{Beweis}

\section{Metrische Räume}
\subsection{Metrischer Räume}
\begin{Definition}[metric-space: metrischer Raum]
Man bezeichet $(M,d)$ mit $d\colon M^2\to\R$ genau dann als
metrischen Raum, wenn die folgenden Axiome erfüllt sind:
\begin{gather*}
\text{(M1)}\quad d(x,y)=0\iff x=y,\\
\text{(M2)}\quad d(x,y)=d(y,x),\\
\text{(M3)}\quad d(x,y)\le d(x,z)+d(z,y).
\end{gather*}
\end{Definition}

\subsection{Normierte Räume}
\begin{Definition}[normed-space: normierter Raum]\label{def:normed-space}
Sei $V$ ein Vektorraum über dem Körper der rellen oder komplexen
Zahlen. Sei $N(x)=\|x\|$ eine Abbildung, die jedem $x\in V$ eine
reelle Zahl zuordnet. Man nennt $(V,N)$ genau dann einen
normierten Raum, wenn die folgenden Axiome erfüllt sind:
\begin{align*}
\text{(N1)}\quad &\|x\|=0\iff x=0,&&\text{(Definitheit)}\\
\text{(N2)}\quad &\|\lambda x\|=|\lambda|\|x\|,&&\text{(betragsmäßige Homogenität)}\\
\text{(N3)}\quad &\|x+y\| \le \|x\|+\|y\|.&&\text{(Dreiecksungleichung)}
\end{align*}
\end{Definition}

\begin{Satz}[umgekehrte Dreiecksungleichung]\label{rev-tineq}
In jedem normierten Raum gilt
\[|\|x\|-\|y\|| \le \|x-y\|.\]
\end{Satz}
\begin{Beweis}
Auf beiden Seiten von Def. \ref{def:normed-space} (normed-space)
Axiom (N3) wird $\|y\|$ subtrahiert.
Es ergibt sich
\[\|x+y\| - \|y\| \le \|x\|.\]
Substitution $x:=x-y$ bringt nun
\[\|x\| - \|y\| \le \|x-y\|.\]
Vertauscht man nun $x$ und $y$, dann ergibt sich
\[\|y\|-\|x\| \le \|y-x\| \iff -(\|x\|-\|y\|)\le \|x-y\|.\]
Wir haben nun die Situation $a\le b$ und $-a\le b$. Multipliziert
man die letzte Ungleichung mit $-1$, dann ergibt sich $a\ge -b$.
Somit ist $-b\le a\le b$, kurz $|a|\le b$.\,\qedsymbol
\end{Beweis}


