\documentclass{beamer}
\usetheme{Antibes}
\useinnertheme{rectangles}
\useoutertheme{infolines}
\usepackage[utf8]{inputenc}
\usepackage[T1]{fontenc}
\usepackage[ngerman]{babel}

% patch the look of +, = in arev
\usefonttheme{serif} 

\usepackage{arev}
\usepackage{amsmath}
\usepackage{amssymb}

\setbeamertemplate{footline}{%
\begin{beamercolorbox}[ht=3.0ex,dp=1ex]{title in head/foot}
\hfill\footnotesize\insertpagenumber\enspace\enspace\end{beamercolorbox}}

\definecolor{brown1}{rgb}{0.26,0.14,0}
\definecolor{brown2}{rgb}{0.20,0.10,0}
\setbeamercolor*{palette primary}{fg=white,bg=brown1}
\setbeamercolor*{palette secondary}{fg=white,bg=brown2}
\setbeamercolor*{palette tertiary}{fg=white,bg=brown2}
\setbeamercolor{itemize item}{fg=black}
\newcommand{\modest}[1]{{\small\color{gray}#1}}

\newcommand{\ee}{\mathrm e}
\newcommand{\ui}{\mathrm i}
\newcommand{\real}{\operatorname{Re}}
\newcommand{\imag}{\operatorname{Im}}
\newcommand{\uv}[1]{\underline{#1}}
\newcommand{\bv}[1]{\mathbf{#1}}

\newcommand{\N}{\mathbb N}
\newcommand{\Z}{\mathbb Z}
\newcommand{\Q}{\mathbb Q}
\newcommand{\R}{\mathbb R}
\newcommand{\C}{\mathbb C}

\newcommand{\id}{\operatorname{id}}
\newcommand{\sgn}{\operatorname{sgn}}
\newcommand{\Abb}{\operatorname{Abb}}
\newcommand{\unit}[1]{\mathrm{#1}}
\newcommand{\chem}[1]{\mathrm{#1}}
\newcommand{\strong}[1]{\textsf{\textbf{#1}}}
\newcommand{\defiff}{\quad:\Longleftrightarrow\quad}

\title{Was ist Ableiten?}
\date{}

\begin{document}

\begin{frame}
\maketitle
\end{frame}

\begin{frame}
Betrachten wir eine beliebige reelle Funktion $f$ an einer beliebigen
Stelle $x_0$. Betrachten wir den Graph am Punkt $(x_0,f(x_0))$
unter einer Lupe oder einem Mikroskop.
\end{frame}

\begin{frame}
Bei vielen Funktionen schaut der Graph unter hinreichend starker
Vergrößerung aus wie eine Gerade. Nehmen wir einmal an, das ist
auch bei $f$ der Fall.
\end{frame}

\begin{frame}
Sei $h$ eine hinreichend kleine Zahl, also $h\approx 0$, aber $h\ne 0$.
Nun ist $f(x_0+h)$ näherungsweise als lineare Funktion in $h$
beschreibbar, also $f(x_0+h)\approx f(x_0)+mh$, wobei $m$ ein
unbekannter Anstieg ist.
\end{frame}

\begin{frame}
Den Wert $f(x_0+h)$ können wir allerdings ausrechnen, da wir $f$
ja vorliegen haben. Umformung der Gleichung nach $m$ ergibt
\[m \approx \frac{f(x_0+h)-f(x_0)}{h}.\]
\end{frame}

\begin{frame}
Das muss umso genauer werden, je kleiner $h$ ist. Das bringt uns
auf die folgende Idee.
\end{frame}

\begin{frame}
\strong{Defintion.} Eine Funktion $f$ heißt \emph{differenzierbar}
an der Stelle $x_0$, wenn der Grenzwert
\[Df(x_0) = \lim_{h\to 0}\frac{f(x_0+h)-f(x_0)}{h}\]
existiert.
\end{frame}

\begin{frame}
Geometrische Interpretation: $Df(x_0)$ ist der Anstieg der Tangente
von $f$ an der Stelle $x_0$.
\end{frame}

\begin{frame}
Wozu braucht man das?
\end{frame}

\begin{frame}
Nun, hat eine differenzierbare Funktion $f$ an einer Stelle $x_0$ ein
Extremum, dann muss dort eine waagerechte Tangente vorliegen.

\vspace{1em}
Umgekehrt können wir also durch Lösen der Gleichung $Df(x)=0$ nach
solchen Stellen fischen. Zwar müssen nicht alle Lösungen auch
Extremstellen sein, aber es kann keine Extremstelle geben, die nicht
Lösung dieser Gleichung ist. {\footnotesize
--- Ausgenommen davon sind die Randstellen des Definitionsbereichs.}
\end{frame}

\begin{frame}
Solche Extremwertaufgaben, auch Optimierungsprobleme genannt,
sind für
\begin{itemize}
\item die Mathematik,
\item alle Naturwissenschaften,
\item Ingenieurswissenschaften und
\item die Wirtschaftslehre
\end{itemize}
von \emph{grundlegender} Bedeutung.
\end{frame}

\begin{frame}
Außerdem ist die Differentialrechnung mit vielen Konzepten der
Analysis eng verwoben.
\end{frame}

\begin{frame}
Ende.
\vfill\hfill
\modest{Creative Commons CC0}
\end{frame}


\end{document}


