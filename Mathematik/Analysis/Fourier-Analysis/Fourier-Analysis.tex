\documentclass[a4paper,11pt,fleqn,twoside,dvipdfmx]{article}
\usepackage[utf8]{inputenc}
\usepackage[T1]{fontenc}
\usepackage[ngerman]{babel}
\usepackage{amsmath}
\usepackage{amssymb}
\usepackage{amsthm}

\iffalse
  \usepackage{lmodern}
\else
% \renewcommand\familydefault{\sfdefault}
% \usepackage{sfmath}
% \usepackage[scaled=.87]{helvet}
% \usepackage[charter]{mathdesign}
  \usepackage[libertine,cmintegrals]{newtxmath}
  \usepackage{libertine}
  \renewcommand\ttdefault{lmtt}
\fi

% == patch: cmintegrals for mathdesign ==
% \DeclareSymbolFont{rmlargesymbols}{OMX}{cmex}{m}{n}
% \DeclareMathSymbol{\rmintop}{\mathop}{rmlargesymbols}{82}
% \renewcommand{\int}{\rmintop\nolimits}

\usepackage{color}
\definecolor{c1}{RGB}{0,40,80}
\usepackage[colorlinks=true,linkcolor=c1]{hyperref}
\usepackage{geometry}
\geometry{a4paper,left=36mm,right=30mm,top=26mm,bottom=34mm}

\newcommand{\ee}{\mathrm e}
\newcommand{\ui}{\mathrm i}
\newcommand{\real}{\operatorname{Re}}
\newcommand{\imag}{\operatorname{Im}}

\newcommand{\N}{\mathbb N}
\newcommand{\Z}{\mathbb Z}
\newcommand{\R}{\mathbb R}
\newcommand{\C}{\mathbb C}
\newcommand{\emdef}[1]{\emph{#1}}

\theoremstyle{definition}
\newtheorem{Definition}{Definition}

\theoremstyle{theorem}
\newtheorem{Satz}{Satz}
\newtheorem{Korollar}[Satz]{Koroallar}

\numberwithin{equation}{section}

\begin{document}
\thispagestyle{empty}

\vglue 2em
{\center\LARGE\bfseries Fourier-Analysis\par}

\tableofcontents

\section{Vorbereitungen}
\subsection{Periodische Funktionen}

\begin{Definition}
Eine Funktion $f\colon\R\to\R$ heißt \emdef{periodisch}, wenn es eine
Zahl $P\ne 0$ gibt, so dass
\begin{equation}\label{eq:periodisch}
\forall x\in\R\colon\, f(x+P)=f(x)
\end{equation}
gilt. Eine solche Zahl $P$ wird \emdef{Periode} oder
\emdef{Periodenlänge} genannt.
\end{Definition}

\begin{Korollar}
Wenn eine Funktion die Periode $P$ besitzt, so hat sie auch die
Perioden $kP$ für jede ganze Zahl $k\ne 0$.
\end{Korollar}
\noindent
\textbf{Beweis.} Für $k=2$ ist
\begin{equation}
f(x+2P) = f(x+P+P) \stackrel{\eqref{eq:periodisch}}=
f(x+P) \stackrel{\eqref{eq:periodisch}}= f(x).
\end{equation}
Für $k=-1$ ist außerdem
\begin{equation}
f(x-P) \stackrel{\eqref{eq:periodisch}}= f(x-P+P) = f(x).
\end{equation}
Und jetzt noch einmal allgemein. Für $n\ge 0$ ergibt sich der
Induktionsschritt
\begin{equation}
f(x+(n+1)P) = f(x+nP+P) \stackrel{\eqref{eq:periodisch}}= f(x+nP).
\end{equation}
Für $n\ge 0$ ergibt sich außerdem der Induktionsschritt
\begin{equation}
f(x-(n+1)P) = f(x-nP-P) \stackrel{\eqref{eq:periodisch}}=
f(x-nP-P+P) = f(x-nP).
\end{equation}
Per Induktion ergibt sich nun die Gleichung $f(x+kP) = f(x)$. $\Box$

\begin{Korollar}
Ist eine $P$-periodische Funktion $f\colon\R\to\R$ auf jedem
kompakten Intervall integrierbar, so gilt
\begin{equation}
\int_0^P f(x)\,\mathrm dx = \int_a^{a+P} f(x)\,\mathrm dx.
\end{equation}
\end{Korollar}
\noindent
\textbf{Beweis.} Verwende die Substitutionsregel:
\begin{equation}
\int_{x=g(0)}^{x=g(a)} f(x)\,\mathrm dx = \int_{u=0}^{u=a}
f(g(u))g'(u)\,\mathrm du
\end{equation}
mit $x=g(u)=u+P$.

Das Integral wird aufgetrennt:
\begin{equation}
\begin{split}
\int_{a}^{a+P}f(x)\,\mathrm dx
=\int_{a}^{P}f(x)\,\mathrm dx+&\underbrace{\int_{P}^{a+P}f(x)\,\mathrm dx}.\\
&\int_0^a f(u+P)\,\mathrm du
\stackrel{\eqref{eq:periodisch}}{=} \int_0^a f(u)\,\mathrm du
\end{split}
\end{equation}
Nun wird es wieder zusammengeführt:
\begin{equation}
\int_0^a f(x)\,\mathrm dx+\int_a^P f(x)\,\mathrm dx
= \int_0^P f(x)\,\mathrm dx.\;\Box
\end{equation}
Oft setzt man $P:=2\pi$ und $a:=-\pi$.

\subsection{Signale}
Eine Funktion $f\colon\R\to\R$ lässt sich auch als ein Signal
interpretieren. Darunter versteht man einfach, dass der
Definitionsbereich von $f$ die Zeit ist. Man schreibt nun $t:=x$,
wobei $t$ für \emph{time} steht. Außerdem schreibt man $T:=P$ und
spricht von \emdef{Periodendauer} anstelle von Periodenlänge.
Die Abkürzung
\begin{equation}\label{eq:Kreisfrequenz}
\omega:=\frac{2\pi}{T}
\end{equation}
ist praktisch, da dieser Ausdruck öfters
auftauchen wird. Die Größe $\omega$ wird \emdef{Kreisfrequenz}
genannt. Die einfache Frequenz $f:=1/T$ wird zugunsten der
Kreisfrequenz keine Verwendung finden. Das ist kein Problem, da
zwischen beiden der einfache Zusammenhang $\omega = 2\pi f$
besteht.

\subsection{Harmonische Schwingungen}
Der Begriff der Schwingung ist sehr allgemein. Darunter versteht
man ein Signal, welches sich zeitlich auf eine gewisse Art und
Weise wiederholt. Es ist schwierig, eine genaue Definition
anzugeben. Wir wollen uns nun zunächst auf die harmonischen
Schwingungen beschränken, der einfachsten Art von Schwingung.

\begin{Definition}
Eine \emdef{harmonische Schwingung} ist ein Signal von der Form
\begin{equation}
f(t) := A\sin(\varphi(t))+B,\quad f\colon\R\to\R
\end{equation}
mit
\begin{equation}
\varphi(t) := \omega t+\varphi_0
\end{equation}
Hierbei nennt man $A$ die \emdef{Amplitude}, $B$ den \emdef{Gleichwert},
$\varphi(t)$ die Phase und $\varphi_0$ die Anfangsphase. Die
Phase ist von der Größenart her ein Winkel, weshalb man auch vom
\emdef{Phasenwinkel} spricht.
\end{Definition}
Beim Parameter $\omega$ muss es sich um die Kreisfrequenz handeln,
denn die Sinusfunktion $\sin(x)$ besitzt die Periode $P=2\pi$.
Nun gilt aber $x=\omega t$ und somit $P=\omega T$. Dann ist
$\omega T=2\pi$ und daher $\omega=2\pi/T$, in Übereinstimmung mit
\eqref{eq:Kreisfrequenz}. Man kann bei dieser Betrachtung
o.\,B.\,d.\,A $\varphi_0=0$ setzen, denn die Verschiebung
der Sinusfunktion um $\varphi_0$ ändert nicht ihre Periodendauer.

Eine harmonische Schwingung erfüllt die Differentialgleichung
\begin{equation}
y'' = \omega^2 B-\omega^2 y,
\end{equation}
wie man durch Einsetzen leicht feststellen kann.


\vfill\noindent
Dieser Text steht unter der Lizenz Creative Commons CC0.

\end{document}
