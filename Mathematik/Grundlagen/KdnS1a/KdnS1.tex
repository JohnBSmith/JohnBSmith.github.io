\documentclass[8pt]{beamer}
\usetheme{Antibes}
\useinnertheme{rectangles}
\useoutertheme{infolines}
\usepackage[utf8]{inputenc}
\usepackage[T1]{fontenc}
\usepackage[ngerman]{babel}

% Patch the look of +, = in arev
\usefonttheme{serif}

\usepackage{arev}
% Patch punctuation to be upright
\DeclareMathSymbol{.}{\mathpunct}{operators}{`.}
\DeclareMathSymbol{,}{\mathpunct}{operators}{`,}

\usepackage{amsmath}
\usepackage{amssymb}
\usepackage{ebproof}
\usepackage{booktabs}
\usepackage{fitch}

\usepackage{listings}
\lstset{basicstyle=\ttfamily}

\newcommand{\inferrulewidth}{0.4608pt}
\ebproofset{rule thickness=\inferrulewidth}
% \the\fontdimen8\textfont3
\newcommand{\inote}[1]{{\small #1}}

\setbeamertemplate{footline}{%
\begin{beamercolorbox}[ht=3.0ex,dp=1ex]{title in head/foot}
\hfill\footnotesize\insertpagenumber\enspace\enspace\end{beamercolorbox}}

\definecolor{bluegreen1}{rgb}{0.0,0.20,0.28}
\definecolor{bluegreen2}{rgb}{0.0,0.20,0.28}
\setbeamercolor*{palette primary}{fg=white,bg=bluegreen1}
\setbeamercolor*{palette secondary}{fg=white,bg=bluegreen2}
\setbeamercolor*{palette tertiary}{fg=white,bg=bluegreen2}
\setbeamercolor{itemize item}{fg=black}
\setbeamercolor{block title}{bg=bluegreen2}
\newcommand{\modest}[1]{{\small\color{gray}#1}}
\hypersetup{colorlinks,urlcolor=magenta}

% highlight colors
% \definecolor{chla}{RGB}{220,0,90}
% \definecolor{chlb}{RGB}{0,180,220}
% \definecolor{chlg}{RGB}{255,190,0}
% \definecolor{chlh}{RGB}{20,20,160}

\definecolor{chla}{RGB}{100,143,255}
\definecolor{chlb}{RGB}{255,176,0}
\definecolor{chlg}{RGB}{160,0,120}
\definecolor{chlh}{RGB}{240,40,140}

\newcommand{\hla}[1]{{\color{chla}#1}}
\newcommand{\hlb}[1]{{\color{chlb}#1}}
\newcommand{\hlg}[1]{{\color{chlg}#1}}
\newcommand{\hlh}[1]{{\color{chlh}#1}}

\newcommand{\unit}[1]{\mathrm{#1}}
\newcommand{\strong}[1]{\textsf{\textbf{#1}}}
\newcommand{\defiff}{\quad:\Longleftrightarrow\quad}
\newcommand{\infernote}[1]{\!\text{\footnotesize #1}}
\renewcommand{\qedsymbol}{\ensuremath{\Box}}
\newcommand{\discharge}[1]{$\sim$#1}
\newcommand{\centerheadline}[1]{%
  \begin{center}\strong{#1}\end{center}}
\newcommand{\parspace}{\vspace{0.8em}}
\newcommand{\cond}{\rightarrow}
\newcommand{\bicond}{\leftrightarrow}

\title{Natürliches Schließen}
\subtitle{Teil 1: Aussagenlogik}
\date{}

\begin{document}

\begin{frame}
\maketitle
\end{frame}

%\begin{frame}
%\hla{\rule{1cm}{1cm}}\hlb{\rule{1cm}{1cm}}\hlg{\rule{1cm}{1cm}}\hlh{\rule{1cm}{1cm}}
%\end{frame}

\begin{frame}
\centerheadline{Deduktive Argumente}
\end{frame}

\begin{frame}
Zur Klärung, wie logisches Denken vonstatten gehen sollte, teilen
wir den Gedankengang in kleine Schritte auf, die wir \emph{Argumente}
nennen.\pause

\parspace
Wir notieren ein Argument in der Form
\[\begin{array}{@{}l@{}}
\text{Prämisse}\\[2pt]
\therefore\text{Konklusion}\end{array}\]
oder auch als
\[\dfrac{\text{Prämisse}}{\text{Konklusion}}.\]
Dies liest sich: \emph{Die Prämisse gilt, also gilt die Konklusion.}\pause

\parspace
Besitzt das Argument zwei Prämissen, notiert man:
\[\begin{array}{@{}l@{}}
\text{Prämisse 1}\\[2pt]
\text{Prämisse 2}\\
\midrule[\inferrulewidth]
\text{Konklusion}
\end{array}\]
\end{frame}

\begin{frame}
Ein Argument heißt \emph{gültig}, wenn die Wahrheit der Prämissen die
Wahrheit der Konklusion nach sich zieht.

\parspace
Ein Argument heißt \emph{korrekt}, wenn es gültig ist und seine
Prämisse tatsächlich wahr sind.\pause

\parspace
Betrachten wir nun dieses Argument:
\[\frac{\text{Es regnet.}}{\text{Die Straße wird nass.}}\]
Die gültigkeit dieses Arguments hängt vom Kontext ab, von der Welt, über
die die Aussagen reden. Das Weltmodell muss hier zugegebenermaßen ein sehr
einfaches sein, damit der Schluss exakt in dieser Weise gültig ist.
Man darf sich dabei etwa ein Computerspiel vorstellen, in dem es eine Straße
gibt, die bei Regen ausnahmslos nass wird.
\end{frame}

\begin{frame}
Anders verhält es sich mit diesem Argument:
\[\begin{array}{@{}l@{}}
\text{Wenn es regnet, wird die Straße nass.}\\[2pt]
\text{Es regnet.}\\
\midrule[\inferrulewidth]
\text{Die Straße wird nass.}
\end{array}\]
Es hängt nicht von der Welt ab, über die die Aussagen reden, sondern
gilt aufgrund seiner Struktur, aufgrund seiner \emph{Form}.\pause

\parspace
Diese Form besteht anscheinend in dem Schema
\[\begin{array}{@{}l@{}}
\text{Wenn $A$, dann $B$}\\[2pt]
\text{$A$}\\
\midrule[\inferrulewidth]
\text{$B$}
\end{array}\]
Es ist eine als \emph{Modus ponens} bekannte Schlussform.
\end{frame}

\begin{frame}
Natürliche Sprachen sind außerordentlich komplex bzw. ermöglichen
übermäßig reichhaltige Ausdrucksformen. Um ein klares Bild von den
möglichen einfachen Schlussformen zu bekommen, möchte man sich daher
auf eine möglichst einfache Sprache beschränken. Es hat sich
herausgestellt, dass die \emph{aussagenlogische Sprache} dafür genügt,
die in der Mathematik auftretenden Schlussformen zu ermöglichen.\pause

\parspace
Weiterhin bildet die Aussagenlogik ein wichtiges Studienobjekt, das
das Verständnis komplizierterer Logiken fördert.
\end{frame}

\begin{frame}
\centerheadline{Aussagenlogische Sprache}
\end{frame}

\begin{frame}
\begin{block}{Formeln der Aussagenlogik}
Die Menge der Formeln sei in folgender Weise induktiv definiert.
\begin{itemize}
\item Jede der aussagenlogischen Variablen $p_0,p_1,p_2,\ldots$ ist eine Formel,
genannt \emph{Primformel} oder \emph{atomare Formel}.
\item Die Symbole $\bot, \top$ sind Formeln.
\item Ist $A$ eine Formel, so ist auch $(\lnot A)$ eine Formel.
\item Sind $A,B$ Formeln, so sind auch $(A\land B), (A\lor B), (A\cond B), (A\bicond B)$ Formeln.
\item Nichts anderes ist eine Formel.
\end{itemize}
\end{block}\pause

Wie bei Punkt- vor Strichrechnung führen wir für die Junktoren zur
Einsparung von Klammern eine Rangfolge ein. Diese ist
\[\lnot, \land, \lor, \cond, \bicond.\]
Die induktive Erklärung der Formeln lässt sich so modifizieren, dass
sie die Rangfolge berücksichtigt. Ich will dies hier allerdings nicht
näher ausführen, da es vom Wesentlichen ablenkt. Es führt in die
Thematik der formalen Grammatiken und des Parserbaus.
\end{frame}

\begin{frame}[t]
\vspace{0.5em}
\strong{Bemerkung zur Nomenklatur}

\vspace{4em}
Es stehen $p, q, r$ alternativ zu $p_0, p_1, p_2$ für Primformeln.

\parspace
Es stehen $P, Q, R$ für Schemavariablen, für die Primformeln
eingesetzt werden dürfen.

\parspace
Es stehen $A, B, C$ oder alternativ $\varphi, \psi, \chi$ für
Schemavariablen, für die beliebige Formeln eingesetzt werden dürfen.

\parspace
Es stehen $\Gamma, \Delta$ für Schemavariablen, für die Ansammlungen
von Formeln eingesetzt werden dürfen.
\end{frame}

\begin{frame}
\centerheadline{Die Schlussregeln}
\end{frame}

\begin{frame}
Zum Beweisen von Aussagen wollen wir nun gern Argumente in formalisierter
Form führen. Wir notieren die Prämissen ab jetzt nebeneinander statt
untereinander. Der Modus ponens sollte insofern lauten
\[\frac{A\cond B\qquad A}{B}.\]\pause
Es verhält sich aber so, dass wir auch Aussagen unter Annahmen ableiten
wollen. Klarheit schafft hierbei die Idee, die gemachten Annahmen
mit der abgeleiteten Aussage mitzutragen. Dies wird notiert als
\emph{Sequenz}
\[A_1,\ldots,A_n\vdash A.\]
Sie drückt aus, dass $A$ unter den Annahmen $A_1,\ldots,A_n$
ableitbar sei.\pause

\parspace
Zur Abkürzung notieren wir $\Gamma\vdash A$ zu einer Ansammlung
$\Gamma$ von Annahmen. Der Modus ponens lässt sich nun allgemein fassen als
\[\frac{\Gamma\vdash A\cond B\qquad\Gamma\vdash A}{\Gamma\vdash B}.\]
In Worten: \emph{Wenn sowohl $A\cond B$ als auch $A$ unter der den
Annahmen $\Gamma$ ableitbar sind, so ist es auch $B$.}
\end{frame}

\begin{frame}
Die Ansammlung $\Gamma$ in der Sequenz $\Gamma\vdash A$ ist eine endliche
Menge, oder sollte sich verhalten wie eine. Die Notation $\Gamma,A\vdash B$
besitzt dahingehend die Bedeutung $\Gamma\cup\{A\}\vdash B$. Entsprechend
besitzt $\Gamma,\Gamma'\vdash A$ die Bedeutung $\Gamma\cup\Gamma'\vdash A$.\pause

\parspace
Die Sequenzen $\Gamma,\Gamma'\vdash A$ und $\Gamma',\Gamma\vdash A$
sind demzufolge gleich.\pause

\parspace
Kodiert man die Ansammlung alternativ als Liste, sollten, insofern sie
sich wie eine Menge verhalten soll, die folgenden
strukturellen Schlussregeln bestehen.

\begin{block}{Vertauschungsregel}
\[\dfrac{\Gamma\vdash A}{\sigma(\Gamma)\vdash A}\quad\text{($\sigma$ sei eine beliebige Permutation)}\]
\end{block}

\begin{block}{Kontraktionsregel}
\[\dfrac{\Gamma,A,A\vdash B}{\Gamma,A\vdash B}\]
\end{block}

Die Umkehrung der Kontraktionsregel wird nicht benötigt, da sie mit
der sogleich erläuterten Abschwächungsregel einhergeht.
\end{frame}

\begin{frame}[t]
\vspace{2em}
Ist eine Aussage aus bestimmten Annahmen ableitbar, sollte dies bei
Hinzufügung weiterer Annahmen erst recht möglich sein. Man sagt, die
klassische Logik wäre \emph{monoton}. Insofern gilt die
\begin{block}{Abschwächungsregel}
\[\dfrac{\Gamma\vdash A}{\Gamma,\Gamma'\vdash A}\]
\end{block}\pause

Vermittels dieser unternehmen wir die folgende Überlegung.

\only<3>{
\[\begin{prooftree}
\hypo{\Gamma,\Gamma'\vdash B}
\end{prooftree}\]
}
\only<4>{
\[\begin{prooftree}
\hypo{\hlg{\Gamma,\Gamma'}\vdash\hlb{B}}
\end{prooftree}\]
\mbox{}\hfill $\dfrac{\hlg{\Gamma}\vdash\hla{A}\to\hlb{B}\qquad
\hlg{\Gamma}\vdash\hla{A}}{\hlg{\Gamma}\vdash\hlb{B}}$
}
\only<5>{
\[\begin{prooftree}
  \hypo{\hlg{\Gamma,\Gamma'}\vdash\hla{A}\to\hlb{B}}
  \hypo{\hlg{\Gamma,\Gamma'}\vdash\hlb{B}}
\infer2{\hlg{\Gamma,\Gamma'}\vdash\hlb{B}}
\end{prooftree}\]
\mbox{}\hfill $\dfrac{\hlg{\Gamma}\vdash\hla{A}\to\hlb{B}\qquad
\hlg{\Gamma}\vdash\hla{A}}{\hlg{\Gamma}\vdash\hlb{B}}$
}
\only<6>{
\[\begin{prooftree}
  \hypo{\Gamma,\Gamma'\vdash A\to B}
  \hypo{\Gamma,\Gamma'\vdash B}
\infer2{\Gamma,\Gamma'\vdash B}
\end{prooftree}\]
}
\only<7>{
\[\begin{prooftree}
  \hypo{\hlg{\Gamma},\hlh{\Gamma'}\vdash\hla{A\to B}}
  \hypo{\Gamma,\Gamma'\vdash B}
\infer2{\Gamma,\Gamma'\vdash B}
\end{prooftree}\]
\mbox{}\hfill $\dfrac{\hlg{\Gamma}\vdash\hla{A}}{\hlg{\Gamma},\hlh{\Gamma'}\vdash\hla{A}}$
}
\only<8>{
\[\begin{prooftree}
    \hypo{\hlg{\Gamma}\vdash\hla{A\to B}}
  \infer1{\hlg{\Gamma},\hlh{\Gamma'}\vdash\hla{A\to B}}
  \hypo{\Gamma,\Gamma'\vdash B}
\infer2{\Gamma,\Gamma'\vdash B}
\end{prooftree}\]
\mbox{}\hfill $\dfrac{\hlg{\Gamma}\vdash\hla{A}}{\hlg{\Gamma},\hlh{\Gamma'}\vdash\hla{A}}$
}
\only<9>{
\[\begin{prooftree}
    \hypo{\Gamma\vdash A\to B}
  \infer1{\Gamma,\Gamma'\vdash A\to B}
  \hypo{\Gamma,\Gamma'\vdash B}
\infer2{\Gamma,\Gamma'\vdash B}
\end{prooftree}\]
}
\only<10>{
\[\begin{prooftree}
    \hypo{\Gamma\vdash A\to B}
  \infer1{\Gamma,\Gamma'\vdash A\to B}
  \hypo{\hlh{\Gamma},\hlg{\Gamma'}\vdash\hla{B}}
\infer2{\Gamma,\Gamma'\vdash B}
\end{prooftree}\]
\mbox{}\hfill $\dfrac{\hlg{\Gamma}\vdash\hla{A}}{\hlg{\Gamma},\hlh{\Gamma'}\vdash\hla{A}}$
}
\only<11>{
\[\begin{prooftree}
    \hypo{\Gamma\vdash A\to B}
  \infer1{\Gamma,\Gamma'\vdash A\to B}
    \hypo{\hlg{\Gamma'}\vdash\hla{B}}
  \infer1{\hlh{\Gamma},\hlg{\Gamma'}\vdash\hla{B}}
\infer2{\Gamma,\Gamma'\vdash B}
\end{prooftree}\]
\mbox{}\hfill $\dfrac{\hlg{\Gamma}\vdash\hla{A}}{\hlg{\Gamma},\hlh{\Gamma'}\vdash\hla{A}}$
}
\only<12>{
\[\begin{prooftree}
    \hypo{\Gamma\vdash A\to B}
  \infer1{\Gamma,\Gamma'\vdash A\to B}
     \hypo{\Gamma'\vdash B}
  \infer1{\Gamma,\Gamma'\vdash B}
\infer2{\Gamma,\Gamma'\vdash B}
\end{prooftree}\]
}
\end{frame}

\begin{frame}[t]
\vspace{0.5em}
\strong{Regeln zur Subjunktion}

\parspace
Ergo gilt der Modus ponens in der folgenden allgemeinen Form, die
wir Subjunktionsbeseitigung nennen wollen.

\begin{block}{Subjunktionsbeseitigung}
\[\dfrac{\Gamma\vdash A\to B\qquad\Gamma'\vdash A}{\Gamma,\Gamma'\vdash B}\]
\end{block}\pause

Wie verläuft nun der Beweis einer Subjunktion $A\cond B$? Mit ihr
möchte man wie gesagt von $A$ zu $B$ gelangen dürfen. Aber genau dies
erreicht man, indem man $B$ unter Annahme von $A$ ableitet.\pause

\parspace
Dies motiviert die

\begin{block}{Subjunktionseinführung}
\[\dfrac{\Gamma,A\vdash B}{\Gamma\vdash A\cond B}\]
\end{block}
\end{frame}

\begin{frame}[t]
\vspace{0.5em}
\strong{Regeln zur Negation}

\parspace
Die Negation $\lnot A$ versteht sich als gleichbedeutend zu $A\cond\bot$,
oder darf auch als Abkürzung dafür stehen. Insofern ergeben sich Regeln
analog zu denen der Subjunktion.

\begin{block}{Negationsbeseitigung}
\[\dfrac{\Gamma\vdash\lnot A\qquad\Gamma'\vdash A}{\Gamma,\Gamma'\vdash\bot}\]
\end{block}

\begin{block}{Negationseinführung}
\[\dfrac{\Gamma,A\vdash\bot}{\Gamma\vdash\lnot A}\]
\end{block}
\end{frame}

\begin{frame}[t]
\vspace{0.5em}
\strong{Regeln zur Konjunktion}

\parspace
Der Umgang mit der Konjunktion $A\land B$ sollte intuitiv verständlich
sein. Aus $A\land B$ will sich sowohl $A$ als auch $B$ ableiten lassen.
Gelten umgekehrt sowohl $A$ als auch $B$, soll auch $A\land B$ gelten.

\begin{block}{Konjunktionsbeseitigung}
\[\dfrac{\Gamma\vdash A\land B}{\Gamma\vdash A},\quad
\dfrac{\Gamma\vdash A\land B}{\Gamma\vdash B}\]
\end{block}

\begin{block}{Konjunktionseinführung}
\[\dfrac{\Gamma\vdash A\qquad\Gamma'\vdash B}{\Gamma,\Gamma'\vdash A\land B}\]
\end{block}
\end{frame}

\begin{frame}[t]
\vspace{0.5em}
\strong{Regeln zur Disjunktion}

\parspace
Der Umgang mit der Disjunktion $A\lor B$ ist ein wenig schwieriger.
Ihre Beseitigungsregel kodiert eine Fallunterscheidung. Ist $C$
nämlich sowohl aus $A$ als auch aus $B$ ableitbar, will $C$ auch aus
$A\lor B$ ableitbar sein.

\begin{block}{Disjunktionsbeseitigung}
\[\dfrac{\Gamma\vdash A\lor B\qquad\Gamma',A\vdash C\qquad\Gamma'',B\vdash C}
{\Gamma,\Gamma',\Gamma''\vdash C}
\quad\text{bzw.}\quad
\dfrac{\Gamma',A\vdash C\qquad\Gamma'',B\vdash C}
{\Gamma',\Gamma'',A\lor B\vdash C}\]
\end{block}

Mit der Einführung verhält es sich einfacher. Gilt $A$, so gilt erst
recht $A\lor B$. Gilt $B$, so gilt erst recht $A\lor B$.

\begin{block}{Disjunktionseinführung}
\[\dfrac{\Gamma\vdash A}{\Gamma\vdash A\lor B},\quad
\dfrac{\Gamma\vdash B}{\Gamma\vdash A\lor B}\]
\end{block}
\end{frame}

\begin{frame}[t]
\vspace{0.5em}
\strong{Regeln zur Bijunktion}

\parspace
Die Bijunktion $A\bicond B$ versteht sich als bleichbedeutend zu
$(A\cond B)\land (B\cond A)$, oder darf als Abkürzung dafür stehen.
Die Regeln ergeben sich somit analog zu denen der Konjunktion.

\begin{block}{Bijunktionsbeseitigung}
\[\dfrac{\Gamma\vdash A\bicond B}{\Gamma\vdash A\cond B},\quad
\dfrac{\Gamma\vdash A\bicond B}{\Gamma\vdash B\cond A}\]
\end{block}

\begin{block}{Bijunktionseinführung}
\[\dfrac{\Gamma\vdash A\cond B\qquad\Gamma'\vdash B\cond A}
{\Gamma,\Gamma'\vdash A\bicond B}\]
\end{block}
\end{frame}

\begin{frame}[t]
\vspace{0.5em}
\strong{Regel für Annahmen}

\parspace
Die Besonderheit des natürlichen Schließens liegt in der Machbarkeit,
Aussagen unter Annahmen abzuleiten. Die Abhängigkeit von einer
Annahme wird hierbei im Zuge der Subjunktions- sowie Negationseinführung
wieder getilgt. Bislang fehlt aber noch eine Regel für Annahmen.\pause

\parspace
Zweifelsfrei als richtig erkennt man den trivialen Sachverhalt an, dass
eine Aussage unter ihrer eigenen Annahme gilt. Das heißt, zu jeder
Aussage $A$ will die Sequenz $A\vdash A$ ex nihilo eingeführt werden
dürfen.

\begin{block}{Grundsequenzenregel}
\[\dfrac{}{A\vdash A}\]
\end{block}

Die Notation drückt eine Regel ohne Prämissen aus.

\parspace
In den Grundsequenzen und ggf. Axiomen beginnt ein Beweis. Alle anderen
Regeln führen eine Sequenz sukzessive auf Grundsequenzen oder Axiome
bzw. bereits bewiesene Theoreme zurück.
\end{frame}

\begin{frame}[t]
\vspace{0.5em}
\strong{Regeln der intuitionistischen Logik}

\parspace
Die bislang aufgeführten Regeln charakterisieren die \emph{Minimallogik}.
Sie wird zur \emph{intuitionistischen Logik} bei Hinzunahme der Regel

\begin{block}{EFQ: Ex falso quodlibet}
\[\dfrac{\Gamma\vdash\bot}{\Gamma\vdash A}\]
\end{block}

In Worten besagt sie, dass aus einem Widerspruch jede beliebige
Aussage folgt.

\parspace
Intuitionistische Logik dient als Basis der konstruktiven Mathematik.
In ihr müssen alle mathematischen Objekte konstruiert werden, statt sie
aus einem einfach so als existent angenommenen Universum aussondern
zu dürfen.

\parspace
\begin{small}
Die bislang aufgestellten Regeln lassen sich über die
Curry-Howard-Korrespondenz in Bezug zum einfach getypten $\lambda$-Kalkül
stellen, dergestalt dass jeder Aussage ein Typ und ihrem Beweis ein
Term des Typs entspricht. Die Schlussregeln drücken dabei aus, auf
welche Arten Terme konstruiert werden dürfen. Die $\beta$-Reduktion
entspricht der Normalisierung von Beweisen.

\parspace
Die kategorielle Semantik ordnet jedem Typ daraufhin ein Objekt und jedem
Term einen Morphismus innerhalb einer kartesisch abgeschlossenen Kategorie
zu. Speziell mit der Kategorie der Mengen erhält man das volle
mengentheoretische Modell.
\end{small}
\end{frame}

\begin{frame}[t]
\vspace{0.5em}
\strong{Regeln der klassischen Logik}

\parspace
Von der intuitionistischen Logik aus gelangt man letztlich zur
klassischen, indem der Satz vom ausgeschlossenen Dritten
hinzugenommen wird.

\begin{block}{LEM: Law of excluded middle (TND: Tertium non datur)}
\[\dfrac{}{\vdash A\lor\lnot A}\]
\end{block}

Alternativ gelangt man direkt von der minimalen zur klassischen
durch Hinzunahme der Regel zur Beseitigung der Doppelnegation.

\begin{block}{DNE: Double negation elimination}
\[\dfrac{\Gamma\vdash\lnot\lnot A}{\Gamma\vdash A}\]
\end{block}

Die Regeln LEM, DNE sind in der intuitionistischen Logik nicht gültig.
Sie besitzt eine andere Semantik als die klassische.
\end{frame}

\begin{frame}
\centerheadline{Logisches Schließen}
\end{frame}

\begin{frame}[t]
\vspace{1em}
Ein Beweis wird erbracht, indem vermittels der Schlussregeln ein
\emph{Beweisbaum} konstruiert wird, auch \emph{Herleitungs-} oder
\emph{Ableitungsbaum} genannt. Ich will dies anhand eines Beispiels
verdeutlichen.

\parspace
Wir wollen den Beweis von $(A\to B)\to (\lnot B\to\lnot A)$ führen.

\parspace
Wir bauen den Beweis rückwärts, lassen den Baum also von der Wurzel
bis zu den Blättern wachsen. Zur Verdeutlichung werden die Schemavariablen
der genutzten Schlussregeln bunt eingefärbt, wobei der jeweiligen
Einsetzung an der Fortführung des Baums die entsprechende
Farbe zukommt.

\only<2>{
\[\begin{prooftree}
\hypo{\vdash (A\to B)\to (\lnot B\to\lnot A)}
\end{prooftree}\]
}
\only<3>{
\[\begin{prooftree}
\hypo{\vdash \hla{(A\to B)}\to \hlb{(\lnot B\to\lnot A)}}
\end{prooftree}\]
\mbox{}\hfill $\displaystyle\frac{\hlg{\Gamma},\hla{A}\vdash\hlb{B}}
{\hlg{\Gamma}\vdash\hla{A}\to\hlb{B}}$
}
\only<4>{
\[\begin{prooftree}
  \hypo{\hla{A\to B}\vdash\hlb{\lnot B\to\lnot A}}
\infer1{\vdash\hla{(A\to B)}\to \hlb{(\lnot B\to\lnot A)}}
\end{prooftree}\]
\mbox{}\hfill $\displaystyle\frac{\hlg{\Gamma},\hla{A}\vdash\hlb{B}}
{\hlg{\Gamma}\vdash\hla{A}\to\hlb{B}}$
}
\only<5>{
\[\begin{prooftree}
  \hypo{A\to B\vdash\lnot B\to\lnot A}
\infer1{\vdash (A\to B)\to(\lnot B\to\lnot A)}
\end{prooftree}\]
}
\only<6>{
\[\begin{prooftree}
  \hypo{\hlg{A\to B}\vdash\hla{\lnot B}\to\hlb{\lnot A}}
\infer1{\vdash (A\to B)\to(\lnot B\to\lnot A)}
\end{prooftree}\]
\mbox{}\hfill $\displaystyle\frac{\hlg{\Gamma},\hla{A}\vdash\hlb{B}}
{\hlg{\Gamma}\vdash\hla{A}\to\hlb{B}}$
}
\only<7>{
\[\begin{prooftree}
    \hypo{\hlg{A\to B},\hla{\lnot B}\vdash\hlb{\lnot A}}
  \infer1{\hlg{A\to B}\vdash\hla{\lnot B}\to\hlb{\lnot A}}
\infer1{\vdash (A\to B)\to(\lnot B\to\lnot A)}
\end{prooftree}\]
\mbox{}\hfill $\displaystyle\frac{\hlg{\Gamma},\hla{A}\vdash\hlb{B}}
{\hlg{\Gamma}\vdash\hla{A}\to\hlb{B}}$
}
\only<8>{
\[\begin{prooftree}
    \hypo{A\to B,\lnot B\vdash\lnot A}
  \infer1{A\to B\vdash\lnot B\to\lnot A}
\infer1{\vdash (A\to B)\to(\lnot B\to\lnot A)}
\end{prooftree}\]
}
\only<9>{
\[\begin{prooftree}
    \hypo{\hlg{A\to B,\lnot B}\vdash\lnot\hla{A}}
  \infer1{A\to B\vdash\lnot B\to\lnot A}
\infer1{\vdash (A\to B)\to(\lnot B\to\lnot A)}
\end{prooftree}\]
\mbox{}\hfill $\displaystyle\frac{\hlg{\Gamma},\hla{A}\vdash\bot}
{\hlg{\Gamma}\vdash\lnot\hla{A}}$
}
\only<10>{
\[\begin{prooftree}
      \hypo{\hlg{A\to B,\lnot B},\hla{A}\vdash\bot}
    \infer1{\hlg{A\to B,\lnot B}\vdash\lnot\hla{A}}
  \infer1{A\to B\vdash\lnot B\to\lnot A}
\infer1{\vdash (A\to B)\to(\lnot B\to\lnot A)}
\end{prooftree}\]
\mbox{}\hfill $\displaystyle\frac{\hlg{\Gamma},\hla{A}\vdash\bot}
{\hlg{\Gamma}\vdash\lnot\hla{A}}$
}
\only<11>{
\[\begin{prooftree}
      \hypo{A\to B,\lnot B,A\vdash\bot}
    \infer1{A\to B,\lnot B\vdash\lnot A}
  \infer1{A\to B\vdash\lnot B\to\lnot A}
\infer1{\vdash (A\to B)\to(\lnot B\to\lnot A)}
\end{prooftree}\]
}
\only<12>{
\[\begin{prooftree}
      \hypo{\hlh{A\to B},\hlg{\lnot B},\hlh{A}\vdash\bot}
    \infer1{A\to B,\lnot B\vdash\lnot A}
  \infer1{A\to B\vdash\lnot B\to\lnot A}
\infer1{\vdash (A\to B)\to(\lnot B\to\lnot A)}
\end{prooftree}\]
\mbox{}\hfill $\displaystyle\frac{\hlg{\Gamma}\vdash\lnot\hla{A}\qquad\hlh{\Gamma'}\vdash\hla{A}}
{\hlg{\Gamma},\hlh{\Gamma'}\vdash\bot}$
}
\only<13>{
\[\begin{prooftree}
        \hypo{\hlg{\lnot B}\vdash\lnot\hla{B}}
        \hypo{\hlh{A\to B, A}\vdash\hla{B}}
      \infer2{\hlh{A\to B},\hlg{\lnot B},\hlh{A}\vdash\bot}
    \infer1{A\to B,\lnot B\vdash\lnot A}
  \infer1{A\to B\vdash\lnot B\to\lnot A}
\infer1{\vdash (A\to B)\to(\lnot B\to\lnot A)}
\end{prooftree}\]
\mbox{}\hfill $\displaystyle\frac{\hlg{\Gamma}\vdash\lnot\hla{A}\qquad\hlh{\Gamma'}\vdash\hla{A}}
{\hlg{\Gamma},\hlh{\Gamma'}\vdash\bot}$
}
\only<14>{
\[\begin{prooftree}
        \hypo{\lnot B\vdash\lnot B}
        \hypo{A\to B, A\vdash B}
      \infer2{A\to B,\lnot B,A\vdash\bot}
    \infer1{A\to B,\lnot B\vdash\lnot A}
  \infer1{A\to B\vdash\lnot B\to\lnot A}
\infer1{\vdash (A\to B)\to(\lnot B\to\lnot A)}
\end{prooftree}\]
}
\only<15>{
\[\begin{prooftree}
        \hypo{\hla{\lnot B}\vdash\hla{\lnot B}}
        \hypo{A\to B, A\vdash B}
      \infer2{A\to B,\lnot B,A\vdash\bot}
    \infer1{A\to B,\lnot B\vdash\lnot A}
  \infer1{A\to B\vdash\lnot B\to\lnot A}
\infer1{\vdash (A\to B)\to(\lnot B\to\lnot A)}
\end{prooftree}\]
\mbox{}\hfill $\displaystyle\frac{}{\hla{A}\vdash\hla{A}}$
}
\only<16>{
\[\begin{prooftree}
        \infer0{\hla{\lnot B}\vdash\hla{\lnot B}}
        \hypo{A\to B, A\vdash B}
      \infer2{A\to B,\lnot B,A\vdash\bot}
    \infer1{A\to B,\lnot B\vdash\lnot A}
  \infer1{A\to B\vdash\lnot B\to\lnot A}
\infer1{\vdash (A\to B)\to(\lnot B\to\lnot A)}
\end{prooftree}\]
\mbox{}\hfill $\displaystyle\frac{}{\hla{A}\vdash\hla{A}}$
}
\only<17>{
\[\begin{prooftree}
        \infer0{\lnot B\vdash\lnot B}
        \hypo{A\to B, A\vdash B}
      \infer2{A\to B,\lnot B,A\vdash\bot}
    \infer1{A\to B,\lnot B\vdash\lnot A}
  \infer1{A\to B\vdash\lnot B\to\lnot A}
\infer1{\vdash (A\to B)\to(\lnot B\to\lnot A)}
\end{prooftree}\]
}
\only<18>{
\[\begin{prooftree}
        \infer0{\lnot B\vdash\lnot B}
        \hypo{\hlg{A\to B}, \hlh{A}\vdash\hlb{B}}
      \infer2{A\to B,\lnot B,A\vdash\bot}
    \infer1{A\to B,\lnot B\vdash\lnot A}
  \infer1{A\to B\vdash\lnot B\to\lnot A}
\infer1{\vdash (A\to B)\to(\lnot B\to\lnot A)}
\end{prooftree}\]
\mbox{}\hfill $\displaystyle\frac{\hlg{\Gamma}\vdash\hla{A}\to\hlb{B}\qquad\hlh{\Gamma'}\vdash\hla{A}}
{\hlg{\Gamma},\hlh{\Gamma'}\vdash\hlb{B}}$
}
\only<19>{
\[\begin{prooftree}
        \infer0{\lnot B\vdash\lnot B}
          \hypo{\hlg{A\to B}\vdash\hla{A}\to\hlb{B}}
          \hypo{\hlh{A}\vdash\hla{A}}
        \infer2{\hlg{A\to B}, \hlh{A}\vdash\hlb{B}}
      \infer2{A\to B,\lnot B,A\vdash\bot}
    \infer1{A\to B,\lnot B\vdash\lnot A}
  \infer1{A\to B\vdash\lnot B\to\lnot A}
\infer1{\vdash (A\to B)\to(\lnot B\to\lnot A)}
\end{prooftree}\]
\mbox{}\hfill $\displaystyle\frac{\hlg{\Gamma}\vdash\hla{A}\to\hlb{B}\qquad\hlh{\Gamma'}\vdash\hla{A}}
{\hlg{\Gamma},\hlh{\Gamma'}\vdash\hlb{B}}$
}
\only<20>{
\[\begin{prooftree}
        \infer0{\lnot B\vdash\lnot B}
          \hypo{A\to B\vdash A\to B}
          \hypo{A\vdash A}
        \infer2{A\to B, A\vdash B}
      \infer2{A\to B,\lnot B,A\vdash\bot}
    \infer1{A\to B,\lnot B\vdash\lnot A}
  \infer1{A\to B\vdash\lnot B\to\lnot A}
\infer1{\vdash (A\to B)\to(\lnot B\to\lnot A)}
\end{prooftree}\]
}
\only<21>{
\[\begin{prooftree}
        \infer0{\lnot B\vdash\lnot B}
          \hypo{\hla{A\to B}\vdash\hla{A\to B}}
          \hypo{A\vdash A}
        \infer2{A\to B, A\vdash B}
      \infer2{A\to B,\lnot B,A\vdash\bot}
    \infer1{A\to B,\lnot B\vdash\lnot A}
  \infer1{A\to B\vdash\lnot B\to\lnot A}
\infer1{\vdash (A\to B)\to(\lnot B\to\lnot A)}
\end{prooftree}\]
\mbox{}\hfill $\displaystyle\frac{}{\hla{A}\vdash\hla{A}}$
}
\only<22>{
\[\begin{prooftree}
        \infer0{\lnot B\vdash\lnot B}
          \infer0{\hla{A\to B}\vdash\hla{A\to B}}
          \hypo{A\vdash A}
        \infer2{A\to B, A\vdash B}
      \infer2{A\to B,\lnot B,A\vdash\bot}
    \infer1{A\to B,\lnot B\vdash\lnot A}
  \infer1{A\to B\vdash\lnot B\to\lnot A}
\infer1{\vdash (A\to B)\to(\lnot B\to\lnot A)}
\end{prooftree}\]
\mbox{}\hfill $\displaystyle\frac{}{\hla{A}\vdash\hla{A}}$
}
\only<23>{
\[\begin{prooftree}
        \infer0{\lnot B\vdash\lnot B}
          \infer0{A\to B\vdash A\to B}
          \hypo{A\vdash A}
        \infer2{A\to B, A\vdash B}
      \infer2{A\to B,\lnot B,A\vdash\bot}
    \infer1{A\to B,\lnot B\vdash\lnot A}
  \infer1{A\to B\vdash\lnot B\to\lnot A}
\infer1{\vdash (A\to B)\to(\lnot B\to\lnot A)}
\end{prooftree}\]
}
\only<24>{
\[\begin{prooftree}
        \infer0{\lnot B\vdash\lnot B}
          \infer0{A\to B\vdash A\to B}
          \hypo{\hla{A}\vdash\hla{A}}
        \infer2{A\to B, A\vdash B}
      \infer2{A\to B,\lnot B,A\vdash\bot}
    \infer1{A\to B,\lnot B\vdash\lnot A}
  \infer1{A\to B\vdash\lnot B\to\lnot A}
\infer1{\vdash (A\to B)\to(\lnot B\to\lnot A)}
\end{prooftree}\]
\mbox{}\hfill $\displaystyle\frac{}{\hla{A}\vdash\hla{A}}$
}
\only<25>{
\[\begin{prooftree}
        \infer0{\lnot B\vdash\lnot B}
          \infer0{A\to B\vdash A\to B}
          \infer0{\hla{A}\vdash\hla{A}}
        \infer2{A\to B, A\vdash B}
      \infer2{A\to B,\lnot B,A\vdash\bot}
    \infer1{A\to B,\lnot B\vdash\lnot A}
  \infer1{A\to B\vdash\lnot B\to\lnot A}
\infer1{\vdash (A\to B)\to(\lnot B\to\lnot A)}
\end{prooftree}\]
\mbox{}\hfill $\displaystyle\frac{}{\hla{A}\vdash\hla{A}}$
}
\only<26>{
\[\begin{prooftree}
        \infer0{\lnot B\vdash\lnot B}
          \infer0{A\to B\vdash A\to B}
          \infer0{A\vdash A}
        \infer2{A\to B, A\vdash B}
      \infer2{A\to B,\lnot B,A\vdash\bot}
    \infer1{A\to B,\lnot B\vdash\lnot A}
  \infer1{A\to B\vdash\lnot B\to\lnot A}
\infer1{\vdash (A\to B)\to(\lnot B\to\lnot A)}
\end{prooftree}\]
Quod erat demonstrandum.
}
\end{frame}

\begin{frame}[t]
\vspace{0.5em}
\strong{Darstellung: Gentzen-Style}

\parspace
Die Ausführung des Beweisbaums führt zu einem hohen Schreibaufwand,
insofern dieselben Antezedenzen bis zu ihrer Tilgung wieder und wieder
notiert werden müssen. Dementsprechend überlegt man sich Kurzformen.

\parspace
\begin{center}
\begin{tabular}{l@{\qquad\quad}l}
$\begin{prooftree}
        \infer0{1\vdash\neg B}
          \infer0{2\vdash A\cond B}
          \infer0{3\vdash A}
        \infer2{2, 3\vdash B}
      \infer2{1, 2, 3\vdash \bot}
    \infer1{1, 2\vdash \neg A}
  \infer1{2\vdash \neg B\cond\neg A}
\infer1{\vdash (A\cond B)\cond (\neg B\cond\neg A)}
\end{prooftree}$
&
$\begin{prooftree}
        \infer0[\infernote{1}]{\neg B}
          \infer0[\infernote{2}]{A\cond B}
          \infer0[\infernote{3}]{A}
        \infer2{B}
      \infer2{\bot}
    \infer1[\infernote{\discharge 3}]{\neg A}
  \infer1[\infernote{\discharge 1}]{\neg B\cond\neg A}
 \infer1[\infernote{\discharge 2}]{(A\cond B)\cond (\neg B\cond\neg A)}
\end{prooftree}$
\end{tabular}
\end{center}

\parspace
Die linke Form kürzt die Antezedenzen durch Nummern ab. In der rechten
entfallen sie gänzlich, ihre Nummern erscheinen dafür einmal bei ihrer
Annahme und einmal bei ihrer Tilgung.
\end{frame}

\begin{frame}[t]
\vspace{0.5em}
\strong{Darstellung: Suppes-Style}

\parspace
Eine weitere, sehr systematische Darstellung setzt den Beweis aus
einer Liste von Tabellenzeilen zusammen.

\begin{center}
\begin{tabular}{cclcl}
\toprule
\strong{\small Nr.} & \strong{\small Abh.}
& \strong{\small Aussage} & \strong{\small Regel}
& \strong{\small auf}\\
\midrule
1 & 1 & $\neg B$ & hypo &\\
2 & 2 & $(A\cond B)$ & hypo &\\
3 & 3 & $A$ & hypo &\\
4 & 2, 3 & $B$ & subj elim & 2, 3\\
5 & 1, 2, 3 & $\bot$ & neg elim & 1, 4\\
6 & 1, 2 & $\neg A$ & neg intro & 5\\
7 & 2 & $\neg B\cond\neg A$ & subj intro & 6\\
8 & $\emptyset$ & $(A\cond B)\cond (\neg B\cond\neg A)$ & subj intro & 7\\
\bottomrule
\end{tabular}
\end{center}

\vspace{1em}
Jede Zeile enthält eine Aussage zuzüglich der Information, wie und woraus
sie abgeleitet wurde. Vor der Aussage befinden sich die
Zeilennummern der Annahmen, von denen sie abhängt. Der Vergleich mit dem
Beweisbaum verrät, dass in jeder Zeile eine Sequenz aufgeführt ist, die
Abhängigkeiten dabei nichts anderes als ihre Antezedenzen sind.

\end{frame}

\begin{frame}[t,fragile]

\vspace{0.5em}
\strong{Maschinengestützes Beweisen}

\vspace{2em}
Zum Ausschließen von Flüchtigkeitsfehlern nutzen wir
\emph{maschinengestütztes Beweisen}.
Das Programm \texttt{nd.py}, ein minimalistischer
Beweisprüfer, erhält die folgende ASCII-Eingabe, die den Beweis
im Suppes"=Style wiedergibt.

\begin{lstlisting}[xleftmargin=4em]
1. 1 |- ~B, hypo.
2. 2 |- A -> B, hypo.
3. 3 |- A, hypo.
4. 2, 3 |- B, subj_elim 2 3.
5. 1, 2, 3 |- false, neg_elim 1 4.
6. 2, 1 |- ~A, neg_intro 5.
7. 2 |- ~B -> ~A, subj_intro 6.
8. |- (A -> B) -> (~B -> ~A), subj_intro 7.
\end{lstlisting}

Diese wird dazu mit einem Texteditor in eine Datei \texttt{Beweise.txt}
geschrieben. Der anschließende Programmaufruf

\begin{lstlisting}[xleftmargin=4em]
nd.py Beweise.txt
\end{lstlisting}

prüft den Beweis daraufhin.
\end{frame}

\begin{frame}[t]
\vspace{0.5em}
\strong{Darstellung: Fitch-Style}

\parspace
Einige bevorzugen die Darstellung nach Fitch, engl.
\emph{Fitch notation} oder \emph{Fitch-style proof} genannt.
Bei dieser wird durch jede Annahme ein neuer, von einer senkrechten
Linie umfasster Bereich eröffnet, in dem sie zur Verfügung steht. Die
Annahme steht am Anfang des Bereichs über einer waagerechten Linie. Der
Bereich endet mit der Tilgung der Annahme.

\[\begin{nd}
\open
  \hypo {1} {A\cond B}
  \open
    \hypo {2} {\neg B}
    \open
    \hypo {3} {A}
    \have {4} {B} \ie{1,3}
    \have {5} {\bot} \ne{2,4}
    \close
  \have {6} {\neg A} \ni{5}
  \close
\have {7} {\neg B\cond\neg A} \ii{6}
\close
\have {8} {(A\cond B)\cond (\neg B\cond\neg A)} \ii{7}
\end{nd}\]
\end{frame}

\begin{frame}[t]
\vspace{0.5em}
\strong{Darstellung: In Worten}

\parspace
Zu guter Letzt verbleibt die klassische Darstellung der Beweisführung
aufzuführen. Die in Worten. Sie zeichnet sich durch die Auslassung mühseliger
technischer Details und blumige Formulierungen aus, soll aber genug Information
enthalten, dass man im Zweifel die zuvor gezeigte Formalisierung des Beweises
erstellen und verifizieren kann.

\parspace
\strong{Satz.} \emph{Es gilt $(A\cond B)\cond (\lnot B\cond\lnot A)$ für
beliebige Aussagen $A,B$.}

\parspace
\strong{Beweis.}
Unter der Annahme $A\cond B$ ist $\lnot B\cond\lnot A$ zu zeigen,
unter der weiteren Annahme $\lnot B$ also $\lnot A$. Dazu leiten wir
aus $A$ einen Widerspruch ab. Aus $A\cond B$ und $A$ folgt zunächst $B$.
Aus $\lnot B$ und $B$ folgt daraufhin bereits der Widerspruch.\,\qedsymbol\pause

\parspace
Streng genommen handelt es sich hier um ein Theoremschema. Erst mit
der Einsetzung konkreter Formeln für die Schemavariablen $A,B$ entsteht
ein eigentliches Theorem. Mit den Setzungen $A:=(p\land q)$ und $B:=q$
erhält man bpsw.
\[(p\land q\cond p)\cond (\lnot p\cond\lnot (p\land q)).\]
Und weil $p\land q\cond p$ ebenfalls beweisbar ist, folgt sogleich
\[\lnot p\cond\lnot (p\land q).\]
\end{frame}

\begin{frame}
\centerheadline{Zulässige Regeln}
\end{frame}

\begin{frame}
Natürliches Schließen zeigt sich nicht nur für den Beweis von Aussagen
dienlich. Bewegt man sich auf der metalogischen Ebene, kann es auch
zur Ableitung neuer Regeln Verwendung finden. Eine Schlussregel, die
man durch eine metalogische Überlegung zeigt, wollen wir als
\emph{zulässig} bezeichnen. Ein Beispiel dafür ist die

\begin{block}{Kontraposition}
\[\dfrac{\Gamma\vdash A\cond B}{\Gamma\vdash\lnot B\cond\lnot A}\]
\end{block}

Nämlich findet sich der kurze Beweisbaum:

\[\begin{prooftree}
  \hypo{\Gamma\vdash A\cond B}
  \infer0[\inote{bereits gezeigt}]{\vdash (A\cond B)\cond (\lnot B\cond\lnot A)}
\infer2[\inote{subj elim}]{\Gamma\vdash\lnot B\cond\lnot A}
\end{prooftree}\]

Die Überlegung fällt unter das folgende allgemeine Prinzip.
\begin{block}{Lemma}
Wenn $\vdash A\cond B$ ableitbar ist, dann ist $\dfrac{\Gamma\vdash A}{\Gamma\vdash B}$ zulässig.
\end{block}
\end{frame}

\begin{frame}
Deraufhin erhält man sogleich den

\begin{block}{Modus tollens}
\[\dfrac{\Gamma\vdash A\to B\qquad\Gamma'\vdash\lnot B}{\Gamma,\Gamma'\vdash\lnot A}\]
\end{block}

Dies bestätigt sich durch den kurzen Beweisbaum:
\[\begin{prooftree}
    \hypo{\Gamma\vdash A\cond B}
  \infer1{\Gamma\vdash\lnot B\cond\lnot A}
  \hypo{\Gamma'\vdash\lnot B}
\infer2{\Gamma,\Gamma'\vdash\lnot A}
\end{prooftree}\]

Summa summarum steht eine Vorgehensweise zur Ableitung der klassischen
Schlussformen zur Verfügung.
\end{frame}

\begin{frame}
\centerheadline{Klassische Semantik der Aussagenlogik}
\end{frame}

\begin{frame}
Bislang trat Logik als syntaktisches System auf. Um die Wahrheit
von Aussagen bzw. die Gültigkeit von Argumenten beurteilen zu können,
muss den Formeln eine inhaltlichte Bedeutung zukommen.\pause

\begin{block}{Definition. (Wertung)}
Eine \emph{Wertung} V, engl. \emph{valuation}, ordnet jeder
Aussagenvariable (Primformel) $P$ einen Wahrheitswert $V(P)\in\{0,1\}$ zu.
\end{block}
\end{frame}

\begin{frame}
\begin{block}{Definition. (Interpretation)}
Wir definieren die \emph{Interpretation} $I = (V)$ als Erweiterung der
Wertung $V$ auf sämtliche Formeln rekursiv gemäß
\[\begin{array}{@{}r@{}l@{\qquad}r@{}l@{}}
I(P) &{}:= V(P), & I(A\land B) &{}:= I(A)\;\texttt{and}\; I(B)\\[2pt]
I(\lnot A) &{}:= \texttt{not}\; I(A), & I(A\lor B) &{}:= I(A)\;\texttt{or}\; I(B)\\[2pt]
I(\bot) &{}:= 0, & I(A\cond B) &{}:= (\texttt{not}\; I(A))\;\texttt{or}\; I(B),\\[2pt]
I(\top) &{}:= 1, & I(A\bicond B) &{}:=  I(A\cond B)\;\texttt{and}\; I(B\cond A),
\end{array}\]
wobei \texttt{not}, \texttt{and}, \texttt{or}
die mit der Wahrheitstafel
\begin{center}
\begin{tabular}{ccccc}
\toprule
$a$ & $b$ & $\texttt{not}\;a$ & $a\;\texttt{and}\;b$ & $a\;\texttt{or}\;b$\\
\midrule[\heavyrulewidth]
0 & 0 & 1 & 0 & 0 \\
0 & 1 & 1 & 0 & 1 \\
1 & 0 & 0 & 0 & 1 \\
1 & 1 & 0 & 1 & 1 \\
\bottomrule
\end{tabular}
\end{center}
festgelegten Wahrheitsfunktionen seien.
\end{block}
\end{frame}

\begin{frame}
\begin{block}{Definition. (Erfüllung)}
Eine Interpretation $I$ \emph{erfüllt} eine Formel $A$, kurz
$I\models A$, wenn $I(A)=1$ ist. Des Weiteren \emph{erfüllt} $I$ einen
Kontext $\Gamma$, kurz $I\models\Gamma$, wenn $I(A)=1$ für jede Formel
$A\in\Gamma$.
\end{block}\pause

\begin{block}{Definition. (Gültigkeit einer Sequenz)}
Eine Sequenz $\Gamma\vdash A$ heißt \emph{gültig}, wenn
\[\forall I\colon (I\models\Gamma)\Rightarrow (I\models A),\]
also wenn jede Interpretation, die $\Gamma$ erfüllt, auch $A$ erfüllt.
\end{block}\pause

\begin{block}{Definition. (Gültigkeit einer Regel)}
Eine Schlussregel
\[\dfrac{\Gamma_1\vdash A_1\qquad\ldots\qquad\Gamma_n\vdash A_n}{\Gamma_1,\ldots,\Gamma_n\vdash B}\]
heißt \emph{gültig}, wenn
\[(\Gamma_1\models A_1)\land\ldots\land (\Gamma_n\models A_n)\Rightarrow (\Gamma_1\ldots,\Gamma_n\models B).\]
\end{block}
\end{frame}

\begin{frame}
\centerheadline{Korrektheit des natürlichen Schließens}
\end{frame}

\begin{frame}
\begin{block}{Satz. (Korrektheit des natürlichen Schließens)}
\emph{Ist die Sequenz $\Gamma\vdash A$ ableitbar, folgt $\Gamma\models A$.}
\end{block}

\strong{Beweisskizze.} Zu zeigen gilt, dass sämtliche Schlussregeln
gültig sind. Dann ergibt sich per struktureller Induktion über den Aufbau
des Beweises, dass alle ableitbaren Sequenzen gültig sind. Die
Induktionsanfänge ergeben sich hierbei durch die Regeln ohne Prämissen,
also die Regel zur Einführung der Grundsequenzen.\pause

\parspace
Zur Einführung von Grundsequzen. Es ist $A\models A$ zu zeigen,
also $(I\models A)\Rightarrow (I\models A)$ für jede Interpretation $I$.
Unter der Annahme $I\models A$ gilt $I\models A$ aber breits trivial.\pause

\parspace
Zur Beseitigung der Konjunktion. Zu zeigen ist, dass $\Gamma\models A$
aus $\Gamma\models A\land B$ folgt. Sei $J$ fest, aber beliebig.
Es gelte $J\models\Gamma$. Zu zeigen ist $J\models A$. Laut Prämisse
gilt
\[\forall I\colon (I\models\Gamma)\Rightarrow (I\models A\land B).\]
Wir spezialisieren $I:=J$, und folgern damit $J\models A\land B$.
Laut Semantik bedeutet dies $J(A\land B)=1$, also
$(J(A)\;\texttt{and}\;J(B))=1$, womit sowohl $J(A)=1$ als auch $J(B)=1$
gilt. Ergo gilt $J\models A$.\pause

\parspace
Die Gültigkeit der restlichen Regeln zu bestätigen, bleibt der Leserin
als Übung überlassen.\,\qedsymbol
\end{frame}

\begin{frame}
Sobald die später erläuterten Regeln für Quantoren zur Verfügung stehen,
lassen sich die Beweise übrigens ebenfalls im natürlichen Schließens
formalisieren, was allerdings in der metalogischen Ebene stattfindet.\pause

\parspace
Bspw. findet sich der Beweisbaum:
\[\begin{prooftree}
                      \infer0[\inote{hypo}]{1\vdash (\Gamma\models A\land B)}
                    \infer1[\inote{laut Def. Gültigkeit}]{1\vdash\forall I\colon (I\models\Gamma)\Rightarrow (I\models A\land B)}
                  \infer1[\inote{uq elim}]{1\vdash (J\models\Gamma)\Rightarrow (J\models A\land B)}
                  \infer0[\inote{hypo}]{2\vdash (J\models\Gamma)}
               \infer2[\inote{subj elim}]{1, 2\vdash (J\models A\land B)}
             \infer1[\inote{laut Semantik}]{1, 2\vdash J(A)=1\land J(B)=1}
           \infer1[\inote{conj elim}]{1, 2\vdash J(A)=1}
         \infer1[\inote{laut Def. Erfüllung}]{1, 2\vdash (J\models A)}
       \infer1[\inote{subj intro}]{1\vdash (J\models\Gamma)\Rightarrow (J\models A)}
     \infer1[\inote{uq intro}]{1\vdash\forall I\colon (I\models\Gamma)\Rightarrow (I\models A)}
  \infer1[\inote{laut Def. Gültigkeit}]{1\vdash (\Gamma\models A)}
\infer1[\inote{subj intro}]{\vdash (\Gamma\models A\land B)\Rightarrow (\Gamma\models A)}
\end{prooftree}\]
\end{frame}

\begin{frame}
\centerheadline{Übungsaufgaben}
\end{frame}

\begin{frame}[t,fragile]
\vspace{3em}
\strong{Aufgabe 1.}
Der Beweis von $A\land B\cond A$ ist gesucht.\pause

\parspace
Beweisbaum:

\[\begin{prooftree}
    \infer0[\inote{hypo}]{A\land B\vdash A\land B}
  \infer1[\inote{conj eliml}]{A\land B\vdash A}
\infer1[\inote{subj intro}]{\vdash A\land B\cond A}
\end{prooftree}\]

\vspace{1em}
Eingabe für den Beweisprüfer:

\begin{lstlisting}[xleftmargin=4em]
1. 1 |- A /\ B, hypo.
2. 1 |- A, conj_eliml 1.
3. |- A /\ B -> A, subj_intro 2.
\end{lstlisting}
\end{frame}

\begin{frame}[t,fragile]
\vspace{3em}
\strong{Aufgabe 2.}
Der Beweis von $A\lor B\cond B\lor A$ ist gesucht.\pause

\parspace
Beweisbaum:

\[\begin{prooftree}
    \infer0[\inote{hypo}]{A\lor B\vdash A\lor B}
      \infer0[\inote{hypo}]{A\vdash A}
    \infer1[\inote{disj intror}]{A\vdash B\lor A}
      \infer0[\inote{hypo}]{B\vdash B}
    \infer1[\inote{disj introl}]{B\vdash B\lor A}
  \infer3[\inote{disj elim}]{A\lor B\vdash B\lor A}
\infer1[\inote{subj intro}]{\vdash A\lor B\cond B\lor A}
\end{prooftree}\]

\vspace{1em}
Eingabe für den Beweisprüfer:

\begin{lstlisting}[xleftmargin=4em]
1. 1 |- A \/ B, hypo.
2. 2 |- A, hypo.
3. 2 |- B \/ A, disj_intror 2.
4. 4 |- B, hypo.
5. 4 |- B \/ A, disj_introl 4.
6. 1 |- B \/ A, disj_elim 1 3 5.
7. |- A \/ B -> B \/ A, subj_intro 6.
\end{lstlisting}
\end{frame}

\begin{frame}[t]
\vspace{3em}
\strong{Literatur}
\begin{itemize}
\item Gerhard Gentzen: \emph{Untersuchungen über das logische Schließen}.\\
In: \emph{Mathematische Zeitschrift}. Band 39, 1935, S. 176--210, S. 405--431.
\href{https://gdz.sub.uni-goettingen.de/id/PPN266833020_0039}{Band 39 online via GDZ}.
\item Gerhard Gentzen: \emph{Die Widerspruchsfreiheit der reinen
Zahlentheorie}.\\
In: \emph{Mathematische Annalen}. Band 112, 1936, S. 493--565.
\href{https://gdz.sub.uni-goettingen.de/id/PPN235181684_0112}{Band 112 online via GDZ}.
---Zum KdnS von Sequenzen.
\item Samuel Mimram: \emph{Program = Proof}.
\href{https://www.lix.polytechnique.fr/Labo/Samuel.Mimram/publications/}{Link (Open Access)}.
\item Ingebrigt Johansson: \emph{Der Minimalkalkül, ein reduzierter
intuitionistischer Formalismus}. In: \emph{Compositio Mathematica}.
Band 4, 1937, S. 119--136.
\item Andrzej Indrzejczak: \href{https://iep.utm.edu/natural-deduction/}{\emph{Natural Deduction}}.
In: \emph{The Internet Encyclopedia of Philosophy}.
\item Francis Jeffry Pelletier, Allen Hazen:
\href{https://plato.stanford.edu/entries/natural-deduction/}{\emph{Natural Deduction Systems in Logic}}.
In: \emph{The Stanford Encyclopedia of Philosophy}.
\item Eckart Menzler-Trott: \emph{Gentzens Problem. Mathematische Logik
  im nationalsozialistischen Deutschland}. Birkhäuser, Basel 2001.
\end{itemize}
\end{frame}

\begin{frame}
Ende.
\vfill\hfill\modest{August 2025}\\
\hfill\modest{Creative Commons CC0 1.0}
\end{frame}


\end{document}


