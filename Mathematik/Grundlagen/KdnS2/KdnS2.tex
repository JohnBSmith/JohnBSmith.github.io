\documentclass[8pt]{beamer}
\usetheme{Antibes}
\useinnertheme{rectangles}
\useoutertheme{infolines}
\usepackage[utf8]{inputenc}
\usepackage[T1]{fontenc}
\usepackage[ngerman]{babel}

% Patch the look of +, = in arev
\usefonttheme{serif}

\usepackage{arev}
% Patch punctuation to be upright
\DeclareMathSymbol{.}{\mathpunct}{operators}{`.}
\DeclareMathSymbol{,}{\mathpunct}{operators}{`,}

\usepackage{amsmath}
\usepackage{amssymb}
\usepackage{proof}
\usepackage{booktabs}

\setbeamertemplate{footline}{%
\begin{beamercolorbox}[ht=3.0ex,dp=1ex]{title in head/foot}
\hfill\footnotesize\insertpagenumber\enspace\enspace\end{beamercolorbox}}

\definecolor{bluegreen1}{rgb}{0.0,0.20,0.28}
\definecolor{bluegreen2}{rgb}{0.0,0.20,0.28}
\setbeamercolor*{palette primary}{fg=white,bg=bluegreen1}
\setbeamercolor*{palette secondary}{fg=white,bg=bluegreen2}
\setbeamercolor*{palette tertiary}{fg=white,bg=bluegreen2}
\setbeamercolor{itemize item}{fg=black}
\setbeamercolor{block title}{bg=bluegreen2}
\newcommand{\modest}[1]{{\small\color{gray}#1}}
\hypersetup{colorlinks,urlcolor=magenta}

\newcommand{\unit}[1]{\mathrm{#1}}
\newcommand{\strong}[1]{\textsf{\textbf{#1}}}
\newcommand{\defiff}{\quad:\Longleftrightarrow\quad}
\newcommand{\infernote}[1]{\!\text{\footnotesize #1}}
\renewcommand{\qedsymbol}{\ensuremath{\Box}}
\newcommand{\discharge}[1]{$\sim$#1}
\newcommand{\centerheadline}[1]{%
  \begin{center}\strong{#1}\end{center}}
\newcommand{\parspace}{\vspace{0.8em}}
\newcommand{\cond}{\rightarrow}
\newcommand{\Z}{\mathbb Z}
\newcommand{\R}{\mathbb R}
\newcommand{\FV}{\mathrm{FV}}

\title{Der Kalkül des natürlichen Schließens}
\subtitle{Teil 2: Prädikatenlogik}
\date{}

\begin{document}

\begin{frame}
\maketitle
\end{frame}

\begin{frame}
\centerheadline{Freie Variablen}
\end{frame}

\begin{frame}
In der Formel $A\land P(x)$ taucht die Variable $x$ frei auf,
wogegen sie in der Formel $\forall x\colon A\land P(x)$ nur gebunden
vorkommt.\pause

\parspace
Wir bezeichnen mit $\FV(A)$ die Menge der freien Variablen
der Formel $A$. Man kann sie rekursiv über den Formelaufbau definieren:
\begin{align*}
& \FV(A\land B) = \FV(A)\cup\FV(B), && \FV(A\cond B) = \FV(A)\cup\FV(B),\\
& \FV(A\lor B)  = \FV(A)\cup\FV(B), && \FV(A\leftrightarrow B) = \FV(A)\cup\FV(B),\\
& \FV(\forall x\colon A) = \FV(A)\setminus\{x\}, && \FV(\neg A) = \FV(A),\\
& \FV(\exists x\colon A) = \FV(A)\setminus\{x\}, && \FV(\bot) = \FV(\top) = \emptyset
\end{align*}
und
\[\FV(P(t_1,\ldots,t_n)) = \FV(t_1)\cup\ldots\cup\FV(t_n),\]
wobei $\FV(t)$ die im Term $t$ auftauchenden Variablen sind, was
man abermals rekursiv definieren kann.
\end{frame}

\begin{frame}
\centerheadline{Substitution}
\end{frame}

\begin{frame}
Man bezeichnet mit $A[x:=t]$, auch $A[t/x]$ geschrieben, die Formel,
die dadurch entsteht, dass in der Formel $A$ jedes freie Vorkommen
der Variable $x$ gegen den Term $t$ ersetzt wird.\pause

\parspace
Beispielsweise resultiert $(x^2\ge 0)[x:=a+1]$ in der Formel $(a+1)^2\ge 0$.\pause

\parspace
Zusätzlich soll gelten, dass bei $A[x:=t]$ die Variablen in $t$ nicht
eingefangen im Sinne von überschattet werden dürfen.\pause

\parspace
Beispielsweise resultiert $(\exists x\colon P(x,y))[y:=a]$ in
$\exists x\colon P(x,a)$. Jedoch ist $(\exists x\colon P(x,y))[y:=x]$
nicht $\exists x\colon P(x,x)$, sondern $\exists u\colon P(u,x)$.
Es musste die gebundene Variable hier also umbenannt werden, damit
es nicht zur Überschattung kommt.
\end{frame}

\begin{frame}
\centerheadline{Universalquantifizierungen}
\end{frame}

\begin{frame}
Von »Alle Enten haben einen Schnabel« dürfen wir schließen auf
»Donald hat einen Schnabel«.\pause
\begin{block}{Beseitigung der Universalquantifizierung}
\[\dfrac{\Gamma\vdash\forall x\colon A}{\Gamma\vdash A[x:=t]}\]
\end{block}\pause
Sei $E(x)$ beispielsweise das Prädikat »$x$ ist eine Ente« und $S(x)$
das Prädikat »$x$ hat einen Schnabel«. Man darf schließen:
\[
\infer[\infernote{$\cond$B}]{\vdash S(\text{Donald})}{
  \infer[\infernote{$\forall$B}]{\vdash E(\text{Donald})\cond S(\text{Donald})}{
    \vdash\forall x\colon E(x)\cond S(x)}
& \vdash E(\text{Donald})}
\]
\end{frame}

\begin{frame}
Es gelte $0<x$. Addiert man $1$ zu beiden Seiten,
erhält man $1<x+1$. Wegen $0<1$ gilt folglich erst recht
$0<x+1$. Wir haben also
\[0<x\vdash 0<x+1.\]\pause
Da keine weiteren Annahmen über $x$ gemacht wurden, dürfen
wir eine Universalquantifizierung einführen: 
\[
\infer[\infernote{$\forall$E}]{\vdash\forall x\colon 0<x\cond 0<x+1}{
  \infer[\infernote{$\cond$E}]{\vdash 0<x\cond 0<x+1}{
    0<x\vdash 0<x+1}}
\]\pause

\begin{footnotesize}
\strong{Bemerkung.} Stillschweigend vorausgesetzt wurde die Gestalt des
Diskursuniversums $U$, über das alle Quantifizierungen laufen. Das heißt,
mit $\forall x$ ist immer $\forall x\in U$ gemeint. Wir haben hier
beispielsweise $U=\Z$ oder $U=\R$. Ein Diskursuniversum muss immer
nichtleer sein, da man sonst weitere Spitzfindigkeiten berücksichtigen
müsste.
\end{footnotesize}
\end{frame}

\begin{frame}
\begin{block}{Einführung der Universalquantifizierung}
\[\dfrac{\Gamma\vdash A}{\Gamma\vdash\forall x\colon A}(x\notin\mathrm{FV}(\Gamma))\]
\end{block}
\end{frame}

\begin{frame}
Bei der allgemeinen Regel zur gemachten Schlussform ist zu beachten,
dass die Variable, über die allquantifiziert wird, nicht in einer
Aussage im Kontext vorkommt. Hierdurch könnte nämlich die allgemeine
Gültigkeit der Aussage, über die quantifiziert wird, eingeschränkt
werden.\pause

\parspace
Beispiel für das Unheil einer irrigen Ableitung:
\[
\infer[\infernote{$\forall$B, x:=0}]{\vdash 1 < 0}{
  \infer[\infernote{$\cond$B}]{\vdash\forall x\colon 1 < x}{
    \infer[\infernote{$\forall$B, x:=2}]{\vdash 1 < 2\cond\forall x\colon 1 < x}{
      \infer[\infernote{$\forall$E}]{\vdash \forall x\colon 1 < x\cond\forall x\colon 1 < x}{
        \infer[\infernote{$\cond$E}]{\vdash 1 < x\cond\forall x\colon 1 < x}{
          \infer[\infernote{$\forall$E irrig}]{1 < x\vdash\forall x\colon 1 < x}{
            \infer[\infernote{Ax}]{1 < x\vdash 1 < x}{}}}}}
  & \infer{\vdash 1<2}{\text{\footnotesize unwichtig}}
  }
}
\]
\end{frame}

\begin{frame}[t]
\vspace{7em}
Beispiel einer korrekten Ableitung:

\vspace{1em}
\onslide*<1>{\[
\vdash A\subseteq A
\]}
\onslide*<2>{\[
\infer[\infernote{per Def.}]{\vdash A\subseteq A}{
  \vdash\forall x\colon x\in A\cond x\in A}
\]}
\onslide*<3>{\[
\infer[\infernote{per Def.}]{\vdash A\subseteq A}{
  \infer[\infernote{$\forall$E}]{\vdash\forall x\colon x\in A\cond x\in A}{
    \vdash x\in A\cond x\in A}}
\]}
\onslide*<4>{\[
\infer[\infernote{per Def.}]{\vdash A\subseteq A}{
  \infer[\infernote{$\forall$E}]{\vdash\forall x\colon x\in A\cond x\in A}{
    \infer[\infernote{$\cond$E}]{\vdash x\in A\cond x\in A}{
      \infer[\infernote{Ax}]{x\in A\vdash x\in A}{}}}}
\]}
\end{frame}

\begin{frame}[t]
\vspace{6em}
Beispiel einer korrekten Ableitung in der reinen Logik:

\vspace{1em}
\onslide*<1>{\[
\vdash A\land (\forall x\colon P(x))\cond (\forall x\colon A\land P(x))
\]}
\onslide*<2>{\[
\infer[\infernote{$\cond$E}]{\vdash A\land (\forall x\colon P(x))\cond (\forall x\colon A\land P(x))}{
  1\vdash \forall x\colon A\land P(x)}
\]}
\onslide*<3>{\[
\infer[\infernote{$\cond$E}]{\vdash A\land (\forall x\colon P(x))\cond (\forall x\colon A\land P(x))}{
  \infer[\infernote{$\forall$E}]{1\vdash \forall x\colon A\land P(x)}{
    1\vdash A\land P(x)}}
\]}
\onslide*<4>{\[
\infer[\infernote{$\cond$E}]{\vdash A\land (\forall x\colon P(x))\cond (\forall x\colon A\land P(x))}{
  \infer[\infernote{$\forall$E}]{1\vdash \forall x\colon A\land P(x)}{
    \infer[\infernote{$\land$E}]{1\vdash A\land P(x)}{
      1\vdash A
    & 1\vdash P(x)}}}
\]}
\onslide*<5>{\[
\infer[\infernote{$\cond$E}]{\vdash A\land (\forall x\colon P(x))\cond (\forall x\colon A\land P(x))}{
  \infer[\infernote{$\forall$E}]{1\vdash \forall x\colon A\land P(x)}{
    \infer[\infernote{$\land$E}]{1\vdash A\land P(x)}{
      \infer[\infernote{$\land$B}]{1\vdash A}{
        \infer[\infernote{Ax}]{1\equiv A\land(\forall x\colon P(x))}{}}
    & \infer[\infernote{$\forall$B}]{1\vdash P(x)}{
        1\vdash\forall x\colon P(x)}}}}
\]}
\onslide*<6>{\[
\infer[\infernote{$\cond$E}]{\vdash A\land (\forall x\colon P(x))\cond (\forall x\colon A\land P(x))}{
  \infer[\infernote{$\forall$E}]{1\vdash \forall x\colon A\land P(x)}{
    \infer[\infernote{$\land$E}]{1\vdash A\land P(x)}{
      \infer[\infernote{$\land$B}]{1\vdash A}{
        \infer[\infernote{Ax}]{1\equiv A\land(\forall x\colon P(x))}{}}
    & \infer[\infernote{$\forall$B}]{1\vdash P(x)}{
        \infer[\infernote{$\land$B}]{1\vdash\forall x\colon P(x)}{
          \infer[\infernote{Ax}]{1\equiv A\land (\forall x\colon P(x))}{}}}}}}
\]}
\end{frame}

\begin{frame}
\centerheadline{Existenzquantifizierungen}
\end{frame}

\begin{frame}
Es ist ja $2\in\Z$ und $2^2=4$. Ergo ist $x:=2$ ein Zeuge für
$\exists x\colon x\in\Z\land x^2=4$.\pause

\parspace
Zur Einführung muss also ein Zeuge der Existenzaussage
gefunden werden.
\begin{block}{Einführung der Existenzquantifizierung}
\[\dfrac{\Gamma\vdash A[x:=t]}{\Gamma\vdash\exists x\colon A}\]
\end{block}
\end{frame}

\begin{frame}
Die Existenzaussage $\exists x\colon P(x)$ wird wie folgt beseitigt.
Mir ihr liegt ein Zeuge $a$ mit $P(a)$ vor. Unter Verwendung
der Aussage $P(a)$ wird eine Aussage $B$ hergeleitet, in der $a$ nicht
frei vorkommt. Somit ist $B$ unabhängig vom gewählten Zeugen, was
erforderlich ist, da unbekannt bleibt, welcher der Zeugen vorliegt.\pause

\begin{block}{Beseitigung der Existenzquantifizierung}
\[\dfrac{\Gamma\vdash\exists a\colon A\qquad\Gamma',A\vdash B}{\Gamma,\Gamma'\vdash B}
(a\notin\mathrm{FV}(\Gamma,\Gamma',B))\]
\end{block}
\end{frame}

\begin{frame}[t]
\vspace{6em}
Beispiel einer korrekten Ableitung:

\vspace{1em}
\onslide*<1>{\[
\vdash A\land (\exists x\colon P(x))\cond (\exists x\colon A\land P(x))
\]}
\onslide*<2>{\[
\infer[\infernote{$\cond$E}]{\vdash A\land (\exists x\colon P(x))\cond (\exists x\colon A\land P(x))}{
  1\vdash\exists x\colon A\land P(x)}
\]}
\onslide*<3>{\[
\infer[\infernote{$\cond$E}]{\vdash A\land (\exists x\colon P(x))\cond (\exists x\colon A\land P(x))}{
  \infer[\infernote{$\exists$B}]{1\vdash\exists x\colon A\land P(x)}{
    1\vdash\exists a\colon P(a)
  & 1, P(a)\vdash\exists x\colon A\land P(x)}}
\]}
\onslide*<4>{\[
\infer[\infernote{$\cond$E}]{\vdash A\land (\exists x\colon P(x))\cond (\exists x\colon A\land P(x))}{
  \infer[\infernote{$\exists$B}]{1\vdash\exists x\colon A\land P(x)}{
    \infer[\infernote{$\land$B}]{1\vdash\exists a\colon P(a)}{
      \infer[\infernote{Ax}]{1\equiv A\land (\exists x\colon P(x))}{}}
  & \infer[\infernote{$\exists$E}]{1, P(a)\vdash\exists x\colon A\land P(x)}{
      1, P(a)\vdash A\land P(a)}}}
\]}
\onslide*<5>{\[
\infer[\infernote{$\cond$E}]{\vdash A\land (\exists x\colon P(x))\cond (\exists x\colon A\land P(x))}{
  \infer[\infernote{$\exists$B}]{1\vdash\exists x\colon A\land P(x)}{
    \infer[\infernote{$\land$B}]{1\vdash\exists a\colon P(a)}{
      \infer[\infernote{Ax}]{1\equiv A\land (\exists x\colon P(x))}{}}
  & \infer[\infernote{$\exists$E}]{1, P(a)\vdash\exists x\colon A\land P(x)}{
      \infer[\infernote{$\land$B}]{1, P(a)\vdash A\land P(a)}{
        1\vdash A
      & \infer[\infernote{Ax}]{P(a)\vdash P(a)}{}}}}}
\]}
\onslide*<6>{\[
\infer[\infernote{$\cond$E}]{\vdash A\land (\exists x\colon P(x))\cond (\exists x\colon A\land P(x))}{
  \infer[\infernote{$\exists$B}]{1\vdash\exists x\colon A\land P(x)}{
    \infer[\infernote{$\land$B}]{1\vdash\exists a\colon P(a)}{
      \infer[\infernote{Ax}]{1\equiv A\land (\exists x\colon P(x))}{}}
  & \infer[\infernote{$\exists$E}]{1, P(a)\vdash\exists x\colon A\land P(x)}{
      \infer[\infernote{$\land$B}]{1, P(a)\vdash A\land P(a)}{
        \infer[\infernote{$\land$B}]{1\vdash A}{
          \infer[\infernote{Ax}]{1\equiv A\land (\exists x\colon P(x))}{}}
      & \infer[\infernote{Ax}]{P(a)\vdash P(a)}{}}}}}
\]

\parspace
Hier ist $a$ mit $A\land P(a)$ ein Zeuge für $\exists x\colon A\land P(x)$.
}
\end{frame}

\begin{frame}
Ende.
\vfill\hfill\modest{November 2022}\\
\hfill\modest{Creative Commons CC0 1.0}
\end{frame}


\end{document}


