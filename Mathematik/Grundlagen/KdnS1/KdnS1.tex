\documentclass[8pt]{beamer}
\usetheme{Antibes}
\useinnertheme{rectangles}
\useoutertheme{infolines}
\usepackage[utf8]{inputenc}
\usepackage[T1]{fontenc}
\usepackage[ngerman]{babel}

% Patch the look of +, = in arev
\usefonttheme{serif}

\usepackage{arev}
% Patch punctuation to be upright
\DeclareMathSymbol{.}{\mathpunct}{operators}{`.}
\DeclareMathSymbol{,}{\mathpunct}{operators}{`,}

\usepackage{amsmath}
\usepackage{amssymb}
\usepackage{proof}
\usepackage{booktabs}
\usepackage{fitch}

\setbeamertemplate{footline}{%
\begin{beamercolorbox}[ht=3.0ex,dp=1ex]{title in head/foot}
\hfill\footnotesize\insertpagenumber\enspace\enspace\end{beamercolorbox}}

\definecolor{bluegreen1}{rgb}{0.0,0.20,0.28}
\definecolor{bluegreen2}{rgb}{0.0,0.20,0.28}
\setbeamercolor*{palette primary}{fg=white,bg=bluegreen1}
\setbeamercolor*{palette secondary}{fg=white,bg=bluegreen2}
\setbeamercolor*{palette tertiary}{fg=white,bg=bluegreen2}
\setbeamercolor{itemize item}{fg=black}
\setbeamercolor{block title}{bg=bluegreen2}
\newcommand{\modest}[1]{{\small\color{gray}#1}}
\hypersetup{colorlinks,urlcolor=magenta}

\newcommand{\unit}[1]{\mathrm{#1}}
\newcommand{\strong}[1]{\textsf{\textbf{#1}}}
\newcommand{\defiff}{\quad:\Longleftrightarrow\quad}
\newcommand{\infernote}[1]{\!\text{\footnotesize #1}}
\renewcommand{\qedsymbol}{\ensuremath{\Box}}
\newcommand{\discharge}[1]{$\sim$#1}
\newcommand{\centerheadline}[1]{%
  \begin{center}\strong{#1}\end{center}}
\newcommand{\parspace}{\vspace{0.8em}}
\newcommand{\cond}{\rightarrow}

\title{Natürliches Schließen}
\subtitle{Teil 1: Aussagenlogik}
\date{}

\begin{document}

\begin{frame}
\maketitle
\end{frame}

\begin{frame}
\centerheadline{Schließen}
\end{frame}

\begin{frame}
Allgemein wird beim logischen Schließen in jedem Schritt aus Prämissen
eine Konklusion gewonnen.

\parspace
So darf von den Prämissen »Stockholm liegt in Schweden« und
»Gustaf wohnt in Stockholm« auf die Konklusion »Gustaf wohnt in Schweden«
geschlossen werden.\pause

\parspace
Man notiert
\[\dfrac{\text{Stockholm liegt in Schweden}\qquad\qquad\text{Gustaf wohnt in Stockholm}}{
\text{Gustaf wohnt in Schweden}}.\]
\end{frame}

\begin{frame}
\centerheadline{Konjunktionen}
\end{frame}

\begin{frame}[t]
\vspace{2em}
Sind zwei Aussagen wahr, dürfen wir auf ihre Konjunktion schließen:
\[\dfrac{\text{Es schneit}\qquad\text{Es regnet}}{
  \text{Es fällt Schneeregen}}\]\pause
Von der Konjunktion dürfen wir auf die Bestandteile schließen:
\[\dfrac{\text{Es fällt Schneeregen}}{\text{Es regnet}}\qquad\quad
  \dfrac{\text{Es fällt Schneeregen}}{\text{Es schneit}}\]\pause
Insofern bestehen Schlussregeln:
\begin{center}
\begin{tabular}{c@{\qquad\quad}c}
\strong{\small Einführung der Konjunktion}
& \strong{\small Beseitigung der Konjunktion}\\[6pt]
$\dfrac{A\qquad B}{A\land B}$
& $\dfrac{A\land B}{A}\qquad\dfrac{A\land B}{B}$\\
\end{tabular}
\end{center}\pause

\parspace
\begin{footnotesize}
\strong{Bemerkung.} Es stehen hier Variablen $A,B,C,D$ nicht nur
für atomare logische Variablen, sondern für beliebige Formeln.
Insofern handelt es sich bei ihnen eigentlich um Metavariablen.
In der Metatheorie der Logik würde man eher $\varphi,\chi,\psi$
schreiben.
\end{footnotesize}
\end{frame}

\begin{frame}
\centerheadline{Sequenzen}
\end{frame}

\begin{frame}
Nun kann es aber sein, dass eine Aussage lediglich relativ unter
Voraussetzung einer anderen Aussage wahr ist. Angenommen, es schneit
zwischen 0:00 und 2:00 Uhr. Ebenfalls angenommen, es regnet zwischen
1:00 und 3:00 Uhr. Dann dürfen wir nicht einfach schließen, dass
Schneeregen fällt. Der Schneeregen fällt lediglich zwischen
1:00 und 2:00 Uhr.\pause

\parspace
Berücksichtigung relativer Wahrheit notiert man als \emph{Sequenz}
\[A_1,A_2,\ldots,A_n\vdash B.\]
Dies soll bedeuten: Unter den Voraussetzungen $A_1$ bis $A_n$
gilt die Aussage $B$.\pause

\parspace
Es sind die $A_1$ bis $A_n$ die \emph{Antezedenzen} oder
\emph{Vorderformeln} der Sequenz. Die Aussage $B$ ist die
\emph{Sukzedenz} oder \emph{Hinterformel} der Sequenz.
\end{frame}

\begin{frame}
Wir kürzen $\Gamma := [A_1,A_2,\ldots,A_n]$ ab, wobei $\Gamma$
als Liste oder auch als endliche Menge von Antezedenzen betrachtet wird.
Die Sequenz lautet nun
\[\Gamma\vdash B.\]
Man bezeichnet $\Gamma$ auch als den \emph{Kontext}.\pause

\parspace
Wir schreiben
$\Gamma,\Gamma'\vdash B$ für $\Gamma\cup\Gamma'\vdash B$ bzw.
\[A_1,A_2,\ldots,A_n,A_1',A_2',\ldots,A_n'\vdash B.\]

\parspace
\begin{footnotesize}
\strong{Bemerkung.} Zu bemerken ist, dass für Sequenzen und Ableitbarkeit
dieselbe Symbolik benutzt wird. Streng genommen ist Ableitbarkeit zunächst
allerdings ein metalogischer, von Sequenzen zu unterscheidender Begriff.
Zur Schaffung von Klarheit ist es daher günstig, Sequenzen bei der
metalogischen Untersuchung eine andere Symbolik zu geben,
beispielsweise $\Gamma\succ B$.
\end{footnotesize}
\end{frame}

\begin{frame}
\centerheadline{Nochmals Konjunktionen}
\end{frame}

\begin{frame}
Die gemachte Betrachtung führt zur folgenden allgemeinen Regel.
\begin{block}{Schlussregel zur Einführung der Konjunktion}
\[\dfrac{\Gamma\vdash A\qquad\Gamma'\vdash B}{\Gamma,\Gamma'\vdash A\land B}\]
\end{block}\pause
Man setze beispielsweise
\begin{align*}
& A := \text{Es schneit},\\
& B := \text{Es regnet},\\
& \Gamma := [\text{Es ist zwischen 0:00 und 2:00 Uhr}],\\
& \Gamma' := [\text{Es ist zwischen 1:00 und 3:00 Uhr}].
\end{align*}
\end{frame}

\begin{frame}
\begin{block}{Schlussregeln zur Beseitigung der Konjunktion}
\[\dfrac{\Gamma\vdash A\land B}{\Gamma\vdash A},\qquad
\dfrac{\Gamma\vdash A\land B}{\Gamma\vdash B}\]
\end{block}
\end{frame}

\begin{frame}
\centerheadline{Implikationen}
\end{frame}

\begin{frame}
Aus der Antezedenz »Es regnet« leiten wir ab, »Die Erde wird nass«.
Insofern ist die Implikation »Wenn es regnet, dann wird die Erde
nass« gültig.\pause

\parspace
Vergessen wir für einen Moment, wie es sich mit Regen verhält.
Ist die genannte Implikation und außerdem »Es regnet« als
Antezedenz gegeben, leiten wir trotzdem ab, »Die Erde wird nass«.\pause

\parspace
Insofern bestehen Regeln des Schließens.
\begin{block}{Schlussregeln}
\begin{center}
\begin{tabular}{c@{\qquad\quad}c}
\strong{\small Einführung der Implikation}
& \strong{\small Beseitigung der Implikation}\\[6pt]
$\dfrac{\Gamma,A\vdash B}{\Gamma\vdash A\cond B}$
& $\dfrac{\Gamma\vdash A\cond B\qquad\Gamma'\vdash A}{\Gamma,\Gamma'\vdash B}$\\
\end{tabular}
\end{center}
\end{block}\pause

\parspace
Bemerkung:
\begin{itemize}
\item Die Einführung der Implikation spiegelt das Deduktionstheorem wider.
\item Die Beseitigung der Implikation stellt den Modus ponens dar.
\end{itemize}
\end{frame}

\begin{frame}
\centerheadline{Ein Axiom}
\end{frame}

\begin{frame}
Manche Sequenzen darf man als offenkundig gültig betrachten. Beispielsweise
\[\text{Es regnet}\vdash\text{Es regnet}.\]\pause
Insofern darf man $A\vdash A$ für jede Aussage $A$
axiomatisch als gültig befinden. Eine Sequenz dieser Form heißt
\emph{Grundsequenz}.\pause

\parspace
Wir können diesen Umstand auch als Schlussregel darstellen, bei der
die Sequenz aus dem Nichts abgeleitet wird.
\begin{block}{Axiom}
\[\dfrac{}{A\vdash A}\]
\end{block}
\end{frame}

\begin{frame}
Eine Bemerkung noch. Offenbar dürfen Sequenzen abgeschwächt werden.
Das heißt, wenn
$\Gamma\vdash A$ gilt, dann gilt erst recht $\Gamma,\Gamma'\vdash A$.\pause
\begin{block}{Abschwächungsregel}
\[\dfrac{\Gamma\vdash A}{\Gamma,\Gamma'\vdash A}\]
\end{block}\pause
Setzt man in der Einführungsregel der Konjunktion $\Gamma=\Gamma'$, vereinfacht sie sich zu
\[\dfrac{\Gamma\vdash A\qquad\Gamma\vdash B}{\Gamma\vdash A\land B}.\]
Aus dieser speziellen Form kann die allgemeine Regel allerdings immer noch
hervorgebracht werden, denn per Abschwächungsregel findet sich:
\[
\infer{\Gamma,\Gamma'\vdash A\land B}{
  \infer{\Gamma,\Gamma'\vdash A}{\Gamma\vdash A}
& \infer{\Gamma,\Gamma'\vdash B}{\Gamma'\vdash B}}
\]
\end{frame}

\begin{frame}
\centerheadline{Ein erster Beweis}
\end{frame}

\begin{frame}
\begin{center}
\begin{tabular}{cl}
\toprule
\strong{Abkürzung} & \strong{Bedeutung}\\
\midrule
Ax & Axiom\\
$\land$E & Einführung der Konjunktion\\
$\land$B & Beseitigung der Konjunktion\\
$\cond$E & Einführung der Implikation\\
$\cond$B & Beseitigung der Implikation\\
\bottomrule
\end{tabular}
\end{center}
\end{frame}

\begin{frame}[t]
\strong{Darstellung: Beweisbaum}

\vspace{4em}
Als erste Aufgabe soll der Beweis der Aussage
$(A\cond B)\land A\cond B$ erbracht werden, die dem
Modus ponens entspricht.

\parspace
Es sind verschiedene Vorgehensweisen denkbar.

\parspace
Zum Erfolg führt oft schon, den Beweis rückwärts zu konstruieren:

\vspace{1em}
\onslide*<1>{
\[\vdash (A\cond B)\land A\cond B\]}
\onslide*<2>{
\[\infer[\infernote{$\cond$E}]{\vdash (A\cond B)\land A\cond B}{
  (A\cond B)\land A\vdash B}\]
}
\onslide*<3>{
\[\infer[\infernote{$\cond$E}]{\vdash (A\cond B)\land A\cond B}{
  \infer[\infernote{$\cond$B}]{(A\cond B)\land A\vdash B}{
    (A\cond B)\land A\vdash A\cond B
  & (A\cond B)\land A\vdash A}}\]
}
\onslide*<4>{
\[\infer[\infernote{$\cond$E}]{\vdash (A\cond B)\land A\cond B}{
  \infer[\infernote{$\cond$B}]{(A\cond B)\land A\vdash B}{
    \infer[\infernote{$\land$B}]{(A\cond B)\land A\vdash A\cond B}{
      \infer[\infernote{Ax}]{(A\cond B)\land A\vdash (A\cond B)\land A}{}}
  & \infer[\infernote{$\land$B}]{(A\cond B)\land A\vdash A}{
      \infer[\infernote{Ax}]{(A\cond B)\land A\vdash (A\cond B)\land A}{}}}}\]
}
\end{frame}

\begin{frame}[t]
\strong{Darstellung: Beweisbaum in Kurzform}

\vspace{3em}
Ein recht hoher Schreibaufwand.\pause

\parspace
Kürzen wir die Antezedenzen doch durch Nummern ab:
\[\infer{\vdash (A\cond B)\land A\cond B}{
  \infer{1\vdash B}{
    \infer{1\vdash A\cond B}{
      \infer{1\equiv (A\cond B)\land A}{}}
  & \infer{1\vdash A}{
      \infer{1\equiv (A\cond B)\land A}{}}}}\]\pause

Oder noch kürzer:
\[\infer[\infernote{\discharge 1}]{(A\cond B)\land A\cond B}{
  \infer{B}{
    \infer{A\cond B}{
      \infer[\infernote{1}]{(A\cond B)\land A}{}}
  & \infer{A}{
      \infer[\infernote{1}]{(A\cond B)\land A}{}}}}\]
Hier bezeichnet 1 die Annahme~1 und \discharge{1} ihre Tilgung.
\end{frame}

\begin{frame}[t]
\strong{Darstellung: Liste}

\vspace{3em}
Eine weitere, sehr systematische Darstellung setzt den Beweis aus
einer Liste von Tabellenzeilen zusammen.

\vspace{1em}
\begin{center}
\begin{tabular}{cclcl}
\toprule
\strong{\small Abhängigkeiten} & \strong{\small Nr.}
& \strong{\small Aussage} & \strong{\small Regel}
& \strong{\small angewendet auf}\\
\midrule
1 & 1 & $(A\cond B)\land A$ & Ax &\\
1 & 2 & $A\cond B$ & $\land$B & 1\\
1 & 3 & $A$ & $\land$B & 1\\
1 & 4 & $B$ & $\cond$B & 2, 3\\
$\emptyset$ & 5 & $(A\cond B)\land A\cond B$ & $\cond$E & 1, 4\\
\bottomrule
\end{tabular}
\end{center}\pause

\vspace{1em}
Jede Zeile enthält eine Aussage und dahinter zusätzlich die
Information, wie und woraus die Aussage abgeleitet wurde.
Die Abhängigkeiten sind die Antezedenzen der Sequenz.
\end{frame}

\begin{frame}[t]
\strong{Darstellung: Fitch-Style}

\vspace{3em}
Einige bevorzugen die Darstellung nach Fitch, engl.
\emph{Fitch notation} oder \emph{Fitch-style proof} genannt.
Bei dieser wird durch jede Annahme ein neuer, von einer senkrechten
Linie umfasster Bereich der Gültigkeit eröffnet. Die Annahme steht am
Anfang des Bereichs über einer waagerechten Linie.
\[\begin{nd}
\open
\hypo {1} {(A\cond B)\land A}
  \have {2} {A\cond B} \ae{1}
  \have {3} {A} \ae{1}
  \have {4} {B} \ie{2,3}
\close
\have {5} {(A\cond B)\land A\cond B} \ii{1,4}
\end{nd}\]
\end{frame}

\begin{frame}[t]
\strong{Darstellung: In Worten}

\vspace{4em}
\strong{Satz.} \emph{Die Formel $(A\cond B)\land A\cond B$
ist allgemeingültig.}

\parspace
\strong{Beweis.}
Angenommen, es gilt $(A\cond B)\land A$. Dann liegt
sowohl $A\cond B$ als auch $A$ vor. Per Modus ponens erhält
man somit $B$. Die Einführung der Implikation
$(A\cond B)\land A\cond B$ tilgt schließlich die
gemachte Annahme.\,\qedsymbol\pause

\vspace{2em}
Die Klassische Darstellung der Beweisführung. Charakteristisch sind
blumige Formulierungen und vor allem die Auslassung mühseliger
technischer Details.
\end{frame}

\begin{frame}
\centerheadline{Negationen}
\end{frame}

\begin{frame}
Die Verneinung $\neg A$ kann allgemein durch $A\cond\bot$
definiert werden, da beide Formeln im Minimalkalkül, und damit sowohl
in klassischer als auch in intuitionistischer Logik äquivalent sind.
Hiermit kann die Einführung und Beseitigung der Verneinung auf die
Regeln der Implikation zurückgeführt werden.\pause

\parspace
Man kommt zum folgenden Resultat.
\begin{block}{Schlussregeln}
\begin{center}
\begin{tabular}{c@{\qquad\quad}c}
\strong{\small Einführung der Negation}
& \strong{\small Beseitigung der Negation}\\[6pt]
$\dfrac{\Gamma,A\vdash\bot}{\Gamma\vdash\neg A}$
& $\dfrac{\Gamma\vdash\neg A\qquad\Gamma'\vdash A}{\Gamma,\Gamma'\vdash\bot}$\\
\end{tabular}
\end{center}
\end{block}
\end{frame}

\begin{frame}
\centerheadline{Zulässige Schlussregeln}
\end{frame}

\begin{frame}
Schlussregeln ermöglichen nicht nur die Schaffung von Beweisen,
sie bieten auch ein Werkzeug zur Herleitung weiterer Schlussregeln.
Hergeleitete bezeichnet man als \emph{zulässige Schlussregeln}.\pause{}

\parspace
Umstände wie der folgende sind beispielsweise
dem Verständnis dienlich.

\begin{block}{Lemma}
Ist $A\cond B$ eine allgemeingültige
Formel, dann ist
\[\dfrac{\Gamma\vdash A}{\Gamma\vdash B}\]
eine zulässige Regel.
\end{block}\pause

\parspace
\strong{Beweis.} Dies wird bereits durch den kurzen Beweisbaum
\[\dfrac{\vdash A\cond B\qquad \Gamma\vdash A}{
  \Gamma\vdash B}\;\,\infernote{$\cond$B}\]
bestätigt.\,\qedsymbol
\end{frame}

\begin{frame}
Wir werden uns nun wie Münchhausen an den eigenen Haaren aus dem
Sumpf ziehen. Unterfangen ist die Herleitung der zulässigen Regeln
\emph{Kontraposition} und \emph{Modus tollens}.\pause

\parspace
Hierfür muss zunächst bewiesen werden, dass es sich bei
\[(A\cond B)\cond (\neg B\cond\neg A)\]
um eine allgemeingültige Formel handelt.
\end{frame}

\begin{frame}[t]

\vspace{6em}
Konstruktion des Beweisbaums:

\vspace{1em}
\onslide*<1>{
\[\vdash (A\cond B)\cond (\neg B\cond\neg A)\]
}
\onslide*<2>{
\[\infer[\infernote{$\cond$E}]{
  \vdash (A\cond B)\cond (\neg B\cond\neg A)
}{
  A\cond B\vdash \neg B\cond\neg A}
\]
}
\onslide*<3>{
\[\infer[\infernote{$\cond$E}]{
  \vdash (A\cond B)\cond (\neg B\cond\neg A)
}{
  \infer[\infernote{$\cond$E}]{
    A\cond B\vdash \neg B\cond\neg A
  }{
    A\cond B, \neg B\vdash \neg A}}
\]
}
\onslide*<4>{
\[\infer[\infernote{$\cond$E}]{
  \vdash (A\cond B)\cond (\neg B\cond\neg A)
}{
  \infer[\infernote{$\cond$E}]{
    A\cond B\vdash \neg B\cond\neg A
  }{
    \infer[\infernote{$\neg$E}]{A\cond B, \neg B\vdash \neg A}{
      A\cond B, \neg B, A\vdash \bot}}}
\]
}
\onslide*<5>{
\[\infer[\infernote{$\cond$E}]{
  \vdash (A\cond B)\cond (\neg B\cond\neg A)
}{
  \infer[\infernote{$\cond$E}]{
    A\cond B\vdash \neg B\cond\neg A
  }{
    \infer[\infernote{$\neg$E}]{A\cond B, \neg B\vdash \neg A}{
      \infer[\infernote{$\neg$B}]{A\cond B, \neg B, A\vdash \bot}{
        \neg B\vdash\neg B
      & A\cond B, A\vdash B}}}}
\]
}
\onslide*<6>{
\[\infer[\infernote{$\cond$E}]{
  \vdash (A\cond B)\cond (\neg B\cond\neg A)
}{
  \infer[\infernote{$\cond$E}]{
    A\cond B\vdash \neg B\cond\neg A
  }{
    \infer[\infernote{$\neg$E}]{A\cond B, \neg B\vdash \neg A}{
      \infer[\infernote{$\neg$B}]{A\cond B, \neg B, A\vdash \bot}{
        \infer[\infernote{Ax}]{\neg B\vdash\neg B}{}
      & \infer[\infernote{$\cond$B}]{A\cond B, A\vdash B}{
          \infer[\infernote{Ax}]{A\cond B\vdash A\cond B}{}
        & \infer[\infernote{Ax}]{A\vdash A}{}}}}}}
\]
}
\end{frame}

\begin{frame}
Beweisbaum in Kurzform:

\parspace
\begin{center}
\begin{tabular}{l@{\qquad\quad}l}
$\infer{\vdash (A\cond B)\cond (\neg B\cond\neg A)}{
  \infer{2\vdash \neg B\cond\neg A}{
    \infer{1, 2\vdash \neg A}{
      \infer{1, 2, 3\vdash \bot}{
        \infer{1\equiv\neg B}{}
      & \infer{2, 3\vdash B}{
          \infer{2\equiv A\cond B}{}
        & \infer{3\equiv A}{}}}}}}$
&
$\infer[\infernote{\discharge 2}]{\vdash (A\cond B)\cond (\neg B\cond\neg A)}{
  \infer[\infernote{\discharge 1}]{\neg B\cond\neg A}{
    \infer[\infernote{\discharge 3}]{\neg A}{
      \infer{\bot}{
        \infer[\infernote{1}]{\neg B}{}
      & \infer{B}{
          \infer[\infernote{2}]{A\cond B}{}
        & \infer[\infernote{3}]{A}{}}}}}}$
\end{tabular}
\end{center}

\parspace
Liste:
\begin{center}
\begin{tabular}{cclcl}
\toprule
\strong{\small Abh.} & \strong{\small Nr.}
& \strong{\small Aussage} & \strong{\small Regel}
& \strong{\small auf}\\
\midrule
1 & 1 & $\neg B$ & Ax &\\
2 & 2 & $(A\cond B)$ & Ax &\\
3 & 3 & $A$ & Ax &\\
2, 3 & 4 & $B$ & $\cond$B & 2, 3\\
1, 2, 3 & 5 & $\bot$ & $\neg$B & 1, 4\\
1, 2 & 6 & $\neg A$ & $\neg$E & 5\\
2 & 7 & $\neg B\cond\neg A$ & $\cond$E & 6\\
$\emptyset$ & 8 & $(A\cond B)\cond (\neg B\cond\neg A)$ & $\cond$E & 7\\
\bottomrule
\end{tabular}
\end{center}
\end{frame}

\begin{frame}
Fitch-Style:
\[\begin{nd}
\open
  \hypo {1} {A\cond B}
  \open
    \hypo {2} {\neg B}
    \open
    \hypo {3} {A}
    \have {4} {B} \ie{1,3}
    \have {5} {\bot} \ne{2,4}
    \close
  \have {6} {\neg A} \ni{5}
  \close
\have {7} {\neg B\cond\neg A} \ii{6}
\close
\have {8} {(A\cond B)\cond (\neg B\cond\neg A)} \ii{7}
\end{nd}\]
\end{frame}

\begin{frame}
In Worten:

\parspace
\strong{Beweis.}
Um $(A\cond B)\cond (\neg B\cond\neg A)$
zu zeigen, ist unter Annahme von sowohl $A\cond B$
als auch $\neg B$ als auch $A$ ein Widerspruch abzuleiten.
Zunächst erhält man $B$ aus $A\cond B$ und $A$
per Modus ponens. Nun steht $\neg B$ bereits im Widerspruch
zu $B$.\,\qedsymbol
\end{frame}

\begin{frame}
Die Regel der Kontraposition erhält man nun unverzüglich unter
Anwendung des Lemmas als Korollar.

\begin{block}{Kontraposition}
\[\dfrac{\Gamma\vdash A\cond B}{\Gamma\vdash\neg B\cond\neg A}\]
\end{block}\pause

Deraufhin erhält man sogleich:
\begin{block}{Modus tollens}
\[\dfrac{\Gamma\vdash A\cond B\qquad\Gamma'\vdash\neg B}{\Gamma,\Gamma'\vdash\neg A}\]
\end{block}\pause
\strong{Beweis.} Zeigt der kurze Beweisbaum:
\[\infer[\infernote{$\cond$B}]{\Gamma,\Gamma'\vdash\neg A}{
  \infer{\Gamma\vdash\neg B\cond\neg A}{\Gamma\vdash A\cond B}
& \Gamma'\vdash\neg B}
\]
\end{frame}

\begin{frame}
\centerheadline{Disjunktionen}
\end{frame}

\begin{frame}
Die Regeln zur Einführung sind eigentlich erwartungsgemäß.
\begin{block}{Einführung der Disjunktion}
\[\dfrac{\Gamma\vdash A}{\Gamma\vdash A\lor B},\qquad
\dfrac{\Gamma\vdash B}{\Gamma\vdash A\lor B}\]
\end{block}\pause

\parspace
Bei der Beseitigung macht man eine Fallunterscheidung. Wenn nämlich
eine Aussage $C$ sowohl aus $A$ als auch aus $B$ folgt, darf man doch
als gültig befinden, dass $C$ auch aus der Disjunktion $A\lor B$ folgt.\pause

\begin{block}{Beseitigung der Disjunktion}
\[\dfrac{\Gamma\vdash A\lor B\qquad\Gamma',A\vdash C\qquad\Gamma'',B\vdash C}{
  \Gamma,\Gamma',\Gamma''\vdash C}\]
\end{block}
\end{frame}

\begin{frame}[t]

\vspace{6em}
Es drängt sich auf, als erste Aufgabe die Allgemeingültigkeit von
$A\lor B\cond B\lor A$ zu bestätigen.

\parspace
Konstruktion des Beweisbaums:

\vspace{1em}
\onslide*<1>{\[
\vdash A\lor B\cond B\lor A
\]}
\onslide*<2>{\[
\infer[\infernote{$\cond$E}]{\vdash A\lor B\cond B\lor A}{
  A\lor B\vdash B\lor A}
\]}
\onslide*<3>{\[
\infer[\infernote{$\cond$E}]{\vdash A\lor B\cond B\lor A}{
  \infer[\infernote{$\lor$B}]{A\lor B\vdash B\lor A}{
    \infer[\infernote{Ax}]{A\lor B\vdash A\lor B}{}
  & A\vdash B\lor A
  & B\vdash B\lor A}}
\]}
\onslide*<4>{\[
\infer[\infernote{$\cond$E}]{\vdash A\lor B\cond B\lor A}{
  \infer[\infernote{$\lor$B}]{A\lor B\vdash B\lor A}{
    \infer[\infernote{Ax}]{A\lor B\vdash A\lor B}{}
  & \infer[\infernote{$\lor$E}]{A\vdash B\lor A}{
      \infer[\infernote{Ax}]{A\vdash A}{}}
  & \infer[\infernote{$\lor$E}]{B\vdash B\lor A}{
      \infer[\infernote{Ax}]{B\vdash B}{}}}}
\]}
\end{frame}

\begin{frame}
\centerheadline{Erweiterungen}
\end{frame}

\begin{frame}[t]
\strong{Intuitionistische Logik}

\vspace{5em}
Die bisherigen Schlussregeln stellen den Minimalkalkül dar.
Um die intuitionistische Logik zu erhalten, bedarf es noch einer
weiteren Regel.\pause
\begin{block}{Ex falso quotlibet}
\[\dfrac{\Gamma\vdash\bot}{\Gamma\vdash A}\]
\end{block}
\end{frame}

\begin{frame}[t]
\strong{Klassische Logik}

\vspace{3em}
Zur Klassischen Logik führt:
\begin{block}{Beseitigung der Doppelnegation}
\[\dfrac{\Gamma\vdash\neg\neg A}{\Gamma\vdash A}\]
\end{block}\pause

\parspace
Ebenfalls zur klassischen Logik führt ein zusätzliches Axiom:
\begin{block}{Satz vom ausgeschlossenen Dritten}
\[\dfrac{}{\vdash A\lor\neg A}\]
\end{block}
\end{frame}

\begin{frame}[t]
\vspace{3em}
\strong{Literatur}
\begin{itemize}
\item Gerhard Gentzen: \emph{Untersuchungen über das logische Schließen}.\\
In: \emph{Mathematische Zeitschrift}. Band 39, 1935, S. 176--210, S. 405--431.
\href{https://gdz.sub.uni-goettingen.de/id/PPN266833020_0039}{Band 39 online via GDZ}.
\item Gerhard Gentzen: \emph{Die Widerspruchsfreiheit der reinen
Zahlentheorie}.\\
In: \emph{Mathematische Annalen}. Band 112, 1936, S. 493--565.
\href{https://gdz.sub.uni-goettingen.de/id/PPN235181684_0112}{Band 112 online via GDZ}.
---Zum KdnS von Sequenzen.
\item Samuel Mimram: \emph{Program = Proof}.
\href{https://www.lix.polytechnique.fr/Labo/Samuel.Mimram/publications/}{Link (Open Access)}.
\item Ingebrigt Johansson: \emph{Der Minimalkalkül, ein reduzierter
intuitionistischer Formalismus}. In: \emph{Compositio Mathematica}.
Band 4, 1937, S. 119--136.
\item Andrzej Indrzejczak: \href{https://iep.utm.edu/natural-deduction/}{\emph{Natural Deduction}}.
In: \emph{The Internet Encyclopedia of Philosophy}.
\item Francis Jeffry Pelletier, Allen Hazen:
\href{https://plato.stanford.edu/entries/natural-deduction/}{\emph{Natural Deduction Systems in Logic}}.
In: \emph{The Stanford Encyclopedia of Philosophy}.
\item Eckart Menzler-Trott: \emph{Gentzens Problem. Mathematische Logik
  im nationalsozialistischen Deutschland}. Birkhäuser, Basel 2001.
\end{itemize}
\end{frame}

\begin{frame}
Ende.
\vfill\hfill\modest{November 2022}\\
\hfill\modest{Creative Commons CC0 1.0}
\end{frame}


\end{document}


