
\chapter{Ansätze zur Problemlösung}

\section{Substitution}

\subsection{Quadratische Gleichungen}%
\index{quadratische Gleichung}\index{Gleichung!quadratische}

Vorgelegt ist eine quadratische Gleichung in Normalform
\begin{equation}\index{Normalform!einer quadratischen Gleichung}
x^2+px+q = 0.
\end{equation}
Interessanterweise lässt sich der lineare Term $px$ durch Darstellung
der Gleichung über eine Translation $x=u+d$ eliminieren. Einsetzen
dieser Substitution bringt
\begin{align}
0 &= (u+d)^2+p(u+d)+q = u^2+2ud+d^2+pu+pd+q\\
&= u^2+(p+2d)u+(d^2+pd+q).
\end{align}
Setzt man nun $p+2d=0$, dann ergibt sich daraus $d=-p/2$ und somit
\begin{align}
q' &:= d^2+pd+q = \Big(-\frac{p}{2}\Big)^2-p\cdot\frac{p}{2}+q
= \frac{p^2}{4}-\frac{p^2}{2}+q\\
&= \frac{p^2}{4}-2\frac{p^2}{4}+q = -\frac{p^2}{4}+q.
\end{align}
Zu lösen ist nunmehr die quadratische Gleichung
\begin{equation}
u^2+q' = 0.
\end{equation}
Aber das ist ganz einfach, die Lösungen sind $u_1=+\sqrt{-q'}$
und $u_2=-\sqrt{-q'}$, sofern $q'\le 0$,
bzw. äquivalent $-q'\ge 0$. Wir schreiben kurz $u=\pm\sqrt{-q'}$.
Resubstitution von $u=x-d$ und $q'$ führt zu
\begin{equation}
x-d = x+\frac{p}{2} = \pm\sqrt{\frac{p^2}{4}-q} = \pm\frac{1}{2}\sqrt{p^2-4q}.
\end{equation}
Man erhält die Lösungsformel
\begin{equation}
x = -\frac{p}{2}\pm\frac{1}{2}\sqrt{p^2-4q}.
\end{equation}

\subsection{Biquadratische Gleichungen}%
\index{biquadratische Gleichung}\index{Gleichung!biquadratische}
Die biquadratische Gleichung
\begin{equation}
x^4+px^2+q = 0
\end{equation}
lässt sich über die Substitution $u=x^2$ auf die quadratische Gleichung
\begin{equation}
u^2+pu+q
\end{equation}
reduzieren. Für $p^2-4q\ge 0$ ergeben sich zwei Lösungen $u_1,u_2$,
wobei eventuell $u_1=u_2$ ist. Nun können sich bis zu vier Lösungen
für die ursprüngliche Gleichung ergeben. Das ist der Fall,
wenn $u_1\ne u_2$ und $u_1,u_2>0$. Dann
ergibt sich
\begin{equation}
x_1=\sqrt{u_1},\quad x_2=-\sqrt{u_1},\quad
x_3=\sqrt{u_2},\quad x_4=-\sqrt{u_2}
\end{equation}
