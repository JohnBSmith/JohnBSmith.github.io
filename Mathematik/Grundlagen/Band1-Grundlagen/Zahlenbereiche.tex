
\chapter{Zahlenbereiche}

\section{Ganze Zahlen}

\subsection{Konstruktion}

\begin{Definition}[Ganze Zahlen]\mbox{}\\*
Auf $\N_0\times\N_0$ wird die folgende Äquivalenrelation definiert:
\[(x_1,y_1)\sim (x_2,y_2) \defiff x_1+y_2 = x_2+y_1.\]
Die Quotientenmenge $\Q := (\N_0\times\N_0)/{\sim}$ nennt man
die ganzen Zahlen.
\end{Definition}

\begin{Satz}[Ring der ganzen Zahlen]\mbox{}\\*
Die Operationen
\begin{align*}
[(x_1,y_1)]+[(x_2,y_2)] &:= [(x_1+x_2,y_1+y_2)],\\
[(x_1,y_1)]\cdot [(x_2,y_2)] &:= [(x_1x_2+y_1y_2,x_1y_2+x_2y_1)]
\end{align*}
sind auf $\Q$ wohldefiniert und $(\Q,+,\cdot)$ bildet einen
kommutativen unitären Ring.
\end{Satz}
\strong{Beweis.} Wohldefiniert heißt, dass das Ergebnis der Operationen
nicht von den gewählten Repräsentanten der Argumente abhängig ist.
Sei dazu $(x_1,y_1)\sim(a_1,b_1)$ und
$(x_2,y_2)\sim (a_2,b_2)$. Zu zeigen ist nun
\begin{gather*}
(x_1+x_2,y_1+y_2)\sim (a_1+a_2,b_1+b_2)\\
\iff (x_1+x_2)+(b_1+b_2) = (a_1+a_2)+(y_1+y_2).
\end{gather*}
Gemäß Voraussetzung ist $x_1+b_1=a_1+y_1$ und $x_2+b_2=a_2+y_2$.
Man bekommt damit auf der linken Seite
\[x_1+x_2+b_1+b_2 = a_1+y_1+a_2+y_2,\]
was wiederum mit der rechten Seite übereinstimmt.

Mit der Multiplikation verhält es sich etwas komplizierter.
Zu Vereinfachung wird zunächst gezeigt:
\begin{gather*}
[(x_1,y_1)]\cdot [(x_2,y_2)] = [(a_1,b_1)]\cdot [(x_2,y_2)]\\
\iff (x_1x_2+y_1y_2,x_1y_2+y_1x_2)\sim (a_1x_2+b_1y_2,a_1y_2+b_1x_2)\\
\iff x_1x_2+y_1y_2 + a_1y_2+b_1x_2 = a_1x_2+b_1y_2 + x_1y_2+y_1x_2\\
\iff (x_1+b_1)x_2 + (a_1+y_1)y_2 = (a_1+y_1)x_2 + (x_1+b_1)y_2.
\end{gather*}
Diese Gleichung ist gemäß Voraussetzung $(x_1,y_1)\sim (a_1,b_1)$
bzw. $x_1+b_1=a_1+y_1$ erfüllt.

Analog bestätigt man
\[[(a_1,b_1)]\cdot [(x_2,y_2)] = [(a_1,b_1)]\cdot [(a_2,b_2)].\]
Gemäß Transitivität ergibt sich somit
\[[(x_1,y_1)]\cdot [(x_2,y_2)] = [(a_1,b_1)]\cdot [(a_2,b_2)].\]
Es ist nun zu bestätigen, dass $(\Q,+)$ eine kommutative Gruppe ist.
Das Assozativgesetz:
\begin{gather*}
([(x_1,y_1)]+[(x_2,y_2)])+[(x_3,y_3)]
= [(x_1+x_2,y_1+y_2)] + [(x_3,y_3)]\\
= [(x_1+x_2+x_3,y_1+y_2+y_3)]
= [(x_1,y_1)]+[(x_2+x_3,y_2+y_3)]\\
= [(x_1,y_1)]+([(x_2,y_2)]+[(x_3,y_3)]).
\end{gather*}
Das neutrale Element ist $[(0,0)]$:
\[[(x,y)]+[(0,0)] = [(x+0,y+0)] = [(x,y)].\]
Das inverse Element zu $[(x,y)]$ ist $[(y,x)]$, denn es gilt
\begin{gather*}
[(x,y)]+[(y,x)] = [(x+y,y+x)] = [(0,0)]\\
\iff (x+y,y+x)\sim (0,0)\iff x+y+0 = y+x+0.
\end{gather*}
Das Kommutativgesetz:
\begin{gather*}
[(x_1,y_1)]+[(x_2,y_2)] = [(x_1+x_2,y_1+y_2)]
= [(x_2+x_1,y_2+y_1)]\\
= [(x_2,y_2)]+[(x_1,y_1)].
\end{gather*}
Es ist nun zu bestätigen, dass $(\Q,\cdot)$ ein kommutatives
Monoid bildet. Das Assoziativgesetz:
\begin{gather*}
([(x_1,y_1)]\cdot [(x_2,y_2)])\cdot [(x_3,y_3)]
= [(x_1x_2+y_1y_2,x_1y_2+x_2y_1)]\cdot [(x_3,y_3)]\\
= [(x_1x_2x_3+x_3y_1y_2+x_1y_2y_3+x_2y_1y_3,\;
x_1x_2y_3+y_1y_2y_3+x_1x_3y_2+x_2x_3y_1)]\\
= [(x_1,y_1)]\cdot [(x_2x_3+y_2y_3,x_2y_3+x_3y_2)]
= [(x_1,y_1)]\cdot ([(x_2,y_2)]\cdot [(x_3,y_3)]).
\end{gather*}
Das Kommutativgesetz:
\begin{gather*}
[(x_1,y_1)]\cdot [(x_2,y_2)] = [(x_1x_2+y_1y_2,\;x_1y_2+y_1x_2)]\\
= [(x_2x_1+y_2y_1,\;x_2y_1+x_1y_2)] = [(x_2,y_2)]\cdot [(x_1,y_1)].
\end{gather*}
Das neutrale Element ist $[(1,0)]$, denn es gilt
\[[(x,y)]\cdot [(1,0)] = [(x\cdot 1+y\cdot 0,\;1\cdot y+x\cdot 0)]
= [(x,y)].\]
Schließlich ist noch das Distributivgesetz zu bestätigen.
Man findet
\begin{gather*}
[(a,b)]\cdot ([(x_1,y_1)]+[(x_2,y_2)])
= [(a,b)]\cdot [(x_1+x_2,y_1+y_2)]\\
= [(ax_1+ax_2+by_1+by_2,\;ay_1+ay_2+bx_1+bx_2)]\\
= [(ax_1+by_1,ay_1+bx_1)]+[(ax_2+by_2,ay_2+bx_2)]\\
= [(a,b)]\cdot [(x_1,y_1)] + [(a,b)]\cdot [(x_2,y_2)].
\end{gather*}
Somit sind alle Axiome bestätigt.\;\qedsymbol

\begin{Definition}[Monoidhomomorphismus]\mbox{}\\*
Seien $(M,+)$ und $(M',+')$ zwei Monoide. Eine Abbildung
$\varphi\colon M\to M'$ heißt Monoidhomomorphismus, wenn
für alle $a,b\in M$ gilt
\[\varphi(a+b) = \varphi(a)+\varphi(b)\]
und $\varphi(0)=0'$ ist.
\end{Definition}
Einen injektiven Homomorphismus nennt man Monomorphismus. Ein
Monomorphismus charakterisiert eine Einbettung einer Struktur als
Unterstruktur einer anderen.

\begin{Satz}[Einbettung der natürlichen Zahlen in die ganzen]%
\label{embedding-nat-int}\mbox{}\\*
Die Abbildung $\varphi\colon\N_0\to\Z$ mit $\varphi(n)=[(n,0)]$
ist ein Monoidmonomorphismus.
\end{Satz}
\strong{Beweis.} Es ergibt sich
\[\varphi(a+b) = [(a+b,0)] = [(a,0)]+[(b,0)] = \varphi(a)+\varphi(b).\]
Außerdem ist $\varphi(0)=[(0,0)]$, und $[(0,0)]$
ist das neutrale Element von $(\Z,+)$.

Schließlich ist noch die Injektivität zu prüfen:
\begin{gather*}
[(a,0)] = \varphi(a) = \varphi(b)  = [(b,0)]\iff (a,0)\sim (b,0)\\
\iff a+0 = b+0 \iff a=b.\;\qedsymbol
\end{gather*}
Anstelle von $\varphi(n)=[(n,0)]$ darf man daher einfach schreiben
$n=[(n,0)]$. Außerdem definiert man $a-b:=a+(-b)$. Daraus
ergibt sich nun
\[[(x,y)] = [(x,0)]+[(0,y)] = [(x,0)] - [(y,0)] = x-y.\]
Die umständliche Schreibweise $[(x,y)]$ wird ab jetzt nicht
mehr benötigt.

\begin{Definition}[Totalordnung der ganzen Zahlen]\mbox{}\\*
Auf $\Z$ wird die folgende strenge Totalordnung definiert:
\[[(x_1,y_1)] < [(x_2,y_2)] \defiff x_1+y_2<x_2+y_1.\]
\end{Definition}

\begin{Satz}[Einbettung der Totalordnung]\mbox{}\\*
Die Abbildung $\varphi$ aus Satz \ref{embedding-nat-int}
genügt der Forderung
\[a<b \implies \varphi(a)<\varphi(b).\]
\end{Satz}
\strong{Beweis.} Nach den Definitionen ist
\[\varphi(a)<\varphi(b)\iff [(a,0)]<[(b,0)]\iff a+0<0+b\iff a<b.\;\qedsymbol\]


\newpage
\section{Rationale Zahlen}

\subsection{Konstruktion}

\begin{Definition}[Rationale Zahlen]\mbox{}\\*
Auf $\Z\times\N$ wird die folgende Äquivalenzrelation definiert:
\[(x_1,y_1)\sim (x_2,y_2) \defiff x_1y_2 = x_2y_1.\]
Die Quotientenmenge $\Q := (\Z\times\N)/{\sim}$ nennt man
die rationalen Zahlen.
\end{Definition}
Für die Äquivalenzklasse $[(x,y)]$ schreibt man $\frac{x}{y}$.

\begin{Satz}[Körper der rationalen Zahlen]\mbox{}\\*
Die Operationen
\[\frac{x_1}{y_1}+\frac{x_2}{y_2} := \frac{x_1y_2+x_2y_1}{y_1y_2},
\qquad\frac{x_1}{y_1}\cdot \frac{x_2}{y_2} := \frac{x_1x_2}{y_1y_2}\]
sind auf $\Q$ wohldefiniert und $(\Q,+,\cdot)$ bildet einen Körper.
\end{Satz}
\strong{Beweis.} Wohldefiniert bedeutet, dass das Ergebnis der
Operationen nicht von den gewählten Repräsentanten der Argumente
abhängig ist. Sei dazu $(a_1,b_1)\sim (x_1,y_1)$ und
$(a_2,b_2)\sim (x_2,y_2)$. Zu zeigen ist nun
\begin{align*}
&(a_1b_2+a_2b_1,b_1b_2)\sim (x_1y_2+x_2y_1,y_1y_2)\\
&\iff (a_1b_2+a_2b_1)(y_1y_2) = (x_1y_2+x_2y_1)(b_1b_2)\\
&\iff a_1b_2 y_1y_2 + a_2b_1y_1y_2 = x_1y_2b_1b_2+x_2y_1b_1b_2.
\end{align*}
Substituiert man $a_1y_1=x_1b_1$ und $a_2y_2=x_2b_2$ auf
der linken Seite der Gleichung, ergibt sich die rechte Seite.
Zu zeigen ist weiterhin
\begin{align*}
(a_1a_2,b_1b_2)\sim (x_1x_2,y_1y_2)
\iff a_1a_2y_1y_2 = x_1x_2b_1b_2.
\end{align*}
Wieder wird linke Seite der Gleichung über $a_1y_1=x_1b_1$
und $a_2y_2=x_2b_2$ in die rechte Seite überführt.
Die Wohldefiniertheit der Operationen ist damit bestätigt.

Es bleibt zu prüfen, dass $(\Q,+,\cdot)$ allen Körperaxiomen genügt.
Das neutrale Element der Addition ist $0/1$, denn es gilt
\[\frac{x}{y}+\frac{0}{1} = \frac{x\cdot 1+0\cdot y}{y\cdot 1} = \frac{x}{y}.\]
Das neutrale Element der Multiplikation ist $1/1$, denn es gilt
\[\frac{x}{y}\cdot\frac{1}{1} = \frac{x\cdot 1}{y\cdot 1} = \frac{x}{y}.\]
Die Assoziativität der Addition ergibt sich ohne größere Umstände:
\begin{align*}
\bigg(\frac{x_1}{y_1}+\frac{x_2}{y_2}\bigg)+\frac{x_3}{y_3}
&= \frac{x_1y_2+x_2y_1}{y_1y_2} + \frac{x_3}{y_3}
= \frac{x_1y_2y_3+x_2y_1y_3+x_3y_1y_2}{y_1y_2y_3},\\
\frac{x_1}{y_1}+\bigg(\frac{x_2}{y_2}+\frac{x_3}{y_3}\bigg)
&= \frac{x_1}{y_1}+\frac{x_2y_3+x_3y_2}{y_2y_3}
= \frac{x_1y_2y_3+x_2y_1y_3+x_3y_1y_2}{y_1y_2y_3}.
\end{align*}
Die Assozativität der Multiplikation ist etwas einfacher:
\[\bigg(\frac{x_1}{y_1}\cdot\frac{x_2}{y_2}\bigg)\cdot\frac{x_3}{y_3}
= \frac{x_1x_2}{y_1y_2}\cdot\frac{x_3}{y_3} = \frac{x_1x_2x_3}{y_1y_2y_3}
= \frac{x_1}{y_1}\cdot\frac{x_2x_3}{y_2y_3}
= \frac{x_1}{y_1}\cdot\bigg(\frac{x_2}{y_2}\cdot\frac{x_3}{y_3}\bigg).\]
Das Kommutativgesetz der Addition:
\[\frac{x_1}{y_1}+\frac{x_2}{y_2} = \frac{x_1y_2+x_2y_1}{y_1y_2}
= \frac{x_2y_1+x_1y_2}{y_2y_1}
= \frac{x_2}{y_2}+\frac{x_1}{y_1}.\]
Das Kommutativgesetz der Multiplikation:
\[\frac{x_1}{y_2}\cdot\frac{x_2}{y_2}
= \frac{x_1x_2}{y_1y_2} = \frac{x_2x_1}{y_2y_1}
= \frac{x_2}{y_2}\cdot\frac{x_1}{y_1}.\]
Das additiv inverse Element zu $x/y$ ist $(-x)/y$, denn es gilt
\[\frac{x}{y}+\frac{-x}{y} = \frac{xy+(-x)y}{y^2}
= \frac{0}{y^2} = \frac{0}{1}.\]
Das multiplikativ inverse Element zu $x/y$ mit $x\ne 0$
ist $y/x$, denn es gilt
\[\frac{x}{y}\cdot\frac{y}{x} = \frac{xy}{xy} = \frac{1}{1}.\]
Schließlich findet bestätigt man noch das Distributivgesetz:
\begin{align*}
&\frac{a}{b}\cdot\bigg(\frac{x_1}{y_1}+\frac{x_2}{y_2}\bigg)
= \frac{a}{b}\cdot\frac{x_1y_2+x_2y_1}{y_1y_2}
= \frac{ax_1y_2+ax_2y_1}{by_1y_2},\\
&\frac{ax_1}{by_1}+\frac{ax_2}{by_2}
= \frac{ax_1by_2+ax_2by_1}{by_1by_2}
= \frac{b}{b}\cdot\frac{ax_1y_2+ax_2y_1}{by_1y_2}.
\end{align*}
Hierbei beachtet man, dass $b/b=1/1$ das multiplikativ
neutrale Element ist.\;\qedsymbol

\newpage
\begin{Definition}[Ringhomomorphismus]\mbox{}\\*
Seien $(R,+,*)$ und $(R',+',*')$ zwei Ringe. Die Abbildung
$\varphi\colon R\to R'$ heißt Ringhomomorphismus, wenn für alle
$a,b\in R$ gilt:
\begin{align*}
\varphi(a+b) = \varphi(a)+'\varphi(b),\qquad
\varphi(a*b) = \varphi(a)*'\varphi(b).
\end{align*}
Besitzt $R$ ein Einselement $1$ und $R'$ ein Einselement $1'$,
dann nennt man $\varphi$ Eins"=erhaltend, wenn $\varphi(1)=1'$ ist.
\end{Definition}
Einen injektiver Homomorphimus wird Monomorphismus genannt. Ein
Monomorphimus charakterisiert eine Einbettung einer Unterstruktur
in eine andere Struktur.
\begin{Satz}[Einbettung der ganzen Zahlen in die rationalen]\mbox{}\\*
Sei $\varphi\colon\Z\to\Q$ mit $\varphi(z):=z/1$. Die
Abbildung $\varphi$ ist Eins"=erhaltender Ringmonomorphismus.
\end{Satz}
\strong{Beweis.} Die Erhaltung des Einselements ergibt sich
trivial. Ferner findet man
\[\varphi(a+b) = \frac{a+b}{1} = \frac{a\cdot 1+b\cdot 1}{1\cdot 1}
= \frac{a}{1}+\frac{b}{1} = \varphi(a)+\varphi(b)\]
und
\[\varphi(ab) = \frac{ab}{1} = \frac{ab}{1\cdot 1} = \frac{a}{1}\cdot\frac{b}{1}
= \varphi(a)\cdot\varphi(b).\;\qedsymbol\]
Gemäß der Einbettung können wir die ganze Zahl $z$ ab jetzt
mit der rationalen Zahl $z/1$ identifizieren. D.\,h. man schreibt
einfach $z=z/1$ anstelle von $\varphi(z)=z/1$.

\begin{Definition}[Division rationaler Zahlen]\mbox{}\\*
Wie in jedem Körper ist die Division für $a,b\in\Q$
definiert als $a/b := ab^{-1}$.
\end{Definition}
Die Division ist also gerade die Multiplikation des Kehrwertes
des Nenners:
\[\frac{x_1}{y_1}/\frac{x_2}{y_2} = \frac{x_1}{y_1}\cdot\frac{y_2}{x_2}.\]
Die Division muss natürlich mit der Notation für rationale Zahlen
kompatibel sein, sonst dürfte man nicht die gleiche Schreibweise
verwenden. Zur Unterscheidung schreiben wir Division für einen
Augenblick mit Doppelstrich als $a\doubleslash b$. Man findet
\[\frac{x}{y} = \frac{x}{1}\cdot\frac{1}{y}
= \frac{x}{1}\doubleslash\frac{y}{1} = x\doubleslash y.\]
Tatsächlich führt beides zum gleichen Ergebnis.

Da die rationalen Zahlen einen Körper bilden, gilt $a/a=aa^{-1}=1$
für jede ratonale Zahl $a$.

\begin{Satz}[Addition, Subtraktion, Multiplikation von Brüchen]\mbox{}\\*
Seien $a,b,c,d$ rationale Zahlen mit $b\ne 0$ und $d\ne 0$. Es gilt
\[\frac{a}{b}+\frac{c}{d} = \frac{ad+bc}{bd},
\qquad \frac{a}{b}-\frac{c}{d} = \frac{ad-bc}{bd},
\qquad \frac{a}{b}\cdot\frac{c}{d} = \frac{ac}{bd}.\]
\end{Satz}
Der Beweis wird dem Leser überlassen.
