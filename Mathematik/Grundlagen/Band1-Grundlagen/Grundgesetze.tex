
\chapter{Grundgesetze der Mathematik}

\section{Mengenlehre}

\subsection{Der Mengenbegriff}\index{Menge}

Eine Menge ist im Wesentlichen ein Beutel, der unterschiedliche
Objekte enthält. Es gibt die leere Menge, das ist der leere Beutel.
Das besondere an einer Menge ist nun, dass das selbe Objekt immer
nur ein einziges mal im Beutel enthalten ist. Legt man zweimal
das selbe Objekt in den Beutel, dann ist dieses darin trotzdem nur
einmal zu finden.

Man kann sich dabei z.\,B. einen Einkaufsbeutel vorstellen,
in welchem sich nur ein Apfel, eine Birne, eine Weintraube usw.
befinden darf. Möchte man mehrere Birnen im Einkaufsbeutel haben,
dann müssen diese unterschieden werden, z.\,B. indem jede Birne
eine unterschiedliche Nummer bekommt.

Möchte man eine Menge aufschreiben, werden die Objekte einfach
in einer beliebigen Reihenfolge aufgelistet und diese Liste in
geschweifte Klammern gesetzt. Z.\,B.:%
\[\{\mathrm{Afpel}, \mathrm{Birne}, \mathrm{Weintraube}\}.\]
Nennen wir den Apfel $A$, die Birne $B$
und die Weintraube $W$. Eine Menge mit zwei Äpfeln und drei
Birnen würde man so schreiben:%
\[\{A_1, A_2, B_1, B_2, B_3\}.\]
Erlaubt sind auch Beutel in Beuteln. Eine Menge mit zwei Äpfeln
und einer Menge mit vier Weintrauben wird beschrieben durch%
\[\{A_1, A_2, \{W_1,W_2,W_3,W_4\}\}.\]
Die Reihenfolge spielt wie gesagt keine Rolle:%
\[\{A_1,A_2\} = \{A_2,A_1\}.\]
Ein leerer Beutel ist etwas anderes als ein Beutel, welcher einen
leeren Beutel enthält:%
\[\{\} \ne \{\{\}\}.\]
Die Notation $x\in M$ bedeutet, dass $x$ in der Menge $M$ enthalten
ist. Man sagt, $x$ ist ein Element von $M$. Z.\,B. ist
\[A_1\in\{A_1,A_2\}.\]

\newpage
\subsection{Teilmengen}

\begin{Definition}[Teilmengenrelation]\index{Teilmenge}
Hat man zwei Mengen $M,N$, dann nennt man $M$ eine Teilmenge von $N$,
wenn jedes Element von $M$ auch ein Element von $N$ ist.
Als Formel:
\[M\subseteq N\defiff \text{für jedes $x\in M$ gilt $x\in N$}.\]
Anders formuliert, aber gleichbedeutend:
\[M\subseteq N\defiff \text{für jedes $x$ gilt:}\; (x\in M\implies x\in N).\]
\end{Definition}
Z.\,B. ist die Aussage $\{1,2\}\subseteq\{1,2,3\}$ wahr.
Die Aussage $\{1,2,3\}\subseteq\{1,2\}$ ist jedoch falsch,
weil $3$ kein Element von $\{1,2\}$ ist. Für jede Menge $M$ gilt
$M\subseteq M$, denn die Aussage
\[x\in M\implies x\in M\]
ist immer wahr, da die Formel »$\varphi\Rightarrow\varphi$«
tautologisch ist.

\subsection{Mengen von Zahlen}
\index{Zahlenbereiche}\index{Natürliche Zahlen}\index{ganze Zahlen}%
\index{reelle Zahlen}\index{Dezimalzahl}

Einige Mengen kommen häufiger vor, was dazu führte, dass man für
diese Mengen kurze Symbole definiert hat.

Die Menge der natürlichen Zahlen mit der Null:
\[\N_0 := \{0,1,2,3,4,\ldots\}.\]
Die Menge der natürlichen Zahlen ohne die Null:
\[\N := \{1,2,3,4,\ldots\}.\]
Die Menge der ganzen Zahlen:
\[\Z := \{\ldots,-4,-3,-2,-1,0,1,2,3,4,\ldots\}.\]
Dann gibt es noch die rationalen Zahlen $\Q$, das sind alle
Brüche der Form $m/n$, wobei $m,n$ ganze Zahlen sind
und $n\ne 0$ ist. Rationale Zahlen lassen sich immer als
Dezimalbruch schreiben, dessen Ziffern irgendwann periodisch
werden.

\begin{table}[h]
\centering
\begin{tabular}{c|l|l}
\strong{Zahl} & \strong{als Dezimalzahl} & \strong{kurz}\\
$1/2$ & $0.5000000000\ldots$ & $0.5\overline{0}$\\
$1/3$ & $0.3333333333\ldots$ & $0.\overline{3}$\\
$1241/1100$ & $1.1281818181\ldots$ & $0.12\overline{81}$
\end{tabular}
\caption{Jeder Bruch lässt sich als Dezimalzahl
schreiben, deren Ziffern in eine periodische Zifferngruppe münden.
Über die periodische Zifferngruppe setzt man einen waagerechten
Strich.}
\end{table}

\noindent
Schließlich gibt es noch die reellen Zahlen $\R$. Darin enthalten sind
alle Dezimalzahlen -- auch solche, deren Ziffern niemals in eine
periodische Zifferengruppe münden. Die reellen Zahlen haben
eine recht komplizierte Struktur, und wir benötigen Mittel
der Analysis um diese verstehen zu können. Solange diese Werkzeuge
noch nicht bekannt sind, kann man die reellen Zahlen einfach
als kontinuierliche Zahlengerade betrachten. Die rationalen
Zahlen haben Lücken in dieser Zahlengerade, z.\,B. ist die Zahl
$\sqrt{2}$ nicht rational, wie sich zeigen lässt. Die reellen
Zahlen schließen diese Lücken.

\subsection{Vergleich von Mengen}\index{Menge!Vergleich von Mengen}

Wie können wir denn wissen, wann zwei Mengen $A,B$, gleich sind?
Zwei Mengen sind ja gleich, wenn sie beide die gleichen Elemente
enthalten. Aber wie lässt sich das als mathematische Aussage
formulieren?

Jedes Element von $A$ muss doch auch ein Element von $B$ sein,
sonst gäbe es Elemente in $A$, die nicht in $B$ enthalten wären.
Umgekehrt muss auch jedes Element von $B$ ein Element von $A$ sein.
Also ist $A\subseteq B$ und $B\subseteq A$ eine notwendige Bedingung.
Diese Bedingung ist sogar hinreichend.

Gehen wir mal von der Kontraposition aus -- sind die beiden Mengen
$A,B$ verschieden, dann muss es ein Element in $A$ geben, welches nicht
in $B$ enthalten ist, oder eines in $B$, welches nicht $A$ enthalten
ist. Als Formel:%
\[A\ne B \implies \exists(x\in A)(x\notin B)\lor\exists(x\in B)(x\notin A).\]
Hiervon bildet man wieder die Kontraposition. Gemäß den
De Morganschen Gesetzen und den verallgemeinerten
De Morganschen Gesetzen ergibt sich%
\[\forall(x\in A)(x\in B)\land\forall(x\in B)(x\in A)\implies A=B.\]
Auf der linken Seite stehen aber nach Definition
Teilmengenbeziehungen, es ergibt sich%
\[A\subseteq B\land B\subseteq A\implies A=B.\]
\begin{Definition}[Gleichheit von Mengen]\index{Gleichheit!von Mengen}
Zwei Mengen $A,B$ sind genau dann gleich, wenn jedes Element von
$A$ auch in $B$ enthalten ist, und jedes von $B$ auch in $A$ enthalten:%
\[A=B\defiff A\subseteq B\land B\subseteq A.\]
\end{Definition}
\begin{Satz}\label{set-eq}
Es gilt
\[A=B\iff \forall x(x\in A\iff x\in B).\]
\end{Satz}
\strong{Beweis.} Wir müssen ein wenig Prädikatenlogik bemühen:%
\begin{align*}
A\subseteq B\land B\subseteq A
&\iff \forall(x\in A)(x\in B)\land\forall(x\in B)(x\in A)\\
&\iff \forall x(x\in A\implies x\in B)\land\forall x(x\in B\implies x\in A)\\
&\iff \forall x((x\in A\implies x\in B)\land (x\in B\implies x\in A))\\
&\iff \forall x(x\in A\iff x\in B).
\end{align*}
Im letzten Schritt wurde ausgenutzt, dass
$\varphi\Leftrightarrow\psi$ definitionsgemäß gleichbedeutend
mit $(\varphi\Rightarrow\psi)\land(\psi\Rightarrow\varphi)$
ist.\;\qedsymbol

%\newpage
\subsection{Beschreibende Angabe von Mengen}

Umso mehr Elemente eine Menge enthält, umso umständlicher wird
die Auflistung all dieser Elemente. Außerdem hantiert man in der
Mathematik normalerweise auch ständig mit Mengen herum, die
unendlich viele Elemente enthalten. Eine explizite Auflistung ist
demnach unmöglich.

Wir entgehen der Auflistung aller Elemente durch eine Beschreibung
der Menge. Die Menge der ganzen Zahlen, welche kleiner als vier sind,
wird so beschrieben:%
\[\{n\in\Z\mid n<4\}.\]
In Worten: Die Menge der $n\in\Z$, für die gilt: $n<4$.

Mit dieser Notation kann man nun z.\,B. schreiben:%
\begin{align*}
\N_0 &= \{n\in\Z\mid n\ge 0\},\\
\N &= \{n\in\Z\mid n>0\}.
\end{align*}
Mit der folgenden formalen Defintion wird die beschreibende Angabe
auf ein festes Fundament gebracht.

\begin{Definition}[Beschränkte Beschreibung einer Menge]%
\label{def:set-builder-bounded}\index{Menge!Comprehension}
Die Menge der $x\in M$, welche die Aussage $P(x)$ erfüllen,
ist definiert durch die folgende logische Äquivalenz:%
\[a\in\{x\in M\mid P(x)\} \defiff a\in M\land P(a).\]
\end{Definition}
Das schaut ein wenig kompliziert aus, ist aber ganz einfach zu
benutzen. Sei z.\,B. $A:=\{n\in\Z\mid n<4\}$. Zu beantworten ist
die Frage, ob $2\in A$ gilt. Eingesetzt in die Definition
ergibt sich%
\[2\in\{n\in\Z\mid n<4\}\iff 2\in\Z\land 2<4.\]
Da $2\in\Z$ und $2<4$ wahre Aussagen sind, ist die rechte Seite
erfüllt, und damit auch die linke Seite der Äquivalenz.

Die geraden Zahlen lassen sich so definieren:%
\[2\Z:=\{n\in\Z\mid\text{es gibt ein $k\in\Z$ mit $n=2k$}\}.\]
Es lässt sich zeigen:
\[a\in 2\Z\implies a^2\in 2\Z.\]
Nach Definition von $2\Z$ gibt es $k\in\Z$ mit $a=2k$.
Dann ist $a^2=(2k)^2=4k^2=2(2k^2)$. Benennt man $k':=2k^2$, dann
gilt also $a^2=2k'$. Also gibt es es ein $k'\in\Z$
mit $a^2=2k'$, und daher ist $a^2\in 2\Z$.

Die graden Zahlen sind ganze Zahlen, welche ohne Rest durch zwei teilbar
sind. Die ganzen Zahlen, welche ohne Rest durch $m$ teilbar sind,
lassen sich formal so definieren:%
\[m\Z:=\{n\in\Z\mid\text{es gibt ein $k\in\Z$ mit $n=mk$}\}.\]
Man zeige:
\begin{align*}
& (1.)\;\;a\in 2\Z\implies a^2\in 4\Z, && (3.)\;\;2\Z\subseteq\Z,\\
& (2.)\;\;a\in 4\Z\implies a\in 2\Z,   && (4.)\;\;4\Z\subseteq 2\Z.
\end{align*}


\begin{Definition}[Beschreibende Angabe einer Menge]%
\label{def:set-builder}
Stellt man sich unter $G$ die Grundmenge vor, welche
alle Elemente enthält, die überhaupt in Betracht kommen können,
dann schreibt man kurz%
\[\{x\mid P(x)\} := \{x\in G\mid P(x)\}\]
und nennt dies die Beschreibung einer Menge.
\end{Definition}
\begin{Satz}
Es gilt
\begin{gather}
\label{eq:set-builder}
a\in\{x\mid P(x)\}\iff P(a),\\
\label{eq:bound-conversion}
\{x\in A\mid P(x)\} = \{x\mid x\in A\land P(x)\}.
\end{gather}
\end{Satz}
\strong{Beweis.} Gemäß Definition \ref{def:set-builder}
und \ref{def:set-builder-bounded} gilt%
\[a\in\{x\mid P(x)\} \iff a\in\{x\in G\mid P(x)\}
\iff a\in G\land P(a)\iff P(a),\]
denn $a\in G$ ist immer erfüllt, wenn $G$ die Grundmenge ist.
Die Aussage $a\in G$ kann daher in der Konjunktion gemäß dem
Neutralitätsgesetz der booleschen Algebra entfallen.

Aussage \eqref{eq:bound-conversion} wird mit Satz \ref{set-eq}
expandiert. Zu zeigen ist nun
\[a\in\{x\in A\mid P(x)\}\iff a\in\{x\mid x\in A\land P(x)\},\]
was gemäß Definition \ref{def:set-builder-bounded} und der schon
bewiesenen Aussage \eqref{eq:set-builder} aber vereinfacht
werden kann zu
\[a\in A\land P(a)\iff a\in A\land P(a).\;\qedsymbol\]


\subsection{Bildmengen}\index{Bildmenge}

Oft kommt auch die Angabe einer Menge als Bildmenge vor, dabei
handelt es sich um eine spezielle Beschreibung der Menge. Ist
$T(x)$ ein Term und $A:=\{a_1,a_2,\ldots,a_n\}$ eine endliche
Menge, dann wird das Bild von $A$ unter $T(x)$ so beschrieben:
\[\{T(x)\mid x\in A\} := \{T(a_1),T(a_2),\ldots, T(a_n)\}.\]
Lies: Die Menge der $T(x)$, für die $x\in A$ gilt.
Für $T(x):=x^2$ und $A:=\{1,2,3,4\}$ ist z.\,B.
\[\{T(x)\mid x\in A\} = \{T(1), T(2), T(3), T(4)\}
= \{1^2,2^2,3^2,4^2\} = \{1,4,9,16\}.\]
Nun kann es aber sein, dass die Menge $A$ unendlich viele Elemente
enthält, eine Auflistung dieser somit unmöglich ist. Eine Auflistung
lässt umgehen, indem man nur logisch die Existenz eines Bildes
zu jedem $x\in A$ verlagt, dieses aber nicht mehr explizit angibt.
Man definiert also allgemein
\[\{T(x)\mid x\in A\} := \{y\mid\text{es gibt ein $x\in A$, für das gilt: $y=T(x)$}\}.\]
Das hatten wir bei den geraden Zahlen
\[2\Z := \{2k\mid k\in\Z\} = \{n\mid\text{es gibt ein $k\in\Z$, für das gilt: $n=2k$}\}\]
schon kennengelernt. Hierbei ist es unwesentlich, ob man $n\in\Z$ verlangt
oder nicht, denn dies wird bereits durch $k\in\Z$ erzwungen.

\newpage
\subsection{Mengenoperationen}

Mengen sind mathematische Objekte, mit denen sich rechnen lässt.
So wie es für Zahlen Rechenoperationen gibt, gibt es auch für
Mengen Rechenoperationen.
\begin{Definition}[Vereinigungsmenge]%
\index{Vereinigungsmenge}\index{Menge!Vereinigung}
Die Vereinigungsmenge von zwei Mengen $A,B$ ist die Menge aller Elemente,
welche in $A$ oder in $B$ vorkommen:
\[A\cup B := \{x\mid x\in A\lor x\in B\}.\]
\end{Definition}
Man nimmt also einen neuen Beutel und schüttet den Inhalt von $A$
und $B$ in diesen Beutel.

Beispiele:
\begin{gather*}
\{1,2\}\cup\{5,7,9\} = \{1,2,5,7,9\},\\
\{1,2\}\cup\{1,3,5\} = \{1,2,3,5\}.
\end{gather*}

\begin{Definition}[Schnittmenge]%
\index{Schnittmenge}\index{Menge!Schnitt}
Die Schnittmenge von zwei Mengen $A,B$ ist die Menge aller Elemente,
welche sowohl in $A$ also auch in $B$ vorkommen:
\[A\cap B := \{x\mid x\in A\land x\in B\}.\]
\end{Definition}
\begin{Satz}
Bei der Beschreibung der Schnittmenge $A\cap B$ genügt es, $A\cup B$ als Grundmenge
zu verwenden, denn es gilt
\[A\cap B = \{x\in A\cup B\mid x\in A\land x\in B\}\]
\end{Satz}
\strong{Beweis.}
Die Formel wird mit Satz \ref{set-eq} expandiert. Zu zeigen ist demnach
\[a\in A\cap B\iff a\in \{x\in A\cup B\mid x\in A\land x\in B\}.\]
Das ist nach \eqref{eq:set-builder} und Definition
\ref{def:set-builder-bounded} gleichbedeutend mit
\begin{align*}
a\in A\land a\in B&\iff a\in A\cup B\land a\in A\land a\in B\\
&\iff (a\in A\lor a\in B)\land a\in A\land a\in B.
\end{align*}
Nun gilt für beliebige Aussagen $\varphi,\psi$ gemäß boolescher Algebra aber
\begin{align*}
(\varphi\lor\psi)\land\varphi\land\psi
&\iff (\varphi\land\varphi\land\psi)\lor(\psi\land\varphi\land\psi)\\
&\iff (\varphi\land\psi)\lor(\varphi\land\psi)\\
&\iff \varphi\land\psi.
\end{align*}
Auf beiden Seiten der Äquivalenz steht jetzt die gleiche Aussage:%
\[a\in A\land a\in B\iff a\in A\land a\in B.\;\qedsymbol\]

\newpage
\section{Gleichungen}
\subsection{Begriff der Gleichung}\index{Gleichung}

Bei einer Gleichung verhält es sich wie bei einer Balkenwaage. Liegt
in einer der Waagschalen eine Masse von 2g und in der anderen
Waagschale zwei Massen von jeweils 1g, dann bleibt die Waage im
Gleichgewicht. Als Gleichung gilt
\[2=1+1.\]
Eine Gleichung kann wahr oder falsch sein, z.\,B. ist $2=2$
wahr, während $2=3$ falsch ist. Das bedeutet aber nicht, dass man
eine falsche Gleichung nicht aufschreiben dürfe. Vielmehr ist eine
Gleichung ein mathematisches Objekt, dem sich ein Wahrheitswert
zuordnen lässt. Zumindest sollte man eine falsche Gleichung nicht
ohne zusätzliche Erklärung aufschreiben, so dass der Eindruck
entstünde, sie könnte wahr sein.

\subsection{Äquivalenzumformungen}%
\index{Aequivalenzumformung@Äquivalenzumformung!von Gleichungen}

Fügt man zu beiden Schalen einer Balkenwaage das gleiche Gewicht
hinzu, dann bleibt die Waage so wie sie vorher war. War sie im
Gleichgewicht, bleibt sie dabei. War sie im Ungleichgewicht,
bleibt sie auch dabei. Ebenso verhält es sich mit einer Gleichung.
Addition der gleichen Zahl auf beide Seiten einer Gleichung bewirkt
keine Veränderung des Aussagengehalts der Gleichung.

Diese Überlegung gilt natürlich auf für die Subtraktion einer Zahl
auf beiden Seiten, welche dem Entfernen des gleichen Gewichtes von
beiden Waagschalen entspricht.

\begin{Satz}[Äquivalenzumformungen]\label{eq-add}
Seien $a,b,c$ beliebige Zahlen. Dann gilt
\begin{align*}
a=b&\iff a+c=b+c,\\
a=b&\iff a-c=b-c.
\end{align*}
\end{Satz}

\noindent
Auch eine Verdopplung des Gewichtes in beiden Schalen der Balkenwaage
ändert nicht ihr Gleichgewicht oder Ungleichgewicht.

\begin{Satz}[Äquivalenzumformungen]\label{eq-mul-int}
Seien $a,b$ beliebige Zahlen und $n\in\Z$ mit $n\ne 0$. Dann gilt
\begin{align*}
a=b&\iff na=nb.
\end{align*}
\end{Satz}
\strong{Beweis.} Gemäß Satz \ref{eq-add} gilt
\begin{align*}
na = nb &\iff 0 = na-nb = n(a-b)\iff n=0\lor a-b=0\\
&\iff a-b=0\iff a=b.
\end{align*}
Dabei wurde ausgenutzt, dass ein Produkt nur null sein kann,
wenn einer der Faktoren null ist. Gemäß Voraussetzung $n\ne 0$ muss
dann aber $a-b=0$ sein.\;\qedsymbol

\begin{Satz}[Äquivalenzumformungen]
Seien $a,b$ beliebige Zahlen und $r\in\Q$ mit $r\ne 0$. Dann gilt
\[a=b\iff ra=rb\iff a/r=b/r.\]
\end{Satz}
\strong{Beweis.}
Die Zahl $r$ ist von der Form $r=m/n$, wobei $m,n\in\Z$ und $m,n\ne 0$.
Daher gilt
\begin{align*}
ra=rb&\iff \frac{m}{n}a=\frac{m}{n}b
\stackrel{\text{Satz \ref{eq-mul-int}}}\iff n\cdot\frac{m}{n}a=n\cdot\frac{m}{n}b\\
&\iff ma=mb\stackrel{\text{Satz \ref{eq-mul-int}}}\iff a=b.
\end{align*}
Daraufhin gilt auch
\[\frac{a}{r}=\frac{b}{r}\iff r\cdot\frac{a}{r}=r\cdot\frac{b}{r}
\iff a=b.\;\qedsymbol\]

\begin{Satz}[Äquivalenzumformungen]
Seien $a,b,r\in\R$ und sei $r\ne 0$. Dann gilt
\[a=b\iff ra=rb\iff a/r=b/r.\]
\end{Satz}
\strong{Beweis.} Man rechnet wieder
\begin{align*}
ra = rb&\iff ra-rb=0\iff (a-b)r=0\iff r=0\lor a-b=0\\
&\iff a-b=0\iff a=b.
\end{align*}
Es wurde wieder ausgenutzt, dass ein Produkt nur dann null sein
kann, wenn einer der Faktoren null ist. Daraufhin gilt auch
\[\frac{a}{r}=\frac{b}{r}\iff r\cdot\frac{a}{r}=r\cdot\frac{b}{r}
\iff a=b.\;\qedsymbol\]

\noindent


\newpage
\section{Ungleichungen}

\subsection{Begriff der Ungleichung}%
\index{Ungleichung}

Man stelle sich zwei Körbe vor, in die Äpfel gelegt werden.
In den rechten Korb werden zwei Äpfel gelegt, in den linken drei.
Dann befinden sich im rechten Korb weniger Äpfel als im linken.
Man sagt, zwei ist kleiner als drei, kurz $2<3$. Man spricht von
einer \emph{Ungleichung}, in Anbetracht dessen, dass die beiden
Körbe nicht die gleiche Anzahl von Äpfeln enthalten.

Der Aussagengehalt einer Ungleichung kann wahr oder falsch sein.
Die Ungleichung $2<3$ ist wahr, die Ungleichungen $3<3$ und
$4<3$ sind falsch.

\begin{Definition}[Ungleichungsrelation]
Die Notation $a<b$ bedeutet »Die Zahl $a$ ist kleiner als
die Zahl $b$«. Die Notation $a\le b$ bedeutet »Die Zahl $a$
ist kleiner als oder gleich der Zahl $b$«. Die Notation
$b>a$ ist eine andere Schreibweise für $a<b$ und bedeutet
»Die Zahl $b$ ist größer als die Zahl $a$«. Die Notation
$b\ge a$ ist eine andere Schreibweise für $a\le b$ und
bedeutet »Die Zahl $b$ ist größer oder gleich der Zahl $a$«.
\end{Definition}

\subsection{Äquivalenzumformungen}%
\index{Aequivalenzumformung@Äquivalenzumformung!von Ungleichungen}

Wir stellen uns wieder einen linken Korb mit zwei Äpfeln und
einen rechten Korb mit drei Äpfeln vor. Legt man nun in beide
Körbe jeweils zusätzlich 10 Äpfel hinein, dann befinden sich
im linken Korb 12 Äpfel und im rechten 13. Der linke Korb
enthält also immer noch weniger Äpfel als im rechten.

Befindet sich eine Balkenwaage im Ungleichgewicht, und legt man
in beide Waagschalen zusätzlich die gleiche Masse von Gewichten,
dann wird sich das Ungleichgewicht der Balkenwaage nicht verändern.

Für die Herausnahme von Äpfeln oder Gewichten ist diese Argumentation
analog. Ist stattdessen eine falsche Ungleichung gegeben,
dann lässt sich durch Addition der selben Zahl auf beiden Seiten
daraus keine wahre Ungleichung gewinnen. Die analoge Argumentation
gilt für die Subtraktion der selben Zahl. Anstelle von ganzen
Äpfeln kann man natürlich auch Apfelhälften hinzufügen, oder
allgmein Apfelbruchteile. Die Argumentation gilt unverändert.

Wir halten fest. 

\begin{Satz}[Äquivalenzumformungen von Ungleichungen]
Seien $a,b,c$ beliebige Zahlen. Dann sind die folgenden
Äquivalenzen gültig:
\begin{gather}
\label{lt-add} a<b\iff a+c<b+c,\\
\label{lt-sub} a<b\iff a-c<b-c,\\
\label{le-add} a\le b\iff a+c\le b+c,\\
\label{le-sub} a\le b\iff a-c\le b-c.
\end{gather}
\end{Satz}

\noindent
In Worten: Wenn auf beiden Seiten einer Ungleichung die gleiche
Zahl addiert oder subtrahiert wird, dann ändert sich der Aussagengehalt
dieser Ungleichung nicht.

Gibt es noch andere Äquivalenzumformungen?

Im linken Korb seien wieder zwei Äpfel, im rechten drei. Verdoppelt
man nun die Anzahl in beiden Körben, dann sind linken vier Äpfel,
im rechten sechs. Verzehnfacht man die Anzahl, dann sind im linken
20 Äpfel, im rechten 30. Offenbar verändert sich der Aussagengehalt
nicht, wenn die Anzahl auf beiden Seiten der Ungleichung mit
der gleichen natürlichen Zahl $n$ multipliziert wird.

Jedoch muss $n=0$ ausgeschlossen werden. Wenn $a<b$ ist, und man
multipliziert auf beiden Seiten mit null, dann ergibt sich
$0<0$, was falsch ist. Aus der wahren Ungleichung wurde damit eine
falsche gemacht, also kann es sich nicht um eine Äquivalenzumformung
handeln.

Auch bei der Ungleichung $a\le b$ muss $n=0$ ausgeschlossen werden.
Warum muss man das tun? Die Ungleichung $0\le 0$ ist doch auch
wahr?

Nun, wenn der Aussagengehalt von $a\le b$ falsch ist, z.\,B. $4\le 3$,
und man multipliziert auf beiden Seiten mit null, dann ergibt sich
$0\le 0$, also eine wahre Ungleichung. Aus einer falschen wurde damit
eine wahre gemacht. Bei einer Äquivalenzumformung ist dies ebenfalls
verboten.

\begin{Satz}[Äquivalenzumformungen von Ungleichungen]
Seien $a,b$ beliebige Zahlen und sei $n>0$ eine natürliche Zahl.
Dann sind die folgenden Äquivalenzen gültig:
\begin{gather}
\label{lt-mul-nat} a<b\iff na<nb,\\
\label{le-mul-nat} a\le b\iff na\le nb.
\end{gather}
\end{Satz}

\noindent\strong{Beweis.}
Aus der Ungleichung $a<b$ erhält man mittels \eqref{lt-sub} die
äquivalente Ungleichung $0<b-a$, indem auf beiden Seiten $a$
subtrahiert wird. Die Zahl $b-a$ ist also positiv. Durch Multiplikation
mit einer positiven Zahl lässt sich das Vorzeichen einer Zahl
aber nicht umkehren. Demnach ist $0<n(b-a)$ genau dann,
wenn $0<b-a$ war. Ausmultiplizieren liefert nun
$0<nb-na$ und Anwendung von \eqref{lt-add} bringt dann $na<nb$.

In Kürze formuliert:
\begin{equation}
a<b\iff 0<b-a\iff 0<n(b-a)=nb-na \iff na<nb.
\end{equation}
Für $a\le b$ gilt diese Überlegung analog.\;\qedsymbol

\strong{Alternativer Beweis.}
Mittels \eqref{lt-add} ergibt sich zunächst:
\begin{equation}
a<b\iff \left\{
\begin{matrix}
a+a<b+a\\
a+b<b+b
\end{matrix}
\right\}
\iff 2a<a+b<2b.
\end{equation}
Unter nochmaliger Anwendung von \eqref{lt-add} ergibt sich
nun
\begin{equation}
a<b\iff \left\{
\begin{matrix}
2a<a+b \iff 3a<2a+b\\
2a<2b \iff 2a+b<3b
\end{matrix}
\right\} 3a<2a+b<3b
\end{equation}
Dieses Muster lässt sich induktiv alle natürlichen Zahlen hochschieben:
Aus $na<(n-1)a+b<nb$ sollte sich
$(n+1)a<na+b<(n+1)b$ schlussfolgern lassen und umgekehrt.
Das ist richtig, denn Addition von $a$ gemäß \eqref{lt-add} bringt
\begin{equation}
na<(n-1)a+b \iff (n+1)a < na+b
\end{equation}
und Addition von $b$ gemäß \eqref{lt-add} bringt
\begin{equation}
na<nb \iff na+b < (n+1)b.
\end{equation}
Zusammen ergibt sich daraus der behauptete Induktionsschritt. 
Daraus erhält man $a<b\iff na<nb$. Für $a\le b$ sind diese
Überlegungungen analog.\;\qedsymbol

Wir können sogleich einen Schritt weiter gehen.
\begin{Satz}[Äquivalenzumformungen von Ungleichungen]
Seien $a,b$ beliebige Zahlen und sei $r>0$ eine rationale Zahl,
dann gelten die folgenden Äquivalenzen:
\begin{gather}
\label{lt-mul-rat} a<b\iff ra<rb\iff a/r<b/r,\\
\label{lt-mul-rat} a\le b\iff ra\le rb\iff a/r\le b/r.
\end{gather}
\end{Satz}

\noindent\strong{Beweis.}
Eine rationale Zahl $r>0$ lässt sich immer Zerlegen in einen Quotienten
$r=m/n$, wobei $m,n$ positive natürliche Zahlen sind. Gemäß
\eqref{lt-mul-nat} gilt
\begin{equation}
\frac{m}{n}\cdot a<\frac{m}{n}\cdot b
\iff n\cdot\frac{m}{n}\cdot a<n\cdot\frac{m}{n}\cdot b
\iff ma<mb.
\end{equation}
Gemäß \eqref{lt-mul-nat} gilt aber auch
\begin{equation}
a<b\iff ma<mb.
\end{equation}
Die Zusammenfassung beider Äquivalenzen ergibt
\begin{equation}
a<b\iff \frac{m}{n}\cdot a<\frac{m}{n}\cdot b\iff ra<rb.
\end{equation}
Für $a\le b$ ist die Argumentation analog. Da die Division durch
eine rationale Zahl $r$ die Multiplikation mit ihrem Kehrwert $1/r$ ist,
sind auch die Äquivalenzen für die Division gültig.\;\qedsymbol

Da sich eine reelle Zahl beliebig gut durch eine rationale annähern
lässt, müsste auch der folgende Satz gültig sein.

\begin{Satz}[Äquivalenzumformungen von Ungleichungen]
Seien $a,b$ beliebige Zahlen und sei $r>0$ eine reelle Zahl,
dann gelten die folgenden Äquivalenzen:
\begin{gather}
\label{lt-mul-real} a<b\iff ra<rb\iff a/r<b/r,\\
\label{lt-mul-real} a\le b\iff ra\le rb\iff a/r\le b/r.
\end{gather}
\end{Satz}

\noindent
Der Satz wird sich als richtig erweisen, der Beweis kann in
Analysis"=Lehrbüchern nachgeschlagen werden.

\subsection{Lineare Ungleichungen}

Interessant werden Ungleichungen nun, wenn in ihnen einen Variable
vorkommt. Beispielsweise sei die Ungleichung $x+2<4$ gegeben.
Wird in diese Ungleichung für die Variable $x$ eine Zahl eingesetzt,
dann kann wird die Ungleichung entweder wahr oder falsch sein.
Für $x:=1$ ergibt sich die wahre Ungleichung $1+2<4$. Für $x:=2$
ergibt sich jedoch die falsche Ungleichung $2+2<4$.

Wir interessieren uns nun natürlich für die Menge aller Lösungen
dieser Ungleichung. Das sind die Zahlen, welche die Ungleichung
erfüllen, wenn sie für $x$ eingesetzt werden. Gesucht ist also
die Lösungsmenge
\[L = \{x\mid x+2<4\},\]
d.\,h. die Menge der $x$, welche die Ungleichung $x+2<4$ erfüllen.

Gemäß Äquivalenzumformung \eqref{lt-sub} kommt man aber sofort zu
\[x+2<4 \iff x+2-2<4-2 \iff x<2.\]
Demnach kann die Lösungsmenge als $L=\{x\mid x<2\}$ angegeben werden,
denn Äquivalenzumformungen lassen die Lösungsmenge einer Ungleichung
unverändert.

Die Ungleichung $x+2<4$ ist sicherlich von so einfacher Gestalt,
dass man diese auch gedanklich lösen kann, ohne Äquivalenzumformungen
bemühen zu müssen. Bei komplizierteren Ungleichungen kommen wir dabei
aber mehr oder weniger schnell an unsere mentalen Grenzen.

Schon ein wenig schwieriger ist z.\,B.
\begin{align*}
& 5x+2<3x+10 && |\;{-2}\\
\iff & 5x<3x+8 && |\;{-3x}\\
\iff & 2x<8 && |\;{/2}\\
\iff & x<4.
\end{align*}

\subsection{Monotone Funktionen}%
\index{Monotone Funktion}

\begin{Definition}[Streng monoton steigende Funktion]%
\index{Streng monotone Funktion}\index{Monotone Funktion!strenge Monotonie}
Eine Funktion $f\colon G\to\R$ heißt streng monoton steigend, wenn
\[a<b\implies f(a)<f(b)\]
für alle Zahlen $a,b\in G$ erfüllt ist.
\end{Definition}
Streng monotone Abbildungen sind von besonderer Bedeutung, weil
sie gemäß ihrer Definition auch Äquivalenzumformungen sind:

\begin{Satz}[Allgemeine Äquivalenzumformung]%
\index{Aequivalenzumformung@Äquivalenzumformung!allgemein für Ungleichungen}
Eine streng monoton steigende Funktionen $f$ ist umkehrbar eindeutig.
Die Umkehrfunktion ist auch streng monoton steigend. D.\,h.
\[a<b\iff f(a)<f(b).\]
Demnach ist die Anwendung einer streng monoton steigenden
Funktion eine Äquivalenzumformung.
\end{Satz}
\noindent\strong{Beweis.}
Zu zeigen ist $a\ne b\implies f(a)\ne f(b)$. Wenn aber $a\ne b$
ist, dann ist entweder $a<b$ und daher nach Voraussetzung
$f(a)<f(b)$ oder $b<a$ und daher nach Voraussetzung $f(b)<f(a)$.
In beiden Fällen ist $f(a)\ne f(b)$.

Seien nun $y_1,y_2$ zwei Bilder der streng monotonen Funktion $f$.
Zu zeigen ist $y_1<y_2\implies f^{-1}(y_1)<f^{-1}(y_2)$.
Stattdessen kann auch die Kontraposition
$f^{-1}(y_2)\le f^{-1}(y_1)\implies y_2\le y_1$ gezeigt werden.
Das lässt sich nun aus der strengen Monotonie von $f$ schließen:
\begin{equation}
f^{-1}(y_2)\le f^{-1}(y_1)\implies
\underbrace{f(f^{-1}(y_2))}_{=y_2}\le \underbrace{f(f^{-1}(y_1))}_{=y_1}.\;\qedsymbol
\end{equation}

\begin{Definition}[Streng monoton fallende Funktion]
Eine Funktion $f\colon G\to\R$ heißt streng monoton fallend, wenn
\[a<b\implies f(a)>f(b)\]
für alle Zahlen $a,b\in G$ erfüllt ist.
\end{Definition}

\noindent
Ein entsprechender Satz gilt auch für diese:
\begin{Satz}[Allgemeine Äquivalenzumformung]
Eine streng monoton fallende Funktion $f$ ist umkehrbar eindeutig.
Die Umkehrfunktion ist auch streng monoton fallend. D.\,h.
\[a<b\iff f(a)>f(b).\]
Demnach ist die Anwendung einer streng monoton fallenden
Funktion eine Äquivalenzumformung bei der sich das Relationszeichen
umdreht.
\end{Satz}

\noindent
Tatsächlich haben wir schon streng monoton steigende Funktionen
kennengelernt. Z.\,B. ist \eqref{lt-add} nichts anderes als die strenge
Monotonie für $f(x):=x+c$. Und \eqref{lt-mul-nat} ist die strenge
Monotonie für $f(x):=nx$.

Die Funktion $f\colon\R\to\R$ mit $f(x):=x^2$ ist nicht streng monoton
steigend. Zum Beispiel ist $-4<-2$, aber $16=f(-4)>f(-2)=4$. Auch
ist die Funktion nicht streng monoton fallend, denn $2<4$,
aber $4=f(2)<f(4)=16$. Schränkt man $f$
auf den Definitionsbereich $\R_{>0}$ ein, so ergibt sich jedoch eine
streng monoton steigende Funktion. Das lässt sich wie folgt zeigen.

Nach Voraussetzung sind $a,b\in\R_{>0}$, d.\,h. $a,b>0$.
Also kann gemäß \eqref{lt-mul-real} einerseits mit $a$
und andererseits mit $b$ multipliziert werden:
\[
a<b\iff\begin{Bmatrix}
a^2<ab\\
ab<b^2
\end{Bmatrix}
\iff a^2<ab<b^2.
\]




