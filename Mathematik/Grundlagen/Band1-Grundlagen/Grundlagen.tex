\documentclass[a4paper,fleqn,12pt]{scrbook}
\usepackage[utf8]{inputenc}
\usepackage[T1]{fontenc}
\usepackage[ngerman]{babel}
% \usepackage{microtype}
\usepackage{amsmath}
\usepackage{amssymb}
\usepackage{amsthm}
\usepackage{graphicx}

\usepackage{mdframed}
\usepackage{lipsum}

\usepackage{libertine}
\usepackage[scaled=0.80]{DejaVuSans}
\usepackage[cmintegrals]{newtxmath}
\renewcommand\ttdefault{lmvtt}

\usepackage{geometry}
\geometry{a4paper,left=30mm,right=30mm,top=30mm,bottom=40mm}

\usepackage{color}
\definecolor{c1}{RGB}{0,40,80}
\definecolor{gray1}{RGB}{80,80,80}
\usepackage[colorlinks=true,linkcolor=c1]{hyperref}
% \usepackage[colorlinks=true,linkcolor=black]{hyperref}

\newcommand{\strong}[1]{\textsf{\textbf{#1}}}

\newtheoremstyle{rmbox}%
  {0pt}% space above
  {0pt}% space below
  {}% bodyfont
  {}% indent
  {\normalfont\sffamily\bfseries}% head font
  {\\[2pt]}% punctuation between head and body
  {0pt}% space after theorem head
  {\thmname{#1}\;\thmnumber{#2}.\;\thmnote{#3.}}

\theoremstyle{rmbox}
\newtheorem{Definition}{Definition}
\newtheorem{Satz}{Satz}
\newtheorem{Lemma}[Satz]{Lemma}
\newtheorem{Korollar}[Satz]{Korollar}

\numberwithin{Definition}{chapter}
\numberwithin{Satz}{chapter}

\definecolor{greenblue}{rgb}{0.0,0.42,0.3}
\definecolor{grayblue}{rgb}{0.1,0.2,0.4}
\definecolor{bgreen}{rgb}{0.94,0.94,0.84}
\definecolor{bblue}{rgb}{0.9,0.92,0.94}

\surroundwithmdframed[topline=false,rightline=false,bottomline=false,%
  linecolor=greenblue, linewidth=3.5pt, innerleftmargin=6pt,%
  innertopmargin=4pt, innerbottommargin=4pt,%
  innerrightmargin=6pt, backgroundcolor=bgreen%
]{Definition}

\newcommand{\framedtheorem}[1]{%
\surroundwithmdframed[topline=false,rightline=false,bottomline=false,%
  linecolor=grayblue, linewidth=3.5pt, innerleftmargin=6pt,%
  innertopmargin=4pt, innerbottommargin=4pt,%
  innerrightmargin=6pt, backgroundcolor=bblue%
]{#1}}

\framedtheorem{Satz}
\framedtheorem{Lemma}
\framedtheorem{Korollar}

\newcommand{\N}{\mathbb N}
\newcommand{\Z}{\mathbb Z}
\newcommand{\Q}{\mathbb Q}
\newcommand{\R}{\mathbb R}
\newcommand{\C}{\mathbb C}
\newcommand{\ui}{\mathrm i}
\newcommand{\ee}{\mathrm e}
\newcommand{\defiff}{\;:\Longleftrightarrow\;}
\newcommand{\emdef}[1]{\emph{#1}}
\renewcommand{\qedsymbol}{\ensuremath{\Box}}

\DeclareMathOperator{\id}{id}
\DeclareMathOperator{\sur}{sur}
\DeclareMathOperator{\real}{Re}
\DeclareMathOperator{\imag}{Im}
\DeclareMathOperator{\Bild}{Bild}
\DeclareMathOperator{\Abb}{Abb}

\usepackage{makeidx}
\makeindex

\title{Grundlagen der Mathematik}
\date{Juni 2019}

\begin{document}
\thispagestyle{empty}

\maketitle

Dieses Heft steht unter der Lizenz Creative Commons CC0.

\tableofcontents


\chapter{Grundgesetze der Mathematik}

\section{Aussagenlogik}

\subsection{Aussagenlogische Formeln}

Aussagen in der Aussagenlogik sind entweder wahr oder falsch,
etwas dazwischen gibt es nicht, das nennt man auch das \emph{Prinzip
der Zweiwertigkeit}\index{Prinzip der Zweiwertigkeit}.
Wir schreiben $0=\text{falsch}$ und $1=\text{wahr}$, das ist schön
kurz und knapp.

Für die Aussage »$n$ ist ohne Rest durch $m$ teilbar« bzw.
»$m$ teilt $n$«, schreibt man kurz $m|n$. Aus Aussagen lassen sich
in der Aussagenlogik zusammengesetzte Aussagen bilden, z.\,B.
\[\text{Aus $2|n$ und $3|n$ folgt, dass $6|n$},\]
als Formel:
\[2|n\land 3|n \implies 6|n.\]
Streng genommen handelt es sich hierbei um eine Aussageform, da die
Aussage von einer Variable abhängig ist. Nachdem für $n$ eine Zahl
eingesetzt wurde, ergibt sich daraus eine Aussage, in diesem Fall
immer eine wahre Aussage.

Eine zusammengesetzte Aussage wird auch \emph{aussagenlogische Formel}
genannt. Aussagenlogische Formeln haben eine innere Struktur. Um diese
untersuchen zu können, werden logische Variablen betrachtet,
das sind solche Variablen, die für eine Aussage stehen. Eine
logische Variable wird durch einen lateinischen Großbuchstaben
am Anfang des Alphabetes beschrieben und kann nur mit den
Wahrheitswerten falsch oder wahr belegt werden. Die genannte Formel
besitzt die Struktur
\[A\land B \implies C.\]
In der Formel treten Verknüfungen von Aussagen auf, das sind
$\land$ und $\Rightarrow$. Es gibt die grundlegenden Verknüpfungen
$\neg,\land,\lor,\Rightarrow,\Leftrightarrow$. Die Bindungsstärke
der gelisteten Verknüfpungen ist absteigend, so wie Punktrechnung
vor Strichrechnung gilt. Das $\neg$ bindet stärker als $\land$,
bindet stärker als $\lor$, bindet stärker als $\Rightarrow$,
bindet stärker als $\Leftrightarrow$. Die Verknüpfungen sind
in Tabelle \ref{tab:logische-Verknuepfungen} definiert.
Anstelle von $\neg A$ schreibt man auch $\overline A$.

\begin{table}
\centering
\begin{tabular}{cc}
\begin{tabular}{c|c}
$A$ & $\neg A$\\
\hline
$0$ & $1$\\
$1$ & $0$
\end{tabular}
&\qquad \begin{tabular}{c|c|c|c|c|c}
$A$ & $B$ & $A\land B$ & $A\lor B$ & $A\Rightarrow B$ & $A\Leftrightarrow B$\\
\hline
$0$ & $0$ & $0$ & $0$ & $1$ & $1$\\
$1$ & $0$ & $0$ & $1$ & $0$ & $0$\\
$0$ & $1$ & $0$ & $1$ & $1$ & $0$\\
$1$ & $1$ & $1$ & $1$ & $1$ & $1$
\end{tabular}
\end{tabular}
\caption{Definition der grundlegenden logischen Verknüpfungen.}
\label{tab:logische-Verknuepfungen}
\end{table}

\begin{table}
\centering
\begin{tabular}{c|c|c|c|c}
$A$ & $B$ & $A\land B$ & $B\land A$ & $A{\land}B\Rightarrow B{\land}A$\\
\hline
$0$ & $0$ & $0$ & $0$ & $1$\\
$1$ & $0$ & $0$ & $0$ & $1$\\
$0$ & $1$ & $0$ & $0$ & $1$\\
$1$ & $1$ & $1$ & $1$ & $1$
\end{tabular}
\caption{Wahrheitstafel zu $A\land B\Rightarrow B\land A$.}
\label{tab:Wahrheitstafel1}
\end{table}

Es gibt Formeln, die immer wahr sind, unabhängig davon, mit
welchen Wahrheitswerten die Variablen belegt werden.
\begin{Definition}[Tautologie]\index{Tautologie}
Ist $\varphi$ eine Formel, die bezüglich jeder möglichen
Variablenbelegung erfüllt ist, dann nennt man $\varphi$ eine Tautologie
und schreibt dafür kurz $\models\varphi$.
\end{Definition}
Z.\,B. gilt
\[\models A\land B\implies B\land A.\]
Es lässt sich leicht überprüfen, ob eine Formel tautologisch ist.
Dazu wird einfach die Wahrheitstafel zu dieser Formel aufgestellt,
hier Tabelle $\ref{tab:Wahrheitstafel1}$.
Die Wahrheitstafel\index{Wahrheitstafel} ist eine Wertetabelle,
die zu jeder Variablenbelegung den Wahrheitswert der Formel angibt.
Bei einer tautologischen Formel enthält die Ergebnisspalte in jeder
Zeile den Wert $1$.

Zwei wichtige Metaregeln, die Einsetzungsregel und die
Ersetzungsregel, ermöglichen das Rechnen mit aussagenlogischen
Formeln. Die Einsetzungsregel ermöglicht es, aus schon bekannten
Tautologien neue bilden zu können, ohne jedes mal eine Wahrheitstafel
aufstellen zu müssen. Die Ersetzungsregel ermöglicht die Umformung
von Formeln.

\begin{Satz}[Einsetzungsregel]\index{Einsetzungsregel}
Sei $v$ eine logische Variable. Ist $\varphi$ eine tautologische
Formel, dann ergibt sich wieder eine tautologische Formel, wenn man
jedes Vorkommen von $v$ in $\varphi$ durch eine Formel $\psi$ ersetzt.
Kurz:
\[(\models \varphi )\implies (\models \varphi [v:=\psi]).\]
Das gilt auch für die simultane Substitution:
\[(\models \varphi )\implies
(\models \varphi [v_1:=\psi_1,\ldots ,v_n:=\psi_n]).\]
\end{Satz}
\strong{Begründung.} Die Variable $v$ kann in $\varphi$
frei mit einem Wahrheitswert belegt werden, nach Voraussetzung
ist $\varphi$ dabei immer erfüllt. Somit ist $\varphi$ auch
erfüllt, wenn $v$ mit dem Wahrheitswert von $\psi$ belegt wird.
Dann muss aber auch $\varphi[v:=\psi]$ unter einer beliebigen Belegung
wahr sein.\;\qedsymbol

\newpage
\begin{Satz}[Ersetzungsregel]\index{Ersetzungsregel}
Sei $F(\varphi)$ eine Formel, welche von der Teilformel $\varphi$
abhängig ist. Sei außerdem $\varphi$ äquivalent zu $\psi$.
Dann sind auch $F(\varphi)$ und $F(\psi)$ äquivalent. Kurz:
\[(\models\varphi\Leftrightarrow\psi)
\implies (\models F(\varphi)\Leftrightarrow F(\psi)).\]
\end{Satz}
\strong{Begründung.}
Die Äquivalenz von $\varphi$ und $\psi$ erzwingt, dass $\psi$
unter einer beliebigen Belegung den gleichen Wahrheitswert besitzt
wie $\varphi$. Da $F(0)\Leftrightarrow F(0)$ und
$F(1)\Leftrightarrow F(1)$ gilt, muss also
$F(\varphi)\Leftrightarrow F(\psi)$ gelten.\;\qedsymbol

\begin{Satz}[Kleine Metaregel]
Es gilt $\models\varphi$ und $\models\psi$
genau dann, wenn $\models\varphi\land\psi$.
\end{Satz}
\strong{Beweis.}
Sind $\varphi,\psi$ tautologisch, dann dürfen sie durch
den Wahrheitswert wahr ersetzt werden. Unter dieser Voraussetzung
ist $\varphi\land\psi$ gleichbedeutend mit $1\land 1$, demnach
auch tautologisch.

Sei nun umgekehrt $\varphi\land\psi$ tautologisch. Es müssen
zwingend auch $\varphi$ und $\psi$ wahr sein, denn sonst wäre
$\varphi\land\psi$ falsch.\;\qedsymbol

\begin{Satz}[Kleine Abtrennungsregel]
Aus $\models\varphi$ und $\models\varphi\Rightarrow\psi$
folgt $\models\psi$.\\
Aus $\models\varphi$ und $\models\varphi\Leftrightarrow\psi$
folgt $\models\psi$.
\end{Satz}
\strong{Beweis.}
Ist $\varphi$ tautologisch, dann darf es durch den Wahrheitswert
wahr ersetzt werden. Unter dieser Voraussetzung ist
$\varphi\Rightarrow\psi$ gleichbedeutend mit $1\Rightarrow\psi$.
Diese Formel kann nur erfüllt sein, wenn auch $\psi$ wahr ist.
Da aber $\varphi\Rightarrow\psi$ tautologisch sein soll,
muss damit zwingend auch $\psi$ tautologisch sein.
Für $\varphi\Leftrightarrow\psi$ ist die Argumentation
analog.\;\qedsymbol

\begin{Satz}[Abtrennung von Implikationen]
Aus $\models\varphi\Leftrightarrow\psi$
folgt $\models\varphi\Rightarrow\psi$.
\end{Satz}
\strong{Beweis.}
Man zeigt
\[\models (A\Leftrightarrow B)
\Leftrightarrow (A\Rightarrow B)\land (B\Rightarrow A)\]
mittels Wahrheitstafel. Gemäß der Einsetzungsregel gilt dann auch
\[\models (\varphi\Leftrightarrow\psi)
\Leftrightarrow (\varphi\Rightarrow\psi)\land (\psi\Rightarrow\varphi).\]
Mit der kleinen Abtrennungsregel und der Voraussetzung erhält man
\[\models (\varphi\Rightarrow\psi)\land (\psi\Rightarrow\varphi).\]
Gemäß der kleinen Metaregel ergibt sich schließlich
$\models \varphi\Rightarrow\psi$.\;\qedsymbol

\newpage
\begin{Definition}[Äquivalente Formeln]
Zwei Formeln $\varphi,\psi$ heißen äquivalent, wenn die
Äquivalenz $\varphi\Leftrightarrow\psi$ tautologisch ist, kurz
\[(\varphi\equiv\psi)\defiff(\models\varphi\Leftrightarrow\psi).\]
\end{Definition}

\begin{Satz}
Die Relation $\varphi\equiv\psi$ ist eine Äquivalenzrelation, d.\,h.
es gilt
\begin{align}
& \varphi\equiv\varphi, && (\text{Reflexivität})\\
& (\varphi\equiv\psi)\implies (\psi\equiv\varphi), && (\text{Symmetrie})\\
& (\varphi\equiv\psi)\land (\psi\equiv\chi)\implies (\varphi\equiv\chi). && (\text{Transitivität})
\end{align}
\end{Satz}

\subsection{Boolesche Algebra}

\begin{table}
\begin{center}
\begin{tabular}{c|c|l}
\strong{UND}&
\strong{ODER}&
\strong{Gesetze}\\

$A\land B\equiv B\land A$ &
$A\lor B\equiv B\lor A$ &
Kommutativgesetze\\

$A\land (B\land C)\equiv (A\land B)\land C$ &
$A\lor(B\lor C)\equiv (A\lor B)\lor C$ &
Assoziativgesetze\\

$A\land A\equiv A$ &
$A\lor A\equiv A$ &
Idempotenzgesetze\\

$A\land 1\equiv A$ &
$A\lor 0\equiv A$ &
Neutralitätsgesetze\\

$A\land 0\equiv 0$ &
$A\lor 1\equiv 1$ &
Extremalgesetze\\

$A\land\overline A\equiv 0$ &
$A\lor\overline A\equiv 1$ &
Komplementärgesetze\\

$\overline{A\land B}\equiv\overline A\lor\overline B$ &
$\overline{A\lor B}\equiv\overline A\land\overline B$ &
De Morgansche Gesetze\\

$A\land (A\lor B)\equiv A$ &
$A\lor (A\land B)\equiv A$ &
Absorptionsgesetze

\end{tabular}
\caption{Die Regeln der booleschen Algebra.}
\label{tab:boolesche-Algebra}
\end{center}
\end{table}

Die Regeln in Tablelle \ref{tab:boolesche-Algebra} gewinnt man
alle mittels Warheitstafel. Gemäß der Einsetzungsregel dürfen für
die Variablen auch Formeln eingesetzt werden, die griechischen
Formelvariablen benötigt man somit nicht mehr.

Weiterhin gelten die Distributivgesetze
\begin{align}
A\land(B\lor C) &\equiv (A\land B)\lor (A\land C),\\
A\lor(B\land C) &\equiv (A\lor B)\land (A\lor C).
\end{align}
Schließlich gibt es noch das Involutionsgesetz
\begin{equation}
\overline {\overline A}\equiv A.
\end{equation}
Die Implikation und die Äquivalenz lassen sich auf NICHT, UND, ODER
zurückführen:%
\begin{align}
A\Rightarrow B &\equiv \overline A\lor B,\\
A\Leftrightarrow B &\equiv (A\Rightarrow B)\land (B\Rightarrow A).
\end{align}

\newpage
\section{Mengenlehre}

\subsection{Der Mengenbegriff}\index{Menge}

Eine Menge ist im Wesentlichen ein Beutel, der unterschiedliche
Objekte enthält. Es gibt die leere Menge, das ist der leere Beutel.
Das besondere an einer Menge ist nun, dass das selbe Objekt immer
nur ein einziges mal im Beutel enthalten ist. Legt man zweimal
das selbe Objekt in den Beutel, dann ist dieses darin trotzdem nur
einmal zu finden.

Man kann sich dabei z.\,B. einen Einkaufsbeutel vorstellen,
in welchem sich nur ein Apfel, eine Birne, eine Weintraube usw.
befinden darf. Möchte man mehrere Birnen im Einkaufsbeutel haben,
dann müssen diese unterschieden werden, z.\,B. indem jede Birne
eine unterschiedliche Nummer bekommt.

Möchte man eine Menge aufschreiben, werden die Objekte einfach
in einer beliebigen Reihenfolge aufgelistet und diese Liste in
geschweifte Klammern gesetzt. Z.\,B.:%
\[\{\mathrm{Afpel}, \mathrm{Birne}, \mathrm{Weintraube}\}.\]
Nennen wir den Apfel $A$, die Birne $B$
und die Weintraube $W$. Eine Menge mit zwei Äpfeln und drei
Birnen würde man so schreiben:%
\[\{A_1, A_2, B_1, B_2, B_3\}.\]
Erlaubt sind auch Beutel in Beuteln. Eine Menge mit zwei Äpfeln
und einer Menge mit vier Weintrauben wird beschrieben durch%
\[\{A_1, A_2, \{W_1,W_2,W_3,W_4\}\}.\]
Die Reihenfolge spielt wie gesagt keine Rolle:%
\[\{A_1,A_2\} = \{A_2,A_1\}.\]
Ein leerer Beutel ist etwas anderes als ein Beutel, welcher einen
leeren Beutel enthält:%
\[\{\} \ne \{\{\}\}.\]
Die Notation $x\in M$ bedeutet, dass $x$ in der Menge $M$ enthalten
ist. Man sagt, $x$ ist ein Element von $M$. Z.\,B. ist
\[A_1\in\{A_1,A_2\}.\]

\newpage
\subsection{Teilmengen}

\begin{Definition}[Teilmengenrelation]\index{Teilmenge}
Hat man zwei Mengen $M,N$, dann nennt man $M$ eine Teilmenge von $N$,
wenn jedes Element von $M$ auch ein Element von $N$ ist.
Als Formel:
\[M\subseteq N\defiff \text{für jedes $x\in M$ gilt $x\in N$}.\]
Anders formuliert, aber gleichbedeutend:
\[M\subseteq N\defiff \text{für jedes $x$ gilt:}\; (x\in M\implies x\in N).\]
\end{Definition}
Z.\,B. ist die Aussage $\{1,2\}\subseteq\{1,2,3\}$ wahr.
Die Aussage $\{1,2,3\}\subseteq\{1,2\}$ ist jedoch falsch,
weil $3$ kein Element von $\{1,2\}$ ist. Für jede Menge $M$ gilt
$M\subseteq M$, denn die Aussage
\[x\in M\implies x\in M\]
ist immer wahr, da die Formel »$\varphi\Rightarrow\varphi$«
tautologisch ist.

\subsection{Mengen von Zahlen}
\index{Zahlenbereiche}\index{Natürliche Zahlen}\index{ganze Zahlen}%
\index{reelle Zahlen}\index{Dezimalzahl}

Einige Mengen kommen häufiger vor, was dazu führte, dass man für
diese Mengen kurze Symbole definiert hat.

Die Menge der natürlichen Zahlen mit der Null:
\[\N_0 := \{0,1,2,3,4,\ldots\}.\]
Die Menge der natürlichen Zahlen ohne die Null:
\[\N := \{1,2,3,4,\ldots\}.\]
Die Menge der ganzen Zahlen:
\[\Z := \{\ldots,-4,-3,-2,-1,0,1,2,3,4,\ldots\}.\]
Dann gibt es noch die rationalen Zahlen $\Q$, das sind alle
Brüche der Form $m/n$, wobei $m,n$ ganze Zahlen sind
und $n\ne 0$ ist. Rationale Zahlen lassen sich immer als
Dezimalbruch schreiben, dessen Ziffern irgendwann periodisch
werden.

\begin{table}[h]
\centering
\begin{tabular}{c|l|l}
\strong{Zahl} & \strong{als Dezimalzahl} & \strong{kurz}\\
$1/2$ & $0.5000000000\ldots$ & $0.5\overline{0}$\\
$1/3$ & $0.3333333333\ldots$ & $0.\overline{3}$\\
$1241/1100$ & $1.1281818181\ldots$ & $0.12\overline{81}$
\end{tabular}
\caption{Jeder Bruch lässt sich als Dezimalzahl
schreiben, deren Ziffern in eine periodische Zifferngruppe münden.
Über die periodische Zifferngruppe setzt man einen waagerechten
Strich.}
\end{table}

\noindent
Schließlich gibt es noch die reellen Zahlen $\R$. Darin enthalten sind
alle Dezimalzahlen -- auch solche, deren Ziffern niemals in eine
periodische Zifferengruppe münden. Die reellen Zahlen haben
eine recht komplizierte Struktur, und wir benötigen Mittel
der Analysis um diese verstehen zu können. Solange diese Werkzeuge
noch nicht bekannt sind, kann man die reellen Zahlen einfach
als kontinuierliche Zahlengerade betrachten. Die rationalen
Zahlen haben Lücken in dieser Zahlengerade, z.\,B. ist die Zahl
$\sqrt{2}$ nicht rational, wie sich zeigen lässt. Die reellen
Zahlen schließen diese Lücken.

\subsection{Vergleich von Mengen}\index{Menge!Vergleich von Mengen}

Wie können wir denn wissen, wann zwei Mengen $A,B$, gleich sind?
Zwei Mengen sind ja gleich, wenn sie beide die gleichen Elemente
enthalten. Aber wie lässt sich das als mathematische Aussage
formulieren?

Jedes Element von $A$ muss doch auch ein Element von $B$ sein,
sonst gäbe es Elemente in $A$, die nicht in $B$ enthalten wären.
Umgekehrt muss auch jedes Element von $B$ ein Element von $A$ sein.
Also ist $A\subseteq B$ und $B\subseteq A$ eine notwendige Bedingung.
Diese Bedingung ist sogar hinreichend.

Gehen wir mal von der Kontraposition aus -- sind die beiden Mengen
$A,B$ verschieden, dann muss es ein Element in $A$ geben, welches nicht
in $B$ enthalten ist, oder eines in $B$, welches nicht $A$ enthalten
ist. Als Formel:%
\[A\ne B \implies \exists(x\in A)(x\notin B)\lor\exists(x\in B)(x\notin A).\]
Hiervon bildet man wieder die Kontraposition. Gemäß den
De Morganschen Gesetzen und den verallgemeinerten
De Morganschen Gesetzen ergibt sich%
\[\forall(x\in A)(x\in B)\land\forall(x\in B)(x\in A)\implies A=B.\]
Auf der linken Seite stehen aber nach Definition
Teilmengenbeziehungen, es ergibt sich%
\[A\subseteq B\land B\subseteq A\implies A=B.\]
\begin{Definition}[Gleichheit von Mengen]\index{Gleichheit!von Mengen}
Zwei Mengen $A,B$ sind genau dann gleich, wenn jedes Element von
$A$ auch in $B$ enthalten ist, und jedes von $B$ auch in $A$ enthalten:%
\[A=B\defiff A\subseteq B\land B\subseteq A.\]
\end{Definition}
\begin{Satz}\label{set-eq}
Es gilt
\[A=B\iff \forall x(x\in A\iff x\in B).\]
\end{Satz}
\strong{Beweis.} Wir müssen ein wenig Prädikatenlogik bemühen:%
\begin{align*}
A\subseteq B\land B\subseteq A
&\iff \forall(x\in A)(x\in B)\land\forall(x\in B)(x\in A)\\
&\iff \forall x(x\in A\implies x\in B)\land\forall x(x\in B\implies x\in A)\\
&\iff \forall x((x\in A\implies x\in B)\land (x\in B\implies x\in A))\\
&\iff \forall x(x\in A\iff x\in B).
\end{align*}
Im letzten Schritt wurde ausgenutzt, dass
$\varphi\Leftrightarrow\psi$ definitionsgemäß gleichbedeutend
mit $(\varphi\Rightarrow\psi)\land(\psi\Rightarrow\varphi)$
ist.\;\qedsymbol

%\newpage
\subsection{Beschreibende Angabe von Mengen}

Umso mehr Elemente eine Menge enthält, umso umständlicher wird
die Auflistung all dieser Elemente. Außerdem hantiert man in der
Mathematik normalerweise auch ständig mit Mengen herum, die
unendlich viele Elemente enthalten. Eine explizite Auflistung ist
demnach unmöglich.

Wir entgehen der Auflistung aller Elemente durch eine Beschreibung
der Menge. Die Menge der ganzen Zahlen, welche kleiner als vier sind,
wird so beschrieben:%
\[\{n\in\Z\mid n<4\}.\]
In Worten: Die Menge der $n\in\Z$, für die gilt: $n<4$.

Mit dieser Notation kann man nun z.\,B. schreiben:%
\begin{align*}
\N_0 &= \{n\in\Z\mid n\ge 0\},\\
\N &= \{n\in\Z\mid n>0\}.
\end{align*}
Mit der folgenden formalen Defintion wird die beschreibende Angabe
auf ein festes Fundament gebracht.

\begin{Definition}[Beschränkte Beschreibung einer Menge]%
\label{def:set-builder-bounded}\index{Menge!Comprehension}
Die Menge der $x\in M$, welche die Aussage $P(x)$ erfüllen,
ist definiert durch die folgende logische Äquivalenz:%
\[a\in\{x\in M\mid P(x)\} \defiff a\in M\land P(a).\]
\end{Definition}
Das schaut ein wenig kompliziert aus, ist aber ganz einfach zu
benutzen. Sei z.\,B. $A:=\{n\in\Z\mid n<4\}$. Zu beantworten ist
die Frage, ob $2\in A$ gilt. Eingesetzt in die Definition
ergibt sich%
\[2\in\{n\in\Z\mid n<4\}\iff 2\in\Z\land 2<4.\]
Da $2\in\Z$ und $2<4$ wahre Aussagen sind, ist die rechte Seite
erfüllt, und damit auch die linke Seite der Äquivalenz.

Die geraden Zahlen lassen sich so definieren:%
\[2\Z:=\{n\in\Z\mid\text{es gibt ein $k\in\Z$ mit $n=2k$}\}.\]
Es lässt sich zeigen:
\[a\in 2\Z\implies a^2\in 2\Z.\]
Nach Definition von $2\Z$ gibt es $k\in\Z$ mit $a=2k$.
Dann ist $a^2=(2k)^2=4k^2=2(2k^2)$. Benennt man $k':=2k^2$, dann
gilt also $a^2=2k'$. Also gibt es es ein $k'\in\Z$
mit $a^2=2k'$, und daher ist $a^2\in 2\Z$.

Die graden Zahlen sind ganze Zahlen, welche ohne Rest durch zwei teilbar
sind. Die ganzen Zahlen, welche ohne Rest durch $m$ teilbar sind,
lassen sich formal so definieren:%
\[m\Z:=\{n\in\Z\mid\text{es gibt ein $k\in\Z$ mit $n=mk$}\}.\]
Man zeige:
\begin{align*}
& (1.)\;\;a\in 2\Z\implies a^2\in 4\Z, && (3.)\;\;2\Z\subseteq\Z,\\
& (2.)\;\;a\in 4\Z\implies a\in 2\Z,   && (4.)\;\;4\Z\subseteq 2\Z.
\end{align*}


\begin{Definition}[Beschreibende Angabe einer Menge]%
\label{def:set-builder}
Stellt man sich unter $G$ die Grundmenge vor, welche
alle Elemente enthält, die überhaupt in Betracht kommen können,
dann schreibt man kurz%
\[\{x\mid P(x)\} := \{x\in G\mid P(x)\}\]
und nennt dies die Beschreibung einer Menge.
\end{Definition}
\begin{Satz}
Es gilt
\begin{gather}
\label{eq:set-builder}
a\in\{x\mid P(x)\}\iff P(a),\\
\label{eq:bound-conversion}
\{x\in A\mid P(x)\} = \{x\mid x\in A\land P(x)\}.
\end{gather}
\end{Satz}
\strong{Beweis.} Gemäß Definition \ref{def:set-builder}
und \ref{def:set-builder-bounded} gilt%
\[a\in\{x\mid P(x)\} \iff a\in\{x\in G\mid P(x)\}
\iff a\in G\land P(a)\iff P(a),\]
denn $a\in G$ ist immer erfüllt, wenn $G$ die Grundmenge ist.
Die Aussage $a\in G$ kann daher in der Konjunktion gemäß dem
Neutralitätsgesetz der booleschen Algebra entfallen.

Aussage \eqref{eq:bound-conversion} wird mit Satz \ref{set-eq}
expandiert. Zu zeigen ist nun
\[a\in\{x\in A\mid P(x)\}\iff a\in\{x\mid x\in A\land P(x)\},\]
was gemäß Definition \ref{def:set-builder-bounded} und der schon
bewiesenen Aussage \eqref{eq:set-builder} aber vereinfacht
werden kann zu
\[a\in A\land P(a)\iff a\in A\land P(a).\;\qedsymbol\]


\subsection{Bildmengen}\index{Bildmenge}

Oft kommt auch die Angabe einer Menge als Bildmenge vor, dabei
handelt es sich um eine spezielle Beschreibung der Menge. Ist
$T(x)$ ein Term und $A:=\{a_1,a_2,\ldots,a_n\}$ eine endliche
Menge, dann wird das Bild von $A$ unter $T(x)$ so beschrieben:
\[\{T(x)\mid x\in A\} := \{T(a_1),T(a_2),\ldots, T(a_n)\}.\]
Lies: Die Menge der $T(x)$, für die $x\in A$ gilt.
Für $T(x):=x^2$ und $A:=\{1,2,3,4\}$ ist z.\,B.
\[\{T(x)\mid x\in A\} = \{T(1), T(2), T(3), T(4)\}
= \{1^2,2^2,3^2,4^2\} = \{1,4,9,16\}.\]
Nun kann es aber sein, dass die Menge $A$ unendlich viele Elemente
enthält, eine Auflistung dieser somit unmöglich ist. Eine Auflistung
lässt umgehen, indem man nur logisch die Existenz eines Bildes
zu jedem $x\in A$ verlagt, dieses aber nicht mehr explizit angibt.
Man definiert also allgemein
\[\{T(x)\mid x\in A\} := \{y\mid\text{es gibt ein $x\in A$, für das gilt: $y=T(x)$}\}.\]
Das hatten wir bei den geraden Zahlen
\[2\Z := \{2k\mid k\in\Z\} = \{n\mid\text{es gibt ein $k\in\Z$, für das gilt: $n=2k$}\}\]
schon kennengelernt. Hierbei ist es unwesentlich, ob man $n\in\Z$ verlangt
oder nicht, denn dies wird bereits durch $k\in\Z$ erzwungen.

\newpage
\subsection{Mengenoperationen}

Mengen sind mathematische Objekte, mit denen sich rechnen lässt.
So wie es für Zahlen Rechenoperationen gibt, gibt es auch für
Mengen Rechenoperationen.
\begin{Definition}[Vereinigungsmenge]%
\index{Vereinigungsmenge}\index{Menge!Vereinigung}
Die Vereinigungsmenge von zwei Mengen $A,B$ ist die Menge aller Elemente,
welche in $A$ oder in $B$ vorkommen:
\[A\cup B := \{x\mid x\in A\lor x\in B\}.\]
\end{Definition}
Man nimmt also einen neuen Beutel und schüttet den Inhalt von $A$
und $B$ in diesen Beutel.

Beispiele:
\begin{gather*}
\{1,2\}\cup\{5,7,9\} = \{1,2,5,7,9\},\\
\{1,2\}\cup\{1,3,5\} = \{1,2,3,5\}.
\end{gather*}

\begin{Definition}[Schnittmenge]%
\index{Schnittmenge}\index{Menge!Schnitt}
Die Schnittmenge von zwei Mengen $A,B$ ist die Menge aller Elemente,
welche sowohl in $A$ also auch in $B$ vorkommen:
\[A\cap B := \{x\mid x\in A\land x\in B\}.\]
\end{Definition}
\begin{Satz}
Bei der Beschreibung der Schnittmenge $A\cap B$ genügt es, $A\cup B$ als Grundmenge
zu verwenden, denn es gilt
\[A\cap B = \{x\in A\cup B\mid x\in A\land x\in B\}\]
\end{Satz}
\strong{Beweis.}
Die Formel wird mit Satz \ref{set-eq} expandiert. Zu zeigen ist demnach
\[a\in A\cap B\iff a\in \{x\in A\cup B\mid x\in A\land x\in B\}.\]
Das ist nach \eqref{eq:set-builder} und Definition
\ref{def:set-builder-bounded} gleichbedeutend mit
\begin{align*}
a\in A\land a\in B&\iff a\in A\cup B\land a\in A\land a\in B\\
&\iff (a\in A\lor a\in B)\land a\in A\land a\in B.
\end{align*}
Nun gilt für beliebige Aussagen $\varphi,\psi$ gemäß boolescher Algebra aber
\begin{align*}
(\varphi\lor\psi)\land\varphi\land\psi
&\iff (\varphi\land\varphi\land\psi)\lor(\psi\land\varphi\land\psi)\\
&\iff (\varphi\land\psi)\lor(\varphi\land\psi)\\
&\iff \varphi\land\psi.
\end{align*}
Auf beiden Seiten der Äquivalenz steht jetzt die gleiche Aussage:%
\[a\in A\land a\in B\iff a\in A\land a\in B.\;\qedsymbol\]

\newpage
\section{Gleichungen}
\subsection{Begriff der Gleichung}\index{Gleichung}

Bei einer Gleichung verhält es sich wie bei einer Balkenwaage. Liegt
in einer der Waagschalen eine Masse von 2g und in der anderen
Waagschale zwei Massen von jeweils 1g, dann bleibt die Waage im
Gleichgewicht. Als Gleichung gilt
\[2=1+1.\]
Eine Gleichung kann wahr oder falsch sein, z.\,B. ist $2=2$
wahr, während $2=3$ falsch ist. Das bedeutet aber nicht, dass man
eine falsche Gleichung nicht aufschreiben dürfe. Vielmehr ist eine
Gleichung ein mathematisches Objekt, dem sich ein Wahrheitswert
zuordnen lässt. Zumindest sollte man eine falsche Gleichung nicht
ohne zusätzliche Erklärung aufschreiben, so dass der Eindruck
entstünde, sie könnte wahr sein.

\subsection{Äquivalenzumformungen}%
\index{Aequivalenzumformung@Äquivalenzumformung!von Gleichungen}

Fügt man zu beiden Schalen einer Balkenwaage das gleiche Gewicht
hinzu, dann bleibt die Waage so wie sie vorher war. War sie im
Gleichgewicht, bleibt sie dabei. War sie im Ungleichgewicht,
bleibt sie auch dabei. Ebenso verhält es sich mit einer Gleichung.
Addition der gleichen Zahl auf beide Seiten einer Gleichung bewirkt
keine Veränderung des Aussagengehalts der Gleichung.

Diese Überlegung gilt natürlich auf für die Subtraktion einer Zahl
auf beiden Seiten, welche dem Entfernen des gleichen Gewichtes von
beiden Waagschalen entspricht.

\begin{Satz}[Äquivalenzumformungen]\label{eq-add}
Seien $a,b,c$ beliebige Zahlen. Dann gilt
\begin{align*}
a=b&\iff a+c=b+c,\\
a=b&\iff a-c=b-c.
\end{align*}
\end{Satz}

\noindent
Auch eine Verdopplung des Gewichtes in beiden Schalen der Balkenwaage
ändert nicht ihr Gleichgewicht oder Ungleichgewicht.

\begin{Satz}[Äquivalenzumformungen]\label{eq-mul-int}
Seien $a,b$ beliebige Zahlen und $n\in\Z$ mit $n\ne 0$. Dann gilt
\begin{align*}
a=b&\iff na=nb.
\end{align*}
\end{Satz}
\strong{Beweis.} Gemäß Satz \ref{eq-add} gilt
\begin{align*}
na = nb &\iff 0 = na-nb = n(a-b)\iff n=0\lor a-b=0\\
&\iff a-b=0\iff a=b.
\end{align*}
Dabei wurde ausgenutzt, dass ein Produkt nur null sein kann,
wenn einer der Faktoren null ist. Gemäß Voraussetzung $n\ne 0$ muss
dann aber $a-b=0$ sein.\;\qedsymbol

\begin{Satz}[Äquivalenzumformungen]
Seien $a,b$ beliebige Zahlen und $r\in\Q$ mit $r\ne 0$. Dann gilt
\[a=b\iff ra=rb\iff a/r=b/r.\]
\end{Satz}
\strong{Beweis.}
Die Zahl $r$ ist von der Form $r=m/n$, wobei $m,n\in\Z$ und $m,n\ne 0$.
Daher gilt
\begin{align*}
ra=rb&\iff \frac{m}{n}a=\frac{m}{n}b
\stackrel{\text{Satz \ref{eq-mul-int}}}\iff n\cdot\frac{m}{n}a=n\cdot\frac{m}{n}b\\
&\iff ma=mb\stackrel{\text{Satz \ref{eq-mul-int}}}\iff a=b.
\end{align*}
Daraufhin gilt auch
\[\frac{a}{r}=\frac{b}{r}\iff r\cdot\frac{a}{r}=r\cdot\frac{b}{r}
\iff a=b.\;\qedsymbol\]

\begin{Satz}[Äquivalenzumformungen]
Seien $a,b,r\in\R$ und sei $r\ne 0$. Dann gilt
\[a=b\iff ra=rb\iff a/r=b/r.\]
\end{Satz}
\strong{Beweis.} Man rechnet wieder
\begin{align*}
ra = rb&\iff ra-rb=0\iff (a-b)r=0\iff r=0\lor a-b=0\\
&\iff a-b=0\iff a=b.
\end{align*}
Es wurde wieder ausgenutzt, dass ein Produkt nur dann null sein
kann, wenn einer der Faktoren null ist. Daraufhin gilt auch
\[\frac{a}{r}=\frac{b}{r}\iff r\cdot\frac{a}{r}=r\cdot\frac{b}{r}
\iff a=b.\;\qedsymbol\]

\noindent


\newpage
\section{Ungleichungen}

\subsection{Begriff der Ungleichung}%
\index{Ungleichung}

Man stelle sich zwei Körbe vor, in die Äpfel gelegt werden.
In den rechten Korb werden zwei Äpfel gelegt, in den linken drei.
Dann befinden sich im rechten Korb weniger Äpfel als im linken.
Man sagt, zwei ist kleiner als drei, kurz $2<3$. Man spricht von
einer \emph{Ungleichung}, in Anbetracht dessen, dass die beiden
Körbe nicht die gleiche Anzahl von Äpfeln enthalten.

Der Aussagengehalt einer Ungleichung kann wahr oder falsch sein.
Die Ungleichung $2<3$ ist wahr, die Ungleichungen $3<3$ und
$4<3$ sind falsch.

\begin{Definition}[Ungleichungsrelation]
Die Notation $a<b$ bedeutet »Die Zahl $a$ ist kleiner als
die Zahl $b$«. Die Notation $a\le b$ bedeutet »Die Zahl $a$
ist kleiner als oder gleich der Zahl $b$«. Die Notation
$b>a$ ist eine andere Schreibweise für $a<b$ und bedeutet
»Die Zahl $b$ ist größer als die Zahl $a$«. Die Notation
$b\ge a$ ist eine andere Schreibweise für $a\le b$ und
bedeutet »Die Zahl $b$ ist größer oder gleich der Zahl $a$«.
\end{Definition}

\subsection{Äquivalenzumformungen}%
\index{Aequivalenzumformung@Äquivalenzumformung!von Ungleichungen}

Wir stellen uns wieder einen linken Korb mit zwei Äpfeln und
einen rechten Korb mit drei Äpfeln vor. Legt man nun in beide
Körbe jeweils zusätzlich 10 Äpfel hinein, dann befinden sich
im linken Korb 12 Äpfel und im rechten 13. Der linke Korb
enthält also immer noch weniger Äpfel als im rechten.

Befindet sich eine Balkenwaage im Ungleichgewicht, und legt man
in beide Waagschalen zusätzlich die gleiche Masse von Gewichten,
dann wird sich das Ungleichgewicht der Balkenwaage nicht verändern.

Für die Herausnahme von Äpfeln oder Gewichten ist diese Argumentation
analog. Ist stattdessen eine falsche Ungleichung gegeben,
dann lässt sich durch Addition der selben Zahl auf beiden Seiten
daraus keine wahre Ungleichung gewinnen. Die analoge Argumentation
gilt für die Subtraktion der selben Zahl. Anstelle von ganzen
Äpfeln kann man natürlich auch Apfelhälften hinzufügen, oder
allgmein Apfelbruchteile. Die Argumentation gilt unverändert.

Wir halten fest. 

\begin{Satz}[Äquivalenzumformungen von Ungleichungen]
Seien $a,b,c$ beliebige Zahlen. Dann sind die folgenden
Äquivalenzen gültig:
\begin{gather}
\label{lt-add} a<b\iff a+c<b+c,\\
\label{lt-sub} a<b\iff a-c<b-c,\\
\label{le-add} a\le b\iff a+c\le b+c,\\
\label{le-sub} a\le b\iff a-c\le b-c.
\end{gather}
\end{Satz}

\noindent
In Worten: Wenn auf beiden Seiten einer Ungleichung die gleiche
Zahl addiert oder subtrahiert wird, dann ändert sich der Aussagengehalt
dieser Ungleichung nicht.

Gibt es noch andere Äquivalenzumformungen?

Im linken Korb seien wieder zwei Äpfel, im rechten drei. Verdoppelt
man nun die Anzahl in beiden Körben, dann sind linken vier Äpfel,
im rechten sechs. Verzehnfacht man die Anzahl, dann sind im linken
20 Äpfel, im rechten 30. Offenbar verändert sich der Aussagengehalt
nicht, wenn die Anzahl auf beiden Seiten der Ungleichung mit
der gleichen natürlichen Zahl $n$ multipliziert wird.

Jedoch muss $n=0$ ausgeschlossen werden. Wenn $a<b$ ist, und man
multipliziert auf beiden Seiten mit null, dann ergibt sich
$0<0$, was falsch ist. Aus der wahren Ungleichung wurde damit eine
falsche gemacht, also kann es sich nicht um eine Äquivalenzumformung
handeln.

Auch bei der Ungleichung $a\le b$ muss $n=0$ ausgeschlossen werden.
Warum muss man das tun? Die Ungleichung $0\le 0$ ist doch auch
wahr?

Nun, wenn der Aussagengehalt von $a\le b$ falsch ist, z.\,B. $4\le 3$,
und man multipliziert auf beiden Seiten mit null, dann ergibt sich
$0\le 0$, also eine wahre Ungleichung. Aus einer falschen wurde damit
eine wahre gemacht. Bei einer Äquivalenzumformung ist dies ebenfalls
verboten.

\begin{Satz}[Äquivalenzumformungen von Ungleichungen]
Seien $a,b$ beliebige Zahlen und sei $n>0$ eine natürliche Zahl.
Dann sind die folgenden Äquivalenzen gültig:
\begin{gather}
\label{lt-mul-nat} a<b\iff na<nb,\\
\label{le-mul-nat} a\le b\iff na\le nb.
\end{gather}
\end{Satz}

\noindent\strong{Beweis.}
Aus der Ungleichung $a<b$ erhält man mittels \eqref{lt-sub} die
äquivalente Ungleichung $0<b-a$, indem auf beiden Seiten $a$
subtrahiert wird. Die Zahl $b-a$ ist also positiv. Durch Multiplikation
mit einer positiven Zahl lässt sich das Vorzeichen einer Zahl
aber nicht umkehren. Demnach ist $0<n(b-a)$ genau dann,
wenn $0<b-a$ war. Ausmultiplizieren liefert nun
$0<nb-na$ und Anwendung von \eqref{lt-add} bringt dann $na<nb$.

In Kürze formuliert:
\begin{equation}
a<b\iff 0<b-a\iff 0<n(b-a)=nb-na \iff na<nb.
\end{equation}
Für $a\le b$ gilt diese Überlegung analog.\;\qedsymbol

\strong{Alternativer Beweis.}
Mittels \eqref{lt-add} ergibt sich zunächst:
\begin{equation}
a<b\iff \left\{
\begin{matrix}
a+a<b+a\\
a+b<b+b
\end{matrix}
\right\}
\iff 2a<a+b<2b.
\end{equation}
Unter nochmaliger Anwendung von \eqref{lt-add} ergibt sich
nun
\begin{equation}
a<b\iff \left\{
\begin{matrix}
2a<a+b \iff 3a<2a+b\\
2a<2b \iff 2a+b<3b
\end{matrix}
\right\} 3a<2a+b<3b
\end{equation}
Dieses Muster lässt sich induktiv alle natürlichen Zahlen hochschieben:
Aus $na<(n-1)a+b<nb$ sollte sich
$(n+1)a<na+b<(n+1)b$ schlussfolgern lassen und umgekehrt.
Das ist richtig, denn Addition von $a$ gemäß \eqref{lt-add} bringt
\begin{equation}
na<(n-1)a+b \iff (n+1)a < na+b
\end{equation}
und Addition von $b$ gemäß \eqref{lt-add} bringt
\begin{equation}
na<nb \iff na+b < (n+1)b.
\end{equation}
Zusammen ergibt sich daraus der behauptete Induktionsschritt. 
Daraus erhält man $a<b\iff na<nb$. Für $a\le b$ sind diese
Überlegungungen analog.\;\qedsymbol

Wir können sogleich einen Schritt weiter gehen.
\begin{Satz}[Äquivalenzumformungen von Ungleichungen]
Seien $a,b$ beliebige Zahlen und sei $r>0$ eine rationale Zahl,
dann gelten die folgenden Äquivalenzen:
\begin{gather}
\label{lt-mul-rat} a<b\iff ra<rb\iff a/r<b/r,\\
\label{lt-mul-rat} a\le b\iff ra\le rb\iff a/r\le b/r.
\end{gather}
\end{Satz}

\noindent\strong{Beweis.}
Eine rationale Zahl $r>0$ lässt sich immer Zerlegen in einen Quotienten
$r=m/n$, wobei $m,n$ positive natürliche Zahlen sind. Gemäß
\eqref{lt-mul-nat} gilt
\begin{equation}
\frac{m}{n}\cdot a<\frac{m}{n}\cdot b
\iff n\cdot\frac{m}{n}\cdot a<n\cdot\frac{m}{n}\cdot b
\iff ma<mb.
\end{equation}
Gemäß \eqref{lt-mul-nat} gilt aber auch
\begin{equation}
a<b\iff ma<mb.
\end{equation}
Die Zusammenfassung beider Äquivalenzen ergibt
\begin{equation}
a<b\iff \frac{m}{n}\cdot a<\frac{m}{n}\cdot b\iff ra<rb.
\end{equation}
Für $a\le b$ ist die Argumentation analog. Da die Division durch
eine rationale Zahl $r$ die Multiplikation mit ihrem Kehrwert $1/r$ ist,
sind auch die Äquivalenzen für die Division gültig.\;\qedsymbol

Da sich eine reelle Zahl beliebig gut durch eine rationale annähern
lässt, müsste auch der folgende Satz gültig sein.

\begin{Satz}[Äquivalenzumformungen von Ungleichungen]
Seien $a,b$ beliebige Zahlen und sei $r>0$ eine reelle Zahl,
dann gelten die folgenden Äquivalenzen:
\begin{gather}
\label{lt-mul-real} a<b\iff ra<rb\iff a/r<b/r,\\
\label{lt-mul-real} a\le b\iff ra\le rb\iff a/r\le b/r.
\end{gather}
\end{Satz}

\noindent
Der Satz wird sich als richtig erweisen, der Beweis kann in
Analysis"=Lehrbüchern nachgeschlagen werden.

\subsection{Lineare Ungleichungen}

Interessant werden Ungleichungen nun, wenn in ihnen einen Variable
vorkommt. Beispielsweise sei die Ungleichung $x+2<4$ gegeben.
Wird in diese Ungleichung für die Variable $x$ eine Zahl eingesetzt,
dann kann wird die Ungleichung entweder wahr oder falsch sein.
Für $x:=1$ ergibt sich die wahre Ungleichung $1+2<4$. Für $x:=2$
ergibt sich jedoch die falsche Ungleichung $2+2<4$.

Wir interessieren uns nun natürlich für die Menge aller Lösungen
dieser Ungleichung. Das sind die Zahlen, welche die Ungleichung
erfüllen, wenn sie für $x$ eingesetzt werden. Gesucht ist also
die Lösungsmenge
\[L = \{x\mid x+2<4\},\]
d.\,h. die Menge der $x$, welche die Ungleichung $x+2<4$ erfüllen.

Gemäß Äquivalenzumformung \eqref{lt-sub} kommt man aber sofort zu
\[x+2<4 \iff x+2-2<4-2 \iff x<2.\]
Demnach kann die Lösungsmenge als $L=\{x\mid x<2\}$ angegeben werden,
denn Äquivalenzumformungen lassen die Lösungsmenge einer Ungleichung
unverändert.

Die Ungleichung $x+2<4$ ist sicherlich von so einfacher Gestalt,
dass man diese auch gedanklich lösen kann, ohne Äquivalenzumformungen
bemühen zu müssen. Bei komplizierteren Ungleichungen kommen wir dabei
aber mehr oder weniger schnell an unsere mentalen Grenzen.

Schon ein wenig schwieriger ist z.\,B.
\begin{align*}
& 5x+2<3x+10 && |\;{-2}\\
\iff & 5x<3x+8 && |\;{-3x}\\
\iff & 2x<8 && |\;{/2}\\
\iff & x<4.
\end{align*}

\subsection{Monotone Funktionen}%
\index{Monotone Funktion}

\begin{Definition}[Streng monoton steigende Funktion]%
\index{Streng monotone Funktion}\index{Monotone Funktion!strenge Monotonie}
Eine Funktion $f\colon G\to\R$ heißt streng monoton steigend, wenn
\[a<b\implies f(a)<f(b)\]
für alle Zahlen $a,b\in G$ erfüllt ist.
\end{Definition}
Streng monotone Abbildungen sind von besonderer Bedeutung, weil
sie gemäß ihrer Definition auch Äquivalenzumformungen sind:

\begin{Satz}[Allgemeine Äquivalenzumformung]%
\index{Aequivalenzumformung@Äquivalenzumformung!allgemein für Ungleichungen}
Eine streng monoton steigende Funktionen $f$ ist umkehrbar eindeutig.
Die Umkehrfunktion ist auch streng monoton steigend. D.\,h.
\[a<b\iff f(a)<f(b).\]
Demnach ist die Anwendung einer streng monoton steigenden
Funktion eine Äquivalenzumformung.
\end{Satz}
\noindent\strong{Beweis.}
Zu zeigen ist $a\ne b\implies f(a)\ne f(b)$. Wenn aber $a\ne b$
ist, dann ist entweder $a<b$ und daher nach Voraussetzung
$f(a)<f(b)$ oder $b<a$ und daher nach Voraussetzung $f(b)<f(a)$.
In beiden Fällen ist $f(a)\ne f(b)$.

Seien nun $y_1,y_2$ zwei Bilder der streng monotonen Funktion $f$.
Zu zeigen ist $y_1<y_2\implies f^{-1}(y_1)<f^{-1}(y_2)$.
Stattdessen kann auch die Kontraposition
$f^{-1}(y_2)\le f^{-1}(y_1)\implies y_2\le y_1$ gezeigt werden.
Das lässt sich nun aus der strengen Monotonie von $f$ schließen:
\begin{equation}
f^{-1}(y_2)\le f^{-1}(y_1)\implies
\underbrace{f(f^{-1}(y_2))}_{=y_2}\le \underbrace{f(f^{-1}(y_1))}_{=y_1}.\;\qedsymbol
\end{equation}

\begin{Definition}[Streng monoton fallende Funktion]
Eine Funktion $f\colon G\to\R$ heißt streng monoton fallend, wenn
\[a<b\implies f(a)>f(b)\]
für alle Zahlen $a,b\in G$ erfüllt ist.
\end{Definition}

\noindent
Ein entsprechender Satz gilt auch für diese:
\begin{Satz}[Allgemeine Äquivalenzumformung]
Eine streng monoton fallende Funktion $f$ ist umkehrbar eindeutig.
Die Umkehrfunktion ist auch streng monoton fallend. D.\,h.
\[a<b\iff f(a)>f(b).\]
Demnach ist die Anwendung einer streng monoton fallenden
Funktion eine Äquivalenzumformung bei der sich das Relationszeichen
umdreht.
\end{Satz}

\noindent
Tatsächlich haben wir schon streng monoton steigende Funktionen
kennengelernt. Z.\,B. ist \eqref{lt-add} nichts anderes als die strenge
Monotonie für $f(x):=x+c$. Und \eqref{lt-mul-nat} ist die strenge
Monotonie für $f(x):=nx$.

Die Funktion $f\colon\R\to\R$ mit $f(x):=x^2$ ist nicht streng monoton
steigend. Zum Beispiel ist $-4<-2$, aber $16=f(-4)>f(-2)=4$. Auch
ist die Funktion nicht streng monoton fallend, denn $2<4$,
aber $4=f(2)<f(4)=16$. Schränkt man $f$
auf den Definitionsbereich $\R_{>0}$ ein, so ergibt sich jedoch eine
streng monoton steigende Funktion. Das lässt sich wie folgt zeigen.

Nach Voraussetzung sind $a,b\in\R_{>0}$, d.\,h. $a,b>0$.
Also kann gemäß \eqref{lt-mul-real} einerseits mit $a$
und andererseits mit $b$ multipliziert werden:
\[
a<b\iff\begin{Bmatrix}
a^2<ab\\
ab<b^2
\end{Bmatrix}
\iff a^2<ab<b^2.
\]






\chapter{Ansätze zur Problemlösung}

\section{Substitution}

\subsection{Quadratische Gleichungen}%
\index{quadratische Gleichung}\index{Gleichung!quadratische}

Vorgelegt ist eine quadratische Gleichung in Normalform
\begin{equation}\index{Normalform!einer quadratischen Gleichung}
x^2+px+q = 0.
\end{equation}
Interessanterweise lässt sich der lineare Term $px$ durch Darstellung
der Gleichung über eine Translation $x=u+d$ eliminieren. Einsetzen
dieser Substitution bringt
\begin{align}
0 &= (u+d)^2+p(u+d)+q = u^2+2ud+d^2+pu+pd+q\\
&= u^2+(p+2d)u+(d^2+pd+q).
\end{align}
Setzt man nun $p+2d=0$, dann ergibt sich daraus $d=-p/2$ und somit
\begin{align}
q' &:= d^2+pd+q = \Big(-\frac{p}{2}\Big)^2-p\cdot\frac{p}{2}+q
= \frac{p^2}{4}-\frac{p^2}{2}+q\\
&= \frac{p^2}{4}-2\frac{p^2}{4}+q = -\frac{p^2}{4}+q.
\end{align}
Zu lösen ist nunmehr die quadratische Gleichung
\begin{equation}
u^2+q' = 0.
\end{equation}
Aber das ist ganz einfach, die Lösungen sind $u_1=+\sqrt{-q'}$
und $u_2=-\sqrt{-q'}$, sofern $q'\le 0$,
bzw. äquivalent $-q'\ge 0$. Wir schreiben kurz $u=\pm\sqrt{-q'}$.
Resubstitution von $u=x-d$ und $q'$ führt zu
\begin{equation}
x-d = x+\frac{p}{2} = \pm\sqrt{\frac{p^2}{4}-q} = \pm\frac{1}{2}\sqrt{p^2-4q}.
\end{equation}
Man erhält die Lösungsformel
\begin{equation}
x = -\frac{p}{2}\pm\frac{1}{2}\sqrt{p^2-4q}.
\end{equation}

\subsection{Biquadratische Gleichungen}%
\index{biquadratische Gleichung}\index{Gleichung!biquadratische}
Die biquadratische Gleichung
\begin{equation}
x^4+px^2+q = 0
\end{equation}
lässt sich über die Substitution $u=x^2$ auf die quadratische Gleichung
\begin{equation}
u^2+pu+q
\end{equation}
reduzieren. Für $p^2-4q\ge 0$ ergeben sich zwei Lösungen $u_1,u_2$,
wobei eventuell $u_1=u_2$ ist. Nun können sich bis zu vier Lösungen
für die ursprüngliche Gleichung ergeben. Das ist der Fall,
wenn $u_1\ne u_2$ und $u_1,u_2>0$. Dann
ergibt sich
\begin{equation}
x_1=\sqrt{u_1},\quad x_2=-\sqrt{u_1},\quad
x_3=\sqrt{u_2},\quad x_4=-\sqrt{u_2}
\end{equation}


\chapter{Kombinatorik}

\section{Endliche Mengen}

\subsection{Indikatorfunktion}

\begin{Definition}[Iverson-Klammer]\newlinefirst
Für eine Aussage $A$ der klassischen Aussagenlogik definiert man
\[[A] := \begin{cases}
1 &\text{wenn}\;A,\\
0 &\text{sonst}.
\end{cases}\]
\end{Definition}

\begin{Satz}\label{iverson-basic-rules}
Es gilt
\begin{gather*}
[A\land B] = [A][B],\\
[A\lor B] = [A]+[B]-[A][B],\\
[\neg A] = 1-[A],\\
[A\to B] = 1-[A](1-[B]).
\end{gather*}
\end{Satz}
\begin{Beweis} Trivial mittels Wertetabelle.\,\qedsymbol
\end{Beweis}

\begin{Satz}\label{indicator-set-op}
Für die Indikatorfunktion $1_M(x):=[x\in M]$ gilt
\begin{gather*}
1_{A\cap B} = 1_A 1_B,\\
1_{A\cup B} = 1_A + 1_B - 1_{A\cap B}.
\end{gather*}
\end{Satz}
\begin{Beweis}
Gemäß Satz \ref{iverson-basic-rules} gelten die
Rechnungen
\begin{align*}
1_{A\cap B}(x) = [x\in A\cap B]
= [x\in A\land x\in B] = [x\in A][x\in B] = 1_A(x)1_B(x)
\end{align*}
und
\begin{align*}
1_{A\cup B}(x) &= [x\in A\cup B] = [x\in A\lor x\in B]
= [x\in A] + [x\in B] - [x\in A][x\in B]\\
&= 1_A(x) + 1_B(x) - 1_{A\cap B}(x).\,\qedsymbol
\end{align*}
\end{Beweis}

\begin{Satz}
Für endliche Mengen $A,B$ gilt $|A\cup B| = |A|+|B|-|A\cap B|$.
\end{Satz}
\begin{Beweis}
Gemäß Satz \ref{indicator-set-op} darf man rechnen
\begin{align*}
|A\cup B| &= \sum_{x\in G} 1_{A\cup B}(x)
= \sum_{x\in G} (1_A(x) + 1_B(x) - 1_{A\cap B}(x))\\
&= \sum_{x\in G} 1_A(x) + \sum_{x\in G} 1_B(x) - \sum_{x\in G} 1_{A\cap B}(x)
= |A| + |B| - |A\cap B|.\,\qedsymbol
\end{align*}
\end{Beweis}

\begin{Satz}
Für endliche Mengen $A,B$ mit $A\subseteq B$ gilt $|A|\le |B|$.
\end{Satz}
\begin{Beweis}
Mithilfe der Indiktorfunktion findet sich
\begin{gather*}
A\subseteq B \iff (\forall x\colon 1_A(x)\le 1_B(x))
\iff (\forall x\colon 0\le 1_B(x)-1_A(x))\\
\implies 0\le \sum_{x\in B}(1_B(x)-1_A(x))
= \sum_{x\in B} 1_B(x) - \sum_{x\in B} 1_A(x) = |B| - |A|\\
\implies |A|\le |B|.\,\qedsymbol
\end{gather*}
\end{Beweis}

\newpage
\subsection{Endliche Abbildungen}

\begin{Satz}[Anzahl der Abbildungen]\newlinefirst
Seien $X,Y$ endliche Mengen mit $|X| = k$ und $|Y|=n$. Die Menge
der Abbildungen $X\to Y$ enthält $n^k$ Elemente.
\end{Satz}
\begin{Beweis}
Induktion über $k$. Im Anfang $k=0$ ist $X=\emptyset$. Es gibt genau
eine Abbildung $\emptyset\to Y$, nämlich die leere Abbildung.
Gleichermaßen ist $n^0=1$.

Zum Induktionsschritt. Induktionsvoraussetzung sei die Gültigkeit
für $k-1$. Es  sei $|X|=k$ und $|Y|=n$. Gesucht ist die Anzahl
der Möglichkeiten zur Festlegung der Abbildung $f\colon X\to Y$.
Sei $x\in X$ fest. Für die Festlegung $f(x)=y$ bestehen nun genau $n$
Möglichkeiten, nämlich so viele, wie es Elemente $y\in Y$ gibt.
Für die Festlegung der übrigen Werte betrachtet man $f$ als Abbildung%
\[f\colon X\setminus\{x\}\to Y,\]
von denen es laut Voraussetzung $n^{k-1}$ gibt. Wir haben also
$n$ mal $n^{k-1}$ Möglichkeiten, das sind $n^k$.\,\qedsymbol
\end{Beweis}

\begin{Satz}[Anzahl der Bijektionen]\newlinefirst
Seien $X,Y$ endliche Mengen, wobei $|X|=|Y|=n$ gelte.
Die Menge der Bijektionen $X\to Y$ enthält $n!$ Elemente.
\end{Satz}
\begin{Beweis}
Induktion über $n$. Im Anfang $n=0$ ist $X=\emptyset$
und $Y=\emptyset$. Es existiert genau eine Bijektion
$\emptyset\to\emptyset$, nämlich die leere Abbildung.
Bei der Fakultät gilt ebenfalls $0! = 1$ laut
Def. \ref{def:factorial}.

Zum Induktionsschritt. Induktionsvoraussetzung sei die Gültigkeit für
$n-1$. Es sei $|X|=n$. Gesucht ist die Anzahl der Möglichkeiten zur
Festlegung der Bijektion $f\colon X\to Y$. Sei $x\in X$ fest. Für
die Festlegung $f(x)=y$ bestehen genau $n$ Möglichkeiten, nämlich
so viele, wie es Elemente $y\in Y$ gibt. Bei der Festlegung der übrigen
Werte entfällt $y$ aufgrund der Injektivität von $f$. Für die Festlegung
betrachtet man $f$ daher als Bijektion%
\[f\colon X\setminus\{x\}\to Y\setminus\{y\},\]
von denen es laut Voraussetzung $(n-1)!$ gibt. Wir haben also
$n$ mal $(n-1)!$ Möglichkeiten, was gemäß Def. \ref{def:factorial}
gleich $n!$ ist.\,\qedsymbol
\end{Beweis}

\begin{Satz}[Anzahl der Injektionen]\newlinefirst
Seien $X,Y$ endliche Mengen, wobei $|X|=k$ und $|Y|=n$ gelte.
Die Menge der Injektionen $X\to Y$ enthält $n^{\underline k}$
Elemente.
\end{Satz}
\begin{Beweis}
Induktion über $k$. Im Anfang $k=0$ ist $X=\emptyset$. Es gibt
genau eine Injektion $\emptyset\to Y$, nämlich die leere Abbildung.
Gleichermaßen gilt $n^{\underline 0} = 1$.

Zum Schritt. Voraussetzung sei die Gültigkeit
für $k-1$. Es sei $|X|=k$ und $|Y|=n$. Gesucht ist die Anzahl
der Möglichkeiten zur Festlegung der Injektion $f\colon X\to Y$.
Sei $x\in X$ fest. Für die Festlegung $f(x)=y$ bestehen genau
$n$ Möglichkeiten, nämlich so viele, wie es Elemente $y\in Y$ gibt.
Bei der Festlegung der übrigen entfällt $y$ aufgrund der Injektivität
von $f$. Für die Festlegung betrachtet man $f$ daher als Injektion%
\[f\colon X\setminus\{x\}\to Y\setminus\{y\},\]
von denen es laut Voraussetzung $(n-1)^{\underline{k-1}}$ gibt. Es sind
also $n$ mal $(n-1)^{\underline{k-1}}$ Möglichkeiten, was gleich
$n^{\underline k}$ ist.\,\qedsymbol
\end{Beweis}

\newpage
\begin{Satz}\label{bijection-from-k-subsets-to-orbits}
Seien $X,Y$ endliche Mengen und sei $|X|=k$. Sei $C_k(Y)$
die Menge der $k$-elementigen Teilmengen von $Y$. Für zwei
Injektionen $X\to Y$ sei ferner die Äquivalenzrelation
\[f\sim g \defiff \exists\pi\in S_k\colon f=g\circ \pi\]
definiert, wobei mit den $\pi\in S_k$ Permutationen gemeint sind.
Zwischen der Quotientenmenge $\operatorname{Inj}(X, Y)/S_k$
und $C_k(Y)$ besteht eine kanonische Bijektion.
\end{Satz}
\begin{Beweis}
Wir definieren diese Bijektion als
\[\varphi\colon \operatorname{Inj}(X, Y)/S_k\to C_k(Y),
\quad \varphi([f]) := f(X),\]
wobei $[f]=f\circ S_k$ die Äquivalenzklasse des Repräsentanten $f$
bezeichne. Sie ist wohldefiniert, denn für $f\sim g$ gilt
\[f(X) = (g\circ\pi)(X) = g(\pi(X)) = g(X).\]
Für die Injektivität von $\varphi$ ist zu zeigen, dass $f(X) = g(X)$
die Aussage $[f]=[g]$ impliziert, also die Existenz einer Permutation
$\pi$ mit $f=g\circ\pi$. Weil $g$ injektiv ist, existiert eine
Linksinverse $g^{-1}$, so dass wir $\pi:=g^{-1}\circ f$ wählen
können. Es verbleibt somit die Gleichung $f=g\circ g^{-1}\circ f$ zu
zeigen. Zwar ist $g^{-1}$ im Allgemeinen keine Rechtsinverse, ihre
Einschränkung auf $g(X)$ aber schon. Wegen $f(X)=g(X)$ hebt sich
$g\circ g^{-1}$ daher wie gewünscht auf $f(X)$ weg.

Zur Surjektivität von $\varphi$. Hier ist zu zeigen, dass es zu jeder
Menge $B\in C_k(Y)$ eine Injektion $f$ mit $f(X)=B$ gibt. Betrachten
wir sie als Bijektion $f\colon X\to B$. Eine solche besteht,
weil $X$ und $B$ gleichmächtig sind.\,\qedsymbol
\end{Beweis}

\begin{Satz}[Anzahl der Kombinationen]\newlinefirst
Sei $Y$ eine $|Y|=n$ Elemente enthaltende endliche Menge und $C_k(Y)$
die Menge der $k$-elementigen Teilmengen von $Y$.
Es gilt $|C_k(Y)| = \binom{n}{k}$.
\end{Satz}
\begin{Beweis}[Beweis 1]
Sei $X$ eine Menge mit $|X|=k$. Es gilt
\[|C_k(Y)|
\stackrel{\text{(1)}}= |\operatorname{Inj}(X,Y)/S_k|
\stackrel{\text{(2)}}= \frac{|\operatorname{Inj}(X,Y)|}{|S_k|}
= \frac{n^{\underline k}}{k!} = \binom{n}{k}.\]
Gleichung (1) gilt hierbei laut Satz
\ref{bijection-from-k-subsets-to-orbits}.
Die Einsicht von (2) erhält man mit der folgenden Überlegung.
Für jede Gruppe $G$ gilt die Bahnformel $|G| = |f\circ G|\cdot |G_f|$.
Ist nun die Fixgruppe $G_f$ trivial, ist $|G_f|=1$ und
infolge $|f\circ G|=|G|$. Dies ist bei der symmetrischen Gruppe
$G=S_k$ der Fall. Aus diesem Grund enthält jede Bahn $f\circ S_k$
gleich viele Elemente, $|S_k|$ an der Zahl. Weil die Bahnen außerdem
paarweise disjunkt sind, erhält man die Faktorisierung
\[|\operatorname{Inj}(X,Y)| = |S_k|\cdot |\operatorname{Inj}(X,Y)/S_k|.\,\qedsymbol\]
\end{Beweis}
\begin{Beweis}[Beweis 2]
Induktion über $(n, k)$. Im Anfang ist $k=0$ oder $k=n$. Der abstruse
Fall $k=0$ sucht nach Teilmengen ohne Elemente. Es existiert genau eine
solche Menge, nämlich die leere Menge, womit $C_0(Y)=1$ ist. Der Fall
$k=n$ sucht nach Teilmengen, die so viele Elemente haben wie $Y$.
Dies kann nur $Y$ selbst sein, womit $C_n(Y)=1$ gilt.
Gleichermaßen gilt $\binom{n}{0}=1$ und $\binom{n}{n}=1$.

Induktionsvoraussetzung sei die Gültigkeit für $(n-1, k-1)$ und
$(n-1, k)$. Man nimmt nun ein Element $y$ aus $Y$ heraus,
womit darin $n-1$ verbleiben. Entscheidet man sich,
$y$ zur Teilmenge hinzuzufügen, verbleiben noch $k-1$ Elemente
auszuwählen. Entscheidet man sich dagegen, verbleibt die Teilmenge
unverändert, womit nach wie vor $k$ Elemente auszuwählen sind.
Die Anzahl der Möglichkeiten ist somit
\[|C_k(Y)| = |C_{k-1}(Y\setminus\{y\})| + |C_k(Y\setminus\{y\})|
\stackrel{\mathrm{IV}}=\binom{n-1}{k-1} + \binom{n-1}{k}
= \binom{n}{k}.\,\qedsymbol\]
\end{Beweis}

\begin{Satz}[Gitterweg-Interpretation]\newlinefirst
Ein Gitterweg\index{Gitterweg} auf dem Gitter $\Z\times\Z$ heißt
monoton, wenn von $(x,y)$ aus lediglich der Schritt nach $(x+1,y)$
oder der Schritt nach $(x,y+1)$ gewährt ist. Die Anzahl der monotonen
Gitterwege von $(0,0)$ zu $(x,y)$ beträgt
\[\frac{(x+y)!}{x!y!} = \binom{x+y}{x} = \binom{x+y}{y}.\]
\end{Satz}
\begin{Beweis}[Beweis 1]
Alle Gitterwege besitzen dieselbe Länge $x+y$. Die Knoten des jeweiligen
Wegs nummerieren wir der Reihe nach mit Ausnahme des letzten. Nun
ist von den $x+y$ Nummern eine Teilmenge von $y$ Nummern auszuwählen,
an denen ein Schritt nach oben stattfinden soll. Dafür gibt es
$\binom{x+y}{y}$ Möglichkeiten.\,\qedsymbol
\end{Beweis}

\begin{Beweis}[Beweis 2]
Es bezeichne $f(x,y)$ die Anzahl der Wege von $(0,0)$
zu $(x,y)$. Zum Erreichen eines Randpunktes besteht immer nur eine
einzige Möglichkeit, womit $f(x,0)=1$ und $f(0,y)=1$ gilt. Der nicht
auf dem Rand befindliche Punkt $(x,y)$ kann nun von $(x-1,y)$ oder von
$(x,y-1)$ aus erreicht werden, womit
\[f(x,y) = f(x-1,y) + f(x,y-1)\]
gelten muss. Man sieht nun, dass diese Rekursion ein gedrehtes
pascalsches Dreieck erzeugt. Wir setzen daher $f(x,y) = C(x+y,x)$ und
führen die Koordinatentransformation $x+y=n$ und $x=k$ aus. Die
Rekurrenz nimmt damit die Form
\begin{gather*}
C(x+y,x) = C(x-1+y,x-1) + C(x+y-1,x)\\
\iff C(n,k) = C(n-1,k-1) + C(n-1,k).
\end{gather*}
an. Die Randbedingungen führen zu $C(n,n)=1$ und $C(n,0)=1$. Durch diese
Rekurrenz ist eindeutig der Binomialkoeffizient $C(n,k)=\binom{n}{k}$
bestimmt, womit
\[f(x,y) = C(x+y,x) = \binom{x+y}{x}\]
gelten muss.\,\qedsymbol
\end{Beweis}

\begin{Satz}[Rekursionsformel der Potenzmengenabbildung]\newlinefirst
Für $x\notin M$ gilt $\mathcal P(M\cup\{x\}) = \mathcal P(M)\cup\{A\cup\{x\}\mid A\in\mathcal P(M)\}$.
\end{Satz}
\begin{Beweis}
Die Gleichung ist äquivalent zu
\[T\subseteq M\cup\{x\}\iff T\subseteq M\lor\exists A\subseteq M\colon T=A\cup\{x\}.\]
Nehmen wir die rechte Seite an. Im Fall $T\subseteq M$ gilt erst
recht $T\subseteq M\cup\{x\}$. Im anderen Fall liegt ein $A\subseteq M$
vor, womit $A\cup\{x\}\subseteq M\cup\{x\}$ gilt. Wegen $T=A\cup\{x\}$
gilt also ebenfalls $T\subseteq M\cup\{x\}$.

Nehmen wir die linke Seite an. Mit $T\subseteq M\cup\{x\}$ und
$x\notin M$ folgt per Satz \ref{subseteq-diff}
\[T\setminus\{x\}\subseteq (M\cup\{x\})\setminus\{x\} = M,
\;\text{also}\; T\setminus\{x\}\subseteq M.\]
Im Fall $x\notin T$ ist
$T=T\setminus\{x\}$, womit man $T\subseteq M$
erhält. Im Fall $x\in T$ wird $A:=T\setminus\{x\}$
als Zeuge der Existenzaussage gewählt.\,\qedsymbol
\end{Beweis}

\newpage
\section{Endliche Summen}

\subsection{Allgemeine Regeln}

\begin{Definition}[Summe]
Sei $(G,+,0)$ eine kommutative Gruppe und $a_k\in G$. Die Summe ist
rekursiv definiert als
\[\sum_{k=m}^{m-1} a_k := 0,\quad \sum_{k=m}^n a_k
:= a_n + \sum_{k=m}^{n-1} a_k.\]
\end{Definition}

\begin{Satz}\label{sum-add}
Es gilt
\[\sum_{k=m}^n (a_k + b_k) = \sum_{k=m}^n a_k + \sum_{k=m}^n b_k.\]
\end{Satz}
\begin{Beweis} Induktion über $n$. Im Anfang $n=m-1$ haben
beide Seiten der Gleichung den Wert null. Induktionsschritt:
\begin{align*}
\sum_{k=m}^n (a_k+b_k) &= a_n + b_n + \sum_{k=m}^{n-1} (a_k+b_k)
\stackrel{\mathrm{IV}}= a_n + b_n + \sum_{k=m}^{n-1} a_k + \sum_{k=m}^{n-1} b_k\\
&= \sum_{k=m}^n a_k + \sum_{k=m}^n b_k.\,\qedsymbol
\end{align*}
\end{Beweis}

\begin{Satz}\label{sum-scale}
Sei $R$ ein Ring und $c,a_k\in R$. Sei $c$ eine
Konstante. Es gilt
\[\sum_{k=m}^n ca_k = c\sum_{k=m}^n a_k.\]
\end{Satz}
\begin{Beweis} Induktion über $n$. Im Anfang $n=m-1$ haben beide
Seiten der Gleichung den Wert null. Induktionsschritt:
\[\sum_{k=m}^n ca_k = ca_n + \sum_{k=m}^{n-1} ca_k
\stackrel{\mathrm{IV}}= ca_n + c\sum_{k=m}^{n-1} a_k
= c(a_n + \sum_{k=m}^{n-1} a_k) = c\sum_{k=m}^n a_k.\,\qedsymbol\]
\end{Beweis}

\begin{Satz}[Aufteilung einer Summe]\label{sum-split}
Für $m\le p\le n$ gilt
\[\sum_{k=m}^n a_k = \sum_{k=m}^{p-1} a_k + \sum_{k=p}^n a_k.\]
\end{Satz}
\begin{Beweis} Induktion über $n$. Im Induktionsanfang ist $n=p$
und folglich:
\[\sum_{k=m}^p a_k = \sum_{k=m}^{p-1} a_k + p_k
= \sum_{k=m}^{p-1} a_k + \sum_{k=p}^p a_k.\]
Induktionsschritt:
\[\sum_{k=m}^n a_k = a_n + \sum_{k=m}^{n-1} a_k
\stackrel{\mathrm{IV}}= a_n + \sum_{k=m}^{p-1} a_k + \sum_{k=p}^{n-1} a_k
= \sum_{k=m}^{p-1} a_k + \sum_{k=p}^n a_k.\,\qedsymbol\]
\end{Beweis}

\begin{Satz}[Indexshift]\label{sum-indexshift}\newlinefirst
Für die Indexverschiebung der Distanz $d\in\Z$ gilt
\[\textstyle\sum_{k=m}^n a_k = \sum_{k=m+d}^{n+d} a_{k-d}.\]
\end{Satz}
\begin{Beweis}[Beweis 1]
Induktion über $n$. Im Anfang $n = m-1$ haben beide Seiten
der Gleichung den Wert null. Induktionsschritt:
\[\sum_{k=m}^n a_k = a_n+\sum_{k=m}^{n-1}a_k \stackrel{\mathrm{IV}}=
a_{(n+d)-d}+\sum_{k=m+d}^{n+d-1}a_{k-d}
= \sum_{k=m+d}^{n+d}a_{k-d}.\,\qedsymbol\]
\end{Beweis}
\begin{Beweis}[Beweis 2]
Mit der Substitution $k=k'-d$ findet sich die Umformung
\[\sum_{k=m}^n a_k \stackrel{\text{(1)}}= \sum_{m\le k\le n} a_k
\stackrel{\text{(2)}}= \sum_{m\le k'-d\le n} a_{k'-d}
\stackrel{\text{(3)}}= \sum_{m+d\le k'\le n+d} a_{k'-d}
\stackrel{\text{(4)}}= \sum_{k'=m+d}^{n+d} a_{k'-d},\]
wobei (1), (4) gemäß Satz \ref{sum-set-is-range} gelten
und (2), (3) eine andere Schreibweise für die Substitutionsregel
\ref{sum-set-subs} ist.\,\qedsymbol
\end{Beweis}
\strong{Bemerkung.} Der zweite Beweis ist eigentlich zirkulär,
weil der Beweis der Substitutionsregel über den Beweis von
Satz \ref{sum-set-well-defined} in transitiver Abhängigkeit zum
generalisierten Kommutativgesetz \ref{sum-perm-index} steht, dessen
Beweis einen Indexshift enthält.

\begin{Satz} Es gilt
\[\sum_{i=m}^n \sum_{j=m'}^{n'} a_{ij} = \sum_{j=m'}^{n'}\sum_{i=m}^n a_{ij}.\]
\end{Satz}
\begin{Beweis}
Induktion über $n$ und $n'$. Im Anfang bei $n=m-1$ und $n'=m-1$
haben beide Seiten der Gleichung den Wert null. Induktionsschritt für $n$:
\[\sum_{i=m}^n\sum_{j=m'}^{n'} a_{ij}
= \!\!\sum_{j=m'}^{n'} a_{nj}
+ \!\sum_{i=m}^{n-1}\sum_{j=m'}^{n'} a_{ij}
\stackrel{\mathrm{IV}}=
\!\sum_{j=m'}^{n'} a_{nj}
+ \!\sum_{j=m'}^{n'}\sum_{i=m}^{n-1} a_{ij}
= \!\!\sum_{j=m'}^{n'} (a_{nj}+\sum_{i=m}^{n-1} a_{ij})
= \!\!\sum_{j=m'}^{n'} \sum_{i=m}^n a_{ij}.\]
Induktionsschritt für $n'$:
\[\sum_{i=m}^n\sum_{j=m'}^{n'} a_{ij}
= \!\!\sum_{i=m}^n (a_{in'}+\!\!\sum_{j=m'}^{n'-1}a_{ij})
= \!\!\sum_{i=m}^n a_{in'}+\!\!\sum_{i=m}^n\sum_{j=m'}^{n'-1}a_{ij}
\stackrel{\mathrm{IV}}=
\!\sum_{i=m}^n a_{in'}+\sum_{j=m'}^{n'-1}\sum_{i=m}^n a_{ij}
= \!\!\sum_{j=m'}^{n'}\sum_{i=m}^n a_{ij}.\]
Weil immer ein Schritt nach rechts oder ein Schritt nach oben durchführbar ist,
werden alle Punkte $(n,n')$ im Gitter $\Z_{\ge m-1}\times\Z_{\ge m'-1}$ erreicht.\,\qedsymbol
\end{Beweis}

\begin{Satz}[Umkehrung der Reihenfolge]\label{sum-rev}\newlinefirst
Es gilt $\sum_{k=0}^n a_k = \sum_{k=0}^n a_{n-k}$.
\end{Satz}
\begin{Beweis}
Induktion über $n$. Im Anfang $n=-1$ haben beide Seiten der Gleichung
den Wert null. Der Induktionsschritt ist
\begin{align*}
\sum_{k=0}^n a_{n-k} &= a_{n-n} + \sum_{k=0}^{n-1} a_{n-k}
\stackrel{\mathrm{IV}}= a_0+\sum_{k=0}^{n-1} a_{n-(n-1-k)}a_k\\
&= a_0+\sum_{k=0}^{n-1} a_{k-1}
\stackrel{\text{(1)}}= \sum_{k=0}^0 a_k+\sum_{k=1}^n a_k
\stackrel{\text{(2)}}= \sum_{k=0}^n a_k,
\end{align*}
wobei (1) gemäß Indexshift \ref{sum-indexshift} und
(2) gemäß Aufteilung \ref{sum-split} gilt.\,\qedsymbol
\end{Beweis}

\newpage
\begin{Satz}[Generalisiertes Kommutativgesetz]%
\label{sum-perm-index}\newlinefirst
Sei $M=\{k\in\Z\mid m\le k\le n\}$. Für jede Permutation $\pi\colon M\to M$ gilt
\[\sum_{k=m}^n a_k = \sum_{k=m}^n a_{\pi(k)}.\]
\end{Satz}
\begin{Beweis} Induktiv. Sei ohne Beschränkung der Allgemeinheit $m=1$.
Im Induktionsanfang $n=0$ und $n=1$ ist die Gleichung offenkundig erfüllt.

Induktionsschritt. Induktionsvoraussetzung sei die Gültigkeit für $M$.
Zu zeigen ist die Gültigkeit für $M\cup\{n+1\}$.

Sei $t$ ein fester Parameter mit $1\le t\le n+1$.
Im Fall $\pi(t) = n+1$ geht man wie folgt vor.
Man setze $\sigma(k):=\pi(k)$ für $1\le k\le t-1$. Man setze
$\sigma(k):=\pi(k+1)$ für $t\le k\le n$. Weil $n+1$ kein Wert von
$\sigma$ ist, muss $\sigma$ eine Permutation $\sigma\colon M\to M$ sein.
Ergo gilt
\begin{align*}
\sum_{k=1}^{n+1} a_{\pi(k)} &= \sum_{k=1}^{t-1} a_{\pi(k)}
+ a_{\pi(t)} + \sum_{k=t+1}^{n+1} a_{\pi(k)}
= a_{\pi(t)} + \sum_{k=1}^{t-1} a_{\pi(k)}
+ \sum_{k=t}^n a_{\pi(k+1)}\\
&= a_{n+1} + \sum_{k=1}^{t-1} a_{\sigma(k)}
+ \sum_{k=t}^n a_{\sigma(k)}
= a_{n+1} + \sum_{k=1}^n a_{\sigma(k)}\\
&\stackrel{\mathrm{IV}}= a_{n+1} + \sum_{k=1}^n a_k
= \sum_{k=1}^{n+1} a_k.
\end{align*}
Man beachte, dass in den beiden Randfällen $t=1$ und $t=n+1$ die
jeweilige Randsumme den Wert null hat und somit verschwindet.\,\qedsymbol
\end{Beweis}

\begin{Definition}\label{def:sum-set}
Für eine endliche Menge $M$ definiert man
\[\sum_{k\in M} a_k := \sum_{i=m}^n a_{f(i)},\]
wobei $f\colon \{m,\ldots,n\}\to M$ eine frei wählbare Bijektion ist.
\end{Definition}

\begin{Satz}\label{sum-set-well-defined}
Der Wert Summe auf der rechten Seite von Def. \ref{def:sum-set}
ist unabhängig von der gewählten Bijektion.
\end{Satz}
\begin{Beweis} Seien $f,g$ zwei solche Bijektionen. Dann existiert
$\pi$ mit $f=g\circ\pi$, womit%
\[\sum_{i=m}^n a_{f(i)} = \sum_{i=m}^n a_{g(\pi(i))} =
\sum_{i=m}^n a_{g(i)}\]
laut Satz \ref{sum-perm-index} gilt.\,\qedsymbol
\end{Beweis}

\begin{Satz}\label{sum-set-is-range}
Für $M = \{k\in\Z\mid m\le k\le n\}$ gilt
\[\sum_{m\le k\le n} a_k := \sum_{k\in M} a_k = \sum_{k=m}^n a_k.\]
\end{Satz}
\begin{Beweis} Es gilt
$\sum_{k\in M} a_k = \sum_{k=m}^n a_{\id(k)} = \sum_{k=m}^n a_k$.\,\qedsymbol
\end{Beweis}

\begin{Satz}[Substitutionsregel]\label{sum-set-subs}
Ist $\varphi\colon M'\to M$ eine Bijektion, gilt
\[\sum_{k\in M} a_k = \sum_{k'\in M'} a_{\varphi(k')}.\]
\end{Satz}
\begin{Beweis} Zur Bijektion $f\colon\{1,\ldots,|M|\}\to M$ existiert die Bijektion
$g$ mit $f = \varphi\circ g$.

Infolge gilt
\[\sum_{k\in M} a_k = \sum_{i=1}^{|M|} a_{f(i)}
= \sum_{i=1}^{|M|} a_{\varphi(g(i))}
= \sum_{k'\in M'} a_{\varphi(k')}.\,\qedsymbol\]
\end{Beweis}

\begin{Satz} Es gilt
$\sum\limits_{k\in M} ca_k = c\sum\limits_{k\in M} a_k$ und
$\sum\limits_{k\in M} (a_k + b_k)
= \sum\limits_{k\in M} a_k + \sum\limits_{k\in M} b_k$.
\end{Satz}
\begin{Beweis}
Laut Definition gilt
\begin{gather*}
\sum_{k\in M} ca_k = \sum_{i=1}^{|M|} ca_{f(i)}
= c\sum_{i=1}^{|M|} a_{f(i)} = c\sum_{k\in M} a_k,\\
\sum_{k\in M} (a_k+b_k) = \sum_{i=1}^{|M|} (a_{f(i)}+b_{f(i)}) =
\sum_{i=1}^{|M|} a_{f(i)} + \sum_{i=1}^{|M|} b_{f(i)}
= \sum_{k\in M} a_k + \sum_{k\in M} b_k.\,\qedsymbol
\end{gather*}
\end{Beweis}

\begin{Satz} Es gilt
\[\sum_{k\in M}\sum_{l\in N} a_{kl} = \sum_{l\in N}\sum_{k\in M} a_{kl}.\]
\end{Satz}
\begin{Beweis}
Laut Definition gilt
\[\sum_{k\in M}\sum_{l\in N} a_{kl}
= \sum_{i=1}^{|M|}\sum_{j=1}^{|N|} a_{f(i),g(j)}
= \sum_{j=1}^{|N|}\sum_{i=1}^{|M|} a_{f(i),g(j)}
= \sum_{l\in N}\sum_{k\in M} a_{k,l}.\]
\end{Beweis}

\begin{Satz}\label{sum-set-disjoint}
Für $M\cap N=\emptyset$ gilt
\[\sum_{k\in M\cup N} a_k = \sum_{k\in M} a_k + \sum_{k\in N} a_k.\]
\end{Satz}
\begin{Beweis}
Sei $m:=|M|$ und $n:=|N|$. Laut Prämisse existiert eine Bijektion
$f\colon \{1,\ldots, m+n\}\to M\cup N$ mit $f(i)\in M$ für
$1\le i\le m$ und $f(i)\in N$ für $m+1\le i\le m+n$. Das macht
\[\sum_{k\in M\cup N} a_k = \sum_{i=1}^{m+n} a_{f(i)}
= \sum_{i=1}^m a_{f(i)} + \sum_{i=m+1}^{m+n} a_{f(i)}
= \sum_{k\in M} a_k + \sum_{k\in N} a_k.\,\qedsymbol\]
\end{Beweis}

\begin{Satz}\label{sum-partition}
Für eine disjunkte Zerlegung $M = \bigcup_{i\in I} M_i$ gilt
\[\sum_{k\in M} a_k = \sum_{i\in I}\sum_{k\in M_i} a_k.\]
\end{Satz}
\begin{Beweis}
Induktion über $I$. Im Anfang $I=\emptyset$ haben beide Seiten
den Wert null. Induktionsvoraussetzung sei die Gültigkeit
für $I$. Zu zeigen ist die Gültigkeit für $I\cup\{n\}$ mit $n\notin I$.
Der Induktionsschritt ist
\[\sum_{k\in M_n\cup M} a_k
= \sum_{k\in M_n} a_k + \sum_{k\in M} a_k
\stackrel{\mathrm{IV}}= \sum_{k\in M_n} a_k +
\sum_{i\in I}\sum_{k\in M_i} a_k
= \sum_{i\in I\cup\{n\}}\sum_{k\in M_i} a_k.\,\qedsymbol\]
\end{Beweis}

\begin{Satz}
Es gilt
\[\sum_{t\in M\times N} a_t = \sum_{k\in M}\sum_{l\in N} a_{(k,l)}.\]
\end{Satz}
\begin{Beweis}
Es ist $M = \bigcup_{k\in M} \{k\}$ und weiter $M\times N =
\bigcup_{k\in M} (\{k\}\times N)$ eine disjunkte Zerlegung. Hiermit
findet sich die Umformung
\[\sum_{t\in M\times N} a_t
\stackrel{\text{(1)}}= \sum_{k\in M}\;\sum_{t\in \{k\}\times N} a_t
\stackrel{\text{(2)}}= \sum_{k\in M}\sum_{l\in N} a_{(k,l)},\]
wobei (1) laut Satz \ref{sum-partition} gilt und (2) per Substitutionsregel
\ref{sum-set-subs} mit der Bijektion $\varphi\colon N\to\{k\}\times N$
mit $\varphi(l):=(k,l)$ und $t=\varphi(l)$.
\end{Beweis}

\begin{Satz} Mit der Indikatorfunktion $1_A\colon M\to\{0,1\}$
für $A\subseteq M$ gilt
\[\sum_{k\in M} 1_A(k)a_k = \sum_{k\in A} a_k.\]
\end{Satz}
\begin{Beweis}
Mit disjunkter Zerlegung $M=A\cup (M\setminus A)$
und Satz \ref{sum-set-disjoint} gilt
\[\sum_{k\in M} 1_A(k)a_k = \sum_{k\in A} \underbrace{1_A(k)}_{1} a_k
+ \sum_{k\in M\setminus A}\underbrace{1_A(k)}_{0} a_k
= \sum_{k\in A} a_k.\,\qedsymbol\]
\end{Beweis}

\begin{Satz} Allgemein gilt
\[\sum_{k\in A\cup B} a_k = \sum_{k\in A} a_k + \sum_{k\in B} a_k
- \sum_{k\in A\cap B} a_k.\]
\end{Satz}
\begin{Beweis} Sei $G=A\cup B$ die Grundmenge. Gemäß Satz
\ref{indicator-set-op} darf man rechnen
\begin{align*}
\sum_{k\in G} a_k &= \sum_{k\in G} 1_{A\cup B}(k)a_k
= \sum_{k\in G} 1_A(k)a_k + \sum_{k\in G} 1_B(k)a_k
- \sum_{k\in G} 1_{A\cap B}(k)a_k\\
&= \sum_{k\in A} a_k + \sum_{k\in B} a_k - \sum_{k\in A\cap B} a_k.\,\qedsymbol
\end{align*}
\end{Beweis}

\begin{Definition}[Differenzenfolge]\index{Differenzenfolge}
Zu einer Folge $(a_k)$ definiert man
\[(\Delta a)_k := a_{k+1} - a_k.\]
\end{Definition}

\begin{Satz}[Teleskopsumme]\label{sum-tele} Es gilt
\[\sum_{k=m}^{n-1} (\Delta a)_k = \sum_{k=m}^{n-1} (a_{k+1} - a_k) = a_n - a_m.\]
\end{Satz}
\begin{Beweis}[Beweis 1] Induktion über $n$. Im Anfang $n=m$ haben beide Seiten
der Gleichung den Wert null. Induktionsschritt:
\[\sum_{k=m}^n (a_{k+1} - a_k) = (a_{n+1} - a_n) + \!\!\sum_{k=m}^{n-1} (a_{k+1} - a_n)
\stackrel{\mathrm{IV}}= a_{n+1} - a_n + a_n - a_m = a_{n+1} - a_m.\,\qedsymbol\]
\end{Beweis}
\begin{Beweis}[Beweis 2]
Per Indexshift \ref{sum-indexshift}
gilt $\sum\limits_{k=m}^{n-1} a_{k+1} = \!\!\sum\limits_{k=m+1}^n\!\! a_k
= a_n - a_m + \sum\limits_{k=m}^{n-1} a_k$.
Somit ist%
\[\sum_{k=m}^{n-1} (a_{k+1} - a_k) = \sum_{k=m}^{n-1} a_{k+1} - \sum_{k=m}^{n-1} a_k
= a_n - a_m + \sum_{k=m}^{n-1} a_k - \sum_{k=m}^{n-1} a_k = a_n - a_m.\,\qedsymbol\]
\end{Beweis}

\begin{Satz}
Zum Beweis einer Formel
\[\sum_{k=m}^{n-1} a_k = s_n\]
genügt es, $s_m=0$ und $(\Delta s)_n = a_n$ zu zeigen.
\end{Satz}
\begin{Beweis}[Beweis 1]
Induktion über $n$. Im Anfang $n=m$ haben beide Seiten der Gleichung laut
der Prämisse den Wert null. Induktionsschritt:
\[\sum_{k=m}^n a_k = a_n + \sum_{k=m}^{n-1} a_k
\stackrel{\mathrm{IV}}= a_n + s_n
= (\Delta s)_n + s_n = s_{n+1} - s_n + s_n = s_{n+1}.\,\qedsymbol\]
\end{Beweis}
\begin{Beweis}[Beweis 2]
Spezialisierung von Satz \ref{sum-tele}.\,\qedsymbol
\end{Beweis}

\begin{Satz} Der Differenzoperator ist linear. Das heißt,
für alle Folgen $(a_n), (b_n)$ und jede Konstante $c$ gilt
\begin{align*}
& \Delta(a+b) = \Delta a + \Delta b, && ((a+b)_n := a_n + b_n)\\
& \Delta(ca) = c\Delta a. && ((ca)_n := ca_n)
\end{align*}
\end{Satz}
\begin{Beweis} Man findet
\begin{align*}
(\Delta(a+b))_n &= (a+b)_{n+1} - (a+b)_n
= (a_{n+1}+b_{n+1}) - (a_n+b_n)\\
&= a_{n+1}-a_n + b_{n+1}-b_n = (\Delta a)_n + (\Delta b)_n
= (\Delta a + \Delta b)_n
\end{align*}
und
\[(\Delta(ca))_n = (ca)_{n+1} - (ca)_n = ca_{n+1} - ca_n
= c(a_{n+1} - a_n) = c(\Delta a)_n = (c\Delta a)_n.\,\qedsymbol\]
\end{Beweis}

\begin{Definition}[Shiftoperator] Man definiert
\[(Ta)_n := a_{n+1}.\]
\end{Definition}

\begin{Satz}[Iterierter Differenzoperator]\newlinefirst
Für jede Folge $(a_n)$ und $m\in\Z_{\ge 0}$ gilt
\[(\Delta^m a)_n = (-1)^m\sum_{k=0}^m\binom{m}{k} (-1)^k a_{n+k}.\]
\end{Satz}
\begin{Beweis} Es gilt $\Delta = T-\id$. Weil $T$ und $\id$ kommutieren,
ist der binomische Lehrsatz anwendbar. Es ergibt sich
\[\Delta^m = (T-\id)^m = \sum_{k=0}^m\binom{m}{k} (-1)^{m-k} T^k\id^{m-k}
= (-1)^m \sum_{k=0}^m\binom{m}{k} (-1)^k T^k.\,\qedsymbol\]
\end{Beweis}

\begin{Satz}\label{delta-deg}
Sei $f$ eine Polynomfunktion. Dann ist $\Delta_h f$ eine
Polynomfunktion mit niedrigerem Grad.
\end{Satz}
\begin{Beweis} Für $f(x)=\sum_{n=0}^m a_n x^n$ gilt
\begin{gather*}
\Delta_h f(x) = f(x+h) - f(x) = \sum_{n=0}^m a_n (x+h)^n - \sum_{n=0}^m a_n x^n
= \sum_{n=0}^m a_n ((x+h)^n - x^n)\\
= \sum_{n=0}^m a_n (x^n + \sum_{k=0}^{n-1}\binom{n}{k}x^k h^{n-k} - x^n)
= \sum_{n=0}^m a_n \sum_{k=0}^{n-1}\binom{n}{k}x^k h^{n-k}.
\end{gather*}
In der Summe treten nur Monome bis $x^{m-1}$ auf.\,\qedsymbol
\end{Beweis}

\begin{Satz} Sei $f$ ein Polynom vom Grad $N$. Für $n,a\in\Z$ und $n\ge a$ gilt
\[f(n) = \sum_{k=0}^N \frac{(\Delta^k f)(a)}{k!}(n-a)^{\underline k}
= \sum_{k=0}^N \binom{n-a}{k}(\Delta^k f)(a).\]
\end{Satz}
\begin{Beweis}
Es gilt $T=\Delta+\id$. Für jede nichtnegative ganze Zahl $m$ gilt
\[T^m = (\Delta+\id)^m = \sum_{k=0}^m\binom{m}{k}\Delta^k\]
mit dem binomischen Lehrsatz, da $\Delta$ und $\id$ kommutieren. Das macht
\[f(a + m) = \sum_{k=0}^m\binom{m}{k}(\Delta^k f)(a).\]
Man substituiere nun $n = a+m$. Für $n\ge a$ gilt dann
\[f(n) = \sum_{k=0}^{n-a}\binom{n-a}{k}(\Delta^k f)(a)
= \sum_{k=0}^N\binom{n-a}{k}(\Delta^k f)(a).\]
Der Indexbereich der Summierung durfte auf bis $k=N$ geändert werden, weil
$\Delta^k f = 0$ für $k>N$ laut Satz \ref{delta-deg} gilt. Dass nun
Summanden mit $k>n-a$ auftreten können, ist nicht weiter schlimm, weil
in diesem Fall $\binom{n-a}{k}=0$ ist.\,\qedsymbol
\end{Beweis}

\newpage
\subsection{Klassische Partialsummen}

\begin{Satz}[Partialsummen der konstanten Folge]%
\label{sum-const}\newlinefirst
Es gilt $\displaystyle\sum_{k=m}^n 1 = n-m+1$.
\end{Satz}
\begin{Beweis}
Induktion über $n$. Im Anfang $n=m-1$ haben beide Seiten
der Gleichung den Wert null. Induktionsschritt:
\[\sum_{k=m}^n 1 = 1 + \sum_{k=m}^{n-1}
\stackrel{\mathrm{IV}}= 1 + n-1-m+1 = n-m+1.\,\qedsymbol\]
\end{Beweis}

\begin{Satz}[Partialsummen der arithmetischen Folge]%
\index{arithmetische Folge}\newlinefirst
Es gilt $\displaystyle\sum_{k=0}^n k = \frac{n}{2}(n+1)$.
\end{Satz}
\begin{Beweis}[Beweis 1]
Induktion über $n$. Im Anfang $n=-1$ haben beide Seiten der Gleichung
den Wert null. Induktionsschritt:
\[\sum_{k=0}^n k = n + \sum_{k=0}^{n-1} k
\stackrel{\mathrm{IV}}= n + \frac{n-1}{2}(n-1+1)
= \frac{n}{2}(2 + n-1) = \frac{n}{2}(n+1).\,\qedsymbol\]
\end{Beweis}
\begin{Beweis}[Beweis 2]
Klassischer Beweis. Man findet die Umformung
\[2\!\sum_{k=0}^n k = \!\sum_{k=0}^n k + \!\sum_{k=0}^n k
\stackrel{\text{(1)}}= \!\sum_{k=0}^n k + \!\sum_{k=0}^n (n-k)
\stackrel{\text{(2)}}= \!\sum_{k=0}^n (k+n-k)
= \!\sum_{k=0}^n n \stackrel{\text{(3)}}= n\!\sum_{k=0}^n 1
\stackrel{\text{(4)}}= n(n+1),\]
wobei (1), (2), (3), (4) gemäß Satz
\ref{sum-rev}, \ref{sum-add}, \ref{sum-scale}, \ref{sum-const}
gelten.\,\qedsymbol
\end{Beweis}

\begin{Satz}[Partialsummen der geometrischen Folge]%
\label{sum-geom-seq}\index{geometrische Folge}\newlinefirst
Für $m\ge 0$ und $z\in\C\setminus\{1\}$ gilt
$\displaystyle\sum_{k=m}^{n-1} z^k = \frac{z^n-z^m}{z-1}$.
\end{Satz}
\begin{Beweis}
Induktion über $n$. Im Anfang $n=m-1$ haben beiden Seiten der Gleichung
den Wert null. Induktionsschritt:
\[\sum_{k=m}^n z^k = z^n + \sum_{k=m}^{n-1} z^k
\stackrel{\mathrm{IV}}= z^n + \frac{z^n-z^m}{z-1}
= \frac{(z-1)z^n+z^n-z^m}{z-1}
= \frac{z^{n+1}-z^m}{z-1}.\,\qedsymbol\]
\end{Beweis}

\begin{Satz}
Für $m\ge 0$ und $z\in\C\setminus\{1\}$ gilt
\[\sum_{k=m}^{n-1} kz^k
= \frac{(nz^n-mz^m)(z-1) - (z^n-z^m)z}{(z-1)^2}.\]
\end{Satz}
\begin{Beweis}
Die Gleichung von Satz \ref{sum-geom-seq} für $m\ge 1$ auf beiden
Seiten nach $z$ ableiten und anschließend beide Seiten mit $z$
multiplizieren. Den Fall $m=0$ und in diesem den Summand zu $k=0$
explizit betrachten, sonst aber auf dieselbe Weise vorgehen.\,\qedsymbol
\end{Beweis}

\newpage
\begin{Satz} Es gilt
\[\sum_{k=1}^n (-1)^k k = (-1)^n\left\lfloor\frac{n+1}{2}\right\rfloor.\]
\end{Satz}
\begin{Beweis}
Induktion über $n$. Im Anfang $n=0$ haben beiden Seiten den Wert null.

Induktionsschritt:
\[\sum_{k=1}^n (-1)^k k = (-1)^n n + \sum_{k=1}^{n-1} (-1)^k k
\stackrel{\mathrm{IV}}= (-1)^n n + (-1)^{n-1}\left\lfloor\frac{n}{2}\right\rfloor
= (-1)^n (n-\left\lfloor\frac{n}{2}\right\rfloor).\]
Zu zeigen verbleibt die Gleichung
\[n-\left\lfloor\frac{n}{2}\right\rfloor = \left\lfloor\frac{n+1}{2}\right\rfloor
\iff n = \left\lfloor\frac{n}{2}\right\rfloor + 
\left\lfloor\frac{n+1}{2}\right\rfloor.\]
Wir nehmen die Fallunterscheidung zwischen geraden und ungeraden
Zahlen vor, um Satz \ref{floor-add-int} und \ref{floor-is-zero}
nutzen zu können. Im geraden Fall $n=2k$ bestätigt sich
\[\left\lfloor\frac{2k}{2}\right\rfloor +  \left\lfloor\frac{2k+1}{2}\right\rfloor
= \lfloor k\rfloor + \left\lfloor k + \frac{1}{2}\right\rfloor = k + k = 2k.\]
Im ungeraden Fall $n=2k+1$ bestätigt sich
\[\left\lfloor\frac{2k+1}{2}\right\rfloor + \left\lfloor\frac{2k+1+1}{2}\right\rfloor
= \left\lfloor k + \frac{1}{2}\right\rfloor + \left\lfloor k+1\right\rfloor
= k + k + 1 = 2k + 1.\,\qedsymbol\]
\end{Beweis}

\section{Funktionen}

\subsection{Floor und Ceil}

\begin{Definition}[Floorfunktion]%
\label{def:floor}\index{Floorfunktion}
Für $x\in\R$ definiert man
\[y = \lfloor x\rfloor\defiff y\in\Z\land 0\le x-y < 1.\]
\end{Definition}

\begin{Definition}[Ceilfunktion]%
\label{def:ceil}\index{Ceilfunktion}
Für $x\in\R$ definiert man
\[y = \lceil x\rceil\defiff y\in\Z\land 0\le y-x < 1.\]
\end{Definition}

\begin{Satz}\label{floor-add-int}
Für jede ganze Zahl $k$ gilt $\lfloor k + x\rfloor = k + \lfloor x\rfloor$.
\end{Satz}
\begin{Beweis} Aufgrund der Prämisse $k\in\Z$ ist $y\in\Z$ äquivalent
zu $y-k\in\Z$. Unter dieser Gegebenheit findet sich mit
Def. \ref{def:floor} die äquivalente Umformung
\begin{align*}
y = \lfloor k+x\rfloor &\iff y\in\Z\land 0\le (k+x)-y < 1\iff y-k\in\Z\land 0\le x-(y-k) < 1\\
&\iff y - k = \lfloor x\rfloor \iff y = k + \lfloor x\rfloor.\,\qedsymbol
\end{align*}
\end{Beweis}

\begin{Satz}\label{floor-is-zero}
Für $0\le x < 1$ gilt $\lfloor x\rfloor = 0$.
\end{Satz}
\begin{Beweis}
Dies folgt unmittelbar aus Def. \ref{def:floor}.\,\qedsymbol
\end{Beweis}

\newpage
\subsection{Faktorielle}

\begin{Definition}[Fakultät]%
\label{def:factorial}\index{Fakultaet@Fakultät}\newlinefirst
Für eine nichtnegative ganze Zahl $n$ definiert man $n!$ rekursiv durch
\[0! := 1,\quad (n+1)! := (n+1)n!.\]
\end{Definition}

\begin{Definition}[Fallende Faktorielle]\label{def:falling-factorial}%
\index{Faktorielle!fallende}\index{fallende Faktorielle}\newlinefirst
Für $k\in\Z_{\ge 0}$ und $n\in\Z$ (oder allgemeiner $n\in\C$)
definiert man $n^{\underline k}$ rekursiv durch
\[n^{\underline 0} := 1,\quad n^{\underline {k+1}}:=n(n-1)^{\underline k}.\]
\end{Definition}

\begin{Definition}[Steigende Faktorielle]\label{def:raising-factorial}%
\index{Faktorielle!steigende}\index{steigende Faktorielle}\newlinefirst
Für $k\in\Z_{\ge 0}$ und $n\in\Z$ (oder allgemeiner $n\in\C$)
definiert man $n^{\overline k}$ rekursiv durch
\[n^{\overline 0} := 1,\quad n^{\overline {k+1}}:=n(n+1)^{\overline k}.\]
\end{Definition}

\begin{Satz}\label{relation-ff-factorial}
Für $n,k\in\Z_{\ge 0}$ und $k\le n$ gilt
\[n^{\underline k} = \frac{n!}{(n-k)!}.\]
\end{Satz}
\begin{Beweis}
Induktion über $k$. Im Anfang $k=0$ resultieren beide Seiten der
Gleichung im gleichen Wert~1. Der Induktionsschritt ist
\[n^{\underline k} = n(n-1)^{\underline{k-1}}
\stackrel{\text{IV}}= n\frac{(n-1)!}{((n-1)-(k-1))!}
= \frac{n(n-1)!}{(n-k)!}
= \frac{n!}{(n-1)!}.\,\qedsymbol\]
\end{Beweis}

\begin{Satz}
Für ganze Zahlen $n,k$ mit $n\ge 1$ und $k\ge 1-n$ gilt
\[n^{\overline k} = \frac{(n+k-1)!}{(n-1)!}.\]
\end{Satz}
\begin{Beweis}
Induktion über $k$. Im Anfang $k=0$ resultieren beide Seiten der
Gleichung im gleichen Wert~1. Der Induktionsschritt ist
\begin{align*}
n^{\overline k} &= n(n+1)^{\overline{k-1}}
\stackrel{\text{IV}}= n\frac{(n+1+k-1-1)!}{(n+1-1)!}
= \frac{n(n+k-1)!}{n!}\\
&= \frac{n(n+k-1)!}{n(n-1)!}
= \frac{(n+k-1)!}{(n-1)!}.\,\qedsymbol
\end{align*}
\end{Beweis}

\begin{Satz}
Für jedes $n\in\Z_{\ge 0}$ gilt $n!\le n^n$.
\end{Satz}
\begin{Beweis}
Induktion über $n$. Im Induktionsanfang $n=0$ hat man $0! = 1$ und $0^0=1$.
Zum Induktionsschritt unternimmt man zunächst die äquivalente Umformung
\[(n+1)!\le (n+1)^{n+1} \iff (n+1)n!\le (n+1)(n+1)^n
\iff n!\le (n+1)^n.\]
Diese Ungleichung bestätigt die Rechnung
\[(n+1)^n = \sum_{k=0}^n\binom{n}{k}n^k =
n^n+\sum_{k=0}^{n-1}\binom{n}{k}n^k\ge n^n
\stackrel{\mathrm{IV}}\ge n!.\,\qedsymbol\]
\end{Beweis}

\newpage
\subsection{Binomialkoeffizient}

\begin{Definition}[Binomialkoeffizient]%
\label{def:binom}\index{Binomialkoeffizient}\newlinefirst
Für $k\in\Z_{\ge 0}$ und $n\in\Z$ (oder allgemeiner $n\in\C$)
definiert man
\[\binom{n}{k} := \frac{n^{\underline k}}{k!}.\]
\end{Definition}

\begin{Satz}
Für $n\in\Z$ mit $k\le n$ gilt
\[\binom{n}{k} = \frac{n!}{k!(n-k)!}\]
\end{Satz}
\begin{Beweis}
Folgt direkt aus Def. \ref{def:binom} und Satz
\ref{relation-ff-factorial}.\,\qedsymbol
\end{Beweis}

\begin{Satz}
Für $k\ge 1$ und $n\in\Z$ (oder allgemeiner $n\in\C$) gilt
\[\binom{n}{k} = \frac{n}{k}\binom{n-1}{k-1}.\]
\end{Satz}
\begin{Beweis}
Es findet sich die Umformung
\[\binom{n}{k} = \frac{n^{\underline k}}{k!}
= \frac{n(n-1)^{\underline {k-1}}}{k(k-1)!}
= \frac{n}{k}\binom{n-1}{k-1}.\,\qedsymbol\]
\end{Beweis}

\begin{Satz}
Für $k\ge 1$ und $n\in\Z$ (oder allgemeiner $n\in\C$) gilt
\[\binom{n}{k} = \binom{n-1}{k-1} + \binom{n-1}{k}.\]
\end{Satz}
\begin{Beweis}
Es findet sich die Umformung
\begin{align*}
\binom{n-1}{k-1} + \binom{n-1}{k}
&= \frac{(n-1)^{\underline{k-1}}}{(k-1)!} + \frac{(n-1)^{\underline k}}{k!}
= \frac{k(n-1)^{\underline{k-1}}}{k!} + \frac{(n-1)^{\underline{k-1}}(n-k)}{k!}\\
&= \frac{(n-1)^{\underline{k-1}}}{k!}(k + n -k)
= \frac{n(n-1)^{\underline{k-1}}}{k!}
= \frac{n^{\underline k}}{k!} = \binom{n}{k}.\,\qedsymbol
\end{align*}
\end{Beweis}


\printindex

\end{document}


