\documentclass[a4paper,fleqn,11pt]{scrbook}
\usepackage[utf8]{inputenc}
\usepackage[T1]{fontenc}
\usepackage[ngerman]{babel}
\usepackage{amsmath}
\usepackage{amssymb}
\usepackage{amsthm}

\usepackage{mdframed}
\usepackage{lipsum}

\usepackage{libertine}
\usepackage[scaled=0.80]{DejaVuSans}
\usepackage[cmintegrals]{newtxmath}
\renewcommand\ttdefault{lmvtt}

\usepackage{geometry}
\geometry{a4paper,left=34mm,right=34mm,top=30mm,bottom=40mm}

\usepackage{color}
\definecolor{c1}{RGB}{0,40,80}
\definecolor{gray1}{RGB}{80,80,80}
\usepackage[colorlinks=true,linkcolor=c1]{hyperref}
% \usepackage[colorlinks=true,linkcolor=black]{hyperref}

\newcommand{\strong}[1]{\textsf{\textbf{#1}}}

\newtheoremstyle{rmbox}%
  {0pt}% space above
  {0pt}% space below
  {}% bodyfont
  {}% indent
  {\normalfont\sffamily\bfseries}% head font
  {\\[2pt]}% punctuation between head and body
  {0pt}% space after theorem head
  {\thmname{#1}\;\thmnumber{#2}.\;\thmnote{#3.}}

\theoremstyle{rmbox}
\newtheorem{Definition}{Definition}
\newtheorem{Satz}{Satz}
\newtheorem{Lemma}[Satz]{Lemma}
\newtheorem{Korollar}[Satz]{Korollar}

\numberwithin{Definition}{chapter}
\numberwithin{Satz}{chapter}

\definecolor{greenblue}{rgb}{0.0,0.32,0.4}
\definecolor{grayblue}{rgb}{0.2,0.2,0.4}

\surroundwithmdframed[topline=false,rightline=false,bottomline=false,%
  linecolor=greenblue, linewidth=3.5pt, innerleftmargin=6pt,%
  innertopmargin=2pt, innerbottommargin=4pt,%
  innerrightmargin=0pt%
]{Definition}

\newcommand{\framedtheorem}[1]{%
\surroundwithmdframed[topline=false,rightline=false,bottomline=false,%
  linecolor=grayblue, linewidth=3.5pt, innerleftmargin=6pt,%
  innertopmargin=2pt, innerbottommargin=4pt,%
  innerrightmargin=0pt%
]{#1}}

\framedtheorem{Satz}
\framedtheorem{Lemma}
\framedtheorem{Korollar}

\newcommand{\N}{\mathbb N}
\newcommand{\Z}{\mathbb Z}
\newcommand{\R}{\mathbb R}
\newcommand{\C}{\mathbb C}
\newcommand{\ui}{\mathrm i}
\newcommand{\ee}{\mathrm e}
\newcommand{\defiff}{\;:\Longleftrightarrow\;}
\newcommand{\emdef}[1]{\emph{#1}}
\renewcommand{\qedsymbol}{\ensuremath{\Box}}

\DeclareMathOperator{\id}{id}
\DeclareMathOperator{\sur}{sur}
\DeclareMathOperator{\real}{Re}
\DeclareMathOperator{\imag}{Im}
\DeclareMathOperator{\Bild}{Bild}

\title{Band 1: Grundlagen der Mathematik}
\date{Mai 2019}

\begin{document}
\thispagestyle{empty}

\maketitle

Dieses Heft steht unter der Lizenz Creative Commons CC0.

\tableofcontents


\chapter{Grundgesetze der Mathematik}

\section{Ungleichungen}

\subsection{Begriff der Ungleichung}

Man stelle sich zwei Körbe vor, in die Äpfel gelegt werden.
In den rechten Korb werden zwei Äpfel gelegt, in den linken drei.
Dann befinden sich im rechten Korb weniger Äpfel als im linken.
Man sagt, zwei ist kleiner als drei, kurz $2<3$. Man spricht von
einer \emph{Ungleichung}, in Anbetracht dessen, dass die beiden
Körbe nicht die gleiche Anzahl von Äpfeln enthalten.

Der Aussagengehalt einer Ungleichung kann wahr oder falsch sein.
Die Ungleichung $2<3$ ist wahr, die Ungleichungen $3<3$ und
$4<3$ sind falsch.

\begin{Definition}[Ungleichungsrelation]
Die Notation $a<b$ bedeutet »Die Zahl $a$ ist kleiner als
die Zahl $b$«. Die Notation $a\le b$ bedeutet »Die Zahl $a$
ist kleiner als oder gleich der Zahl $b$«. Die Notation
$b>a$ ist eine andere Schreibweise für $a<b$ und bedeutet
»Die Zahl $b$ ist größer als die Zahl $a$«. Die Notation
$b\ge a$ ist eine andere Schreibweise für $a\le b$ und
bedeutet »Die Zahl $b$ ist größer oder gleich der Zahl $a$«.
\end{Definition}

\subsection{Äquivalenzumformungen}

Wir stellen uns wieder einen linken Korb mit zwei Äpfeln und
einen rechten Korb mit drei Äpfeln vor. Legt man nun in beide
Körbe jeweils zusätzlich 10 Äpfel hinein, dann befinden sich
im linken Korb 12 Äpfel und im rechten 13. Der linke Korb
enthält also immer noch weniger Äpfel als im rechten.

Befindet sich eine Balkenwaage im Ungleichgewicht, und legt man
in beide Waagschalen zusätzlich die gleiche Masse von Gewichten,
dann wird sich das Ungleichgewicht der Balkenwaage nicht verändern.

Für die Herausnahme von Äpfeln oder Gewichten ist diese Argumentation
analog. Ist stattdessen eine falsche Ungleichung gegeben,
dann lässt sich durch Addition der selben Zahl auf beiden Seiten
daraus keine wahre Ungleichung gewinnen. Die analoge Argumentation
gilt für die Subtraktion der selben Zahl. Anstelle von ganzen
Äpfeln kann man natürlich auch Apfelhälften hinzufügen, oder
allgmein Apfelbruchteile. Die Argumentation gilt unverändert.

Wir halten fest. 

\pagebreak
\begin{Satz}[Äquivalenzumformungen von Ungleichungen]
Seien $a,b,c$ beliebige Zahlen. Dann sind die folgenden
Äquivalenzen gültig:
\begin{gather}
\label{lt-add} a<b\iff a+c<b+c,\\
\label{lt-sub} a<b\iff a-c<b-c,\\
\label{le-add} a\le b\iff a+c\le b+c,\\
\label{le-sub} a\le b\iff a-c\le b-c.
\end{gather}
\end{Satz}

\noindent
In Worten: Wenn auf beiden Seiten einer Ungleichung die gleiche
Zahl addiert oder subtrahiert wird, dann ändert sich der Aussagengehalt
dieser Ungleichung nicht.

Gibt es noch andere Äquivalenzumformungen?

Im linken Korb seien wieder zwei Äpfel, im rechten drei. Verdoppelt
man nun die Anzahl in beiden Körben, dann sind linken vier Äpfel,
im rechten sechs. Verzehnfacht man die Anzahl, dann sind im linken
20 Äpfel, im rechten 30. Offenbar verändert sich der Aussagengehalt
nicht, wenn die Anzahl auf beiden Seiten der Ungleichung mit
der gleichen natürlichen Zahl $n$ multipliziert wird.

Jedoch muss $n=0$ ausgeschlossen werden. Wenn $a<b$ ist, und man
multipliziert auf beiden Seiten mit null, dann ergibt sich
$0<0$, was falsch ist. Aus der wahren Ungleichung wurde damit eine
falsche gemacht, also kann es sich nicht um eine Äquivalenzumformung
handeln.

Auch bei der Ungleichung $a\le b$ muss $n=0$ ausgeschlossen werden.
Warum muss man das tun? Die Ungleichung $0\le 0$ ist doch auch
wahr?

Nun, wenn der Aussagengehalt von $a\le b$ falsch ist, z.\,B. $4\le 3$,
und man multipliziert auf beiden Seiten mit null, dann ergibt sich
$0\le 0$, also eine wahre Ungleichung. Aus einer falschen wurde damit
eine wahre gemacht. Bei einer Äquivalenzumformung ist dies ebenfalls
verboten.

\begin{Satz}[Äquivalenzumformungen von Ungleichungen]
Seien $a,b$ beliebige Zahlen und sei $n>0$ eine natürliche Zahl.
Dann sind die folgenden Äquivalenzen gültig:
\begin{gather}
\label{lt-mul-nat} a<b\iff na<nb,\\
\label{le-mul-nat} a\le b\iff na\le nb.
\end{gather}
\end{Satz}

\noindent\strong{Beweis.}
Aus der Ungleichung $a<b$ erhält man mittels \eqref{lt-sub} die
äquivalente Ungleichung $0<b-a$, indem auf beiden Seiten $a$
subtrahiert wird. Die Zahl $b-a$ ist also positiv. Durch Multiplikation
mit einer positiven Zahl lässt sich das Vorzeichen einer Zahl
aber nicht umkehren. Demnach ist $0<n(b-a)$ genau dann,
wenn $0<b-a$ war. Ausmultiplizieren liefert nun
$0<nb-na$ und Anwendung von \eqref{lt-add} bringt dann $na<nb$.

In Kürze formuliert:
\begin{equation}
a<b\iff 0<b-a\iff 0<n(b-a)=nb-na \iff na<nb.
\end{equation}
Für $a\le b$ gilt diese Überlegung analog.\;\qedsymbol

\strong{Alternativer Beweis.}
Mittels \eqref{lt-add} ergibt sich zunächst:
\begin{equation}
a<b\iff \left\{
\begin{matrix}
a+a<b+a\\
a+b<b+b
\end{matrix}
\right\}
\iff 2a<a+b<2b.
\end{equation}
Unter nochmaliger Anwendung von \eqref{lt-add} ergibt sich
nun
\begin{equation}
a<b\iff \left\{
\begin{matrix}
2a<a+b \iff 3a<2a+b\\
2a<2b \iff 2a+b<3b
\end{matrix}
\right\} 3a<2a+b<3b
\end{equation}
Dieses Muster lässt sich induktiv alle natürlichen Zahlen hochschieben:
Aus $na<(n-1)a+b<nb$ sollte sich
$(n+1)a<na+b<(n+1)b$ schlussfolgern lassen und umgekehrt.
Das ist richtig, denn Addition von $a$ gemäß \eqref{lt-add} bringt
\begin{equation}
na<(n-1)a+b \iff (n+1)a < na+b
\end{equation}
und Addition von $b$ gemäß \eqref{lt-add} bringt
\begin{equation}
na<nb \iff na+b < (n+1)b.
\end{equation}
Zusammen ergibt sich daraus der behauptete Induktionsschritt. 
Daraus erhält man $a<b\iff na<nb$. Für $a\le b$ sind diese
Überlegungungen analog.\;\qedsymbol

Wir können sogleich einen Schritt weiter gehen.
\begin{Satz}[Äquivalenzumformungen von Ungleichungen]
Seien $a,b$ beliebige Zahlen und sei $r>0$ eine rationale Zahl,
dann gelten die folgenden Äquivalenzen:
\begin{gather}
\label{lt-mul-rat} a<b\iff ra<rb\iff a/r<b/r,\\
\label{lt-mul-rat} a\le b\iff ra\le rb\iff a/r\le b/r.
\end{gather}
\end{Satz}

\noindent\strong{Beweis.}
Eine rationale Zahl $r>0$ lässt sich immer Zerlegen in einen Quotienten
$r=m/n$, wobei $m,n$ positive natürliche Zahlen sind. Gemäß
\eqref{lt-mul-nat} gilt
\begin{equation}
\frac{m}{n}\cdot a<\frac{m}{n}\cdot b
\iff n\cdot\frac{m}{n}\cdot a<n\cdot\frac{m}{n}\cdot b
\iff ma<mb.
\end{equation}
Gemäß \eqref{lt-mul-nat} gilt aber auch
\begin{equation}
a<b\iff ma<mb.
\end{equation}
Die Zusammenfassung beider Äquivalenzen ergibt
\begin{equation}
a<b\iff \frac{m}{n}\cdot a<\frac{m}{n}\cdot b\iff ra<rb.
\end{equation}
Für $a\le b$ ist die Argumentation analog. Da die Division durch
eine rationale Zahl $r$ die Multiplikation mit ihrem Kehrwert $1/r$ ist,
sind auch die Äquivalenzen für die Division gültig.\;\qedsymbol

Da sich eine reelle Zahl beliebig gut durch eine rationale annähern
lässt, müsste auch der folgende Satz gültig sein.

\begin{Satz}[Äquivalenzumformungen von Ungleichungen]
Seien $a,b$ beliebige Zahlen und sei $r>0$ eine reelle Zahl,
dann gelten die folgenden Äquivalenzen:
\begin{gather}
\label{lt-mul-real} a<b\iff ra<rb\iff a/r<b/r,\\
\label{lt-mul-real} a\le b\iff ra\le rb\iff a/r\le b/r.
\end{gather}
\end{Satz}

\noindent
Der Satz wird sich als richtig erweisen, der Beweis kann in
Analysis"=Lehrbüchern nachgeschlagen werden.

\subsection{Lineare Ungleichungen}

Interessant werden Ungleichungen nun, wenn in ihnen einen Variable
vorkommt. Beispielsweise sei die Ungleichung $x+2<4$ gegeben.
Wird in diese Ungleichung für die Variable $x$ eine Zahl eingesetzt,
dann kann wird Ungleichung entweder wahr oder falsch sein.
Für $x:=1$ ergibt sich die wahre Ungleichung $1+2<4$. Für $x:=2$
ergibt sich jedoch die falsche Ungleichung $2+2<4$.

Wir interessieren uns nun natürlich für die Menge aller Lösungen
dieser Ungleichung. Das sind die Zahlen, welche die Ungleichung
erfüllen, wenn sie für $x$ eingesetzt werden. Gesucht ist also
die Lösungsmenge
\begin{equation}
L = \{x\mid x+2<4\},
\end{equation}
d.\,h. die Menge der $x$, welche die Ungleichung $x+2<4$ erfüllen.

Gemäß Äquivalenzumformung \eqref{lt-sub} kommt man aber sofort zu
\begin{equation}
x+2<4 \iff x+2-2<4-2 \iff x<2.
\end{equation}
Demnach kann die Lösungsmenge als $L=\{x\mid x<2\}$ angegeben werden,
denn Äquivalenzumformungen lassen die Lösungsmenge einer Ungleichung
unverändert.

Die Ungleichung $x+2<4$ ist sicherlich von so einfacher Gestalt,
dass man diese auch gedanklich lösen kann, ohne Äquivalenzumformungen
bemühen zu müssen. Bei komplizierteren Ungleichungen kommen wir dabei
aber mehr oder weniger schnell an unsere mentalen Grenzen.

Schon ein wenig schwieriger ist z.\,B.
\begin{align}
& 5x+2<3x+10 && |\;{-2}\\
\iff & 5x<3x+8 && |\;{-3x}\\
\iff & 2x<8 && |\;{:2}\\
\iff & x<4.
\end{align}

\subsection{Monotone Funktionen}

\begin{Definition}[Streng monotone Funktionen]
Eine Funktion $f\colon G\to\R$ heißt streng monoton, wenn die Äquivalenz
\[a<b\iff f(a)<f(b)\]
für beliebige Zahlen $a,b\in G$ erfüllt ist.
\end{Definition}
Streng monotone Abbildung sind von besonderer Bedeutung, weil
sie gemäß ihrer Definition auch Äquivalenzumformungen sind.

Tatsächlich haben wir schon monotone Abbildungen kennengelernt.
Z.\,B. ist \eqref{lt-add} nichts anderes als die strenge Monotonie
für $f(x):=x+c$. Und \eqref{lt-mul-nat} ist die strenge Monotonie
für $f(x):=nx$.

Die Funktion $f\colon\R\to\R$ mit $f(x):=x^2$ ist nicht streng Monoton.
Zum Beispiel ist $-4<-2$, aber $16=f(-4)>f(-2)=4$. Schränkt man $f$
auf den Definitionsbereich $\R_{>0}$ ein, so ergibt sich jedoch eine
streng Monotone Funktion. Das lässt sich wie folgt zeigen.

Nach Voraussetzung sind $a,b\in\R_{>0}$, d.\,h. $a,b>0$.
Also kann gemäß \eqref{lt-mul-real} einerseits mit $a$
und andererseits mit $b$ multipliziert werden:
\begin{equation}
a<b\iff\begin{Bmatrix}
a^2<ab\\
ab<b^2
\end{Bmatrix}
\iff a^2<ab<b^2.
\end{equation}







\end{document}


