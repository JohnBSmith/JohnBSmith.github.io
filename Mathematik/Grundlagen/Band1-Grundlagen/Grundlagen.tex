\documentclass[a4paper,fleqn,12pt]{scrbook}
\usepackage[utf8]{inputenc}
\usepackage[T1]{fontenc}
\usepackage[ngerman]{babel}
\usepackage{microtype}
\usepackage{amsmath}
\usepackage{amssymb}
\usepackage{amsthm}
\usepackage{graphicx}

\usepackage{mdframed}
\usepackage{lipsum}

\usepackage{libertine}
\usepackage[scaled=0.80]{DejaVuSans}
\usepackage[cmintegrals]{newtxmath}
\renewcommand\ttdefault{lmvtt}

\usepackage{geometry}
\geometry{a4paper,left=30mm,right=30mm,top=30mm,bottom=40mm}

\usepackage{color}
\definecolor{c1}{RGB}{0,40,80}
\definecolor{gray1}{RGB}{80,80,80}
\usepackage[colorlinks=true,linkcolor=c1]{hyperref}
% \usepackage[colorlinks=true,linkcolor=black]{hyperref}

\newcommand{\strong}[1]{\textsf{\textbf{#1}}}

\newtheoremstyle{rmbox}%
  {0pt}% space above
  {0pt}% space below
  {}% bodyfont
  {}% indent
  {\normalfont\sffamily\bfseries}% head font
  {\\[2pt]}% punctuation between head and body
  {0pt}% space after theorem head
  {\thmname{#1}\;\thmnumber{#2}.\;\thmnote{#3.}}

\theoremstyle{rmbox}
\newtheorem{Definition}{Definition}
\newtheorem{Satz}{Satz}
\newtheorem{Lemma}[Satz]{Lemma}
\newtheorem{Korollar}[Satz]{Korollar}

\numberwithin{Definition}{chapter}
\numberwithin{Satz}{chapter}

\definecolor{greenblue}{rgb}{0.0,0.42,0.3}
\definecolor{grayblue}{rgb}{0.1,0.2,0.4}
\definecolor{bgreen}{rgb}{0.94,0.94,0.84}
\definecolor{bblue}{rgb}{0.9,0.92,0.94}

\surroundwithmdframed[topline=false,rightline=false,bottomline=false,%
  linecolor=greenblue, linewidth=3.5pt, innerleftmargin=6pt,%
  innertopmargin=4pt, innerbottommargin=4pt,%
  innerrightmargin=6pt, backgroundcolor=bgreen%
]{Definition}

\newcommand{\framedtheorem}[1]{%
\surroundwithmdframed[topline=false,rightline=false,bottomline=false,%
  linecolor=grayblue, linewidth=3.5pt, innerleftmargin=6pt,%
  innertopmargin=4pt, innerbottommargin=4pt,%
  innerrightmargin=6pt, backgroundcolor=bblue%
]{#1}}

\framedtheorem{Satz}
\framedtheorem{Lemma}
\framedtheorem{Korollar}

\newcommand{\N}{\mathbb N}
\newcommand{\Z}{\mathbb Z}
\newcommand{\Q}{\mathbb Q}
\newcommand{\R}{\mathbb R}
\newcommand{\C}{\mathbb C}
\newcommand{\ui}{\mathrm i}
\newcommand{\ee}{\mathrm e}
\newcommand{\defiff}{\;:\Longleftrightarrow\;}
\newcommand{\emdef}[1]{\emph{#1}}
\renewcommand{\qedsymbol}{\ensuremath{\Box}}
\newcommand{\doubleslash}{/\!/}

\DeclareMathOperator{\id}{id}
\DeclareMathOperator{\sur}{sur}
\DeclareMathOperator{\real}{Re}
\DeclareMathOperator{\imag}{Im}
\DeclareMathOperator{\Bild}{Bild}
\DeclareMathOperator{\Abb}{Abb}

\usepackage{makeidx}
\makeindex

\title{Grundlagen der Mathematik}
\date{Juli 2019}

\begin{document}
\thispagestyle{empty}

\maketitle

Dieses Buch steht unter der Lizenz Creative Commons CC0.

\tableofcontents


\chapter{Grundbegriffe der Mathematik}

\section{Aussagenlogik}

\subsection{Aussagenlogische Formeln}

Aussagen in der Aussagenlogik sind entweder wahr oder falsch,
etwas dazwischen gibt es nicht, das nennt man auch das \emph{Prinzip
der Zweiwertigkeit}\index{Prinzip der Zweiwertigkeit}.
Wir schreiben $0=\text{falsch}$ und $1=\text{wahr}$, das ist schön
kurz und knapp.

Für die Aussage »$n$ ist ohne Rest durch $m$ teilbar« bzw.
»$m$ teilt $n$«, schreibt man kurz $m|n$. Aus Aussagen lassen sich
in der Aussagenlogik zusammengesetzte Aussagen bilden, z.\,B.
\[\text{Aus $2|n$ und $3|n$ folgt, dass $6|n$},\]
als Formel:
\[2|n\land 3|n \implies 6|n.\]
Streng genommen handelt es sich hierbei um eine Aussageform, da die
Aussage von einer Variable abhängig ist. Nachdem für $n$ eine Zahl
eingesetzt wurde, ergibt sich daraus eine Aussage, in diesem Fall
immer eine wahre Aussage.

Eine zusammengesetzte Aussage wird auch \emph{aussagenlogische Formel}
genannt. Aussagenlogische Formeln haben eine innere Struktur. Um diese
untersuchen zu können, werden logische Variablen betrachtet,
das sind solche Variablen, die für eine Aussage stehen. Eine
logische Variable wird durch einen lateinischen Großbuchstaben
am Anfang des Alphabetes beschrieben und kann nur mit den
Wahrheitswerten falsch oder wahr belegt werden. Die genannte Formel
besitzt die Struktur
\[A\land B \implies C.\]
In der Formel treten Verknüpfungen von Aussagen auf, das sind
$\land$ und $\Rightarrow$. Es gibt die grundlegenden Verknüpfungen
$\neg,\land,\lor,\Rightarrow,\Leftrightarrow$. Die Bindungsstärke
der gelisteten Verknüpfungen ist absteigend, so wie Punktrechnung
vor Strichrechnung gilt. Das $\neg$ bindet stärker als $\land$,
bindet stärker als $\lor$, bindet stärker als $\Rightarrow$,
bindet stärker als $\Leftrightarrow$. Die Verknüpfungen sind
in Tabelle \ref{tab:logische-Verknuepfungen} definiert.
Anstelle von $\neg A$ schreibt man auch $\overline A$.

\begin{table}
\centering
\begin{tabular}{cc}
\begin{tabular}{c|c}
$A$ & $\neg A$\\
\hline
$0$ & $1$\\
$1$ & $0$
\end{tabular}
&\qquad \begin{tabular}{c|c|c|c|c|c}
$A$ & $B$ & $A\land B$ & $A\lor B$ & $A\Rightarrow B$ & $A\Leftrightarrow B$\\
\hline
$0$ & $0$ & $0$ & $0$ & $1$ & $1$\\
$1$ & $0$ & $0$ & $1$ & $0$ & $0$\\
$0$ & $1$ & $0$ & $1$ & $1$ & $0$\\
$1$ & $1$ & $1$ & $1$ & $1$ & $1$
\end{tabular}
\end{tabular}
\caption{Definition der grundlegenden logischen Verknüpfungen.}
\label{tab:logische-Verknuepfungen}
\end{table}

\begin{table}
\centering
\begin{tabular}{c|c|c|c|c}
$A$ & $B$ & $A\land B$ & $B\land A$ & $A{\land}B\Rightarrow B{\land}A$\\
\hline
$0$ & $0$ & $0$ & $0$ & $1$\\
$1$ & $0$ & $0$ & $0$ & $1$\\
$0$ & $1$ & $0$ & $0$ & $1$\\
$1$ & $1$ & $1$ & $1$ & $1$
\end{tabular}
\caption{Wahrheitstafel zu $A\land B\Rightarrow B\land A$.}
\label{tab:Wahrheitstafel1}
\end{table}

Es gibt Formeln, die immer wahr sind, unabhängig davon, mit
welchen Wahrheitswerten die Variablen belegt werden.
\begin{Definition}[Tautologie]\index{Tautologie}\mbox{}\\*
Ist $\varphi$ eine Formel, die bezüglich jeder möglichen
Variablenbelegung erfüllt ist, dann nennt man $\varphi$ eine Tautologie
und schreibt dafür kurz $\models\varphi$.
\end{Definition}
Z.\,B. gilt
\[\models A\land B\implies B\land A.\]
Es lässt sich leicht überprüfen, ob eine Formel tautologisch ist.
Dazu wird einfach die Wahrheitstafel zu dieser Formel aufgestellt,
hier Tabelle $\ref{tab:Wahrheitstafel1}$.
Die Wahrheitstafel\index{Wahrheitstafel} ist eine Wertetabelle,
die zu jeder Variablenbelegung den Wahrheitswert der Formel angibt.
Bei einer tautologischen Formel enthält die Ergebnisspalte in jeder
Zeile den Wert $1$.

Zwei wichtige Metaregeln, die Einsetzungsregel und die
Ersetzungsregel, ermöglichen das Rechnen mit aussagenlogischen
Formeln. Die Einsetzungsregel ermöglicht es, aus schon bekannten
Tautologien neue bilden zu können, ohne jedes mal eine Wahrheitstafel
aufstellen zu müssen. Die Ersetzungsregel ermöglicht die Umformung
von Formeln.

\begin{Satz}[Einsetzungsregel]\index{Einsetzungsregel}\mbox{}\\*
Sei $v$ eine logische Variable. Ist $\varphi$ eine tautologische
Formel, dann ergibt sich wieder eine tautologische Formel, wenn man
jedes Vorkommen von $v$ in $\varphi$ durch eine Formel $\psi$ ersetzt.
Kurz:
\[(\models \varphi )\implies (\models \varphi [v:=\psi]).\]
Das gilt auch für die simultane Substitution:
\[(\models \varphi )\implies
(\models \varphi [v_1:=\psi_1,\ldots ,v_n:=\psi_n]).\]
\end{Satz}
\strong{Begründung.} Die Variable $v$ kann in $\varphi$
frei mit einem Wahrheitswert belegt werden, nach Voraussetzung
ist $\varphi$ dabei immer erfüllt. Somit ist $\varphi$ auch
erfüllt, wenn $v$ mit dem Wahrheitswert von $\psi$ belegt wird.
Dann muss aber auch $\varphi[v:=\psi]$ unter einer beliebigen Belegung
wahr sein.\;\qedsymbol

\newpage
\begin{Satz}[Ersetzungsregel]\index{Ersetzungsregel}%
\label{Ersetzungsregel}\mbox{}\\*
Sei $F(\varphi)$ eine Formel, welche von der Teilformel $\varphi$
abhängig ist. Sei außerdem $\varphi$ äquivalent zu $\psi$.
Dann sind auch $F(\varphi)$ und $F(\psi)$ äquivalent. Kurz:
\[(\models\varphi\Leftrightarrow\psi)
\implies (\models F(\varphi)\Leftrightarrow F(\psi)).\]
\end{Satz}
\strong{Begründung.}
Die Äquivalenz von $\varphi$ und $\psi$ erzwingt, dass $\psi$
unter einer beliebigen Belegung den gleichen Wahrheitswert besitzt
wie $\varphi$. Da $F(0)\Leftrightarrow F(0)$ und
$F(1)\Leftrightarrow F(1)$ gilt, muss also
$F(\varphi)\Leftrightarrow F(\psi)$ gelten.\;\qedsymbol

\begin{Satz}[Kleine Metaregel]\mbox{}\\*
Es gilt $\models\varphi$ und $\models\psi$
genau dann, wenn $\models\varphi\land\psi$.
\end{Satz}
\strong{Beweis.}
Sind $\varphi,\psi$ tautologisch, dann dürfen sie durch
den Wahrheitswert wahr ersetzt werden. Unter dieser Voraussetzung
ist $\varphi\land\psi$ gleichbedeutend mit $1\land 1$, demnach
auch tautologisch.

Sei nun umgekehrt $\varphi\land\psi$ tautologisch. Es müssen
zwingend auch $\varphi$ und $\psi$ wahr sein, denn sonst wäre
$\varphi\land\psi$ falsch.\;\qedsymbol

\begin{Satz}[Kleine Abtrennungsregel]\mbox{}\\*
Aus $\models\varphi$ und $\models\varphi\Rightarrow\psi$
folgt $\models\psi$.\\
Aus $\models\varphi$ und $\models\varphi\Leftrightarrow\psi$
folgt $\models\psi$.
\end{Satz}
\strong{Beweis.}
Ist $\varphi$ tautologisch, dann darf es durch den Wahrheitswert
wahr ersetzt werden. Unter dieser Voraussetzung ist
$\varphi\Rightarrow\psi$ gleichbedeutend mit $1\Rightarrow\psi$.
Diese Formel kann nur erfüllt sein, wenn auch $\psi$ wahr ist.
Da aber $\varphi\Rightarrow\psi$ tautologisch sein soll,
muss damit zwingend auch $\psi$ tautologisch sein.
Für $\varphi\Leftrightarrow\psi$ ist die Argumentation
analog.\;\qedsymbol

\begin{Satz}[Abtrennung von Implikationen]\mbox{}\\*
Aus $\models\varphi\Leftrightarrow\psi$
folgt $\models\varphi\Rightarrow\psi$.
\end{Satz}
\strong{Beweis.}
Man zeigt
\[\models (A\Leftrightarrow B)
\Leftrightarrow (A\Rightarrow B)\land (B\Rightarrow A)\]
mittels Wahrheitstafel. Gemäß der Einsetzungsregel gilt dann auch
\[\models (\varphi\Leftrightarrow\psi)
\Leftrightarrow (\varphi\Rightarrow\psi)\land (\psi\Rightarrow\varphi).\]
Mit der kleinen Abtrennungsregel und der Voraussetzung erhält man
\[\models (\varphi\Rightarrow\psi)\land (\psi\Rightarrow\varphi).\]
Gemäß der kleinen Metaregel ergibt sich schließlich
$\models \varphi\Rightarrow\psi$.\;\qedsymbol

\newpage
\begin{Definition}[Äquivalente Formeln]\mbox{}\\*
Zwei Formeln $\varphi,\psi$ heißen äquivalent, wenn die
Äquivalenz $\varphi\Leftrightarrow\psi$ tautologisch ist, kurz
\[(\varphi\equiv\psi)\defiff(\models\varphi\Leftrightarrow\psi).\]
\end{Definition}

\begin{Satz}
Die Relation $\varphi\equiv\psi$ ist eine Äquivalenzrelation, d.\,h.
es gilt
\begin{align}
& \varphi\equiv\varphi, && (\text{Reflexivität})\\
& (\varphi\equiv\psi)\implies (\psi\equiv\varphi), && (\text{Symmetrie})\\
& (\varphi\equiv\psi)\land (\psi\equiv\chi)\implies (\varphi\equiv\chi). && (\text{Transitivität})
\end{align}
\end{Satz}

\subsection{Boolesche Algebra}%
\index{boolesche Algebra}

\begin{table}
\begin{center}
\begin{tabular}{c|c|l}
\strong{UND}&
\strong{ODER}&
\strong{Gesetze}\\

$A\land B\equiv B\land A$ &
$A\lor B\equiv B\lor A$ &
Kommutativgesetze\\

$A\land (B\land C)\equiv (A\land B)\land C$ &
$A\lor(B\lor C)\equiv (A\lor B)\lor C$ &
Assoziativgesetze\\

$A\land A\equiv A$ &
$A\lor A\equiv A$ &
Idempotenzgesetze\\

$A\land 1\equiv A$ &
$A\lor 0\equiv A$ &
Neutralitätsgesetze\\

$A\land 0\equiv 0$ &
$A\lor 1\equiv 1$ &
Extremalgesetze\\

$A\land\overline A\equiv 0$ &
$A\lor\overline A\equiv 1$ &
Komplementärgesetze\\

$\overline{A\land B}\equiv\overline A\lor\overline B$ &
$\overline{A\lor B}\equiv\overline A\land\overline B$ &
De Morgansche Gesetze\\

$A\land (A\lor B)\equiv A$ &
$A\lor (A\land B)\equiv A$ &
Absorptionsgesetze

\end{tabular}
\caption{Die Regeln der booleschen Algebra.}
\label{tab:boolesche-Algebra}
\end{center}
\end{table}

Die Regeln in Tabelle \ref{tab:boolesche-Algebra} gewinnt man
alle mittels Wahrheitstafel. Gemäß der Einsetzungsregel dürfen für
die Variablen auch Formeln eingesetzt werden, die griechischen
Formelvariablen benötigt man somit nicht mehr.

Weiterhin gelten die Distributivgesetze
\begin{align}
A\land(B\lor C) &\equiv (A\land B)\lor (A\land C),\\
A\lor(B\land C) &\equiv (A\lor B)\land (A\lor C).
\end{align}
Schließlich gibt es noch das Involutionsgesetz
\begin{equation}
\overline {\overline A}\equiv A.
\end{equation}
Die Implikation und die Äquivalenz lassen sich auf NICHT, UND, ODER
zurückführen:%
\begin{align}
A\Rightarrow B &\equiv \overline A\lor B,\\
A\Leftrightarrow B &\equiv (A\Rightarrow B)\land (B\Rightarrow A).
\end{align}
Mit den bisher genannten Regeln lassen sich aussagenlogische Formeln
auf einfache Art umformen. Z.\,B. ist die Formel $1\Rightarrow A$
äquivalent zu $A$. Man findet
\[1\Rightarrow A\enspace\equiv\enspace\overline 1\lor A\enspace\equiv\enspace 0\lor A\enspace\equiv\enspace A.\]
Natürlich kann man alternativ mittels Wahrheitstafel auch
\[\models(1\Rightarrow A)\Leftrightarrow A\]
überprüfen.

\begin{Satz}[Formel zum Modus ponens]%
\label{Formel-Modus-ponens}\mbox{}\\*
Es gilt $\models A\land (A\Rightarrow B)\Rightarrow B$.
\end{Satz}
\strong{Beweis.} Gemäß den Regeln der booleschen Algebra ergibt sich
\begin{align}
& A\land (A\Rightarrow B)\Rightarrow B\\
& \equiv A\land (\overline A\lor B)\Rightarrow B && \text{(Zerlegung von »$\Rightarrow$«)}\\
& \equiv \overline{A\land (\overline A\lor B)}\lor B && \text{(Zerlegung von »$\Rightarrow$«)}\\
& \equiv \overline A\lor\overline{\overline A\lor B}\lor B && \text{(De Morgan)}\\
& \equiv \overline A\lor (\overline{\overline A}\land\overline B)\lor B && \text{(De Morgan)}\\
& \equiv \overline A\lor (A\land\overline B)\lor B && \text{(Involutionsgesetz)}\\
& \equiv ((\overline A\lor A)\land(\overline A\lor\overline B))\lor B && \text{(Distributivgesetz)}\\
& \equiv (1\land(\overline A\lor\overline B))\lor B && \text{(Komplementärgesetz)}\\
& \equiv (\overline A\lor\overline B)\lor B && \text{(Absorptionsgesetz)}\\
& \equiv \overline A\lor(\overline B\lor B) && \text{(Assoziativgesetz)}\\
& \equiv \overline A\lor 1 && \text{(Komplementärgesetz)}\\
& \equiv 1.\;\qedsymbol && \text{(Absorptionsgesetz)}
\end{align}
Von der Ersetzungsregel (Satz \ref{Ersetzungsregel}), also
\[\varphi\equiv\psi\enspace
\text{impliziert}\enspace F(\varphi)\equiv F(\psi),\]
wurde ständig stillschweigend Gebrauch gemacht, nämlich bei jeder
Umformung einer Teilformel.

\begin{Satz}[Regel zur Kontraposition]\index{Kontraposition}\mbox{}\\*
Es gilt $A\Rightarrow B\equiv \overline B\Rightarrow\overline A$.
\end{Satz}
\strong{Beweis.} Man findet
\begin{align}
&A\Rightarrow B\\
&\equiv \overline A\lor B && (\text{Zerlegung von »$\Rightarrow$«})\\
&\equiv B\lor\overline A && (\text{Kommutativgesetz})\\
&\equiv \overline{\overline B}\lor\overline A && (\text{Involutionsgesetz})\\
&\equiv \overline B\Rightarrow\overline A.\;\qedsymbol
  && (\text{Zerlegung von »$\Rightarrow$«)}
\end{align}

\newpage
\subsection{Formale Beweise}
\begin{Definition}[Semantische Implikation]\mbox{}\\*
Sei $M=\{\varphi_1,\varphi_2,\ldots,\varphi_n\}$ eine Menge von
Formeln und sei $\psi$ eine weitere Formel. Man sagt dann, $M$
impliziert $\psi$, kurz $M\models\psi$, wenn jede Belegung von
logischen Variablen, die alle Formeln in $M$ erfüllt, auch $\psi$
erfüllt.
\end{Definition}
Das klingt etwas kompliziert, ist es aber eigentlich nicht. Man schaut
sich die große Wahrheitstafel an, in der alle Formeln vorkommen,
jede Formel in einer neuen Spalte.
Ergibt sich in einer Zeile bei allen Formeln in $M$ eine 1, dann muss
auch $\psi$ in dieser Zeile den Wahrheitswert 1 besitzen.

Die Aussage $\models\varphi$ ist mit $\{\}\models\varphi$
gleichbedeutend, denn bei einer leeren Formelmenge werden keine
Belegungen ausgeschlossen, $\varphi$ muss also jede Belegung
erfüllen. Der Definition nach ist $\varphi$ dann eine Tautologie.

Man beobachtet außerdem, dass $\{\varphi\}\models\psi$ mit
$\models\varphi\Rightarrow\psi$ übereinstimmt. Hat nämlich
$\varphi$ den Wahrheitswert 0, dann ist $\varphi\Rightarrow\psi$
immer erfüllt, ohne dass der Wahrheitswert von $\psi$ dabei eine
Rolle spielt. Solche Belegungen entfallen auch bei
$\{\varphi\}\models\psi$. Nun darf $\varphi$ als wahr vorausgesetzt
werden. Wäre $\psi$ nun falsch, dann ist $\varphi\Rightarrow\psi$
nicht mehr erfüllt, also auch $\models\varphi\Rightarrow\psi$ falsch.
In diesem Fall ist aber auch $\{\varphi\}\models\psi$ falsch.
Es verbleibt nun die Situation, dass sowohl $\varphi$ also auch
$\psi$ wahr sind. Mit diesen Belegungen bleibt dann auch
$\{\varphi\}\models\psi$ unverletzt.

\begin{Satz}[Deduktionstheorem]\index{Deduktionstheorem}\mbox{}\\*
Es gilt $M\cup\{\varphi\}\models\psi$ genau dann, wenn
$M\models\varphi\Rightarrow\psi$.
\end{Satz}
\strong{Beweis.} Man hat
\begin{equation}
M\cup\varphi = \{\varphi_1,\ldots,\varphi_n,\varphi\}.
\end{equation}
Dass alle diese Formeln unter einer Belegung erfüllt sein sollen,
ist aber gleichbedeutend damit, dass die Aussage%
\begin{equation}
\varphi_1\land\ldots\land\varphi_n\land\varphi
\end{equation}
unter dieser Belegung erfüllt ist. Wie bereits erläutert, gilt%
\begin{equation}
(\{\varphi_1\land\ldots\land\varphi_n\land\varphi\}\models\psi)
\iff (\models \varphi_1\land\ldots\land\varphi_n\land\varphi\Rightarrow\psi).
\end{equation}
Mittels boolescher Algebra findet man nun
\begin{align}
& \varphi_1\land\ldots\land\varphi_n\land\varphi\Rightarrow\psi\\
& \equiv \overline{\varphi_1\land\ldots\land\varphi_n\land\varphi}\lor\psi\\
& \equiv \overline{\varphi_1\land\ldots\land\varphi_n}\lor\overline\varphi\lor\psi\\
& \equiv \overline{\varphi_1\land\ldots\land\varphi_n}\lor(\varphi\Rightarrow\psi)\\
& \equiv \varphi_1\land\ldots\land\varphi_n\Rightarrow (\varphi\Rightarrow\psi)
\end{align}
Schließlich gilt aber auch wieder
\begin{equation}
(\models \varphi_1\land\ldots\land\varphi_n\Rightarrow(\varphi\Rightarrow\psi))
\iff (\{\varphi_1\land\ldots\land\varphi_n\}\models\varphi\Rightarrow\psi).\;\qedsymbol
\end{equation}

\newpage
\begin{Definition}[Schlussregel]\index{Schlussregel}\mbox{}\\*
Sei $M$ eine Menge von Formeln, die Formelvariablen enthalten
und $\psi$ eine Formelvariable. Ist die Aussage $M\models\psi$ wahr,
unabhängig davon, welche
Formeln für die Formelvariablen eingesetzt werden, dann spricht man
von einer Schlussregel.
\end{Definition}

\begin{Satz}[Modus ponens]\index{Modus ponens}\mbox{}\\*
Es gilt die Schlussregel $\{\varphi,\varphi\Rightarrow\psi\}\models\psi$.
\end{Satz}
\strong{Beweis.} Gemäß Deduktionstheorem gilt
\[(\{\varphi,\varphi\Rightarrow\psi\}\models\psi)
\iff (\models \varphi\land(\varphi\Rightarrow\psi)\Rightarrow\psi).\]
Gemäß Satz \ref{Formel-Modus-ponens} ist die rechte
Seite wahr.\;\qedsymbol

Schlussregeln ermöglichen es uns, aus wahren Aussagen weitere
wahre Aussagen zu gewinnen. Die Belegung mit logischen Variablen
tritt nun in den Hintergrund, besonders dann, wenn die Formeln keine
logischen Variablen mehr enthalten. Sind $A$ und $A\Rightarrow B$
wahre Aussagen, dann muss gemäß Modus ponens auch $B$ eine wahre
Aussage sein.

Ein Beispiel dazu. Sei $A(n):=(2|n)$ die Aussage »$2$ teilt $n$«
und $B(n):=(4|n^2)$ die Aussage »$4$ teilt $n^2$«.
Nun gilt $A(n)\Rightarrow B(n)$ für jede beliebige ganze
Zahl, welche für $n$ eingesetzt wird. Gemäß Modus ponens ist der Schluss%
\[\{A(n), A(n)\Rightarrow B(n)\}\models B(n)\]
richtig. Ausgehend von »$2$ teilt $10$« können wir
damit »$4$ teilt $100$« schlussfolgern.

\begin{Definition}[Beweis]\index{Beweis}\mbox{}\\*
Eine Aussage ist sicher dann wahr, wenn sie mittels Schlussregeln
aus schon bekannten wahren Aussagen gefolgert werden kann. Die Kette
von Schlüssen heißt Beweis dieser Aussage.
\end{Definition}
Ein Beispiel dazu. Angenommen wir wissen, dass die Aussage $A$ wahr
ist. Außerdem ist bekannt, dass $A\Rightarrow B$ und $B\Rightarrow C$
wahr sind. Gemäß Modus ponens ist dann auch $B$ wahr. Nochmalige
Anwendung des Modus ponens liefert die Wahrheit von $C$.

Der formale Beweis von $C$ schaut so aus:
\[
\begin{tabular}{ll}
1.\;$A$, & (Prämisse)\\
2.\;$A\Rightarrow B$, & (Prämisse)\\
3.\;$B\Rightarrow C$, & (Prämisse)\\
4.\;$B$, & (MP, 1, 2)\\
5.\;$C$ & (MP, 4, 3)
\end{tabular}
\]
In Klammern steht immer die Begründung für die jeweilige Aussage.
Der Modus ponens wurde mit MP abgekürzt.

\newpage
\subsection{Notwendige und hinreichende Bedingungen}

Manchmal sagt man, eine Bedingung ist für eine bestimmte Aussage
notwendig. Das ist ein schon bekannter logischer Zusammenhang.
Sei $B$ die Bedingung und $A$ die Aussage. Ist $B$ falsch, dann
kann $A$ niemals wahr sein. Ist $B$ wahr, dann ist $A$ beliebig,
denn nur weil die notwendige Bedingung $B$ zutrifft, heißt das nicht,
dass die Aussage $A$ zwingend wahr sein muss. Dieser Zusammenhang
wird nun gerade genau durch die Wahrheitstafel von $A\Rightarrow B$
wiedergegeben. Man erhält%
\[(\text{$B$ ist notwendig für $A$})\;\equiv\; (A\Rightarrow B).\]
Die Sprechweise »$B$ ist hinreichend für $A$« drückt dagegen aus,
dass die Wahrheit von $A$ mit der Wahrheit von $B$ sichergestellt
ist. Falls $B$ jedoch falsch ist, ist der Wahrheitsgehalt von $A$
beliebig. Dieser Zusammenhang wird gerade durch die Wahrheitstafel
von $B\Rightarrow A$ wiedergegeben. Man erhält%
\[(\text{$B$ ist hinreichend für $A$})\;\equiv\; (B\Rightarrow A).\]
Um sich pedantischer ausdrücken, sprechen manche von notwendigen,
aber nicht hinreichenden Bedingungen, bzw. von hinreichenden,
aber nicht notwendigen Bedingungen.

Gemäß
$(A\Leftrightarrow B)\equiv (A\Rightarrow B)\land (B\Rightarrow A)$
ergibt sich
\[(\text{$B$ ist notwendig und hinreichend für $A$})\;\equiv\;(B\Leftrightarrow A).\]

\noindent
Weitere Sprechweisen für $A\Rightarrow B$ sind »$A$ impliziert $B$«
und »$A$ zieht $B$ nach sich« sowie »aus $A$ folgt $B$«.

Ist $B$ hinreichend für $A$, dann kann man sich bei $A$ sicher sein,
sofern die Bedingung $B$ überprüft wurde.

Ist $B$ nur notwendig für $A$, dann ist durch eine Überprüfung von $B$
nicht viel Wissen über $A$ gewonnen, man darf sich nicht sicher sein,
dass $A$ wahr ist. Lediglich falls $B$ falsch
ist, lässt sich mittels Kontraposition%
\[A\Rightarrow B\;\equiv\;\overline B\Rightarrow\overline A\]
ableiten, dass dann auch $A$ falsch sein muss.

Man gewinnt den folgenden Zusammenhang:
\[(\text{$B$ ist notwendig für $A$})\;\equiv\;
(\text{$\overline B$ ist hinreichend für $\overline A$}).\]
Sind für eine Aussage $A$ mehrere Bedingungen $B_k$ notwendig,
dann heißt das, $A$ ist schon falsch, wenn nur eine der $B_k$
falsch ist. Die Formel dazu ist
\[A\Rightarrow B_1\land B_2\land\ldots\land B_n.\]
Sind für eine Aussage $A$ mehrere Bedingungen $B_k$ hinreichend,
dann heißt das, $A$ ist schon dann richtig, wenn nur eine der $B_k$
richtig ist. Die Formel dazu ist
\[B_1\lor B_2\lor\ldots\lor B_n\Rightarrow A.\]

\newpage
\subsection{Widerspruchsbeweise}
Mittels boolescher Algebra oder einer Wahrheitstafel überzeugt man
sich leicht von
\[\models (A\Rightarrow B)\land (A\Rightarrow\overline B)\Rightarrow\overline A.\]
Unter Heranziehung des Deduktionstheorems ist das äquivalent zu
\[\{A\Rightarrow B,\;A\Rightarrow\overline B\}\models\overline A.\]
Angenommen, man konnte die Aussagen $B$ und $\overline B$
unter Annahme von $A$ beweisen, dann gilt $\{A\}\models B$
und $\{A\}\models\overline B$. Gemäß Deduktionstheorem
bedeutet das jedoch $\models A\Rightarrow B$ und
$\models A\models\overline B$. Da diese Bedingungen tautologisch
sind, können sie entfallen, übrig bleibt $\models\overline A$.

Wir gelangen zur folgenden Schlussregel.

\begin{Satz}[Reductio ad absurdum]\mbox{}\\*
Kann man unter Annahme einer Prämisse $\varphi$ sowohl
$\psi$ als auch $\overline\psi$ beweisen, dann muss
die Negation von $\varphi$ tautologisch sein:
\[\text{$\{\varphi\}\models\psi$
und $\{\varphi\}\models\overline\psi$
impliziert $\models\overline\varphi$}.\]
\end{Satz}

\noindent
Diese Schlussregel lässt sich noch ein wenig verallgemeinern.
Man überzeugt sich mittels boolescher Algebra oder Wahrheitstafel von
\[\models (K\land A\Rightarrow B)\land (K\land A\Rightarrow\overline B)
\Rightarrow (K\Rightarrow\overline A).\]
Nochmals wird das Deduktionstheorem angewendet:
\[\{K\land A\Rightarrow B,\;K\land A\Rightarrow\overline B\}
\models K\Rightarrow\overline A.\]
Für $K$ lässt sich eine konjunktive Aussage
$\varphi_1\land\ldots\land\varphi_n$ einsetzen.
Definiert man $M:=\{\varphi_1,\ldots,\varphi_n\}$, dann gilt
\[(M\cup\{A\}\models\psi)\iff (\models K\land A\Rightarrow\psi)\]
gemäß Deduktionstheorem. Die restliche Überlegung gestaltet
sich wie zuvor. Insgesamt erhält man das folgende Ergebnis.
\begin{Satz}[Reductio ad absurdum]\mbox{}\\*
Sei $M$ eine endliche Formelmenge. Es gilt:
\[\text{$M\cup\{\varphi\}\models\psi$
und $M\cup\{\varphi\}\models\overline\psi$
impliziert $M\models\overline\varphi$}.\]
\end{Satz}

\noindent
Aus der Reductio ad absurdum lässt sich nun ein
Beweisverfahren erstellen. Man setzt $\varphi\equiv\overline A$ ein
und beachtet $A\equiv\neg\neg A$. Aus $\overline A\models\psi$
und $\overline A\models\overline\psi$ lässt sich wie gezeigt
$\models A$ schlussfolgern. Nimmt man also $\overline A$ an,
und zeigt damit den Widerspruch, dass sowohl $\psi$ als auch
$\overline\psi$, dann hat man einen Beweis für $A$.

\newpage
\section{Prädikatenlogik}\index{Prädikatenlogik}
\subsection{Endliche Bereiche}
In diesem Abschnitt wird der Übergang von der Aussagenlogik in
die Prädikatenlogik beschrieben. Eine Prädikat $P$ ist eine Aussageform,
die einem Objekt $x$ einen Wahrheitswert $P(x)$ zuordnet. Z.\,B.
ist $P(x)\equiv (x<4)$ ein Prädikat. Je nachdem was für eine Zahl
für $x$ eingesetzt wird, ergibt sich entweder wahr oder falsch.

\begin{Definition}[Allquantor]\index{Allquantor}\mbox{}\\*
Der Allquantor für endliche Objektbereiche
ist rekursiv definiert gemäß%
\[\bigwedge_{k=1}^0 P(x_k) :\equiv 1,\qquad
\bigwedge_{k=1}^n P(x_k) :\equiv P(x_n)\land\bigwedge_{k=1}^{n-1} P(x_k).\]
\end{Definition}

\begin{Definition}[Existenzquantor]\index{Existenzquantor}\mbox{}\\*
Der Existenzquantor für endliche Objektbereiche
ist rekursiv definiert gemäß%
\[\bigvee_{k=1}^0 P(x_k) :\equiv 0,\qquad
\bigvee_{k=1}^n P(x_k) :\equiv P(x_n)\lor\bigvee_{k=1}^{n-1} P(x_k).\]
\end{Definition}

\noindent
Das allquantifizierte Prädikat ist nur dann wahr, wenn $P(x_k)$ für
jedes $x_k$ erfüllt ist. Man bekommt die aussagenlogische Formel%
\[\bigwedge_{k=1}^n P(x_k)
\enspace\equiv\enspace P(x_1)\land P(x_2)\land\ldots\land P(x_n).\]
Meistens benutzen wir die Schreibweisen
\[(\forall x\in M)P(x) \equiv \bigwedge_{k=1}^n P(x_k),\qquad
(\exists x\in M)P(x) \equiv \bigvee_{k=1}^n P(x_k),\]
wobei $M=\{x_1,x_2,\ldots,x_n\}$ die Zusammenfassung
der Objekte ist. In allen diesen Schreibweisen haben die
Quantoren die gleiche Operatorrangfolge wie die Negation. Z.\,B. wird
die Formel
\[(\forall x\in M)P(x)\land A\]
gelesen als
\[((\forall x\in M)P(x))\land A,\]
Davon zu unterscheiden ist die Formel
\[(\forall x\in M)(P(x)\land A).\]


\begin{Satz}[Distributivgesetze]\mbox{}\\*
Ist $M$ endlich, $A$ eine Aussage und $P(x)$ ein Prädikat auf $M$,
dann gilt
\begin{align*}
A\lor(\forall x\in M)P(x) &\equiv (\forall x\in M)(A\lor P(x)),\\
A\land(\exists x\in M)P(x) &\equiv (\exists x\in M)(A\land P(x)).
\end{align*}
\end{Satz}
\strong{Beweis.} Induktiv mittels boolescher Algebra. Induktionsanfang:
\begin{gather*}
A\lor\bigwedge_{k=1}^0 P(x_k) \equiv A\lor 1 \equiv 1 \equiv \bigwedge_{k=1}^0 (A\lor P(x_k)).
\end{gather*}
Induktionsschritt:
\begin{gather*}
A\lor\bigwedge_{k=1}^n P(x_k) \equiv
A\lor (P(x_n)\land\bigwedge_{k=1}^{n-1} P(x_k))
\equiv (A\lor P(x_n))\land (A\lor\bigwedge_{k=1}^{n-1} P(x_k))\\
\equiv (A\lor P(x_n))\land \bigwedge_{k=1}^{n-1} (A\lor P(x_k))
\equiv \bigwedge_{k=1}^n (A\lor P(x_k)).
\end{gather*}
Für den Existenzquantor ist die Argumentation analog.\;\qedsymbol

\begin{Satz}[De Morgansche Gesetze]\mbox{}\\*
Ist $M$ endlich und $P(x)$ ein Prädikat auf $M$, dann gilt
\begin{align*}
\neg(\forall x\in M)P(x) &\equiv (\exists x\in M)\;\neg P(x),\\
\neg(\exists x\in M)P(x) &\equiv (\forall x\in M)\;\neg P(x).
\end{align*}
\end{Satz}
\strong{Beweis.} Induktionsanfang:
\begin{gather*}
\neg\bigwedge_{k=1}^0 P(x_k) \equiv \neg 1 \equiv 0 \equiv\bigvee_{k=1}^0 \neg P(x_k).
\end{gather*}
Induktionsschritt:
\begin{gather*}
\neg\bigwedge_{k=1}^n P(x_k)
\equiv \neg(P(x_n)\land\bigwedge_{k=1}^{n-1} P(x_k))
\equiv \neg P(x_n)\lor\neg\bigwedge_{k=1}^{n-1} P(x_k)\\
\equiv \neg P(x_n)\lor\bigvee_{k=1}^{n-1} \neg P(x_k)
\equiv \bigvee_{k=1}^n \neg P(x_k).
\end{gather*}
Für den Existenzquantor ist die Argumentation analog.\;\qedsymbol

\newpage
\begin{Satz}[Verträglichkeitsgesetze]%
\label{finite-quantifier-compatibility}\mbox{}\\*
Ist $M$ endlich und sind $P(x),Q(x)$ Prädikate auf $M$, dann gilt
\begin{align*}
(\forall x\in M)(P(x)\land Q(x)) \equiv (\forall x\in M)P(x)\land(\forall x\in M)Q(x),\\
(\exists x\in M)(P(x)\lor Q(x)) \equiv (\exists x\in M)P(x)\lor(\exists x\in M)Q(x).
\end{align*}
\end{Satz}
\strong{Beweis.} Induktionsanfang:
\begin{gather*}
\bigwedge_{k=1}^0 (P(x_k)\land Q(x_k)) \equiv 1 \equiv 1\land 1
\equiv \bigwedge_{k=1}^0 P(x_k)\land\bigwedge_{k=1}^0 Q(x_k).
\end{gather*}
Induktionsschritt:
\begin{gather*}
\bigwedge_{k=1}^n (P(x_k)\land Q(x_k))
\equiv (P(x_n)\land Q(x_n))\land\bigwedge_{k=1}^{n-1} (P(x_k)\land Q(x_k))\\
\equiv P(x_n)\land Q(x_n)\land\bigwedge_{k=1}^{n-1} P(x_k)\land\bigwedge_{k=1}^{n-1} Q(x_k)\\
\equiv P(x_n)\land\bigwedge_{k=1}^{n-1} P(x_k)\land Q(x_n)\land\bigwedge_{k=1}^{n-1} Q(x_k)
\equiv \bigwedge_{k=1}^n P(x_k)\land\bigwedge_{k=1}^n Q(x_k).
\end{gather*}
Die Argumentation für den Existenzquantor ist analog.\;\qedsymbol

% \newpage
\begin{Satz}[Vertauschbarkeit gleichartiger Quantoren]\mbox{}\\*
Sind $M,N$ endlich, dann gilt
\begin{align*}
(\forall x\in M)(\forall x\in N)P(x,y)&\equiv (\forall x\in N)(\forall x\in M)P(x,y),\\
(\exists x\in M)(\exists x\in N)P(x,y)&\equiv (\exists x\in N)(\exists x\in M)P(x,y).
\end{align*}
\end{Satz}
\strong{Beweis.}
Induktionsanfang:
\[\bigwedge_{i=1}^0\bigwedge_{j=1}^n P(x_i,y_j)
\equiv 1 \equiv \bigwedge_{j=1}^n 1 \equiv
\bigwedge_{j=1}^n\bigwedge_{i=1}^0 P(x_i,y_j).\]
Induktionsschritt:
\begin{gather*}
\bigwedge_{i=1}^m\bigwedge_{j=1}^n P(x_i,y_j)
\equiv \bigwedge_{j=1}^n P(x_m,y_j)\land\bigwedge_{i=1}^{m-1}\bigwedge_{j=1}^n P(x_i,y_j)\\
\equiv \bigwedge_{j=1}^n P(x_m,y_j)\land\bigwedge_{j=1}^n \bigwedge_{i=1}^{m-1} P(x_i,y_j)\\
\stackrel{(*)}\equiv \bigwedge_{j=1}^n \bigg(P(x_m,y_j)\land\bigwedge_{i=1}^{m-1} P(x_i,y_j)\bigg)
\equiv \bigwedge_{j=1}^n \bigwedge_{i=1}^m P(x_i,y_j).
\end{gather*}
Die Äquivalenz $(*)$ gilt gemäß Satz \ref{finite-quantifier-compatibility}.

Für den Existenzquantor ist die Argumentation analog.\;\qedsymbol

\subsection{Allgemeine Regeln}

Man denkt sich nun ein Universum $U$, das alle denkbaren Objekte
enthält. Das Prädikat $P(x)$ sei für jedes $x\in U$ definiert.
Anstelle von $(\forall x\in U)P(x)$ schreibt man kürzer
$(\forall x)P(x)$. Anstelle von $(\exists x\in U)P(x)$ schreibt
man kürzer $(\exists x)P(x)$. Das Universum darf unendlich sein,
aber nicht leer, es muss immer mindestens ein Element enthalten.
\begin{Definition}[Allquantor]\mbox{}\\*
Es gilt $(\forall x)P(x)$ genau dann, wenn $P(x)$ für jedes
beliebige $x$ wahr ist.
\end{Definition}
\begin{Definition}[Existenzquantor]\mbox{}\\*
Es gilt $(\exists x)P(x)$ genau dann, wenn ein $x$ gefunden
werden kann, das $P(x)$ erfüllt.
\end{Definition}
Das Problem das sich jetzt stellt, ist, dass zur Überprüfung
von prädikatenlogischen Formeln unendlich viele Wahrheitstafeln
aufgestellt werden müssten, nämlich für jedes der unendlich vielen
Objekte, welche für eine Objektvariable eingesetzt werden können, und das
auch noch für jedes Prädikat, welches in eine Prädikatvariable
eingesetzt werden kann. Wir müssen also anders vorgehen.

Zunächst überzeugt man sich davon, dass die Einsetzungsregel und die
Ersetzungsregel gültig bleiben. Außerdem definiert man für zwei
prädikatenlogische Formeln $\varphi,\psi$ die Äquivalenz als
\[(\varphi\equiv\psi)\defiff (\{\varphi\}\models\psi)\land (\{\psi\}\models\varphi).\]
Bei der semantischen Implikation werden nun nicht nur Aussagenvariablen mit
Wahrheitswerten belegt. Auch Prädikatvariablen werden mit Prädikaten
belegt. Da es unendlich viele Prädikate gibt, lässt sich das natürlich
praktisch nicht mehr durchführen.
\begin{Satz}
Es gilt $A\equiv (\forall x)A$ und $A\equiv(\exists x)A$.
\end{Satz}
\strong{Beweis.} Im Fall $A\equiv 0$ ist auch
$(\forall x)0$ falsch, da $0$ für kein $x$ erfüllt ist.
Im Fall $A\equiv 1$ ist auch $(\forall x)1$ wahr, da $1$ für jedes
beliebige $x$ erfüllt ist. Für den Existenzquantor ist die
Argumentation analog.\;\qedsymbol

Vorsicht, das Universum darf nicht leer sein,
denn $(\forall x{\in}\{\})\,0\equiv 1$.

\begin{Satz}[Verallgemeinerte Distributivgesetze]\mbox{}\\*
Es gilt
\begin{align*}
A\lor (\forall x)P(x) &\equiv (\forall x)(A\lor P(x)),\\
A\land(\exists x)P(x) &\equiv (\exists x)(A\land P(x)).
\end{align*}
\end{Satz}
\strong{Beweis.} Im Fall $A\equiv 0$ ergibt sich
\[A\lor(\forall x)P(x) \equiv 0\lor(\forall x)P(x)
\equiv (\forall x)P(x)\equiv (\forall x)(0\lor P(x))\equiv (\forall x)(A\lor P(x)).\]
Im Fall $A\equiv 1$ ergibt sich
\[A\lor(\forall x)P(x)\equiv 1\lor(\forall x)P(x) \equiv 1 \equiv (\forall x)1
\equiv (\forall x)(1\lor P(x))\equiv (\forall x)(A\lor P(x)).\]
Für den Existenzquantor ist die Argumentation analog.\;\qedsymbol

\begin{Satz}[Verallgemeinerte De Morgansche Gesetze]\mbox{}\\*
Es gilt
\begin{align*}
\neg (\forall x)P(x) &\equiv (\exists x)\;\neg P(x),\\
\neg (\exists x)P(x) &\equiv (\forall x)\;\neg P(x).
\end{align*}
\end{Satz}
\strong{Beweis.} Gilt $(\forall x)P(x)$, dann ist $P(x)\equiv 1$.
Es ergibt sich
\begin{equation}
\neg(\forall x)P(x) \equiv \neg 1 \equiv 0 \equiv (\exists x)0
\equiv (\exists x)\;\neg 1\equiv (\exists x)\;\neg P(x).
\end{equation}
Gilt $(\forall x)P(x)$ nicht, dann muss es ein $x$ mit
$\neg P(x)\equiv 1$ geben und es gilt
\begin{equation}
\neg(\forall x)P(x) \equiv \neg 0\equiv 1\equiv (\exists x)1
\equiv (\exists x)\;\neg P(x).
\end{equation}
Die Argumentation für den Existenzquantor ist analog.\;\qedsymbol

\begin{Satz}[Verträglichkeitsgesetze]\mbox{}\\*
Es gilt
\begin{align*}
(\forall x)(P(x)\land Q(x)) &\equiv (\forall x)P(x)\land (\forall x)Q(x),\\
(\exists x)(P(x)\lor Q(x)) &\equiv (\exists x)P(x)\lor (\exists x)Q(x).
\end{align*}
\end{Satz}
\strong{Beweis.} Angenommen, die linke Seite ist wahr. Dann muss
$P(x)\land Q(x)\equiv 1$ sein, und daher auch $P(x)\equiv 1$ und
$Q(x)\equiv 1$. Dann ist aber auch
$(\forall x)P(x)\equiv 1$ und $(\forall x)Q(x)\equiv 1$. Somit gilt
\begin{equation}
(\forall x)(P(x)\land Q(x)) \equiv 1 \equiv
1\land 1 \equiv (\forall x)P(x)\land (\forall x)Q(x).
\end{equation}
Angenommen, die linke Seite ist falsch. Dann gibt es
ein $x$, für welches $P(x)\land Q(x)\equiv 0$ ist. Für dieses $x$
muss also $P(x)\equiv 0$ oder $Q(x)\equiv 0$ sein, oder beides.
Dann ist auch $(\forall x)P(x)\equiv 0$ oder $(\forall x)Q(x)\equiv 0$.
Somit ist
\begin{equation}
(\forall x)P(x)\land(\forall x)Q(x)\equiv 0.
\end{equation}
Für den Existenzquantor ist die Argumentation analog. Alternativ
ergibt sich nach den De Morganschen und verallgemeinerten De Morganschen
Gesetzen
\begin{gather}
(\exists x)(P(x)\lor Q(x))
\equiv \neg(\forall x)\;\neg (P(x)\lor Q(x))\\
\equiv \neg(\forall x)(\neg P(x)\land \neg Q(x))
\equiv \neg((\forall x)\;\neg P(x)\land (\forall x)\;\neg Q(x))\\
\equiv \neg(\forall x)\;\neg P(x)\lor\neg(\forall x)\;\neg Q(x)
\equiv (\exists x) P(x)\lor(\exists x) Q(x).\;\qedsymbol
\end{gather}


\newpage
\subsection{Beschränkte Quantifizierung}
\begin{Definition}[Beschränkte Quantifizierung]\mbox{}\\*
Ist $P$ ein Prädikat auf $U$ und $M\subseteq U$ eine Teilmenge von
$U$, dann definiert man
\begin{align*}
(\forall x\in M)P(x) &:\equiv (\forall x)(x\in M\Rightarrow P(x)),\\
(\exists x\in M)P(x) &:\equiv (\exists x)(x\in M\land P(x)).
\end{align*}
\end{Definition}
Zuweilen schreibt man auch
\begin{align}
(\forall R(x))P(x) &:\equiv (\forall x)(R(x)\Rightarrow P(x)),\\
(\exists R(x))P(x) &:\equiv (\exists x)(R(x)\land P(x)),
\end{align}
solange klar bleibt, dass $x$ die gebundene Variable ist.
Z.\,B. $(\forall x{<}4)P(x)$ und ähnlich.

\begin{Satz}[Verallgemeinerte Distributivgesetze]%
\label{logical-dist-general}\mbox{}\\*
Es gilt
\begin{align*}
A\lor (\forall x\in M)P(x) &\equiv (\forall x\in M)(A\lor P(x)),\\
A\land(\exists x\in M)P(x) &\equiv (\exists x\in M)(A\land P(x)).
\end{align*}
\end{Satz}
\strong{Beweis.} Für den Allquantor gilt
\begin{gather}
A\lor (\forall x\in M)P(x)
\equiv A\lor(\forall x)(x\in M\Rightarrow P(x))\\
\equiv A\lor(\forall x)(\neg x\in M\lor P(x))
\equiv (\forall x)(A\lor\neg x\in M\lor P(x))\\
\equiv (\forall x)(x\in M\Rightarrow A\lor P(x))
\equiv (\forall x\in M)(A\lor P(x)).
\end{gather}
Für den Existenzquantor gilt
\begin{gather}
A\land (\exists x\in M)P(x)
\equiv A\land(\exists x)(x\in M\land P(x))\\
\equiv (\exists x)(A\land x\in M\land P(x))
\equiv (\exists x)(x\in M\land A\land P(x))\\
\equiv (\exists x\in M)(A\land P(x)).\;\qedsymbol
\end{gather}

\begin{Satz}[Verallgemeinerte De Morgansche Gesetze]\mbox{}\\*
Es gilt
\begin{align*}
\neg (\forall x\in M)P(x) &\equiv (\exists x\in M)\;\neg P(x),\\
\neg (\exists x\in M)P(x) &\equiv (\forall x\in M)\;\neg P(x).
\end{align*}
\end{Satz}
\strong{Beweis.} Es gilt
\begin{gather}
\neg(\forall x\in M)P(x) \equiv \neg(\forall x)(x\in M\Rightarrow P(x))
\equiv \neg(\forall x)(\neg x\in M\lor P(x))\\
\equiv (\exists x)(x\in M\land \neg P(x))
\equiv (\exists x\in M)\;\neg P(x).
\end{gather}
Die Argumentation für den Existenzquantor ist analog.\;\qedsymbol

\newpage
\section{Mengenlehre}

\subsection{Der Mengenbegriff}\index{Menge}

Eine Menge ist im Wesentlichen ein Beutel, der unterschiedliche
Objekte enthält. Es gibt die leere Menge, das ist der leere Beutel.
Das besondere an einer Menge ist nun, dass das selbe Objekt immer
nur ein einziges mal im Beutel enthalten ist. Legt man zweimal
das selbe Objekt in den Beutel, dann ist dieses darin trotzdem nur
einmal zu finden.

Man kann sich dabei z.\,B. einen Einkaufsbeutel vorstellen,
in welchem sich nur ein Apfel, eine Birne, eine Weintraube usw.
befinden darf. Möchte man mehrere Birnen im Einkaufsbeutel haben,
dann müssen diese unterschieden werden, z.\,B. indem jede Birne
eine unterschiedliche Nummer bekommt.

Möchte man eine Menge aufschreiben, werden die Objekte einfach
in einer beliebigen Reihenfolge aufgelistet und diese Liste in
geschweifte Klammern gesetzt. Z.\,B.:%
\[\{\mathrm{Afpel}, \mathrm{Birne}, \mathrm{Weintraube}\}.\]
Nennen wir den Apfel $A$, die Birne $B$
und die Weintraube $W$. Eine Menge mit zwei Äpfeln und drei
Birnen würde man so schreiben:%
\[\{A_1, A_2, B_1, B_2, B_3\}.\]
Erlaubt sind auch Beutel in Beuteln. Eine Menge mit zwei Äpfeln
und einer Menge mit vier Weintrauben wird beschrieben durch%
\[\{A_1, A_2, \{W_1,W_2,W_3,W_4\}\}.\]
Die Reihenfolge spielt wie gesagt keine Rolle:%
\[\{A_1,A_2\} = \{A_2,A_1\}.\]
Ein leerer Beutel ist etwas anderes als ein Beutel, welcher einen
leeren Beutel enthält:%
\[\{\} \ne \{\{\}\}.\]
Die Notation $x\in M$ bedeutet, dass $x$ in der Menge $M$ enthalten
ist. Man sagt, $x$ ist ein Element von $M$. Z.\,B. ist
\[A_1\in\{A_1,A_2\}.\]

\newpage
\subsection{Teilmengen}

\begin{Definition}[Teilmengenrelation]\index{Teilmenge}\mbox{}\\*
Hat man zwei Mengen $M,N$, dann nennt man $M$ eine Teilmenge von $N$,
wenn jedes Element von $M$ auch ein Element von $N$ ist.
Als Formel:
\[M\subseteq N\defiff \text{für jedes $x\in M$ gilt $x\in N$}.\]
Anders formuliert, aber gleichbedeutend:
\[M\subseteq N\defiff \text{für jedes $x$ gilt:}\; (x\in M\implies x\in N).\]
\end{Definition}
Z.\,B. ist die Aussage $\{1,2\}\subseteq\{1,2,3\}$ wahr.
Die Aussage $\{1,2,3\}\subseteq\{1,2\}$ ist jedoch falsch,
weil $3$ kein Element von $\{1,2\}$ ist. Für jede Menge $M$ gilt
$M\subseteq M$, denn die Aussage
\[x\in M\implies x\in M\]
ist immer wahr, da die Formel »$\varphi\Rightarrow\varphi$«
tautologisch ist.

\subsection{Mengen von Zahlen}
\index{Zahlenbereiche}\index{Natürliche Zahlen}\index{ganze Zahlen}%
\index{reelle Zahlen}\index{Dezimalzahl}

Einige Mengen kommen häufiger vor, was dazu führte, dass man für
diese Mengen kurze Symbole definiert hat.

Die Menge der natürlichen Zahlen mit der Null:
\[\N_0 := \{0,1,2,3,4,\ldots\}.\]
Die Menge der natürlichen Zahlen ohne die Null:
\[\N := \{1,2,3,4,\ldots\}.\]
Die Menge der ganzen Zahlen:
\[\Z := \{\ldots,-4,-3,-2,-1,0,1,2,3,4,\ldots\}.\]
Dann gibt es noch die rationalen Zahlen $\Q$, das sind alle
Brüche der Form $m/n$, wobei $m,n$ ganze Zahlen sind
und $n\ne 0$ ist. Rationale Zahlen lassen sich immer als
Dezimalbruch schreiben, dessen Ziffern irgendwann periodisch
werden.

\begin{table}[h]
\centering
\begin{tabular}{c|l|l}
\strong{Zahl} & \strong{als Dezimalzahl} & \strong{kurz}\\
$1/2$ & $0.5000000000\ldots$ & $0.5\overline{0}$\\
$1/3$ & $0.3333333333\ldots$ & $0.\overline{3}$\\
$1241/1100$ & $1.1281818181\ldots$ & $0.12\overline{81}$
\end{tabular}
\caption{Jeder Bruch lässt sich als Dezimalzahl
schreiben, deren Ziffern in eine periodische Zifferngruppe münden.
Über die periodische Zifferngruppe setzt man einen waagerechten
Strich.}
\end{table}

\noindent
Schließlich gibt es noch die reellen Zahlen $\R$. Darin enthalten sind
alle Dezimalzahlen -- auch solche, deren Ziffern niemals in eine
periodische Zifferngruppe münden. Die reellen Zahlen haben
eine recht komplizierte Struktur, und wir benötigen Mittel
der Analysis um diese verstehen zu können. Solange diese Werkzeuge
noch nicht bekannt sind, kann man die reellen Zahlen einfach
als kontinuierliche Zahlengerade betrachten. Die rationalen
Zahlen haben Lücken in dieser Zahlengerade, z.\,B. ist die Zahl
$\sqrt{2}$ nicht rational, wie sich zeigen lässt. Die reellen
Zahlen schließen diese Lücken.

\subsection{Vergleich von Mengen}\index{Menge!Vergleich von Mengen}

Wie können wir denn wissen, wann zwei Mengen $A,B$, gleich sind?
Zwei Mengen sind ja gleich, wenn sie beide die gleichen Elemente
enthalten. Aber wie lässt sich das als mathematische Aussage
formulieren?

Jedes Element von $A$ muss doch auch ein Element von $B$ sein,
sonst gäbe es Elemente in $A$, die nicht in $B$ enthalten wären.
Umgekehrt muss auch jedes Element von $B$ ein Element von $A$ sein.
Also ist $A\subseteq B$ und $B\subseteq A$ eine notwendige Bedingung.
Diese Bedingung ist sogar hinreichend.

Gehen wir mal von der Kontraposition aus -- sind die beiden Mengen
$A,B$ verschieden, dann muss es ein Element in $A$ geben, welches nicht
in $B$ enthalten ist, oder eines in $B$, welches nicht $A$ enthalten
ist. Als Formel:%
\[A\ne B \implies (\exists x\in A)(x\notin B)\lor(\exists x\in B)(x\notin A).\]
Hiervon bildet man wieder die Kontraposition. Gemäß den
De Morganschen Gesetzen und den verallgemeinerten
De Morganschen Gesetzen ergibt sich%
\[(\forall x\in A)(x\in B)\land(\forall x\in B)(x\in A)\implies A=B.\]
Auf der linken Seite stehen aber nach Definition
Teilmengenbeziehungen, es ergibt sich%
\[A\subseteq B\land B\subseteq A\implies A=B.\]
\begin{Definition}[Gleichheit von Mengen]%
\index{Gleichheit!von Mengen}\mbox{}\\*
Zwei Mengen $A,B$ sind genau dann gleich, wenn jedes Element von
$A$ auch in $B$ enthalten ist, und jedes von $B$ auch in $A$ enthalten:%
\[A=B\defiff A\subseteq B\land B\subseteq A.\]
\end{Definition}
\begin{Satz}\label{set-eq}
Es gilt
\[A=B\iff (\forall x)(x\in A\iff x\in B).\]
\end{Satz}
\strong{Beweis.} Wir müssen ein wenig Prädikatenlogik bemühen:%
\begin{align*}
A\subseteq B\land B\subseteq A
&\iff (\forall x\in A)(x\in B)\land(\forall x\in B)(x\in A)\\
&\iff (\forall x)(x\in A\implies x\in B)\land(\forall x)(x\in B\implies x\in A)\\
&\iff (\forall x)((x\in A\implies x\in B)\land (x\in B\implies x\in A))\\
&\iff (\forall x)(x\in A\iff x\in B).
\end{align*}
Im letzten Schritt wurde ausgenutzt, dass die Äquivalenz
$\varphi\Leftrightarrow\psi$ gleichbedeutend
mit der Formel $(\varphi\Rightarrow\psi)\land(\psi\Rightarrow\varphi)$
ist.\;\qedsymbol

%\newpage
\subsection{Beschreibende Angabe von Mengen}

Umso mehr Elemente eine Menge enthält, umso umständlicher wird
die Auflistung all dieser Elemente. Außerdem hantiert man in der
Mathematik normalerweise auch ständig mit Mengen herum, die
unendlich viele Elemente enthalten. Eine explizite Auflistung ist
demnach unmöglich.

Wir entgehen der Auflistung aller Elemente durch eine Beschreibung
der Menge. Die Menge der ganzen Zahlen, welche kleiner als vier sind,
wird so beschrieben:%
\[\{n\in\Z\mid n<4\}.\]
In Worten: Die Menge der $n\in\Z$, für die gilt: $n<4$.

Mit dieser Notation kann man nun z.\,B. schreiben:%
\begin{align*}
\N_0 &= \{n\in\Z\mid n\ge 0\},\\
\N &= \{n\in\Z\mid n>0\}.
\end{align*}
Mit der folgenden formalen Definition wird die beschreibende Angabe
auf ein festes Fundament gebracht.

\begin{Definition}[Beschränkte Beschreibung einer Menge]%
\label{def:set-builder-bounded}\index{Menge!Comprehension}\mbox{}\\*
Die Menge der $x\in M$, welche die Aussage $P(x)$ erfüllen,
ist definiert durch die folgende logische Äquivalenz:%
\[a\in\{x\in M\mid P(x)\} \defiff a\in M\land P(a).\]
\end{Definition}
Das schaut ein wenig kompliziert aus, ist aber ganz einfach zu
benutzen. Sei z.\,B. $A:=\{n\in\Z\mid n<4\}$. Zu beantworten ist
die Frage, ob $2\in A$ gilt. Eingesetzt in die Definition
ergibt sich%
\[2\in\{n\in\Z\mid n<4\}\iff 2\in\Z\land 2<4.\]
Da $2\in\Z$ und $2<4$ wahre Aussagen sind, ist die rechte Seite
erfüllt, und damit auch die linke Seite der Äquivalenz.

Die geraden Zahlen lassen sich so definieren:%
\[2\Z:=\{n\in\Z\mid\text{es gibt ein $k\in\Z$ mit $n=2k$}\}.\]
Es lässt sich zeigen:
\[a\in 2\Z\implies a^2\in 2\Z.\]
Nach Definition von $2\Z$ gibt es $k\in\Z$ mit $a=2k$.
Dann ist $a^2=(2k)^2=4k^2=2(2k^2)$. Benennt man $k':=2k^2$, dann
gilt also $a^2=2k'$. Also gibt es es ein $k'\in\Z$
mit $a^2=2k'$, und daher ist $a^2\in 2\Z$.

Die geraden Zahlen sind ganze Zahlen, welche ohne Rest durch zwei
teilbar sind. Die ganzen Zahlen, welche ohne Rest durch $m$ teilbar
sind, lassen sich formal so definieren:%
\[m\Z:=\{n\in\Z\mid\text{es gibt ein $k\in\Z$ mit $n=mk$}\}.\]
Man zeige:
\begin{align*}
& (1.)\;\;a\in 2\Z\implies a^2\in 4\Z, && (3.)\;\;2\Z\subseteq\Z,\\
& (2.)\;\;a\in 4\Z\implies a\in 2\Z,   && (4.)\;\;4\Z\subseteq 2\Z.
\end{align*}


\begin{Definition}[Beschreibende Angabe einer Menge]%
\label{def:set-builder}\mbox{}\\*
Stellt man sich unter $G$ die Grundmenge vor, welche
alle Elemente enthält, die überhaupt in Betracht kommen können,
dann schreibt man kurz%
\[\{x\mid P(x)\} := \{x\in G\mid P(x)\}\]
und nennt dies die Beschreibung einer Menge.
\end{Definition}
\begin{Satz}
Es gilt
\begin{gather}
\label{eq:set-builder}
a\in\{x\mid P(x)\}\iff P(a),\\
\label{eq:bound-conversion}
\{x\in A\mid P(x)\} = \{x\mid x\in A\land P(x)\}.
\end{gather}
\end{Satz}
\strong{Beweis.} Gemäß Definition \ref{def:set-builder}
und \ref{def:set-builder-bounded} gilt%
\[a\in\{x\mid P(x)\} \iff a\in\{x\in G\mid P(x)\}
\iff a\in G\land P(a)\iff P(a),\]
denn $a\in G$ ist immer erfüllt, wenn $G$ die Grundmenge ist.
Die Aussage $a\in G$ kann daher in der Konjunktion gemäß dem
Neutralitätsgesetz der booleschen Algebra entfallen.

Aussage \eqref{eq:bound-conversion} wird mit Satz \ref{set-eq}
expandiert. Zu zeigen ist nun
\[a\in\{x\in A\mid P(x)\}\iff a\in\{x\mid x\in A\land P(x)\},\]
was gemäß Definition \ref{def:set-builder-bounded} und der schon
bewiesenen Aussage \eqref{eq:set-builder} aber vereinfacht
werden kann zu
\[a\in A\land P(a)\iff a\in A\land P(a).\;\qedsymbol\]


\subsection{Bildmengen}\index{Bildmenge}

Oft kommt auch die Angabe einer Menge als Bildmenge vor, dabei
handelt es sich um eine spezielle Beschreibung der Menge. Ist
$T(x)$ ein Term und $A:=\{a_1,a_2,\ldots,a_n\}$ eine endliche
Menge, dann wird das Bild von $A$ unter $T(x)$ so beschrieben:
\[\{T(x)\mid x\in A\} := \{T(a_1),T(a_2),\ldots, T(a_n)\}.\]
Lies: Die Menge der $T(x)$, für die $x\in A$ gilt.
Für $T(x):=x^2$ und $A:=\{1,2,3,4\}$ ist z.\,B.
\[\{T(x)\mid x\in A\} = \{T(1), T(2), T(3), T(4)\}
= \{1^2,2^2,3^2,4^2\} = \{1,4,9,16\}.\]
Nun kann es aber sein, dass die Menge $A$ unendlich viele Elemente
enthält, eine Auflistung dieser somit unmöglich ist. Eine Auflistung
lässt umgehen, indem man nur logisch die Existenz eines Bildes
zu jedem $x\in A$ verlangt, dieses aber nicht mehr explizit angibt.
Man definiert also allgemein
\[\{T(x)\mid x\in A\} := \{y\mid\text{es gibt ein $x\in A$, für das gilt: $y=T(x)$}\}.\]
Das hatten wir bei den geraden Zahlen
\[2\Z := \{2k\mid k\in\Z\} = \{n\mid\text{es gibt ein $k\in\Z$, für das gilt: $n=2k$}\}\]
schon kennengelernt. Hierbei ist es unwesentlich, ob man $n\in\Z$ verlangt
oder nicht, denn dies wird bereits durch $k\in\Z$ erzwungen.

\newpage
\subsection{Mengenoperationen}

Mengen sind mathematische Objekte, mit denen sich rechnen lässt.
So wie es für Zahlen Rechenoperationen gibt, gibt es auch für
Mengen Rechenoperationen.
\begin{Definition}[Vereinigungsmenge]%
\index{Vereinigungsmenge}\index{Menge!Vereinigung}\mbox{}\\*
Die Vereinigungsmenge von zwei Mengen $A,B$ ist die Menge aller Elemente,
welche in $A$ oder in $B$ vorkommen:
\[A\cup B := \{x\mid x\in A\lor x\in B\}.\]
\end{Definition}
Man nimmt also einen neuen Beutel und schüttet den Inhalt von $A$
und $B$ in diesen Beutel.

Beispiele:
\begin{gather*}
\{1,2\}\cup\{5,7,9\} = \{1,2,5,7,9\},\\
\{1,2\}\cup\{1,3,5\} = \{1,2,3,5\}.
\end{gather*}

\begin{Definition}[Schnittmenge]%
\index{Schnittmenge}\index{Menge!Schnitt}\mbox{}\\*
Die Schnittmenge von zwei Mengen $A,B$ ist die Menge aller Elemente,
welche sowohl in $A$ also auch in $B$ vorkommen:
\[A\cap B := \{x\mid x\in A\land x\in B\}.\]
\end{Definition}
\begin{Satz}
Bei der Beschreibung der Schnittmenge $A\cap B$ genügt es, $A\cup B$ als Grundmenge
zu verwenden, denn es gilt
\[A\cap B = \{x\in A\cup B\mid x\in A\land x\in B\}\]
\end{Satz}
\strong{Beweis.}
Die Formel wird mit Satz \ref{set-eq} expandiert. Zu zeigen ist demnach
\[a\in A\cap B\iff a\in \{x\in A\cup B\mid x\in A\land x\in B\}.\]
Das ist nach \eqref{eq:set-builder} und Definition
\ref{def:set-builder-bounded} gleichbedeutend mit
\begin{align*}
a\in A\land a\in B&\iff a\in A\cup B\land a\in A\land a\in B\\
&\iff (a\in A\lor a\in B)\land a\in A\land a\in B.
\end{align*}
Nun gilt für beliebige Aussagen $\varphi,\psi$ gemäß boolescher Algebra aber
\begin{align*}
(\varphi\lor\psi)\land\varphi\land\psi
&\iff (\varphi\land\varphi\land\psi)\lor(\psi\land\varphi\land\psi)\\
&\iff (\varphi\land\psi)\lor(\varphi\land\psi)\\
&\iff \varphi\land\psi.
\end{align*}
Auf beiden Seiten der Äquivalenz steht jetzt die gleiche Aussage:%
\[a\in A\land a\in B\iff a\in A\land a\in B.\;\qedsymbol\]

\begin{Definition}[Vereinigung beliebig vieler Mengen]\mbox{}\\*
Sei $M$ eine Menge von Mengen. Die Vereinigung der $A\in M$ ist
definiert gemäß
\[\bigcup M = \bigcup_{A\in M} A := \{x\mid (\exists A)(A\in M\land x\in A)\}.\]
\end{Definition}
Für $M=\{\}$ ist $\bigcup_{A\in M} A = \{\}$.

Das logische ODER findet seine Entsprechung genau in der Vereinigung
von zwei Mengen. Dazu passend findet der Existenzquantor seine
Entsprechung genau in der Vereinigung beliebig vieler Mengen.
Aus diesem Grund lassen sich Regeln der booleschen Algebra direkt
auf die Mengenoperationen übertragen. Z.\,B. lautet das
Distributivgesetz für Mengen
\[B\cap\bigcup_{A\in M} A = \bigcup_{A\in M}(B\cap A).\]
Diese Gleichung lässt sich nämlich expandieren in die logische Formel
\[x\in B\land(\exists A\in M)(x\in A) \iff (\exists A\in M)(x\in B\land x\in A).\]
Die Äquivalenz ist wie gesagt gültig gemäß Satz
\ref{logical-dist-general}.

\begin{Definition}[Schnitt beliebig vieler Mengen]\mbox{}\\*
Sei $M$ eine nichtleere Menge von Mengen. Der Schnitt der $A\in M$
ist definiert gemäß
\[\bigcap M = \bigcap_{A\in M} A := \{x\mid(\forall A)(A\in M\Rightarrow x\in A)\}.\]
\end{Definition}
Im Gegensatz zur Vereinigung wurde der Schnitt $\bigcap_{A\in M} A$ 
für $M=\{\}$ undefiniert gelassen. Hier gibt es zwei Möglichkeiten.
Zum einen könnte man die Bedingung $M\ne\{\}$ einfach fallen
lassen, dann ergibt sich beim leeren Schnitt immer die Grundmenge
$G=\{x\mid 1\}$. Im allgemeinen Mengenuniversum ist $G$ die
Allklasse. Diese ist nach den ZFC"=Axiomen jedoch keine Menge mehr.

Aus diesen Grund gibt es noch die alternative Definition
\[\bigcap_{A\in M} A :=
\{x\in\bigcup_{A\in M} A\mid(\forall A)(A\in M\Rightarrow x\in A)\}.\]
Eine Familie $(A_i)$ von Mengen $A_i$ mit $i\in I$ ist eine Abbildung
$A\colon I\to Z$, wobei $Z$ eine Zielmenge ist, welche die
$A_i$ als Elemente enthält. Die Menge $I$ wird in diesem Zusammenhang
auch Indexmenge genannt. Man definiert dafür
\[\bigcup_{i\in I} A_i := \bigcup A(I)
= \bigcup\{X\mid(\exists i\in I)(X=A_i)\} = \{x\mid (\exists i\in I)(x\in A_i)\},\]
wobei mit $A(I)$ das Bild von $I$ unter $A$ gemeint ist. Gemäß Def. \ref{def:img} bekommt man
\begin{align*}
\bigcup_{i\in I} A_i
&= \{x\mid (\exists X)(X\in \{X\mid(\exists i\in I)(X=A_i)\}\land x\in X)\}\\
&= \{x\mid (\exists X)((\exists i\in I)(X=A_i)\land x\in X)\}
= \{x\mid (\exists X)(\exists i\in I)(X=A_i\land x\in X)\}\\
&= \{x\mid (\exists X)(\exists i\in I)(x\in A_i)\}
= \{x\mid (\exists i\in I)(x\in A_i)\}.
\end{align*}
Für $I\ne\{\}$ definiert man entsprechend
\[\bigcap_{i\in I} A_i := \bigcap A(I) = \{x\mid (\exists i\in I)(x\in A_i)\}.\]

\newpage
\begin{Definition}[Differenzmenge]\mbox{}\\*
Für zwei Mengen $A,B$ ist
$A\setminus B := \{x\mid x\in A\land x\notin B\}$.
\end{Definition}
\begin{Definition}[Komplementärmenge]\mbox{}\\*
Ist $G$ eine festgelegte Grundmenge und $A\subseteq G$, dann ist
$A^\compc := G\setminus A$.
\end{Definition}
Die Komplementärmenge entspricht der logischen Negation, denn
\[A^\compc = \{x\mid x\in G\land x\notin A\}
= \{x\in G\mid x\notin A\}.\]
Hat man eine Grundmenge festgelegt, so dass alle betrachteten
Mengen Teilmengen dieser Grundmenge sind, dann genügen die
Operationen $A^{\mathrm c}$, $A\cap B$, $A\cup B$
den gleichen Regeln wie ihre logischen Entsprechungen $\neg A$,
$A\land B$, $A\lor B$. Nämlich bilden diese eine boolesche Algebra.
Definiert man axiomatisch, was unter einer booleschen Algebra
zu verstehen ist, dann lassen sich damit Regeln herleiten, die
sowohl für die Aussagenlogik als auch für die Mengenlehre
gültig sein müssen.

Um eine axiomatische Präzisierung kümmern wir uns später.
Zunächst übertragen wir weitere wichtige Rechenregeln ausgehend
von der Aussagenlogik.

Aus $\neg\neg A\equiv A$ für eine Aussage $A$ folgt
$(A^\compc)^\compc = A$ für eine Menge $A$, denn
\begin{gather*}
x\in (A^\compc)^\compc \iff x\in G\land \neg x\in\{x\mid x\in G\land \neg x\in A\}\\
\iff x\in G\land \neg (x\in G\land \neg x\in A)
\iff x\in G\land (\neg x\in G\lor \neg\neg x\in A)\\
\iff 0\lor x\in G\land x\in A
\iff x\in G\land x\in A
\iff x\in A.
\end{gather*}
Die letzte Äquivalenz gilt wegen $A\subseteq G$. Käme die Grundmenge
dabei nicht in den Weg, würde sich die Rechnung zu
\begin{gather*}
x\in (A^\compc)^\compc \iff \neg x\in\{x\mid \neg x\in A\}
\iff \neg\neg x\in A\iff x\in A
\end{gather*}
vereinfachen. Zum einen müsste man dann aber
die Allklasse als »Grundmenge« benutzen, zum anderen ist die Regel
so nicht für jede beliebige Grundmenge $G$ mit $A\subseteq G$
gezeigt.

Die trivialen Regeln $\{\}^\compc = G$ und $G^\compc=\{\}$
entsprechen $\neg 0\equiv 1$ und $\neg 1\equiv 0$.
Dem logischen Wert wahr entspricht demnach die Grundmenge
und dem logischen Wert falsch die leere Menge.

Der Leser zeige zur Übung auch die Übertragung der De Morganschen
Gesetze
\begin{gather*}
(A\cup B)^\compc = A^\compc\cap B^\compc,\qquad
(A\cap B)^\compc = A^\compc\cap B^\compc.
\end{gather*}
Die Komplementärgesetze:
\begin{gather*}
A\cup A^\compc = G,\qquad A\cap A^\compc = \{\}.
\end{gather*}
Der Kontraposition entspricht die Formel
$A^\compc\cup B=(B^\compc)^\compc\cup A^\compc$. Im Zusammenhang
mit der Teilmengenrelation hat die Kontraposition aber auch noch
ein anderes Analogon, das ist%
\begin{gather*}
A\subseteq B \iff B^\compc\subseteq A^\compc.
\end{gather*}


\newpage
\subsection{Produktmengen}
Zwei Objekte $a,b$ kann man zu einem geordneten Paar $(a,b)$
zusammenfassen. Zwei Paare sind definitionsgemäß genau dann
gleich, wenn sie komponentenweise gleich sind:
\[(a_1,b_1) = (a_2,b_2) \defiff a_1=a_2\land b_1=b_2.\]

\begin{Definition}[Kartesisches Produkt]\mbox{}\\*
Das kartesische Produkt der Mengen $A,B$ ist die Menge der
Paare $(a,b)$, für die $a\in A$ und $b\in B$ ist, kurz
\[A\times B = \{(a,b)\mid a\in A\land b\in B\}.\]
\end{Definition}
Zu beachten ist, dass hier eine Bildmenge vorliegt, d.\,h. es gilt
\begin{align*}
A\times B &= \{t\mid(\exists a\in A)(\exists b\in B)(t=(a,b))\}\\
&= \{t\mid(\exists a)(\exists b)(a\in A\land b\in B\land t=(a,b))\}.
\end{align*}

\begin{Satz}
Für das kartesische Produkt mit der leeren Menge gilt $A\times\emptyset=\emptyset$
und $\emptyset\times B=\emptyset$.
\end{Satz}
\strong{Beweis.} Das kann man einfach nachrechnen.
Unter Anwendung von Satz \ref{set-eq} und \eqref{eq:set-builder}
bekommt man zunächst die äquivalente Aussage
\begin{gather*}
t\in A\times\emptyset \iff (\exists a)(\exists b)(a\in A\land b\in\emptyset\land t=(a,b)).
\end{gather*}
Nun ist aber $b\in\emptyset$ niemals wahr, da die leere Menge keine
Elemente enthält. Demnach ergibt sich
\begin{gather*}
(\exists a)(\exists b)(a\in A\land b\in\emptyset\land t=(a,b))
\iff (\exists a)(\exists b)\, 0 \iff (\exists a)\,0\iff 0.
\end{gather*}
Die Aussage $t\in A\times\emptyset$ ist also immer falsch,
daher kann $A\times\emptyset$ keine Elemente enthalten.\;\qedsymbol

\begin{Satz}
Ist $A\subseteq X$ und $B\subseteq Y$, dann ist
$A\times B\subseteq X\times Y$.
\end{Satz}
\strong{Beweis.} Sei $t$ ein Paar, das in
$A\times B$ enthalten ist. Dann gibt es nach Definition
$a\in A$ und $b\in B$, so dass $t=(a,b)$. Wegen
$A\subseteq X$ ist aber auch $a\in X$ und wegen $B\subseteq Y$
ist auch $b\in Y$. Daher gibt es $a\in X$ und $b\in Y$, so dass
$t=(a,b)$. Gemäß Definition heißt das $t\in X\times Y$. Gemäß
Definition ist $A\times B$ daher eine Teilmenge von $X\times Y$.\;\qedsymbol

\newpage
\section{Abbildungen}
\subsection{Grundbegriffe}
Seien zwei beliebige Mengen $A,B$ gegeben. Eine Abbildung
$f\colon A\to B$ ist eine Zuordnung, die jedem Element $x\in A$
genau ein Element $y\in B$ zuordnet. Man schreibt $y=f(x)$
oder $x\mapsto y$, um auszudrücken, dass dem Element $x$
das Element $y$ zugeordnet wird.

Ausgesprochen wird $f(x)$ als »$f$ von $x$«, oder auch
»das Bild von $x$ unter $f$«. Die Schreibweise $x\mapsto y$
wird ausgesprochen als »$x$ zu $y$«, oder auch
»$x$ wird abgebildet auf $y$«. Die Schreibweise
$f\colon A\to B$ wird ausgesprochen als »$f$ ist eine
Abbildung von $A$ nach $B$«.

Man nennt $A$ die Definitionsmenge oder den Definitionsbereich
der Abbildung und $B$ die Zielmenge der Abbildung. Gibt es zu
einem $y\in B$ ein $x\in A$, so dass $y=f(x)$, dann nennt man
$x$ ein Urbildelement zu $y$.

Abbildungen sind für die Mathematik fundamental. Eine
Formalisierung dieses Begriffs mittels Prädikatenlogik
und Mengenlehre erscheint deshalb erstrebenswert.

\begin{Definition}[Abbildung]\mbox{}\\*
Sei $G\subseteq A\times B$. Man nennt ein Tripel $f=(G,A,B)$ eine
Abbildung, wenn die folgenden zwei Bedingungen erfüllt sind.
1. Zu jedem $x\in A$ gibt es mindestens ein Bild:
\[(\forall x\in A)(\exists y\in B)((x,y)\in G).\]
2. Zu jedem $x\in A$ gibt es höchstens ein Bild:
\[(\forall (x_1,y_1),(x_2,y_2)\in G)(x_1=x_2\implies y_1=y_2).\]
Man definiert außerdem
\[y=f(x)\defiff (x,y)\in G.\]
\end{Definition}

\begin{Definition}[Bildmenge]\label{def:img}\mbox{}\\*
Sei $f\colon A\to B$ eine Abbildung.
Für eine Menge $M\subseteq A$ nennt man die Menge
\[f(M) := \{y\mid(\exists x\in M)(y=f(x))\}\]
das Bild von $M$ unter $f$.
\end{Definition}

\begin{Definition}[Urbildmenge]\mbox{}\\*
Sei $f\colon A\to B$ eine Abbildung. Für eine Menge $N$ nennt man
\[f^{-1}(N) := \{x\in A\mid f(x)\in N\}\]
das Urbild von $N$ bezüglich $f$.
\end{Definition}

\newpage
\subsection{Verkettung von Abbildungen}
\begin{Definition}[Verkettung]\mbox{}\\*
Sei $f\colon A\to B$ und $g\colon B\to C$. Die Abbildung
\[(g\circ f)\colon A\to C,\quad (g\circ f)(x):=g(f(x))\]
heißt Verkettung von $f$ und $g$, sprich »$g$ nach $f$«.
\end{Definition}

\noindent
Oft hat man die Situation vorliegen, bei der $f\colon A\to B$
und $g\colon B'\to C$, wobei $B\subseteq B'$ ist. Das ist aber
nicht so schlimm. Man nimmt die folgende unproblematische
Definitionserweiterung vor:
\[(g\circ f)\colon A\to C,\quad g\circ f := g|_B\circ f.\]
Mit $g|_B$ ist hierbei die Einschränkung der Abbildung $g$
auf den Definitionsbereich $B$ gemeint.

\begin{Definition}[Einschränkung]\mbox{}\\*
Für $f\colon A\to B$ und $M\subseteq A$ nennt man
\[f|_M\colon M\to B,\quad f|_M(x):=f(x)\]
die Einschränkung von $f$ auf $M$.
\end{Definition}

\noindent
Schwerwiegender ist die Situation $f\colon A\to B$
und $g\colon B'\to C$ mit $B'\subseteq B$. Hier dürfen nur
solche $x\in A$ im neuen Definitionsbereich vorkommen, bei
denen $f(x)\in B'$ ist. Gemäß der Definition des Urbildes
gilt wiederum
\[f(x)\in B'\iff x\in f^{-1}(B').\]
Man kann nun die Verkettung definieren gemäß
\[h\colon f^{-1}(B')\to C,\quad h(x):=g(f(x)).\]

\begin{Satz}[Bildmenge unter Verkettungen]\label{img-comp}\mbox{}\\*
Seien $f\colon A\to B$ und $g\colon B\to C$, dann gilt
$(g\circ f)(M) = g(f(M))$.
\end{Satz}
\strong{Beweis.} Die Gleichung gemäß Definition expandieren:
\[(\exists x)(x\in M\land z
= (g\circ f)(x))\iff (\exists y)(y\in f(M)\land z=g(y)).\]
Auf der rechten Seite ergibt sich nun
\begin{gather*}
(\exists y)(y\in f(M)\land z=g(y))
\equiv (\exists y)((\exists x)(x\in M\land y=f(x))\land z=g(y))\\
\equiv (\exists y)(\exists x)(x\in M\land y=f(x)\land z=g(y)))\\
\equiv (\exists x)(x\in M\land\exists y(y=f(x)\land z=g(y)))\\
\equiv (\exists x)(x\in M\land z=g(f(x)).\;\qedsymbol
\end{gather*}


\newpage
\subsection{Injektionen, Surjektionen, Bijektionen}

\begin{Definition}[Injektive Abbidlung]\mbox{}\\*
Eine Abbildung $f\colon A\to B$ heißt injektiv, wenn
\[(\forall x_1,x_2\in A)(f(x_1)=f(x_2)\implies x_1=x_2)\]
bzw.
\[(\forall x_1,x_2\in A)(x_1\ne x_2\implies f(x_1)\ne f(x_2)).\]
\end{Definition}

\begin{Definition}[Surjektive Abbildung]\mbox{}\\*
Eine Abbildung $f\colon A\to B$ heißt surjektiv,
wenn $f(A)=B$ ist.
\end{Definition}
Bemerkung: Da immer $f(A)\subseteq B$ ist, braucht man bloß $B\subseteq f(A)$
zu zeigen.

\begin{Definition}[Bijektive Abbildung]\mbox{}\\*
Eine Abbildung heißt bijektiv, wenn sie sowohl injektiv als
auch surjektiv ist.
\end{Definition}

\begin{Satz}
Sei $f\colon A\to B$ und $g\colon B\to C$. Es gilt:
\begin{align*}
1.\; & \text{Sind $f$ und $g$ injektiv, dann auch $g\circ f$}.\\
2.\; & \text{Sind $f$ und $g$ surjektiv, dann auch $g\circ f$}.\\
3.\; & \text{Sind $f$ und $g$ bijektiv, dann auch $g\circ f$}.
\end{align*}
\end{Satz}
\strong{Beweis.} Mühelos. Seien $f,g$ injektiv, dann gilt
\begin{gather*}
g(f(x_1)) = (g\circ f)(x_1) = (g\circ f)(x_2) = g(f(x_2))\\
\implies f(x_1) = f(x_2)\\
\implies x_1 = x_2.
\end{gather*}
Somit ist auch $g\circ f$ injektiv. Seien $f,g$ nun surjektiv,
dann ergibt sich
\[(g\circ f)(A) = g(f(A)) = g(B) = C\]
gemäß Satz \ref{img-comp}.
Somit ist auch $g\circ f$ surjektiv.\;\qedsymbol

\newpage
\section{Relationen}
\subsection{Grundbegriffe}

\begin{Definition}[Relation]\mbox{}\\*
Seien $A,B$ zwei Mengen und sei $G\subseteq A\times B$.
Das Tripel $R=(G,A,B)$ heißt Relation zwischen $A$ und $B$.
Man schreibt
\[R(x,y) \defiff (x,y)\in G.\]
\end{Definition}
Eine Relation lässt sich natürlich als wahrheitswertige Funktion
interpretieren:
\[R\colon A\times B\to\{0,1\},\quad R(x,y):=((x,y)\in G).\]
Eine Relation ist somit auch ein Prädikat auf $A\times B$.

\subsection{Äquivalenzrelationen}
\begin{Definition}[Äquivalenzrelation]\mbox{}\\*
Seien $A$ eine Menge und seien $x,y,z\in A$. Sei $R(x,y):=(x\sim y)$ eine
Relation. Man nennt $R$ Äquivalenzrelation, wenn gilt:
\[\begin{array}{ll}
x\sim x, &\text{(Reflexivität)}\\
x\sim y \implies y\sim x, & \text{(Symmetrie)}\\
x\sim y\land y\sim z\implies x\sim z.\quad & \text{(Transitivität)}
\end{array}\]
\end{Definition}

\begin{Definition}[Äquivalenzklasse]\mbox{}\\*
Sei $M$ eine Menge und $x\sim y$ eine Äquivalenzrelation für $x,y\in M$.
Die Menge%
\[[a] := \{x\in M\mid x\sim a\}\]
nennt man die Äquivalenzklasse zum Repräsentanten $a\in M$.
\end{Definition}

\begin{Satz}[Äquivalenzrelation induziert Zerlegung]\mbox{}\\*
Eine Menge wird durch eine Äquivalenzrelation in disjunkte
Äquivalenzklassen zerlegt, lat. partitioniert.
\end{Satz}
\strong{Beweis.} Sei $M$ die Menge und $x\sim y$ die Äquivalenzrelation.
Zu zeigen ist, dass kein Element von $M$ in mehr als einer
Äquivalenzklasse vorkommt. Seien $a,b,c\in M$, sei $c\in [a]$
und $c\in [b]$. Aufgrund von $c\sim a$ sowie $c\sim b$ und der
Transitivität gilt%
\[x\in [a]\iff x\sim a\iff x\sim c\iff x\sim b\iff x\in [b].\]
Man hat also
\[(\forall x\in M)(x\in [a]\Leftrightarrow x\in [b])\iff [a]=[b].\]
Wenn also $[a]\ne [b]$ ist, kann nicht gleichzeitig $c\in [a]$ und $c\in [b]$
sein.\;\qedsymbol

\begin{Satz}[Zerlegung induziert Äquivalenzrelation]\mbox{}\\*
Sei $M$ eine Menge. Die Familie $(A_k)$ von Mengen $A_k\subseteq M$
bilde eine Zerlegung von $M$, d.\,h. dass die Vereinigung aller
$A_k$ die Menge $M$ überdeckt und dass paarweise $A_i\cap A_j=\{\}$
für $i\ne j$ ist. Dann ist%
\[x\sim y\defiff (\exists k)(x\in A_k\land y\in A_k)\]
eine Äquivalenzrelation auf $M$.
\end{Satz}
\strong{Beweis.} Da die $A_k$ die Menge $M$ überdecken,
muss es für ein beliebiges $x\in M$ mindestens eine Menge $A_k$
geben, so dass $x\in A_k$. Daher gilt $x\sim x$.

Die Symmetrie ergibt sich trivial.

Zur Transitivität. Voraussetzung ist $x\sim y$ und $y\sim z$.
Es gibt also ein $i$ mit $x\in A_i$ und $y\in A_i$. Außerdem gibt
es ein $j$ mit $y\in A_j$ und $z\in A_j$. Somit gilt%
\[(\exists i)(\exists j)(x\in A_i\land y\in A_i\land y\in A_j\land z\in A_j).\]
Wegen
\[A_i\cap A_j = \{\} \iff (\forall y)(y\in A_i\land y\in A_j\iff 0)\]
für $i\ne j$ kann $y\in A_i\land y\in A_j$ aber nur erfüllt sein,
wenn $i=j$ ist. Daher ergibt sich%
\[(\exists i)(x\in A_i\land z\in A_i),\]
d.\,h. $x\sim z$.\;\qedsymbol

\begin{Definition}[Quotientenmenge]\mbox{}\\*
Für eine gegebene Äquivalenzrelation wird die aus allen
Äquivalenzklassen bestehende Menge
\[M/{\sim} := \{[x]\mid x\in M\}\]
als Quotientenmenge oder Faktormenge bezeichnet.
\end{Definition}

\begin{Definition}[Quotientenabbildung]\mbox{}\\*
Für eine gegebene Äquivalenzrelation ist die Projektion
\[\pi\colon M\to M/{\sim},\quad \pi(x):=[x]\]
surjektiv und wird Quotientenabbildung genannt.
\end{Definition}

\newpage
\begin{Definition}[Repräsentantensystem]\mbox{}\\*
Für eine gegebene Äquivalenzrelation auf $M$ nennt man eine
Teilemenge $A\subseteq M$ ein vollständiges Repräsentantensystem,
wenn die Einschränkung $\pi|_A$ bijektiv ist, wobei mit $\pi$
die Quotientenabbildung gemeint ist.
\end{Definition}
Repräsentantensysteme ermöglichen die einfache Handhabung von
Äquivalenzklassen. Möchte man wissen, ob ein Element $x$ in der
Äquivalenklasse $[a]$ enthalten ist, dann braucht man bloß
zu überprüfen, ob $x\sim a$ ist.

Eine große Fülle von Äquivalenzrelationen lässt auf die folgende
einfache Art konstruieren. Hat man eine beliebige Abbildung
$f\colon A\to B$, dann sind die Urbilder $f^{-1}(\{y_1\})$ und
$f^{-1}(\{y_2\})$ disjunkt, sofern $y_1\ne y_2$:%
\begin{align*}
f^{-1}(\{y_1\})\cap f^{-1}(\{y_2\})
&= \{x\mid f(x)=y_1\}\cap\{x\mid f(x)=y_2\}\\
&= \{x\mid f(x)=y_1\land f(x)=y_2\} = \{\}.
\end{align*}
Im letzten Schritt wurde beachtet, dass eine Abbildung für
das selbe Argument definitionsgemäß keine zwei unterschiedlichen
Werte annehmen kann.

Demnach ist gemäß
\[Z = A/{\sim} = \{f^{-1}(\{y\})\mid y\in f(A)\}\]
eine Zerlegung des Definitionsbereichs $A$ gegeben und somit auch eine
Äquivalenzrelation. Für $x_1,x_2\in A$ gilt%
\[x_1\sim x_2 \iff f(x_1) = f(x_2).\]

\begin{Satz}[Charakterisierung von Äquivalenzklassen]\mbox{}\\*
Sei auf der Menge $M$ eine Äquivalenzrelation gegeben. Eine
Teilmenge $A\subseteq M$ ist genau dann eine Äquivalenzklasse,
wenn%
\begin{align*}
1.\;& A\ne\{\},\\
2.\;& x,y\in A\implies x\sim y,\\
3.\;& x\in A\land y\in M\land x\sim y\implies y\in A.
\end{align*}
\end{Satz}
\strong{Beweis.} Angenommen, $A$ ist eine Äquivalenzklasse.
Dann gibt es definitionsgemäß ein $a$ mit $A=[a]$. Daher ist
mindestens $a\in A$ und somit $A\ne\{\}$. Mit $x,y\in A$ ergibt
sich $A=[x]=[y]$. Aufgrund von%
\[x\sim y \iff [a]=[b]\]
muss somit $x\sim y$ sein. Sei nun $x\in A$ und $y\in M$ mit
$x\sim y$. Es folgt $A=[x]=[y]$. Daher muss $y\in A$ sein.

Umgekehrt angenommen, die drei Eigenschaften sind erfüllt.
Zu zeigen ist, dass es ein $a$ gibt mit $A=[a]$. Da $A$ gemäß 1.
nichtleer ist, enthält es mindestens ein Element, dieses nennen wir
$a$. Für jedes weitere Element $x\in A$ ergibt sich
$x\sim a$, da sonst 2. verletzt sein würde. Schließlich muss man
noch wissen, ob $x\in A$, wenn $x\sim a$ und $x\in M$ ist.
Dies ist aber mit 3. gesichert. Es gibt also
tatsächlich ein $a$ mit $A=\{x\in M\mid x\sim a\}$.\;\qedsymbol

\newpage
\section{Gleichungen}
\subsection{Begriff der Gleichung}\index{Gleichung}

Bei einer Gleichung verhält es sich wie bei einer Balkenwaage. Liegt
in einer der Waagschalen eine Masse von 2g und in der anderen
Waagschale zwei Massen von jeweils 1g, dann bleibt die Waage im
Gleichgewicht. Als Gleichung gilt
\[2=1+1.\]
Eine Gleichung kann wahr oder falsch sein, z.\,B. ist $2=2$
wahr, während $2=3$ falsch ist. Das bedeutet aber nicht, dass man
eine falsche Gleichung nicht aufschreiben dürfe. Vielmehr ist eine
Gleichung ein mathematisches Objekt, dem sich ein Wahrheitswert
zuordnen lässt. Zumindest sollte man eine falsche Gleichung nicht
ohne zusätzliche Erklärung aufschreiben, so dass der Eindruck
entstünde, sie könnte wahr sein.

\subsection{Äquivalenzumformungen}%
\index{Aequivalenzumformung@Äquivalenzumformung!von Gleichungen}

Fügt man zu beiden Schalen einer Balkenwaage das gleiche Gewicht
hinzu, dann bleibt die Waage so wie sie vorher war. War sie im
Gleichgewicht, bleibt sie dabei. War sie im Ungleichgewicht,
bleibt sie auch dabei. Ebenso verhält es sich mit einer Gleichung.
Addition der gleichen Zahl auf beide Seiten einer Gleichung bewirkt
keine Veränderung des Aussagengehalts der Gleichung.

Diese Überlegung gilt natürlich auf für die Subtraktion einer Zahl
auf beiden Seiten, welche dem Entfernen des gleichen Gewichtes von
beiden Waagschalen entspricht.

\begin{Satz}[Äquivalenzumformungen]\label{eq-add}\mbox{}\\*
Seien $a,b,c$ beliebige Zahlen. Dann gilt
\begin{align*}
a=b&\iff a+c=b+c,\\
a=b&\iff a-c=b-c.
\end{align*}
\end{Satz}

\noindent
Auch eine Verdopplung des Gewichtes in beiden Schalen der Balkenwaage
ändert nicht ihr Gleichgewicht oder Ungleichgewicht.

\begin{Satz}[Äquivalenzumformungen]\label{eq-mul-int}\mbox{}\\*
Seien $a,b$ beliebige Zahlen und $n\in\Z$ mit $n\ne 0$. Dann gilt
\begin{align*}
a=b&\iff na=nb.
\end{align*}
\end{Satz}
\strong{Beweis.} Gemäß Satz \ref{eq-add} gilt
\begin{align*}
na = nb &\iff 0 = na-nb = n(a-b)\iff n=0\lor a-b=0\\
&\iff a-b=0\iff a=b.
\end{align*}
Dabei wurde ausgenutzt, dass ein Produkt nur null sein kann,
wenn einer der Faktoren null ist. Gemäß Voraussetzung $n\ne 0$ muss
dann aber $a-b=0$ sein.\;\qedsymbol

\begin{Satz}[Äquivalenzumformungen]\mbox{}\\*
Seien $a,b$ beliebige Zahlen und $r\in\Q$ mit $r\ne 0$. Dann gilt
\[a=b\iff ra=rb\iff a/r=b/r.\]
\end{Satz}
\strong{Beweis.}
Die Zahl $r$ ist von der Form $r=m/n$, wobei $m,n\in\Z$ und $m,n\ne 0$.
Daher gilt
\begin{align*}
ra=rb&\iff \frac{m}{n}a=\frac{m}{n}b
\stackrel{\text{Satz \ref{eq-mul-int}}}\iff n\cdot\frac{m}{n}a=n\cdot\frac{m}{n}b\\
&\iff ma=mb\stackrel{\text{Satz \ref{eq-mul-int}}}\iff a=b.
\end{align*}
Daraufhin gilt auch
\[\frac{a}{r}=\frac{b}{r}\iff r\cdot\frac{a}{r}=r\cdot\frac{b}{r}
\iff a=b.\;\qedsymbol\]

\begin{Satz}[Äquivalenzumformungen]\mbox{}\\*
Seien $a,b,r\in\R$ und sei $r\ne 0$. Dann gilt
\[a=b\iff ra=rb\iff a/r=b/r.\]
\end{Satz}
\strong{Beweis.} Man rechnet wieder
\begin{align*}
ra = rb&\iff ra-rb=0\iff (a-b)r=0\iff r=0\lor a-b=0\\
&\iff a-b=0\iff a=b.
\end{align*}
Es wurde wieder ausgenutzt, dass ein Produkt nur dann null sein
kann, wenn einer der Faktoren null ist. Daraufhin gilt auch
\[\frac{a}{r}=\frac{b}{r}\iff r\cdot\frac{a}{r}=r\cdot\frac{b}{r}
\iff a=b.\;\qedsymbol\]

\noindent


\newpage
\section{Ungleichungen}

\subsection{Begriff der Ungleichung}%
\index{Ungleichung}

Man stelle sich zwei Körbe vor, in die Äpfel gelegt werden.
In den rechten Korb werden zwei Äpfel gelegt, in den linken drei.
Dann befinden sich im rechten Korb weniger Äpfel als im linken.
Man sagt, zwei ist kleiner als drei, kurz $2<3$. Man spricht von
einer \emph{Ungleichung}, in Anbetracht dessen, dass die beiden
Körbe nicht die gleiche Anzahl von Äpfeln enthalten.

Der Aussagengehalt einer Ungleichung kann wahr oder falsch sein.
Die Ungleichung $2<3$ ist wahr, die Ungleichungen $3<3$ und
$4<3$ sind falsch.

\begin{Definition}[Ungleichungsrelation]\mbox{}\\*
Die Notation $a<b$ bedeutet »Die Zahl $a$ ist kleiner als
die Zahl $b$«. Die Notation $a\le b$ bedeutet »Die Zahl $a$
ist kleiner als oder gleich der Zahl $b$«. Die Notation
$b>a$ ist eine andere Schreibweise für $a<b$ und bedeutet
»Die Zahl $b$ ist größer als die Zahl $a$«. Die Notation
$b\ge a$ ist eine andere Schreibweise für $a\le b$ und
bedeutet »Die Zahl $b$ ist größer oder gleich der Zahl $a$«.
\end{Definition}

\subsection{Äquivalenzumformungen}%
\index{Aequivalenzumformung@Äquivalenzumformung!von Ungleichungen}

Wir stellen uns wieder einen linken Korb mit zwei Äpfeln und
einen rechten Korb mit drei Äpfeln vor. Legt man nun in beide
Körbe jeweils zusätzlich 10 Äpfel hinein, dann befinden sich
im linken Korb 12 Äpfel und im rechten 13. Der linke Korb
enthält also immer noch weniger Äpfel als im rechten.

Befindet sich eine Balkenwaage im Ungleichgewicht, und legt man
in beide Waagschalen zusätzlich die gleiche Masse von Gewichten,
dann wird sich das Ungleichgewicht der Balkenwaage nicht verändern.

Für die Herausnahme von Äpfeln oder Gewichten ist diese Argumentation
analog. Ist stattdessen eine falsche Ungleichung gegeben,
dann lässt sich durch Addition der selben Zahl auf beiden Seiten
daraus keine wahre Ungleichung gewinnen. Die analoge Argumentation
gilt für die Subtraktion der selben Zahl. Anstelle von ganzen
Äpfeln kann man natürlich auch Apfelhälften hinzufügen, oder
allgemein Apfelbruchteile. Die Argumentation gilt unverändert.

Wir halten fest. 

\begin{Satz}[Äquivalenzumformungen von Ungleichungen]\mbox{}\\*
Seien $a,b,c$ beliebige Zahlen. Dann sind die folgenden
Äquivalenzen gültig:
\begin{gather}
\label{lt-add} a<b\iff a+c<b+c,\\
\label{lt-sub} a<b\iff a-c<b-c,\\
\label{le-add} a\le b\iff a+c\le b+c,\\
\label{le-sub} a\le b\iff a-c\le b-c.
\end{gather}
\end{Satz}

\noindent
In Worten: Wenn auf beiden Seiten einer Ungleichung die gleiche
Zahl addiert oder subtrahiert wird, dann ändert sich der
Aussagengehalt dieser Ungleichung nicht.

Gibt es noch andere Äquivalenzumformungen?

Im linken Korb seien wieder zwei Äpfel, im rechten drei. Verdoppelt
man nun die Anzahl in beiden Körben, dann sind linken vier Äpfel,
im rechten sechs. Verzehnfacht man die Anzahl, dann sind im linken
20 Äpfel, im rechten 30. Offenbar verändert sich der Aussagengehalt
nicht, wenn die Anzahl auf beiden Seiten der Ungleichung mit
der gleichen natürlichen Zahl $n$ multipliziert wird.

Jedoch muss $n=0$ ausgeschlossen werden. Wenn $a<b$ ist, und man
multipliziert auf beiden Seiten mit null, dann ergibt sich
$0<0$, was falsch ist. Aus der wahren Ungleichung wurde damit eine
falsche gemacht, also kann es sich nicht um eine Äquivalenzumformung
handeln.

Auch bei der Ungleichung $a\le b$ muss $n=0$ ausgeschlossen werden.
Warum muss man das tun? Die Ungleichung $0\le 0$ ist doch auch
wahr?

Nun, wenn der Aussagengehalt von $a\le b$ falsch ist, z.\,B. $4\le 3$,
und man multipliziert auf beiden Seiten mit null, dann ergibt sich
$0\le 0$, also eine wahre Ungleichung. Aus einer falschen wurde damit
eine wahre gemacht. Bei einer Äquivalenzumformung ist dies ebenfalls
verboten.

\begin{Satz}[Äquivalenzumformungen von Ungleichungen]\mbox{}\\*
Seien $a,b$ beliebige Zahlen und sei $n>0$ eine natürliche Zahl.
Dann sind die folgenden Äquivalenzen gültig:
\begin{gather}
\label{lt-mul-nat} a<b\iff na<nb,\\
\label{le-mul-nat} a\le b\iff na\le nb.
\end{gather}
\end{Satz}

\noindent\strong{Beweis.}
Aus der Ungleichung $a<b$ erhält man mittels \eqref{lt-sub} die
äquivalente Ungleichung $0<b-a$, indem auf beiden Seiten $a$
subtrahiert wird. Die Zahl $b-a$ ist also positiv. Durch Multiplikation
mit einer positiven Zahl lässt sich das Vorzeichen einer Zahl
aber nicht umkehren. Demnach ist $0<n(b-a)$ genau dann,
wenn $0<b-a$ war. Ausmultiplizieren liefert nun
$0<nb-na$ und Anwendung von \eqref{lt-add} bringt dann $na<nb$.

In Kürze formuliert:
\begin{equation}
a<b\iff 0<b-a\iff 0<n(b-a)=nb-na \iff na<nb.
\end{equation}
Für $a\le b$ gilt diese Überlegung analog.\;\qedsymbol

\strong{Alternativer Beweis.}
Mittels \eqref{lt-add} ergibt sich zunächst:
\begin{equation}
a<b\iff \left\{
\begin{matrix}
a+a<b+a\\
a+b<b+b
\end{matrix}
\right\}
\iff 2a<a+b<2b.
\end{equation}
Unter nochmaliger Anwendung von \eqref{lt-add} ergibt sich
nun
\begin{equation}
a<b\iff \left\{
\begin{matrix}
2a<a+b \iff 3a<2a+b\\
2a<2b \iff 2a+b<3b
\end{matrix}
\right\} 3a<2a+b<3b
\end{equation}
Dieses Muster lässt sich induktiv alle natürlichen Zahlen hochschieben:
Aus $na<(n-1)a+b<nb$ sollte sich
$(n+1)a<na+b<(n+1)b$ schlussfolgern lassen und umgekehrt.
Das ist richtig, denn Addition von $a$ gemäß \eqref{lt-add} bringt
\begin{equation}
na<(n-1)a+b \iff (n+1)a < na+b
\end{equation}
und Addition von $b$ gemäß \eqref{lt-add} bringt
\begin{equation}
na<nb \iff na+b < (n+1)b.
\end{equation}
Zusammen ergibt sich daraus der behauptete Induktionsschritt. 
Daraus erhält man $a<b\iff na<nb$. Für $a\le b$ sind diese
Überlegungen analog.\;\qedsymbol

Wir können sogleich einen Schritt weiter gehen.
\begin{Satz}[Äquivalenzumformungen von Ungleichungen]\mbox{}\\*
Seien $a,b$ beliebige Zahlen und sei $r>0$ eine rationale Zahl,
dann gelten die folgenden Äquivalenzen:
\begin{gather}
\label{lt-mul-rat} a<b\iff ra<rb\iff a/r<b/r,\\
\label{lt-mul-rat} a\le b\iff ra\le rb\iff a/r\le b/r.
\end{gather}
\end{Satz}

\noindent\strong{Beweis.}
Eine rationale Zahl $r>0$ lässt sich immer Zerlegen in einen Quotienten
$r=m/n$, wobei $m,n$ positive natürliche Zahlen sind. Gemäß
\eqref{lt-mul-nat} gilt
\begin{equation}
\frac{m}{n}\cdot a<\frac{m}{n}\cdot b
\iff n\cdot\frac{m}{n}\cdot a<n\cdot\frac{m}{n}\cdot b
\iff ma<mb.
\end{equation}
Gemäß \eqref{lt-mul-nat} gilt aber auch
\begin{equation}
a<b\iff ma<mb.
\end{equation}
Die Zusammenfassung beider Äquivalenzen ergibt
\begin{equation}
a<b\iff \frac{m}{n}\cdot a<\frac{m}{n}\cdot b\iff ra<rb.
\end{equation}
Für $a\le b$ ist die Argumentation analog. Da die Division durch
eine rationale Zahl $r$ die Multiplikation mit ihrem Kehrwert $1/r$ ist,
sind auch die Äquivalenzen für die Division gültig.\;\qedsymbol

Da sich eine reelle Zahl beliebig gut durch eine rationale annähern
lässt, müsste auch der folgende Satz gültig sein.

\begin{Satz}[Äquivalenzumformungen von Ungleichungen]\mbox{}\\*
Seien $a,b$ beliebige Zahlen und sei $r>0$ eine reelle Zahl,
dann gelten die folgenden Äquivalenzen:
\begin{gather}
\label{lt-mul-real} a<b\iff ra<rb\iff a/r<b/r,\\
\label{lt-mul-real} a\le b\iff ra\le rb\iff a/r\le b/r.
\end{gather}
\end{Satz}

\noindent
Der Satz wird sich als richtig erweisen, der Beweis kann in
Analysis"=Lehrbüchern nachgeschlagen werden.

\subsection{Lineare Ungleichungen}

Interessant werden Ungleichungen nun, wenn in ihnen einen Variable
vorkommt. Beispielsweise sei die Ungleichung $x+2<4$ gegeben.
Wird in diese Ungleichung für die Variable $x$ eine Zahl eingesetzt,
dann kann wird die Ungleichung entweder wahr oder falsch sein.
Für $x:=1$ ergibt sich die wahre Ungleichung $1+2<4$. Für $x:=2$
ergibt sich jedoch die falsche Ungleichung $2+2<4$.

Wir interessieren uns nun natürlich für die Menge aller Lösungen
dieser Ungleichung. Das sind die Zahlen, welche die Ungleichung
erfüllen, wenn sie für $x$ eingesetzt werden. Gesucht ist also
die Lösungsmenge
\[L = \{x\mid x+2<4\},\]
d.\,h. die Menge der $x$, welche die Ungleichung $x+2<4$ erfüllen.

Gemäß Äquivalenzumformung \eqref{lt-sub} kommt man aber sofort zu
\[x+2<4 \iff x+2-2<4-2 \iff x<2.\]
Demnach kann die Lösungsmenge als $L=\{x\mid x<2\}$ angegeben werden,
denn Äquivalenzumformungen lassen die Lösungsmenge einer Ungleichung
unverändert.

Die Ungleichung $x+2<4$ ist sicherlich von so einfacher Gestalt,
dass man diese auch gedanklich lösen kann, ohne Äquivalenzumformungen
bemühen zu müssen. Bei komplizierteren Ungleichungen kommen wir dabei
aber mehr oder weniger schnell an unsere mentalen Grenzen.

Schon ein wenig schwieriger ist z.\,B.
\begin{align*}
& 5x+2<3x+10 && |\;{-2}\\
\iff & 5x<3x+8 && |\;{-3x}\\
\iff & 2x<8 && |\;{/2}\\
\iff & x<4.
\end{align*}

\subsection{Monotone Funktionen}%
\index{Monotone Funktion}

\begin{Definition}[Streng monoton steigende Funktion]%
\index{Streng monotone Funktion}%
\index{Monotone Funktion!strenge Monotonie}\mbox{}\\*
Eine Funktion $f\colon G\to\R$ heißt streng monoton steigend, wenn
\[a<b\implies f(a)<f(b)\]
für alle Zahlen $a,b\in G$ erfüllt ist.
\end{Definition}
Streng monotone Abbildungen sind von besonderer Bedeutung, weil
sie gemäß ihrer Definition auch Äquivalenzumformungen sind:

\begin{Satz}[Allgemeine Äquivalenzumformung]%
\index{Aequivalenzumformung@Äquivalenzumformung!allgemein für Ungleichungen}%
\mbox{}\\*
Eine streng monoton steigende Funktionen $f$ ist umkehrbar eindeutig.
Die Umkehrfunktion ist auch streng monoton steigend. D.\,h.
\[a<b\iff f(a)<f(b).\]
Demnach ist die Anwendung einer streng monoton steigenden
Funktion eine Äquivalenzumformung.
\end{Satz}
\noindent\strong{Beweis.}
Zu zeigen ist $a\ne b\implies f(a)\ne f(b)$. Wenn aber $a\ne b$
ist, dann ist entweder $a<b$ und daher nach Voraussetzung
$f(a)<f(b)$ oder $b<a$ und daher nach Voraussetzung $f(b)<f(a)$.
In beiden Fällen ist $f(a)\ne f(b)$.

Seien nun $y_1,y_2$ zwei Bilder der streng monotonen Funktion $f$.
Zu zeigen ist $y_1<y_2\implies f^{-1}(y_1)<f^{-1}(y_2)$.
Stattdessen kann auch die Kontraposition
$f^{-1}(y_2)\le f^{-1}(y_1)\implies y_2\le y_1$ gezeigt werden.
Das lässt sich nun aus der strengen Monotonie von $f$ schließen:
\begin{equation}
f^{-1}(y_2)\le f^{-1}(y_1)\implies
\underbrace{f(f^{-1}(y_2))}_{=y_2}\le \underbrace{f(f^{-1}(y_1))}_{=y_1}.\;\qedsymbol
\end{equation}

\begin{Definition}[Streng monoton fallende Funktion]\mbox{}\\*
Eine Funktion $f\colon G\to\R$ heißt streng monoton fallend, wenn
\[a<b\implies f(a)>f(b)\]
für alle Zahlen $a,b\in G$ erfüllt ist.
\end{Definition}

\noindent
Ein entsprechender Satz gilt auch für diese:
\begin{Satz}[Allgemeine Äquivalenzumformung]\mbox{}\\*
Eine streng monoton fallende Funktion $f$ ist umkehrbar eindeutig.
Die Umkehrfunktion ist auch streng monoton fallend. D.\,h.
\[a<b\iff f(a)>f(b).\]
Demnach ist die Anwendung einer streng monoton fallenden
Funktion eine Äquivalenzumformung bei der sich das
Relationszeichen umdreht.
\end{Satz}

\noindent
Tatsächlich haben wir schon streng monoton steigende Funktionen
kennengelernt. Z.\,B. ist \eqref{lt-add} nichts anderes als die strenge
Monotonie für $f(x):=x+c$. Und \eqref{lt-mul-nat} ist die strenge
Monotonie für $f(x):=nx$.

Die Funktion $f\colon\R\to\R$ mit $f(x):=x^2$ ist nicht streng monoton
steigend. Zum Beispiel ist $-4<-2$, aber $16=f(-4)>f(-2)=4$. Auch
ist die Funktion nicht streng monoton fallend, denn $2<4$,
aber $4=f(2)<f(4)=16$. Schränkt man $f$
auf den Definitionsbereich $\R_{>0}$ ein, so ergibt sich jedoch eine
streng monoton steigende Funktion. Das lässt sich wie folgt zeigen.

Nach Voraussetzung sind $a,b\in\R_{>0}$, d.\,h. $a,b>0$.
Also kann gemäß \eqref{lt-mul-real} einerseits mit $a$
und andererseits mit $b$ multipliziert werden:
\[
a<b\iff\begin{Bmatrix}
a^2<ab\\
ab<b^2
\end{Bmatrix}
\iff a^2<ab<b^2.
\]




\chapter{Zahlenbereiche}

\section{Die natürlichen Zahlen}

\subsection{Modelle der natürlichen Zahlen}

Zahlen spielen in den Mathematik eine maßgebliche Rolle, und dies nicht
nur bei der quantitativen Erfassung von Größen, sondern auch bei der
logischen Klärung der Gegenstände. Zum Beispiel
setzen wesentliche Teile der Analysis, der linearen Algebra und der
Stochastik den Begriff der reellen Zahlen voraus. Und dies sind die
Kerngebiete der Mathematik, die das Fundament für die Natur- und
Ingenieurwissenschaften bilden.

Den auf den Zahlenbereichen definierten arithmetischen Operationen
wohnen bestimmte Rechenregeln inne. Anfangs mag man diese Regeln im
Rahmen einer Theoriefindung klären und daraufhin zu Grundregeln gelangen,
die man axiomatisch voraussetzt. Stattdessen will ich die Zahlenbereiche
aber sogleich vermittels Mengenlehre modellieren und aufzeigen, dass
die Modelle die Grundregeln erfüllen. Die Zahlenbereiche werden dabei
schrittweise aufeinander aufgebaut.

Wir starten mit den natürlichen Zahlen.

\begin{Definition}[Peano-Axiome]\newlinefirst
Eine Menge $\N$ ist als Struktur $(\N,0,s)$ mit $0\in\N$ und $s\colon\N\to\N$
ein Modell der \emph{natürlichen Zahlen}, wenn gilt
\[\begin{array}{@{}ll@{}}
\text{(P1)} & \text{$s$ ist injektiv},\\[3pt]
\text{(P2)} & \forall n\in\N\colon s(n)\ne 0,\\[3pt]
\text{(P3)} & A(0)\land (\forall n\in\N\colon A(n)\cond A(s(n)))\cond (\forall n\in\N\colon A(n)).
\end{array}\]
\end{Definition}
Zusätzlich muss, wie bereits diskutiert, der Rekursionssatz vorausgesetzt
werden, um die arithmetischen Operationen definieren zu können. Das
Axiom (P3), das das Prinzip der Induktion beschreibt, habe ich als Schema
gefasst, um in der Logik erster Stufe verbleiben zu können.
Es vermittelt zu jeder Aussageform $A(n)$ ein Axiom.
Die ursprüngliche Fassung der Peano"=Axiome substituiert $A$ allerdings
gegen eine Prädikatvariable und setzt somit die Logik zweiter
Stufe voraus.

\begin{Definition}[Addition natürlicher Zahlen]%
\label{def:N-Addition}\newlinefirst
Die \emph{Addition} zweier natürlicher Zahlen ist rekursiv definiert als
\[a+0 := a,\qquad a+s(b) := s(a+b).\]
\end{Definition}
\begin{Definition}[Multiplikation natürlicher Zahlen]\newlinefirst
Die \emph{Multiplikation} zweier natürlicher Zahlen ist rekursiv definiert als
\[a\cdot 0 := 0,\qquad a\cdot s(b) := a\cdot b + a.\]
\end{Definition}

\noindent
Diese rekursiven Bildungsvorschriften lassen sich direkt so im Computer
implementieren. Gleichwohl ist das Prozedere außerordentlich
ineffizient und somit praktisch nur für sehr kleine Zahlen brauchbar.

\begin{Definition}[Ordnung der natürlichen Zahlen]\newlinefirst
Die \emph{Ordnung} der natürlichen Zahlen ist definiert als
\[a\le b\,:\bicond\,\exists n\in\N\colon a+n = b.\]
\end{Definition}

\noindent
Werden die natürlichen Zahlen als formales System dargestellt, in dem
gerechnet werden kann, entfällt der Rekursionssatz vorläufig aus der
Betrachtung. Dafür werden Addition und Multiplikation axiomatisch erklärt.
Man nennt dieses System die \emph{Peano"=Arithmetik}, kurz PA. Beschränkt
man sich dabei auf die intuitionistische Logik, spricht man von der
\emph{Heyting"=Arithmetik}, kurz HA.
\begin{Definition}[Peano-Arithmetik]\newlinefirst
Die Struktur $(\N,0,s,+,\cdot,\le)$ sei ein Modell der
\emph{Peano"=Arithmetik}, wenn $(\N,0,s)$ die Peano"=Axiome
erfüllt und zusätzlich für alle $a,b\in\N$ gilt
\[\begin{array}{@{}l@{\quad\;\;}l@{\quad\;\;}l@{}}
a+0 = a, & a+s(b) = s(a+b), &
a\le b \bicond \exists n\in\N\colon a+n = b.\\[3pt]
a\cdot 0 = 0, & a\cdot s(b) = a\cdot b + a.
\end{array}\]
\end{Definition}

\noindent
Für die natürlichen Zahlen finden sich diverse Modelle, darunter jenes
bereits diskutierte, das sie als endliche Ordinalzahlen darstellt.
Dahingehend tut sich die Frage auf, ob jedes dieser Modelle
gleichberechtigt ist.

\begin{Definition}[Isomorphismus zwischen Peano-Strukturen]%
\label{def:Isomorphismus-Peano}\newlinefirst
Eine Bijektion $f\colon\N\to\N'$ ist ein Isomorphismus zwischen
zwei Peano"=Strukturen $(\N,0,s)$ und $(\N',0',s')$, wenn
$f(0)=0'$ und $f(s(n))=s'(f(n))$ gilt.
\end{Definition}

\noindent
Man rechnet unschwer nach, dass die Umkehrabbildung eines Isomorphismus'
ebenfalls ein Isomorphismus ist.

\begin{Satz}[Isomorphiesatz von Dedekind]\newlinefirst
Je zwei Modelle der natürlichen Zahlen sind isomorph.
\end{Satz}
\begin{Beweis}
Es seien $(\N,0,s)$, $(\N',0',s')$ zwei Modelle der natürlichen Zahlen.
Vermittels des Rekursionssatzes definiert man
\[\begin{array}{lr@{\;}lr@{\;}l}
f\colon\N\to\N', & f(0) &:= 0', & f(s(n)) &:= s'(f(n)),\\[3pt]
g\colon\N'\to\N, & g(0') &:= 0, & g(s'(n)) &:= s(g(n)),\\[3pt]
h\colon\N\to\N, & h(0) &:= 0, & h(s(n)) &:= s(h(n)).
\end{array}\]
Man rechnet unschwer nach, dass $g\circ f = h$ und $h = \id$ gilt.
Analog erhält man $f\circ g = \id$, womit $f,g$ bijektiv sind.
Und per Definition sind $f,g$ Isomorphismen zwischen Peano"=Strukturen.\,\qedsymbol
\end{Beweis}

\noindent
Man macht sich außerdem klar, dass die rekursive Definition des
Isomorphismus in Def. \ref{def:Isomorphismus-Peano} enthalten ist.
Das heißt, der Isomorphismus ist zudem noch eindeutig bestimmt.
Insofern es also nur einen, und somit einen ausgezeichneten Isomorphismus
gibt, kann man sagen, dass je zwei Modelle der natürlichen Zahlen
kanonisch isomorph sind.

\begin{Satz}
Mit $1:=s(0)$ gilt $s(a)=a+1$.
\end{Satz}
\begin{Beweis}
Es findet sich $a+1 = a+s(0) = s(a+0) = s(a)$.\,\qedsymbol
\end{Beweis}

\begin{Satz}[Assoziativgesetz der Addition]\newlinefirst
Für alle $a,b,c\in\N$ gilt $(a+b)+c = a+(b+c)$.
\end{Satz}
\begin{Beweis}
Induktion über $c$. Im Anfang $c=0$ gilt
\[(a+b)+c = (a+b)+0 = a+b = a+(b+0) = a+(b+c).\]
Mit der Induktionsvoraussetzung $(a+b)+c = a+(b+c)$ findet sich
\[(a+b)+s(c) = s((a+b)+c) \stackrel{\mathrm{IV}}= s(a+(b+c))
= a+s(b+c) = a+(b+s(c)).\,\qedsymbol\]
\end{Beweis}

\begin{Satz}[Neutrales Element der Addition]%
\label{nat-zero}\newlinefirst
Für alle $a\in\N$ gilt $0+a = a+0 = a$.
\end{Satz}
\begin{Beweis}
Per Definition gilt $a+0=a$. Zu $0+a$ per Induktion über $a$.
Im Anfang $a=0$ ist $0+a = 0$ per Definition. Zum Schritt.
Induktionsvoraussetzung sei $0+a=a$. Man findet
\[0+s(a) = s(0+a) \stackrel{\mathrm{IV}}= s(a).\,\qedsymbol\]
\end{Beweis}

\begin{Satz}[Kommutativgesetz der Addition]\newlinefirst
Für alle $a,b\in\N$ gilt $a+b = b+a$.
\end{Satz}
\begin{Beweis}
Zunächst $a+1 = 1+a$ per Induktion über $a$. Im Anfang $a=0$ gilt
die Aussage gemäß Satz \ref{nat-zero}. Zum Schritt. Man findet
\[1+s(a) = s(1+a) \stackrel{\mathrm{IV}}= s(a+1) = a+s(1).\]
Nun $a+b=b+a$ per Induktion über $b$. Der Fall $b=0$
gilt gemäß Satz \ref{nat-zero}. Anfang sei $b=1$. Dieser
wurde zuvor gezeigt. Zum Schritt findet sich
\begin{align*}
s(b) + a &= (b+1)+a = b+(1+a) = b+(a+1) = b+s(a)\\
&= s(b+a)\stackrel{\mathrm{IV}}= s(a+b) = a+s(b).\,\qedsymbol
\end{align*}
\end{Beweis}

\begin{Satz}[Distributivgesetz der Multiplikation]\newlinefirst
Für alle $a,b,c\in\N$ gilt $(a+b)c = ac+bc$.
\end{Satz}
\begin{Beweis} Induktion über $c$. Im Anfang $c=0$ resultieren
beide Seiten der Gleichung im Wert~$0$. Zum Schritt findet sich
\[(a+b)s(c) = (a+b)c+(a+b) \stackrel{\mathrm{IV}}= ac+bc+a+b = ac+a+bc+b
= as(c) + bs(c).\qedsymbol\]
\end{Beweis}

\begin{Satz}\label{nat-mul-zero}
Für alle $a\in\N$ gilt $0a=0$.
\end{Satz}
\begin{Beweis}
Induktion über $a$. Im Anfang $a=0$ ist per Definition $0a=0$.
Der Schritt ist
\[0s(a) = 0a+0 \stackrel{\mathrm{IV}}= 0+0 = 0.\]
\end{Beweis}

\begin{Satz}[Neutrales Element der Multiplikation]%
\label{nat-mul-one}\newlinefirst
Für alle $a\in\N$ gilt $a\cdot 1=a$ und $1\cdot a=a$.
\end{Satz}
\begin{Beweis}
Die Formel $a\cdot 1 = a$ folgt unmittelbar aus der Definition
und bereits bekannten Regeln.

Die Formel $1\cdot a = a$ per Induktion über $a$. Im Anfang $a=0$ folgt
die Regel unmittelbar aus der Definition. Der Schritt ist
\[1s(a) = 1a+1 \stackrel{\mathrm{IV}}= a+1 = s(a).\,\qedsymbol\]
\end{Beweis}

\begin{Satz}[Kommutativgesetz der Multiplikation]\newlinefirst
Für alle $a,b\in\N$ gilt $ab=ba$.
\end{Satz}
\begin{Beweis}
Induktion über $b$. Im Anfang $b=0$ gilt die Regel gemäß
Lemma \ref{nat-mul-zero}. Der Schritt ist
\[as(b) = ab+a \stackrel{\mathrm{IV}}= ba+a = ba + 1a = (b+1)a = s(b)a.\,\qedsymbol\]
\end{Beweis}

\begin{Satz}
Aus $a\le b$ folgt $a+c\le b+c$.
\end{Satz}
\begin{Beweis}
Mit der Prämisse liegt ein $n$ mit $a+n = b$ vor. Somit folgt
$a+c+n = b+c$, womit $n$ auch ein Zeuge für $a+c\le b+c$ ist.\,\qedsymbol
\end{Beweis}

\section{Die ganzen Zahlen}

\subsection{Konstruktion}

\begin{Definition}[Ganze Zahlen]\newlinefirst
Auf $\N_0\times\N_0$ wird die folgende Äquivalenzrelation definiert:
\[(x,y)\sim (x',y')\,:\bicond\, x+y' = x'+y.\]
Die Quotientenmenge $\Z := (\N_0\times\N_0)/{\sim}$ nennt man
die \emph{ganzen Zahlen}.
\end{Definition}

\begin{Satz}[Ring der ganzen Zahlen]\newlinefirst
Die Operationen
\begin{align*}
[(x,y)]+[(x',y')] &:= [(x+x',y+y')],\\
[(x,y)]\cdot [(x',y')] &:= [(xx'+yy',xy'+x'y)]
\end{align*}
sind auf $\Z$ wohldefiniert und $(\Z,+,\cdot)$ bildet einen
kommutativen unitären Ring.
\end{Satz}
\strong{Beweis.} Wohldefiniert heißt, dass das Ergebnis der Operationen
nicht von den gewählten Repräsentanten der Argumente abhängig ist.
Sei dazu $(x,y)\sim(a,b)$ und
$(x',y')\sim (a',b')$. Zu zeigen ist nun
\[(x+x',y+y')\sim (a+a',b+b'),\]
was laut Definition zu
\[(x+x')+(b+b') = (a+a')+(y+y').\]
äquivalent ist. Gemäß Voraussetzung ist $x+b=a+y$ und $x'+b'=a'+y'$.
Man bekommt damit auf der linken Seite
\[x+x'+b+b' = x+b+x'+b' = a+y+a'+y',\]
was wiederum mit der rechten Seite übereinstimmt.

Mit der Multiplikation verhält es sich etwas komplizierter.
Zu Vereinfachung wird zunächst gezeigt:
\begin{gather*}
[(x,y)]\cdot [(x',y')] = [(a,b)]\cdot [(x',y')],\\
\iff (xx'+yy',xy'+yx')\sim (ax'+by',ay'+bx')\\
\iff xx'+yy' + ay'+bx' = ax'+by' + xy'+yx'\\
\iff (x+b)x' + (a+y)y' = (a+y)x' + (x+b)y'.
\end{gather*}
Diese Gleichung ist gemäß Voraussetzung $(x,y)\sim (a,b)$
bzw. $x+b=a+y$ erfüllt.

Analog bestätigt man
\[[(a,b)]\cdot [(x',y')] = [(a,b)]\cdot [(a',b')].\]
Gemäß Transitivität ergibt sich somit
\[[(x,y)]\cdot [(x',y')] = [(a,b)]\cdot [(a',b')].\]
Es ist nun zu bestätigen, dass $(\Z,+)$ eine kommutative Gruppe ist.
Das Assoziativgesetz:
\begin{gather*}
([(x,y)]+[(x',y')])+[(x'',y'')]
= [(x+x',y+y')] + [(x'',y'')]\\
= [(x+x'+x'',y+y'+y'')]
= [(x,y)]+[(x'+x'',y'+y'')]\\
= [(x,y)]+([(x',y')]+[(x'',y'')]).
\end{gather*}
Das neutrale Element ist $[(0,0)]$:
\[[(x,y)]+[(0,0)] = [(x+0,y+0)] = [(x,y)].\]
Das inverse Element zu $[(x,y)]$ ist $[(y,x)]$, denn es gilt
\begin{gather*}
[(x,y)]+[(y,x)] = [(x+y,y+x)] = [(0,0)]\\
\iff (x+y,y+x)\sim (0,0)\iff x+y+0 = y+x+0.
\end{gather*}
Das Kommutativgesetz:
\[[(x,y)]+[(x',y')] = [(x+x',y+y')] = [(x'+x,y'+y)]
= [(x',y')]+[(x,y)].\]
Es ist nun zu bestätigen, dass $(\Z,\cdot)$ ein kommutatives
Monoid bildet. Das Assoziativgesetz:
\begin{gather*}
([(x,y)]\cdot [(x',y')])\cdot [(x'',y'')]
= [(xx'+yy',xy'+x'y)]\cdot [(x'',y'')]\\
= [(xx'x''+x''yy'+xy'y''+x'yy'',\;
xx'y''+yy'y''+xx''y'+x'x''y)]\\
= [(x,y)]\cdot [(x'x''+y'y'',x'y''+x''y')]
= [(x,y)]\cdot ([(x',y')]\cdot [(x'',y'')]).
\end{gather*}
Das Kommutativgesetz:
\begin{gather*}
[(x,y)]\cdot [(x',y')] = [(xx'+yy',\;xy'+yx')]\\
= [(x'x+y'y,\;x'y+xy')] = [(x',y')]\cdot [(x,y)].
\end{gather*}
Das neutrale Element ist $[(1,0)]$, denn es gilt
\[[(x,y)]\cdot [(1,0)] = [(x\cdot 1+y\cdot 0,\;1\cdot y+x\cdot 0)]
= [(x,y)].\]
Schließlich ist noch das Distributivgesetz zu bestätigen.
Man findet
\begin{gather*}
[(a,b)]\cdot ([(x,y)]+[(x',y')])
= [(a,b)]\cdot [(x+x',y+y')]\\
= [(ax+ax'+by+by',\;ay+ay'+bx+bx')]\\
= [(ax+by,ay+bx)]+[(ax'+by',ay'+bx')]\\
= [(a,b)]\cdot [(x,y)] + [(a,b)]\cdot [(x',y')].
\end{gather*}
Somit sind alle Axiome bestätigt.\;\qedsymbol

\begin{Definition}[Monoidhomomorphismus]\newlinefirst
Seien $(M,+)$ und $(M',+')$ zwei Monoide. Eine Abbildung
$\varphi\colon M\to M'$ heißt Monoidhomomorphismus, wenn
für alle $a,b\in M$ gilt
\[\varphi(a+b) = \varphi(a)+\varphi(b)\]
und $\varphi(0)=0'$ ist.
\end{Definition}
Einen injektiven Homomorphismus nennt man Monomorphismus. Ein
Monomorphismus charakterisiert eine Einbettung einer Struktur als
Unterstruktur einer anderen.

\begin{Satz}[Einbettung der natürlichen Zahlen in die ganzen]%
\label{embedding-nat-int}\mbox{}\\*
Die Abbildung $\varphi\colon\N_0\to\Z$ mit $\varphi(n):=[(n,0)]$
ist ein Monoidmonomorphismus.
\end{Satz}
\strong{Beweis.} Es ergibt sich
\[\varphi(a+b) = [(a+b,0)] = [(a,0)]+[(b,0)] = \varphi(a)+\varphi(b).\]
Außerdem ist $\varphi(0)=[(0,0)]$, und $[(0,0)]$
ist das neutrale Element von $(\Z,+)$.

Schließlich ist noch die Injektivität zu prüfen:
\begin{gather*}
[(a,0)] = \varphi(a) = \varphi(b)  = [(b,0)]\iff (a,0)\sim (b,0)\\
\iff a+0 = b+0 \iff a=b.\;\qedsymbol
\end{gather*}
Anstelle von $\varphi(n)=[(n,0)]$ darf man daher einfach schreiben
$n=[(n,0)]$. Außerdem definiert man $a-b:=a+(-b)$. Daraus
ergibt sich nun
\[[(x,y)] = [(x,0)]+[(0,y)] = [(x,0)] - [(y,0)] = x-y.\]
Die umständliche Schreibweise $[(x,y)]$ wird ab jetzt nicht
mehr benötigt.

\begin{Definition}[Totalordnung der ganzen Zahlen]\newlinefirst
Auf $\Z$ wird die folgende strenge Totalordnung definiert:
\[[(x,y)] < [(x',y')]\,:\bicond\, x+y'<x'+y.\]
\end{Definition}

\begin{Satz}[Einbettung der Totalordnung]\newlinefirst
Die Abbildung $\varphi$ aus Satz \ref{embedding-nat-int}
genügt der Forderung
\[a<b\,\cond\, \varphi(a)<\varphi(b).\]
\end{Satz}
\strong{Beweis.} Nach den Definitionen ist
\[\varphi(a)<\varphi(b)\iff [(a,0)]<[(b,0)]\iff a+0<0+b\iff a<b.\;\qedsymbol\]


\section{Die rationalen Zahlen}

\subsection{Konstruktion}

\begin{Definition}[Rationale Zahlen]\newlinefirst
Auf $\Z\times\N_1$ wird die folgende Äquivalenzrelation definiert:
\[(x,y)\sim (x',y')\,:\bicond\, xy' = x'y.\]
Die Quotientenmenge $\Q := (\Z\times\N_1)/{\sim}$ nennt man
die \emph{rationalen Zahlen}.
\end{Definition}
Für die Äquivalenzklasse $[(x,y)]$ schreibt man $\frac{x}{y}$.

\begin{Satz}[Körper der rationalen Zahlen]\newlinefirst
Die Operationen
\[\frac{x}{y}+\frac{x'}{y'} := \frac{xy'+x'y}{yy'},
\qquad\frac{x}{y}\cdot \frac{x'}{y'} := \frac{xx'}{yy'}\]
sind auf $\Q$ wohldefiniert und $(\Q,+,\cdot)$ bildet einen Körper.
\end{Satz}
\strong{Beweis.} Wohldefiniert bedeutet, dass das Ergebnis der
Operationen nicht von den gewählten Repräsentanten der Argumente
abhängig ist. Sei dazu $(a,b)\sim (x,y)$ und
$(a',b')\sim (x',y')$. Zu zeigen ist nun
\begin{align*}
&(ab'+a'b,bb')\sim (xy'+x'y,yy'),\\
&\iff (ab'+a'b)(yy') = (xy'+x'y)(bb')\\
&\iff ab' yy' + a'byy' = xy'bb'+x'ybb'.
\end{align*}
Substituiert man $ay=xb$ und $a'y'=x'b'$ auf
der linken Seite der Gleichung, ergibt sich die rechte Seite.
Zu zeigen ist weiterhin
\begin{align*}
(aa',bb')\sim (xx',yy')
\iff aa'yy' = xx'bb'.
\end{align*}
Wieder wird linke Seite der Gleichung über $ay=xb$
und $a'y'=x'b'$ in die rechte Seite überführt.
Die Wohldefiniertheit der Operationen ist damit bestätigt.

Es bleibt zu prüfen, dass $(\Q,+,\cdot)$ allen Körperaxiomen genügt.
Das neutrale Element der Addition ist $0/1$, denn es gilt
\[\frac{x}{y}+\frac{0}{1} = \frac{x\cdot 1+0\cdot y}{y\cdot 1} = \frac{x}{y}.\]
Das neutrale Element der Multiplikation ist $1/1$, denn es gilt
\[\frac{x}{y}\cdot\frac{1}{1} = \frac{x\cdot 1}{y\cdot 1} = \frac{x}{y}.\]
Die Assoziativität der Addition ergibt sich ohne größere Umstände:
\begin{align*}
\bigg(\frac{x}{y}+\frac{x'}{y'}\bigg)+\frac{x''}{y''}
&= \frac{xy'+x'y}{yy'} + \frac{x''}{y''}
= \frac{xy'y''+x'yy''+x''yy'}{yy'y''},\\
\frac{x}{y}+\bigg(\frac{x'}{y'}+\frac{x''}{y''}\bigg)
&= \frac{x}{y}+\frac{x'y''+x''y'}{y'y''}
= \frac{xy'y''+x'yy''+x''yy'}{yy'y''}.
\end{align*}
Die Assoziativität der Multiplikation ist etwas einfacher:
\[\bigg(\frac{x}{y}\cdot\frac{x'}{y'}\bigg)\cdot\frac{x''}{y''}
= \frac{xx'}{yy'}\cdot\frac{x''}{y''} = \frac{xx'x''}{yy'y''}
= \frac{x}{y}\cdot\frac{x'x''}{y'y''}
= \frac{x}{y}\cdot\bigg(\frac{x'}{y'}\cdot\frac{x''}{y''}\bigg).\]
Das Kommutativgesetz der Addition:
\[\frac{x}{y}+\frac{x'}{y'} = \frac{xy'+x'y}{yy'}
= \frac{x'y+xy'}{y'y}
= \frac{x'}{y'}+\frac{x}{y}.\]
Das Kommutativgesetz der Multiplikation:
\[\frac{x}{y'}\cdot\frac{x'}{y'}
= \frac{xx'}{yy'} = \frac{x'x}{y'y}
= \frac{x'}{y'}\cdot\frac{x}{y}.\]
Das additiv inverse Element zu $x/y$ ist $(-x)/y$, denn es gilt
\[\frac{x}{y}+\frac{-x}{y} = \frac{xy+(-x)y}{y^2}
= \frac{0}{y^2} = \frac{0}{1}.\]
Das multiplikativ inverse Element zu $x/y$ mit $x\ne 0$
ist $y/x$, denn es gilt
\[\frac{x}{y}\cdot\frac{y}{x} = \frac{xy}{xy} = \frac{1}{1}.\]
Schließlich findet bestätigt man noch das Distributivgesetz:
\begin{align*}
&\frac{a}{b}\cdot\bigg(\frac{x}{y}+\frac{x'}{y'}\bigg)
= \frac{a}{b}\cdot\frac{xy'+x'y}{yy'}
= \frac{axy'+ax'y}{byy'},\\
&\frac{ax}{by}+\frac{ax'}{by'}
= \frac{axby'+ax'by}{byby'}
= \frac{b}{b}\cdot\frac{axy'+ax'y}{byy'}.
\end{align*}
Hierbei beachtet man, dass $b/b=1/1$ das multiplikativ
neutrale Element ist.\;\qedsymbol

\begin{Definition}[Ringhomomorphismus]\newlinefirst
Seien $(R,+,*)$ und $(R',+',*')$ zwei Ringe. Die Abbildung
$\varphi\colon R\to R'$ heißt \emph{Ringhomomorphismus}, wenn für alle
$a,b\in R$ gilt:
\begin{align*}
\varphi(a+b) = \varphi(a)+'\varphi(b),\qquad
\varphi(a*b) = \varphi(a)*'\varphi(b).
\end{align*}
Besitzt $R$ ein Einselement $1$ und $R'$ ein Einselement $1'$,
dann nennt man $\varphi$ \emph{Eins"=erhaltend}, wenn $\varphi(1)=1'$ ist.
\end{Definition}
Einen injektiver Homomorphismus wird Monomorphismus genannt. Ein
Monomorphismus charakterisiert eine Einbettung einer Unterstruktur
in eine andere Struktur.

\newpage
\begin{Satz}[Einbettung der ganzen Zahlen in die rationalen]\newlinefirst
Sei $\varphi\colon\Z\to\Q$ mit $\varphi(z):=z/1$. Die
Abbildung $\varphi$ ist ein Eins"=erhaltender Ringmonomorphismus.
\end{Satz}
\strong{Beweis.} Die Erhaltung des Einselements ergibt sich
trivial. Ferner findet man
\[\varphi(a+b) = \frac{a+b}{1} = \frac{a\cdot 1+b\cdot 1}{1\cdot 1}
= \frac{a}{1}+\frac{b}{1} = \varphi(a)+\varphi(b)\]
und
\[\varphi(ab) = \frac{ab}{1} = \frac{ab}{1\cdot 1} = \frac{a}{1}\cdot\frac{b}{1}
= \varphi(a)\cdot\varphi(b).\;\qedsymbol\]
Gemäß der Einbettung können wir die ganze Zahl $z$ ab jetzt
mit der rationalen Zahl $z/1$ identifizieren. Das heißt, man schreibt
einfach $z=z/1$ statt $\varphi(z)=z/1$.

\begin{Definition}[Division rationaler Zahlen]\newlinefirst
Wie in jedem Körper ist die Division für $a,b\in\Q$
definiert als $a/b := ab^{-1}$.
\end{Definition}
Die Division ist also gerade die Multiplikation des Kehrwertes
des Nenners:
\[\frac{x}{y}/\frac{x'}{y'} = \frac{x}{y}\cdot\frac{y'}{x'}.\]
Die Division muss natürlich mit der Notation für rationale Zahlen
kompatibel sein, sonst dürfte man nicht die gleiche Schreibweise
verwenden. Zur Unterscheidung schreiben wir Division für einen
Augenblick mit Doppelstrich als $a\doubleslash b$. Man findet
\[\frac{x}{y} = \frac{x}{1}\cdot\frac{1}{y}
= \frac{x}{1}\doubleslash\frac{y}{1} = x\doubleslash y.\]
Tatsächlich führt beides zum gleichen Ergebnis.

Da die rationalen Zahlen einen Körper bilden, gilt $a/a=aa^{-1}=1$
für jede rationale Zahl $a$.

\begin{Satz}[Addition, Subtraktion, Multiplikation von Brüchen]\newlinefirst
Seien $a,b,c,d$ rationale Zahlen mit $b\ne 0$ und $d\ne 0$. Es gilt
\[\frac{a}{b}+\frac{c}{d} = \frac{ad+bc}{bd},
\qquad \frac{a}{b}-\frac{c}{d} = \frac{ad-bc}{bd},
\qquad \frac{a}{b}\cdot\frac{c}{d} = \frac{ac}{bd}.\]
\end{Satz}
Der Beweis wird dem Leser überlassen.

\newpage
\subsection{Beträge rationaler Zahlen}

\begin{Definition}[Signum]\newlinefirst
Für jede rationale Zahl $x$ setzt man
\[\sgn(x) := \left\{\begin{array}{@{}rl@{}}
1, & \mathrm{wenn}\; x>0,\\
0, & \mathrm{wenn}\; x=0,\\
-1, & \mathrm{wenn}\;x<0.
\end{array}\right.\]
\end{Definition}
Vermittels Fallunterscheidung rechnet man unschwer nach,%
\[\sgn(xy) = \sgn(x)\sgn(y).\]
Für $x\ne 0$ ist außerdem $\sgn(x)^2=1$ unmittelbar einsichtig,
allgemein gilt demnach $\sgn(x)^2=[x\ne 0]$. Infolge
erhält man $\tfrac{1}{\sgn(x)}=\sgn(x)$ für $x\ne 0$, und somit%
\[\sgn(\tfrac{1}{x}) = \tfrac{1}{\sgn(x)} = \sgn(x),
\;\text{denn}\;1 = \sgn(1) = \sgn(x\cdot\tfrac{1}{x})
= \sgn(x)\sgn(\tfrac{1}{x}).\]

\begin{Definition}[Betrag]\newlinefirst
Man setzt $|x| := \sgn(x)\cdot x$ für jede rationale Zahl $x$.
\end{Definition}

\noindent
Aus der Definition leiten sich einige elementare Rechenregeln ab. Es folgt%
\[|xy| = \sgn(xy)xy = \sgn(x)\sgn(y)xy
= \sgn(x)x\sgn(y)y = |x||y|.\]
Multipliziert man die definierende Gleichung im Fall $x\ne 0$ auf
beiden Seiten mit $\sgn(x)$, erhält man außerdem%
\[\sgn(x)|x| = \sgn(x)^2\cdot x = 1\cdot x = x.\]
Im Fall $x=0$ ist $x=\sgn(x)|x|$ aber auch erfüllt. Dies bedeutet, dass
sich jede Zahl in ihren Betrag und ihr Vorzeichen zerlegen lässt, wobei
der Null kein Vorzeichen zukommen braucht.


\begin{Satz}[Dreiecksungleichung]\newlinefirst
Die Ungleichung $|x+y| \le |x| + |y|$ gilt für alle rationalen Zahlen $x,y$.
\end{Satz}
\begin{Beweis}
Für jede rationale Zahl $q$ gilt $q\le |q|$. Mit der Setzung $q:=xy$ erhält man
$xy\le |xy| = |x||y|$. Zusammen mit $|x+y|^2 = (x+y)^2$ sowie $|x|^2 = x^2$
und $|y|^2 = y^2$ ergibt sich somit
\begin{align*}
|x+y|^2 = x^2+2xy+y^2 \le x^2 + 2|xy| + y^2 = (|x|+|y|)^2.
\end{align*}
Weil Quadrieren auf den nichtnegativen rationalen Zahlen eine monoton
steigende Funktion ist, und somit die Ordnung erhält, gelangt man zur
gesuchten Ungleichung letztlich mit der Äquivalenz%
\[|x+y| \le |x| + |y| \,\Leftrightarrow\, |x+y|^2 \le (|x|+|y|)^2.\,\qedsymbol\]
\end{Beweis}

\newpage
\section{Die reellen Zahlen}

\subsection{Konstruktion}

\begin{Definition}[Offene Kugel]\newlinefirst
Zu $\varepsilon\in\Q_{\ge 0}$ und $q\in\Q$ nennt man
$U_\varepsilon(q):=\{p\in\Q\mid |p-q|<\varepsilon\}$
die \emph{offene Kugel} vom Radius $\varepsilon$ mit Zentrum $q$ oder
kurz die $\varepsilon$-Umgebung von $q$.
\end{Definition}

\begin{Definition}[Konvergente rationale Folge]\newlinefirst
Eine Folge $x\colon\N\to\Q$ heißt \emph{konvergent} gegen $L$, falls
\[\forall\varepsilon\in\Q_{>0}\colon\exists n_0\in\N\colon\forall n\ge n_0\colon |x_n-L| < \varepsilon.\]
\end{Definition}

\begin{Satz}
Eine Folge rationaler Zahlen konvergiert gegen höchstens einen Grenzwert.
\end{Satz}
\begin{Beweis}
Sei $x\colon\N\to\Q$ konvergent gegen $L_1,L_2$. Angenommen, es wäre $L_1\ne L_2$.
Wir wählen $\varepsilon$ so klein, dass die Umgebungen $U_\varepsilon(L_1)$
und $U_\varepsilon(L_2)$ disjunkt sind. Nun liegt ein $n_1$ vor,
so dass $x_n\in U_\varepsilon(L_1)$ für $n\ge n_1$. Entsprechend ein
$n_2$, so dass $x_n\in U_\varepsilon(L_2)$ für $n\ge n_2$. Für
$n:=\max(n_1,n_2)$ wäre $x_n$ absurderweise in beiden Umgebungen.
Also muss die Annahme falsch sein.\,\qedsymbol
\end{Beweis}

\begin{Definition}[Grenzwert]\newlinefirst
Konvergiert die Folge $x$ gegen $L$, dann definieren wir ihren \emph{Grenzwert} als
\[\lim_{n\to\infty} x_n := L.\]
\end{Definition}

\begin{Definition}[Cauchy-Folge]\newlinefirst
Eine Folge $x\colon\N\to\Q$ heißt \emph{Cauchy"=Folge}, falls
\[\forall\varepsilon\in\Q_{>0}\colon\exists n_0\in\N\colon
\forall m\ge n_0\colon\forall n\ge n_0\colon |x_m - x_n| < \varepsilon.\]
\end{Definition}

\begin{Definition}[Reelle Zahlen]\newlinefirst
Auf $C_\Q$, der Menge aller Cauchy"=Folgen auf $\Q$, definiert man
die \emph{reellen Zahlen} als die Quotientenmenge $\R:=C_\Q/{\sim}$
bezüglich der Äquivalenzrelation
\[x\sim y \;:\Leftrightarrow\; \lim_{n\to\infty}(x_n-y_n) = 0\]
und legt Addition und Multiplikation reeller Zahlen fest als
\[[x]+[y] := [x+y],\quad\;\; [x]\cdot [y] := [x\cdot y].\]
\end{Definition}

\begin{Satz}
Sind $x,y$ Cauchy"=Folgen, so ist $(x+y)_n := x_n+y_n$ ebenfalls eine.
\end{Satz}
\begin{Beweis}
Die beiden Voraussetzungen sind
\begin{gather*}
\forall\varepsilon_1\in\Q_{>0}\colon\exists n_1\in\N\colon
  \forall m\ge n_1\colon\forall n\ge n_1\colon |x_m-x_n|<\varepsilon_1,\\
\forall\varepsilon_2\in\Q_{>0}\colon\exists n_2\in\N\colon
  \forall m\ge n_2\colon\forall n\ge n_2\colon |y_m-y_n|<\varepsilon_2.
\end{gather*}
Es sei $\varepsilon\in\Q_{>0}$ fest, aber beliebig. Damit wird mit
$\varepsilon_1:=\tfrac{1}{2}\varepsilon$ und $\varepsilon_2:=\tfrac{1}{2}\varepsilon$
spezialisiert. Man setze $n_0:=\max(n_1,n_2)$, daraufhin gilt für $m\ge n_0$ und $n\ge n_0$ sowohl
$|x_m-x_n|<\tfrac{1}{2}\varepsilon$ als auch $|y_m-y_n|<\tfrac{1}{2}\varepsilon$.
Vermittels der Dreiecksungleichung findet sich nun die gesuchte Ungleichung
\[|(x+y)_m - (x+y)_n| = |(x_m-x_n) + (y_m-y_n)| \le |x_m-x_n| + |y_m-y_n| < \varepsilon.\,\qedsymbol\]
\end{Beweis}

\begin{Satz}
Jede Cauchy"=Folge ist beschränkt.
\end{Satz}
\begin{Beweis}
Sei $x$ eine Cauchy"=Folge. Spezialisiert man diese Voraussetzung mit
$\varepsilon:=1$, erhält man ein $n_0$, so dass $|x_m-x_n|<1$ für
$m\ge n_0$ und $n\ge n_0$ gelten muss. Vermittels der umgekehrten
Dreiecksungleichung folgt daraus
\[|x_m|-|x_n| \le |x_m-x_n| < 1,\;\text{also}\; |x_m| < |x_{n_0}| + 1.\]
Damit ist die Folge zumindest ab $n_0$ beschränkt. Aber für
$m\le n_0$ ist das Maximum $M:=\max_{m\le n_0} |x_m|$ schlicht eine
Schranke. Somit ist $x$ mit $|x_m| < M + 1$ für jedes $m\in\N$
eine beschränkte Folge.\,\qedsymbol
\end{Beweis}

\begin{Satz}
Sind $x,y$ Cauchy"=Folgen, so ist $(x\cdot y)_n := x_n\cdot y_n$ ebenfalls eine.
\end{Satz}
\begin{Beweis}
Mit dem Ansatz
\[(xy)_m - (xy)_n = x_m y_m - x_n y_n = (x_m y_m - x_m y_n) + (x_m y_n - x_n y_n)\]
und der Dreiecksungleichung erhält man
\[|(xy)_m - (xy)_n| \le |x_m y_m - x_m y_n| + |x_m y_n - x_n y_n|
= |x_m||y_m-y_n| + |y_n||x_m-x_n|.\]
Wir setzen wieder $n_0:=\max(n_1,n_2)$, so dass sowohl $|x_m-x_n|<\varepsilon$
als auch $|y_m-y_n|<\varepsilon$ für $m\ge n_0$ und $n\ge n_0$. Weil $x,y$
beschränkt sind, existiert außerdem eine Schranke $s$ mit $|x_m|\le s$
für jedes $m$ und $|y_n|\le s$ für jedes $n$. Ergo hat man
\[|x_m||y_m-y_n| + |y_n||x_m-x_n| < |x_m|\varepsilon + |y_n|\varepsilon
\le s\varepsilon + s\varepsilon = 2s\varepsilon.\]
Ist $\varepsilon'$ nun fest, aber beliebig, kann man mit
$\varepsilon:=\frac{\varepsilon'}{2s}$ spezialisieren. Für jedes $\varepsilon'\in\Q_{>0}$
existiert also $n_0\in\N$ mit $|(xy)_m - (xy)_n| < \varepsilon'$
für $m\ge n_0$ und $n\ge n_0$.\,\qedsymbol
\end{Beweis}

\begin{Satz}
Die Addition und Multiplikation reeller Zahlen ist wohldefiniert.
\end{Satz}
\begin{Beweis}
Es gelte $x\sim x'$ und $y\sim y'$. Zu zeigen ist $x+y\sim x'+y'$
und $x\cdot y\sim x'\cdot y'$. Zu jedem $\varepsilon$ existiert laut
den Voraussetzungen ein $n_0$ mit $|x_n-x_n'| < \varepsilon$
und $|y_n-y_n'|<\varepsilon$ für jedes $n\ge n_0$. Vermittels der
Dreiecksungleichung erhält man somit die Ungleichung
\[|(x+y)_n - (x'+y')_n| = |x_n-x_n' + y_n-y'_n|
\le |x_n-x_n'| + |y_n-y_n'| < 2\varepsilon.\]
Also unterbietet die Folge $(x+y)-(x'+y')$ jedes $\varepsilon'\in\Q_{>0}$
bezüglich $\varepsilon:=\tfrac{1}{2}\varepsilon'$, womit sie ebenfalls
gegen null konvergiert.

Zu den Produktfolgen macht man wieder den Ansatz
\[(xy)_n - (x'y')_n = x_n y_n - x_n' y_n' = (x_n y_n - x_n' y_n) + (x_n' y_n - x_n' y_n').\]
Mit der Dreiecksungleichung ergibt sich daher
\[|(xy)_n - (x'y')_n| \le |y_n||x_n-x_n'| + |x_n'||y_n-y_n'|
< |y_n|\varepsilon + |x_n'|\varepsilon \le 2s\varepsilon,\]
wobei $s$ wieder eine Schranke ist, so dass $|y_n|<s$ und $|x_n'|<s$ für jedes $n\in\N$.
Sie existiert, weil $y_n$ und $x_n'$ als Cauchy"=Folgen beschränkt sind.
Also unterbietet die Folge $xy-x'y'$ jedes $\varepsilon'\in\Q_{>0}$
bezüglich $\varepsilon:=\tfrac{\varepsilon'}{2s}$, womit sie ebenfalls
gegen null konvergiert.\,\qedsymbol
\end{Beweis}


\chapter{Ansätze zur Problemlösung}

\section{Substitution}

\subsection{Quadratische Gleichungen}%
\index{quadratische Gleichung}\index{Gleichung!quadratische}

Vorgelegt ist eine quadratische Gleichung in Normalform
\begin{equation}\index{Normalform!einer quadratischen Gleichung}
x^2+px+q = 0.
\end{equation}
Interessanterweise lässt sich der lineare Term $px$ durch Darstellung
der Gleichung über eine Translation $x=u+d$ eliminieren. Einsetzen
dieser Substitution bringt
\begin{align}
0 &= (u+d)^2+p(u+d)+q = u^2+2ud+d^2+pu+pd+q\\
&= u^2+(p+2d)u+(d^2+pd+q).
\end{align}
Setzt man nun $p+2d=0$, dann ergibt sich daraus $d=-p/2$ und somit
\begin{align}
q' &:= d^2+pd+q = \Big(-\frac{p}{2}\Big)^2-p\cdot\frac{p}{2}+q
= \frac{p^2}{4}-\frac{p^2}{2}+q\\
&= \frac{p^2}{4}-2\frac{p^2}{4}+q = -\frac{p^2}{4}+q.
\end{align}
Zu lösen ist nunmehr die quadratische Gleichung
\begin{equation}
u^2+q' = 0.
\end{equation}
Aber das ist ganz einfach, die Lösungen sind $u_1=+\sqrt{-q'}$
und $u_2=-\sqrt{-q'}$, sofern $q'\le 0$,
bzw. äquivalent $-q'\ge 0$. Wir schreiben kurz $u=\pm\sqrt{-q'}$.
Resubstitution von $u=x-d$ und $q'$ führt zu
\begin{equation}
x-d = x+\frac{p}{2} = \pm\sqrt{\frac{p^2}{4}-q} = \pm\frac{1}{2}\sqrt{p^2-4q}.
\end{equation}
Man erhält die Lösungsformel
\begin{equation}
x = -\frac{p}{2}\pm\frac{1}{2}\sqrt{p^2-4q}.
\end{equation}

\subsection{Biquadratische Gleichungen}%
\index{biquadratische Gleichung}\index{Gleichung!biquadratische}
Die biquadratische Gleichung
\begin{equation}
x^4+px^2+q = 0
\end{equation}
lässt sich über die Substitution $u=x^2$ auf die quadratische Gleichung
\begin{equation}
u^2+pu+q
\end{equation}
reduzieren. Für $p^2-4q\ge 0$ ergeben sich zwei Lösungen $u_1,u_2$,
wobei eventuell $u_1=u_2$ ist. Nun können sich bis zu vier Lösungen
für die ursprüngliche Gleichung ergeben. Das ist der Fall,
wenn $u_1\ne u_2$ und $u_1,u_2>0$. Dann
ergibt sich
\begin{equation}
x_1=\sqrt{u_1},\quad x_2=-\sqrt{u_1},\quad
x_3=\sqrt{u_2},\quad x_4=-\sqrt{u_2}
\end{equation}


\chapter{Kombinatorik}

\section{Endliche Mengen}

\subsection{Indikatorfunktion}

\begin{Definition}[Iverson-Klammer]\newlinefirst
Für eine Aussage $A$ der klassischen Aussagenlogik definiert man
\[[A] := \begin{cases}
1 &\text{wenn}\;A,\\
0 &\text{sonst}.
\end{cases}\]
\end{Definition}

\begin{Satz}\label{iverson-basic-rules}
Es gilt
\begin{gather*}
[A\land B] = [A][B],\\
[A\lor B] = [A]+[B]-[A][B],\\
[\neg A] = 1-[A],\\
[A\to B] = 1-[A](1-[B]).
\end{gather*}
\end{Satz}
\begin{Beweis} Trivial mittels Wertetabelle.\,\qedsymbol
\end{Beweis}

\begin{Satz}\label{indicator-set-op}
Für die Indikatorfunktion $1_M(x):=[x\in M]$ gilt
\begin{gather*}
1_{A\cap B} = 1_A 1_B,\\
1_{A\cup B} = 1_A + 1_B - 1_{A\cap B}.
\end{gather*}
\end{Satz}
\begin{Beweis}
Gemäß Satz \ref{iverson-basic-rules} gelten die
Rechnungen
\begin{align*}
1_{A\cap B}(x) = [x\in A\cap B]
= [x\in A\land x\in B] = [x\in A][x\in B] = 1_A(x)1_B(x)
\end{align*}
und
\begin{align*}
1_{A\cup B}(x) &= [x\in A\cup B] = [x\in A\lor x\in B]
= [x\in A] + [x\in B] - [x\in A][x\in B]\\
&= 1_A(x) + 1_B(x) - 1_{A\cap B}(x).\,\qedsymbol
\end{align*}
\end{Beweis}

\begin{Satz}
Für endliche Mengen $A,B$ gilt $|A\cup B| = |A|+|B|-|A\cap B|$.
\end{Satz}
\begin{Beweis}
Gemäß Satz \ref{indicator-set-op} darf man rechnen
\begin{align*}
|A\cup B| &= \sum_{x\in G} 1_{A\cup B}(x)
= \sum_{x\in G} (1_A(x) + 1_B(x) - 1_{A\cap B}(x))\\
&= \sum_{x\in G} 1_A(x) + \sum_{x\in G} 1_B(x) - \sum_{x\in G} 1_{A\cap B}(x)
= |A| + |B| - |A\cap B|.\,\qedsymbol
\end{align*}
\end{Beweis}

\begin{Satz}
Für endliche Mengen $A,B$ mit $A\subseteq B$ gilt $|A|\le |B|$.
\end{Satz}
\begin{Beweis}
Mithilfe der Indiktorfunktion findet sich
\begin{gather*}
A\subseteq B \iff (\forall x\colon 1_A(x)\le 1_B(x))
\iff (\forall x\colon 0\le 1_B(x)-1_A(x))\\
\implies 0\le \sum_{x\in B}(1_B(x)-1_A(x))
= \sum_{x\in B} 1_B(x) - \sum_{x\in B} 1_A(x) = |B| - |A|\\
\implies |A|\le |B|.\,\qedsymbol
\end{gather*}
\end{Beweis}

\newpage
\subsection{Endliche Abbildungen}

\begin{Satz}[Anzahl der Abbildungen]\newlinefirst
Seien $X,Y$ endliche Mengen mit $|X| = k$ und $|Y|=n$. Die Menge
der Abbildungen $X\to Y$ enthält $n^k$ Elemente.
\end{Satz}
\begin{Beweis}
Induktion über $k$. Im Anfang $k=0$ ist $X=\emptyset$. Es gibt genau
eine Abbildung $\emptyset\to Y$, nämlich die leere Abbildung.
Gleichermaßen ist $n^0=1$.

Zum Induktionsschritt. Induktionsvoraussetzung sei die Gültigkeit
für $k-1$. Es  sei $|X|=k$ und $|Y|=n$. Gesucht ist die Anzahl
der Möglichkeiten zur Festlegung der Abbildung $f\colon X\to Y$.
Sei $x\in X$ fest. Für die Festlegung $f(x)=y$ bestehen nun genau $n$
Möglichkeiten, nämlich so viele, wie es Elemente $y\in Y$ gibt.
Für die Festlegung der übrigen Werte betrachtet man $f$ als Abbildung%
\[f\colon X\setminus\{x\}\to Y,\]
von denen es laut Voraussetzung $n^{k-1}$ gibt. Wir haben also
$n$ mal $n^{k-1}$ Möglichkeiten, das sind $n^k$.\,\qedsymbol
\end{Beweis}

\begin{Satz}[Anzahl der Bijektionen]\newlinefirst
Seien $X,Y$ endliche Mengen, wobei $|X|=|Y|=n$ gelte.
Die Menge der Bijektionen $X\to Y$ enthält $n!$ Elemente.
\end{Satz}
\begin{Beweis}
Induktion über $n$. Im Anfang $n=0$ ist $X=\emptyset$
und $Y=\emptyset$. Es existiert genau eine Bijektion
$\emptyset\to\emptyset$, nämlich die leere Abbildung.
Bei der Fakultät gilt ebenfalls $0! = 1$ laut
Def. \ref{def:factorial}.

Zum Induktionsschritt. Induktionsvoraussetzung sei die Gültigkeit für
$n-1$. Es sei $|X|=n$. Gesucht ist die Anzahl der Möglichkeiten zur
Festlegung der Bijektion $f\colon X\to Y$. Sei $x\in X$ fest. Für
die Festlegung $f(x)=y$ bestehen genau $n$ Möglichkeiten, nämlich
so viele, wie es Elemente $y\in Y$ gibt. Bei der Festlegung der übrigen
Werte entfällt $y$ aufgrund der Injektivität von $f$. Für die Festlegung
betrachtet man $f$ daher als Bijektion%
\[f\colon X\setminus\{x\}\to Y\setminus\{y\},\]
von denen es laut Voraussetzung $(n-1)!$ gibt. Wir haben also
$n$ mal $(n-1)!$ Möglichkeiten, was gemäß Def. \ref{def:factorial}
gleich $n!$ ist.\,\qedsymbol
\end{Beweis}

\begin{Satz}[Anzahl der Injektionen]\newlinefirst
Seien $X,Y$ endliche Mengen, wobei $|X|=k$ und $|Y|=n$ gelte.
Die Menge der Injektionen $X\to Y$ enthält $n^{\underline k}$
Elemente.
\end{Satz}
\begin{Beweis}
Induktion über $k$. Im Anfang $k=0$ ist $X=\emptyset$. Es gibt
genau eine Injektion $\emptyset\to Y$, nämlich die leere Abbildung.
Gleichermaßen gilt $n^{\underline 0} = 1$.

Zum Schritt. Voraussetzung sei die Gültigkeit
für $k-1$. Es sei $|X|=k$ und $|Y|=n$. Gesucht ist die Anzahl
der Möglichkeiten zur Festlegung der Injektion $f\colon X\to Y$.
Sei $x\in X$ fest. Für die Festlegung $f(x)=y$ bestehen genau
$n$ Möglichkeiten, nämlich so viele, wie es Elemente $y\in Y$ gibt.
Bei der Festlegung der übrigen entfällt $y$ aufgrund der Injektivität
von $f$. Für die Festlegung betrachtet man $f$ daher als Injektion%
\[f\colon X\setminus\{x\}\to Y\setminus\{y\},\]
von denen es laut Voraussetzung $(n-1)^{\underline{k-1}}$ gibt. Es sind
also $n$ mal $(n-1)^{\underline{k-1}}$ Möglichkeiten, was gleich
$n^{\underline k}$ ist.\,\qedsymbol
\end{Beweis}

\newpage
\begin{Satz}\label{bijection-from-k-subsets-to-orbits}
Seien $X,Y$ endliche Mengen und sei $|X|=k$. Sei $C_k(Y)$
die Menge der $k$-elementigen Teilmengen von $Y$. Für zwei
Injektionen $X\to Y$ sei ferner die Äquivalenzrelation
\[f\sim g \defiff \exists\pi\in S_k\colon f=g\circ \pi\]
definiert, wobei mit den $\pi\in S_k$ Permutationen gemeint sind.
Zwischen der Quotientenmenge $\operatorname{Inj}(X, Y)/S_k$
und $C_k(Y)$ besteht eine kanonische Bijektion.
\end{Satz}
\begin{Beweis}
Wir definieren diese Bijektion als
\[\varphi\colon \operatorname{Inj}(X, Y)/S_k\to C_k(Y),
\quad \varphi([f]) := f(X),\]
wobei $[f]=f\circ S_k$ die Äquivalenzklasse des Repräsentanten $f$
bezeichne. Sie ist wohldefiniert, denn für $f\sim g$ gilt
\[f(X) = (g\circ\pi)(X) = g(\pi(X)) = g(X).\]
Für die Injektivität von $\varphi$ ist zu zeigen, dass $f(X) = g(X)$
die Aussage $[f]=[g]$ impliziert, also die Existenz einer Permutation
$\pi$ mit $f=g\circ\pi$. Weil $g$ injektiv ist, existiert eine
Linksinverse $g^{-1}$, so dass wir $\pi:=g^{-1}\circ f$ wählen
können. Es verbleibt somit die Gleichung $f=g\circ g^{-1}\circ f$ zu
zeigen. Zwar ist $g^{-1}$ im Allgemeinen keine Rechtsinverse, ihre
Einschränkung auf $g(X)$ aber schon. Wegen $f(X)=g(X)$ hebt sich
$g\circ g^{-1}$ daher wie gewünscht auf $f(X)$ weg.

Zur Surjektivität von $\varphi$. Hier ist zu zeigen, dass es zu jeder
Menge $B\in C_k(Y)$ eine Injektion $f$ mit $f(X)=B$ gibt. Betrachten
wir sie als Bijektion $f\colon X\to B$. Eine solche besteht,
weil $X$ und $B$ gleichmächtig sind.\,\qedsymbol
\end{Beweis}

\begin{Satz}[Anzahl der Kombinationen]\newlinefirst
Sei $Y$ eine $|Y|=n$ Elemente enthaltende endliche Menge und $C_k(Y)$
die Menge der $k$-elementigen Teilmengen von $Y$.
Es gilt $|C_k(Y)| = \binom{n}{k}$.
\end{Satz}
\begin{Beweis}[Beweis 1]
Sei $X$ eine Menge mit $|X|=k$. Es gilt
\[|C_k(Y)|
\stackrel{\text{(1)}}= |\operatorname{Inj}(X,Y)/S_k|
\stackrel{\text{(2)}}= \frac{|\operatorname{Inj}(X,Y)|}{|S_k|}
= \frac{n^{\underline k}}{k!} = \binom{n}{k}.\]
Gleichung (1) gilt hierbei laut Satz
\ref{bijection-from-k-subsets-to-orbits}.
Die Einsicht von (2) erhält man mit der folgenden Überlegung.
Für jede Gruppe $G$ gilt die Bahnformel $|G| = |f\circ G|\cdot |G_f|$.
Ist nun die Fixgruppe $G_f$ trivial, ist $|G_f|=1$ und
infolge $|f\circ G|=|G|$. Dies ist bei der symmetrischen Gruppe
$G=S_k$ der Fall. Aus diesem Grund enthält jede Bahn $f\circ S_k$
gleich viele Elemente, $|S_k|$ an der Zahl. Weil die Bahnen außerdem
paarweise disjunkt sind, erhält man die Faktorisierung
\[|\operatorname{Inj}(X,Y)| = |S_k|\cdot |\operatorname{Inj}(X,Y)/S_k|.\,\qedsymbol\]
\end{Beweis}
\begin{Beweis}[Beweis 2]
Induktion über $(n, k)$. Im Anfang ist $k=0$ oder $k=n$. Der abstruse
Fall $k=0$ sucht nach Teilmengen ohne Elemente. Es existiert genau eine
solche Menge, nämlich die leere Menge, womit $C_0(Y)=1$ ist. Der Fall
$k=n$ sucht nach Teilmengen, die so viele Elemente haben wie $Y$.
Dies kann nur $Y$ selbst sein, womit $C_n(Y)=1$ gilt.
Gleichermaßen gilt $\binom{n}{0}=1$ und $\binom{n}{n}=1$.

Induktionsvoraussetzung sei die Gültigkeit für $(n-1, k-1)$ und
$(n-1, k)$. Man nimmt nun ein Element $y$ aus $Y$ heraus,
womit darin $n-1$ verbleiben. Entscheidet man sich,
$y$ zur Teilmenge hinzuzufügen, verbleiben noch $k-1$ Elemente
auszuwählen. Entscheidet man sich dagegen, verbleibt die Teilmenge
unverändert, womit nach wie vor $k$ Elemente auszuwählen sind.
Die Anzahl der Möglichkeiten ist somit
\[|C_k(Y)| = |C_{k-1}(Y\setminus\{y\})| + |C_k(Y\setminus\{y\})|
\stackrel{\mathrm{IV}}=\binom{n-1}{k-1} + \binom{n-1}{k}
= \binom{n}{k}.\,\qedsymbol\]
\end{Beweis}

\begin{Satz}[Gitterweg-Interpretation]\newlinefirst
Ein Gitterweg\index{Gitterweg} auf dem Gitter $\Z\times\Z$ heißt
monoton, wenn von $(x,y)$ aus lediglich der Schritt nach $(x+1,y)$
oder der Schritt nach $(x,y+1)$ gewährt ist. Die Anzahl der monotonen
Gitterwege von $(0,0)$ zu $(x,y)$ beträgt
\[\frac{(x+y)!}{x!y!} = \binom{x+y}{x} = \binom{x+y}{y}.\]
\end{Satz}
\begin{Beweis}[Beweis 1]
Alle Gitterwege besitzen dieselbe Länge $x+y$. Die Knoten des jeweiligen
Wegs nummerieren wir der Reihe nach mit Ausnahme des letzten. Nun
ist von den $x+y$ Nummern eine Teilmenge von $y$ Nummern auszuwählen,
an denen ein Schritt nach oben stattfinden soll. Dafür gibt es
$\binom{x+y}{y}$ Möglichkeiten.\,\qedsymbol
\end{Beweis}

\begin{Beweis}[Beweis 2]
Es bezeichne $f(x,y)$ die Anzahl der Wege von $(0,0)$
zu $(x,y)$. Zum Erreichen eines Randpunktes besteht immer nur eine
einzige Möglichkeit, womit $f(x,0)=1$ und $f(0,y)=1$ gilt. Der nicht
auf dem Rand befindliche Punkt $(x,y)$ kann nun von $(x-1,y)$ oder von
$(x,y-1)$ aus erreicht werden, womit
\[f(x,y) = f(x-1,y) + f(x,y-1)\]
gelten muss. Man sieht nun, dass diese Rekursion ein gedrehtes
pascalsches Dreieck erzeugt. Wir setzen daher $f(x,y) = C(x+y,x)$ und
führen die Koordinatentransformation $x+y=n$ und $x=k$ aus. Die
Rekurrenz nimmt damit die Form
\begin{gather*}
C(x+y,x) = C(x-1+y,x-1) + C(x+y-1,x)\\
\iff C(n,k) = C(n-1,k-1) + C(n-1,k).
\end{gather*}
an. Die Randbedingungen führen zu $C(n,n)=1$ und $C(n,0)=1$. Durch diese
Rekurrenz ist eindeutig der Binomialkoeffizient $C(n,k)=\binom{n}{k}$
bestimmt, womit
\[f(x,y) = C(x+y,x) = \binom{x+y}{x}\]
gelten muss.\,\qedsymbol
\end{Beweis}

\begin{Satz}[Rekursionsformel der Potenzmengenabbildung]\newlinefirst
Für $x\notin M$ gilt $\mathcal P(M\cup\{x\}) = \mathcal P(M)\cup\{A\cup\{x\}\mid A\in\mathcal P(M)\}$.
\end{Satz}
\begin{Beweis}
Die Gleichung ist äquivalent zu
\[T\subseteq M\cup\{x\}\iff T\subseteq M\lor\exists A\subseteq M\colon T=A\cup\{x\}.\]
Nehmen wir die rechte Seite an. Im Fall $T\subseteq M$ gilt erst
recht $T\subseteq M\cup\{x\}$. Im anderen Fall liegt ein $A\subseteq M$
vor, womit $A\cup\{x\}\subseteq M\cup\{x\}$ gilt. Wegen $T=A\cup\{x\}$
gilt also ebenfalls $T\subseteq M\cup\{x\}$.

Nehmen wir die linke Seite an. Mit $T\subseteq M\cup\{x\}$ und
$x\notin M$ folgt per Satz \ref{subseteq-diff}
\[T\setminus\{x\}\subseteq (M\cup\{x\})\setminus\{x\} = M,
\;\text{also}\; T\setminus\{x\}\subseteq M.\]
Im Fall $x\notin T$ ist
$T=T\setminus\{x\}$, womit man $T\subseteq M$
erhält. Im Fall $x\in T$ wird $A:=T\setminus\{x\}$
als Zeuge der Existenzaussage gewählt.\,\qedsymbol
\end{Beweis}

\newpage
\section{Endliche Summen}

\subsection{Allgemeine Regeln}

\begin{Definition}[Summe]
Sei $(G,+,0)$ eine kommutative Gruppe und $a_k\in G$. Die Summe ist
rekursiv definiert als
\[\sum_{k=m}^{m-1} a_k := 0,\quad \sum_{k=m}^n a_k
:= a_n + \sum_{k=m}^{n-1} a_k.\]
\end{Definition}

\begin{Satz}\label{sum-add}
Es gilt
\[\sum_{k=m}^n (a_k + b_k) = \sum_{k=m}^n a_k + \sum_{k=m}^n b_k.\]
\end{Satz}
\begin{Beweis} Induktion über $n$. Im Anfang $n=m-1$ haben
beide Seiten der Gleichung den Wert null. Induktionsschritt:
\begin{align*}
\sum_{k=m}^n (a_k+b_k) &= a_n + b_n + \sum_{k=m}^{n-1} (a_k+b_k)
\stackrel{\mathrm{IV}}= a_n + b_n + \sum_{k=m}^{n-1} a_k + \sum_{k=m}^{n-1} b_k\\
&= \sum_{k=m}^n a_k + \sum_{k=m}^n b_k.\,\qedsymbol
\end{align*}
\end{Beweis}

\begin{Satz}\label{sum-scale}
Sei $R$ ein Ring und $c,a_k\in R$. Sei $c$ eine
Konstante. Es gilt
\[\sum_{k=m}^n ca_k = c\sum_{k=m}^n a_k.\]
\end{Satz}
\begin{Beweis} Induktion über $n$. Im Anfang $n=m-1$ haben beide
Seiten der Gleichung den Wert null. Induktionsschritt:
\[\sum_{k=m}^n ca_k = ca_n + \sum_{k=m}^{n-1} ca_k
\stackrel{\mathrm{IV}}= ca_n + c\sum_{k=m}^{n-1} a_k
= c(a_n + \sum_{k=m}^{n-1} a_k) = c\sum_{k=m}^n a_k.\,\qedsymbol\]
\end{Beweis}

\begin{Satz}[Aufteilung einer Summe]\label{sum-split}
Für $m\le p\le n$ gilt
\[\sum_{k=m}^n a_k = \sum_{k=m}^{p-1} a_k + \sum_{k=p}^n a_k.\]
\end{Satz}
\begin{Beweis} Induktion über $n$. Im Induktionsanfang ist $n=p$
und folglich:
\[\sum_{k=m}^p a_k = \sum_{k=m}^{p-1} a_k + p_k
= \sum_{k=m}^{p-1} a_k + \sum_{k=p}^p a_k.\]
Induktionsschritt:
\[\sum_{k=m}^n a_k = a_n + \sum_{k=m}^{n-1} a_k
\stackrel{\mathrm{IV}}= a_n + \sum_{k=m}^{p-1} a_k + \sum_{k=p}^{n-1} a_k
= \sum_{k=m}^{p-1} a_k + \sum_{k=p}^n a_k.\,\qedsymbol\]
\end{Beweis}

\begin{Satz}[Indexshift]\label{sum-indexshift}\newlinefirst
Für die Indexverschiebung der Distanz $d\in\Z$ gilt
\[\textstyle\sum_{k=m}^n a_k = \sum_{k=m+d}^{n+d} a_{k-d}.\]
\end{Satz}
\begin{Beweis}[Beweis 1]
Induktion über $n$. Im Anfang $n = m-1$ haben beide Seiten
der Gleichung den Wert null. Induktionsschritt:
\[\sum_{k=m}^n a_k = a_n+\sum_{k=m}^{n-1}a_k \stackrel{\mathrm{IV}}=
a_{(n+d)-d}+\sum_{k=m+d}^{n+d-1}a_{k-d}
= \sum_{k=m+d}^{n+d}a_{k-d}.\,\qedsymbol\]
\end{Beweis}
\begin{Beweis}[Beweis 2]
Mit der Substitution $k=k'-d$ findet sich die Umformung
\[\sum_{k=m}^n a_k \stackrel{\text{(1)}}= \sum_{m\le k\le n} a_k
\stackrel{\text{(2)}}= \sum_{m\le k'-d\le n} a_{k'-d}
\stackrel{\text{(3)}}= \sum_{m+d\le k'\le n+d} a_{k'-d}
\stackrel{\text{(4)}}= \sum_{k'=m+d}^{n+d} a_{k'-d},\]
wobei (1), (4) gemäß Satz \ref{sum-set-is-range} gelten
und (2), (3) eine andere Schreibweise für die Substitutionsregel
\ref{sum-set-subs} ist.\,\qedsymbol
\end{Beweis}
\strong{Bemerkung.} Der zweite Beweis ist eigentlich zirkulär,
weil der Beweis der Substitutionsregel über den Beweis von
Satz \ref{sum-set-well-defined} in transitiver Abhängigkeit zum
generalisierten Kommutativgesetz \ref{sum-perm-index} steht, dessen
Beweis einen Indexshift enthält.

\begin{Satz} Es gilt
\[\sum_{i=m}^n \sum_{j=m'}^{n'} a_{ij} = \sum_{j=m'}^{n'}\sum_{i=m}^n a_{ij}.\]
\end{Satz}
\begin{Beweis}
Induktion über $n$ und $n'$. Im Anfang bei $n=m-1$ und $n'=m-1$
haben beide Seiten der Gleichung den Wert null. Induktionsschritt für $n$:
\[\sum_{i=m}^n\sum_{j=m'}^{n'} a_{ij}
= \!\!\sum_{j=m'}^{n'} a_{nj}
+ \!\sum_{i=m}^{n-1}\sum_{j=m'}^{n'} a_{ij}
\stackrel{\mathrm{IV}}=
\!\sum_{j=m'}^{n'} a_{nj}
+ \!\sum_{j=m'}^{n'}\sum_{i=m}^{n-1} a_{ij}
= \!\!\sum_{j=m'}^{n'} (a_{nj}+\sum_{i=m}^{n-1} a_{ij})
= \!\!\sum_{j=m'}^{n'} \sum_{i=m}^n a_{ij}.\]
Induktionsschritt für $n'$:
\[\sum_{i=m}^n\sum_{j=m'}^{n'} a_{ij}
= \!\!\sum_{i=m}^n (a_{in'}+\!\!\sum_{j=m'}^{n'-1}a_{ij})
= \!\!\sum_{i=m}^n a_{in'}+\!\!\sum_{i=m}^n\sum_{j=m'}^{n'-1}a_{ij}
\stackrel{\mathrm{IV}}=
\!\sum_{i=m}^n a_{in'}+\sum_{j=m'}^{n'-1}\sum_{i=m}^n a_{ij}
= \!\!\sum_{j=m'}^{n'}\sum_{i=m}^n a_{ij}.\]
Weil immer ein Schritt nach rechts oder ein Schritt nach oben durchführbar ist,
werden alle Punkte $(n,n')$ im Gitter $\Z_{\ge m-1}\times\Z_{\ge m'-1}$ erreicht.\,\qedsymbol
\end{Beweis}

\begin{Satz}[Umkehrung der Reihenfolge]\label{sum-rev}\newlinefirst
Es gilt $\sum_{k=0}^n a_k = \sum_{k=0}^n a_{n-k}$.
\end{Satz}
\begin{Beweis}
Induktion über $n$. Im Anfang $n=-1$ haben beide Seiten der Gleichung
den Wert null. Der Induktionsschritt ist
\begin{align*}
\sum_{k=0}^n a_{n-k} &= a_{n-n} + \sum_{k=0}^{n-1} a_{n-k}
\stackrel{\mathrm{IV}}= a_0+\sum_{k=0}^{n-1} a_{n-(n-1-k)}a_k\\
&= a_0+\sum_{k=0}^{n-1} a_{k-1}
\stackrel{\text{(1)}}= \sum_{k=0}^0 a_k+\sum_{k=1}^n a_k
\stackrel{\text{(2)}}= \sum_{k=0}^n a_k,
\end{align*}
wobei (1) gemäß Indexshift \ref{sum-indexshift} und
(2) gemäß Aufteilung \ref{sum-split} gilt.\,\qedsymbol
\end{Beweis}

\newpage
\begin{Satz}[Generalisiertes Kommutativgesetz]%
\label{sum-perm-index}\newlinefirst
Sei $M=\{k\in\Z\mid m\le k\le n\}$. Für jede Permutation $\pi\colon M\to M$ gilt
\[\sum_{k=m}^n a_k = \sum_{k=m}^n a_{\pi(k)}.\]
\end{Satz}
\begin{Beweis} Induktiv. Sei ohne Beschränkung der Allgemeinheit $m=1$.
Im Induktionsanfang $n=0$ und $n=1$ ist die Gleichung offenkundig erfüllt.

Induktionsschritt. Induktionsvoraussetzung sei die Gültigkeit für $M$.
Zu zeigen ist die Gültigkeit für $M\cup\{n+1\}$.

Sei $t$ ein fester Parameter mit $1\le t\le n+1$.
Im Fall $\pi(t) = n+1$ geht man wie folgt vor.
Man setze $\sigma(k):=\pi(k)$ für $1\le k\le t-1$. Man setze
$\sigma(k):=\pi(k+1)$ für $t\le k\le n$. Weil $n+1$ kein Wert von
$\sigma$ ist, muss $\sigma$ eine Permutation $\sigma\colon M\to M$ sein.
Ergo gilt
\begin{align*}
\sum_{k=1}^{n+1} a_{\pi(k)} &= \sum_{k=1}^{t-1} a_{\pi(k)}
+ a_{\pi(t)} + \sum_{k=t+1}^{n+1} a_{\pi(k)}
= a_{\pi(t)} + \sum_{k=1}^{t-1} a_{\pi(k)}
+ \sum_{k=t}^n a_{\pi(k+1)}\\
&= a_{n+1} + \sum_{k=1}^{t-1} a_{\sigma(k)}
+ \sum_{k=t}^n a_{\sigma(k)}
= a_{n+1} + \sum_{k=1}^n a_{\sigma(k)}\\
&\stackrel{\mathrm{IV}}= a_{n+1} + \sum_{k=1}^n a_k
= \sum_{k=1}^{n+1} a_k.
\end{align*}
Man beachte, dass in den beiden Randfällen $t=1$ und $t=n+1$ die
jeweilige Randsumme den Wert null hat und somit verschwindet.\,\qedsymbol
\end{Beweis}

\begin{Definition}\label{def:sum-set}
Für eine endliche Menge $M$ definiert man
\[\sum_{k\in M} a_k := \sum_{i=m}^n a_{f(i)},\]
wobei $f\colon \{m,\ldots,n\}\to M$ eine frei wählbare Bijektion ist.
\end{Definition}

\begin{Satz}\label{sum-set-well-defined}
Der Wert Summe auf der rechten Seite von Def. \ref{def:sum-set}
ist unabhängig von der gewählten Bijektion.
\end{Satz}
\begin{Beweis} Seien $f,g$ zwei solche Bijektionen. Dann existiert
$\pi$ mit $f=g\circ\pi$, womit%
\[\sum_{i=m}^n a_{f(i)} = \sum_{i=m}^n a_{g(\pi(i))} =
\sum_{i=m}^n a_{g(i)}\]
laut Satz \ref{sum-perm-index} gilt.\,\qedsymbol
\end{Beweis}

\begin{Satz}\label{sum-set-is-range}
Für $M = \{k\in\Z\mid m\le k\le n\}$ gilt
\[\sum_{m\le k\le n} a_k := \sum_{k\in M} a_k = \sum_{k=m}^n a_k.\]
\end{Satz}
\begin{Beweis} Es gilt
$\sum_{k\in M} a_k = \sum_{k=m}^n a_{\id(k)} = \sum_{k=m}^n a_k$.\,\qedsymbol
\end{Beweis}

\begin{Satz}[Substitutionsregel]\label{sum-set-subs}
Ist $\varphi\colon M'\to M$ eine Bijektion, gilt
\[\sum_{k\in M} a_k = \sum_{k'\in M'} a_{\varphi(k')}.\]
\end{Satz}
\begin{Beweis} Zur Bijektion $f\colon\{1,\ldots,|M|\}\to M$ existiert die Bijektion
$g$ mit $f = \varphi\circ g$.

Infolge gilt
\[\sum_{k\in M} a_k = \sum_{i=1}^{|M|} a_{f(i)}
= \sum_{i=1}^{|M|} a_{\varphi(g(i))}
= \sum_{k'\in M'} a_{\varphi(k')}.\,\qedsymbol\]
\end{Beweis}

\begin{Satz} Es gilt
$\sum\limits_{k\in M} ca_k = c\sum\limits_{k\in M} a_k$ und
$\sum\limits_{k\in M} (a_k + b_k)
= \sum\limits_{k\in M} a_k + \sum\limits_{k\in M} b_k$.
\end{Satz}
\begin{Beweis}
Laut Definition gilt
\begin{gather*}
\sum_{k\in M} ca_k = \sum_{i=1}^{|M|} ca_{f(i)}
= c\sum_{i=1}^{|M|} a_{f(i)} = c\sum_{k\in M} a_k,\\
\sum_{k\in M} (a_k+b_k) = \sum_{i=1}^{|M|} (a_{f(i)}+b_{f(i)}) =
\sum_{i=1}^{|M|} a_{f(i)} + \sum_{i=1}^{|M|} b_{f(i)}
= \sum_{k\in M} a_k + \sum_{k\in M} b_k.\,\qedsymbol
\end{gather*}
\end{Beweis}

\begin{Satz} Es gilt
\[\sum_{k\in M}\sum_{l\in N} a_{kl} = \sum_{l\in N}\sum_{k\in M} a_{kl}.\]
\end{Satz}
\begin{Beweis}
Laut Definition gilt
\[\sum_{k\in M}\sum_{l\in N} a_{kl}
= \sum_{i=1}^{|M|}\sum_{j=1}^{|N|} a_{f(i),g(j)}
= \sum_{j=1}^{|N|}\sum_{i=1}^{|M|} a_{f(i),g(j)}
= \sum_{l\in N}\sum_{k\in M} a_{k,l}.\]
\end{Beweis}

\begin{Satz}\label{sum-set-disjoint}
Für $M\cap N=\emptyset$ gilt
\[\sum_{k\in M\cup N} a_k = \sum_{k\in M} a_k + \sum_{k\in N} a_k.\]
\end{Satz}
\begin{Beweis}
Sei $m:=|M|$ und $n:=|N|$. Laut Prämisse existiert eine Bijektion
$f\colon \{1,\ldots, m+n\}\to M\cup N$ mit $f(i)\in M$ für
$1\le i\le m$ und $f(i)\in N$ für $m+1\le i\le m+n$. Das macht
\[\sum_{k\in M\cup N} a_k = \sum_{i=1}^{m+n} a_{f(i)}
= \sum_{i=1}^m a_{f(i)} + \sum_{i=m+1}^{m+n} a_{f(i)}
= \sum_{k\in M} a_k + \sum_{k\in N} a_k.\,\qedsymbol\]
\end{Beweis}

\begin{Satz}\label{sum-partition}
Für eine disjunkte Zerlegung $M = \bigcup_{i\in I} M_i$ gilt
\[\sum_{k\in M} a_k = \sum_{i\in I}\sum_{k\in M_i} a_k.\]
\end{Satz}
\begin{Beweis}
Induktion über $I$. Im Anfang $I=\emptyset$ haben beide Seiten
den Wert null. Induktionsvoraussetzung sei die Gültigkeit
für $I$. Zu zeigen ist die Gültigkeit für $I\cup\{n\}$ mit $n\notin I$.
Der Induktionsschritt ist
\[\sum_{k\in M_n\cup M} a_k
= \sum_{k\in M_n} a_k + \sum_{k\in M} a_k
\stackrel{\mathrm{IV}}= \sum_{k\in M_n} a_k +
\sum_{i\in I}\sum_{k\in M_i} a_k
= \sum_{i\in I\cup\{n\}}\sum_{k\in M_i} a_k.\,\qedsymbol\]
\end{Beweis}

\begin{Satz}
Es gilt
\[\sum_{t\in M\times N} a_t = \sum_{k\in M}\sum_{l\in N} a_{(k,l)}.\]
\end{Satz}
\begin{Beweis}
Es ist $M = \bigcup_{k\in M} \{k\}$ und weiter $M\times N =
\bigcup_{k\in M} (\{k\}\times N)$ eine disjunkte Zerlegung. Hiermit
findet sich die Umformung
\[\sum_{t\in M\times N} a_t
\stackrel{\text{(1)}}= \sum_{k\in M}\;\sum_{t\in \{k\}\times N} a_t
\stackrel{\text{(2)}}= \sum_{k\in M}\sum_{l\in N} a_{(k,l)},\]
wobei (1) laut Satz \ref{sum-partition} gilt und (2) per Substitutionsregel
\ref{sum-set-subs} mit der Bijektion $\varphi\colon N\to\{k\}\times N$
mit $\varphi(l):=(k,l)$ und $t=\varphi(l)$.
\end{Beweis}

\begin{Satz} Mit der Indikatorfunktion $1_A\colon M\to\{0,1\}$
für $A\subseteq M$ gilt
\[\sum_{k\in M} 1_A(k)a_k = \sum_{k\in A} a_k.\]
\end{Satz}
\begin{Beweis}
Mit disjunkter Zerlegung $M=A\cup (M\setminus A)$
und Satz \ref{sum-set-disjoint} gilt
\[\sum_{k\in M} 1_A(k)a_k = \sum_{k\in A} \underbrace{1_A(k)}_{1} a_k
+ \sum_{k\in M\setminus A}\underbrace{1_A(k)}_{0} a_k
= \sum_{k\in A} a_k.\,\qedsymbol\]
\end{Beweis}

\begin{Satz} Allgemein gilt
\[\sum_{k\in A\cup B} a_k = \sum_{k\in A} a_k + \sum_{k\in B} a_k
- \sum_{k\in A\cap B} a_k.\]
\end{Satz}
\begin{Beweis} Sei $G=A\cup B$ die Grundmenge. Gemäß Satz
\ref{indicator-set-op} darf man rechnen
\begin{align*}
\sum_{k\in G} a_k &= \sum_{k\in G} 1_{A\cup B}(k)a_k
= \sum_{k\in G} 1_A(k)a_k + \sum_{k\in G} 1_B(k)a_k
- \sum_{k\in G} 1_{A\cap B}(k)a_k\\
&= \sum_{k\in A} a_k + \sum_{k\in B} a_k - \sum_{k\in A\cap B} a_k.\,\qedsymbol
\end{align*}
\end{Beweis}

\begin{Definition}[Differenzenfolge]\index{Differenzenfolge}
Zu einer Folge $(a_k)$ definiert man
\[(\Delta a)_k := a_{k+1} - a_k.\]
\end{Definition}

\begin{Satz}[Teleskopsumme]\label{sum-tele} Es gilt
\[\sum_{k=m}^{n-1} (\Delta a)_k = \sum_{k=m}^{n-1} (a_{k+1} - a_k) = a_n - a_m.\]
\end{Satz}
\begin{Beweis}[Beweis 1] Induktion über $n$. Im Anfang $n=m$ haben beide Seiten
der Gleichung den Wert null. Induktionsschritt:
\[\sum_{k=m}^n (a_{k+1} - a_k) = (a_{n+1} - a_n) + \!\!\sum_{k=m}^{n-1} (a_{k+1} - a_n)
\stackrel{\mathrm{IV}}= a_{n+1} - a_n + a_n - a_m = a_{n+1} - a_m.\,\qedsymbol\]
\end{Beweis}
\begin{Beweis}[Beweis 2]
Per Indexshift \ref{sum-indexshift}
gilt $\sum\limits_{k=m}^{n-1} a_{k+1} = \!\!\sum\limits_{k=m+1}^n\!\! a_k
= a_n - a_m + \sum\limits_{k=m}^{n-1} a_k$.
Somit ist%
\[\sum_{k=m}^{n-1} (a_{k+1} - a_k) = \sum_{k=m}^{n-1} a_{k+1} - \sum_{k=m}^{n-1} a_k
= a_n - a_m + \sum_{k=m}^{n-1} a_k - \sum_{k=m}^{n-1} a_k = a_n - a_m.\,\qedsymbol\]
\end{Beweis}

\begin{Satz}
Zum Beweis einer Formel
\[\sum_{k=m}^{n-1} a_k = s_n\]
genügt es, $s_m=0$ und $(\Delta s)_n = a_n$ zu zeigen.
\end{Satz}
\begin{Beweis}[Beweis 1]
Induktion über $n$. Im Anfang $n=m$ haben beide Seiten der Gleichung laut
der Prämisse den Wert null. Induktionsschritt:
\[\sum_{k=m}^n a_k = a_n + \sum_{k=m}^{n-1} a_k
\stackrel{\mathrm{IV}}= a_n + s_n
= (\Delta s)_n + s_n = s_{n+1} - s_n + s_n = s_{n+1}.\,\qedsymbol\]
\end{Beweis}
\begin{Beweis}[Beweis 2]
Spezialisierung von Satz \ref{sum-tele}.\,\qedsymbol
\end{Beweis}

\begin{Satz} Der Differenzoperator ist linear. Das heißt,
für alle Folgen $(a_n), (b_n)$ und jede Konstante $c$ gilt
\begin{align*}
& \Delta(a+b) = \Delta a + \Delta b, && ((a+b)_n := a_n + b_n)\\
& \Delta(ca) = c\Delta a. && ((ca)_n := ca_n)
\end{align*}
\end{Satz}
\begin{Beweis} Man findet
\begin{align*}
(\Delta(a+b))_n &= (a+b)_{n+1} - (a+b)_n
= (a_{n+1}+b_{n+1}) - (a_n+b_n)\\
&= a_{n+1}-a_n + b_{n+1}-b_n = (\Delta a)_n + (\Delta b)_n
= (\Delta a + \Delta b)_n
\end{align*}
und
\[(\Delta(ca))_n = (ca)_{n+1} - (ca)_n = ca_{n+1} - ca_n
= c(a_{n+1} - a_n) = c(\Delta a)_n = (c\Delta a)_n.\,\qedsymbol\]
\end{Beweis}

\begin{Definition}[Shiftoperator] Man definiert
\[(Ta)_n := a_{n+1}.\]
\end{Definition}

\begin{Satz}[Iterierter Differenzoperator]\newlinefirst
Für jede Folge $(a_n)$ und $m\in\Z_{\ge 0}$ gilt
\[(\Delta^m a)_n = (-1)^m\sum_{k=0}^m\binom{m}{k} (-1)^k a_{n+k}.\]
\end{Satz}
\begin{Beweis} Es gilt $\Delta = T-\id$. Weil $T$ und $\id$ kommutieren,
ist der binomische Lehrsatz anwendbar. Es ergibt sich
\[\Delta^m = (T-\id)^m = \sum_{k=0}^m\binom{m}{k} (-1)^{m-k} T^k\id^{m-k}
= (-1)^m \sum_{k=0}^m\binom{m}{k} (-1)^k T^k.\,\qedsymbol\]
\end{Beweis}

\begin{Satz}\label{delta-deg}
Sei $f$ eine Polynomfunktion. Dann ist $\Delta_h f$ eine
Polynomfunktion mit niedrigerem Grad.
\end{Satz}
\begin{Beweis} Für $f(x)=\sum_{n=0}^m a_n x^n$ gilt
\begin{gather*}
\Delta_h f(x) = f(x+h) - f(x) = \sum_{n=0}^m a_n (x+h)^n - \sum_{n=0}^m a_n x^n
= \sum_{n=0}^m a_n ((x+h)^n - x^n)\\
= \sum_{n=0}^m a_n (x^n + \sum_{k=0}^{n-1}\binom{n}{k}x^k h^{n-k} - x^n)
= \sum_{n=0}^m a_n \sum_{k=0}^{n-1}\binom{n}{k}x^k h^{n-k}.
\end{gather*}
In der Summe treten nur Monome bis $x^{m-1}$ auf.\,\qedsymbol
\end{Beweis}

\begin{Satz} Sei $f$ ein Polynom vom Grad $N$. Für $n,a\in\Z$ und $n\ge a$ gilt
\[f(n) = \sum_{k=0}^N \frac{(\Delta^k f)(a)}{k!}(n-a)^{\underline k}
= \sum_{k=0}^N \binom{n-a}{k}(\Delta^k f)(a).\]
\end{Satz}
\begin{Beweis}
Es gilt $T=\Delta+\id$. Für jede nichtnegative ganze Zahl $m$ gilt
\[T^m = (\Delta+\id)^m = \sum_{k=0}^m\binom{m}{k}\Delta^k\]
mit dem binomischen Lehrsatz, da $\Delta$ und $\id$ kommutieren. Das macht
\[f(a + m) = \sum_{k=0}^m\binom{m}{k}(\Delta^k f)(a).\]
Man substituiere nun $n = a+m$. Für $n\ge a$ gilt dann
\[f(n) = \sum_{k=0}^{n-a}\binom{n-a}{k}(\Delta^k f)(a)
= \sum_{k=0}^N\binom{n-a}{k}(\Delta^k f)(a).\]
Der Indexbereich der Summierung durfte auf bis $k=N$ geändert werden, weil
$\Delta^k f = 0$ für $k>N$ laut Satz \ref{delta-deg} gilt. Dass nun
Summanden mit $k>n-a$ auftreten können, ist nicht weiter schlimm, weil
in diesem Fall $\binom{n-a}{k}=0$ ist.\,\qedsymbol
\end{Beweis}

\newpage
\subsection{Klassische Partialsummen}

\begin{Satz}[Partialsummen der konstanten Folge]%
\label{sum-const}\newlinefirst
Es gilt $\displaystyle\sum_{k=m}^n 1 = n-m+1$.
\end{Satz}
\begin{Beweis}
Induktion über $n$. Im Anfang $n=m-1$ haben beide Seiten
der Gleichung den Wert null. Induktionsschritt:
\[\sum_{k=m}^n 1 = 1 + \sum_{k=m}^{n-1}
\stackrel{\mathrm{IV}}= 1 + n-1-m+1 = n-m+1.\,\qedsymbol\]
\end{Beweis}

\begin{Satz}[Partialsummen der arithmetischen Folge]%
\index{arithmetische Folge}\newlinefirst
Es gilt $\displaystyle\sum_{k=0}^n k = \frac{n}{2}(n+1)$.
\end{Satz}
\begin{Beweis}[Beweis 1]
Induktion über $n$. Im Anfang $n=-1$ haben beide Seiten der Gleichung
den Wert null. Induktionsschritt:
\[\sum_{k=0}^n k = n + \sum_{k=0}^{n-1} k
\stackrel{\mathrm{IV}}= n + \frac{n-1}{2}(n-1+1)
= \frac{n}{2}(2 + n-1) = \frac{n}{2}(n+1).\,\qedsymbol\]
\end{Beweis}
\begin{Beweis}[Beweis 2]
Klassischer Beweis. Man findet die Umformung
\[2\!\sum_{k=0}^n k = \!\sum_{k=0}^n k + \!\sum_{k=0}^n k
\stackrel{\text{(1)}}= \!\sum_{k=0}^n k + \!\sum_{k=0}^n (n-k)
\stackrel{\text{(2)}}= \!\sum_{k=0}^n (k+n-k)
= \!\sum_{k=0}^n n \stackrel{\text{(3)}}= n\!\sum_{k=0}^n 1
\stackrel{\text{(4)}}= n(n+1),\]
wobei (1), (2), (3), (4) gemäß Satz
\ref{sum-rev}, \ref{sum-add}, \ref{sum-scale}, \ref{sum-const}
gelten.\,\qedsymbol
\end{Beweis}

\begin{Satz}[Partialsummen der geometrischen Folge]%
\label{sum-geom-seq}\index{geometrische Folge}\newlinefirst
Für $m\ge 0$ und $z\in\C\setminus\{1\}$ gilt
$\displaystyle\sum_{k=m}^{n-1} z^k = \frac{z^n-z^m}{z-1}$.
\end{Satz}
\begin{Beweis}
Induktion über $n$. Im Anfang $n=m-1$ haben beiden Seiten der Gleichung
den Wert null. Induktionsschritt:
\[\sum_{k=m}^n z^k = z^n + \sum_{k=m}^{n-1} z^k
\stackrel{\mathrm{IV}}= z^n + \frac{z^n-z^m}{z-1}
= \frac{(z-1)z^n+z^n-z^m}{z-1}
= \frac{z^{n+1}-z^m}{z-1}.\,\qedsymbol\]
\end{Beweis}

\begin{Satz}
Für $m\ge 0$ und $z\in\C\setminus\{1\}$ gilt
\[\sum_{k=m}^{n-1} kz^k
= \frac{(nz^n-mz^m)(z-1) - (z^n-z^m)z}{(z-1)^2}.\]
\end{Satz}
\begin{Beweis}
Die Gleichung von Satz \ref{sum-geom-seq} für $m\ge 1$ auf beiden
Seiten nach $z$ ableiten und anschließend beide Seiten mit $z$
multiplizieren. Den Fall $m=0$ und in diesem den Summand zu $k=0$
explizit betrachten, sonst aber auf dieselbe Weise vorgehen.\,\qedsymbol
\end{Beweis}

\newpage
\begin{Satz} Es gilt
\[\sum_{k=1}^n (-1)^k k = (-1)^n\left\lfloor\frac{n+1}{2}\right\rfloor.\]
\end{Satz}
\begin{Beweis}
Induktion über $n$. Im Anfang $n=0$ haben beiden Seiten den Wert null.

Induktionsschritt:
\[\sum_{k=1}^n (-1)^k k = (-1)^n n + \sum_{k=1}^{n-1} (-1)^k k
\stackrel{\mathrm{IV}}= (-1)^n n + (-1)^{n-1}\left\lfloor\frac{n}{2}\right\rfloor
= (-1)^n (n-\left\lfloor\frac{n}{2}\right\rfloor).\]
Zu zeigen verbleibt die Gleichung
\[n-\left\lfloor\frac{n}{2}\right\rfloor = \left\lfloor\frac{n+1}{2}\right\rfloor
\iff n = \left\lfloor\frac{n}{2}\right\rfloor + 
\left\lfloor\frac{n+1}{2}\right\rfloor.\]
Wir nehmen die Fallunterscheidung zwischen geraden und ungeraden
Zahlen vor, um Satz \ref{floor-add-int} und \ref{floor-is-zero}
nutzen zu können. Im geraden Fall $n=2k$ bestätigt sich
\[\left\lfloor\frac{2k}{2}\right\rfloor +  \left\lfloor\frac{2k+1}{2}\right\rfloor
= \lfloor k\rfloor + \left\lfloor k + \frac{1}{2}\right\rfloor = k + k = 2k.\]
Im ungeraden Fall $n=2k+1$ bestätigt sich
\[\left\lfloor\frac{2k+1}{2}\right\rfloor + \left\lfloor\frac{2k+1+1}{2}\right\rfloor
= \left\lfloor k + \frac{1}{2}\right\rfloor + \left\lfloor k+1\right\rfloor
= k + k + 1 = 2k + 1.\,\qedsymbol\]
\end{Beweis}

\section{Funktionen}

\subsection{Floor und Ceil}

\begin{Definition}[Floorfunktion]%
\label{def:floor}\index{Floorfunktion}
Für $x\in\R$ definiert man
\[y = \lfloor x\rfloor\defiff y\in\Z\land 0\le x-y < 1.\]
\end{Definition}

\begin{Definition}[Ceilfunktion]%
\label{def:ceil}\index{Ceilfunktion}
Für $x\in\R$ definiert man
\[y = \lceil x\rceil\defiff y\in\Z\land 0\le y-x < 1.\]
\end{Definition}

\begin{Satz}\label{floor-add-int}
Für jede ganze Zahl $k$ gilt $\lfloor k + x\rfloor = k + \lfloor x\rfloor$.
\end{Satz}
\begin{Beweis} Aufgrund der Prämisse $k\in\Z$ ist $y\in\Z$ äquivalent
zu $y-k\in\Z$. Unter dieser Gegebenheit findet sich mit
Def. \ref{def:floor} die äquivalente Umformung
\begin{align*}
y = \lfloor k+x\rfloor &\iff y\in\Z\land 0\le (k+x)-y < 1\iff y-k\in\Z\land 0\le x-(y-k) < 1\\
&\iff y - k = \lfloor x\rfloor \iff y = k + \lfloor x\rfloor.\,\qedsymbol
\end{align*}
\end{Beweis}

\begin{Satz}\label{floor-is-zero}
Für $0\le x < 1$ gilt $\lfloor x\rfloor = 0$.
\end{Satz}
\begin{Beweis}
Dies folgt unmittelbar aus Def. \ref{def:floor}.\,\qedsymbol
\end{Beweis}

\newpage
\subsection{Faktorielle}

\begin{Definition}[Fakultät]%
\label{def:factorial}\index{Fakultaet@Fakultät}\newlinefirst
Für eine nichtnegative ganze Zahl $n$ definiert man $n!$ rekursiv durch
\[0! := 1,\quad (n+1)! := (n+1)n!.\]
\end{Definition}

\begin{Definition}[Fallende Faktorielle]\label{def:falling-factorial}%
\index{Faktorielle!fallende}\index{fallende Faktorielle}\newlinefirst
Für $k\in\Z_{\ge 0}$ und $n\in\Z$ (oder allgemeiner $n\in\C$)
definiert man $n^{\underline k}$ rekursiv durch
\[n^{\underline 0} := 1,\quad n^{\underline {k+1}}:=n(n-1)^{\underline k}.\]
\end{Definition}

\begin{Definition}[Steigende Faktorielle]\label{def:raising-factorial}%
\index{Faktorielle!steigende}\index{steigende Faktorielle}\newlinefirst
Für $k\in\Z_{\ge 0}$ und $n\in\Z$ (oder allgemeiner $n\in\C$)
definiert man $n^{\overline k}$ rekursiv durch
\[n^{\overline 0} := 1,\quad n^{\overline {k+1}}:=n(n+1)^{\overline k}.\]
\end{Definition}

\begin{Satz}\label{relation-ff-factorial}
Für $n,k\in\Z_{\ge 0}$ und $k\le n$ gilt
\[n^{\underline k} = \frac{n!}{(n-k)!}.\]
\end{Satz}
\begin{Beweis}
Induktion über $k$. Im Anfang $k=0$ resultieren beide Seiten der
Gleichung im gleichen Wert~1. Der Induktionsschritt ist
\[n^{\underline k} = n(n-1)^{\underline{k-1}}
\stackrel{\text{IV}}= n\frac{(n-1)!}{((n-1)-(k-1))!}
= \frac{n(n-1)!}{(n-k)!}
= \frac{n!}{(n-1)!}.\,\qedsymbol\]
\end{Beweis}

\begin{Satz}
Für ganze Zahlen $n,k$ mit $n\ge 1$ und $k\ge 1-n$ gilt
\[n^{\overline k} = \frac{(n+k-1)!}{(n-1)!}.\]
\end{Satz}
\begin{Beweis}
Induktion über $k$. Im Anfang $k=0$ resultieren beide Seiten der
Gleichung im gleichen Wert~1. Der Induktionsschritt ist
\begin{align*}
n^{\overline k} &= n(n+1)^{\overline{k-1}}
\stackrel{\text{IV}}= n\frac{(n+1+k-1-1)!}{(n+1-1)!}
= \frac{n(n+k-1)!}{n!}\\
&= \frac{n(n+k-1)!}{n(n-1)!}
= \frac{(n+k-1)!}{(n-1)!}.\,\qedsymbol
\end{align*}
\end{Beweis}

\begin{Satz}
Für jedes $n\in\Z_{\ge 0}$ gilt $n!\le n^n$.
\end{Satz}
\begin{Beweis}
Induktion über $n$. Im Induktionsanfang $n=0$ hat man $0! = 1$ und $0^0=1$.
Zum Induktionsschritt unternimmt man zunächst die äquivalente Umformung
\[(n+1)!\le (n+1)^{n+1} \iff (n+1)n!\le (n+1)(n+1)^n
\iff n!\le (n+1)^n.\]
Diese Ungleichung bestätigt die Rechnung
\[(n+1)^n = \sum_{k=0}^n\binom{n}{k}n^k =
n^n+\sum_{k=0}^{n-1}\binom{n}{k}n^k\ge n^n
\stackrel{\mathrm{IV}}\ge n!.\,\qedsymbol\]
\end{Beweis}

\newpage
\subsection{Binomialkoeffizient}

\begin{Definition}[Binomialkoeffizient]%
\label{def:binom}\index{Binomialkoeffizient}\newlinefirst
Für $k\in\Z_{\ge 0}$ und $n\in\Z$ (oder allgemeiner $n\in\C$)
definiert man
\[\binom{n}{k} := \frac{n^{\underline k}}{k!}.\]
\end{Definition}

\begin{Satz}
Für $n\in\Z$ mit $k\le n$ gilt
\[\binom{n}{k} = \frac{n!}{k!(n-k)!}\]
\end{Satz}
\begin{Beweis}
Folgt direkt aus Def. \ref{def:binom} und Satz
\ref{relation-ff-factorial}.\,\qedsymbol
\end{Beweis}

\begin{Satz}
Für $k\ge 1$ und $n\in\Z$ (oder allgemeiner $n\in\C$) gilt
\[\binom{n}{k} = \frac{n}{k}\binom{n-1}{k-1}.\]
\end{Satz}
\begin{Beweis}
Es findet sich die Umformung
\[\binom{n}{k} = \frac{n^{\underline k}}{k!}
= \frac{n(n-1)^{\underline {k-1}}}{k(k-1)!}
= \frac{n}{k}\binom{n-1}{k-1}.\,\qedsymbol\]
\end{Beweis}

\begin{Satz}
Für $k\ge 1$ und $n\in\Z$ (oder allgemeiner $n\in\C$) gilt
\[\binom{n}{k} = \binom{n-1}{k-1} + \binom{n-1}{k}.\]
\end{Satz}
\begin{Beweis}
Es findet sich die Umformung
\begin{align*}
\binom{n-1}{k-1} + \binom{n-1}{k}
&= \frac{(n-1)^{\underline{k-1}}}{(k-1)!} + \frac{(n-1)^{\underline k}}{k!}
= \frac{k(n-1)^{\underline{k-1}}}{k!} + \frac{(n-1)^{\underline{k-1}}(n-k)}{k!}\\
&= \frac{(n-1)^{\underline{k-1}}}{k!}(k + n -k)
= \frac{n(n-1)^{\underline{k-1}}}{k!}
= \frac{n^{\underline k}}{k!} = \binom{n}{k}.\,\qedsymbol
\end{align*}
\end{Beweis}


\chapter{Zahlentheorie}
\section{Kongruenzen und Teilbarkeit}

\begin{Definition}[Teiler] Für $a,m\in\Z$ ist die Relation »$m$ teilt
$a$« definiert als
\[m\mid a\defiff \exists k\in\Z\colon a = km.\]
\end{Definition}

\begin{Definition}[Kongruenz]\label{def:cong}
Für $a,b,m\in\Z$ ist die Relation »$a$ ist kongruent zu $b$ modulo
$m$« definiert als
\[a\equiv b \pmod m \defiff m\mid (a-b).\]
\end{Definition}

\begin{Satz}
Es gilt $m\mid a$ genau dann, wenn $a\equiv 0\pmod m$.
\end{Satz}
\begin{Beweis}
Spezialisierung von Def. \ref{def:cong} mit $b:=0$.\,\qedsymbol
\end{Beweis}

\begin{Definition}[Teilermenge] Die Teilermenge
einer ganzen Zahl $a$ ist definiert durch
\[m\in T(a)\defiff m\mid a.\]
\end{Definition}
\strong{Bemerkung.} Wir setzen $T_{\ge 1}(a):=T(a)\cap\Z_{\ge 1}$ und
$T_{\ge 0}(a):=T(a)\cap\Z_{\ge 0}$.

\begin{Satz}\label{divisor-divisor-subset} Es gilt
\[a\mid b \iff T(a)\subseteq T(b).\]
\end{Satz}
\begin{Beweis}
Zur Implikation von rechts nach links.
Die Aussage $T(a)\subseteq T(b)$ ist laut ihrer Definition
äquivalent zu $\forall m\colon m\mid a\Rightarrow m\mid b$.
Die Anwendung dieser Prämsse auf den Ansatz $a\mid a$
liefert sofort $a\mid b$.

Zur Implikation von links nach rechts. Expandiert man die Aussage
durch Einsetzen der Definitionen, ist
\[(\exists k\colon b=ka)\implies
(\forall m\colon (\exists x\colon a=xm)\implies (\exists y\colon b=ym)).\]
zu zeigen. Wir haben einen Zeugen $k$ für $b=ka$ gegeben. 
Gemäß Prämisse liegt ein Zeuge $x$ für $a=xm$ vor, wonach $ka=kxm$
gilt, also $b=kxm$. Ergo ist $y:=kx$ ein Zeuge für
$\exists y\colon b=ym$.\,\qedsymbol
\end{Beweis}

\begin{Satz}
Die Teilerrelation ist transitiv, das heißt,
$a\mid b$ und $b\mid c$ impliziert $a\mid c$.
\end{Satz}
\begin{Beweis}
Unter Nutzung von Satz \ref{divisor-divisor-subset}
nimmt die Aussage die Gestalt
\[T(a)\subseteq T(b)\land T(b)\subseteq T(c)\implies T(a)\subseteq T(c)\]
an. Diese Aussage ist nun aber offenkundig, da die Relation »ist
Teilmenge von« eine transitive ist.\,\qedsymbol
\end{Beweis}

\begin{Satz}\label{divides-weaken}
Gilt $m\mid a$ und $b\in\Z$, so gilt auch $m\mid (ab)$.
\end{Satz}
\begin{Beweis}
Die Prämisse liefert einen Zeugen $x$ mit $a=xm$, womit $ab=bxm$
gilt. Ergo existiert mit $y:=bx$ ein Zeuge für
$\exists y\colon ab=ym$.\,\qedsymbol
\end{Beweis}

\begin{Satz}
Die Kongruenzrelation ist eine Äquivalenzrelation. Das heißt, es gilt
\begin{align*}
& a\equiv a\pmod m, && \text{(Reflexivität)}\\
& a\equiv b \implies b\equiv a\pmod m, && \text{(Symmetrie)}\\
& a\equiv b\land b\equiv c\implies a\equiv c\pmod m. && \text{(Transitivität)}
\end{align*}
\end{Satz}
\begin{Beweis}
Zur Reflexivität. Es gilt die äquivalente Umformung
\[a\equiv a\pmod m \iff m\mid (a-a) \iff m\mid 0\iff (\exists k\colon 0 = km).\]
Die letzte Aussage ist durch $k=0$ erfüllt.

Zur Symmetrie. Die Prämisse liefert einen Zeugen $x$ mit $a-b = xm$,
womit $b-a = -xm$ gilt. Ergo ist $y:=-x$ ein Zeuge für die
Existenzaussage $\exists y\colon b-a = ym$.

Zur Reflexivität. Die Prämissen liefern Zeugen $x$ mit $a-b=xm$
und $y$ mit $b-c=ym$. Infolge gilt
\[a-c = (a-b) + (b-c) = xm + ym = (x+y)m.\]
Ergo ist $z:=x+y$ ein Zeuge für $\exists z\colon a-c = zm$.\,\qedsymbol
\end{Beweis}

\begin{Satz}\label{cong-shift}
Für ganze Zahlen $a,b,c$ gilt
\begin{gather*}
a\equiv b\pmod m \iff a+c\equiv b+c \pmod m,\\
a\equiv b\pmod m \iff a-c\equiv b-c \pmod m.
\end{gather*}
\end{Satz}
\begin{Beweis}
Unter Nutzung der Definition findet sich die äquivalente Umformung
\begin{align*}
a+c\equiv b+c\pmod m &\iff m\mid ((a+c)-(b+c))
\iff m\mid (a-b)\\
&\iff a\equiv b\pmod m.
\end{align*}
Der Beweis der zweiten Aussage verläuft analog.\,\qedsymbol
\end{Beweis}

\begin{Satz}\label{cong-scale}
Für ganze Zahlen $a,b,c$ gilt
\[a\equiv b\pmod m\implies ac=bc\pmod m.\]
\end{Satz}
\begin{Beweis} Die Prämisse liefert einen Zeugen $x$ mit
$a-b = xm$, womit $ac-bc = cxm$ gilt. Ergo existiert mit $y:=cx$ ein Zeuge
für $\exists y\colon ac-bc = ym$.\,\qedsymbol
\end{Beweis}

\begin{Satz}\label{cong-add-sub-mul}
Gilt $a\equiv a'\pmod m$ und $b\equiv b'\pmod m$,
so gilt auch
\begin{gather*}
a+b\equiv a'+b'\pmod m,\\
a-b\equiv a'-b'\pmod m,\\
ab\equiv a'b'\pmod m.
\end{gather*}
\end{Satz}
\begin{Beweis}
Laut Prämisse liegen Zeugen $x$ mit $a=a'+xm$ und $y$ mit $b=b'+ym$ vor.
Infolge gilt
\[a+b = a'+xm+b'+ym = a'+b'+(x+y)m.\]
Ergo existiert mit $z:=x+y$ ein Zeuge für
$\exists z\colon (a+b)=(a'+b')+zm$. Der Beweis der zweiten Regel
verläuft analog. Bei der Multiplikation gilt
\[ab = (a'+xm)(b'+ym) = a'b' + a'ym+b'xm + xym^2
= a'b' + (a'y+b'x+xym)m.\]
Ergo existiert mit $z:=a'y+b'x+xym$ ein Zeuge für
$\exists z\colon ab = a'b'+zm$.\,\qedsymbol
\end{Beweis}

\begin{Satz}\label{cong-pow}
Gilt $a\equiv a'\pmod m$ und $n\in\Z_{\ge 0}$, so gilt auch
$a^n\equiv (a')^n\pmod m$.
\end{Satz}
\begin{Beweis}
Induktion über $n$ mit Induktionsanfang bei $n=0$. Mit $a^0=1$ und
$(a')^0=1$ wird die Behauptung in diesem Fall zu $1\equiv 1$, die
aufgrund der Reflexivität gilt. Induktionsschritt.
Wendet man Satz \ref{cong-add-sub-mul} auf die Prämisse
$a\equiv a'$ und die Induktionsvoraussetzung $a^n\equiv (a')^n$
an, findet sich $aa^n \equiv a'(a')^n$, also
$a^{n+1}\equiv (a')^{n+1}$.\,\qedsymbol
\end{Beweis}

\begin{Satz}\label{cong-sum}
Gilt $a_k\equiv a_k'\pmod m$ für alle $k$, so gilt auch\\
$\sum_{k=0}^{n-1} a_k\equiv \sum_{k=0}^{n-1} a_k'\pmod m$.
\end{Satz}
\begin{Beweis}
Induktion über $n$. Für $n=0$ wird die Behauptung zu $0\equiv 0$, die
aufgrund der Reflexivität gilt. Induktionsschritt. Wendet man Satz
\ref{cong-add-sub-mul} auf die
Prämisse $a_n\equiv a_n'$ und die Induktionsvoraussetzung
$\sum_{k=0}^{n-1} a_k\equiv \sum_{k=0}^{n-1} a_k'$ an, folgt
\[\sum_{k=0}^{n-1} a_k = a_n + \sum_{k=0}^{n-1} a_k\equiv a_n' + \sum_{k=0}^{n-1} a_k'
= \sum_{k=0}^{n-1} a_k'.\,\qedsymbol\]
\end{Beweis}

\begin{Satz}
Sei $p\in\Z[X]$, also ein Polynom mit ganzzahligen Koeffizienten.
Gilt $x\equiv x'\pmod m$, so gilt auch $p(x)\equiv p(x')\pmod m$.
\end{Satz}
\begin{Beweis}
Laut Satz \ref{cong-pow} gilt $x^k\equiv (x')^k$ für jedes
$k\ge 0$. Im weiteren Fortgang gilt $a_k x^k\equiv a_k (x')^k$
wegen Satz \ref{cong-shift}. Mit Satz \ref{cong-sum} erhält man
schließlich
\[p(x) = \sum_{k=0}^n a_k x^k\equiv\sum_{k=0}^n a_k (x')^k = p(x').\,\qedsymbol\]
\end{Beweis}

\begin{Satz}
Für jede ganze Zahl $n$ gilt $2\mid n \iff 2\mid n^2$.
\end{Satz}
\begin{Beweis}
Die Implikation von links nach rechts gilt gemäß
Satz \ref{divides-weaken} mit $a:=n$ und $b:=n$.
Zur Implikation von rechts nach links. Wir zeigen die Kontraposition
\[\neg(2\mid n) \implies \neg(2\mid n^2).\]
Laut Prämisse ist $n$ ungerade, also von der Form $n=2k+1$ mit $k\in\Z$.
Nun gilt
\[n^2 = (2k+1)^2 = 4k^2 + 4k + 1 = 2(2k^2 + 2k) + 1,\]
wonach $n^2$ ebenfalls ungerade sein muss.\,\qedsymbol
\end{Beweis}

\section{Primzahlen}

\begin{Definition}[Teilerfunktion]\newlinefirst
Für eine positive ganze Zahl $n$ definiert man
\[\sigma_k(n) := \sum_{d\mid n} d^k,\]
wobei mit $d\mid n$ die positiven Teiler $d\in T_{\ge 1}(n)$ gemeint
sind.
\end{Definition}

\begin{Definition}[Primzahl]\newlinefirst
Eine positive ganze Zahl $n$ wird Primzahl genannt, wenn sie
zwei unterschiedliche positive Teiler besitzt, womit $\sigma_0(n)=2$
gemeint ist.
\end{Definition}

\begin{Satz}
Eine Zahl $n$ ist genau dann eine Primzahl, wenn $n\ge 2$ ist und
ihre einzigen beiden positiven Teiler 1 und $n$ selbst sind.
\end{Satz}
\begin{Beweis}
Jede ganze Zahl besitzt 1 und sich selbst als Teiler. Somit muss
$\sigma_0(n)=2$ für $n\ge 2$ äquivalent zu $T_{\ge 1}(n)=\{1,n\}$
sein.\,\qedsymbol
\end{Beweis}

\begin{Satz}[Satz des Euklid]
Es gibt unendlich viele Primzahlen.
\end{Satz}
\begin{Beweis}[Klassischer Beweis]
Sei $M=\{p_1,\ldots,p_n\}$ eine endliche Menge von Primzahlen.
Es wird gezeigt, dass eine weitere Primzahl $p\notin M$ existiert.
Man bilde dazu das Produkt $m=\prod_{k=1}^n p_k$. Nun ist $m+1$ entweder
prim oder nicht. Falls $m+1$ prim ist, ist mit $p=m+1$ eine weitere
Primzahl gefunden. Sei also $m+1$ nicht prim, womit mindestens
ein Primfaktor $p$ enthalten ist. Angenommen, es wäre $p=p_k$ für eines
der $k$. Dann gälte $p_k\mid m+1$. Es gilt gemäß Konstruktion von $m$
aber auch $p_k\mid m$. Ergo wäre $p_k$ ebenso ein Teiler der
Differenz $(m+1)-m = 1$. Das ist absurd, weil keine Primzahl
ein Teiler der Zahl~1 ist.\,\qedsymbol
\end{Beweis}



\printindex

\end{document}


