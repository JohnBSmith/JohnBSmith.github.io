
\chapter{Elemente der Algebra}

\section{Gruppentheorie}

\subsection{Elementare Gesetzmäßigkeiten}
\begin{Definition}[Gruppe]\mbox{}\\*
Sei $G$ eine Menge und $*\colon G\times G\to\Omega$ eine Verknüfung.
Die Menge $G$ bildet bezüglich der Verknüpfung eine Gruppe $(G,*)$,
wenn die folgenden Axiome erfüllt sind:
\begin{itemize}
\item[\strong{(E)}]
Es darf $\Omega=G$ sein, d.\,h. die Verknüpfung führt nicht aus $G$ heraus.
\item[\strong{(A)}]
Das Assoziativgesetz $a*(b*c)=(a*b)*c$ gilt für alle $a,b,c\in G$.
\item[\strong{(N)}]
Es gibt ein neutrales Element $e$, so dass $g*e=e*g=g$ für jedes
$g\in G$ gilt.
\item[\strong{(I)}]
Zu jedem $g\in G$ gibt es ein Element $h\in G$ mit $g*h=h*g=e$,
wobei $e$ ein neutrales Element ist.
\end{itemize}
\end{Definition}
Anstelle von $g*h$ schreibt man auch kurz $gh$. Es gibt auch Gruppen,
die additiv geschrieben werden, da schreibt man $g+h$ anstelle von
$gh$ und $ng$ anstelle von $g^n$ für $n\in\Z$.

\begin{Satz} Das neutrale Element einer Gruppe ist eindeutig bestimmt,
d.\,h. es kann keine zwei unterschiedlichen neutralen Elemente geben.
\end{Satz}
\strong{Beweis.} Seien $e$ und $e'$ zwei neutrele Elemente.
Zu zeigen ist, dass dann schon $e'=e$ gilt. Nach Voraussetzung
gilt $ae=a$ und $e'b=b$ für alle $a,b$. Setzt man $a:=e'$ und
$b:=e$ ein, dann ergibt sich $e' = e'e = e.$\;\qedsymbol

\begin{Satz}
In jeder Gruppe gilt die Linkskürzbarkeit $ga=gb\implies a=b$
und die Rechtskürzbarkeit $ag=bg\implies a=b$.
\end{Satz}
\strong{Beweis.} Die Gleichung $ga=gb$ multipliziert man auf beiden
Seiten mit $g^{-1}$, dann gilt
\[ga=gb \implies g^{-1}ga=g^{-1}gb \iff ea=eb \iff a=b.\]
Für die Gleichung $ag=bg$ geht das analog.\;\qedsymbol

\begin{Satz} Zu jedem Element ist das inverse Element eindeutig
bestimmt, d.\,h. es kann keine zwei unterschiedlichen inversen
Elemente geben.
\end{Satz}
\strong{Beweis.} Seien $h$ und $h'$ invers zu $g$. Dann gilt
$gh=e$ und $gh'=e$. Daher ist $gh=gh'$. Gemäß Linkskürzbarkeit
folgt daraus $h=h'$.\;\qedsymbol

