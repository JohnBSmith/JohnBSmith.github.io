
\chapter{Elemente der Algebra}

\section{Gruppentheorie}

\subsection{Elementare Gesetzmäßigkeiten}
\begin{Definition}[Gruppe]\mbox{}\\*
Sei $G$ eine Menge und $*\colon G\times G\to\Omega$ eine Verknüpfung.
Die Menge $G$ bildet bezüglich der Verknüpfung eine Gruppe $(G,*)$,
wenn die folgenden Axiome erfüllt sind:
\begin{itemize}
\item[\strong{(E)}]
Es darf $\Omega=G$ sein, d.\,h. die Verknüpfung führt nicht aus $G$ heraus.
\item[\strong{(A)}]
Das Assoziativgesetz $a*(b*c)=(a*b)*c$ gilt für alle $a,b,c\in G$.
\item[\strong{(N)}]
Es gibt ein neutrales Element $e$, so dass $g*e=e*g=g$ für jedes
$g\in G$ gilt.
\item[\strong{(I)}]
Zu jedem $g\in G$ gibt es ein Element $h\in G$ mit $g*h=h*g=e$,
wobei $e$ ein neutrales Element ist. Dieses $h$ wird
inverses Element zu $g$ genannt.
\end{itemize}
\end{Definition}
Anstelle von $g*h$ schreibt man auch kurz $gh$. Für das inverse Element
zu $g$ schreibt man $g^{-1}$. Es gibt auch Gruppen,
bei denen die Verknüpfung als Addition geschrieben wird, da schreibt
man $g+h$ anstelle von $gh$ und $ng$ anstelle von $g^n$ für $n\in\Z$.
Für das inverse Element $-g$ anstelle von $g^{-1}$.

\begin{Definition}[Abelsche Gruppe]\mbox{}\\*
Zwei Elemente $a,b$ kommutieren, wenn $a*b=b*a$ ist.
Eine Gruppe $G$ heißt abelsch oder kommutativ, wenn alle
Elemente der Gruppe kommutieren, d.\,h. wenn
das Kommutativgesetz $(\forall a,b\in G)(a*b=b*a)$ erfüllt ist.
\end{Definition}
Bei den allermeisten Verknüpfungen, die als Addition 
geschrieben werden, ist das Kommutativgesetz erfüllt.

\begin{Satz} Das neutrale Element einer Gruppe ist eindeutig bestimmt,
d.\,h. es kann keine zwei unterschiedlichen neutralen Elemente geben.
\end{Satz}
\strong{Beweis.} Seien $e$ und $e'$ zwei neutrale Elemente.
Zu zeigen ist, dass dann schon $e'=e$ gilt. Nach Voraussetzung
gilt $ae=a$ und $e'b=b$ für alle $a,b$. Setzt man $a:=e'$ und
$b:=e$ ein, dann ergibt sich $e' = e'e = e.$\;\qedsymbol

\begin{Satz}
In jeder Gruppe gilt die Linkskürzbarkeit $ga=gb\implies a=b$
und die Rechtskürzbarkeit $ag=bg\implies a=b$.
\end{Satz}
\strong{Beweis.} Die Gleichung $ga=gb$ multipliziert man auf beiden
Seiten mit $g^{-1}$, dann gilt
\[ga=gb \implies g^{-1}ga=g^{-1}gb \iff ea=eb \iff a=b.\]
Für die Gleichung $ag=bg$ geht das analog.\;\qedsymbol

Kürzbarkeit bedeutet, eine Multiplikation auf beiden Seiten
rückgängig machen zu können. Das Rückgänig"=machen"=können
ist wiederum die charakteristische Eigenschaft einer injektiven
Abbildung. Unter diesem Aspekt gesehen bedeutet die Kürzbarkeit,
dass die Links- und Rechts"=Translation
\begin{gather}
\label{eq:ltrans} l_g\colon G\to G,\quad l_g(x):=gx,\\
\label{eq:rtrans} r_g\colon G\to G,\quad r_g(x):=xg
\end{gather}
injektiv sind. Wie man leicht nachrechnet, sind sie sogar bijektiv.
Für die Umkehrabbildungen gilt $(l_g)^{-1}=l_{g^{-1}}$ und
$(r_g)^{-1}=r_{g^{-1}}$.

\begin{Satz} Zu jedem Element ist das inverse Element eindeutig
bestimmt, d.\,h. es kann keine zwei unterschiedlichen inversen
Elemente geben.
\end{Satz}
\strong{Beweis.} Seien $h$ und $h'$ invers zu $g$. Dann gilt
$gh=e$ und $gh'=e$. Daher ist $gh=gh'$. Gemäß Linkskürzbarkeit
folgt daraus $h=h'$.\;\qedsymbol

\begin{Definition}[Untergruppe]\mbox{}\\*
Sei $(G,*)$ eine Gruppe und $U\subseteq G$.
Man nennt $U$ Untergruppe von $G$, kurz $U\le G$, wenn
$(U,*)$ die Gruppenaxiome bezüglich der selben Verknüpfung $*$
erfüllt.
\end{Definition}

\begin{Satz}[Untergruppenkriterium]\mbox{}\\*
Sei $G$ eine Gruppe. Eine nichtleere Teilmenge $H\subseteq G$ ist eine
Untergruppe von $G$, wenn mit $a,b\in H$ auch $ab\in H$ und
mit $a\in H$ auch $a^{-1}\in H$ ist.
\end{Satz}
\strong{Beweis.}
Da $H$ nichtleer ist, gibt es mindestens ein
Element $a\in H$. Nach Voraussetzung ist dann auch $a^{-1}\in H$,
und daher auch das neutrale Element $e=aa^{-1}\in H$.

Das Assoziativgesetz gilt in $H$, weil es in $G$ gilt.
Die Abgeschlossenheit und die Existenz der inversen Elemente
stehen direkt in der Voraussetzung. Damit sind alle Axiome
überprüft.\;\qedsymbol

\begin{Definition}[Homomorphismus zwischen Gruppen]\mbox{}\\*
Seien $(G,*)$ und $(G',*')$ zwei Gruppen. Eine Abbildung
$\varphi\colon G\to G'$ wird Homomorphismus genannt, wenn die
Gleichung
$\varphi(a*b) = \varphi(a)*'\varphi(b)$
für alle $a,b\in G$ erfüllt ist.
\end{Definition}

\begin{Satz} Sei $\varphi\colon G\to G'$ ein Homomorphismus.
Sind $e\in G$ und $e'\in G'$ die neutralen Elemente, dann gilt
$e'=\varphi(e)$. Außerdem ist $\varphi(g)^{-1}=\varphi(g^{-1})$
für jedes $g\in G$.
\end{Satz}
\strong{Beweis.} Es gilt
$e'\varphi(e) = \varphi(e) = \varphi(ee) = \varphi(e)\varphi(e)$.
Kürzen ergibt $e'=\varphi(e)$. Daraus folgt
\[e' = \varphi(e) = \varphi(g^{-1}g) = \varphi(g^{-1})\,\varphi(g).\]
Damit bekommt man
\[\varphi(g)^{-1} = e'\varphi(g)^{-1}
= \varphi(g^{-1})\,\varphi(g)\,\varphi(g)^{-1}
= \varphi(g^{-1})\;\qedsymbol\]

\begin{Satz}\label{hom-img-subgroup}
Sei $\varphi\colon G\to G'$ ein Homomorphismus. Die Bildmenge
$\varphi(G)$ ist eine Untergruppe von $G'$.
\end{Satz}
\strong{Beweis.} Zu prüfen sind die Voraussetzungen des
Untergruppenkriteriums. Wegen $\varphi(a)\varphi(b)
= \varphi(ab)\in\varphi(G)$ und $\varphi(a)^{-1}
= \varphi(a^{-1})\in\varphi(G)$ sind diese erfüllt.\;\qedsymbol

Für injektive, surjektive, bijektive Homomorphismen gibt es
eigene Bezeichnungen. Die injektiven nennt man Monomorphismen,
die surjektiven Epimorphismen und die bijektiven Isomorphismen.

Gibt es zwischen zwei Gruppen $G,G'$ einen Isomorphismus, dann nennt
man die beiden Gruppen isomorph zueinander, man schreibt dafür
$G\simeq G'$. Zwei Gruppen die isomorph zueinander sind, sind
im Wesentlichen gleich. Isomorphie ist eine Äquivalenzrelation.

Monomorphismen charakterisieren die Einbettung einer Gruppe
in eine andere Gruppe. Man kann Einbettungen als Verallgemeinerung
der Untergruppenbeziehung sehen. Hat man nämlich einen Monomorphismus
$\varphi\colon H\to G$, dann erhält man bei Einschränkung der
Zielmenge auf die Bildmenge einen Isomorphismus, d.\,h. es gilt
$H\simeq\varphi(H)$. Die Gruppen $H$ und $\varphi(H)$ sind also im
Wesentlichen gleich. Andererseits ist $\varphi(H)\le G$ gemäß Satz
\ref{hom-img-subgroup}.

\subsection{Gruppenaktionen}

\begin{Definition}[Linksaktion]\mbox{}\\*
Eine Abbildung $f\colon G\times X\to X$ heißt Gruppenlinksaktion,
kurz Linksaktion, wenn für das neutrale Element $e\in G$ und alle
$g,h\in G$ gilt
\begin{gather*}
f(e,x) = x,\qquad f(gh,x) = f(g,f(h,x)).
\end{gather*}
\end{Definition}
Anstelle von $f(g,x)$ schreibt man für gewöhnlich einfach $gx$,
bzw. $g+x$ bei einer additiv geschriebenen Verknüpfung.

\begin{Definition}[Rechtsaktion]\mbox{}\\*
Eine Abbildung $f\colon X\times G\to X$ heißt Gruppenrechtsaktion,
kurz Rechtsaktion, wenn für das neutrale Element $e\in G$ und
alle $g,h\in G$ gilt
\begin{gather*}
f(x,e) = x,\qquad f(x,gh) = f(f(x,g),h).
\end{gather*}
\end{Definition}
Bei diesen Axiomen ist für $X$ eine beliebige Menge zugelassen.
Es kann auch $X=G$ sein. Beispiele dafür haben wir bereits
kennengelernt, nämlich ist die Linkstranslation \eqref{eq:ltrans} eine
Linksaktion und die Rechtstranslation \eqref{eq:rtrans} eine
Rechtsaktion.

\section{Ringtheorie}

Es gibt in der Mathematik Objekte wie Restklassen, Matrizen
und Polynome, für die wie bei den ganzen Zahlen eine Addition und
eine Multiplikation definiert ist. Die Addition und Multiplikation
von zwei Matrizen ergibt z.\,B. wieder eine Matrix. In jedem Fall
genügen die Addition und Multiplikation einem bestimmten Muster, den
Ring"=Axiomen. Das legt nahe, aus den Axiomen allgemeine Rechenregeln
und Gesetzmäßigkeiten abzuleiten, die somit in allen Ringen gültig
sind.

Wir erhalten dadurch als neues Werkzeug ein verallgemeinertes Rechnen.
Das ist für uns ganz besonders wichtig, da eine enorme Anzahl
von mathematischen Strukturen die Struktur eines Rings enthält.
Z.\,B. ist jeder Körper auch ein Ring. Die rationalen, reellen
und komplexen Zahlen bilden jeweils einen Körper. Allein schon
dieser Umstand, dass die wichtigsten grundlegenden Zahlenbereiche
einen Körper bilden, macht es sinnvoll, Ringe und Körper näher zu
studieren.

Ringe sind außerdem bedeutsam als Grundlage für die
Konzepterweiterungen Modul und assoziative Algebra. 
Diesen beiden Begriffen ist auf bestimmte Art geometrische
Information eingeimpft, sie sind von großer Tragweite in der
linearen Algebra. Z.\,B. ist jeder Vektorraum, und damit insbesondere
jeder euklidische Vektorraum ein Modul. Beispiele für assoziative
Algebren sind die Tensoralgebra, die äußere Algebra und die
Clifford"=Algebra.

Überraschend treten auch in der Analysis solche geometrisch
motivierten Konzepte auf. So wurde die Analysis zur Funktionalanalysis
weiterentwickelt, die auch mit Vektorräumen arbeitet. Als assoziative
Algebren kommen hier die Banachalgebren hinzu.

Neben kontinuierlichen Strukturen sind für die Algebra auch
diskrete Strukturen wie Restklassenringe typisch. Die Restklassenringe
bilden eine Grundlage für die Zahlentheorie.

Schließlich sind Ringe auch tief in der abstrakten Algebra verwurzelt.
Es scheint so, als ergäbe sich dort eine nur schwer überschaubare Fülle
von Strukturen. Das mag richtig sein, allerdings bringen die mit der
axiomatischen Methode gewonnenen allgemeinen Gesetzmäßigkeiten eine
gewisse Ordnung.

\subsection{Elementare Gesetzmäßigkeiten}

\begin{Definition}[Ring]\mbox{}\\*
Eine Struktur $(R,+,\cdot)$ heißt Ring, wenn
\begin{enumerate}
\item $(R,+)$ eine kommutative Gruppe ist,
\item $(R,\cdot)$ eine Halbgruppe ist,
\item die Distributivgesetze $a(b+c)=ab+ac$ und $(a+b)c=ac+bc$
für alle $a,b,c\in R$ erfüllt sind.
\end{enumerate}
\end{Definition}
Es gibt hier einen Unterschied zwischen Linksdistributivgesetz
und Rechtsdistributivgesetz, weil die Multiplikation nicht
kommutativ sein braucht.

\begin{Definition}[Unitärer Ring]\mbox{}\\*
Ein Ring $(R,+,\cdot)$ heißt unitär oder Ring mit Eins, wenn
$(R,\cdot)$ ein Monoid ist.
\end{Definition}
D.\,h. ein unitärer Ring ist ein Ring $R$, in dem es ein Einselement
$e$ gibt, so dass $e\cdot a=a\cdot e=a$ für alle $a\in R$. Man kann
$e=1$ schreiben, muss aber beachten, dass damit ein abstraktes Element
gemeint ist. Unter Umständen verbietet sich das auch aufgrund von
Zweideutigkeit. Z.\,B. ist im Matrizenring das Einselement die
Einheitsmatrix. Diese schreibt man $E$ oder $I$ und nicht $1$, um sie
von der dort ebenfalls vorkommenden Skalarmultiplikation mit der
Zahl $1$ unterscheiden zu können.

\begin{Korollar}
Sei $R$ ein Ring und $0\in R$ das Nullelement, dann gilt
$0\cdot a = a\cdot 0 = 0$ für jedes $a\in R$.
\end{Korollar}
\strong{Beweis.} Man darf rechnen $0a = (0+0)a = 0a+0a$. Subtrahiert
man auf beiden Seiten der Gleichung $0a$, ergibt sich $0=0a$.
Für $a\cdot 0$ ist die Rechnung analog.\;\qedsymbol



