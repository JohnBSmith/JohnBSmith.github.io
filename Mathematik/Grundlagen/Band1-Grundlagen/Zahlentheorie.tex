
\chapter{Zahlentheorie}

\section{Kongruenzen}

\begin{Definition}[Kongruenz]\index{Kongruenz}
Zwei ganze Zahlen $a,b$ heißen kongruent modulo $m$, wenn ihre Differenz
$(b-a)$ durch $m$ teilbar ist:%
\[a\equiv b\pmod{m}\defiff (\exists k\in\Z)(b-a=km).\]
\end{Definition}
Anstelle von »$(\mathrm{mod}\;m)$« schreibt man beim Rechnen meist
kürzer »$(m)$«.

\begin{Satz}
Die Kongruenz ist eine Äquivalenzrelation, d.\,h. es gilt
\begin{align*}
&a\equiv a\pmod{m},&&\text{(Reflexivität)}\\
&a\equiv b\implies b\equiv a\pmod{m},&&\text{(Symmetrie)}\\
&a\equiv b\land b\equiv c\implies a\equiv c\pmod{m}.&&\text{(Transitivität)}
\end{align*}
\end{Satz}
\strong{Beweis.} Für die Reflexivität ist ein $k$ mit $0=a-a=km$
zu finden. Setze $k=0$.

Bei der Symmetrie gibt es nach Voraussetzung
ein $k$ mit $b-a=km$. Dann ist $a-b=-km$. Setze $k'=-k$.
Es gibt also $k'$ mit $a-b=k'm$, somit gilt $b\equiv a$.

Bei der Transitivität gibt es nach Voraussetzung $k$ mit
$b-a=km$ und $l$ mit $b-c=lm$. Das heißt, es gilt
\[b = a+km = c+lm\implies c-a = km-lm = (k-l)m.\]
Setze $k'=k-l$. Es gibt also $k'$ mit $c-a=k'm$.
Somit gilt $a\equiv c$.\;\qedsymbol

\begin{Satz}\label{Kongruenz-add-sub}
Sind $a,b,c$ ganze Zahlen, dann gilt
\begin{align*}
a\equiv b\pmod{m}&\iff a+c\equiv b+c\pmod{m},\\
a\equiv b\pmod{m}&\iff a-c\equiv b-c\pmod{m}.
\end{align*}
\end{Satz}
\strong{Beweis.}
Unter Beachtung von $(b+c)-(a+c)=b-a$ findet man
\begin{gather*}
a\equiv b\pmod{m}
\iff (\exists k\in\Z)(b-a=km)\\
\iff (\exists k\in\Z)((b+c)-(a+c)=km)\\
\iff a+c\equiv b+c\pmod{m}.
\end{gather*}
Für die Subtraktion von $c$ ist die Überlegung analog.\;\qedsymbol

\newpage
\begin{Satz}\label{Kongruenz-mul}
Sind $a,b,c$ ganze Zahlen, dann gilt
\[a\equiv b\pmod{m} \implies ac\equiv bc\pmod{m}.\]
\end{Satz}
\strong{Beweis.}
Unter der Voraussetzung $a\equiv b\pmod{m}$ gibt es ein
$k$ mit $b-a=km$. Es gilt
\[b-a=km\iff (b-a)c=kcm \iff bc-ac=k'm\]
mit $k':=kc$. Man hat also
\[(\exists k'\in\Z)(bc-ac=k'm)\iff ac\equiv bc\pmod{m}.\;\qedsymbol\]

\begin{Satz}
Gilt $a\equiv a'\pmod{m}$ und
$b\equiv b'\pmod{m}$, dann gilt auch
\begin{align*}
a+b&\equiv a'+b'\pmod{m},\\
a-b&\equiv a'-b'\pmod{m},\\
ab&\equiv a'b'\pmod{m}.
\end{align*}
\end{Satz}
\strong{Beweis.} Man findet
\begin{equation}
\left.\begin{aligned}
a\equiv a'&\implies a+b\equiv a'+b\\
b\equiv b'&\implies a'+b\equiv a'+b'
\end{aligned}\right\}
\implies a+b\equiv a'+b\equiv a'+b'\pmod{m}.
\end{equation}
Für die Subtraktion ist die Überlegung analog. Für die Multiplikation
ebenfalls:%
\begin{equation}
\left.\begin{aligned}
a\equiv a'&\implies ab\equiv a'b\\
b\equiv b'&\implies a'b\equiv a'b'
\end{aligned}\right\}
\implies ab\equiv a'b\equiv a'b'\pmod{m}.\;\qedsymbol
\end{equation}

\begin{Satz}
Addition des Moduls führt auf eine kongruente Zahl:%
\[a\equiv a+m\equiv a-m\pmod{m}.\]
\end{Satz}
\strong{Beweis.}
Es gilt
\[a\equiv a+m\pmod{m}\iff (\exists k\in\Z)(km=(a+m)-a=m).\]
Setze $k=1$. Bei
\[a\equiv a-m\pmod{m}\iff (\exists k\in\Z)(km=(a-m)-a=-m)\]
setze $k=-1$.\;\qedsymbol

\section{Der Restklassenring}

Wir könnten nun beginnen, mit der Kongruenzenrechnung interessante
Probleme zu lösen. Zunächst möchte ich aber erläutern, wie die
Kongruenzenrechnung mit dem Restklassenring zusammenhängt. Unter
diesem Blickwinkel bekommen wir ein tieferes Verständnis und können
Mittel der Ringtheorie und Gruppentheorie anwenden.

Zu einer ganzen Zahl $a$ ist die Restklasse modulo $m$ definiert als 
\[[a]_m := \{x\mid x\equiv a\pmod m\}.\]
Eine alternative Schreibweise für $[a]_m$ ist $a+m\Z$. Weil die
Kongruenz eine Äquivalenzrelation ist, handelt es sich bei den
Restklassen um Äquivalenzklassen. Wir betrachten nun die
Quotientenmenge
\[\Z/m\Z := \{[a]_m\mid a\in\Z\}.\]
Nun können wir die Addition und Multiplikation von Restklassen
definieren.
\begin{Satz}
Auf $\Z/m\Z$ sind die beiden Operationen
\begin{align*}
[a]_m + [b]_m &:= [a+b]_m,\\
[a]_m\cdot [b]_m &:= [ab]_m
\end{align*}
wohldefiniert.
\end{Satz}
\strong{Beweis.} Zu zeigen ist, dass $a+b\equiv x+y$ gilt, sofern
$a\equiv x$ und $b\equiv y$ ist. Gemäß Satz \ref{Kongruenz-add-sub} gilt
\begin{align*}
a\equiv x &\iff a+b\equiv x+b,\\
b\equiv y &\iff x+b\equiv x+y.
\end{align*}
Aus den beiden Prämssen erhalten wir demzufolge
$a+b\equiv x+b\equiv x+y$.
Die Argumentation zur Multiplikation ist analog, wobei
Satz \ref{Kongruenz-mul} zur Anwendung kommt.\,\qedsymbol

Die Struktur $(\Z/m\Z,+,\cdot)$ nennt man den \emph{Restklassenring}
zum Modul $m$.

\begin{Lemma}\label{Surjektion-strukturerhaltend}
Sei $M$ eine Struktur mit einer Verknüpfung, von
Magma bis kommutative Gruppe. Ist
$\varphi\colon M\to M'$ eine strukturerhaltende Surjektion,
dergestalt dass $\varphi(ab)=\varphi(a)\varphi(b)$, dann ist
$M'$ von derselben Struktur und $\varphi$ ein Homomorphismus.
\end{Lemma}
\strong{Beweis.} Die Verknüpfung auf $M$ sei abgeschlossen. Weil
$\varphi$ surjektiv ist, gibt es zu $a',b'\in M'$ immer
$a,b\in M$ mit $a'=\varphi(a)$ und $b'=\varphi(b)$. Somit gilt
\[a'b' = \varphi(a)\varphi(b) = \varphi(ab)\in M'.\]
Die Verknüpfung auf $M$ erfülle das Assoziativgesetz. Dann gilt
\[(a'b')c' = \varphi(ab)\varphi(c) = \varphi(abc)
= \varphi(a)\varphi(bc) = a'(b'c').\]
Die Verknüpfung auf $M$ habe ein neutrales Element $e$. Dann gilt
\[\varphi(a)=\varphi(ea) = \varphi(e)\varphi(a),\quad
\varphi(a)=\varphi(ae)=\varphi(a)\varphi(e).\]
Demzufolge besitzt $M'$ mit $e':=\varphi(e)$ ein neutrales Element.

Zur Verknüpfung auf $M$ gebe es zu jedem Element ein inverses. Dann gilt
\[\varphi(e) = \varphi(aa^{-1}) = \varphi(a)\varphi(a^{-1}),\quad
\varphi(e) = \varphi(a^{-1}a) = \varphi(a^{-1})\varphi(a).\]
Demzufolge ist $\varphi(a)^{-1}=\varphi(a^{-1})$.

Die Verknüpfung auf $M$ sei kommutativ. Dann gilt
\[a'b' = \varphi(a)\varphi(b) = \varphi(ab) = \varphi(ba) = \varphi(b)\varphi(a) = b'a'.\]
Somit ist die Verknüpfung auf $M'$ kommutativ.\,\qedsymbol

\begin{Satz} Jeder Restklassenring ist ein kommutativer unitärer Ring.
\end{Satz}
\strong{Beweis.} Bereits bewiesen wurde, dass die ganzen Zahlen einen
kommutativen unitären Ring bilden. Wir betrachten nun die
Quotientenabbildung
\[\pi\colon\Z \to \Z/m\Z,\quad \pi(a):=[a]_m.\]
Die Operationen wurden so definiert dass
$\pi(a+b)=\pi(a)+\pi(b)$ und $\pi(ab)=\pi(a)\pi(b)$ gilt.
Gemäß Lemma \ref{Surjektion-strukturerhaltend} ist $(\Z/m\Z,+)$ also
eine kommutative Gruppe und $(\Z/m\Z,\cdot)$ ein kommutatives
Monoid. Es verbleibt noch das Distributivgesetz zu prüfen. Man rechnet
\begin{align*}
a'(b'+c') &= \pi(a)(\pi(b)+\pi(c)) = \pi(a)(\pi(b+c))
= \pi(a(b+c)) = \pi(ab+ac)\\
&= \pi(ab)+\pi(ac) = \pi(a)\pi(b)+\pi(a)\pi(c) = a'b'+a'c'.
\end{align*}
Damit ist der Satz gezeigt, und ferner ist gezeigt dass $\pi$ ein
Eins"=erhaltender Ringhomomorphismus ist.\,\qedsymbol
