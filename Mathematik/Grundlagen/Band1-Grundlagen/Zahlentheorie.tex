
\chapter{Zahlentheorie}

\section{Kongruenzen}

\begin{Definition}[Kongruenz]\index{Kongruenz}
Zwei ganze Zahlen $a,b$ heißen kongruent modulo $m$, wenn ihre Differenz
$(b-a)$ durch $m$ teilbar ist:%
\[a\equiv b\pmod{m}\defiff\exists k\in\Z\colon b-a=km.\]
\end{Definition}
Anstelle von »$(\mathrm{mod}\;m)$« schreibt man beim Rechnen meist
kürzer »$(m)$«.

\begin{Satz}
Die Kongruenz ist eine Äquivalenzrelation, d.\,h. es gilt
\begin{align*}
&a\equiv a\pmod{m},&&\text{(Reflexivität)}\\
&a\equiv b\implies b\equiv a\pmod{m},&&\text{(Symmetrie)}\\
&a\equiv b\land b\equiv c\implies a\equiv c\pmod{m}.&&\text{(Transitivität)}
\end{align*}
\end{Satz}
\strong{Beweis.} Für die Reflexivität ist ein $k$ mit $0=a-a=km$
zu finden. Setze $k=0$.

Bei der Symmetrie gibt es nach Voraussetzung
ein $k$ mit $b-a=km$. Dann ist $a-b=-km$. Setze $k'=-k$.
Es gibt also $k'$ mit $a-b=k'm$, somit gilt $b\equiv a$.

Bei der Transitivität gibt es nach Voraussetzung $k$ mit
$b-a=km$ und $l$ mit $b-c=lm$. Das heißt, es gilt
\[b = a+km = c+lm\implies c-a = km-lm = (k-l)m.\]
Setze $k'=k-l$. Es gibt also $k'$ mit $c-a=k'm$.
Somit gilt $a\equiv c$.\;\qedsymbol

\begin{Satz}\label{Kongruenz-add-sub}
Sind $a,b,c$ ganze Zahlen, dann gilt
\begin{align*}
a\equiv b\pmod{m}&\iff a+c\equiv b+c\pmod{m},\\
a\equiv b\pmod{m}&\iff a-c\equiv b-c\pmod{m}.
\end{align*}
\end{Satz}
\strong{Beweis.}
Unter Beachtung von $(b+c)-(a+c)=b-a$ findet man
\begin{gather*}
a\equiv b\pmod{m}
\iff (\exists k\in\Z\colon b-a=km)\\
\iff (\exists k\in\Z\colon (b+c)-(a+c)=km)\\
\iff a+c\equiv b+c\pmod{m}.
\end{gather*}
Für die Subtraktion von $c$ ist die Überlegung analog.\;\qedsymbol

\newpage
\begin{Satz}\label{Kongruenz-mul}
Sind $a,b,c$ ganze Zahlen, dann gilt
\[a\equiv b\pmod{m} \implies ac\equiv bc\pmod{m}.\]
\end{Satz}
\strong{Beweis.}
Unter der Voraussetzung $a\equiv b\pmod{m}$ gibt es ein
$k$ mit $b-a=km$. Es gilt
\[b-a=km\iff (b-a)c=kcm \iff bc-ac=k'm\]
mit $k':=kc$. Man hat also
\[(\exists k'\in\Z\colon bc-ac=k'm)\iff ac\equiv bc\pmod{m}.\;\qedsymbol\]

\begin{Satz}
Gilt $a\equiv a'\pmod{m}$ und
$b\equiv b'\pmod{m}$, dann gilt auch
\begin{align*}
a+b&\equiv a'+b'\pmod{m},\\
a-b&\equiv a'-b'\pmod{m},\\
ab&\equiv a'b'\pmod{m}.
\end{align*}
\end{Satz}
\strong{Beweis.} Man findet
\begin{equation}
\left.\begin{aligned}
a\equiv a'&\implies a+b\equiv a'+b\\
b\equiv b'&\implies a'+b\equiv a'+b'
\end{aligned}\right\}
\implies a+b\equiv a'+b\equiv a'+b'\pmod{m}.
\end{equation}
Für die Subtraktion ist die Überlegung analog. Für die Multiplikation
ebenfalls:%
\begin{equation}
\left.\begin{aligned}
a\equiv a'&\implies ab\equiv a'b\\
b\equiv b'&\implies a'b\equiv a'b'
\end{aligned}\right\}
\implies ab\equiv a'b\equiv a'b'\pmod{m}.\;\qedsymbol
\end{equation}

\begin{Satz}
Addition des Moduls führt auf eine kongruente Zahl:%
\[a\equiv a+m\equiv a-m\pmod{m}.\]
\end{Satz}
\strong{Beweis.}
Es gilt
\[a\equiv a+m\pmod{m}\iff (\exists k\in\Z\colon km=(a+m)-a=m).\]
Setze $k=1$. Bei
\[a\equiv a-m\pmod{m}\iff (\exists k\in\Z\colon km=(a-m)-a=-m)\]
setze $k=-1$.\;\qedsymbol

\section{Der Restklassenring}

Wir könnten nun beginnen, mit der Kongruenzenrechnung interessante
Probleme zu lösen. Zunächst möchte ich aber erläutern, wie die
Kongruenzenrechnung mit dem Restklassenring zusammenhängt. Unter
diesem Blickwinkel bekommen wir ein tieferes Verständnis und können
Mittel der Ringtheorie und Gruppentheorie anwenden.

Zu einer ganzen Zahl $a$ ist die Restklasse modulo $m$ definiert als 
\[[a]_m := \{x\mid x\equiv a\pmod m\}.\]
Eine alternative Schreibweise für $[a]_m$ ist $a+m\Z$. Weil die
Kongruenz eine Äquivalenzrelation ist, handelt es sich bei den
Restklassen um Äquivalenzklassen. Wir betrachten nun die
Quotientenmenge
\[\Z/m\Z := \{[a]_m\mid a\in\Z\}.\]
Nun können wir die Addition und Multiplikation von Restklassen
definieren.
\begin{Satz}
Auf $\Z/m\Z$ sind die beiden Operationen
\begin{align*}
[a]_m + [b]_m &:= [a+b]_m,\\
[a]_m\cdot [b]_m &:= [ab]_m
\end{align*}
wohldefiniert.
\end{Satz}
\strong{Beweis.} Zu zeigen ist, dass $a+b\equiv x+y$ gilt, sofern
$a\equiv x$ und $b\equiv y$ ist. Gemäß Satz \ref{Kongruenz-add-sub} gilt
\begin{align*}
a\equiv x &\iff a+b\equiv x+b,\\
b\equiv y &\iff x+b\equiv x+y.
\end{align*}
Aus den beiden Prämssen erhalten wir demzufolge
$a+b\equiv x+b\equiv x+y$.
Die Argumentation zur Multiplikation ist analog, wobei
Satz \ref{Kongruenz-mul} zur Anwendung kommt.\,\qedsymbol

Die Struktur $(\Z/m\Z,+,\cdot)$ nennt man den \emph{Restklassenring}
zum Modul $m$.

\begin{Korollar}
Jeder Restklassenring ist ein kommutativer unitärer Ring.
\end{Korollar}
\strong{Beweis.} Bereits bewiesen wurde, dass die ganzen Zahlen einen
kommutativen unitären Ring bilden. Aufgrund der Wohldefiniertheit der
Addition und Multiplikation ist Kongruenz modulo $m$ eine
Kongruenzrelation. Satz \ref{Kongruenz-Ring-Quotient} zeigt somit die
Behauptung.\,\qedsymbol

Zudem ist die Quotientenabbildung
\[\varphi\colon\Z \to \Z/m\Z,\quad \varphi(a):=[a]_m.\]
ein Eins-erhaltender Ringhomomorphismus, wie aus dem Beweis
von Satz \ref{Kongruenz-Ring-Quotient} hervorgeht.

\newpage
\section{Euklidische Division}

\begin{Lemma}[Lemma zur euklidischen Division]\mbox{}\\*
Zu je zwei ganzen Zahlen $a,b$ mit $b\ne 0$ gibt es zwei eindeutig
bestimmte Zahlen $q,r$ mit $0\le r<|b|$, so dass $a=bq+r$. Man nennt
$q$ den \emph{Quotient} und $r$ den \emph{Rest}.
\end{Lemma}
\strong{Beweis der Existenz.}
Betrachten wir zunächst den Fall, bei dem $a\ge 0$
und $b>0$ ist. Man kann dann sooft $b$ von $a$ abziehen, bis sich eine
Zahl $\ge 0$ und $<b$ ergibt. Formal bilden wir die
Folge $r_k := a-bk$. Nun muss für irgendein $k\ge 0$ schließlich
$0\le r_k<b$ sein. Damit ist $q=k$ und $r=r_k$ gefunden.

Sei nun $a<0$. Wie bereits gezeigt gibt es $q,r$ mit $-a = bq+r$.
Für $r=0$ haben wir dann mit $q':=-q$ und $r':=0$ einen Quotient
und einen Rest. Sei nun $r\ne 0$. Dann gilt
\[a = -bq-r = -(q+1)b + b - r.\]
Mit $q':=-(q+1)$ und $r':=b-r$ gibt es somit auch in diesem Fall einen
Quotient und einen Rest. Der Rest erfüllt auch die gewünschte
Ungleichung, denn aus $r<b$ ergibt sich $0<r'$ und aus $0<r$ ergibt
sich $r'<b$.

Sei nun $a$ beliebig und $b<0$. Dann gibt es $q,r$ mit $a=(-b)q+r$.
Setze also $q':=-q$ und $r':=r$. Damit gilt $a=bq'+r'$, womit
auch in diesem Fall ein Quotient und ein Rest gefunden ist.\,\qedsymbol

\strong{Beweis der Eindeutigkeit.} Das Paar $q,r$ erfülle
$a=bq+r$ und $q',r'$ erfülle ebenfalls $a=bq'+r'$. Dann gilt
\[bq+r = bq'+r',\iff b(q-q') = r'-r,\implies |b| |q-q'| = |r'-r|.\]
Aus $0<r<|b|$ und $0<r'<|b|$ erhält man außerdem $|r'-r|<|b|$. Somit
muss $|b| |q-q'| < |b|$ sein, also $|q-q'| < 1$. Eine nichtnegative
ganze Zahl kann aber nur dann kleiner als eins sein, wenn sie null
ist. Damit hat man
\[|q-q'|=0,\iff q-q'=0,\iff q=q'.\]
Entsprechend folgt $r=r'$.\,\qedsymbol

Bei der euklidischen Division $a:b$ ist $(a\bmod b)$ eine geläufige
Schreibweise für den Rest. Das Lemma zur euklidischen Division sagt uns,
dass jede Restklasse von $\Z/m\Z$ einen kanonischen Repräsentant
besitzt. Nämlich besitzt die Restklasse $[a]_m$ den kanonischen
Repräsentant $r=(a\bmod m)$, denn $a=mq+r$ bedeutet dass $a-r$ durch
$m$ teilbar ist, also
\[a\equiv r\pmod m,\quad\text{bzw.}\quad [a]_m = [r]_m.\]

