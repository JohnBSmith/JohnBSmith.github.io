
\chapter{Zahlentheorie}

\section{Kongruenzen}

\begin{Definition}[Kongruenz]\index{Kongruenz}
Zwei ganze Zahlen $a,b$ heißen kongruent modulo $m$, wenn ihre Differenz
$(b-a)$ durch $m$ teilbar ist:%
\[a\equiv b\pmod{m}\defiff (\exists k\in\Z)(b-a=km).\]
\end{Definition}
Anstelle von »$(\mathrm{mod}\;m)$« schreibt man beim Rechnen meist
kürzer »$(m)$«.

\begin{Satz}
Die Kongruenz ist eine Äquivalenzrelation, d.\,h. es gilt
\begin{align*}
&a\equiv a\pmod{m},&&\text{(Reflexivität)}\\
&a\equiv b\implies b\equiv a\pmod{m},&&\text{(Symmetrie)}\\
&a\equiv b\land b\equiv c\implies a\equiv c\pmod{m}.&&\text{(Transitivität)}
\end{align*}
\end{Satz}
\strong{Beweis.} Für die Reflexivität ist ein $k$ mit $0=a-a=km$
zu finden. Setze $k=0$.

Bei der Symmetrie gibt es nach Voraussetzung
ein $k$ mit $b-a=km$. Dann ist $a-b=-km$. Setze $k'=-k$.
Es gibt also $k'$ mit $a-b=k'm$, somit gilt $b\equiv a$.

Bei der Transitivität gibt es nach Voraussetzung $k$ mit
$b-a=km$ und $l$ mit $b-c=lm$. D.\,h. es gilt
\[b = a+km = c+lm\implies c-a = km-lm = (k-l)m.\]
Setze $k'=k-l$. Es gibt also $k'$ mit $c-a=k'm$. Somit gilt $a\equiv c$.\;\qedsymbol

\begin{Satz}
Sind $a,b,c$ ganze Zahlen, dann gilt
\begin{align*}
a\equiv b\pmod{m}&\iff a+c\equiv b+c\pmod{m},\\
a\equiv b\pmod{m}&\iff a-c\equiv b-c\pmod{m}.
\end{align*}
\end{Satz}
\strong{Beweis.}
Unter Beachtung von $(b+c)-(a+c)=b-a$ findet man
\begin{gather*}
a\equiv b\pmod{m}
\iff (\exists k\in\Z)(b-a=km)\\
\iff (\exists k\in\Z)((b+c)-(a+c)=km)\\
\iff a+c\equiv b+c\pmod{m}.
\end{gather*}
Für die Subtraktion von $c$ ist die Überlegung analog.\;\qedsymbol

\begin{Satz}
Sind $a,b,c$ ganze Zahlen, dann gilt
\[a\equiv b\pmod{m} \implies ac\equiv bc\pmod{m}.\]
\end{Satz}
\strong{Beweis.}
Unter der Voraussetzung $a\equiv b\pmod{m}$ gibt es ein
$k$ mit $b-a=km$. Es gilt
\[b-a=km\iff (b-a)c=kcm \iff bc-ac=k'm\]
mit $k':=kc$. Man hat also
\[(\exists k'\in\Z)(bc-ac=k'm)\iff ac\equiv bc\pmod{m}.\;\qedsymbol\]

\begin{Satz}
Gilt $a\equiv a'\pmod{m}$ und
$b\equiv b'\pmod{m}$, dann gilt auch
\begin{align*}
a+b&\equiv a'+b'\pmod{m},\\
a-b&\equiv a'-b'\pmod{m},\\
ab&\equiv a'b'\pmod{m}.
\end{align*}
\end{Satz}
\strong{Beweis.} Man findet
\begin{equation}
\left.\begin{aligned}
a\equiv a'&\implies a+b\equiv a'+b\\
b\equiv b'&\implies a'+b\equiv a'+b'
\end{aligned}\right\}
\implies a+b\equiv a'+b\equiv a'+b'\pmod{m}.
\end{equation}
Für die Subtraktion ist die Überlegung analog. Für die Multiplikation
ebenfalls:%
\begin{equation}
\left.\begin{aligned}
a\equiv a'&\implies ab\equiv a'b\\
b\equiv b'&\implies a'b\equiv a'b'
\end{aligned}\right\}
\implies ab\equiv a'b\equiv a'b'\pmod{m}.\;\qedsymbol
\end{equation}

\begin{Satz}
Addition des Moduls führt auf eine kongruente Zahl:%
\[a\equiv a+m\equiv a-m\pmod{m}.\]
\end{Satz}
\strong{Beweis.}
Es gilt
\[a\equiv a+m\pmod{m}\iff (\exists k\in\Z)(km=(a+m)-a=m).\]
Setze $k=1$. Bei
\[a\equiv a-m\pmod{m}\iff (\exists k\in\Z)(km=(a-m)-a=-m)\]
setze $k=-1$.\;\qedsymbol
