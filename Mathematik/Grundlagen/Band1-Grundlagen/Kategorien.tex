
\chapter{Kategorientheorie}

\section{Grundbegriffe}

\begin{Definition}[Kategorie]
Eine Kategorie ist ein Tripel $C=(\mathrm{Ob},\mathrm{Hom},{\circ})$,
sofern die folgenden beiden Axiome erfüllt sind:
\begin{enumerate}
\item Für $f\colon A\to B$, $g\colon B\to C$, $h\colon C\to D$ gilt
das Assoziativgesetz $h\circ (g\circ f) = (h\circ g)\circ f$.
\item Für jedes Objekt $X$ existiert die Identität $\id_X\colon X\to X$,
so dass $f\circ\id_A = \id_B\circ f=f$ für alle Objekte $A,B$
und $f\colon A\to B$.
\end{enumerate}
\end{Definition}
Die Elemente der Klasse $\mathrm{Ob}$ nennt man Objekte. Die
Elemente der Klasse $\mathrm{Hom}$ nennt man Morphismen. Die
Verknüpfung $g\circ f$, sprich $g$ nach $f$, nennt man Verkettung
von $g$ und $f$.

Die Schreibweise ist $f\colon X\to Y$ gleichbedeutend mit
$f\in\operatorname{Hom}(X,Y)$, wobei $X,Y\in\mathrm{Ob}$.
Mit $\operatorname{Hom}(X,Y)$ ist die Teilklasse von
$\mathrm{Hom}$ gemeint, die alle Morphismen von $X$ nach $Y$
enthält. Man schreibt $\dom(f) = X$ und $\cod(f) = Y$.

Nun gut, man macht hier zunächst zwei Beobachtungen. Erstens
erinnern die Axiome an die Monoid"=Axiome, haben aber den Unterschied,
dass die Morphismen kompatibel sein müssen. D.\,h. um $g\circ f$
bilden zu können, muss $\cod(f)=\dom(g)$ sein.

Zweitens erinnern die Axiome an die Regeln für die Verkettung
von Abbildungen. Tatsächlich bilden die Abbildungen eine Kategorie.

\begin{Satz}[Kategorie der Mengen]\mbox{}\\*
Sei $\Omega$ das Mengenuniversum und für $A,B\in\Omega$ sei
$\mathrm{Hom}(A,B):=\mathrm{Abb}(A,B)$. Sei $g\circ f$ die Verkettung
von Abbildungen. Dann bildet $\strong{Set}:=(\Omega,\mathrm{Hom},\circ)$
eine Kategorie.
\end{Satz}
\strong{Beweis.} Trivial.\;\qedsymbol

\begin{Satz}[Kategorie der Gruppen]\mbox{}\\*
Sei $\Omega$ die Klasse aller Gruppen und für $G,H\in\Omega$ sei
$\mathrm{Hom}(G,H)$ die Klasse der Homomorphismen von $G$ nach $H$.
Sei $g\circ f$ die Verkettung von Homomorphismuen.
Dann bildet $\strong{Group}:=(\Omega,\mathrm{Hom},\circ)$
eine Kategorie.
\end{Satz}
\strong{Beweis.} Homomorphismen sind Abbildungen, die Axiome
daher wie bei der Kategorie der Mengen erfüllt. Die Verkettung
zweier Homomorphismen ist ja auch ein Homomorphismus.\;\qedsymbol

Entsprechend bilden Ringe mit Ringhomomorphismen, Körper mit
Körperhomomorphismen, Vektorräume mit Vektorraumhomomorphismen
usw. Kategorien.

Nun ist es so, dass Gruppen auch Mengen und Homomorphismen
auch Abbildungen sind. Die Kategorie der Gruppen ist gewissermaßen
in der Kategorie der Mengen enthalten. Um das zu präzisieren,
benötigen wir den Begriff des Vergissfunktors.

\begin{Definition}[Kovarianter Funktor]\mbox{}\\*
Sind $C,D$ Kategorien, dann nennt man $F\colon C\to D$ einen
kovarianten Funktor, wenn jedem Objekt $X\in\mathrm{Ob}(C)$ ein Objekt
$F(X)\in\mathrm{Ob}(D)$ zugeordnet wird und jedem Morphismus
$f\in\mathrm{Hom}_C(X,Y)$ ein ein Morphismus
$F(f)\in\mathrm{Hom}_D(F(X),F(Y))$ zugeordnet wird,
so dass die folgenden beiden Verträglichkeitsaxiome erfüllt sind:
\begin{gather*}
F(g\circ f) = F(g)\circ F(f),\\
F(\id_X) = \id_{F(X)}.
\end{gather*}
\end{Definition}
\begin{Definition}[Kontravarianter Funktor]\mbox{}\\*
Wie beim kovarianten Funktor, mit dem Unterschied
$F(g\circ f) = F(f)\circ F(g)$.
\end{Definition}
Bemerkung: Die Notation ist überladen. Nämlich ist die Zuordnung
$F\colon\mathrm{Ob}(C)\to\mathrm{Ob}(D)$ zu unterscheiden
von
\[\tilde F\colon\mathrm{Hom}_C(X,Y)\to\mathrm{Hom}_D(F(X),F(Y)).\]
Das Paar $(F,\tilde F)$ kodiert dann eigentlich den Funktor
$C\to D$.

\begin{Satz}[Vergissfunktor]\mbox{}\\*
Sei $F\colon\strong{Group}\to\strong{Set}$ mit $F((G,*,e)):=G$,
und jedem Gruppenhomomorphismus
\[\varphi\colon (G,*,e)\to (G',*',e')\]
sei die Abbildunge $F(\varphi)\colon G\to G'$ mit
$F(\varphi)(x):=\varphi(x)$ zugeordnet. Bei $F$ handelt
es sich um einen kovarianten Funktor.
\end{Satz}
\strong{Beweis.}
Es gilt $F(\id)(x) = \id(x)$, und daher $F(\id)=\id$.
Außerdem gilt
\begin{gather*}
F(\varphi_2\circ\varphi_1)(x) = (\varphi_2\circ\varphi_1)(x)
= \varphi_2(\varphi_1(x))
= F(\varphi_2)(F(\varphi_1)(x))
= (F(\varphi_2)\circ F(\varphi_1))(x),
\end{gather*}
und daher $F(\varphi_2\circ\varphi_1)
= F(\varphi_2)\circ F(\varphi_1)$.\;\qedsymbol
