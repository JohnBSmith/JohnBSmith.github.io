
\chapter{Kategorientheorie}

\section{Grundbegriffe}

\begin{Definition}[Kategorie]
Eine Kategorie ist ein Tripel $C=(\mathrm{Ob},\mathrm{Hom},{\circ})$,
sofern die folgenden beiden Axiome erfüllt sind:
\begin{enumerate}
\item Für $f\colon A\to B$, $g\colon B\to C$, $h\colon C\to D$ gilt
das Assoziativgesetz $h\circ (g\circ f) = (h\circ g)\circ f$.
\item Für jedes Objekt $X$ existiert die Identität $\id_X\colon X\to X$,
so dass $f\circ\id_A = \id_B\circ f=f$ für alle Objekte $A,B$
und $f\colon A\to B$.
\end{enumerate}
\end{Definition}
Die Elemente der Klasse $\mathrm{Ob}$ nennt man Objekte. Die
Elemente der Klasse $\mathrm{Hom}$ nennt man Morphismen. Die
Verknüpfung $g\circ f$, sprich $g$ nach $f$, nennt man Verkettung
von $g$ und $f$.

Die Schreibweise ist $f\colon X\to Y$ gleichbedeutend mit
$f\in\operatorname{Hom}(X,Y)$, wobei $X,Y\in\mathrm{Ob}$.
Mit $\operatorname{Hom}(X,Y)$ ist die Teilklasse von
$\mathrm{Hom}$ gemeint, die alle Morphismen von $X$ nach $Y$
enthält. Man schreibt $\dom(f) = X$ und $\cod(f) = Y$.

Nun gut, man macht hier zunächst zwei Beobachtungen. Erstens
erinnern die Axiome an die Monoid"=Axiome, haben aber den Unterschied,
dass die Morphismen kompatibel sein müssen. D.\,h. um $g\circ f$
bilden zu können, muss $\cod(f)=\dom(g)$ sein.

Zweitens erinnern die Axiome an die Regeln für die Verkettung
von Abbildungen. Tatsächlich bilden die Abbildungen eine Kategorie.

\begin{Satz}[Kategorie der Mengen]\mbox{}\\*
Sei $\Omega$ das Mengenuniversum und für $A,B\in\Omega$ sei
$\mathrm{Hom}(A,B):=\mathrm{Abb}(A,B)$. Sei $g\circ f$ die Verkettung
von Abbildungen. Dann bildet $\strong{Set}:=(\Omega,\mathrm{Hom},\circ)$
eine Kategorie.
\end{Satz}
\strong{Beweis.} Trivial.\;\qedsymbol

\begin{Satz}[Kategorie der Gruppen]\mbox{}\\*
Sei $\Omega$ die Klasse aller Gruppen und für $G,H\in\Omega$ sei
$\mathrm{Hom}(G,H)$ die Klasse der Homomorphismen von $G$ nach $H$.
Sei $g\circ f$ die Verkettung von Homomorphismen.
Dann bildet $\strong{Group}:=(\Omega,\mathrm{Hom},\circ)$
eine Kategorie.
\end{Satz}
\strong{Beweis.} Homomorphismen sind Abbildungen, die Axiome
daher wie bei der Kategorie der Mengen erfüllt. Die Verkettung
zweier Homomorphismen ist ja auch ein Homomorphismus.\;\qedsymbol

Entsprechend bilden Ringe mit Ringhomomorphismen, Körper mit
Körperhomomorphismen, Vektorräume mit Vektorraumhomomorphismen
usw. Kategorien.

Nun ist es so, dass Gruppen auch Mengen und Homomorphismen
auch Abbildungen sind. Die Kategorie der Gruppen ist gewissermaßen
in der Kategorie der Mengen enthalten. Um das zu präzisieren,
benötigen wir den Begriff des Vergissfunktors.

\begin{Definition}[Kovarianter Funktor]\mbox{}\\*
Sind $C,D$ Kategorien, dann nennt man $F\colon C\to D$ einen
kovarianten Funktor, wenn jedem Objekt $X\in\mathrm{Ob}(C)$ ein Objekt
$F(X)\in\mathrm{Ob}(D)$ zugeordnet wird und jedem Morphismus
$f\in\mathrm{Hom}_C(X,Y)$ ein ein Morphismus
$F(f)\in\mathrm{Hom}_D(F(X),F(Y))$ zugeordnet wird,
so dass die folgenden beiden Verträglichkeitsaxiome erfüllt sind:%
\begin{gather*}
F(g\circ f) = F(g)\circ F(f),\\
F(\id_X) = \id_{F(X)}.
\end{gather*}
\end{Definition}
\begin{Definition}[Kontravarianter Funktor]\mbox{}\\*
Wie beim kovarianten Funktor, mit dem Unterschied
$F(g\circ f) = F(f)\circ F(g)$.
\end{Definition}
Bemerkung: Die Notation ist überladen. Nämlich ist die Zuordnung
$F\colon\mathrm{Ob}(C)\to\mathrm{Ob}(D)$ zu unterscheiden
von
\[\tilde F\colon\mathrm{Hom}_C(X,Y)\to\mathrm{Hom}_D(F(X),F(Y)).\]
Das Paar $(F,\tilde F)$ kodiert dann eigentlich den Funktor
$C\to D$.

\begin{Satz}[Vergissfunktor]\mbox{}\\*
Sei $F\colon\strong{Group}\to\strong{Set}$ mit $F((G,*,e)):=G$,
und jedem Gruppenhomomorphismus%
\[\varphi\colon (G,*,e)\to (G',*',e')\]
sei die Abbildung $F(\varphi)\colon G\to G'$ mit
$F(\varphi)(x):=\varphi(x)$ zugeordnet. Bei $F$ handelt
es sich um einen kovarianten Funktor.
\end{Satz}
\strong{Beweis.}
Es gilt $F(\id)(x) = \id(x)$, und daher $F(\id)=\id$.
Außerdem gilt%
\begin{gather*}
F(\varphi_2\circ\varphi_1)(x) = (\varphi_2\circ\varphi_1)(x)
= \varphi_2(\varphi_1(x))
= F(\varphi_2)(F(\varphi_1)(x))
= (F(\varphi_2)\circ F(\varphi_1))(x),
\end{gather*}
und daher $F(\varphi_2\circ\varphi_1)
= F(\varphi_2)\circ F(\varphi_1)$.\;\qedsymbol

\begin{Satz} Sei $P(X)=2^X$ die Potenzmenge von $X$. Dann ist
wie folgt ein kovarianter Funktor gegeben:
\[P\colon\strong{Set}\to\strong{Set},\quad
P(X):=2^X,\quad P(f)(M):=f(M),\]
wobei $f$ eine beliebige Abbildung
und $f(M)$ die Bildmenge von $M$ unter $f$ ist.
\end{Satz}
\strong{Beweis.} Nach Satz \ref{img-comp} gilt
\begin{gather*}
P(g\circ f)(M) = (g\circ f)(M) = g(f(M))
= P(g)(P(f)(M)) = (P(g)\circ P(f))(M).
\end{gather*}
Daher ist $P(g\circ f)=P(g)\circ P(f)$. Außerdem ist
\[P(\id_X)(M) = \id_X(M) = M = \id_{P(X)}(M)\]
und daher $P(\id_X)=\id_{P(X)}$.\;\qedsymbol

Zum Funktor $P$ kommt noch ein weiterer Aspekt hinzu.
Für eine Abbildung $f$ kann man ganz pedantisch das
Bild $f(x)$ von der Bildmenge $f(\{x\})$ unterscheiden.
Aufgrund der Gleichung $f(\{x\})=\{f(x)\}$ verschwimmt diese
Unterscheidung aber gewissermaßen.
Die Abbildungen $f$ und $P(f)$ verhalten sich also
gewissermaßen gleich. Man kann sagen, dass $f$
auf ganz natürliche Art und Weise die Abbildung $P(f)$
zugeordnet ist. Definiert man
\[\eta(X)\colon X\to 2^X,\quad \eta(X)(x):=\{x\},\]
dann kommutiert das folgende Diagramm:
\[\xymatrix{
X \ar[r]^f \ar[d]_{\eta(X)} & Y \ar[d]^{\eta(Y)} \\
2^X \ar[r]_{P(f)} & 2^Y }\]
D.\,h. es gilt $\eta(Y)\circ f = P(f)\circ\eta(X)$.
Die Zuordnung $\eta$ ist eine sogenannte natürliche Transformation.

\begin{Definition}[Natürliche Transformation]
Seien $C,D$ Kategorien und $F,G\colon C\to D$ Funktoren.
Dann schreibt man $\eta\colon F\to G$ und nennt $\eta$ natürliche
Transformation, wenn die folgenden beiden Axiome erfüllt sind:
\begin{enumerate}
\item Jedes Objekt $X\in\mathrm{Ob}(C)$ bekommt einen Morphismus
$\eta(X)\colon F(X)\to G(X)$.
\item Für jeden Morphismus $f\colon X\to Y$ gilt
$\eta(Y)\circ F(f)=G(f)\circ\eta(X)$.
\end{enumerate}
\end{Definition}
Die zweite Bedingung lässt sich übersichtlich als kommutierendes Diagramm
darstellen:
\[\xymatrix{
F(X) \ar[r]^{F(f)} \ar[d]_{\eta(X)} & F(Y) \ar[d]^{\eta(Y)} \\
G(X) \ar[r]_{G(f)} & G(Y)}\]
Ein weiteres Beispiel ergibt sich bezüglich Äquivalenzrelationen
in Erinnerung an \eqref{eq:induzierte-Abbildung}.
Eine Abbildung $f\colon M\to M'$ heiße \emph{induzierend}, wenn%
\begin{equation}
x\sim a \implies f(x)\sim' f(a).
\end{equation}
\begin{Satz}
Die Paare $(M,\sim)$, bestehend aus Menge und Äquivalenzrelation,
bilden mit den induzierenden Abbildungen
als Morphismen bezüglich Verkettung eine Kategorie.
\end{Satz}
\strong{Beweis.}
Die identische Abbildung ist offensichtlich induzierend. Hat man
neben $f\colon M\to M'$ eine weitere induzierende Abbildung $g\colon M'\to M''$, dann
folgt $g(y)\sim'' g(b)$ aus $y\sim' b$. Aus $x\sim a$ folgt
mit $y:=f(x)$ und $b:=f(a)$ somit $g(f(x))\sim'' g(f(x))$.
Daher ist auch $g\circ f$ induzierend.\;\qedsymbol

Genau dann wenn $f$ induzierend ist, existiert eine induzierte
Abbildung
\begin{equation}
I(f)\colon M/\sim\to M'/\sim',\;\text{so dass}\;I(f)\circ\pi = \pi'\circ f,
\end{equation}
wobei $\pi,\pi'$ jeweils die kanonische Projektion ist.
\begin{Satz}
Bei der Induktion $I$ handelt es sich um einen kovarianten
Funktor.
\end{Satz}
\strong{Beweis.} Man betrachte das folgende kommutierende Diagramm:
\[\xymatrix{
M \ar[r]^{f} \ar[d]_{\pi}
& M' \ar[r]^{g} \ar[d]^{\pi'}
& M'' \ar[d]^{\pi''}\\
M/\sim \ar[r]_{I(f)}
& M'/\sim' \ar[r]_{I(g)}
& M''/\sim''}\]
Die Induktion $I$ besitzt die Eigenschaften
\begin{gather}
I(f)\circ\pi = \pi'\circ f,\\
I(g)\circ\pi' = \pi''\circ g,\\
I(g\circ f)\circ\pi = \pi''\circ (g\circ f).
\end{gather}
Damit kann man nun rechnen
\begin{equation}
I(g\circ f)\circ\pi = \pi''\circ g\circ f
= I(g)\circ\pi'\circ f = I(g)\circ I(f)\circ\pi.
\end{equation}
Infolge gilt $I(g\circ f)=I(g)\circ I(f)$, da die kanonische
Projektion $\pi$ eine Surjektion ist. Aus der Forderung $I(\id)\circ\pi
= \pi\circ\id = \pi$ ergibt sich $I(\id) = \id$,
da $\pi$ surjektiv ist.\;\qedsymbol

Die Abbildung $\eta((M,\sim)):=\pi$, die jeder Menge mit
Äquivalenzrelation ihre kanonische Projektion zuordnet,
ist eine natürliche Transformation.
