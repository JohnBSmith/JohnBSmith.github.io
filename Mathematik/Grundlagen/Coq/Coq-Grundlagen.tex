\documentclass[a4paper,10pt,fleqn,twocolumn,twoside,dvipdfmx]{scrartcl}
\usepackage[utf8]{inputenc}
\usepackage[T1]{fontenc}
\usepackage[ngerman]{babel}
\usepackage{microtype}

% Bugfix. Siehe https://komascript.de/node/2295
\unsettoc{toc}{onecolumn}

\usepackage{amsmath}
\usepackage{amssymb}
\usepackage{booktabs}
\usepackage{listings}
\lstset{basicstyle=\ttfamily\small,commentstyle=\ttfamily}

\usepackage{libertine}
\usepackage[libertine,smallerops]{newtxmath}
% \addtokomafont{disposition}{\rmfamily}
% \renewcommand\ttdefault{lmtt}
\usepackage[scaled=0.84]{DejaVuSans}
\usepackage[scaled=0.84]{DejaVuSansMono}

\usepackage{graphicx}
\usepackage{color}
\definecolor{c1}{RGB}{60,0,40}

\usepackage{geometry}
\geometry{a4paper,left=25mm,right=12mm,top=20mm,bottom=28mm}
\setlength{\columnsep}{6mm}

\numberwithin{equation}{section}

\usepackage[colorlinks=true,linkcolor=black,citecolor=black,%
urlcolor=black]{hyperref}

\newcommand{\R}{\mathbb R}
\newcommand{\ee}{\mathrm e}
\newcommand{\unit}[1]{\mathrm{#1}}
\newcommand{\strong}[1]{{\sffamily\bfseries #1}}

\lstdefinelanguage{Coq}{
  morekeywords = {Definition, fun, match, with, end, Fixpoint, struct,
    where, Inductive, forall},
  sensitive=false,
  morecomment=[s]{(*}{*)},
  morestring=[b]",
}

\begin{document}

%\thispagestyle{empty}

\section*{\huge Grundlagen von Coq}

\vspace{1em}\noindent
{\large Oktober 2021}

\tableofcontents

\subsection*{Vorwort}

Im Laufe der letzten Jahrhunderte scheint sich in unserer Welt mit der
Vertiefung der Erkenntnis und dem neu hinzugekommenen Wissen immer mehr
Komplexität aufgetan zu haben. Insbesondere kommen in der Mathematik
immer wieder verwickelte Sätze und Theorien hinzu, bei deren Beweisen
ein unaufhörliches Wachstum der Länge zu beobachten ist. Eine ähnliche
Schwierigkeit keimt in der Informatik auf, wo mit der wachsenden
Vielfalt an Anforderungen die Länge von Programmen fortlaufend
zunimmt.

Aus diesem Grund kommen wir letztendlich irgendwann in den Zugzwang,
dieser Komplexität auf irgendeine Art Einhalt gebieten zu müssen.

Der folgende Text erkläutert, wie sich unter Zuhilfenahme von Computern
mathematische Beweise verfizieren lassen. 
Grundlegend dafür ist der Curry"=Howard"=Isomorphismus, ein
fundementaler Zusammenhang zwischen mathematischen Aussagen
und Typsignaturen. Hierauf baut der Konstruktionenkalkül auf,
mit dessen Erweiterung, dem Kalkül der induktiven Konstruktionen,
man auf systematische Art Beweise in Programmterme umwandeln kann.

Wie läuft das ab? Zu jeder mathematischen Aussage gehört eine
äquivalente Typsignatur. Findet man nun einen Programmterm, welcher
einen Wert des Typs berechnet, so ist der Typ offenbar nichtleer.
Diese Feststellung bedeutet aber nichts anderes, als dass der Beweis
der ursprünglichen Aussage erbracht ist.

Damit diese Argumentation stichhaltig bleibt, darf der Term
ausschließlich \emph{reine} Funktionen enthalten. Das sind solche
Funktionen, bei denen der Rückgabewert ausschließlich von den
Argumenten abhängig ist. Die Berechnung läuft hierbei strikt
funktional ab, -- damit ist gemeint dass alle Daten unveränderlich
sind und die Berechnung somit frei von Seiteneffekten ist.
Außerdem muss jede Berechnung stets terminieren, darf sich also niemals
in einer Endlosschleife verfangen. Mathematisch gesprochen bedeutet
dies, dass jede auftretende Funktion \emph{total} ist, dass also keine
partiellen Funktionen vorkommen. Beide Forderungen gemeinsam,
Reinheit und Totalität, bedeuten, dass wir es mit Funktionen
im mathematischen Sinn zu tun haben.

Schließlich muss das Typsystem reichhaltig genug sein, um alle
erdenklichen Aussagen kodieren zu können. Hierzu bedarf es der
\emph{parametrischen Polymorphie}, d.\,h. der Parametrisierung
von Typen über Typparameter, die bei vielen modernen
Programmiersprachen Grundlage für die generische Programmierung
bildet. Darüber hinaus sind \emph{abhängige Typen} notwendig, die die
Parametrisierung so erweitern, dass als Typparameter nicht nur Typen,
sondern auch Werte erlaubt sind.

Der vertrauenswürdige Programmkern, die \emph{trusted computing base},
ist klein. Es ist dieser Flaschenhals, der uns ruhiger Schlafen lässt.

Ich hoffe der Text ist auch horizonterweiternd, wenn man nun nicht
jeden Beweis mit diesem System verfizieren möchte. So handelt es
sich hier um eine Berührregion zwischen Mathematik und Informatik.

\section{Gallinas Typsystem}
\subsection{Programmieren in Gallina}

\section{Logik}

\subsection{Modus ponens}

Wir wollen den Modus ponens%
\begin{equation}
A\land (A\rightarrow B)\rightarrow B
\end{equation}
beweisen. Der Typ zu dieser Aussage ist%
\begin{equation}
A\times (A\rightarrow B)\rightarrow B.
\end{equation}
Die Aussage ist wahr, wenn dieser Typ mindestens einen Wert enhält.
Gesucht ist also eine Funktion, die einen Wert vom Typ $A$ und
eine Funktion vom Typ $A\rightarrow B$ nimmt und einen Wert vom
Typ $B$ liefert. Die kann man direkt angeben:%
\begin{equation}
(a, f)\mapsto f(a).
\end{equation}
Die freien Variablen müssen wir noch allquantifizieren, damit später
im Programm keine ungebundenen Variablen mehr auftauchen. Das macht%
\begin{equation}
\forall_{A, B}(A\times (A\to B)\to B).
\end{equation}
Ein Wert dieses Typs ist entsprechend%
\begin{equation}
(A, B) \mapsto (a, f)\colon A\times (A\to B)\mapsto f(a).
\end{equation}
In Coq geschrieben lautet der Term:%
\begin{lstlisting}[language=Coq]
Definition modus_ponens:
forall A B: Type, A*(A -> B) -> B :=
  fun (A B: Type) =>
    fun (t: A*(A -> B)) =>
      match t with (a, f) => f a end.
\end{lstlisting}
Allgemein gilt die äquivalente Umformung%
\begin{equation}
\begin{aligned}
& A\land B\to C \iff \overline{A\land B}\lor C
\iff\overline A\lor\overline B\lor C\\
&\iff A\to \overline B\lor C
\iff A\to B\to C.
\end{aligned}
\end{equation}
Dies entspricht dem Schönfinkeln der Funktion. Demnach kodiert
der Typ%
\begin{equation}
A\to (A\to B)\to B.
\end{equation}
ebenfalls den Modus ponens. In Coq sind geschönfinkelte Formulierungen
natürlicher. Das sieht man hier am Entfallen des match"=Operators:%
\begin{lstlisting}[language=Coq]
Definition modus_ponens:
forall A B: Type, A -> (A -> B) -> B :=
  fun (A B: Type) =>
    fun (a: A) =>
      fun (f: A -> B) => f a.
\end{lstlisting}
Nun sind in Coq Kurzschreibweisen erlaubt, die das Schönfinkeln
syntaktisch rückgängig machen:%
\begin{lstlisting}[language=Coq]
Definition modus_ponens:
forall A B: Type, A -> (A -> B) -> B :=
  fun (A B: Type) (a: A) (f: A -> B) => f a.
\end{lstlisting}

\subsection{Formeln der Aussagenlogik}

Betrachten wir $A\land B\to A$, der Typ dazu ist
\[A\times B\to B.\]
Ein Wert dieses Typs ist ja einfach die Projektion auf das linke
Element, also
\[(a, b)\mapsto a.\]
Das Programm:
\begin{lstlisting}[language=Coq]
Definition left: forall A B: Type, A*B -> B :=
  fun (A B: Type) (t: A*B) =>
    match t with (a, b) => a end.
\end{lstlisting}
Zum Kommutativgesetz $A\land B\to B\land A$ gehört der Typ
\begin{equation}
A\times B\to B\times A.
\end{equation}
Ein Wert dieses Typs ist die Funktion, welches linkes und rechtes
Element vertauscht, das ist
\[(a, b)\mapsto (b, a).\]
Das Programm ist entsprechend:
\begin{lstlisting}[language=Coq]
Definition conjunction_commutativity:
forall A B: Type, A*B -> B*A :=
  fun (A B: Type) (t: A*B) =>
    match t with (a, b) => (b, a) end.
\end{lstlisting}
Zum Kommutativgesetz $A\lor B\to B\lor A$ gehört der Typ
\begin{equation}
A+B\to B+A.
\end{equation}
Eine Funktion dieses Typs ergibt sich durch Fallunterscheidung.
Wir finden
\begin{equation}
\begin{cases}
  (\mathrm{left},a)  \mapsto (\mathrm{right},a),\\
  (\operatorname{right},b) \mapsto (\operatorname{left},b).\end{cases}
\end{equation}
Das Programm:
\begin{lstlisting}[language=Coq]
Definition disjunction_commutativity:
forall A B: Type, A + B -> B + A :=
  fun (A B: Type) (s: A + B) =>
    match s with
      | inl a => inr a
      | inr b => inl b
    end.
\end{lstlisting}
Betrachten wir noch $A\to A\lor B$ bzw. $A\to A+B$. Eine Funktion
dieses Typs ist offenbar die Injektion mit Bild im linken Summand,
das ist
\begin{equation}
a\mapsto (\mathrm{left},a).
\end{equation}
Das Programm:
\begin{lstlisting}[language=Coq]
Definition injection_left:
forall A B: Type, A -> A + B :=
  fun (A B: Type) (a: A) => inl a.
\end{lstlisting}
Zeigen wir nun noch das Assoziativgesetz
\begin{equation}
A\land (B\land C) \leftrightarrow (A\land B)\land C.
\end{equation}
Wir betrachten lediglich
\begin{equation}
A\times (B\times C) \to (A\times B)\times C,
\end{equation}
denn die Gegenimplikation geht analog.
Eine Funktion von diesem Typ ist
\begin{equation}
(a, (b, c))\mapsto ((a, b), c).
\end{equation}
Das Programm:
\begin{lstlisting}[language=Coq]
Definition conjunction_assoc1:
forall A B C: Type, A*(B*C) -> (A*B)*C :=
  fun (A B C: Type) (t: A*(B*C)) =>
    match t with (a, (b, c)) => ((a, b), c) end.
\end{lstlisting}

\subsection{Umstieg auf Prop}

Nun ersetzen wir \texttt{Type} durch \texttt{Prop}.
Hierfür sind einige kleine Änderungen notwendig.

Betrachten wir nochmals $A\land B\to P$. Wir müssen $A\times B$
durch $A\land B$ ersetzen. Die Tupelkonstruktion
$(a,b)$ wird ersetzt durch $\operatorname{conj} a\, b$.
Insgesamt bekommt das Programm damit die Form:

\begin{lstlisting}[language=Coq]
Definition left: forall A B: Prop, A /\ B -> A :=
  fun (A B: Prop) (t: A /\ B) =>
    match t with conj a b => a end.
\end{lstlisting}

\noindent
Und nun betrachten wir nochmals $A\lor B\to B\lor A$.
Hier ersetzen wir $A+B$ durch $A\lor B$.
Der Konstruktor \texttt{inl} wird ersetzt durch
\texttt{or\_introl} und \texttt{inr} entsprechend durch
\texttt{or\_intror}. Das Programm ist nun von der Gestalt:

\begin{lstlisting}[language=Coq]
Definition disjunction_commutativity:
forall A B: Prop, A \/ B -> B \/ A :=
  fun (A B: Prop) (s: A \/ B) =>
    match s with
      | or_introl a => or_intror a
      | or_intror b => or_introl b
    end.
\end{lstlisting}

\newpage
\begin{thebibliography}{00}
\bibitem{ProofObjects} Benjamin C. Pierce et al.:
»\href{https://softwarefoundations.cis.upenn.edu/lf-current/index.html}{%
\emph{Software Foundations. Volume 1: Logical Foundations}}«.
\end{thebibliography}


\vfill
\noindent
{\small Dieser Text steht unter der Lizenz\\
Creative Commons CC0 1.0.}

\end{document}
