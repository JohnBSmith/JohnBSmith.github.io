\documentclass[a4paper,11pt,fleqn,twocolumn,twoside]{article}
\usepackage[utf8]{inputenc}
\usepackage[T1]{fontenc}

\usepackage{libertine}
\usepackage[libertine,cmintegrals]{newtxmath}
\renewcommand\ttdefault{lmtt}

\usepackage{ngerman}
\usepackage{amsmath}
\usepackage{amssymb}
\usepackage{color}
\definecolor{c1}{RGB}{00,40,60}
\usepackage[colorlinks=true,linkcolor=c1]{hyperref}
\usepackage{geometry}
\geometry{a4paper,left=25mm,right=15mm,top=20mm,bottom=28mm}
\setlength{\columnsep}{4mm}
\numberwithin{equation}{section}

\begin{document}
\thispagestyle{empty}

\begin{huge}
\noindent
\textbf{Grundlagen\\
der Mathematik}
\par
\end{huge}

\tableofcontents

\section{Formale Sprachen}
\subsection{Objektsprache, Metasprache}
Hinfort ist es sinnvoll zwischen Objektsprache und Metasprache
zu unterscheiden. Die Ausdrücke einer Objektsprache werden als
Objekte behandelt über die man spricht. Die Metasprache ist die
Sprache in der man über diese Objekte spricht. Zur Unterscheidung
von werden Ausdrücke der Objektsprache zunächst in Anführungszeichen
gesetzt. Demnach lässt z.\,B. der folgende metasprachliche
Satz formulieren:

Die Aussage »Heute hat es geregnet.« war gestern wahr gewesen.

Natürlich kann es mehrere Ebenen von Metasprachen geben:

Die Aussage »Die Aussage »Heute hat es geregnet.« war gestern wahr
gewesen.« wahr gestern wahr gewesen.

\subsection{Formale Sprachen}

Wir können auf ein Blatt Papier oder in den Computer eine Zeichenkette
aus Symbolen schreiben. Um genau zu sein, müssen wir uns zunächst
überlegen welche Symbole erlaut sein sollen. Die erlaubten Symbole
fassen wir zu einer Menge $\Sigma$ zusammen und bezeichnen diese
als \emph{Alphabet}. Aus den Symbolen lassen sich nun Ketten
von Symbolen bilden, die wir als \emph{Wörter} bezeichnen.
Ein Wort ist demnach ein Tupel
\begin{equation}
a := (a_1,a_2,\ldots,a_n), \qquad (a_k\in\Sigma)
\end{equation}
das in vielen Fällen auch kurz als $a_1a_2\ldots a_n$ geschrieben wird.
Z.\,B. schreiben wir »Boot« anstelle von
$(\text{B},\text{o},\text{o},\text{t})$. Dazu passend wird die
Konkatenation zweier Wörter $a$ und $b$ als $ab$ geschrieben.
Also
\begin{equation}
ab := (a_1,\ldots,a_m,b_1,\ldots,b_n)
\end{equation}
für $a=(a_1,\ldots,a_m)$ und $b=(b_1,\ldots,b_n)$.

Die Konkatenation von mehr als zwei Wörtern ist ganz klar und lässt
sich präzise rekursiv definieren:
\begin{equation}
\operatorname{cat}_{k=1}^n [a^{(k)}]
:= \operatorname{cat}_{k=1}^{n-1} [a^{(k)}]a^{(n)}.
\end{equation}
Man definiert auch noch das sogenannte leere Wort $\varepsilon$
als leeres Tupel, kurz $\varepsilon:=()$. Das leere Wort ist zugleich
der Wert der leeren Konkatenation:
\begin{equation}
\operatorname{cat}_{k=1}^0 [a^{(k)}] := \varepsilon.
\end{equation}
Verschiedene Wörter lassen sich nun zu einer Menge $L$
zusammenfassen, die man als \emph{Sprache} bezeichnet.

Die Menge aller möglichen Konkatenationen (einschließlich der leeren
Konkatenation) von Symbolen aus $\Sigma$ bezeichnet man als
\emph{kleensche Hülle} $\Sigma^*$. Schließt man man die leere
Konkatenation aus, so spricht man von der \emph{positiven kleenschen
Hülle} $\Sigma^+$.

Jede Sprache $L$ über dem Alphabet $\Sigma$ ist demnach eine Teilmenge
$L\subseteq\Sigma^*$.

\subsection{Abstrakte Syntaxbäume}

Wenn ein mathematischer Ausdruck syntaktisch analysiert
wurde, kann das Ergebnis dieser Analyse als ein sogenannter
\emph{abstrakter Syntaxbaum}, kurz \emph{AST} für engl.
\emph{abstract syntax tree} dargestellt werden.

Dem Ausdruck »$2x$« ordnen wir z.\,B. den AST
\begin{verbatim}
  (*,2,x)
\end{verbatim}
zu und »$2x+4$« ist dementsprechend
\begin{verbatim}
  (+,(*,2,x),4).
\end{verbatim}
Hier ein paar Beispiele:

1. $f(x)$ wird zu \verb|(f,x)|,

2. $f(x,y)$ wird zu \verb|(f,x,y)|,

3. $x^2+a$ wird zu \verb|(+,(^,x,2),a)|,

4. $2(a+b)$ wird zu \verb|(*,2,(+,a,b))|,

5. $a+b+c$ wird zu \verb|(+,(+,a,b),c)|,

6. $a+b+c$ wird zu \verb|(+,a,b,c)|,

7. $x^2=x$ wird zu \verb|(=,(^,x,2),x)|,

8. $A\land B$ wird zu \verb|(and,A,B)|.

\noindent
Bei der Darstellung von abstrakten Syntaxbäumen im Computer gibt
es zunächst zwei Varianten. In Lisp wird der AST zum Ausdruck »$f(x,y)$«
als verkettete Liste
\begin{verbatim}
  (f x y)
\end{verbatim}
dargestellt. Hier kann nämlich mit \texttt{first}
(auch \texttt{car} genannt) der Funktionsbezeichner \texttt{f}
erhalten werden und mit \texttt{rest} (auch \texttt{cdr} genannt)
die Liste der Argumente.

Bei der Verwendung von dynamischen Feldern anstelle von verketteten
Listen sind die Operationen \texttt{first} und \texttt{rest} eventuell
etwas ineffizienter. Die Darstellung ist z.\,B.:
\begin{verbatim}
  ["f", "x", "y"]
\end{verbatim}
In einer effizienten statisch typisierten
Programmiersprache ließe sich natürlich ein Datentyp für AST-Knoten
formulieren, der auf extra die Bedürfnisse, welche sich ergeben,
zugeschnitten ist.

\section{Lambda-Kalkül}
\subsection{Variablensubstitution}
Angenommen, bei \texttt{t} und \texttt{u} handelt es sich um Ausdrücke
bzw. abstrakte Syntaxbäume. Dann bedeutet die Schreibweise
\begin{verbatim}
  t [x:=u]
\end{verbatim}
die Ersetzung jedes Vorkommens der Variable \texttt{x} im Baum
\texttt{t} durch den Baum \texttt{u}. Verwendet man keine abstrakten
Syntaxbäume, so müssen dabei jedoch im Zusammenhang mit Vorrangregeln
die Regeln der Klammersetzung beachtet werden. Z.\,B. ist
\begin{equation}
  (a*b)\;[b:=x+y] = a*(x+y)
\end{equation}
und nicht $a*x+y$.

Eine solche Variablensubstitution soll außerdem nur für
Variablen durchgeführt werden, welche \emph{frei}, d.\,h. ungebunden
vorkommen. Man spricht daher von einer
\emph{freien Variablensubstitution}. Variablen können ausschließlich
durch lambda-Ausdrücke gebunden werden. Was ein lambda-Ausdruck ist,
wird später erklärt.

\subsection{Lambda-Ausdrücke}
Einer Liste wie \verb|[a,b,c,d]| ordnen wir den AST
\begin{verbatim}
  ([],a,b,c,d)
\end{verbatim}
zu. Somit gehört zur einelementigen Liste \verb|[x]| der Baum
\verb|([],x)|. Zur leeren Liste \verb|[]| gehört der Baum \verb|([])|,
wobei die runden Klammern obligatorisch sind.
Der Begriff \emph{Liste} ist gleichbedeutend mit \emph{Tupel}.

Einen AST der Form
\begin{verbatim}
  (lambda, ([], x), t)
\end{verbatim}
bzw. einen Ausdruck
\begin{equation}
\lambda([x],t)
\end{equation}
wobei \texttt{t} ein beliebiger AST ist,
wollen wir als lambda-Ausdruck bezeichnen. Anstelle von
»$\lambda([x],t)$« schreiben wir kürzer »$|x|\;t$«. Alternative
Schreibweisen sind »$\lambda x.\,t$« oder »$x\mapsto t$«.

Dabei verlangt man nun, dass $|x|$ schwächer bindet als alle anderen
Operationen und rechtsassoziativ ist. Rechtsassoziativ bedeutet, dass
die Klammerung immer auf der rechten Seite gesetzt wird. Z.\,B. ist
\begin{equation}
|x|\;|y|\; x+y = |x|\;(|y|\;(x+y)).
\end{equation}
Ein lambda-Ausdruck lässt sich nun auf einen AST \emph{applizieren}.
Man definiert
\begin{equation}
(|x|\; t)(u) \;:=\; (t\;[x:=u]).
\end{equation}
Z.\,B. ist
\begin{equation}
(|x|\;x^2)(4) = 4^2 = 16
\end{equation}
und
\begin{equation}
\begin{split}
& (|x,y|\;x\cdot y)(a+b,c+d)\\
&= (x\cdot y)\;[x:=a+b,\;y:=c+d]\\
&= (a+b)(c+d).
\end{split}
\end{equation}
Applikation ist linksassoziativ. D.\,h. es gilt
\begin{equation}
f(x)(y) = (f(x))(y).
\end{equation}
Man kann nun die freie Variablensubstitution präzise definieren.
Für Ausdrücke $s,t,u$ und Variablen $x,y$ gelten folgende Regeln:

(s1) $x\;[x:=u] = u$.

(s2) $y\;[x:=u] = y$ wenn $x\ne y$.

(s3) $s(t)\;[x:=u] = (s[x:=u])(t[x:=u])$.

(s4) $(|x|\;t)\;[x:=u] = |x|\;t$.

(s5) $(|y|\;t)\;[x:=u] = |y|\;(t[x:=u])$ wenn $x\ne y$
und $\mathrm{FV}(u)$ disjunkt zu $\mathrm{BV}(|y|\;t)$ ist.

Mit $\mathrm{FV}(t)$ bezeichnet man die Menge der \emph{freien Variablen}
von $t$, engl. \emph{free variables}. Dieser Operator ist wie folgt
definiert:

1. $\mathrm{FV}(x) = \{x\}$ für jede Variable $x$.

2. $\mathrm{FV}(s(t)) = \mathrm{FV}(s)\cup\mathrm{FV}(t)$.

3. $\mathrm{FV}(|x|\; t) = \mathrm{FV}(t)\setminus\{x\}$.

\noindent
Mit $\mathrm{BV}(t)$ bezeichnet man die Menge der \emph{gebundenen
Variablen} von $t$, engl. \emph{bounded variables}.
Dieser Operator ist wie folgt definiert:

1. $\mathrm{BV}(x) = \{\}$ für jede Variable $x$.

2. $\mathrm{BV}(s(t)) = \mathrm{BV}(s)\cup\mathrm{BV}(t)$.

3. $\mathrm{BV}(|x|\; t) = \mathrm{BV}(t)\cup\{x\}$.

\noindent
Entsprechend können wir noch $\mathrm{AV}(t)$, die Menge aller
Variablen von $t$ definieren:

1. $\mathrm{AV}(x) = \{x\}$ für jede Variable $x$.

2. $\mathrm{AV}(s(t)) = \mathrm{AV}(s)\cup\mathrm{AV}(t)$.

3. $\mathrm{AV}(|x|\; t) = \mathrm{AV}(t)\cup\{x\}$.

% \newpage
\noindent
Außerdem definieren wir noch $\mathrm{PV}(t)$, die Menge der
in Ausdrücken präsenten Variablen, wie folgt:

1. $\mathrm{PV}(x) = \{x\}$ für jede Variable $x$.

2. $\mathrm{PV}(s(t)) = \mathrm{PV}(s)\cup\mathrm{PV}(t)$.

3. $\mathrm{PV}(|x|\; t) = \mathrm{PV}(t)$.

\noindent
Man beachte dass eine Variable in einem Ausdruck sowohl
frei als auch gebunden vorkommen kann. Mit anderen
Worten: Eine disjunkte Zerlegung von $\mathrm{AV}(t)$
in $\mathrm{FV}(t)$ und $\mathrm{BV}(t)$ ist nicht immer möglich.

Nun gibt es auch lambda-Ausdrücke mit mehr als einem Argument.
Wir können nun
\begin{equation}
(|x,y|\; t) := (|x|\;|y|\;t)
\end{equation}
definieren. Man bezeichnet eine solche Umformung als \emph{Currying}
oder \emph{Schönfinkeln}.

Bei mehr als einem Argument muss man in solchen Fällen aufpassen,
wo die gleichen Variablen sowohl frei als auch gebunden vorkommen.
Der Ausdruck
\begin{equation}
(|x,y|\; x+y)(y,x)
\end{equation}
unterscheidet sich z.\,B. von
\begin{equation}
((x+y)[x:=y])[y:=x].
\end{equation}
Hier müssen beide Substitutionen »gleichzeitig« vorgenommen werden.
Präziser gesagt müssen die gebundenen Variablen vor der Substitution
umbenannt werden. Betrachten wir nämlich den geschönfinkelten
Ausdruck
\begin{equation}
(|x|\;|y|\;x+y)(y)(x).
\end{equation}
Hier erhält man nach Definition nun
\begin{equation}
((|y| x+y)[x:=y])(x),
\end{equation}
aber die Substitution kann wegen Regel (s5) nicht ausgeführt
werden, denn $\mathrm{FV}(y)=\{y\}$ ist nicht disjunkt zu
$\mathrm{BV}(|y|\;x+y)=\{y\}$. Benennt man die gebundene Variable
$y$ nun in $a$ um, so ergibt sich
\begin{equation}
\begin{split}
&((|a|\;x+a)[x:=y])(x)\\
&= (|a|\;y+a)(x) = y+x.
\end{split}
\end{equation}
Wir hatten gesagt, Variablen können ausschließlich durch
lambda-Ausdrücke gebunden werden. Variablenbindungen ohne
lambda-Bindung müssen immer in solche mit lambda-Bindung umformuliert
werden. Z.\,B. ist
\begin{equation}
\sum_{k=m}^n a_k = \mathrm{sum}(m,n,|k|\;a_k)
\end{equation}
und
\begin{equation}
\int_a^b f(x)\,\mathrm dx = \mathrm{integral}(a,b,|x|\;f(x)).
\end{equation}
Für die Quantoren der Prädikatenlogik ergibt sich:
\begin{gather}
\forall x{\in}M\,[P(x)] = \mathrm{all}(M,|x|\;P(x)),\\
\exists x{\in}M\,[P(x)] = \mathrm{any}(M,|x|\;P(x)).
\end{gather}
Das bedeutet beim Allquantor z.\,B., dass der Baum
\begin{verbatim}
  (all, (in,x,M), (P,x))
\end{verbatim}
zum Baum
\begin{verbatim}
  (all, M, (lambda, ([],x), (P,x)))
\end{verbatim}
umformuliert wird.

\subsection{Auswertung}

Bei der Auswertung von lambda-Termen kann es sein, dass wir in
einen sogenannten \emph{infiniten Regress} gelangen. Man betrachte
z.\,B.
\begin{verbatim}
  Regress = (|x| x(x))(|x| x(x)).
\end{verbatim}
Die Auswertung führt wieder auf den ursprünglichen Term, wird also
niemals enden. Oder man betrachte:
\begin{verbatim}
  (|x| 1+x(x))(|x| 1+x(x))
= 1+(|x| 1+x(x))(|x| 1+x(x))
= 1+(1+(|x| 1+x(x))(|x| 1+x(x)))
= usw.
= 1+(1+(1+(...))).
\end{verbatim}
Das Ergebnis dieser Auswertung wäre ein unendlich großer AST.
Es ist nun aber so, dass bei einer bestimmten Auswertung ein AST
nicht benötigt wird, weil er bei der Auswertung einfach entfällt.
Solches kann z.\,B. bei einer Projektion geschehen. Die Funktion
\begin{verbatim}
  p := |x| |y| x
\end{verbatim}
projiziert auf das erste Argument. Daher ist
\begin{verbatim}
  p(0)(Regress) = 0,
\end{verbatim}
obwohl der problematische Term \texttt{Regress} vorkommt.
Wertet man zuerst \texttt{Regress} aus, so gelangt man in den
infiniten Regress. Wertet man hier aber von links nach rechts aus,
so gelangt man zu
\begin{verbatim}
  (|x| |y| x)(0)(Regress)
  = (|y| 0)(Regress) = 0.
\end{verbatim}
Wir wollen hierbei von einer sogenannten \emph{Lazy Evaluation}
(dt. \emph{Bedarfsauswertung}) sprechen, die sich von der
normalen \emph{Eager Evaluation} unterscheidet.

Normalerweise werden in $p(x,y)$ die Argumente $x,y$ auszuwertet
und danach wird $p$ auf die Werte appliziert. Der Ansatz für die
alternative Vorgehensweise ist wie folgt: Zuerst wird $p$ auf die
Argumente appliziert, wobei jedes Vorkommen einer Variablen in $p$
gegen eine Referenz (einen »Zeiger«) auf den entsprechenden AST
ausgetauscht wird. Nun wird das Ergebnis weiter ausgewertet, wobei
problematische Terme eventuell durch Projektionen verloren gegangen
sind.

Ein Problem hierbei ist jedoch, dass sich der Rechenaufwand bei
mehrfachem Vorkommen einer Variablen erhöht. Man tut daher folgendes:
Bei der ersten Auswertung des ASTs wird dieser für alle Referenzen
gegen das Ergebnis der Auswertung ausgetauscht.

\end{document}





