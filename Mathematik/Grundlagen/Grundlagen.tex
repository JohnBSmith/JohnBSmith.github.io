\documentclass[a4paper,11pt,fleqn,twocolumn,twoside]{article}
\usepackage[utf8]{inputenc}
\usepackage[T1]{fontenc}
% \usepackage{lmodern}
\usepackage{mathptmx}
\usepackage{ngerman}
\usepackage{amsmath}
\usepackage{amssymb}
\usepackage{color}
\definecolor{c1}{RGB}{00,40,60}
\usepackage[colorlinks=true,linkcolor=c1]{hyperref}
\usepackage{geometry}
\geometry{a4paper,left=25mm,right=15mm,top=20mm,bottom=28mm}
\setlength{\columnsep}{6mm}
\numberwithin{equation}{section}

\begin{document}
\thispagestyle{empty}

\begin{huge}
\noindent
\textbf{Grundlagen\\
der Mathematik}
\par
\end{huge}

\tableofcontents

\section{Formale Sprachen}
\subsection{Abstrakte Syntaxbäume}

Wenn ein mathematischer Ausdruck syntaktisch analysiert
wurde, kann das Ergebnis dieser Analyse als ein sogenannter
\emph{abstrakter Syntaxbaum}, kurz \emph{AST} für engl.
\emph{abstract syntax tree} dargestellt werden.

Dem Ausdruck »$2x$« ordnen wir z.\,B. den AST
\begin{verbatim}
  (*,2,x)
\end{verbatim}
zu und »$2x+4$« ist dementsprechend
\begin{verbatim}
  (+,(*,2,x),4).
\end{verbatim}
Hier ein paar Beispiele:

1. $f(x)$ wird zu \verb|(f,x)|,

2. $f(x,y)$ wird zu \verb|(f,x,y)|,

3. $x^2+a$ wird zu \verb|(+,(^,x,2),a)|,

4. $2(a+b)$ wird zu \verb|(*,2,(+,a,b))|,

5. $a+b+c$ wird zu \verb|(+,(+,a,b),c)|,

6. $a+b+c$ wird zu \verb|(+,a,b,c)|,

7. $x^2=x$ wird zu \verb|(=,(^,x,2),x)|,

8. $A\land B$ wird zu \verb|(and,A,B)|.

\noindent
Bei der Darstellung von abstrakten Syntaxbäumen im Computer gibt
es zunächst zwei Varianten. In Lisp wird der AST zum Ausdruck »$f(x,y)$«
als verkettete Liste
\begin{verbatim}
  (f x y)
\end{verbatim}
dargestellt. Hier kann nämlich mit \texttt{first}
(auch \texttt{car} genannt) der Funktionsbezeichner \texttt{f}
erhalten werden und mit \texttt{rest} (auch \texttt{cdr} genannt)
die Liste der Argumente.

Bei der Verwendung von dynamischen Feldern anstelle von verketteten
Listen sind die Operationen \texttt{first} und \texttt{rest} eventuell
etwas ineffizienter. Hier ist die Darstellung
\begin{verbatim}
  ["f", ["x", "y"]]
\end{verbatim}
sinnvoller. In einer effizienten statisch typisierten
Programmiersprache ließe sich natürlich ein Datentyp für AST-Knoten
formulieren, der auf extra die Bedürfnisse, welche sich ergeben,
zugeschnitten ist.

\section{Lambda-Kalkül}
\subsection{Variablensubstitution}
Angenommen, bei \texttt{t} und \texttt{u} handelt es sich um Ausdrücke
bzw. abstrakte Syntaxbäume. Dann bedeutet die Schreibweise
\begin{verbatim}
  t [x:=u]
\end{verbatim}
die Ersetzung jedes Vorkommens der Variable \texttt{x} im Baum
\texttt{t} durch den Baum \texttt{u}. Verwendet man keine abstrakten
Syntaxbäume, so müssen dabei jedoch im Zusammenhang mit Vorrangregeln
die Regeln der Klammersetzung beachtet werden. Z.\,B. ist
\begin{equation}
  a*b\;[b:=x+y] = a*(x+y)
\end{equation}
und nicht $a*x+y$.

Eine solche Variablensubstitution soll außerdem nur für
Variablen durchgeführt werden, welche \emph{frei}, d.\,h. ungebunden
vorkommen. Man spricht daher von einer
\emph{freien Variablensubstitution}. Variablen können ausschließlich
durch lambda-Ausdrücke gebunden werden. Was ein lambda-Ausdruck ist,
wird später erklärt.

\subsection{Lambda-Ausdrücke}
Einer Liste wie \verb|[a,b,c,d]| ordnen wir den AST
\begin{verbatim}
  ([],a,b,c,d)
\end{verbatim}
zu. Somit gehört zur einelementigen Liste \verb|[x]| der Baum
\verb|([],x)|. Zur leeren Liste \verb|[]| gehört der Baum \verb|([])|,
wobei die runden Klammern obligatorisch sind.
Der Begriff \emph{Liste} ist gleichbedeutend mit \emph{Tupel}.

Einen AST der Form
\begin{verbatim}
  (lambda, ([], x), t)
\end{verbatim}
bzw. einen Ausdruck
\begin{equation}
\lambda([x],t)
\end{equation}
wobei \texttt{t} ein beliebiger AST ist,
wollen wir als lambda-Ausdruck bezeichnen. Anstelle von
»$\lambda([x],t)$« schreiben wir kürzer »$|x|\;t$«. Alternative
Schreibweisen sind »$\lambda x.\,t$« oder »$x\mapsto t$«.

Dabei verlangt man nun, dass $|x|$ schwächer bindet als alle anderen
Operationen und rechtsassoziativ ist. Rechtsassoziativ bedeutet, dass
die Klammerung immer auf der rechten Seite gesetzt wird. Z.\,B. ist
\begin{equation}
|x|\;|y|\; x+y = |x|\;(|y|\;x+y).
\end{equation}
Ein lambda-Ausdruck lässt sich nun auf einen AST \emph{applizieren}.
Man definiert
\begin{equation}
(|x|\; t)(u) \;=\; t\;[x:=u].
\end{equation}
Z.\,B. ist
\begin{equation}
(|x|\;x^2)(4) = 4^2 = 16
\end{equation}
und
\begin{equation}
\begin{split}
& (|x,y|\;x\cdot y)(a+b,c+d)\\
&= x\cdot y\;[x:=a+b,\;y:=c+d]\\
&= (a+b)(c+d).
\end{split}
\end{equation}
Applikation ist linksassoziativ. D.\,h. es gilt
\begin{equation}
f(x)(y) = (f(x))(y).
\end{equation}

\end{document}





