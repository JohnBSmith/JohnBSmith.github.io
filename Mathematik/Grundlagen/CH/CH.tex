\documentclass[9pt]{beamer}
\usetheme{Antibes}
\useinnertheme{rectangles}
\useoutertheme{infolines}
\usepackage[utf8]{inputenc}
\usepackage[T1]{fontenc}
\usepackage[ngerman]{babel}

% Patch the look of +, = in arev
\usefonttheme{serif}

\usepackage{arev}

% Patch punctuation to be upright
\DeclareMathSymbol{.}{\mathpunct}{operators}{`.}
\DeclareMathSymbol{,}{\mathpunct}{operators}{`,}

\usepackage{amsmath}
\usepackage{amssymb}
\usepackage{booktabs}

\setbeamertemplate{footline}{%
\begin{beamercolorbox}[ht=3.0ex,dp=1ex]{title in head/foot}
\hfill\footnotesize\insertpagenumber\enspace\enspace\end{beamercolorbox}}

\definecolor{brown1}{rgb}{0.26,0.14,0}
\definecolor{brown2}{rgb}{0.26,0.14,0}
\definecolor{urlcolor0}{rgb}{0.8,0,0.5}
\setbeamercolor*{palette primary}{fg=white,bg=brown1}
\setbeamercolor*{palette secondary}{fg=white,bg=brown2}
\setbeamercolor*{palette tertiary}{fg=white,bg=brown2}
\setbeamercolor{itemize item}{fg=black,bg=white}
\newcommand{\modest}[1]{{\small\color{gray}#1}}
\hypersetup{colorlinks=true,urlcolor=urlcolor0}

\newcommand{\N}{\mathbb N}
\newcommand{\id}{\operatorname{id}}
\newcommand{\Abb}{\operatorname{Abb}}
\newcommand{\strong}[1]{\textsf{\textbf{#1}}}
\newcommand{\defiff}{\quad:\Longleftrightarrow\quad}
\newcommand{\kw}[1]{\textbf{#1}}

\title{Aussagen als Typen}
\subtitle{Curry-Howard für Eilige}
\date{}

\begin{document}

\begin{frame}
\maketitle
\end{frame}

\begin{frame}
In der Aussagenlogik sind zusammengesetzte Aussagen aus einem
Satz von Grundverknüpfungen konstruierbar, meist UND, ODER, NICHT.\pause

\vspace{1em}
Betrachten wir $A\land B$.\pause{} Wertetabelle:

\begin{table}
\begin{tabular}{ccc}
$A$ & $B$ & $A\land B$\\
\midrule
0 & 0 & 0\\
0 & 1 & 0\\
1 & 0 & 0\\
1 & 1 & 1
\end{tabular}
\end{table}\pause

Betrachten wir $A$, $B$ als Zahlen und multiplizieren diese,
kommt doch dasselbe Ergebnis bei heraus.
\end{frame}

\begin{frame}
Wie verhält es sich dann mit $A\lor B$?\pause{} Addieren?\pause

\vspace{1em}
Wertetabelle:
\begin{table}
\begin{tabular}{cccc}
$A$ & $B$ & $A\lor B$ & $A + B$\\
\midrule
0 & 0 & 0 & 0\\
0 & 1 & 1 & 1\\
1 & 0 & 1 & 1\\
1 & 1 & 1 & 2
\end{tabular}
\end{table}\pause
Wir könnten $A+B - AB$ rechnen, dann stimmt es.\pause

\vspace{1em}
Aber angenommen, uns würde nur interessieren ob da null herauskommt
oder nicht, dann könnten wir uns die Subtraktion von $AB$ sparen.
\end{frame}

\begin{frame}
Wir teilen die natürlichen Zahlen in zwei Äquivalenzklassen auf,
einmal $0$ für falsch, und die restlichen Zahlen $x\ne 0$
für wahr.\pause

\vspace{1em}
Dann klappt es:
\begin{table}
\begin{tabular}{cccc}
$A$ & $B$ & $AB$ & $A+B$\\
\midrule
0 & 0 & 0 & 0\\
0 & x & 0 & x\\
x & 0 & 0 & x\\
x & x & x & x
\end{tabular}
\end{table}
(In der Tabelle steht $x$ für die Äquivalenzklasse.)
\end{frame}

\begin{frame}
Eine entsprechende Operation gibt es auch für die
Implikation $A\to B$.\pause

\vspace{1em}
Na?\pause{} Das ist $B^A$.\pause{} Wertetabelle:
\begin{table}
\begin{tabular}{cccc}
$A$ & $B$ & $B^A$ & $A\to B$\\
\midrule
0 & 0 & 1 & 1\\
0 & x & 1 & 1\\
x & 0 & 0 & 0\\
x & x & x & 1
\end{tabular}
\end{table}\pause

Tolle Spielerei. Und was bringt uns das jetzt?
\end{frame}

\begin{frame}
Aus der Mengenlehre, speziell der Kardinalzahltheorie ist bekannt,
dass zu jeder dieser Operationen eine einfache Mengenkonstruktion
gehört.\pause

\vspace{1em}
Die Entsprechungen sind:
\begin{table}
\begin{tabular}{ccc}
Aussagen & Zahlen & Mengen\\
\midrule
$A\land B$ & $AB$  & $A\times B$ \\
$A\lor B$  & $A+B$ & $A\sqcup B$\\
$A\to B$   & $B^A$ & $\Abb(A,B)$
\end{tabular}
\end{table}\pause

Mit $A\sqcup B$ ist die disjunkte Vereinigung
\[(\{0\}\times A)\cup (\{1\}\times B) \cong 
(\{\text{Grün}\}\times A)\cup (\{\text{Blau}\}\times B)\]
gemeint. Das heißt, die Elemente werden vorher eingefärbt,
damit bekannt bleibt, aus welcher der Mengen ein Element der
Vereinigung entstammt.\pause

\vspace{1em}
Mit $\Abb(A,B)$ ist die Menge der Abbildungen von $A$ nach $B$
gemeint.
\end{frame}

\begin{frame}
Eine Menge kann die leere Menge $\emptyset$ sein, oder eine
nichtleere Menge. Wir haben $|\emptyset|=0$ und
$|X|=x\ne 0$ für $X\ne\emptyset$.\pause

\vspace{1em}
Das ist genau das, was wir haben wollen, denn
\begin{gather*}
|A\times B| = |A|\cdot |B|,\\
|A\sqcup B| = |A| + |B|,\\
|\Abb(A,B)| = |B|^{|A|}.
\end{gather*}
\end{frame}

\begin{frame}
Für die Allquantifizierung $\forall_{x\in M}P(x)$ brauchen wir
unbedingt auch eine Entsprechung, wie später klar werden wird.\pause

\vspace{1em}
Sei $M$ zunächst endlich, sagen wir
\[M:=\{x_0,x_1,x_2,x_3\}.\]
Dann gilt
\[\forall_{x\in M}P(x) = P(x_0)\land P(x_1)\land P(x_2)\land P(x_3).
\]\pause
Die Entsprechung ist
\[P(x_0)\times P(x_1)\times P(x_2)\times P(x_3),\]
wobei $P\colon M\to\{0,1\}$ gegen $P'\colon M\to U$
zu ersetzen ist, so dass
\[P(x)\iff |P'(x)|\ne 0.\]
\end{frame}

\begin{frame}
Die Elemente dieses Produkts sind Tupel
\[(y_0, y_1, y_2, y_3).\]\pause
Ein solches Tupel können wir auch als Abbildung
\[\{0\mapsto y_0, 1\mapsto y_1, 2\mapsto y_2, 3\mapsto y_3\}\]
betrachten. Das ist schlicht die Indizierung des Tupels.\pause

\vspace{1em}
Die Menge $M$ enthält nun jedes Element genau einmal.
Dann können wir doch auch $M$ selbst als Indexmenge verwenden,
richtig? Das Tupel ist demnach kodierbar als
\[\{x_0\mapsto y_0, x_1\mapsto y_1, x_2\mapsto y_2, x_3\mapsto y_3\}.\]
\end{frame}

\begin{frame}
Das sind Abbildungen $x\mapsto y(x)$, wobei $x\in M$
und $y(x)\in P'(x)$.\pause

\vspace{1em}
So eine Konstruktion gibt es in der Mathematik tatsächlich.
Es handelt sich um das allgemeine Mengenprodukt
\[\prod_{x\in M} P'(x) := \{f\colon M\to \bigcup_{x\in M}P'(x)\mid \forall x\in M\colon f(x)\in P'(x)\}.\]
\end{frame}

\begin{frame}
Aber was bringt uns das denn jetzt?\pause

\vspace{1em}
Betrachten wir die aussagenlogische Formel
\[A\land B\to A.\]
Wir wollen beweisen dass das eine Tautologie ist.\pause

\vspace{1em}
Es handelt sich genau dann um eine Tautologie, wenn
\[\forall_A\forall_B (A\land B\to A)\]
eine wahre Aussage ist.\pause{} Das ist wiederum genau dann der
Fall, wenn die entsprechende Menge
\[\Pi_A\Pi_B (A\times B\to A)\]
nichtleer ist, wobei wir $A\times B\to A$ für $\Abb(A\times B,A)$ schreiben.
\end{frame}

\begin{frame}
Ein Element dieser Menge ist konstruierbar -- die Projektion
auf das linke Element, das ist
\[\mathrm{proj_0} := A\mapsto B\mapsto (a,b)\mapsto a\;\;\text{mit}\;\;
(a,b)\in A\times B,\]
bzw. anders geschrieben
\[\mathrm{proj_0}(A)(B)((a,b)) := a.\]\pause
Es gibt daher mindestens dieses Element $\mathrm{proj}_0$. Die Menge
ist somit nichtleer, die ursprüngliche Formel daher bewiesen.
\end{frame}

\begin{frame}
Irgendwie erinnert das an ein Programm, wie man es in einer
Programmiersprache konstruieren kann.\pause

\vspace{1em}
\strong{Standard ML}\\[4pt]
\texttt{\kw{val} proj0: 'a * 'b -> 'a = \kw{fn} (a, b) => a;}\\[8pt]

Alternativ:\\[4pt]

\texttt{\kw{fun} proj0((a, b): 'a*'b): 'a = a;}\pause

\vspace{1.4em}
\strong{Rust}\\[4pt]
\texttt{\kw{fn} proj0<A, B>((a, \_b): (A, B)) -> A \{a\}}
\end{frame}

\begin{frame}
Das heißt, den Mengen entsprechen Typen? Ein Typsystem
hilft uns dabei, korrekte Terme zu konstruieren. Dann hilft es
uns auch, korrekte Beweise zu konstruieren?\pause

\vspace{1em}
Ja, das geht. Die genannten Programmiersprachen schaffen das aber
nur teilweise. Wir wollen ja Funktionen im mathematischen
Sinn haben.\pause

\vspace{1em}
Das heißt:
\begin{itemize}
\item Die Funktion muss einem Argument immer denselben Wert
zuordnen, denn eine mathematische Funktion ordnet einem Element
\emph{genau einen} Wert zu. Eine solche Funktion nennt man
\emph{rein}, -- \emph{rein funktional}.

\item Die Funktion muss immer terminieren und damit einen
Wert liefern, denn eine mathematische Funktion ordnet \emph{jedem}
Element einen Wert zu. Eine solche Funktion nennt man
\emph{total}.
\end{itemize}\pause

{\small
Würden wir partielle Funktionen zulassen, könnte man zu jedem Typ
als Term ein Programm einfügen, welches sich in einer
Endlosschleife verfängt, und hätte damit sofort einen »Beweis« für jede
beliebige Aussage.}
\end{frame}

\begin{frame}
Zudem muss das Typsystem reichhaltig genug sein, um die
gewünschten Mengenkonstruktionen überhaupt ausdrücken zu können.\pause

\vspace{1em}
Das Typsystem muss erlauben:
\begin{itemize}
\item algebraische Typen,
\item abhängige Typen.
\end{itemize}\pause

\vspace{1em}
Abhängige Typen sind eine fortgeschrittene Form von parametrischer
Polymorphie, bei denen Typen nicht nur über Typen parametrisiert
werden können, sondern auch über Werte.
\end{frame}

\begin{frame}
Das heißt, wenn wir so eine pedantische Typprüfung bauen würden,
könnten wir\ldots?\pause

\vspace{1em}
Solche Systeme wurden bereits vor langer Zeit geschaffen. Der
Theorembeweisassistent \emph{Coq} enthält eine \emph{Gallina} genannte
Sprache, die zentraler Gegenstand des Programmkerns zur Verifikation
der konstruierten Beweise ist.\pause

\vspace{1em}
\strong{Gallina}\\[4pt]
\texttt{\kw{Definition} proj0: \kw{forall}(A B: Type), A*B -> A :=\\
\ \ \kw{fun} (A B: Type) (t: A*B) =>\\
\ \ \ \ \kw{match} t \kw{with} (a, b) => a \kw{end}.}
\end{frame}

\begin{frame}
Zwischen Aussagen und Typen wird eine enge Entsprechung erkennbar,
die im \emph{Curry-Howard-Isomorphismus} ihre Präzisierung findet.\pause
\begin{table}
\begin{tabular}{rl}
\toprule
\strong{Aussagen} & \strong{Typen}\\
\midrule
Konjunktion --- $A\land B$ & $A\times B$ --- Produkt\\
Disjunktion --- $A\lor B$ & $A+B$ --- Summe\\
Implikation --- $A\to B$ & $A\to B$ --- Funktionentyp\\
Allquant. --- $\forall_{x\in M} P(x)$ & $\Pi_{x\colon M} P(x)$ --- abhängiges Produkt\\
Existenzquant. --- $\exists_{x\in M} P(x)$ & $\Sigma_{x\colon M}P(x)$ --- abhängige Summe\\
\bottomrule
\end{tabular}
\end{table}\pause

Zum Beweis einer Aussage konstruiert man zum entsprechenden Typ
einen Term, dessen Wert von diesem Typ ist. Der Wert, den wir
\emph{Zeuge} nennen, belegt dass der Typ durch mindestens
einen Wert \emph{bewohnt} ist. Dies erbringt den Beweis der
ursprünglichen Aussage.
\end{frame}

\begin{frame}
\strong{Ausblick}

\vspace{1em}
Ausgelassen wurden die folgenden Themen:
\begin{itemize}
\item Summentypen und Pattern-Matching (Fallunterscheidung).
\item Die Negation $\neg A$ muss als $A\to 0$ kodiert werden.
\item Wie verhält es sich mit der Gleichheit?
\item Man erhält auf diese Weise intuitionistische Logik.
  Ist auch klassische Aussagenlogik möglich?
\end{itemize}
\end{frame}

\begin{frame}
\strong{Literatur}

\begin{itemize}
\item Stefan Müller-Stach: »\href{https://export.arxiv.org/abs/1805.05419}%
{Äquivalenz und Wahrheit}«, Kapitel 9:
»Typentheorie und ihre Semantik«.
\item Benjamin C. Pierce et al.:
»\href{https://softwarefoundations.cis.upenn.edu/lf-current/index.html}%
{Software Foundations. Volume 1: Logical Foundations}«.
\end{itemize}
\end{frame}

\begin{frame}
Ende.

\vfill
\modest{Dieser Text steht unter der Lizenz\\
Creative Commons CC0 1.0}
\end{frame}

\end{document}


