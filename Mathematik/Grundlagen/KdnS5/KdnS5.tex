\documentclass[8pt]{beamer}
\usetheme{Antibes}
\useinnertheme{rectangles}
\useoutertheme{infolines}
\usepackage[utf8]{inputenc}
\usepackage[T1]{fontenc}
\usepackage[ngerman]{babel}

% Patch the look of +, = in arev
\usefonttheme{serif}

\usepackage{arev}
% Patch punctuation to be upright
\DeclareMathSymbol{.}{\mathpunct}{operators}{`.}
\DeclareMathSymbol{,}{\mathpunct}{operators}{`,}

\usepackage{amsmath}
\usepackage{amssymb}
\usepackage{proof}
\usepackage{booktabs}

\setbeamertemplate{footline}{%
\begin{beamercolorbox}[ht=3.0ex,dp=1ex]{title in head/foot}
\hfill\footnotesize\insertpagenumber\enspace\enspace\end{beamercolorbox}}

\definecolor{bluegreen1}{rgb}{0.0,0.20,0.28}
\definecolor{bluegreen2}{rgb}{0.0,0.20,0.28}
\setbeamercolor*{palette primary}{fg=white,bg=bluegreen1}
\setbeamercolor*{palette secondary}{fg=white,bg=bluegreen2}
\setbeamercolor*{palette tertiary}{fg=white,bg=bluegreen2}
\setbeamercolor{itemize item}{fg=black}
\setbeamercolor{block title}{bg=bluegreen2}
\newcommand{\modest}[1]{{\small\color{gray}#1}}
\hypersetup{colorlinks,urlcolor=magenta}

\newcommand{\unit}[1]{\mathrm{#1}}
\newcommand{\strong}[1]{\textsf{\textbf{#1}}}
\newcommand{\defiff}{\quad:\Longleftrightarrow\quad}
\newcommand{\infernote}[1]{\!\text{\footnotesize #1}}
\renewcommand{\qedsymbol}{\ensuremath{\Box}}
\newcommand{\discharge}[1]{$\sim$#1}
\newcommand{\centerheadline}[1]{%
  \begin{center}\strong{#1}\end{center}}
\newcommand{\parspace}{\vspace{0.8em}}
\newcommand{\cond}{\rightarrow}
\newcommand{\bicond}{\leftrightarrow}
\newcommand{\lnec}{\Box}
\newcommand{\lpos}{\Diamond}

\title{Natürliches Schließen}
\subtitle{Teil 5: Modallogik}
\date{}

\begin{document}

\begin{frame}
\maketitle
\end{frame}

\begin{frame}
\centerheadline{Formeln der Modallogik}
\end{frame}

\begin{frame}
Die Modallogik fügt zu den Junktoren der Aussagenlogik zwei
einstellige Operatoren hinzu:

\begin{center}
\begin{tabular}{cl}
\toprule
\strong{Aussage} & \strong{Lesung}\\
\midrule[\heavyrulewidth]
$\lnec A$ & notwendigerweise $A$\\
$\lpos A$ & möglicherweise $A$\\
\bottomrule
\end{tabular}
\end{center}
Zur Syntax: Beiden kommt die höchste Operatorrangfolge zu, also
dieselbe wie der Negation.

\end{frame}

\begin{frame}
\centerheadline{System K}
\end{frame}

\begin{frame}
Es gibt unterschiedliche modallogische Systeme. Eine große Familie%
\footnote{Die Familie der normalen Modallogiken.}
baut auf dem System K auf, womit man K als grundlegend betrachten
kann. Insofern wollen wir zunächst K diskutieren, bevor wir
uns den anderen Systemen zuwenden.

\parspace
Das formale System K entsteht dadurch, dass sämtliche Regeln und Axiome
der Aussagenlogik weiterhin erhalten bleiben und diesen lediglich eine
weitere Regel (Regel~N) und ein weiteres Axiom%
\footnote{Eigentlich ein Axiomenschema,
da man beliebige Formeln einfügen darf.}
(Axiom~K) hinzugefügt werden.\pause

\parspace
\begin{block}{Regel N (Nezessisierungsregel)}
\[\dfrac{\vdash A}{\vdash\lnec A}\]
\end{block}
In Worten: Ist eine Formel ein Theorem, so soll deren
Notwendigkeit ebenfalls ein Theorem sein.
\end{frame}

\begin{frame}
\begin{block}{Axiom K (Axiom der Verteilung)}
\[\vdash \lnec (A\cond B) \cond (\lnec A\cond\lnec B)\]
\end{block}\pause
Wir dürfen aus diesem Axiom wie üblich per Modus ponens
eine zulässige Schlussregel ableiten:
\begin{block}{Regel K}
\[\dfrac{\Gamma\vdash \lnec (A\cond B)}{\Gamma\vdash\lnec A\cond\lnec B}\]
\end{block}
\end{frame}

\begin{frame}
Eine erste Aufgabe. Gesucht ist der Beweis von
\[\vdash \lnec (A\land B)\cond\lnec A.\]
Man findet:\pause

\[
\infer[\infernote{K}]{\vdash \lnec (A\land B)\cond\lnec A}{
  \infer[\infernote{N}]{\vdash\lnec (A\land B\cond A)}{
    \infer{\vdash A\land B\cond A}{
      \infer{A\land B\vdash A}{
        \infer{A\land B\vdash A\land B}{}}}}}
\]\pause

Frage: Ist $\lnec A\land\lnec B\cond\lnec (A\land B)$
ebenfalls beweisbar?\pause{} Ja, ist sie. Zur Ausführung benötigen wir
allerdings eine kurze Vorbereitung.
\end{frame}

\begin{frame}
Wir bestätigen zunächst einmal die Regel
\[\dfrac{\vdash A\cond (B\cond C)}{\vdash\lnec A\cond (\lnec B\cond\lnec C)}.\]
Es findet sich:\pause

\[
\infer{\vdash\lnec A\cond (\lnec B\cond\lnec C)}{
  \infer[\infernote{K}]{\lnec A\vdash\lnec B\cond\lnec C}{
    \infer{\lnec A\vdash\lnec (B\cond C)}{
      \infer[\infernote{K}]{\vdash\lnec A\cond\lnec (B\cond C)}{
        \infer[\infernote{N}]{\vdash\lnec (A\cond (B\cond C))}{
          \vdash A\cond (B\cond C)}}
    & \infer{\lnec A\vdash\lnec A}{}}}}
\]
\end{frame}

\begin{frame}
Beachtet man nun die allgemeine Äquivalenz von $A\land B\cond C$
und $A\cond (B\cond C)$, erhält man die Regel in der Form
\[\dfrac{\vdash A\land B\cond C}{\vdash\lnec A\land\lnec B\cond\lnec C}.\]
Hiermit gelingt die gesuchte Ableitung kurzerhand:\pause
\[
\infer{\vdash\lnec A\land\lnec B\cond\lnec (A\land B)}{
  \infer{\vdash A\land B\cond A\land B}{
    \infer{A\land B\vdash A\land B}{}}}
\]
\end{frame}

\begin{frame}
Im Fortgang zur gemachten Überlegung finden wir eine verallgemeinerte
Regel der Nezessisierung.
\begin{block}{Regel N*}
\[\dfrac{A_1,\ldots,A_n\vdash B}{\lnec A_1,\ldots,\lnec A_n\vdash\lnec B}\]
\end{block}\pause
\strong{Beweis.}
Induktion über $n$. Im Anfang $n=0$ nimmt die Regel schlicht die Form
der Nezessisierungsregel an. Den Induktionsschritt bestätigt der Beweisbaum:

\[
\infer{\lnec A_1,\ldots,\lnec A_n,\lnec A_{n+1}\vdash\lnec B}{
  \infer[\infernote{K}]{\lnec A_1,\ldots,\lnec A_n\vdash\lnec A_{n+1}\cond\lnec B}{
    \infer[\infernote{IV}]{\lnec A_1,\ldots,\lnec A_n\vdash\lnec (A_{n+1}\cond B)}{
      \infer{A_1,\ldots,A_n\vdash A_{n+1}\cond B}{
        A_1,\ldots,A_n,A_{n+1}\vdash B}}}
& \infer{\lnec A_{n+1}\vdash\lnec A_{n+1}}{}}
\]
\end{frame}

\begin{frame}
Nun zur Möglichkeit. Sie wird auf eine Formel mit einer
Notwendigkeit zurückgeführt.
\begin{block}{Definition der Möglichkeit}
\[\lpos A :\equiv \neg\lnec\neg A\]
\end{block}\pause

\vspace{2em}
Man bestätigt mühelos die Äquivalenz $\lpos\neg A\equiv\neg\lnec A$,
sofern die Beseitigung der Doppelnegation gewährt ist.\pause{} Nämlich
findet sich die äquivalente Umformung
\[\lpos\neg A\equiv\neg\lnec\neg\neg A\equiv\neg\lnec A.\]
Wir wollen die Äquivalenz aber nochmals durch Beweisbäume herstellen.\pause

\vspace{2em}
\begin{small}
\strong{Bemerkung.} Gleichermaßen erhält man $\neg\lpos\neg A\equiv\lnec A$
mit der Umformung
\[\neg\lpos\neg A\equiv\neg\neg\lnec\neg\neg A\equiv\lnec A.\]
\end{small}
\end{frame}

\begin{frame}
Zu $\lpos\neg A\cond\neg\lnec A$ findet sich:\pause

\[
\infer[\infernote{Def.}]{\vdash\lpos\neg A\cond\neg\lnec A}{
  \infer[\infernote{Kontraposition}]{\vdash\neg\lnec\neg\neg A\cond\neg\lnec A}{
    \infer[\infernote{K}]{\vdash\lnec A\cond\lnec\neg\neg A}{
      \infer[\infernote{N}]{\vdash\lnec (A\cond\neg\neg A)}{
        \infer[\infernote{bekanntlich}]{\vdash A\cond\neg\neg A}{}
      }
    }
  }}
\]\pause

Für $\neg\lnec A\cond\lpos\neg A$ benötigen wir nun die Beseitigung
der Doppelnegation:\pause

\[
\infer[\infernote{Def.}]{\vdash\neg\lnec A\cond\lpos\neg A}{
  \infer[\infernote{Kontraposition}]{\vdash\neg\lnec A\cond\neg\lnec\neg\neg A}{
    \infer[\infernote{K}]{\vdash\lnec\neg\neg A\cond\lnec A}{
      \infer[\infernote{N}]{\vdash\lnec(\neg\neg A\cond A)}{
        \infer[\infernote{DN}]{\vdash\neg\neg A\cond A}{}}}}}
\]
\end{frame}

\begin{frame}
Die Äquivalenz $\lnec\neg A\equiv\neg\lpos A$
ist ebenfalls unschwer zu bestätigen.

\parspace
Es findet sich:\pause

\[
\infer{\vdash \lnec\neg A\bicond\neg\lpos A}{
  \infer[\infernote{Def.}]{\vdash\lnec\neg A\cond\neg\lpos A}{
    \infer[\infernote{$B:=\lnec\neg A$}\quad]{\vdash\lnec\neg A\cond\neg\neg\lnec\neg A}{
      \infer[\infernote{bekanntlich}]{\vdash B\cond\neg\neg B}{}}}
& \infer[\infernote{Def.}]{\vdash\neg\lpos A\cond\lnec\neg A}{
    \infer[\infernote{$B:=\lnec\neg A$}]{\vdash\neg\neg\lnec\neg A\cond\lnec\neg A}{
      \infer[\infernote{DN}]{\vdash\neg\neg B\cond B}{}}}}
\]\pause

\vspace{2em}
\begin{small}
\strong{Bemerkung.}
Man kann diese Regeln als Analogon zu den de morganschen Gesetzen
$\neg\forall x\colon P(x)\equiv\exists x\colon\neg P(x)$
und $\neg\exists x\colon P(x)\equiv\forall x\colon\neg P(x)$
auffassen.
\end{small}
\end{frame}

\begin{frame}
Weiterhin zeigt sich die Formel $\lnec (A\cond B)\cond (\lpos A\cond\lpos B)$
als beweisbar.

\parspace
Man zieht hierfür zweimal die Regel der Kontraposition heran:\pause

\vspace{1em}
\[
\infer{\vdash\lnec (A\cond B)\cond (\lpos A\cond\lpos B)}{
  \infer[\infernote{Def.}]{\lnec (A\cond B)\vdash\lpos A\cond\lpos B}{
    \infer[\infernote{Kontraposition}]{\lnec (A\cond B)\vdash\neg\lnec\neg A\cond\neg\lnec\neg B}{
      \infer[\infernote{K}]{\lnec (A\cond B)\vdash\lnec\neg B\cond\lnec\neg A}{
        \infer{\lnec (A\cond B)\vdash\lnec (\neg B\cond\neg A)}{
          \infer{\lnec (A\cond B)\vdash\lnec (A\cond B)}{}
        & \infer[\infernote{K}]{\vdash\lnec (A\cond B)\cond\lnec (\neg B\cond\neg A)}{
            \infer[\infernote{N}]{\vdash\lnec ((A\cond B)\cond(\neg B\cond\neg A))}{
              \infer[\infernote{Kontraposition}]{\vdash (A\cond B)\cond (\neg B\cond\neg A)}{}}}}}}}}
\]
\end{frame}

\begin{frame}
\centerheadline{Zusätzliche Axiome}
\end{frame}

\begin{frame}
\begin{center}
\begin{tabular}{cl}
\toprule
\strong{Kürzel} & \strong{Axiom}\\
\midrule[\heavyrulewidth]
T & $\vdash\lnec A\cond A$\\
B & $\vdash A\cond\lnec\lpos A$\\
D & $\vdash\lnec A\cond\lpos A$\\
4 & $\vdash\lnec A\cond\lnec\lnec A$\\
5 & $\vdash\lpos A\cond\lnec\lpos A$\\
\bottomrule
\end{tabular}
\end{center}

\begin{center}
\begin{tabular}{cl}
\toprule
\strong{System} & \strong{Axiome}\\
\midrule[\heavyrulewidth]
K & K\\
T & K, T\\
B & K, T, B\\
D & K, D\\
S4 & K, T, 4\\
S5 & K, T, 5\\
\bottomrule
\end{tabular}
\end{center}
\end{frame}

\begin{frame}
Das Axiom T impliziert $A\cond\lpos A$. Nämlich findet sich:

\[
\infer{\vdash A\cond\lpos A}{
  \infer{A\vdash\lpos A}{
    \infer[\infernote{Def.}]{\vdash\neg\neg A\cond\lpos A}{
      \infer[\infernote{Kontraposition}]{\vdash\neg\neg A\cond\neg\lnec\neg A}{
        \infer[\infernote{$B:=\neg A$}]{\vdash\lnec\neg A\cond\neg A}{
          \infer[\infernote{T}]{\vdash\lnec B\cond B}{}}}}
  & \infer{A\vdash\neg\neg A}{}}}
\]\pause

Ist die Beseitigung der Doppelnegation gewährt, impliziert
$A\cond\lpos A$ umgekehrt das Axiom T. Nämlich findet sich:

\[
\infer{\vdash\lnec A\cond A}{
  \infer[\infernote{DN}]{\lnec A\vdash A}{
    \infer{\lnec A\vdash\neg\neg A}{
      \infer[\infernote{Kontraposition}]{\vdash\neg\lpos\neg A\cond\neg\neg A}{
        \infer[\infernote{$B:=\neg A$}]{\vdash\neg A\cond\lpos\neg A}{
          \vdash B\cond\lpos B}}
    & \infer[\infernote{Def.}]{\lnec A\vdash\neg\lpos\neg A}{
        \infer{\lnec A\vdash\neg\neg\lnec\neg\neg A}{
          \infer[\infernote{N*}]{\lnec A\vdash\lnec\neg\neg A}{
            \infer{A\vdash\neg\neg A}{}}}}}}}
\]
\end{frame}

\begin{frame}
Mit der gemachten Vorbetrachtung findet man nun mühelos, dass Axiom~B
in System~S5 ableitbar ist. Der Beweisbaum ist:\pause

\[
\infer{\vdash A\cond\lnec\lpos A}{
  \infer[\infernote{5}]{A\vdash\lnec\lpos A}{
    \infer{A\vdash\lpos A}{
      \infer[\infernote{T}]{\vdash A\cond\lpos A}{}
    & \infer{A\vdash A}{}}}}
\]
Bzw. als schlichter Kettenschluss:
\[
\infer[\infernote{Kettenschluss}]{\vdash A\cond\lnec\lpos A}{
  \infer[\infernote{T}]{\vdash A\cond\lpos A}{}
& \infer[\infernote{5}]{\vdash\lpos A\cond\lnec\lpos A}{}
}
\]
\end{frame}

\begin{frame}
Ferner stellt sich heraus, dass Axiom~4 im System S5 ableitbar ist.
Wir nutzen dazu die Feststellung $\neg\lnec A\equiv\lpos\neg A$
als Hilfsmittel. Alternativ ginge es auch, die zulässige
Ersetzungsregel mit $\neg\neg A\equiv A$ zu nutzen --- wir verzichten drauf,
da sie in Beweisassistenten nicht unbedingt direkt zur
Verfügung steht.

\parspace
Der Beweisbaum fällt ein wenig länger aus:\pause
\[
\infer[\infernote{Kettenschluss}]{\vdash\lnec A\cond\lnec\lnec A}{
  \infer[\infernote{$B:=\lnec A$}]{\vdash\lnec A\cond\lnec\lpos\lnec A}{
    \infer[\infernote{B}]{\vdash B\cond\lnec \lpos B}{}}
& \hspace{-4em}\infer[\infernote{N, K}]{\vdash\lnec\lpos\lnec A\cond\lnec\lnec A}{
    \infer[\infernote{Def.}]{\vdash\lpos\lnec A\cond\lnec A}{
      \infer[\infernote{DN}]{\vdash\neg\lnec\neg\lnec A\cond\lnec A}{
        \infer[\infernote{Kontraposition}]{\vdash\neg\lnec\neg\lnec A\cond\neg\neg\lnec A}{
          \infer[\infernote{Kettenschluss}]{\vdash\neg\lnec A\cond\lnec\neg\lnec A}{
            \infer{\vdash\neg\lnec A\cond\lpos\neg A}{
              \infer{\neg\lnec A\equiv\lpos\neg A}{}}
          & \infer[\infernote{Kettenschluss}]{\vdash\lpos\neg A\cond\lnec\neg\lnec A}{
              \infer[\infernote{$B:=\neg A$}]{\vdash\lpos\neg A\cond\lnec\lpos\neg A}{
                \infer[\infernote{5}]{\vdash\lpos B\cond\lnec\lpos B}{}}
            & \infer[\infernote{N, K}]{\vdash\lnec\lpos\neg A\cond\lnec\neg\lnec A}{
                \infer{\vdash\lpos\neg A\cond\neg\lnec A}{
                  \infer{\neg\lnec A\equiv\lpos\neg A}{}}}}}}}}}}
\]
\end{frame}

\begin{frame}
\strong{Literatur}
\begin{itemize}
\item Johan van Benthem: \href{https://iep.utm.edu/modal-lo/}{\emph{Modal Logic: A Contemporary View}}.
In: \emph{The Internet Encyclopedia of Philosophy}.
\item James Garson: \href{https://plato.stanford.edu/entries/logic-modal/}{\emph{Modal Logic}}.
In: \emph{The Stanford Encyclopedia of Philosophy}.
\item Open Logic Project: \href{https://openlogicproject.org/}{\emph{The Open Logic Text}}.
  Part XI, \emph{Normal Modal Logics}.
\end{itemize}
\end{frame}

\begin{frame}
Ende.
\vfill\hfill\modest{Dezember 2022}\\
\hfill\modest{Creative Commons CC0 1.0}
\end{frame}

\end{document}

