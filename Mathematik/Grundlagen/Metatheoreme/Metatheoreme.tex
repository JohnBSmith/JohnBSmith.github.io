\documentclass{beamer}
\usetheme{Antibes}
\useinnertheme{rectangles}
\useoutertheme{infolines}
\usepackage[utf8]{inputenc}
\usepackage[T1]{fontenc}
\usepackage[ngerman]{babel}

% patch the look of +, = in arev
\usefonttheme{serif} 

\usepackage{arev}
\usepackage{amsmath}
\usepackage{amssymb}

\setbeamertemplate{footline}{%
\begin{beamercolorbox}[ht=3.0ex,dp=1ex]{title in head/foot}
\hfill\footnotesize\insertpagenumber\enspace\enspace\end{beamercolorbox}}

\definecolor{brown1}{rgb}{0.26,0.14,0}
\definecolor{brown2}{rgb}{0.20,0.10,0}
\setbeamercolor*{palette primary}{fg=white,bg=brown1}
\setbeamercolor*{palette secondary}{fg=white,bg=brown2}
\setbeamercolor*{palette tertiary}{fg=white,bg=brown2}
\newcommand{\modest}[1]{{\small\color{gray}#1}}

\newcommand{\ee}{\mathrm e}
\newcommand{\ui}{\mathrm i}
\newcommand{\real}{\operatorname{Re}}
\newcommand{\imag}{\operatorname{Im}}
\newcommand{\uv}[1]{\underline{#1}}
\newcommand{\bv}[1]{\mathbf{#1}}

\newcommand{\N}{\mathbb N}
\newcommand{\Z}{\mathbb Z}
\newcommand{\Q}{\mathbb Q}
\newcommand{\R}{\mathbb R}
\newcommand{\C}{\mathbb C}

\newcommand{\id}{\operatorname{id}}
\newcommand{\sgn}{\operatorname{sgn}}
\newcommand{\Abb}{\operatorname{Abb}}
\newcommand{\unit}[1]{\mathrm{#1}}
\newcommand{\chem}[1]{\mathrm{#1}}
\newcommand{\strong}[1]{\textsf{\textbf{#1}}}
\newcommand{\defiff}{\quad:\Longleftrightarrow\quad}

\title{Metatheoreme über die Aussagenlogik}
\date{}

\begin{document}

\begin{frame}
\maketitle
\vfill\hfill
\modest{Creative Commons CC0}
\end{frame}

\begin{frame}
\strong{Definition. Interpretation.}
Eine \emph{Interpretation} $I\colon V\to\{0,1\}$ ist eine Abbildung,
die jeder logischen Variablen einen Wahrheitswert zuordnet.
Anstelle von Interpretation spricht man auch von einer \emph{Belegung}.
\end{frame}

\begin{frame}
Der Definitionsbereich einer Interpretation wird wie folgt
auf die Menge aller wohlgeformten Formeln erweitert:
\begin{align*}
I(0) &= 0,\\
I(1) &= 1,\\
I(\neg\varphi) &= (\neg I(\varphi)),\\
I(\varphi\land\psi) &= (I(\varphi)\land I(\psi)),\\
I(\varphi\lor\psi) &= (I(\varphi)\lor I(\psi)),\\
I(\varphi\rightarrow\psi) &= (I(\varphi)\rightarrow I(\psi)),\\
I(\varphi\leftrightarrow\psi) &= (I(\varphi)\leftrightarrow I(\psi)).
\end{align*}
Die rechte Seite der jeweiligen Zeile wird hierbei mittels ihrer
Wahrheitstafel ausgewertet.
\end{frame}

\begin{frame}
Für eine Formel $\varphi$ sind die Interpretationen
nichts anderes als die Zeilen der Wahrheitstafel zu $\varphi$.
\end{frame}

\begin{frame}
\strong{Definition. Modell.} Eine Interpretation $I$
heißt \emph{Modell} der Formel $\varphi$, wenn
$I(\varphi)=1$ ist.
\end{frame}

\begin{frame}
\strong{Definition. Modellrelation.}
Ist jede Interpretation, die ein Modell jeder Formel der Formelmenge
\[\Gamma=\{\varphi_1,\ldots,\varphi_n\}\]
ist, auch ein Modell von $\psi$, dann sagt man,
$\Gamma$ \emph{modelliert} $\psi$, und schreibt:
\[\Gamma\models\psi.\]
\end{frame}

\begin{frame}
\strong{Definition. Tautologische Formel.}
Eine Formel, welche unter jeder beliebigen Interpretation
gültig ist, heißt \emph{tautologisch}. Kurz:
\[(\models\varphi) \iff \forall I(I(\varphi)=1).\]
Das heißt:
\[(\models\varphi) \iff (\{\}\models\varphi).\]
\end{frame}

\begin{frame}
Anders ausgedrückt: Eine Formel ist genau dann tautologisch,
wenn jede Zeile der Wahrheitstafel am Ende erfüllt wird.
Dieses einfache Verfahren ist Elektronikern bereits aus der
Schaltalgebra bekannt.
\end{frame}

\begin{frame}
\strong{Es gelten Metatheoreme.}
\end{frame}

\begin{frame}
\strong{Korrektheit der Aussagenlogik.}
Es gilt:
\[(\Gamma\vdash\psi)\implies (\Gamma\models\psi).\]
Das heißt, jede Formel die sich unter Verwendung von
Schlussregeln und bereits gezeigten Sätzen syntaktisch aus $\Gamma$
ableiten lässt, wird auch durch $\Gamma$ modelliert.
\end{frame}

\begin{frame}
\strong{Vollständigkeit der Aussagenlogik.}
Es gilt:
\[(\Gamma\models\psi)\implies (\Gamma\vdash\psi).\]
\end{frame}

\begin{frame}
\strong{Deduktionstheorem (syntaktisch).}
Es gilt:
\[(\Gamma\cup\{\varphi\}\vdash\psi)
\iff(\Gamma\vdash\varphi\rightarrow\psi).\]
Infolge gilt auch:
\[(\{\varphi_1,\ldots,\varphi_n\}\vdash\psi)
\iff (\vdash\varphi_1\land\ldots\land\varphi_n\rightarrow\psi).\]
\end{frame}

\begin{frame}
\strong{Deduktionstheorem (semantisch).}
Es gilt:
\[(\Gamma\cup\{\varphi\}\models\psi)
\iff(\Gamma\models\varphi\rightarrow\psi).\]
Infolge gilt auch:
\[(\{\varphi_1,\ldots,\varphi_n\}\models\psi)
\iff (\models\varphi_1\land\ldots\land\varphi_n\rightarrow\psi).\]
\end{frame}

\begin{frame}
Mit dem Deduktionstheorem ist genauer die Folgerung von links nach
rechts gemeint. Die Folgerung von rechts nach links wird dann
Umkehrung des Deduktionstheorems genannt.
\end{frame}

\begin{frame}
\strong{Einsetzungsregel.} Sei $v$ eine metasprachliche Variable,
welche für eine objektsprachliche Variable steht. Ist $\varphi$
eine tautologische Formel, dann ergibt sich auch eine tautologische
Formel, wenn man jedes Auftreten von $v$ in $\varphi$ gegen die
Formel $\psi$ ersetzt. Kurz:
\[(\models\varphi)\implies (\models\varphi[v:=\psi]).\]
\end{frame}

\begin{frame}
Allgemeiner gilt die Einsetzungsregel auch für simultane Einsetzungen:
\[(\models\varphi)\implies (\models\varphi[v_1:=\psi_1,\ldots,v_n:=\psi_n]).\]
\end{frame}

\begin{frame}
Die Metatheoreme überschatten die Aussagenlogik. Ihre Reichhaltigkeit
macht sich dadurch bemerkbar, dass sie es gestatten,
objektsprachliche Formeln in metasprachliche Schlussregeln
umzuwandeln.
\end{frame}

\begin{frame}
\strong{Beispiel.}
Als Anwendungsbeispiel wollen wir den Modus ponens zeigen.
Zunächst überzeugen wir uns mittels einer Wahrheitstafel, dass
\[\models A\land (A\rightarrow B)\rightarrow B\]
gilt.
\end{frame}

\begin{frame}
Mit der Einsetzungsregel ersetzen wir die beiden Variablen
simultan gegen Formeln. Es ergibt sich
\[\models\varphi\land (\varphi\rightarrow\psi)\rightarrow\psi.\]
\end{frame}

\begin{frame}
Wir nutzen nun die Vollständigkeit und
gewinnen:
\[\vdash\varphi\land (\varphi\rightarrow\psi)\rightarrow\psi.\]
\end{frame}

\begin{frame}
Anwendung des Deduktionstheorems bringt uns schließlich
den Modus ponens:
\[\{\varphi,\varphi\rightarrow\psi\}\vdash\psi.\]
\end{frame}

\begin{frame}
Soeben wurde ein einfaches Verfahren skizziert, das es gestattet, neue
Schlussregeln zu gewinnen.
\end{frame}

\begin{frame}
Die Metatheoreme erlauben es uns, die Aussagenlogik besser
zu beherrschen als es mit alleinigen Anwendung eines Kalküls möglich
wäre.
\end{frame}

\begin{frame}
Ende.
\end{frame}

\end{document}


