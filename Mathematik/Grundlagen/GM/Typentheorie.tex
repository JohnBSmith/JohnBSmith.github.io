
\chapter{Typentheorie}

Mit der Zeit hielt die Typentheorie Einzug in die Grundlagen der Mathematik
und gesellte sich neben die Mengenlehre. Viele Theorembeweisassistenten
arbeiten mit einer Typentheorie, die eine Logik höherer Stufe kodiert.
Um nicht nur oberflächlich mit diesen Assistenten arbeiten zu können, sondern
auch ein gewisses Verständnis zu erhalten, wie sie funktionieren, müssen wir
uns notgedrungen mit diesen Systemen auseinandersetzen. Darüber hinaus
bestehen enge Bezüge zu den Typsystemen in der Informatik.

Wir müssen hier zwischen zwei Begriffen unterscheiden. Die \emph{Typentheorien}
sind ideale Typsysteme, denen vom Wesen her ein logisches System innewohnt.
\emph{Die Typentheorie}, eine Teildisziplin der mathematischen Logik, studiert
diese Typsysteme und klärt ihre Konsistenz ab, wobei ein Bezug zur
Beweistheorie hergestellt wird.

\section{Abhängige Typentheorie}

\subsection{Begrifflichkeiten}

Ich will in den Darlegungen den sorgfältigen Ausführungen von
\cite{Avigad}, \cite{Mimram} und \cite{HoTT} folgen. So wie es mehrere
Systeme der Mengenlehre gibt, haben sich auch mehrere Systeme der
abhängigen Typentheorie herausgebildet. Ich werde mich im Folgenden
auf den \emph{induktiven Konstruktionenkalkül} beziehen.

Im formalen System der Mengenlehre gibt es erstens Terme, die für Mengen
stehen, zweitens Formeln, die für Aussagen stehen und drittens Beweise.
Das sind drei syntaktische Systeme, die voneinander getrennt definiert
werden. In der einfachen Typentheorie gibt es gleichermaßen Terme, Typen und
Beweise. Dagegen fallen Terme, Typen und Beweise in der abhängigen
Typentheorie, sofern sie reiner Art ist, syntaktisch zum \emph{Ausdruck}
zusammen.

Dafür gibt es aber drei unterschiedliche Formen von Urteilen.%
\index{Urteil} Dies sind

1.\;das Urteil $\Gamma\;\mathrm{ctx}$, laut dem $\Gamma$ ein
wohlgeformter Kontext sei,

2.\;das Urteil $\Gamma\vdash a\colon A$, laut dem der Ausdruck
$a$ im Kontext $\Gamma$ vom Typ $A$ sei,

3.\;das Urteil $a\equiv b$, laut dem die Ausdrücke
$a,b$ urteilsmäßig gleich seien.

Ein Kontext $\Gamma$ ist eine Liste der Form
\[\Gamma = [x_1\colon A_1,x_2\colon A_2,\ldots,x_n\colon A_n],\]
die kodiert, dass die jeweilige Variable $x_i$ vom Typ $A_i$ sei. Die $x_i$
stehen hierbei für paarweise verschiedene Variablen, nicht aber für
Metavariablen, für die zusammengesetzte Ausdrücke eingesetzt werden
dürften. Anders als in den herkömmlichen Sequenzen muss der Kontext
eine Liste sein, da es auf die Reihenfolge der Elemente ankommt, denn
im Ausdruck $A_j$ dürfen nur die Variablen $x_i$ mit $i<j$ auftauchen.

Es darf auch $n=0$ sein, was bedeuten soll, dass der leere Kontext
$\Gamma=[]$ ebenfalls ein wohlgeformter ist. Die Schreibweise
$\Gamma,x\colon A$ steht analog zu den herkömmlichen Sequenzen für die
Konkatenation der Listen $\Gamma$ und $[x\colon A]$.

\subsection{Formuierungsregeln}

Von den Schlussregeln der abhängigen Typentheorie gruppieren sich
zunächst einmal die \emph{Formierungsregeln}, die der Aufstellung
wohlgeformter Typen dienen. Dem Formalismus nach wird ausschließlich
mit auf diese Weise geformten Typen gearbeitet.

Jedem Ausdruck kommt ein \emph{Typ} zu, auch den Typen selbst. Hierfür
dient eine Abfolge von Konstantensymbolen
\[U_0,U_1,U_2,\ldots,\]
die wir \emph{Typuniversen}\index{Typuniversum}\index{Universum!von Typen}
nennen wollen. Statt $U_i$ ist auch die längere Schreibweise $\mathrm{Type}_i$
geläufig. Hat ein Ausdruck $A$ den Typ $U_i$ für ein $i$, so wollen wir $A$
selbst als einen \emph{Typ} betrachten. Ein Ausdruck, der kein Typ ist,
heißt \emph{Term}. Es verhält sich so ähnlich wie mit einer Mengenlehre,
in der Klassen die einzige Sorte sind. Die Abgrenzung der Mengen von den
echten Klassen geschieht dahingehend nicht syntaktisch, sondern kommt
inhaltlich aus den Regeln des formalen Systems zum Vorschein.

Da stellt sich die Frage, warum nicht ein Universum, nennen wir es $U$,
genügt. Da diesem selbst ein Typ zukommen müsste, würden wir $U\colon U$
festsetzen, wie es Per Martin"=Löf\index{Martin-Loef, Per@Martin-Löf, Per}
in der ursprünglichen Fassung seiner Typentheorie tat. Allerdings entdeckte
Jean-Yves Girard\index{Girard, Jean-Yves} kurz darauf, dass
das System damit inkonsistent wird, wir also einen Widerspruch ableiten
können. Dieser Umstand, den man nun \emph{girardsches Paradoxon}%
\index{girardsches Paradoxon} nennt, verhält sich analog zum russellschen
Paradoxon.\index{russellsche Antinomie} \cite{Coquand-Type-Theory}

Für Typuniversen $U_i$ bestehen die Regeln 
\[\dfrac{\Gamma\;\mathrm{ctx}}{\Gamma\vdash\typeuniv_i\colon\typeuniv_{i+1}}\;\text{univ-intro},
\qquad\dfrac{\Gamma\vdash A\colon\typeuniv_i}{\Gamma\vdash A\colon\typeuniv_{i+1}}\;\text{univ-cumul}.\]
Kontexte werden eingeführt, erweitert gemäß
\[\dfrac{}{[]\;\mathrm{ctx}}\;\text{ctx-empty},\qquad
\dfrac{\Gamma\vdash A\colon U_i}{\Gamma,x\colon A\;\mathrm{ctx}}\;\text{ctx-ext},\]
mit der Nebenbedingung, dass $x$ nicht bereits in $\Gamma$ vorkommt.

Eine Grundsequenz wird eingeführt via
\[\dfrac{[x_1\colon A_1, \ldots, x_n\colon A_n]\;\mathrm{ctx}}{
x_1\colon A_1, \ldots, x_n\colon A_n\vdash x_i\colon A_i}\;\mathrm{var},\;\;\text{oder kurz}\;\;
\dfrac{\Gamma\;\mathrm{ctx}}{\Gamma\vdash x\colon A}((x\colon A)\in\Gamma).\]
Der Typ abhängiger Funktionen wird formiert gemäß
\[\dfrac{\Gamma\vdash A\colon\typeuniv_i\qquad\Gamma,x\colon A\vdash B\colon\typeuniv_i}{
  \Gamma\vdash\Pi(x\colon A).\,B\colon\typeuniv_i}\;\text{Pi-form}.\]
Beim Typ $\mathrm{Prop}:=U_0$ genügt es aber bereits, wenn $B$ vom Typ $U_0$ ist,
\[\dfrac{\Gamma,x\colon A\vdash B\colon\typeuniv_0}{
  \Gamma\vdash\Pi(x\colon A).\,B\colon\typeuniv_0}\;\text{Pi-form0}.\]
Der Typ $A\to A$ der Funktionen von $A$ zu $A$ wird als
$\Pi(x\colon A).\, A$ dargestellt. Wir würden diesen Typ nun gerne
formieren. Hierbei muss aber klargestellt werden, was $A$ sein soll.
Formiert werden kann zum Beispiel der Typ $\Pi(A\colon\typeuniv_0).\,A\to A$,
womit $\Pi(A\colon\typeuniv_0).\,(A\to A)$ gemeint ist.
Zuvor wird kurz festgestellt, dass die Einführung von Grundsequenzen der
Form $A\colon\typeuniv_i\vdash A\colon\typeuniv_i$ eine zulässige Regel ist.
Es finden sich die Bäume:
\[
\begin{array}{l@{\qquad}l}
\begin{prooftree}[center=false]
      \infer0[ctx-empty]{[]\;\mathrm{ctx}}
    \infer1[univ-intro]{\vdash\typeuniv_i\colon\typeuniv_{i+1}}
  \infer1[ctx-ext]{A\colon\typeuniv_i\;\mathrm{ctx}}
\infer1[var]{A\colon\typeuniv_i\vdash A\colon\typeuniv_i}
\end{prooftree}
&
\begin{prooftree}[center=false]
        \infer0{A\colon\typeuniv_0\vdash A\colon\typeuniv_0}
      \infer1[ctx-ext]{A\colon\typeuniv_0,x\colon A\;\mathrm{ctx}}
    \infer1[var]{A\colon\typeuniv_0,x\colon A\vdash A\colon\typeuniv_0}
  \infer1[Pi-form0]{A\colon\typeuniv_0\vdash\Pi(x\colon A).\, A\colon\typeuniv_0}
\infer1[Pi-form0]{\vdash\Pi(A\colon\typeuniv_0).\, \Pi(x\colon A).\, A\colon\typeuniv_0}
\end{prooftree}
\end{array}
\]
Für $i\ge 1$ liegt der Typ allerdings in einem höhergestuften Universum:
\[\begin{prooftree}
  \infer0{\vdash\typeuniv_i\colon\typeuniv_{i+1}}
      \infer0{A\colon\typeuniv_i\vdash A\colon\typeuniv_i}
      \infer0{A\colon\typeuniv_i,x\colon A\vdash A\colon\typeuniv_i}
    \infer2[Pi-form]{A\colon\typeuniv_i\vdash\Pi(x\colon A).\, A\colon\typeuniv_i}
  \infer1[univ-cumul]{A\colon\typeuniv_i\vdash\Pi(x\colon A).\, A\colon\typeuniv_{i+1}}
\infer2[Pi-form]{\vdash\Pi(A\colon\typeuniv_i).\, \Pi(x\colon A).\, A\colon\typeuniv_{i+1}}
\end{prooftree}\]
Man beachte, dass $A$ hier, anders als in der Regel Pi-form, lediglich
für eine Typvariable steht, also atomar ist. Dies wird durch die
Kontexte klar, die nur Variablen enthalten dürfen.

\subsection{Einführungs- und Beseitigungsregeln}

Die Regeln zur Einführung, Beseitigung abhängiger Funktionen lauten
\[\dfrac{\Gamma,x\colon A\vdash b\colon B}
{\Gamma\vdash (x\mapsto b)\colon\Pi(x\colon A).\, B}\,\text{Pi-intro},\quad\;
\dfrac{\Gamma\vdash f\colon\Pi(x\colon A).\,B\quad\;\;\Gamma\vdash a\colon A}
{\Gamma\vdash f(a)\colon B[x:=a]}\,\text{Pi-elim}.\]
Hierbei sind $a,b,A,B$ Ausdrücke.

Speziell für von $x$ unabhängiges $B$ ergibt sich daraus entsprechend
\[\dfrac{\Gamma,x\colon A\vdash b\colon B}
{\Gamma\vdash (x\mapsto b)\colon A\to B},\quad\;\;
\dfrac{\Gamma\vdash f\colon A\to B\quad\;\;\Gamma\vdash a\colon A}
{\Gamma\vdash f(a)\colon B}.\]
Wie bei jedem lambda"=Kalkül ist auch $\lambda x.\,b$ statt $x\mapsto b$
geläufig, wobei dies Kurzschreibweisen für $\lambda(x\colon A).\,b$ und
$(x\colon A)\mapsto b$ sind. Die Klammern dürfen hierbei entfallen, ich
will sie aber zur besseren Lesbarkeit beibehalten.

Beispielsweise ist $\Pi(A\colon U_i).\, A\to A$ bewohnt, denn es findet sich der Baum:
\[\begin{prooftree}
    \infer0{A\colon U_i,x\colon A\vdash x\colon A}
  \infer1[Pi-intro]{A\colon U_i\vdash (x\mapsto x)\colon A\to A}
\infer1[Pi-intro]{\vdash (A\mapsto x\mapsto x)\colon\Pi(A\colon U_i).\,A\to A}
\end{prooftree}\]

\subsection{Bezug zur Logik}

Den Typ $\Pi(x\colon A).\, B$ deuten wir als die logische Aussage
$\forall (x\colon A).\, B$. Weiterhin deuten wir den Typ $A\to B$ als die
Aussage $A\cond B$.

Für $A,B\colon U_0$ kann man die Konjunktion,\index{Konjunktion} in
analoger Weise wie das schon in der Logik zweiter Stufe möglich ist, kodieren als
\[A\land B := \Pi(X\colon U_0).\, (A\to B\to X)\to X.\]
Dahingehend ist unschwer feststellbar, dass die drei Funktionen
\begin{align*}
\mathrm{conj}&\colon \Pi(A\colon U_0).\,\Pi(B\colon U_0).\,A\to B\to A\land B,\\
\mathrm{proj}_1&\colon \Pi(A\colon U_0).\,\Pi(B\colon U_0).\, A\land B\to A,\\
\mathrm{proj}_2&\colon \Pi(A\colon U_0).\,\Pi(B\colon U_0).\, A\land B\to B
\end{align*}
konstruierbar sind. Nämlich findet sich
\begin{gather*}
\mathrm{conj} := A\mapsto B\mapsto (a\colon A)\mapsto (b\colon B)\mapsto X\mapsto (f\colon A\to B\to X)\mapsto f(a)(b),\\
\mathrm{proj}_1 := A\mapsto B\mapsto (c\colon A\land B)\mapsto c(A)((a\colon A)\mapsto (b\colon B)\mapsto a),\\
\mathrm{proj}_2 := A\mapsto B\mapsto (c\colon A\land B)\mapsto c(B)((a\colon A)\mapsto (b\colon B)\mapsto b).
\end{gather*}
Sollte nicht klar sein, wie diese Terme zustande kommen, kann man zunächst
die Herleitungsbäume für die logischen Aussagen erstellen und daraufhin
die Beweisterme ergänzen.
