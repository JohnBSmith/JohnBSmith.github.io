
\chapter{Typentheorie}

Mit der Zeit hielt die Typentheorie Einzug in die Grundlagen der Mathematik
und gesellte sich neben die Mengenlehre. Viele Theorembeweisassistenten
arbeiten mit einer Typentheorie, die eine Logik höherer Stufe kodiert.
Um nicht nur oberflächlich mit diesen Assistenten arbeiten zu können, sondern
auch ein gewissen Verständnis zu erhalten, wie sie funktionieren, müssen wir
uns notgedrungen mit diesen Systemen auseinandersetzen. Darüber hinaus
bestehen enge Bezüge zu den Typsystemen in der Informatik.

Wir müssen hier zwischen zwei Begriffen unterscheiden. Die \emph{Typentheorien}
sind ideale Typsysteme, denen vom Wesen her ein logisches System innewohnt.
\emph{Die Typentheorie}, eine Teildisziplin der mathematischen Logik, studiert
diese Typsysteme und klärt ihre Konsistenz ab, wobei ein Bezug zur
Beweistheorie hergestellt wird.

\section{Abhängige Typentheorie}

\subsection{Begrifflichkeiten}

Ich will in den Darlegungen den sorgfältigen Ausführungen von
\cite{Avigad}, \cite{Mimram} und \cite{HoTT} folgen. So wie es mehrere
Systeme der Mengenlehre gibt, haben sich auch mehrere Systeme der
abhängigen Typentheorie herausgebildet. Ich werde mich im Folgenden
auf den \emph{induktiven Konstruktionenkalkül} beziehen.

Im formalen System der Mengenlehre gibt es erstens Terme, die für Mengen
stehen, zweitens Formeln, die für Aussagen stehen und drittens Beweise.
Das sind drei syntaktische Systeme, die voneinander getrennt definiert
werden. In der einfachen Typentheorie gibt es gleichermaßen Terme, Typen und
Beweise. Dagegen fallen Terme, Typen und Beweise in der abhängigen
Typentheorie, sofern sie reiner Art ist, syntaktisch zum \emph{Ausdruck}
zusammen.

Dafür gibt es aber drei unterschiedliche Formen von Urteilen. Dies sind

1.\;das Urteil $\Gamma\;\mathrm{ctx}$, laut dem $\Gamma$ ein
wohlgeformter Kontext sei,

2.\;das Urteil $\Gamma\vdash a\colon A$, laut dem der Ausdruck
$a$ im Kontext $\Gamma$ vom Typ $A$ sei,

3.\;das Urteil $a\equiv b$, laut dem die Ausdrücke
$a,b$ urteilmäßig gleich seien.

\subsection{Formuierungsregeln}

Von den Schlussregeln der abhängigen Typentheorie gruppieren sich
zunächst einmal die \emph{Formierungsregeln}, die der Aufstellung
wohlgeformter Typen dienen. Dem Formalismus nach wird ausschließlich
mit auf diese Weise geformten Typen gearbeitet.

Jedem Ausdruck kommt ein \emph{Typ} zu, auch den Typen selbst. Hierfür
dient eine Abfolge von Konstantensymbolen
\[U_0,U_1,U_2,\ldots,\]
die wir \emph{Typuniversen} nennen wollen. Statt $U_i$ ist auch die
längere Schreibweise $\mathrm{Type}_i$ geläufig. Hat ein Ausdruck $A$ den
Typ $U_i$ für ein $i$, so wollen wir $A$ selbst als einen \emph{Typ}
betrachten. Ein Ausdruck, der kein Typ ist, heißt \emph{Term}.
Es verhält sich so ähnlich wie mit einer Mengenlehre, in der Klassen
die einzige Sorte sind. Die Abgrenzung der Mengen von den echten Klassen
geschieht dahingehend nicht syntaktisch, sondern kommt inhaltlich aus
den Regeln des formalen Systems zum Vorschein.

Da stellt sich die Frage, warum nicht ein Universum, nennen wir es $U$,
genügt. Da diesem selbst ein Typ zukommen müsste, würden wir $U\colon U$
festsetzen, wie es Per Martin"=Löf in der ursprünglichen Fassung seiner
Typentheorie tat. Allerdings entdeckte Jean-Yves Girard kurz darauf, das
das System damit inkonsistent wird, wir also einen Widerspruch ableiten
können. Dieser Umstand, den man nun \emph{girardsches Paradoxon}
nennt, verhält sich analog zum russellschen Paradoxon.

Für Typuniversen $U_i$ bestehen die Regeln 
\[
\dfrac{\Gamma\;\mathrm{ctx}}{\Gamma\vdash\typeuniv_i\colon\typeuniv_{i+1}}\;\text{univ-intro},
\qquad\dfrac{\Gamma\vdash A\colon\typeuniv_i}{\Gamma\vdash A\colon\typeuniv_{i+1}}\;\text{univ-cumul}.
\]
Es dürfen Grundsequenzen der Form $A\colon\typeuniv_0\vdash A\colon\typeuniv_0$ eingeführt werden, denn:
\[
\infer[\infernote{var}]{A\colon\typeuniv_0\vdash A\colon\typeuniv_0}{
  \infer[\infernote{ctx-ext}]{A\colon\typeuniv_0\;\mathrm{ctx}}{
    \infer[\infernote{univ-intro}]{\vdash\typeuniv_0\colon\typeuniv_1}{}}}
\]
Der Typ abhängiger Funktionen wird formiert gemäß
\[\dfrac{\Gamma\vdash A\colon\typeuniv_i\qquad\Gamma,x\colon A\vdash B\colon\typeuniv_i}{
  \Gamma\vdash\Pi(x\colon A).\,B\colon\typeuniv_i}\;\text{Pi-form}.\]
Der Typ $A\to A$ der Funktionen von $A$ zu $A$ wird als
$\Pi(x\colon A).\, A$ dargestellt. Wir würden diesen Typ nun gerne
formieren. Hierbei muss aber klargestellt werden, was $A$ sein soll.
Formiert werden kann zum Beispiel $\Pi(A\colon\typeuniv_0).\,A\to A$.
Das geht wie folgt:
\[
\infer[\infernote{Pi-form}]{\vdash\Pi(A\colon\typeuniv_0).\, \Pi(x\colon A).\, A\colon\typeuniv_1}{
  \infer[\infernote{univ-intro}]{\vdash\typeuniv_0\colon\typeuniv_1}{}
  & \infer[\infernote{univ-cumul}]{A\colon\typeuniv_0\vdash\Pi(x\colon A).\, A\colon\typeuniv_1}{
      \infer[\infernote{Pi-form}]{A\colon\typeuniv_0\vdash\Pi(x\colon A).\, A\colon\typeuniv_0}{
        \infer{A\colon\typeuniv_0\vdash A\colon\typeuniv_0}{}
      & \infer[\infernote{wk}]{A\colon\typeuniv_0,x\colon A\vdash A\colon\typeuniv_0}{
          \infer{A\colon\typeuniv_0\vdash A\colon\typeuniv_0}{}}}}}
\]
Man beachte, dass $A$ hier, anders als in der Regel Pi-form, lediglich
für eine Typvariable steht, also atomar ist. Dies wird durch die
Kontexte klar, die nur Variablen enthalten dürfen.

