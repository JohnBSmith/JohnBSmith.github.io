
\chapter{Programmverifikation}

\section{Programme}

Zur Lösung mancher Probleme, Beantwortung mancher Fragen mag man eine
mehr oder weniger festgelegte Methode, ein Verfahren anwenden.
Zur Präzisierung legt man das Verfahren durch einen Algorithmus fest.
Mit konkreten Mitteln, das Verfahren durchführen zu können, entsteht
aus dem Algorithmus ein \emph{Programm}. Sei dem zweiten Weltkrieg kamen
immer leistungsfähigere Rechenmaschinen auf, die immer anspruchsvollere
Aufgaben bewältigten. Die Programme, die auf den Maschinen abliefen,
wurden mit der Zeit immer komplexer.

Reflektiert man eine Weile, kann man zur Sichtweise gelangen, dass
Problemlösung mehr oder weniger allgemein algorithmisch stattfinden
kann. Es scheint, als ob wir uns fast notgedrungen mit Programmen
auseinandersetzen müssen.

Die frühen Computer führten meist numerische Rechnungen aus, die den
Bedürfnissen von Ingenieuren und Naturwissenschaftlern entsprangen.
Leider nicht immer für friedliche Zwecke. Mit der Entwicklung ging
die Auffindung vieler numerischer Verfahren und Ausreizen bereits
bekannter Verfahren einher. Dazu ein kleines Beispiel. Das Programm
\ref{lst:Heron} zeigt die iterative Berechnung der Quadratwurzel $x=\sqrt{a}$
mit dem Heron"=Verfahren
\[x_0 := \frac{a + 1}{2},\quad x_{n+1} := \frac{1}{2}\Big(x_n + \frac{a}{x_n}\Big).\]
Die Folge $(x_n)$ konvergiert rasch gegen $x$.

\begin{figure}
\begin{lstlisting}[language=Python,%
label=lst:Heron,
caption={Programm zur iterativen Berechnung von Quadratwurzeln}]
def sqrt(a, epsilon = 1E-12):
    x = 0.5*(a + 1)
    while True:
        x = 0.5*(x + a/x)
        if abs((x*x - a)/a) < epsilon:
            return x
\end{lstlisting}
\begin{minipage}[t]{.48\textwidth}
\begin{lstlisting}[language=Python,%
label=lst:Potenz,%
caption={\raggedright Programm zur iterativen Berechnung der Potenz $x^n$}]
def power(x, n):
    y = 1
    while n != 0:
        y = y*x
        n = n - 1
    return y
\end{lstlisting}
\end{minipage}
\hfill
\begin{minipage}[t]{.48\textwidth}
\begin{lstlisting}[language=Python,%
caption={\raggedright Programm zur rekursiven Berechnung der Potenz $x^n$}]
def power(x, n):
    if n == 0:
        return 1
    else:
        return x*power(x, n - 1)
\end{lstlisting}
\end{minipage}
\end{figure}

\section{Die Programmiersprache IMP}

Jede Programmiersprache sollte ihre eigene Semantik besitzen, die bestimmt,
was beim Ablauf der Programme passiert. Und selbst wenn die Semantik nie
explizit erklärt wurde, besteht zumindest eine implizite Semantik durch
den Compiler und die Maschine bzw. den Interpreter in ihrer jeweiligen
Version. Nun verhält es sich aber so, dass die Gültigkeit der später
aufgeführten Regeln empfindlich von der Semantik abhängen kann. Daher
sollten wir uns auf eine feste Sprache mit einer festen Semantik einigen.

Vollumfängliche Programmiersprachen sind in der Regel zu kompliziert, zu
verwickelt, als dass ihre Konzepte in Kürze beschrieben werden könnten.
Deshalb betrachten wir erst einmal IMP, eine minimalistische imperative
Programmiersprache, die uns als Studienobjekt dienen wird. Sie enthält
nur die ganz grundlegenden Konzepte der Algorithmik, und ihre Programme
verarbeiten lediglich ganze Zahlen. Insbesondere bleibt das Konzept der
Unterprogramme außenvor, wenngleich dieses eines der ersprießlichsten
und interessantesten sein mag.

Wir grenzen zunächst die beiden Wertemengen ab, deren Elemente als Literale
in der Syntax und als Werte bei der Auswertung von Ausdrücken auftauchen.

Es sei $\mathrm{Int}$ die Menge der ganzen Zahlen im Dezimalsystem. Während
also $\Z$ die Menge der ganzen Zahlen bezeichnet, besteht $\mathrm{Int}$
aus konkreten Darstellungen der Zahlen aus $\Z$. Prinzipiell wäre auch
eine andere Darstellung wie das Dualsystem möglich; beim Parsen des
Quelltextes müsste dann vom Dezimalsystem ins Dualsystem umgerechnet werden.

Entsprechend sei $\mathrm{Bool}:=\{\kw{false},\kw{true}\}$, wobei die
Symbole \kw{false} und \kw{true} syntaktische Darstellungen der
Wahrheitswerte seien. An sich ginge es auch, die Wahrheitswerte als
$0$, $1$ in den ganzen Zahlen unterzubringen, was ich hier aber nicht weiter
verfolgen will. Übrigens benötigt IMP per se kein Typsystem, weil Zahlen
und Wahrheitswerte bereits in der Syntax nicht gemeinsam auftauchen
dürfen, wie im Weiteren klar werden wird.

Es erfolgt nun die genaue Beschreibung der abstrakten Syntax von IMP.

Ein \strong{arithmetischer Ausdruck} $a$, kurz \strong{Term}, sei durch die
folgenden Produktionsregeln festgelegt.
\begin{itemize}[nosep]
\item Eine ganze Zahl aus $\mathrm{Int}$ ist ein Term.
\item Eine Variable aus $\mathrm{Loc}$ ist ein Term.
\item Sind $a,a'$ Terme, so sind auch $a\,\code{+}\,a'$, $a\,\code{-}\,a'$,
  $a\,\code{*}\,a'$, $a\,\code{/}\,a'$ Terme.
\item Nichts anderes ist ein Term.
\end{itemize}
Wir fassen die Terme hierbei als abstrakte Syntaxbäume auf.
Auf die genaue Grammatik und die Konstruktion eines Parsers will ich
an dieser Stelle nicht näher eingehen, damit der Fokus auf die
eigentliche Problemstellung nicht verloren geht. Die Operationen sollen
auf die gewöhnliche Art geschrieben werden, die Operatoren dabei die
gewöhnliche Rangfolge besitzen.

Ein \strong{boolescher Ausdruck} $b$, kurz \strong{Ausdruck}, sei durch
die folgenden Produktionsregeln festgelegt.
\begin{itemize}[nosep]
\item Die Symbole \kw{false} und \kw{true} sind Ausdrücke.
\item Sind $a,a'$ Terme, so sind $a\;\code{=}\;a'$ und $a \;\code{<=}\; a'$ Ausdrücke.
\item Ist $b$ ein Ausdruck, so ist auch $\kw{not}\;b$ ein Ausdruck.
\item Sind $b,b'$ Ausdrücke, so sind auch $b\;\kw{and}\;b$ und $b\;\kw{or}\;b$ Ausdrücke.
\item Nichts anderes ist ein Ausdruck.
\end{itemize}

Ein \strong{Kommando} $c$, auch \strong{Programm} genannt, sei durch die
folgenden Produktionsregeln festgelegt.
\begin{itemize}[nosep]
\item Das Symbol \kw{skip} ist ein Kommando.
\item Ist $X$ eine Variable aus $\mathrm{Loc}$ und $a$ ein Term,
  so ist $X\,\code{:=}\;a$ ein Kommando.
\item Sind $c,c'$ Kommandos, so ist auch $c$; $c'$ ein Kommando.
\item Ist $b$ ein Ausdruck und sind $c,c'$ Kommandos, so ist auch\\
  \kw{if} $b$ \kw{then} $c$ \kw{else} $c'$ \kw{end} ein Kommando.
\item Ist $b$ ein Ausdruck und $c$ ein Kommando, so ist auch\\
  \kw{while} $b$ \kw{do} $c$ \kw{end} ein Kommando.
\item Nichts anderes ist ein Kommando.
\end{itemize}

\section{Denotationelle Semantik}

Hat man bereits viele Programme geschrieben, gelesen und sich die Vorgänge
auf der Assembler"=Ebene angeschaut, mag man ein intuitives Verständnis
dafür bekommen haben, wie Programme ablaufen. Es hat sich allerdings
als sehr förderlich herausgestellt, diese Abläufe formal zu präzisieren.
Das gelingt zum Beispiel, indem eine tatsächliche oder emulierte
Rechenmaschine festgelegt wird. Um aber nicht den Fokus auf das
Wesentliche zu verlieren, abstrahiert man von den Details der Maschine,
indem den syntaktischen Konstrukten der Programmiersprache direkt eine
Semantik zugeordnet wird. Ich will näher auf die \emph{denotationelle
Semantik}\index{denotationelle Semantik} eingehen.

Wir bezeichnen
\begin{itemize}[nosep]
\item mit $\mathrm{Aexp}$ die Menge der arithmetischen Ausdrücke,
\item mit $\mathrm{Bexp}$ die Menge der booleschen Ausdrücke,
\item mit $\mathrm{Com}$ die Menge der Kommandos.
\end{itemize}
Will man nun einen arithmetischen Ausdruck auswerten, denkt
man sich dafür eine Funktion $\evA\colon\mathrm{Aexp}\to\mathrm{Int}$.
Es ist%
\[\evA\qb{1+2} = 3\]
usw. Nun kann ein arithmetischer Ausdruck aber auch Variablen enthalten,
womit bspw. auch der Wert $\evA\qb{\code{x}+1}$ bezüglich $\code{x}\in\mathrm{Loc}$
bestimmt sein muss.

Die Auswertungsfunktion $\evA$ wird daher parametrisiert durch den aktuellen
Zustand $s$. Wir haben also $\evA\colon\mathrm{Aexp}\to (S\to\mathrm{Int})$
bzw. $\evA\colon\mathrm{Aexp}\times S\to\mathrm{Int}$.

Einen \emph{Zustand} $s$ modellieren wir schlicht als die Belegung
der verfügbaren Variablen, also als eine Funktion $s\in S$ mit
$S:=\Abb(\mathrm{Loc},\mathrm{Int})$.

Ein möglicher Zustand $s$ mit $s(\mathtt x)=1$ und $s(\mathtt y)=2$
ist zum Beispiel
\[s(X) := \begin{cases}
1, & \text{wenn $X=\mathtt x$},\\
2, & \text{wenn $X=\mathtt y$},\\
0 & \text{sonst}. 
\end{cases}\]
Die Notation $s[X:=n]$ stehe für die \emph{Variante} des Zustands $s$,
erklärt als derjenige Zustand, der $X$ auf $n$ abbildet, aber sonst
mit $s$ übereinstimmt. Bezüglich $s_0(X):=0$ ist das zuvor genannte
Beispiel somit beschreibbar als $s:=s_0[\mathtt x := 1][\mathtt y := 2]$.

Wir legen $\evA\colon\mathrm{Aexp}\to (S\to\mathrm{Int})$ rekursiv fest gemäß
\begin{gather*}
\evA\qb{n}(s) := n,\\
\evA\qb{X}(s) := s(X),\\
\evA\qb{a\,\code{+}\,a'}(s) := \evA\qb{a}(s) + \evA\qb{a'}(s),\\
\evA\qb{a\,\code{-}\,a'}(s) := \evA\qb{a}(s) - \evA\qb{a'}(s),\\
\evA\qb{a\,\code{*}\,a'}(s) := \evA\qb{a}(s) \cdot \evA\qb{a'}(s),\\
\evA\qb{a\,\code{/}\,a'}(s) := \evA\qb{a}(s)\,/\, \evA\qb{a'}(s).
\end{gather*}
bezüglich $n\in\mathrm{Int}$, $X\in\mathrm{Loc}$ und $a,a'\in\mathrm{Aexp}$.
Die Operationen auf der rechten Seite werden hierbei auf die übliche
Art und Weise berechnet. Wir verwenden die euklidische Ganzzahldivision
und setzen $n\,/\,0:=0$ für jedes $n\in\mathrm{Int}$.

Wir legen $\evB\colon\mathrm{Bexp}\to (S\to\mathrm{Bool})$ rekursiv fest gemäß
\begin{gather*}
\evB\qb{\kw{false}}(s) := \kw{false},\\
\evB\qb{\kw{true}}(s) := \kw{true},\\
\evB\qb{a \;\code{=}\; a'}(s) := (\evA\qb{a}(s) = A\qb{a'}(s)),\\
\evB\qb{a \;\code{<=}\; a'}(s) := (\evA\qb{a}(s)\le A\qb{a'}(s)),\\
\evB\qb{\kw{not}\;b} := \lnot\evB\qb{b}(s),\\
\evB\qb{b\;\kw{and}\;b'} := \evB\qb{b}(s)\land\evB\qb{b'}(s),\\
\evB\qb{(b\;\kw{or}\;b'} := \evB\qb{b}(s)\lor\evB\qb{b'}(s),
\end{gather*}
bezüglich $a,a'\in\mathrm{Aexp}$ und $b,b'\in\mathrm{Bexp}$. Der Wahrheitswert
der Relationen bzw. Verknüpfungen auf der rechten Seite wird hierbei auf die
übliche Art und Weise bestimmt.

Wir legen $\evC\colon\mathrm{Com}\to (S\rightharpoonup S)$ rekursiv fest gemäß
\begin{gather*}
\evC\qb{\kw{skip}}(s) := s,\\
\evC\qb{X:=a}(s) := s[X:=\evA\qb{a}(s)],\\
\evC\qb{c; c'}(s) := \evC\qb{c'}(\evC\qb{c}(s)) = (\evC\qb{c'}\circ\evC\qb{c})(s),\\
\evC\qb{\text{\kw{if} $b$ \kw{then} $c$ \kw{else} $c'$ \kw{end}}}(s) := \begin{cases}
\evC\qb{c}(s),& \text{wenn $B\qb{b}(s)=\kw{true}$},\\
\evC\qb{c'}(s)& \text{sonst}.
\end{cases}\\
\evC\qb{\text{\kw{while} $b$ \kw{do} $c$ \kw{end}}}(s) := \begin{cases}
\varphi_{b,c}(\evC\qb{c}(s)),& \text{wenn $\evB\qb{b}(s)=\kw{true}$},\\
s & \text{sonst}
\end{cases}
\end{gather*}
mit $\varphi_{b,c}(s) := \evC\qb{\text{\kw{while} $b$ \kw{do} $c$ \kw{end}}}(s)$.

Bei $\evC\qb{c}$ handelt es sich um eine partielle Funktion, da die rekursive Auswertung
der while"=Schleife unter Umständen nicht terminiert -- zum Beispiel bei
\[\texttt{\kw{while} \kw{true} \kw{do} \kw{skip} \kw{end}}.\]

\section{Die Zusicherungssprache}

\emph{Zusicherungen} sind Aussagen, die man in einem bestimmten Zustand als
erfüllt sehen will. Zum Beispiel ist nach der Ausführung des Kommandos
$\code{y := x*x}$ die Zusicherung $\texttt{y}\ge 0$
erfüllt.

Zusicherungen sind logische Formeln, die der Sprache der einsortigen Logik
erster Stufe entstammen sollen. Die logische Sprache wird passend zur
Programmiersprache IMP definiert, dergestalt dass das Diskursuniversum
$\mathrm{Int}$ sei und die Variablen aus $\mathrm{Loc}$ in den Termen auftauchen dürfen.
Benötigte Funktionssymbole der Signatur $\mathrm{Int}^n\to\mathrm{Int}$ und
Relationssymbole der Signatur $\mathrm{Int}^n\to\mathrm{Bool}$ zu $n\in\N_{\ge 0}$
kann man je nach Bedarf in ihrer üblichen Bedeutung hinzufügen; die Operatoren
von IMP sollen dabei aber mindestens verfügbar sein.

Wir müssen nun allerdings zwischen Variablen $x,y,z$ und Variablen
$\code{x},\code{y},\code{z}\in\mathrm{Loc}$ unterscheiden. Die kursiven tauchen
als freie und gebundene Variablen in den Formeln auf. Die aufrechten
verhalten sich dagegen wie Konstantensymbole, über sie kann nicht
quantifiziert werden.

Die Notation $I,s\models A$ stehe für die Aussage, dass
die Interpretation $I=(\mathcal M,\beta)$ und der Zustand $s$
die Formel $A$ erfüllen. Hierbei ist $\mathcal M$ eine Struktur, die
die Bedeutung der Funktions- und Relationssymbole festlegt,
und $\beta$ eine Belegung der kursiven Variablen. Der Zustand $s$
belegt die aufrechten Variablen. Die Definition der Erfüllung geschieht
analog wie bei der gewöhnlichen Logik erster Stufe, weshalb ich sie hier
nicht näher ausführen will.

Wir betrachten nur das Modell $\mathcal M_0$, das die Symbole mit ihrer üblichen
Bedeutung versieht. Daher sei $\mathcal I_0:=\{(\mathcal M,\beta)\mid\mathcal M=\mathcal M_0\}$,
das heißt, $\mathcal I_0$ sei die Menge der Interpretationen, deren Struktur
$\mathcal M_0$ ist.

Es stehe später $s\models A$ als Abkürzung für $I,s\models A$,
sofern sich dadurch keine Zweideutigkeiten ergeben.

\section{Der Hoare-Kalkül}

Ein Hoare"=Tripel\index{Hoare-Tripel} $\{A\}\,c\,\{B\}$ soll die logische
Aussage ausdrücken, dass nach dem Durchlaufen des Programms $c$ die Zusicherung
$B$ erfüllt ist, sofern zuvor die Zusicherung $A$ erfüllt war. Hierbei
bezeichnet man $A$ als die \emph{Vorbedingung}\index{Vorbedingung} und $B$
als die \emph{Nachbedingung}\index{Nachbedingung} von $c$. Das Tripel fordert
allerdings keineswegs, dass das Programm $c$ tatsächlich terminiert; lediglich,
dass $B$ erfüllt ist, \emph{falls} $c$ terminiert. Ist das Tripel allgemeingültig,
nennt man das Programm \emph{partiell korrekt}\index{partiell korrekt}
in Bezug auf $A,B$. Terminiert das Programm zusätzlich, nennen wir es
\emph{total korrekt}\index{total korrekt} in Bezug auf $A,B$ und
notieren dies $[A]\,c\,[B]$.

Beispiel für ein allgemeingültiges Tripel:

\begin{lstlisting}[language=IMP, xleftmargin=\mathindent, mathescape]
$\{\kw{true}\}$ y := x*x $\{\code{y}\ge 0\}$
\end{lstlisting}

\noindent
Die Allgemeingültigkeit des Tripels $\{A\}\,c\,\{B\}$ wird dahingehend
definiert als%
\[(\models\{A\}c \{B\}) \,:\Leftrightarrow\, \forall I{\in}\mathcal I_0\colon\forall s{\in}S\colon
(I,s\models A)\Rightarrow\forall s'{\in}S\colon \evC\qb{c}(s)=s'\Rightarrow (I,s'\models B).\]
\strong{Bemerkung 1.} In Bezug auf IMP ist der Wert $s'$, sofern $\evC\qb{c}(s)$ existiert,
eindeutig bestimmt. Denken wir uns nun den ungültigen Zustand $\bot$,
der sich ergeben soll, wenn $c$ eine nicht terminierende Schleife ist,
bekommt man eine totale Funktion $\evC\qb{c}\colon S\cup\{\bot\}\to S\cup\{\bot\}$,
wobei $\evC\qb{c}(\bot):=\bot$ gesetzt wird. Diesbezüglich verkürzt sich
die Allgemeingültigkeit zu%
\[(\models\{A\}c \{B\}) \;\Leftrightarrow\; \forall I\in\mathcal I_0\colon\forall s\in S\colon
(I,s\models A)\Rightarrow (I,\evC\qb{c}(s)\models B).\]
Hierbei verlangt man, dass $I,\bot\models A$ für jede Formel $A$ gilt.

\strong{Bemerkung 2.}
Die Zusicherungssprache ist erweiterbar um eine modale Operation
$\Box_c B$ je Kommando $c$, üblicherweise $[c]B$ geschrieben. Deren Semantik sei
\[(I,s\models [c]B) \;:\Leftrightarrow\; \forall s'\in S\colon R_c(s,s')\Rightarrow (I,s'\models B)\]
mit der Zugänglichkeitsrelation $R_c(s,s')\,:\Leftrightarrow\, \evC\qb{c}(s)=s'$.

Diesbezüglich sind $\{A\}c\{B\}$ und $A\Rightarrow [c]B$ semantisch äquivalent.

Die so erweiterte Logik nennt man die \emph{dynamische Logik} von IMP.
Die denotationelle Semantik nimmt hierbei die Rolle einer Kripke"=Semantik
ein, wobei die Zustände die Kripke"=Welten sind.

Da die Semantik von IMP eine deterministische ist, muss der vom Zustand
$s$ aus via $c$ erreichbare Zustand $s'$, sofern dieser existiert,
gemäß $s'=\evC\qb{c}(s)$ eindeutig bestimmt sein; es gibt also keine weiteren
Zustände, an die wir $B$ fordern müssten. Insofern ergibt sich die Klärung
\[(\models [A]c [B]) \,\Leftrightarrow\, \forall I{\in}\mathcal I_0\colon\forall s{\in}S\colon
(I,s\models A)\Rightarrow\exists s'{\in}S\colon \evC\qb{c}(s)=s'\land (I,s'\models B).\]
In Bezug auf die Modaloperation $\Diamond_c B$ bzw. $\langle c\rangle B$ mit der Semantik
\[(s\models\langle c\rangle B)\;:\Leftrightarrow\; \exists s'\in S\colon R_c(s,s')\land (s'\models B).\]
sind nun $[A]c[B]$ und $A\Rightarrow\langle c\rangle B$ semantisch äquivalent.

Außerdem bestehen die Äquivalenzen
$\langle c\rangle B \Leftrightarrow \lnot [c]\lnot B$ und
$[c]B\Leftrightarrow\lnot\langle c\rangle\lnot B$. Sie bestätigen,
dass $\langle c\rangle B$ in Bezug auf die Theorie der Modallogiken
als die \emph{Möglichkeitsoperation} gedeutet werden kann, die zur
\emph{Notwendigkeitsoperation} $[c]B$ gehört.

Die Formalismen stehen im engen Bezug zum dijkstraschen wp-Kalkül.
Dijkstra notiert $\mathrm{wp}(c,B)$ für die schwächste Vorbedingung,
engl. \emph{weakest precondition}, unter der das Programm $c$ in einem
Zustand terminiert, in dem $B$ erfüllt ist. Für jede Vorbedingung $A$
gilt daher $A\Rightarrow\mathrm{wp}(c,B)$. Die Aussagen $[A]\,c\,[B]$
und $A\Rightarrow\mathrm{wp}(c,B)$ drücken dasselbe aus.

Analog steht $\mathrm{wlp}(c,B)$ für die schwächste liberale
Vorbedingung, die lediglich $B$ fordert, falls das Programm terminiert.
Demnach sind $[c]B$ und $\mathrm{wlp}(c,B)$ bedeutungsgleich.
Entsprechend drücken $\{A\}c\{B\}$ und $A\Rightarrow\mathrm{wlp}(c,B)$
dasselbe aus.

Aufgrund der deterministischen Natur der Semantik von IMP sind des Weiteren
$\langle c\rangle B$ und $\mathrm{wp}(c,B)$ bedeutungsgleich. In anderen
Programmiersprachen mit anderen Semantiken mag diese Äquivalenz allerdings
verloren gehen.

Wir diskutieren nun die \strong{Schlussregeln} des Kalküls.

\begin{Satz}[Regel zum leeren Kommando]\newlinefirst
Die Regel $\dfrac{}{\vdash\{A\}\kw{skip}\{A\}}$ ist gültig.
\end{Satz}
\begin{Beweis}
Wir wollen $\models\{A\}\kw{skip}\{B\}$ zeigen.
Dazu sei $s$ fest, aber beliebig. Es gelte $s\models A$. Des Weiteren
gelte $\evC\qb{\kw{skip}}(s)=s'$. Zu zeigen ist $s'\models A$.

Gemäß der denotationellen Semantik gilt $\evC\qb{\kw{skip}}(s)=s$, womit
wir $s'=s$ und somit bereits die Behauptung haben.\,\qedsymbol
\end{Beweis}

\begin{Satz}[Regel zur Zuweisung]\newlinefirst
Die Regel $\dfrac{}{\vdash\{A[X:=a]\}X:=a\{A\}}$ ist gültig.
\end{Satz}
Eine Schlussregel ohne Prämissen. Mit $A[X:=a]$ ist hierbei die
Formel gemeint, die aus $A$ hervorgeht, indem jedes Vorkommen von
$X$ in $A$ durch den Term $a$ ersetzt wird.

Zum Beispiel erkennt man das Tripel
\begin{lstlisting}[language=IMP, xleftmargin=\mathindent, mathescape]
$\{\code{x}\ge 0\}$
x := x + 1
$\{\code{x}\ge 1\}$
\end{lstlisting}
unschwer als allgemeingültig. Dieses fällt unter die Fittiche der
Regel, indem $X:=\code{x}$, $a:=\code{x}+1$ und $A:=(\code{x}\ge 1)$
gesetzt wird. Damit ergibt sich $A[X:=a]$ zu $\code{x} + 1\ge 1$,
was logisch äquivalent zu $\code{x}\ge 0$ ist. Streng genommen muss
hierbei von Satz \ref{HK-Ersetzung} Gebrauch gemacht werden.

Ob die Rechnung stimmt, prüft man so: Man betrachtet die Nachbedingung,
wendet auf diese die mit der Zuweisung übereinstimmende Substitution an,
und dies muss dann in der Vorbedingung resultieren.

\begin{Beweis}
Wir wollen
\[\models\{A[X:=a]\}X:=a\{A\}\]
zeigen. Sei dazu $s$ fest, aber beliebig. Es gelte $s\models A[X:=a]$.
Des Weiteren gelte $\evC\qb{X:=a}(s)=s'$. Zu zeigen ist $s'\models A$.

Gemäß der denotationellen Semantik gilt $\evC\qb{X:=a}(s)=s[X:=a]$, womit
wir $s'=s[X:=a]$ haben. Schließlich folgt die Behauptung vermittels
der Äquivalenz
\[(s\models A[X:=a])\;\Leftrightarrow\; (s[X:=a]\models A),\]
die man unschwer als richtig erkennt oder pedantisch per struktureller
Induktion über den Aufbau von $A$ beweisen kann.\,\qedsymbol
\end{Beweis}

\begin{Satz}[Regel zur Sequenz von Kommandos]\mbox{}\\[4pt]
Die Regel $\dfrac{\vdash\{A\}c\{B\}\qquad \vdash\{B\}c'\{C\}}{\vdash\{A\}c; c'\{C\}}$
ist gültig.
\end{Satz}
\begin{Beweis}
Wir wollen $\models\{A\} c; c'\{C\}$ zeigen.
Dazu sei $s$ fest, aber beliebig. Es gelte $s\models A$. Des Weiteren
gelte $\evC\qb{c;c'}(s)=s''$. Zu zeigen ist $s''\models C$.

Gemäß der denotationellen Semantik existiert $s'=\evC\qb{c}(s)$ mit
\[\evC\qb{c;c'}(s) = \evC\qb{c'}(s')=s''.\]
Aus $s\models A$ und $\models\{A\}c\{B\}$ folgt nun zunächst $s'\models B$.
Mit $\models\{B\}c'\{C\}$ folgt daraufhin $s''\models C$ aus
$s'\models B$.\,\qedsymbol
\end{Beweis}

\begin{Satz}[Regel zur Verzweigung]\mbox{}\\[4pt]
Die Regel $\dfrac{\vdash\{A\land b\}c\{C\}\qquad\vdash\{A\land\lnot b\}c'\{C\}}{
\vdash\{A\}\text{\kw{if} $b$ \kw{then} $c$ \kw{else} $c'$ \kw{end}}\{C\}}$
ist gültig.
\end{Satz}
\begin{Beweis}
Sei $s$ fest, aber beliebig. Es gelte $s\models A$. Des Weiteren gelte
\[\evC\qb{\text{\kw{if} $b$ \kw{then} $c$ \kw{else} $c'$ \kw{end}}}(s) = s'.\]
Zu zeigen ist $s'\models C$. Fallunterscheidung. Im Fall
$\evB\qb{b}(s)=\kw{true}$ gilt $s\models b$, also $s\models A\land b$.
Außerdem ergibt sich in diesem Fall die Vereinfachung
\[\evC\qb{\text{\kw{if} $b$ \kw{then} $c$ \kw{else} $c'$ \kw{end}}}(s) = \evC\qb{c}(s).\]
Vermittels $\models\{A\land b\}c\{C\}$ erhalten wir somit $s'\models C$.
Die Argumentation im Fall $\evB\qb{b}(s)=\kw{false}$ verläuft analog.\,\qedsymbol
\end{Beweis}

\begin{Satz}[Regel zur Schleife]\mbox{}\\[4pt]
Die Regel $\dfrac{\vdash\{A\land b\}c\{A\}}{
\vdash\{A\}\text{\kw{while} $b$ \kw{do} $c$ \kw{end}}\{A\land\lnot b\}}$
ist gültig.
\end{Satz}
Sofern $A$ also sowohl vor der Schleife gilt, als auch vor und hinter
dem Schleifenrumpf, muss $A$ auch hinter der Schleife gelten. Man nennt
$A$ hierbei eine \emph{Schleifeninvariante}. Die Auffindung einer
zielführenden Invariante stellt eine wesentliche Schwierigkeit bei der
Programmverifikation dar.

\begin{Beweis}[Beweis]
Sei $s$ fest, aber beliebig. Es gelte $s\models A$. Des Weiteren gelte
\[\evC\qb{\text{\kw{while} $b$ \kw{do} $c$ \kw{end}}}(s) = s''.\]
Zu zeigen ist $s''\models A\land\lnot b$. Induktion über die
Anzahl der Durchläufe. Im Anfang findet kein Durchlauf statt. Das
geht aber nur, wenn $\evB\qb{b}(s)=\kw{false}$, also $s\models\lnot b$.
Da dem Zustand $s$ nach der Semantik der Schleife in diesem Fall keine Änderung
widerfährt, gilt $s=s''$. Demnach gilt $s''\models A$ und $s''\models\lnot b$,
also $s''\models A\land\lnot b$.

Zum Induktionsschritt. Da die Schleife durchlaufen wird, existiert
$s'$ mit $\evC\qb{c}(s)=s'$. Die restlichen null der mehr Schleifendurchläufe
führen dann zum Zustand $s''$, das heißt, $\varphi_{b,c}(s')=s''$.
Die Induktionsvoraussetzung ist, dass $s''\models A\land\lnot b$
aus $s'\models A$ folgt. Es verbleibt also $s'\models A$ zu zeigen.
Weil der erste Durchlauf stattfindet, muss des Weiteren $\evB\qb{b}(s)=\kw{true}$,
also $s\models b$ gelten, womit wir $s\models A\land b$ haben.
Vermittels $\models\{A\land b\}c\{A\}$, was ja Kraft der Prämisse der Regel
zur Verfügung steht, erhält man schließlich $s'\models A$.\,\qedsymbol
\end{Beweis}

\begin{Satz}[Regel zur Verstärkung der Vorbedingung]\mbox{}\\[4pt]
Die Regel $\dfrac{\vdash A'\Rightarrow A\qquad\vdash\{A\}c\{B\}}{
\vdash\{A'\}c\{B\}}$ ist gültig.
\end{Satz}
\begin{Beweis}
Es gelte $s\models A'$. Des Weiteren gelte $\evC\qb{c}(s)=s'$. Zu zeigen
ist $s'\models B$. Mit $\models A'\Rightarrow A$ erhält man zunächst
$s\models A$. Mit $\models\{A\}c\{B\}$ daraufhin $s'\models B$.\,\qedsymbol
\end{Beweis}

\begin{Satz}[Regel zur Abschwächung der Nachbedingung]\mbox{}\\[4pt]
Die Regel $\dfrac{\vdash\{A\}c\{B\}\qquad\vdash B\Rightarrow B'}{
\vdash\{A\}c\{B'\}}$ ist gültig.
\end{Satz}
\begin{Beweis}
Es gelte $s\models A$. Des Weiteren gelte $\evC\qb{c}(s)=s'$.
Zu zeigen ist $s'\models B'$. Mit $\models\{A\}c\{B\}$ erhält man
zunächst $s'\models B$. Mit $\models B\Rightarrow B'$ erhält man
daraufhin $s'\models B'$.\,\qedsymbol.
\end{Beweis}

\begin{Satz}[Ersetzungsregeln]\label{HK-Ersetzung}\mbox{}\\[4pt]
Die Regeln
$\dfrac{\vdash A\Leftrightarrow A'\quad\vdash\{A\}c\{B\}}{
  \vdash\{A'\}c\{B\}}$,\,
$\dfrac{\vdash B\Leftrightarrow B'\quad\vdash\{A\}c\{B\}}{
  \vdash\{A\}c\{B'\}}$ sind gültig.
\end{Satz}
\begin{Beweis}
Folgt unmittelbar aus der Regel zur Verstärkung der
Vorbedingung bzw. Abschwächung der Nachbedingung.\,\qedsymbol
\end{Beweis}

\noindent\strong{Beispiel 1.}
Wir betrachten nochmals den Algorithmus zur Exponentiation, also den
zur Berechnung der Potenz $\mathtt{x^n}$ zu gegebenen Zahlen $\mathtt{x,n}$.
Dazu wird das Programm aus Listing \ref{lst:Potenz} auf S. \pageref{lst:Potenz}
in IMP übersetzt und mit einer Hilfsvariable versehen, so dass $\mathtt n$
konstant bleibt.\footnote{Alternativ gäbe es auch die Möglichkeit,
Gebrauch von einer mathematischen Hilfsvariable in den Zusicherungen zu
machen, statt das Programm zu modifizieren; dann bliebe die Hilfsvariable
konstant, während sich $\mathtt n$ ändert.} Des Weiteren sei $\mathtt n\ge 0$
als Vorbedingung verlangt. Gesucht ist also ein Beweis des mit Listing
\ref{lst:IMP-Exp} aufgeführten Tripels.

\begin{figure}
\begin{minipage}[t]{.40\textwidth}
\begin{lstlisting}[language=IMP, mathescape, label=lst:IMP-Exp,
caption={\raggedright Berechnung von $x^n$}]
$\{\mathtt n\ge 0\}$
y := 1; k := n;
while not k = 0 do
   y := y*x;
   k := k - 1
end
$\{\mathtt{y = x^n}\}$
\end{lstlisting}
\end{minipage}
\hfill
\begin{minipage}[t]{.56\textwidth}
\begin{lstlisting}[language=IMP, mathescape, label=lst:IMP-ExpBin,
caption={\raggedright Binäre Exponentiation}]
$\{\mathtt n\ge 0\}$
y := 1; a := x; k := n;
while 2 <= k do
   q := k/2; r := k - 2*q;
   if r = 1 then y := y*a else skip end;
   k := q; a := a*a
end;
if k = 1 then y := y*a else skip end
$\{\mathtt{y = x^n}\}$
\end{lstlisting}
\end{minipage}
\end{figure}

Vermittels der Regel zur Zuweisung -- zuzüglich Reflexivität der
Gleichheit und der Ersetzungsregel -- ergibt sich erst einmal die Ableitung:
\[\begin{prooftree}
  \infer0{\vdash \mathtt n\ge 0\,\Leftrightarrow\,\mathtt n\ge 0\land 1 = 1}
  \infer0{\vdash\{\mathtt n\ge 0\land 1 = 1\}\;\code{y := 1}\;\{\mathtt n\ge 0\land\mathtt y = 1\}}
\infer2{\vdash\{\mathtt n\ge 0\}\;\code{y := 1}\;\{\mathtt n\ge 0\land\mathtt y = 1\}}
\end{prooftree}\]
Wir fügen bereits ermittelte Zusicherungen in den Quelltext ein:
\begin{lstlisting}[language=IMP, xleftmargin=\mathindent, mathescape]
$\{\mathtt n\ge 0\}$
y := 1; k := n;
$\{\mathtt k\ge 0\land\mathtt y = 1\}$
while not k = 0 do
   y := y*x;
   k := k - 1
end
$\{\mathtt{y = x^n}\}?$
\end{lstlisting}

\noindent
Die Zielführende Schleifeninvariante ist hier $\mathtt{y =
x^{n-k}}$. Vor der Schleife gilt ja $\mathtt{n=k}$.
Mit $\mathtt{x}^0 = 1$ kommt man somit auf $\mathtt y = 1$, was
bezüglich $0^0:=1$ auch im Fall $\mathtt x = 0$ gilt.
Wir durchziehen das Programm nun gemäß den Regeln sukzessive mit
Zusicherungen und gelangen daraufhin bereits zum Abschluss der Verifikation:
\begin{lstlisting}[language=IMP, xleftmargin=\mathindent, mathescape]
$\{\mathtt n\ge 0\}$
y := 1; k := n;
$\{\mathtt{k\ge \mathrm 0\land y = \mathrm 1\land y = x^{n-k}}\}$
while not k = 0 do
   $\{\mathtt{y = x^{n-k}}\}$
   $\{\mathtt{y\cdot x = x^{n-k}\cdot x}\}$
   y := y*x;
   $\{\mathtt{y = x^{n-k}\cdot x = x^{n-(k - 1)}}\}$
   k := k - 1
   $\{\mathtt{y = x^{n-k}}\}$
end
$\{\mathtt{y = x^{n-k}\land k=\mathrm 0}\}$
$\{\mathtt{y = x^n}\}$
\end{lstlisting}

\noindent\strong{Beispiel 2}.
Das Listing \ref{lst:IMP-ExpBin} zeigt die effiziente \emph{Binäre
Exponentiation}. Wie \ref{lst:IMP-Exp} berechnet dieses Programm die
Potenz $\mathtt{x^n}$, benötigt dafür aber weniger Multiplikationen. Soll
beispielsweise $\mathtt x^{100}$ berechnet werden, kann dies auf
$(\mathtt x^{50})^2$ zurückgeführt werden. Dafür dass am Ende einmal
quadriert wird, wurden $50$ Multiplikationen eingespart. Der Algorithmus
stellt ein allgemeines Verfahren hierfür dar.

Allerdings ist nicht mehr auf den ersten Blick ersichtlich, ob das
Programm korrekt arbeitet. Zur Analyse wurden wieder die
Hilfsvariablen $\mathtt{a,k}$ eingeführt, damit $\mathtt{x,n}$ konstant
bleiben. Nach dem Durchlaufen des Programms mit der Vorbedingung
$\mathtt n\ge 0$ soll wieder die Nachbedingung $\mathtt{y=x^n}$ bestehen.
Gesucht ist wider ein Beweis des aufgeführten Tripels. Wir betrachten das
Programm zunächst vom Ende aus. Dort findet sich das Tripel
\begin{lstlisting}[language=IMP, xleftmargin=\mathindent, mathescape]
$\{\mathtt{(y = x^n\land k = \mathrm 0)\lor (y\cdot a = x^n\land k = \mathrm 1)}\}$
if k = 1 then y := y*a else skip end
$\{\mathtt{y = x^n}\}.$
\end{lstlisting}
Nimmt man bei der Vorbedingung nämlich die Fallunterscheidung in die
beiden Seiten der Disjunktion vor, gelangt man in beiden Fällen
zur Nachbedingung.

Am Anfang der Schleife gilt $\mathtt{a^k = x^n}$, was etwas mit der
Schleifeninvariante zu tun haben könnte. Ist diese Gleichung so
modifizierbar, dass unter ihr die Zusicherung
\[\mathtt{(y = x^n\land k = \mathrm 0)\lor (y\cdot a = x^n\land k = \mathrm 1)}\]
gilt? Ja, nämlich zu $\mathtt{y\cdot a^k = x^n}$, denn
\begin{gather*}
\text{im Fall $\mathtt k=0$ gilt $\mathtt{y\cdot a^k = y}$},\\
\text{im Fall $\mathtt k=1$ gilt $\mathtt{y\cdot a^k = y\cdot a}$}.
\end{gather*}
Die Gleichung $\mathtt{y\cdot a^k = x^n}$ gilt ebenfalls vor der
Schleife, da dort $\mathtt y = 1$ ist. Tatsächlich stellt sie eine
zielführende Schleifeninvariante dar. Es verbleibt also zu bestätigen,
dass sie auch am Ende des Schleifenrumpfs gilt.

Zunächst ergibt sich
\begin{lstlisting}[language=IMP, xleftmargin=\mathindent, mathescape]
$\{\mathtt{y\cdot a^k = x^n}\}$
q := k/2; r := k - 2*q
$\{\mathtt{y\cdot a^k = x^n\land q = \big\lfloor\frac{k}{\mathrm 2}\big\rfloor
\land r = k - \mathrm{2}q}\}.$
\end{lstlisting}

Wegen $r\in\{0,1\}$ findet sich bei der Verzweigung nun
\begin{lstlisting}[language=IMP, xleftmargin=\mathindent, mathescape]
$\{\mathtt{y\cdot a^k = x^n\land q = \big\lfloor\frac{k}{\mathrm 2}\big\rfloor
\land r = k - \mathrm 2q}\}$
if r = 1 then y := y*a else skip end
$\{\mathtt{((y\cdot a^{k-\mathrm 1} = x^n\land r = \mathrm 1)\lor
(y\cdot a^k = x^n\land r = \mathrm 0))
\land r = k - \mathrm 2q}\}.$
\end{lstlisting}
Die lange Nachbedingung kann man kompakter fassen als
\[\mathtt{y\cdot a^{k - r} = x^n\land r = k - \mathrm 2q},\]
wobei sich des Weiteren $\mathtt{k - r = \mathrm 2q}$ ergibt.

Zum Rest des Schleifenrumpfs findet sich somit schließlich
\begin{lstlisting}[language=IMP, xleftmargin=\mathindent, mathescape]
$\{\mathtt{y\cdot a^{\mathrm 2q} = x^n}\}$
k := q
$\{\mathtt{y\cdot a^{\mathrm 2k} = x^n}\}$
a := a*a
$\{\mathtt{y\cdot a^k = x^n}\}.$
\end{lstlisting}
Quod erat demonstrandum.

\section{Zum Kalkül der schwächsten Vorbedingung}

Zu einem Programm $c$ mag man zu einer Nachbedingung $B$ mehr als eine
Vorbedingung $A$ finden, so dass $\{A\}\,c\,\{B\}$ gilt. Dahingehend
gelangt man zur Überlegung, ob man $A$ nicht zu einer Vorbedingung $A'$
mit $A\cond A'$ abschwächen kann, so dass $\{A'\}\,c\,\{B\}$ gültig
bleibt. Wir nennen $A_0$ \emph{schwächste liberale Vorbedingung},
wenn für jede andere Vorbedingung $A$ die Äquivalenz
\[\{A\}\,c\,\{B\}\bicond (A\cond A_0)\]
besteht. Man schreibt $\mathrm{wlp}(c,B):=A_0$, wobei $\mathrm{wlp}$ für
\emph{weakest liberal precondition} steht. Wie bei Erörterung der
Semantik des Hoare"=Tripels bereits festgestellt wurde, handelt es sich
bei $\mathrm{wlp}(c,B)$ um nichts anderes als die modale Aussage $[c]B$.

\begin{Satz}
Es gilt $[c_1;c_2]B\equiv [c_1][c_2]B$.
\end{Satz}
\begin{Beweis}
Zur Abkürzung sei $\varphi_1:\equiv(\evC\qb{c_1}(s)=s')$ sowie
$\varphi_2:\equiv(\evC\qb{c_2}(s')=s'')$ und $\psi:\equiv (s''\models B)$.
Es findet sich die äquivalente Umformung
\begin{align*}
&(s\models [c_1][c_2]B) \iff (\forall s'\colon\varphi_1\cond \forall s''\colon\varphi_2\cond\psi)\\
\iff\, &(\forall s'\colon\forall s''\colon\varphi_1\land\varphi_2\cond\psi)
\iff (\forall s''\colon\forall s'\colon\varphi_1\land\varphi_2\cond\psi)\\
\iff\, &(\forall s''\colon (\exists s'\colon\varphi_1\land\varphi_2)\cond\psi)
\iff (s\models [c_1;c_2]B).
\end{align*}
Der letzte Schritt gelingt hierbei aufgrund der Äquivalenz
\[(\exists s'\colon\varphi_1\land\varphi_2)\Leftrightarrow
\evC\qb{c_1;c_2}(s)=s''.\,\qedsymbol\]
\end{Beweis}

\begin{Satz}
Es gilt $[X:=a]B\equiv B[X:=a]$.
\end{Satz}
\begin{Beweis}
Es gelte $s\models [X:=a]B$, womit $s'\models B$ für jedes $s'$ mit
$\evC\qb{X:=a}(s)=s'$ gelten muss. Es existiert aber genau ein
solches $s'$, und für dieses gilt $s'=s[X:=n]$ mit
$\evA\qb{a}(s)=n$. Diesbezüglich ist $s'\models B$ gleichbedeutend
mit $s\models B[X:=a]$. Diese Argumentation kann auch
unschwer umgekehrt werden.\,\qedsymbol
\end{Beweis}

Der Umstand, dass man einfach die Symbolik verdrehen kann, ist natürlich
eine glückliche Fügung in Form einer syntaktischen Koinzidenz. Wählt man
eine andere Notation, etwa $\lnec_{X:=a} B\equiv B[a/X]$, geht sie verloren.

\begin{Satz}\label{dl-nec}
Aus $\models A$ folgt $\models [c]A$.
\end{Satz}
\begin{Beweis}
Unter der Annahme $\evC\qb{c}(s)=s'$ muss $s'\models A$ bestätigt
werden. Laut der Prämisse ist $s'\models A$ aber per se erfüllt.\,\qedsymbol
\end{Beweis}

\begin{Satz}\label{dl-K}
Es gilt $[c](A\cond B)\cond ([c]A\cond [c]B)$.
\end{Satz}
\begin{Beweis}
Unter der Annahme $\evC\qb{c}(s)=s'$ muss $s'\models B$ bestätigt
werden. Vermittels der ersten Prämisse erhält man $s'\models A$ aus
der Annahme. Vermittels der zweiten Prämisse erhält man
$s'\models A\cond B$ aus der Annahme, was definitionsgemäß zu
$(s'\models A)\cond (s'\models B)$ äquivalent ist. Per Modus ponens
folgt somit $s'\models B$.\,\qedsymbol
\end{Beweis}

\begin{Satz}
Es gilt $[c](A\land B)\equiv [c]A\land [c]B$.
\end{Satz}
\begin{Beweis}
Dies ist ein allgemeiner Umstand der Modallogik K, die
laut Satz \ref{dl-nec} und Satz \ref{dl-K} je fixem Programm $c$
vorliegt.\,\qedsymbol
\end{Beweis}

\begin{Satz}
Es gilt $[\kw{if}\,b\,\kw{then}\,c_1\,\kw{else}\,c_2\,\kw{end}]C\equiv
(b\cond [c_1]C)\land (\lnot b\cond [c_2]C)$.
\end{Satz}
\begin{Beweis}
Es gelte die linke Seite der Äquivalenz. Zum ersten Teil der Konjunktion.
Aus der Annahme von $s\models b$ und $\evC\qb{c_1}(s) = s'$ ist
$s'\models C$ abzuleiten. Wegen $\evB\qb{b}(s) = \kw{true}$ erhält man
\[\evC\qb{\kw{if}\,b\,\kw{then}\,c_1\,\kw{else}\,c_2\,\kw{end}}(s) = s',\]
woraus $s'\models C$ vermittels der linken Seite folgt. Die Argumentation
zum zweiten Teil der Konjunktion verläuft analog.

Es gelte die rechte Seite der Äquivalenz. Es ist $s'\models C$
abzuleiten. Fallunterscheidung. Im Fall $\evB\qb{b}(s) = \kw{true}$
gilt $s\models b$ und außerdem $\evC\qb{c_1}(s) = s'$. Mit
diesen erhält man $s'\models C$ vermittels des ersten Teils der
Konjunktion. Im Fall $\evB\qb{b}(s) = \kw{false}$ verläuft
die Argumentation analog.\,\qedsymbol
\end{Beweis}

\noindent
Es wäre zu bemerken, dass in klassischer Logik die allgemeine Äquivalenz
\[(B\cond A_1)\land (\lnot B\cond A_2)\Leftrightarrow
(B\land A_1)\lor (\lnot B\land A_2)\]
besteht.
