\documentclass[paper=186mm:234mm,pagesize=auto,fleqn,11pt,dvipdfmx]{scrbook}
\usepackage[utf8]{inputenc}
\usepackage[T1]{fontenc}
\usepackage[ngerman]{babel}
\usepackage{microtype}
\usepackage{amsmath}
\usepackage{amssymb}
\usepackage{amsthm}
\usepackage{proof}
\usepackage{booktabs}
\usepackage{graphicx}
\usepackage[all]{xy}

\usepackage{mdframed}
\usepackage{lipsum}

\usepackage{libertine}
\usepackage[libertine]{newtxmath}
\usepackage[scaled=0.80]{DejaVuSans}
\usepackage[scaled=0.78]{DejaVuSansMono}
% \renewcommand\ttdefault{lmvtt}

\usepackage[automark]{scrlayer-scrpage}
\ohead{\pagemark}
\ihead{\headmark}
\chead{}
\cfoot{}
\ofoot{}

\usepackage{geometry}
\geometry{left=34mm,right=24mm,top=26mm,bottom=23mm,%
footskip=10mm, headsep=6mm}

\usepackage{color}
\definecolor{c1}{RGB}{0,40,80}
\definecolor{gray1}{RGB}{80,80,80}
\usepackage[colorlinks=true,linkcolor=c1,urlcolor=c1,citecolor=c1]{hyperref}
% \usepackage[colorlinks=true,linkcolor=black]{hyperref}

\newcommand{\strong}[1]{\textsf{\textbf{#1}}}

\newtheoremstyle{rmbox}%
  {0pt}% space above
  {0pt}% space below
  {}% bodyfont
  {}% indent
  {\normalfont\sffamily\bfseries}% head font
  {~}% punctuation between head and body
  {0pt}% space after theorem head
  {\thmname{#1} \thmnumber{#2}\thmnote{ (#3)}.}

\theoremstyle{rmbox}
\newtheorem{Definition}{Definition}
\newtheorem{Satz}{Satz}
\newtheorem{Lemma}[Satz]{Lemma}
\newtheorem{Korollar}[Satz]{Korollar}

\numberwithin{Definition}{chapter}
\numberwithin{Satz}{chapter}

\newenvironment{Beweis}[1][Beweis]%
  {\par\noindent\strong{#1.}\;\ignorespaces}%
  {\par\addvspace{\topsep}}

\definecolor{greenblue}{rgb}{0.0,0.42,0.3}
\definecolor{grayblue}{rgb}{0.1,0.2,0.4}
\definecolor{bgreen}{rgb}{0.94,0.94,0.84}
\definecolor{bblue}{rgb}{0.9,0.92,0.94}
\definecolor{bordercolor}{rgb}{0.3,0.3,0.3}
\definecolor{hltext}{rgb}{0.7,0,0.6}

\surroundwithmdframed[topline=false,rightline=false,bottomline=false,%
  linecolor=bordercolor, linewidth=3.5pt, innerleftmargin=6pt,%
  innertopmargin=2pt, innerbottommargin=3pt,%
  innerrightmargin=6pt%
]{Definition}

\newcommand{\framedtheorem}[1]{%
\surroundwithmdframed[topline=false,rightline=false,bottomline=false,%
  linecolor=bordercolor, linewidth=3.5pt, innerleftmargin=6pt,%
  innertopmargin=2pt, innerbottommargin=3pt,%
  innerrightmargin=6pt%
]{#1}}

\framedtheorem{Satz}
\framedtheorem{Lemma}
\framedtheorem{Korollar}

\newcommand{\N}{\mathbb N}
\newcommand{\Z}{\mathbb Z}
\newcommand{\Q}{\mathbb Q}
\newcommand{\R}{\mathbb R}
\newcommand{\C}{\mathbb C}
\newcommand{\ui}{\mathrm i}
\newcommand{\ee}{\mathrm e}
\newcommand{\compc}{\mathsf c}
\newcommand{\defiff}{\;:\Longleftrightarrow\;}
\newcommand{\emdef}[1]{\emph{#1}}
\renewcommand{\qedsymbol}{\ensuremath{\Box}}
\newcommand{\doubleslash}{/\!/}
\newcommand{\category}[1]{#1}
\newcommand{\cond}{\Rightarrow}
\newcommand{\bicond}{\Leftrightarrow}
\newcommand{\lnec}{\Box}
\newcommand{\lpos}{\Diamond}

\DeclareMathOperator{\id}{id}
\DeclareMathOperator{\sur}{sur}
\DeclareMathOperator{\real}{Re}
\DeclareMathOperator{\imag}{Im}
\DeclareMathOperator{\Bild}{Bild}
\DeclareMathOperator{\Abb}{Abb}
\DeclareMathOperator{\dom}{dom}
\DeclareMathOperator{\cod}{cod}
\DeclareMathOperator{\Hom}{Hom}
\DeclareMathOperator{\Ob}{Ob}

\newcommand{\infernote}[1]{\!\text{\scriptsize #1}}
\newcommand{\newlinefirst}{\mbox{}\\}
\renewcommand{\models}{\vDash}
\newcommand{\todo}[1]{{\color{hltext}#1}}

\usepackage{makeidx}
\makeindex

\title{Grundlagen der Mathematik}
\subtitle{(Arbeitstitel, work in progress)}
\date{Mai 2023}

\begin{document}
\thispagestyle{empty}

\maketitle

\noindent
Dieses Buch steht unter der Lizenz Creative Commons CC0 1.0.

\tableofcontents


\chapter{Logisches Schließen}

\section{Grundbegriffe}

\subsection{Schlussregeln}

% Alles Schlussregeln und Axiome müssen unmittelbar einsichtig sein,
% müssen unbedenklich sein. Normalerweise will man in der Wissenschaft
% nicht so informell sein, aber hier sind wir am basalen Grund.
% Selbst in der Semantik wird der hier aufgestellte Formalismus
% benötigt. Der Korrektheitsbeweis würde somit zirkulär.

% Am Anfang ist noch nicht allzu viel zu rechnen. Wir müssen uns
% erst um die Schlussregeln und Axiome bemühen, bevor das erste Theorem
% bewiesen werden kann.

Logisches Schließen findet in einzelnen Schritten statt. Ein Schritt
stellt hierbei immer die Ableitung einer Schlussfolgerung aus einer
oder mehreren Voraussetzungen dar. Die Voraussetzungen heißen
\emph{Prämissen}\index{Praemisse@Prämisse}, die Schlussfolgerung
\emph{Konklusion}\index{Konklusion}. Darstellen wollen wir den Schritt
durch eine waagerechte Linie, wobei die Prämissen oberhalb befindlich
sein sollen, und die Konklusion unterhalb. Der Schritt
\[\dfrac{\text{Wenn es regnet, wird die Straße nass}\qquad\text{Es regnet}}
{\text{Die Straße wird nass}}\]
beschreibt beispielsweise, dass aus den Prämissen »Wenn es regnet, wird
die Straße nass« und »Es regnet« die Konklusion »Die Straße wird nass«
gefolgert wird.

Schlüsse wie der Obige treten in der Mathematik ständig auf. Ihnen allen
liegt ein bestimmtes Muster zugrunde, welches sich durch eine als
\emph{Modus ponens}\index{Modus ponens} oder
\emph{Abtrennungsregel}\index{Abtrennungsregel}
bezeichnete schematische \emph{Schlussregel}\index{Schlussregel}
beschreiben lässt. Es bezeichne hierzu $A\cond B$ die Implikation
»wenn $A$, dann $B$«. Es dürfen nun in
\[\dfrac{A\cond B\qquad A}{B}\]
für $A,B$ beliebige Aussagen eingesetzt werden. So darf »Es regnet»
für $A$ und »Die Straße wird nass« für $B$ eingesetzt werden.

\subsection{Sequenzen}

Das Schließen von Aussagen allein genügt nicht. Um freier argumentieren
zu können, würden wir gerne den Umstand beschreiben können, dass eine
Aussage unter bestimmten Annahmen abgeleitet werden konnte. Diese
Annahmen $A_k$ sind selbst Aussagen. Wir fassen sie zu einer endlichen
Ansammlung
\[\Gamma := [A_1,A_2,\ldots,A_n]\]
zusammen, worunter wir eine endliche Liste, oder auch eine endliche
Menge verstehen wollen, denn man soll mit dieser Liste umgehen können
wie mit einer Menge. Das heißt, es ist nicht von Bedeutung, wie
oft eine Aussage vorkommt oder in welcher Reihenfolge die Aussagen
stehen. Man bezeichnet $\Gamma$ als die \emph{Antezedenz}%
\index{Antezedenz} oder die Liste der \emph{Antezedenzen}. Es wird
$\Gamma$ auch der \emph{Kontext}\index{Kontext}
oder die \emph{Umgebung}\index{Umgebung} genannt, das sind auf die
Typentheorie zurückzuführende Sprechweisen, die einen ganz ähnlichen
Formalismus besitzt. Wir bezeichnen die Symbolik \[\Gamma\vdash A\]
als \emph{Sequenz}\index{Sequenz}. Sie drückt das \emph{Urteil}%
\index{Urteil} aus, dass die Aussage $A$ aus den Annahmen
vermittels Schlussregeln ableitbar ist. Der Modus ponens%
\index{Modus ponens} wird nun in der allgemeinen Form
\[\dfrac{\Gamma\vdash A\cond B\qquad\Gamma\vdash A}{\Gamma\vdash B}\]
beschrieben. Wir argumentieren beim Schließen also ab jetzt nicht mehr
mit den Aussagen selbst, sondern mit den Sequenzen. Dies hat einen wichtigen
Grund, nämlich dass die Berücksichtigung der Abhängigkeit von Annahmen
expliziter Teil des Schließens wird.

Ein Kontext kann auch eine leere Liste sein. Besitzt eine vermittels
Schlussregeln ableitbare Sequenz einen leeren Kontext, so bezeichnet
man die Antezedenz als ein \emph{Theorem}\index{Theorem} im engeren
Sinne. Ein Theorem ist also eine Aussage, die für sich allein gilt,
ohne dass dafür irgendwelche Annahmen getroffen werden müssen.

Für Sequenzen gilt die \emph{Abschwächungsregel}%
\index{Abschwaechungsregel@Abschwächungsregel}. Sie besagt, dass
falls die Aussage $A$ bereits aus $\Gamma$ ableitbar ist, diese
Aussage erst recht ableitbar ist, wenn zu $\Gamma$ weitere Annahmen
$\Gamma'$ hinzugefügt werden. Kurzum gilt die Regel
\[\dfrac{\Gamma\vdash A}{\Gamma,\Gamma'\vdash A}.\]
Hierbei bedeutet $\Gamma,\Gamma'$ die Konkatenation der Listen
$\Gamma$ und $\Gamma'$, also im Wesentlichen dasselbe wie die
Vereinigung $\Gamma\cup\Gamma'$, insofern man die Kontexte als
Mengen betrachtet.

\subsection{Zulässige Schlussregeln}

Wiewohl die Regeln des Schließens den Mechanismus zum Beweis
von Aussagen bilden, ist ihre Rolle sogar noch ein wenig tiefgreifender.
Wir können sie nämlich ebenfalls zur Ableitung \emph{weiterer Regeln}
nutzen. Das heißt, wir können sie dazu nutzen, den logischen Kalkül
selbst zu erweitern. Erweiterungen dieser Art nennen wir
\emph{zulässige Schlussregeln}%
\index{zulaessige Schlussregel@zulässige Schlussregel}.

Mit den bisherigen Regeln ist bereits die zulässige Regel
\[\dfrac{\Gamma\vdash A\cond B\qquad\Gamma'\vdash A}
{\Gamma,\Gamma'\vdash B}\]
ableitbar, die eine allgemeinere Form des Modus ponens darstellt. Man
erhält sie kurzerhand, indem den Prämissen des Modus
ponens jeweils die Abschwächungsregel vorgesetzt wird:
\[\infer{\Gamma,\Gamma'\vdash B}{
  \infer{\Gamma,\Gamma'\vdash A\cond B}{\Gamma\vdash A\cond B}
& \infer{\Gamma,\Gamma'\vdash A}{\Gamma'\vdash A}}
\]
Die einfache Form des Modus ponens erhält man mit $\Gamma':=\Gamma$ als
Spezialfall unter Anwendung der Kontraktionsregel.

\subsection{Implikationseinführung}

Ich möchte mich nun der Frage zuwenden, wie eine Implikation $A\cond B$
bewiesen wird. Intuitiv ist hierzu aus der Annahme $A$ die
Aussage $B$ zu folgern. Das heißt, es genügt die Ableitung
der Sequenz $A\vdash B$. Ein weiteres Mal gilt es noch zu
berücksichtigen, dass ein Beweis auch auf einen vorausgesetzten
Kontext $\Gamma$ beschränkt sein dürfen soll. Reflektiert man darüber
eine Weile, dürfte es der Überlegung nach wohl genügen, dass $A$
einfach dem Kontext $\Gamma$ hinzugefügt wird, woraus $B$ zu folgern
ist. Man gelangt zur Regel
\[\dfrac{\Gamma,A\vdash B}{\Gamma\vdash A\cond B}.\]
Wer diese Regel nicht so leicht fassbar findet, insbesondere nicht
direkt plausibel, ob sie bedenkenlos eingesetzt werden darf, der
ist nicht allein. Es gibt auch logische Kalküle, die diese Regel nicht
explizit enthalten. Sie tritt dennoch als \emph{Deduktionstheorem} in
Erscheinung, ein metalogisches Theorem, dessen Beweis erst erbracht
werden muss. Ich möchte diesen Weg allerdings aus einem bestimmten Grund
nicht gehen. Nämlich ist beim Beweis eigentlich natürliches Schließen
auf der metalogischen Ebene zu verwenden, wenn dies auch in informaler
Weise stattfinden mag. Aber nicht jeder Leser weiß zu diesem Zeitpunkt,
wie akkurates logisches Schließen geht. Der Leser benötigt am Anfang
etwas, um sich an den eigenen Haaren aus dem Sumpf zu ziehen.

\subsection{Axiome}

Zur Komplettierung des Kalküls gesellen sich schließlich auch noch
\emph{Axiome}\index{Axiom} hinzu, das sind gemachte Grundannahmen, die
nicht weiter bewiesen werden müssen. Sie sollten daher möglichst
plausibel, oder besser noch zweifelsfrei einsichtig sein. Für die Logik
selbst genügt das Axiom
\[A\vdash A.\]
Der Kalkül funktioniert dergestalt, dass für $A$ eine beliebige Aussage
eingesetzt werden darf, worunter auch zusammengesetzte Aussagen
wie fallen. Eine gern gewählter Weg der Definition des logischen
Kalküls sieht $A$ als eine metasprachliche Variable, für die eine
beliebige Formel eingesetzt werden darf. Unter dieser Sichtweise
spricht man von einem \emph{Axiomenschema}\index{Axiomenschema}. Wie
eine Schablone produziert es für jede Einsetzung einer konkreten
logischen Formel ein eigenes Axiom.

Anstelle $A,B,C$ werden für metasprachliche Variablen zuweilen auch
die griechischen Buchstaben $\varphi,\psi,\chi$ benutzt. Man muss sie
von atomaren logischen Variablen unterscheiden, für die wir in diesem
Buch, um Missverständnissen aus dem Weg zu gehen, kleine Buchstaben
$a,b,c$ oder $p,q,r$ verwenden werden. Sprachlich suggestiv steht
$\varphi$ für \emph{Formel}\index{Formel} oder \emph{formula}, $a$ für
\emph{Aussage} und $p$ für \emph{proposition}.

In diesem Sinne sind auch die Schlussregeln Schemata. Sofern man
Schlussregeln mit null Prämissen gestattet, lässt sich das
Axiomenschema auch als Regel
\[\dfrac{}{A\vdash A}\]
auffassen. In dieser Weise wollen wir die Anwendung von Axiomen in den
Beweisbäumen darstellen.

Axiome in der Form von Sequenzen heißen auch
\emph{Grundsequenzen}\index{Grundsequenz}.

Wir haben nun die Mittel in der Hand, um erste Theoreme beweisen
zu können. Es ist $A\cond A$ ein Theorem. Der Beweisbaum ist:
\[
\infer[\infernote{Implikationseinführung}]{\vdash A\cond A}{
  \infer[\infernote{Axiom}]{A\vdash A}{}}
\]
Unter der Lesung, dass $A$ eine Metavariable ist, handelt es
eigentlich nicht nur um ein Theorem, sondern um ein Schema von
Theoremen. Setzt man für $A$ bspw. die konkrete Formel $p\cond q$
ein, bekommt man das konkrete Theorem
\[\vdash (p\cond q)\cond (p\cond q).\]
% Es ist zu bemerken, dass die Unterscheidung zwischen Metavariablen
% und atomaren logischen Variablen später durch die Einsetzungsregel
% mehr oder weniger hinfällig wird. Hierbei handelt es sich aber um eine
% höhere Überlegung, deren Beweis der Logiker erbringen will. In der
% Wissenschaft, insbesondere in der Logik, will man den Dingen auf den
% Grund gehen, will alles genau auseinandernehmen. Da möchte man bestimmte
% Regeln nicht einfach so als gegeben voraussetzen.

\subsection{Junktoren}

Bisher traten zusammengesetzte Aussagen alleinig in Form einer
Implikation auf. Will man logische Zusammenhänge beschreiben können,
muss die logische Sprache um weitere Junktoren bereichert werden.
Unter einem \emph{Junktor}\index{Junktor} versteht man eine logische
Verknüpfung von Aussagen zu einer zusammengesetzten Aussage.

Wir werden einen Junktor durch \emph{Einführungsregeln}%
\index{Einfuehrungsregel@Einführungsregel} und
\emph{Beseitigungsregeln}\index{Beseitigungsregel}
charakterisieren. Die Regeln der Implikation
wurden bereits beschrieben; die Einführung geschieht per
Implikationseinführung, die Beseitigung per Modus ponens.
Für die restlichen Junktoren der Aussagenlogik lassen sich die Regeln
wahlweise in Form von Axiomenschemata oder in Form von Schlussregeln
darstellen. Ich möchte das per Schemata machen, weil diese ein wenig
kompakter sind, was sie hoffentlich ein wenig leichter einsichtig macht.
Die entsprechenden Schlussregeln leiten wir anschließend als zulässige
Regeln ab.

Die Konjunktion\index{Konjunktion} $A\land B$, auch Und"=Verknüpfung
genannt, sprich »$A$ und $B$«, ist charakterisiert durch die Sequenzen
\[A,B\vdash A\land B;\qquad A\land B\vdash A;\qquad A\land B\vdash B.\]
Aus dem Fall von sowohl Regen als auch Schnee ist der Fall von
Schneeregen ableitbar. Aus dem Fall von Schneeregen ist der Fall
von Regen ableitbar. Aus dem Fall von Schneeregen ist der Fall
von Schnee ableitbar. So sind diese Sequenzen zu verstehen.

Die Einführung der Konjunktion geschieht mit der Regel
\[\dfrac{\Gamma\vdash A\qquad\Gamma'\vdash B}{\Gamma,\Gamma'\vdash A\land B}.\]
Denn es findet sich der Beweisbaum:
\[
\infer[\infernote{MP}]{\Gamma,\Gamma'\vdash A\land B}{
  \infer[\infernote{MP}]{\Gamma\vdash B\cond A\land B}{
    \infer[\infernote{Impl-Einf.}]{\vdash A\cond (B\cond A\land B)}{
      \infer[\infernote{Impl-Einf.}]{A\vdash B\cond A\land B}{
        \infer[\infernote{Axiom}]{A,B\vdash A\land B}{}}}
  & \Gamma\vdash A}
& \Gamma'\vdash B}
\]
Es steht MP als Abkürzung für Modus ponens, und Impl-Einf. für
Implikationseinführung. Man schreibt alternativ auch das Kürzel
$\cond$E anstelle Impl-Einf. und das Kürzel $\cond$B anstelle
von MP. Hierbei steht E offenkundig für \emph{Einführung} und
B für \emph{Beseitigung}. Aber Vorsicht, in der englischsprachigen
Literatur sind das I für \emph{introduction} und E für
\emph{elimination}.

Die beiden Regeln zur Beseitigung der Konjunktion sind
\[\dfrac{\Gamma\vdash A\land B}{\Gamma\vdash A},
\qquad\dfrac{\Gamma\vdash A\land B}{\Gamma\vdash B}.\]
Denn es findet sich:
\[
\infer[\infernote{MP}]{\Gamma\vdash A}{
  \infer[\infernote{Impl-Einf.}]{\vdash A\land B\cond A}{
    \infer[\infernote{Axiom}]{A\land B\vdash A}{}}
& \Gamma\vdash A\land B}
\]
Die Disjunktion\index{Disjunktion} $A\lor B$, auch Oder"=Verknüpfung
genannt, sprich »$A$ oder $B$«, ist charakterisiert durch die Sequenzen
\[A\vdash A\lor B;\qquad B\vdash A\lor B;\qquad
A\lor B, (A\cond C), (B\cond C)\vdash C.\]
So ist »Die Erde des Beetes ist nass« ableitbar aus »Es hat geregnet
oder das Beet wurde gegossen«. Denn sowohl »Es hat geregnet« als auch
»Das Beet wurde gegossen« impliziert »Die Erde des Beetes ist nass«.

Die beiden Regeln zur Einführung der Disjunktion sind
\[\dfrac{\Gamma\vdash A}{\Gamma\vdash A\lor B},\qquad
\dfrac{\Gamma\vdash B}{\Gamma\vdash A\lor B}.\]
Die Regel zur Beseitigung der Disjunktion ist
\[\dfrac{\Gamma\vdash A\lor B\qquad\Gamma',A\vdash C\qquad\Gamma'',B\vdash C}
{\Gamma,\Gamma',\Gamma''\vdash C}.\]
Die Beweise dieser Regeln seien dem Leser überlassen.

Eine Aussage wie »Bertram wird seine Hausaufgaben nicht machen«
formuliert man gern in der Form »Wenn Bertram seine Hausaufgaben macht,
färbt sich der Mond grün«. In gleichartiger Weise lässt sich die
Verneinung auch in der formalen Logik definieren. Hierzu legt man als
Hilfsbegriff zunächst $\bot$ als die \emph{Kontradiktion}%
\index{Kontradiktion} fest, sie steht für eine widersprüchliche Formel.

Die Negation\index{Negation} $\lnot A$, auch Verneinung genannt, sprich
»nicht $A$«, definiert man als identisch mit $A\cond\bot$. Hierdurch
sind die Regeln zu ihrer Einführung und Beseitigung auf die der
Implikation zurückführbar. Es ergibt sich
\[\dfrac{\Gamma,A\vdash\bot}{\Gamma\vdash\lnot A},
\qquad\dfrac{\Gamma\vdash\lnot A\qquad\Gamma'\vdash A}
{\Gamma,\Gamma'\vdash\bot}.\]
Alternativ ließe sich die Negation durch die Sequenzen
\[(A\cond\bot)\vdash\lnot A;\qquad A,\lnot A\vdash\bot\]
charakterisieren. Man überzeuge sich, dass dies aufs selbe hinausläuft.

Die Äquivalenz\index{Aequivalenz@Äquivalenz} $A\bicond B$, sprich
»$A$ genau dann, wenn $B$«, definiert man als identisch mit
$(A\cond B)\land (B\cond A)$. Insofern sind die Regeln zu ihrer
Einführung und Beseitigung auf die der Konjunktion zurückführbar.
Es ergibt sich
\[\dfrac{\Gamma\vdash A\cond B\qquad\Gamma'\vdash B\cond A}
{\Gamma,\Gamma'\vdash A\bicond B},\qquad
\dfrac{\Gamma\vdash A\bicond B}{\Gamma\vdash A\cond B},\qquad
\dfrac{\Gamma\vdash A\bicond B}{\Gamma\vdash B\cond A}.\]
Die entsprechenden charakterisierenden Sequenzen sind
\[(A\cond B),(B\cond A)\vdash A\bicond B;\qquad (A\bicond B),A\vdash B;
\qquad (A\bicond B),B\vdash A.\]

\subsection{Zur Syntax}

So wie »Punktrechnung vor Strichrechnung« gilt, legt man für jeden Junktor
zur Einsparung von Klammern eine Stufe der Priorität fest. In
absteigender Rangfolge sind dies ${\lnot}, {\land}, {\lor}, {\cond},
{\bicond}$. So wird die Formel
\[\lnot A\land B\lor C\cond D\quad\text{gelesen als}\quad
(((\lnot A)\land B)\lor C)\cond D.\]
Weiterhin legt man die Implikation als rechtsassoziativ fest. So
wird
\[A\cond B\cond C\quad\text{gelesen als}\quad A\cond (B\cond C).\]
Manche Schüler haben Schwierigkeiten, die Struktur von Termen zu
durchschauen. Infolge kann es bei ihnen zu Flüchtigkeitsfehlern
bei der Ersetzung von Variablen durch Terme kommen. Sie vergessen,
dass ein Term vor der Einfügung zunächst geklammert werden muss.
Erst die Operatorrangfolge gewährt es, die Klammern unter Umständen
nachträglich entfallen zu lassen. Für diese Schüler mag es förderlich
sein, einen Term als \emph{abstrakten Syntaxbaum} darzustellen.
Gleichermaßen verhält es sich mit der Programmiersprache Lisp, die
Terme als Schachtelung von Listen darstellt, deren Klammern
obligatorisch sind.  Die Aussage $A\land B\cond C$ ist beispielsweise
beschreibbar als
\[\text{\texttt{(implies (and A B) C)}}.\]
Im Wesentlichen veranschaulicht diese Schachtelung
nichts anderes als den abstrakten Syntaxbaum. Man kann gewissermaßen
sagen, dass Lisp eine Programmiersprache ohne Syntax ist. Fast ohne,
im höheren Sinne ohne.

% In der Logik versucht man, die Objektsprache genau zu fassen.
% Daher nutzt man das Konzept der formalen Sprache. Terme werden
% in ihr durch Produktionsregeln beschrieben. Produktionsregeln
% lassen unter anderem in Form der EBNF notieren.

Schreibt man viele logische Formeln auf, drängt es, zumindest bei privaten
Notizen und Rechnungen, nach Kurzschreibweisen. In der Logik ist für das
Konditional $A\cond B$ auch die Schreibweise $A\rightarrow B$ gebräuchlich,
für das Bikonditional $A\bicond B$ entsprechend $A\leftrightarrow B$.
Insbesondere in der Schaltalgebra schreibt man auch $\overline A$
anstelle von $\lnot A$, $AB$ anstelle von $A\land B$ und $A+B$ anstelle
von $A\lor B$. Hierbei darf die Disjunktion $A+B$ allerdings nicht mit
der Kontravalenz $A\oplus B$ verwechselt werden.


\section{Natürliches Schließen}

\subsection{Darstellungsformen}

Abgeleitet werden soll das Theorem
\[\vdash (A\cond B)\cond (\lnot B\cond\lnot A).\]
Meine favorisierte und in diesem Buch genutzte Form der Darstellung
des natürlichen Schließens fügt die aus den Schlussregeln erhaltenen
Schlüsse wie Legosteine zu einem Baum zusammen, dem
\emph{Beweisbaum}\index{Beweisbaum} oder \emph{Herleitungsbaum}.
Im Eigentlichen stehen in den Blättern die Grundsequenzen, und in der
Wurzel das Theorem. Wie wir es bereits getan haben, arbeitet man
allerdings auch mit Exemplaren, die irgendwelche Sequenzen in
irgendwelche Sequenzen überführen, womit man zulässige Schlussregeln
erhält. Allgemeiner ginge ferner die Formulierung als gerichteter
azyklischer Graph, die bei einigen Beweisen ein wenig den
Schreibaufwand reduzieren würde.

Der Beweisbaum genannten Theorems ist:
\[
\infer[\infernote{$\cond$E}]{\vdash (A\cond B)\cond (\lnot B\cond\lnot A)}{
  \infer[\infernote{$\cond$E}]{A\cond B\vdash \lnot B\cond\lnot A}{
    \infer[\infernote{$\lnot$E}]{A\cond B, \lnot B\vdash\lnot A}{
      \infer[\infernote{$\lnot$B}]{A\cond B, \lnot B, A\vdash\bot}{
        \infer[\infernote{Axiom}]{\lnot B\vdash\lnot B}{}
      & \infer[\infernote{$\cond$B}]{A\cond B, A\vdash B}{
          \infer[\infernote{Axiom}]{A\cond B\vdash A\cond B}{}
        & \infer[\infernote{Axiom}]{A\vdash A}{}}}}}}
\]
Die Ausformulierung der Sequenzen verlangt langwieriges erneutes
Aufschreiben der Antezedenzen. Sobald man das Prozedere einmal
verstanden hat, erscheint es überausführlich. Man kann sich daher
verkürzte Darstellungen der Beweisbäume überlegen:
\[
\begin{tabular}{@{}l@{\qquad\quad}l}
\infer{\vdash (A\cond B)\cond (\lnot B\cond\lnot A)}{
  \infer{2\vdash \lnot B\cond\lnot A}{
    \infer{1, 2\vdash\lnot A}{
      \infer{1, 2, 3\vdash\bot}{
        \infer{1\equiv\lnot B}{}
      & \infer{2, 3\vdash B}{
          \infer{2\equiv A\cond B}{}
        & \infer{3\equiv A}{}}}}}}
&
\infer[\infernote{$\sim$2}]{(A\cond B)\cond (\lnot B\cond\lnot A)}{
  \infer[\infernote{$\sim$1}]{\lnot B\cond\lnot A}{
    \infer[\infernote{$\sim$3}]{\lnot A}{
      \infer{\bot}{
        \infer[\infernote{1}]{\lnot B}{}
      & \infer{B}{
          \infer[\infernote{2}]{A\cond B}{}
        & \infer[\infernote{3}]{A}{}}}}}}
\end{tabular}
\]
Die linke Form kürzt die Antezedenzen durch Nummern ab. In der rechten
Form entfallen die Antezedenzen vollständig. Stattdessen tauchen sie
als in den Blättern gemachte nummerierte \emph{Annahmen} auf, die im
Fortgang zur Wurzel irgendwann zu tilgen sind. Ihre Tilgung erscheint
nun als Randnotiz.

Eine weitere, sehr systematische Darstellung setzt den Beweis aus
einer Liste von Tabellenzeilen zusammen. Allerdings ist sie ein wenig
mühevoll zu lesen. Jede Zeile enthält eine Aussage und dahinter
zusätzlich die Information, wie und woraus die Aussage abgeleitet wurde.
Jede der Aussagen bekommt eine Nummer, siehe Tabelle \ref{tab:Kontraposition}.
Die Nummerierung der Abhängigkeiten ist in derselben Reihenfolge wie
zuvor bei den Bäumen angegeben. Wer die Liste genauer betrachtet, erkennt,
dass die jeweilige Zeile nichts anderes als die Sequenz
$\text{Abh.}\vdash\text{Nr.}$ darstellt.

\begin{table}
\begin{center}
\caption{Beweis in Form einer Liste von Tabellenzeilen}
\label{tab:Kontraposition}
\begin{tabular}{cclll}
\toprule
\strong{Abh.} & \strong{Nr.} & \strong{Aussage} & \strong{Regel} & \strong{auf}\\
\midrule
1 & 1 & $\lnot B$ & Axiom &\\
2 & 2 & $A\cond B$ & Axiom &\\
3 & 3 & $A$ & Axiom &\\
2, 3 & 4 & $B$ & $\cond$B & 2, 3\\
1, 2, 3 & 5 & $\bot$ & $\lnot$B & 1, 4\\
1, 2 & 6 & $\lnot A$ & $\lnot$E & 5\\
2 & 7 & $\lnot B\cond\lnot A$ & $\cond$E & 6\\
$\emptyset$ & 8 & $(A\cond B)\cond(\lnot B\cond\lnot A)$ & $\cond$E & 7\\
\bottomrule
\end{tabular}
\end{center}
\end{table}

Unabhängig von Gentzen entwickelte Jaśkowski das natürliche Schließen
einige Jahre zuvor. Während Gentzen Beweise als Bäume darstellte, nutzte
Jaśkowski zunächst eine grafische Darstellung, die später von Fitch
adaptiert wurde und in dieser Form nun als \emph{Fitch"=Style}%
\index{Fitch-Style} bekannt ist.

Zu guter Letzt muss die klassische Darstellung der Beweisführung
aufgeführt werden. Die in Worten. Sie zeichnet sich durch die Auslassung
mühseliger technischer Details und blumige Formulierungen aus,
soll aber genug Information enthalten, dass der Leser im Zweifel
eine Formalisierung des Beweises erstellen und verifizieren kann.

\begin{Satz}
Es gilt $(A\cond B)\cond (\lnot B\cond\lnot A)$.
\end{Satz}
\strong{Beweis.} Aus der Annahme von sowohl $A\cond B$ als
auch $\lnot B$ als auch $A$ ist ein Widerspruch abzuleiten.
Man erhält $B$ zunächst per Modus ponens aus $A\cond B$ und $A$.
Nun steht $\lnot B$ bereits im Widerspruch zu $B$.\,\qedsymbol

Als komfortablen Bonus erhält man mit dem Theorem nun im Anschluss
kurzerhand eine weitere zulässige Regel, die
\emph{Kontraposition}\index{Kontraposition}
\[\dfrac{\Gamma\vdash A\cond B}{\Gamma\vdash\lnot B\cond\lnot A},
\quad\text{denn}\quad
\dfrac{\vdash (A\cond B)\cond (\lnot B\cond\lnot A)\quad\Gamma\vdash A\cond B}
{\Gamma\vdash \lnot B\cond\lnot A}.\]
Fügt man ihr den Modus ponens an, findet sich der
\emph{Modus tollens}\index{Modus tollens}
\[\dfrac{\Gamma\vdash A\cond B\qquad\Gamma'\vdash\lnot B}
{\Gamma,\Gamma'\vdash\lnot A}.\]

\subsection{Widerspruchsbeweise}

Beim \emph{Beweis durch Widerspruch} widerlegt man eine Aussage, indem
gezeigt wird, dass die Annahme der Aussage zu einem logischen
Widerspruch führt. In manchen Situationen bietet diese Art der
Argumentation eine große Hilfe. So schreibt der britische Mathematiker
Godfrey Harold Hardy in seinem Essay \emph{A Mathematician's Apology}
die Worte
\begin{quote}
»The proof is by \emph{reductio ad absurdum}, and \emph{reductio ad
absurdum}, which Euclid loved so much, is one of a mathematician's
finest weapons. It is a far finer gambit than any chess gambit: a chess
player may offer the sacrifice of a pawn or even a piece, but a
mathematician offers \emph{the game}.«
\end{quote}
Zur Schaffung von Klarheit muss man zunächst zwei inhaltlich
verschiedene Arten des Widerspruchsbeweises unterscheiden.
Präzisieren lässt sich diese Unterscheidung anhand
der Regeln
\[\dfrac{\Gamma,A\vdash\bot}{\Gamma\vdash\lnot A},\qquad
\dfrac{\Gamma,\lnot A\vdash\bot}{\Gamma\vdash A}.\]
Die linke Regel stellt die stets verfügbare Negationseinführung dar,
die man auch als \emph{Widerlegung durch Widerspruch} bezeichnen kann.
In der rechten Regel, der klassischen \emph{Reductio ad absurdum},
gelangt man zunächst per Negationseinführung von
$\Gamma,\lnot A\vdash\bot$ zu $\Gamma\vdash\lnot\lnot A$, und daraufhin
zu $\Gamma\vdash A$. Die Beseitigung der Doppelnegation ist allerdings
lediglich in der klassischen Logik verfügbar, in der intuitionistischen
gilt sie dagegen als unzulässig.




\chapter{Elemente der Modelltheorie}

\section{Semantik der klassischen Aussagenlogik}

\subsection{Die Erfüllungsrelation}

Bislang trat die Logik in der Form eines formalen Systems in
Erscheinung. Gegenstand eines solchen Systems sind im Allgemeinen
\emph{Wörter} einer formalen Sprache; im natürlichen Schließen sind das
die Sequenzen. Einige Wörter, die \emph{Axiome}, werden als gegeben
vorausgesetzt. Unter Anwendung von \emph{Ableitungsregeln}, auch
\emph{Inferenzregeln} genannt, das sind die Schlussregeln, leitet man
aus bereits abgeleiteten Wörtern weitere Wörter der Sprache ab.
In diesem Sinne handelt es sich um ein rein syntaktisches System.

Zum tieferen Verständnis muss man sich im Fortgang damit beschäftigen,
welche inhaltliche Bedeutung den logischen Aussagen beigemessen
wird. Der hierfür wesentliche Schritt besteht in der Definition
einer passenden \emph{Semantik}\index{Semantik}.

Gegenstand der Semantik der Logik ist der Wahrheitsgehalt von Aussagen.
Man hat gefunden, dass es sich mit der Frage nach dem Wesen der Wahrheit
schwierig verhält. Wir wollen daher an dieser Stelle gar nicht erst
versuchen, sie zu ergründen. Stattdessen tritt Wahrheit für uns zunächst
lediglich im leicht fassbaren Rahmen der zweiwertigen booleschen
Algebra auf.

In der klassischen Semantik der Aussagenlogik herrscht das
\emph{Bivalenzprinzip}\index{Bivalenzprinzip}, das besagt, dass jede
Aussage entweder \emph{wahr} oder \emph{falsch} sein muss, also einen
von zwei Wahrheitswerten haben muss. Eine Aussage kann nicht
\emph{ein wenig wahr} oder \emph{halbwegs wahr} sein, noch kann sie
eine von mehreren unterschiedlichen gleichwertigen Wahrheiten haben.
Wir schreiben kurz $0$ für falsch und $1$ für wahr. Enthält eine Formel
logische Variablen, kommt ihr ein Wahrheitswert zu, sobald alle
Variablen durch eine Interpretation\index{Interpretation} mit
einem Wahrheitswert belegt wurden.

Die Art und Weise, wie einer Formel ein Wahrheitswert zukommt,
präzisiert die Erfüllungsrelation. Sie wird als Rekursion über den
Formelaufbau definiert. Der Wahrheitswert einer Formel ist hierbei
einzig und allein durch die Wahrheitswerte ihrer Teilformeln bestimmt.

\begin{Definition}[Erfüllungsrelation]\label{def:sat}\newlinefirst
Eine Interpretation $I$ ist eine Funktion, die jede atomare logische
Variable $v$ mit einem Wahrheitswert $I(v)\in\{0,1\}$ belegt.
Man definiert $I\models A$, sprich »$I$ erfüllt $A$«, rekursiv als
\begin{align*}
(I\models\bot) &\iff 0,\\
(I\models\top) &\iff 1,\\
(I\models v) &\iff I(v),\\
(I\models\lnot A) &\iff \lnot (I\models A),\\
(I\models A\land B) &\iff (I\models A)\land (I\models B),\\
(I\models A\lor B) &\iff (I\models A)\lor (I\models B),\\
(I\models A\cond B) &\iff (I\models A)\cond (I\models B),\\
(I\models A\bicond B) &\iff (I\models A)\bicond (I\models B).
\end{align*}
\end{Definition}
Die rechte Seite ist jeweils metalogisch zu verstehen und per
Wahrheitstafel definiert, siehe Tabelle \ref{tab:Junktoren}.
Die Schreibweise $I\nvDash A$ ist gleichbedeutend mit
$\lnot (I\models A)$. Eine Interpretation wird auch als \emph{Modell}
bezeichnet. Man nennt sie \emph{Modell} einer Formel, falls sie die
Formel erfüllt. Andernfalls spricht man von einem \emph{Kontramodell}
oder \emph{Gegenmodell} der Formel.

\begin{table}
\caption{Wahrheitstafel der Junktoren}
\label{tab:Junktoren}
\centering
\begin{tabular}{cc@{\qquad}c@{\qquad}c@{\qquad}c@{\qquad}c@{\qquad}c}
\toprule
$A$ & $B$ & $\lnot A$ & $A\land B$ & $A\lor B$ & $A\cond B$ & $A\bicond B$\\
\midrule
$0$ & $0$ & $1$ & $0$ & $0$ & $1$ & $1$\\
$1$ & $0$ & $0$ & $0$ & $1$ & $0$ & $0$\\
$0$ & $1$ & $1$ & $0$ & $1$ & $1$ & $0$\\
$1$ & $1$ & $0$ & $1$ & $1$ & $1$ & $1$\\
\bottomrule
\end{tabular}
\end{table}

\begin{Definition}\label{def:sat-context}
Für einen Kontext $\Gamma = \{A_1,\ldots,A_n\}$ setzt man
\[(I\models\Gamma)\defiff (I\models A_1)\land\ldots\land (I\models A_n).\]
\end{Definition}

\subsection{Gültigkeit einer Formel}

Eine wichtige Rolle spielen \emph{allgemeingültige} Formeln, die man
in der Aussagenlogik auch als \emph{Tautologien}\index{Tautologie}
bezeichnet. Sie sind immer wahr, unabhängig davon, mit welchem
Wahrheitswert ihre logischen Variablen belegt werden.

Als allgemeinere Begrifflichkeit wollen auf einen Kontext $\Gamma$
bezogen gültige Formeln $A$ betrachten. Die Idee hierbei ist,
dass wenn die Formeln des Kontextes als wahr angenommen werden, die
Formel $A$ ebenfalls wahr sein muss. Trifft dies auf $A$ zu,
schreibt man $\Gamma\models A$, sprich »im Kontext $\Gamma$ ist
$A$ gültig«, oder auch »$\Gamma$ zieht $A$ nach sich«. Die Bezeichnung
\emph{logische Folgerung} oder \emph{logische Konsequenz} ist
ebenfalls verbreitet.

\begin{Definition}[Gültige Formel]\label{def:valid}\newlinefirst
Eine Formel $A$ heißt genau dann gültig im Kontext $\Gamma$,
wenn jede Interpretation, die sämtliche Formeln von $\Gamma$ erfüllt,
auch $A$ erfüllt. Metalogisch
\[(\Gamma\models A)\defiff \forall I\colon (I\models\Gamma)\cond (I\models A).\]
\end{Definition}
Eine im leeren Kontext gültige aussagenlogische Formel $A$ nennt man
wie gesagt Tautologie. Statt $\emptyset\models A$ schreibt man auch
kurz $\models A$. Wie bei Sequenzen schreibt man auch $\Gamma,A,B\models C$
anstelle von $\Gamma\cup\{A,B\}\models C$.

\subsection{Wahrheitstafeln}

Obgleich der Variablenvorrat unendlich groß sein darf, enthält ein
eine Formel von den Variablen nur endlich viele. Insofern sind für eine
Formel in einem Kontext auch nur endlich viele Interpretationen relevant.
Sind insgesamt $n$ Variablen vorhanden, sind es $2^n$ Interpretationen.

Eine Interpretation $I$ mit der Auswertung $I\models A$ ist nichts
anderes als eine Zeile der Wahrheitstafel der Formel $A$. Eine Formel
ist genau dann tautologisch, wenn in der Ergebnisspalte in jeder
Zeile eine~1 steht.

Die Wahrheitstafel \ref{tab:Tautologie-zur-Kontraposition} bestätigt
\[\models (a\cond b)\bicond (\lnot b\cond\lnot a).\]

\begin{table}
\caption{Wahrheitstafel der Tautologie zur Kontraposition}
\label{tab:Tautologie-zur-Kontraposition}
\centering
\begin{tabular}{cc@{\quad\;\;}c@{\quad\;\;}c@{\quad\;\;}c@{\quad\;\;}c@{\quad\;\;}c}
\toprule
$a$ & $b$ & $\lnot b$ & $\lnot a$ & $a\cond b$ & $\lnot b\cond\lnot a$
& $(a\cond b)\bicond (\lnot b\cond\lnot a)$\\
\midrule
$0$ & $0$ & $1$ & $1$ & $1$ & $1$ & $1$ \\
$1$ & $0$ & $1$ & $0$ & $0$ & $0$ & $1$ \\
$0$ & $1$ & $0$ & $1$ & $1$ & $1$ & $1$ \\
$1$ & $1$ & $0$ & $0$ & $1$ & $1$ & $1$ \\
\bottomrule
\end{tabular}
\end{table}

\noindent
Die Tafel führt zusätzlich die Teilformeln auf, was bei längeren
Formeln recht mühselig erscheinen mag. Eine geschickte Methode zur
Reduzierung des Schreibaufwands erspart die Teilformeln, und setzt ihre
Wahrheitswerte dafür schlicht unter die Junktoren, denn Ziffern
benötigen nicht viel Platz.

Die Prüfung einer logischen Folgerung per Wahrheitstafel
ermöglicht die metalogische Beziehung
\[(A_1,\ldots,A_n\models A) \iff (\models A_1\land\ldots\land A_n\cond A).\]
Nämlich findet sich die äquivalente Umformung
\begin{align*}
(A_1,\ldots,A_n\models A)
&\;\Longleftrightarrow_{\text{(1)}}\;
(\forall I\colon (I\models A_1)\land\ldots\land (I\models A_n)\cond (I\models A))\\
&\;\Longleftrightarrow_{\text{(2)}}\;
(\forall I\colon (I\models A_1\land\ldots\land A_n)\cond (I\models A))\\
&\;\Longleftrightarrow_{\text{(3)}}\;
(\forall I\colon I\models A_1\land\ldots\land A_n\cond A)\\
&\;\Longleftrightarrow_{\text{(4)}}\;
(\models A_1\land\ldots\land A_n\cond A).
\end{align*}
Hierbei gilt (1), (4) gemäß Def. \ref{def:valid}, \ref{def:sat-context}
und (2), (3) gemäß Def. \ref{def:sat}.

\subsection{Korrektheit des natürlichen Schließens}

Jede Schlussregel und jedes Axiom besitzt eine semantische Entsprechung.
Die Beweise dafür werden nun auf metalogischer Ebene mittels natürlichem
Schließen selbst erbracht, was wie eine Art Zirkelschluss erscheinen
mag.

Zu den Grundsequenzen und zur Abschwächungsregel findet sich:
\[
\begin{array}{@{}l@{\qquad}l}
\infer[\infernote{Def. \ref{def:valid}}]{\Gamma\cup\{A\}\models A}{
  \infer[\infernote{$\sim$1}]{(I\models \Gamma\cup\{A\})\cond (I\models A)}{
    \infer{I\models A}{
      \infer[\infernote{Def. \ref{def:sat-context}}]{(I\models\Gamma)\land (I\models A)}{
        \infer[\infernote{1}]{I\models\Gamma\cup\{A\}}{}}}}}
&
\infer[\infernote{Def. \ref{def:valid}}]{\Gamma\cup\Gamma'\models A}{
  \infer[\infernote{$\sim$1}]{(I\models\Gamma\cup\Gamma')\cond (I\models A)}{
    \infer{I\models A}{
      \infer[\infernote{Def. \ref{def:valid}}]{(I\models\Gamma)\cond (I\models A)}{
        \Gamma\models A}
    & \infer[\infernote{Def. \ref{def:sat-context}}]{I\models\Gamma}{
        \infer[\infernote{1}]{I\models\Gamma\cup\Gamma'}{}}}}}
\end{array}
\]
\begin{Satz}[Korrektheit des natürlichen Schließens]\newlinefirst
Ist die Sequenz $\Gamma\vdash A$ ableitbar, so muss auch
$\Gamma\models A$ gelten.
\end{Satz}
\strong{Beweis.}
Strukturelle Induktion über die Konstruktion von Beweisbäumen.
Induktionsanfänge sind die semantischen Entsprechungen
der Grundsequenzen. Induktionsschritte sind die semantischen
Entsprechungen der Schlussregeln. Die Beweise der Entsprechungen
wurden bereits diskutiert.\,\qedsymbol

\newpage
\subsection{Logische Äquivalenz}
\begin{Definition}[Äquivalente Formeln]\newlinefirst
Die Äquivalenz zweier Formeln $A,B$ ist definiert als
\[(A\equiv B)\defiff (\models A\bicond B).\]
\end{Definition}

\noindent
Eine Äquivalenz besteht genau dann, wenn jede der beiden Formeln
eine logische Folgerung der anderen ist. Das heißt, es besteht die
metalogische Beziehung
\[(\models A\bicond B)\iff (A\models B)\land (B\models A).\]
Mit den semantischen Entsprechungen der Schlussregeln
findet sich nämlich:
\[
\begin{array}{@{}l@{\qquad\quad}l}
\infer{A\models B}{
  \infer{\models A\cond B}{
    \models A\bicond B}
& \infer{A\models A}{}}
&
\infer{\models A\bicond B}{
  \infer{\models A\cond B}{
    A\models B}
& \infer{\models B\cond A}{
    B\models A}}
\end{array}
\]
\begin{Satz}\label{log-Aeq-ist-Aeqrel}
Es ist $A\equiv B$ eine Äquivalenzrelation. Das heißt, es gilt
\begin{align*}
& A\equiv A, &&\text{(Reflexivität)}\\
& (A\equiv B)\cond (B\equiv A), &&\text{(Symmetrie)}\\
& (A\equiv B)\land (B\equiv C)\cond (A\equiv C). &&\text{(Transitivität)}
\end{align*}
\end{Satz}
Der Beweis sei dem Leser als kleine Übung überlassen.

Der Satz \ref{log-Aeq-ist-Aeqrel} vermittelt, dass Formeln mit
Äquivalenzen so umgeformt werden dürfen, wie Terme mit Termumformungen.
Es finden sich im Fortgang eine Reihe von grundlegenden Äquivalenzen,
die Regeln der \emph{booleschen Algebra}.



\chapter{Mengenlehre}

\section{Grundbegriffe}

\subsection{Der Mengenbegriff}

Eine \emph{Menge}\index{Menge} darf man sich wie einen Beutel
vorstellen, der einzelne Objekte enthält. Die Objekte heißen
\emph{Elemente}\index{Element} der Menge. Jedoch gilt es hierbei zu
beachten, dass es sich mit einer Menge nicht gänzlich wie mit einem
Beutel verhält, in dem dasselbe Objekt mehrmals zu finden sein kann.
\begin{quote}
»\emph{Unter einer Menge verstehen wir jede Zusammenfassung von
bestimmten wohlunterschiedenen Objekten unserer Anschauung oder
unseres Denkens zu einem Ganzen.}«\\
--- Georg Cantor, 1895 (redigiert aus \cite{Cantor})
\end{quote}
Obwohl blumig anmutend, fassen diese Worte das Konzept recht gut
auf den Punkt. Wichtig ist hier das Wort \emph{wohlunterschieden},
das uns zu verstehen gibt, dass ein Element nicht mehrmals in einer
Menge enthalten sein kann. \emph{Eine Menge ist genau dadurch festgelegt,
welche Elemente sie enthält. Sie enthält Elemente weder mehrmals, noch
in einer bestimmten Reihenfolge.}

Es gibt die \emph{leere Menge}\index{leere Menge}, notiert als
$\emptyset$ oder $\{\}$. Man darf sie sich wie einen leeren Beutel
vorstellen. Dagegen enthält die Menge $\{\emptyset\}$ genau ein Element.
Es ist ein Beutel,  der den leeren Beutel enthält.

Wir schreiben kurz $x\in A$ für »$x$ ist ein Element von $A$«,
auch »$x$ gehört zu $A$« oder »$x$ liegt in $A$«.
Es steht $x\notin A$ für $\lnot x\in A$.

Für eine endliche Menge definiert man
\[x\in\{x_1,\ldots,x_n\}\defiff x=x_1\lor\ldots\lor x=x_n.\]

\subsection{Gleichheit von Mengen}

Es wurde gesagt, eine Menge ist dadurch charakterisiert, welche
Elemente sie enthält. Um diesen Gedanke näher zu erfassen, sollten
wir klären, wie es sich mit der Gleichheit von Mengen verhält.
Insofern Mengen durch ihre Elemente bestimmt sind, darf man doch sagen,
zwei Mengen $A,B$ sind genau dann gleich, falls $A,B$ ein Objekt
gemeinsam enthalten oder gemeinsam nicht enthalten. Das heißt, betrachtet
man ein beliebiges Objekt $x$, so ist $x$ genau dann in $A$ enthalten,
wenn $x$ in $B$ enthalten ist.
\begin{Definition}[Gleichheit von Mengen]\label{def:Mengen-Gleichheit}%
\newlinefirst
Für zwei Mengen $A,B$ definiert man
\[A = B \defiff (\forall x\colon x\in A\bicond x\in B).\]
\end{Definition}
Der Leser wird mühelos bestätigen, dass die Gleichheit die Axiome
einer Äquivalenzrelation erfüllt. Wobei der Begriff der Relation
noch zu definieren wäre. Weil die Gesamtheit aller Mengen eine
sogenannte echte Klasse ist, handelt es sich nicht um eine
Äquivalenzrelation im engeren Sinne.

Es ist zulässig, ein Element bei der aufzählenden Angabe einer Menge
mehrmals aufzuführen. Dies ändert allerdings nichts daran, dass ein
Element stets nur einmal in einer Menge enthalten ist. Zum Beispiel
gilt $\{\emptyset,\emptyset\}=\{\emptyset\}$. Mit der Definition der
aufzählenden Angabe und dem Idempotenzgesetz
der Aussagenlogik findet sich nämlich die äquivalente Umformung%
\[x\in\{\emptyset,\emptyset\}\iff x=\emptyset\lor x=\emptyset
\iff x=\emptyset\iff x\in\{\emptyset\}.\]

\subsection{Beschränkte Quantifizierung}

In der Mathematik erstreckt sich die Quantifizierung meist nicht
über das gesamte Diskursuniversum, sondern bleibt auf eine bestimmte
Menge beschränkt. Eine extra Notation macht dies ergonomisch, wobei
eine Erweiterung der logischen Sprache hierfür nicht nötig ist. Die
beschränkte Quantifizierung wird logisch auf eine unbeschränkte
zurückgeführt.

\begin{Definition}[Beschränkte Quantifizierung]\newlinefirst
Für jede Menge $M$ und jede Aussageform $A(x)$ setzt man
\begin{gather*}
(\forall x\in M\colon A(x))\defiff (\forall x\colon x\in M\cond A(x)),\\
(\exists x\in M\colon A(x))\defiff (\exists x\colon x\in M\land A(x)).
\end{gather*}
\end{Definition}
Die Aussage $\forall x\in\emptyset\colon A(x)$ ist allgemeingültig, man
spricht von der \emph{leeren Wahrheit}\index{leere Wahrheit}, engl.
\emph{vacuous truth}\index{vacuous truth}. Via Ex falso quodlibet
erhält man nämlich:
\[
\infer{\vdash\forall x\colon x\in\emptyset\cond A(x)}{
  \infer[\infernote{EFQ}]{x\in\emptyset\vdash A(x)}{
    \infer{x\in\emptyset\vdash\bot}{
      \infer{\vdash\lnot x\in\emptyset}{}
    & \infer{x\in\emptyset\vdash x\in\emptyset}{}}}}
\]
Viele Regeln zur beschränkten Quantifizierung sind analog zu den
Regeln der unbeschränkten. Beispielsweise gilt%
\[(\forall x\in M\colon A(x)\land B(x)) \iff (\forall x\in M\colon A(x))
\land (\forall x\in M\colon B(x)).\]
Man muss allerdings Vorsicht walten lassen. Nicht bei jeder Äquivalenz
liegt eine direkte Analogie vor. Zwar besteht für eine Formel $A$,
in der $x$ nicht frei vorkommt, die Äquivalenz%
\[(\exists x\colon A) \iff A.\]
Die Analogie ist jedoch von der ein klein wenig intrikateren Form
\[(\exists x\in M\colon A) \iff M\ne\emptyset\land A.\]
Diese Beziehung erklärt sich durch die Umformung
\[(\exists x\in M\colon A)\bicond (\exists x\colon x\in M\land A)
\bicond (\exists x\colon x\in M)\land A \bicond M\ne\emptyset\land A.\]
Die letzte Umformung gilt, weil $M\ne\emptyset$ gleichbedeutend mit
$\exists x\colon x\in M$ ist.

\subsection{Komprehension}

Es wird $\{x\mid A(x)\}$ gelesen als »die Klasse der $x$, für die
$A(x)$ gilt« oder »die Menge der $x$, für die $A(x)$ gilt«.
\begin{Definition}[Komprehension]\newlinefirst
Zu einer Aussageform $A(x)$ definiert man die Klasse
$\{x\mid A(x)\}$ gemäß
\[a\in\{x\mid A(x)\}\defiff A(a).\]
\end{Definition}
Man muss mit der Komprehension ein wenig vorsichtig umgehen, denn nicht
jede Klasse ist eine Menge. Die russellsche Klasse $R := \{x\mid x\notin x\}$
ist das klassische Beispiel. Angenommen, $R$ wäre eine
Menge. Dann dürfte man eine Aussage wie $R\in M$ bezüglich
einer Menge $M$ formulieren, also speziell $R\in R$. Gemäß Definition
von $R$ findet sich die folgende Ableitung:
\[
\infer{\vdash R\in R\bicond R\notin R}{
  \infer[\infernote{laut Def.}]{R\in R\vdash R\notin R}{
    \infer{R\in R\vdash R\in R}{}}
& \infer[\infernote{laut Def.}]{R\notin R\vdash R\in R}{
    \infer{R\notin R\vdash R\notin R}{}}}
\]
Für jede Formel $A$ gilt allerdings das Theorem
\[\vdash (A\bicond\lnot A)\cond\bot.\]
Insgesamt ergibt sich so ein Beweis der Kontradiktion. Irgendetwas kann
also nicht gut sein. Diese von Bertrand Russell im Jahre 1901 entdeckte
Verwicklung, die seit jeher den Namen \emph{russellsche Antinomie}%
\index{russellsche Antinomie} trägt, brachte Gottlob Freges
logizistisches Programm in unangenehme Schwierigkeiten. Frege versuchte,
eine Reduktion der Mathematik auf die Logik zu unternehmen, wurde
daraufhin aber von Russell brieflich in Kenntnis gesetzt, dass sein
Werk \emph{Grundgesetze der Arithmetik} in wesentlicher Weise von der
Antinomie unterhöhlt wird. Russell führte das Programm fort, und
veröffentlichte schließlich die \emph{Principia Mathematica}, die der
Antinomie mithilfe einer Typentheorie aus dem Weg geht.

Man begegnet der Problematik, indem man $R$ als eine echte
Klasse ansieht. Für sie darf die Aussage $R\in M$ nicht formuliert
werden. Man definiert weiterhin eine weniger allgemeine Form der
Komprehension, die \emph{Aussonderung}\index{Aussonderung}. Sie verhält
sich gutartig, da sie nicht mehr ermöglicht, als das Ausfiltern von
Elementen aus einer gegebenen Menge.

\begin{Definition}[Aussonderung]\label{def:Aussonderung}\newlinefirst
Zu einer Menge $M$ und einer Aussageform $A(x)$ definiert man
\[a\in\{x\in M\mid A(x)\}\defiff a\in M\land A(a).\]
\end{Definition}

\subsection{Teilmengen}

Gehört jedes Element einer Menge $A$ auch zu einer Menge $B$, nennt
man $A$ eine \emph{Teilmenge}\index{Teilmenge} von $B$. Man sagt auch,
$B$ umfasse $A$, oder $B$ sei eine Obermenge von $A$. Eine \emph{echte
Teilmenge} sei $A$ dann, wenn zusätzlich $A\ne B$ gilt. Die Menge der
geraden Zahlen ist eine echte Teilmenge der ganzen Zahlen. Jede Menge
ist eine Teilmenge von sich selbst, jedoch keine echte.

Die Menge der Quadrate ist eine Teilmenge der Vierecke, genauer eine
Teilmenge der Rechtecke und auch eine Teilmenge der Rhomben.
Weder ist die Menge der Rechtecke eine Teilmenge der Rhomben, noch
ist die Menge der Rhomben eine Teilmenge der Rechtecke. 
Allerdings ist sowohl die Menge der Rechtecke als auch die der Rhomben
eine Teilmenge der Parallelogramme.

\begin{Definition}[Teilmengenbeziehung]\newlinefirst
Man definiert $A\subseteq B$, gelesen »$A$ ist eine Teilmenge von $B$«, als
\[A\subseteq B\defiff (\forall x\colon x\in A\cond x\in B).\]
\end{Definition}
Unschwer bestätigt sich die Äquivalenz
\[A = B \iff A\subseteq B\land B\subseteq A.\]

\begin{Definition}[Potenzmenge]\newlinefirst
Die Potenzmenge\index{Potenzmenge} einer Menge $M$ ist definiert als
\[\mathcal P(M) := \{A\mid A\subseteq M\}.\]
\end{Definition}
Zum Beispiel ist $\mathcal P(\emptyset) = \{\emptyset\}$ und
$\mathcal P(\{0\}) = \{\emptyset, \{0\}\}$. Des Weiteren
\begin{gather*}
\mathcal P(\{0,1\}) = \{\emptyset, \{0\}, \{1\}, \{0,1\}\},\\
\mathcal P(\{0,1,2\}) = \{\emptyset, \{0\}, \{1\}, \{2\}, \{0,1\}, \{0,2\}, \{1,2\}, \{0,1,2\}\}.
\end{gather*}
Die Teilmengenbeziehung darf als eine Art Ordnung zwischen Mengen
betrachtet werden, jedoch nicht als eine Totalordnung, das heißt,
bei einigen Mengen $A,B$ gilt weder $A\subseteq B$ noch $B\subseteq A$.
So ist wie gesagt weder die Menge der Rechtecke eine Teilmenge der
Rhomben, noch umgekehrt.

Die Potenzmenge einer Grundmenge $G$ bildet mit der Teilmengenbeziehung
eine halbgeordnete Menge, engl. \emph{poset} für
\emph{partially ordered set}. Das heißt, alle Mengen
$A,B,C\in\mathcal P(G)$ erfüllen die drei Axiome
\begin{align*}
& A\subseteq A, &&\text{(Reflexivität)}\\
& A\subseteq B\land B\subseteq A\cond A = B, &&\text{(Antisymmetrie)}\\
& A\subseteq B\land B\subseteq C\cond A\subseteq C. &&\text{(Transitivität)}
\end{align*}


\subsection{Mengenoperationen}

\begin{Definition}[Schnitt, Vereinigung, Differenz]\newlinefirst
Zu zwei Mengen $A,B$ definiert man
\begin{align*}
A\cap B &:= \{x\mid x\in A\land x\in B\}, && \text{(Schnittmenge)}\\
A\cup B &:= \{x\mid x\in A\lor x\in B\}, && \text{(Vereinigungsmenge)}\\
A\setminus B &:= \{x\mid x\in A\land x\notin B\}. && \text{(Differenzmenge)}
\end{align*}
\end{Definition}
Sind $A,B$ Teilmengen einer Grundmenge $G$, so sind auch
$A\cap B$, $A\cup B$ und $A\setminus B$ Teilmengen der Grundmenge.
Der Beweis zu $A\cap B\subseteq G$ ist:
\[
\infer{A\subseteq G\vdash A\cap B\subseteq G}{
  \infer{A\subseteq G\vdash\forall x\colon x\in A\cap B\cond x\in G}{
    \infer{A\subseteq G, x\in A\cap B\vdash x\in G}{
      \infer{x\in A\cap B\vdash x\in A}{
        \infer{x\in A\cap B\vdash x\in A\land x\in B}{
          \infer{x\in A\cap B\vdash x\in A\cap B}{}}}
    & \infer{A\subseteq G\vdash x\in A\cond x\in G}{
        \infer{A\subseteq G\vdash \forall x\colon x\in A\cond x\in G}{
          \infer{A\subseteq G\vdash A\subseteq G}{}}}}}}
\]
In so pedantischer Ausführlichkeit findet man Beweise in Büchern
nicht vor. Erstens wird man stillschweigend zulässige Schlussregeln zur
Verkürzung aufstellen. So nimmt der Baum die konzise Form
\[
\begin{array}{@{}l@{\qquad\quad}l}
\infer{A\subseteq G\vdash A\cap B\subseteq G}{
  \infer{A\subseteq G,x\in A\cap B\vdash x\in G}{
    \infer{x\in A\cap B\vdash x\in A}{
      \infer{x\in A\cap B\vdash x\in A\cap B}{}}
  & \infer{A\subseteq G\vdash A\subseteq G}{}}}
&
\infer[\infernote{$\sim$1}]{A\cap B\subseteq G}{
  \infer{x\in G}{
    \infer{x\in A}{
      \infer[\infernote{1}]{x\in A\cap B}{}}
  & A\subseteq G}}
\end{array}
\]
an. Zweitens formuliert der Mathematiker den Beweis meist in Worten:
Um $A\cap B\subseteq G$ zu zeigen, muss $x\in G$ aus $x\in A\cap B$
abgeleitet werden. Mit $x\in A\cap B$ gilt erst recht $x\in A$.
Wegen $A\subseteq G$ ist somit $x\in G$, was zu zeigen war.\,\qedsymbol

\begin{Definition}[Komplement]\newlinefirst
Bezüglich einer Grundmenge $G$ heißt $A^\compc := G\setminus A$
Komplementärmenge.
\end{Definition}
Es zeigt sich elementar, dass die Potenzmenge einer Grundmenge $G$
mit den Operationen $A\cap B$, $A\cup B$ die Axiome
einer booleschen Algebra\index{boolesche Algebra} erfüllt, wobei
$\emptyset$ das Nullelement und $G$ selbst das Einselement ist. Es
gelten somit analoge Regeln wie in der klassischen Aussagenlogik. Wie
die Notation suggeriert, entspricht der Schnitt der Konjunktion, die
Vereinigung der Disjunktion und das Komplement der Negation.

Wie in jedem Verband ist in einer booleschen Algebra
$(M,\wedge,\vee)$ für $a,b\in M$ eine Halbordnung $a\le b$
definiert, indem $a\le b$ und $a\wedge b = a$ als äquivalent
angesehen werden. Bei der Mengenalgebra entpuppt sie sich
als die Teilmengenrelation. Das heißt, es gilt%
\[A\subseteq B\iff A\cap B = A\iff A\cup B = B.\]
Für mehrere Mengen definiert man
\[\bigcap_{i=1}^n A_i := A_1\cap A_2\cap\ldots\cap A_n,\qquad
\bigcup_{i=1}^n A_i := A_1\cup A_2\cup\ldots\cup A_n.\]
Pedantiker mögen die Schreibweise mit den Auslassungspunkten als
ungenau empfinden. Zufriedenstellend ist für sie die Erklärung, dass
es sich um die rekursive Festlegung%
\[\bigcap_{i=1}^1 A_i := A_1,\qquad
\bigcap_{i=1}^n A_i := A_n\cap\bigcap_{i=1}^{n-1} A_i\]
handelt. Regeln wie $B\cup\bigcap_{i=1}^n A_i
= \bigcap_{i=1}^n (B\cup A_i)$ kann man nun per Induktion über $n$
beweisen. Es geht fast trivial vonstatten, sobald man das Prinzip
verstanden hat. Im Wesentlichen weiten sich hier die Regeln der
booleschen Algebra von den zweistelligen auf die mehrstelligen
Operationen aus.

\begin{Definition}[Allgemeine Vereinigung]\newlinefirst
Sei $M$ eine Menge von Mengen. Die Vereinigung der $A\in M$ ist
\[\bigcup M = \bigcup_{A\in M} A := \{x\mid\exists A\in M\colon x\in A\}.\]
\end{Definition}
Für $M=\emptyset$ ist $\bigcup M = \emptyset$. Die Disjunktion findet
ihre Entsprechung genau in der Vereinigung
von zwei Mengen. Dazu passend findet der Existenzquantor seine
Entsprechung genau in der Vereinigung beliebig vieler Mengen.
Aus diesem Grund weiten sich die Regeln der booleschen Algebra
auf die allgemeine Vereinigung aus. Zum Beispiel lautet das
Distributivgesetz für Mengen%
\[B\cap\bigcup_{A\in M} A = \bigcup_{A\in M}(B\cap A).\]
Entfaltung der Definition führt nämlich zur logischen Äquivalenz
\[x\in B\land(\exists A\in M\colon x\in A) \iff (\exists A\in M\colon x\in B\land x\in A).\]
Ihr Beweis gelingt mühelos.

\begin{Definition}[Allgemeiner Schnitt]\newlinefirst
Sei $M$ eine nichtleere Menge von Mengen. Der Schnitt der $A\in M$ ist%
\[\bigcap M = \bigcap_{A\in M} A := \{x\mid\forall A\in M\colon x\in A\}.\]
\end{Definition}
Im Gegensatz zur Vereinigung wurde der Schnitt $\bigcap M$ 
für $M=\emptyset$ undefiniert gelassen. Hier gibt es zwei Möglichkeiten.
Zum einen könnte man die Bedingung $M\ne\emptyset$ einfach fallen
lassen, dann ergibt im allgemeinen Mengenuniversum beim leeren Schnitt
die Allklasse $\{x\mid\top\}$, die jedoch keine Menge ist.

Aus diesen Grund gibt es noch die alternative Definition
\[\bigcap M :=
\{x\in G\mid\forall A\in M\colon x\in A\}.\]
Hierzu ist eine Grundmenge $G$ festzulegen, so dass $M\subseteq\mathcal P(G)$
gilt, oder man setzt $G:=\bigcup M$, wobei sich da die Frage
nach der Nützlichkeit stellt.

Eine Menge von Mengen nennt man ein \emph{Mengensystem}\index{Mengensystem},
wobei aber einige Autoren diese Begrifflichkeit für eine Familie von
Mengen benutzen, die von einer Menge von Mengen zu unterscheiden ist.
Eine Familie stellt eine Verallgemeinerung einer Folge von Mengen dar.
In ihr darf dieselbe Menge mehrmals vorkommen. Man kann Schnitt und
Vereinigung auch für Familien definieren, was aber eigentlich keine
wesentliche Verallgemeinerung zu den obigen Festlegungen darstellt,
wie ich im Folgenden diskutieren möchte.

Eine \emph{Familie}\index{Familie} $(A_i)$ von Mengen $A_i$ mit $i\in I$
ist eine Abbildung $A\colon I\to Z$, wobei $Z$ eine Zielmenge ist,
welche die $A_i$ als Elemente enthält. Die Menge $I$ wird in diesem
Zusammenhang auch \emph{Indexmenge}\index{Indexmenge} genannt. Man
definiert%
\[\bigcup_{i\in I} A_i := \bigcup A(I)
= \bigcup\{X\mid\exists i\in I\colon X=A_i\} = \{x\mid\exists i\in I\colon x\in A_i\},\]
wobei mit $A(I)$ das Bild von $I$ unter $A$ gemeint ist. Man bekommt%
\begin{align*}
\smash{\bigcup_{i\in I} A_i}
&= \{x\mid\exists X\colon X\in \{X\mid\exists i\in I\colon X=A_i\}\land x\in X\}\\
&= \{x\mid\exists X\colon (\exists i\in I\colon X=A_i)\land x\in X\}\\
&= \{x\mid\exists X\colon \exists i\in I\colon X=A_i\land x\in X\}
= \{x\mid\exists i\in I\colon x\in A_i\}.
\end{align*}
Für $I\ne\emptyset$ definiert man entsprechend
\[\bigcap_{i\in I} A_i := \bigcap A(I) = \{x\mid\forall i\in I\colon x\in A_i\}.\]
Die Operation über eine Familie $(A_i)_{i\in I}$ kann also
auf die jeweilige Operation über das System $A(I)$ zurückgeführt
werden.

Später nützlich ist der

\begin{Satz}\label{Index-in-Schnitt}
Es gilt $\bigcup_{i\in I\cap J} A_i = (\bigcup_{i\in I} A_i)\cap
(\bigcup_{i\in J} A_j)$.
\end{Satz}
\begin{Beweis}
Es findet sich die äquivalente Umformung
\begin{align*}\textstyle
x\in\bigcup_{i\in I\cap J} A_i &\iff (\exists i\colon i\in I\cap J\land x\in A_i)\\
&\iff(\exists i\colon (i\in I\land x\in A_i)\land (i\in J\land x\in A_i))\\
&\iff(\exists i\colon i\in I\land x\in A_i)\land (\exists i\colon i\in J\land x\in A_i)\\
&\iff\textstyle x\in\bigcup_{i\in I} A_i\land x\in\bigcup_{i\in J} A_i\\
&\iff\textstyle x\in(\bigcup_{i\in I} A_i)\cap (\bigcup_{i\in J} A_i).\,\qedsymbol
\end{align*}
\end{Beweis}

\noindent
Gelegentlich hat man es mit einer \emph{disjunkten Vereinigung}%
\index{disjunkte Vereinigung} zu tun. Sie ist bedeutsam in der
Theorie der Kardinalzahlen und in der Informatik bei algebraischen
Datentypen sowie deren Bezug zur Beweistheorie. Genauer gesagt hantiert
man in der Informatik nicht mit Mengen, sondern mit Typen, die sich
in gewissen Zügen analog zu Mengen verhalten. Die disjunkte Vereinigung
zweier Mengen kennzeichnet jedes Element vor der Vereinigung mit einem
Tag, das die Information liefert, aus welcher der Mengen es entstammt.
Man setzt%
\[A\sqcup B := \{(0,a)\mid a\in A\}\cup\{(1,b)\mid b\in B\}.\]
Die Zahlen $0,1$ sind hier die \emph{Tags} oder \emph{Diskriminatoren}.
Anstelle der Zahlen könnten genauso gut zwei beliebige unterschiedliche
Elemente als Tags verwendet werden. Beispielsweise ginge auch%
\[A\sqcup B = \{(\text{Grün},a)\mid a\in A\}\cup\{(\text{Blau},b)\mid b\in B\}.\]
In der Informatik verwendet man gern left, right als Tags.

Mit jeder disjunkten Vereinigung ist eine Fallunterscheidung verbunden.
Liegt ein $x\in A\sqcup B$ vor, so muss entweder $x=(0,a)$ für ein
$a\in A$ oder $x=(1,b)$ für ein $b\in B$ sein. Bei einer gewöhnlichen
Vereinigung besteht dagegen kein ausschließendes Oder.

Wir fassen zwei Objekte $x,y$ zu einem \emph{geordneten Paar}%
\index{Paar}\index{geordnetes Paar} $(x,y)$ zusammen. Die beiden
Objekte müssen nicht unbedingt etwas miteinander zu tun haben, sie
dürfen völlig verschiedener Art sein. Im Unterschied zu einer Menge
spielt bei Paaren die Reihenfolge eine Rolle, auch darf dasselbe Objekt
zweimal vorkommen. Im Paar $t=(x,y)$ ist genau die Information über
$x,y$ enthalten. Das heißt, es lassen sich $x,y$ aus dem Paar
extrahieren. Man schreibt dafür $t_1=x$ und $t_2=y$. Die Schreibweise
$t_i$ heißt \emph{Indizierung}, es ist darin $i$ der
\emph{Index}\index{Index}.

Zwei Paare seien definitionsgemäß genau dann gleich, wenn sie
komponentenweise gleich sind,%
\[(x,y) = (x',y')\defiff x=x'\land y=y'.\]
Die genannten Eigenschaften charakterisieren den Begriff Paar
im Wesentlichen, mehr müssen wir nicht wissen. Man hat sich trotzdem
auch mal überlegt, wie Paare in der reinen Mengenlehre dargestellt
werden können, wo alle Objekte Mengen sein sollen. Nach Kuratowski
sind Paare kodierbar als%
\[(x,y) := \{\{x\},\{x,y\}\}.\]
Unter dem allgemeineren Begriff \emph{Tupel}\index{Tupel} fassen
wir eine beliebige endliche Zahl von Objekten in einer bestimmten
Reihenfolge zusammen. Tupel aus drei Objekten heißen \emph{Tripel},
die aus vier heißen \emph{Quadrupel}. Analog zu den Paaren ist ihre
Gleichheit definiert als%
\[(x_1,\ldots,x_n) = (x_1',\ldots,x_n')
\defiff x_1=x_1'\land\ldots\land x_n=x_n'.\]
Zu zwei Mengen $X,Y$ kann man nun die Menge aller Paare betrachten,
deren erste Komponente $X$ entstammt, und deren zweite
Komponente $Y$ entstammt. Sie heißt \emph{Produktmenge}\index{Produktmenge}
oder \emph{kartesisches Produkt}\index{kartesisches Produkt}
der Mengen $X,Y$.
\begin{Definition}[Produktmenge]\newlinefirst
Das kartesische Produkt zweier Mengen $X,Y$ ist die Menge
\[X\times Y := \{(x,y)\mid x\in X\land y\in Y\}.\]
\end{Definition}
Genau genommen handelt es sich hierbei um eine Bildmenge, was bedeutet,
dass sich im rechten Term Existenzquantoren verstecken. Ausführlich
ausgeschrieben lautet der Term%
\begin{align*}
X\times Y &= \{t\mid\exists x\in X\colon\exists y\in Y\colon t=(x,y)\}\\
&= \{t\mid\exists x\colon\exists y\colon x\in X\land y\in Y\land t=(x,y)\}.
\end{align*}
Wie jede Bildmenge ist das Produkt als Vereinigung darstellbar. Es ist%
\[X\times Y = \bigcup_{x\in X}\bigcup_{y\in Y}\{(x,y)\}.\]
Die beiden kurzen Identitäten $X\times\emptyset = \emptyset$ und
$\emptyset\times Y = \emptyset$ gehen unmittelbar aus der Definition
hervor. Es verhält sich analog wie mit der Multiplikation einer
Zahl mit null. Eine sinnvolle Sichtweise, wie die Theorie der
Kardinalzahlen lehrt.

\begin{Satz}
Ist $A\subseteq X$ und $B\subseteq Y$, dann ist
$A\times B\subseteq X\times Y$.
\end{Satz}
\begin{Beweis}
Es liege $t$ in $A\times B$. Laut Definition existieren mithin
$a\in A$ und $b\in B$, so dass $t=(a,b)$. Wegen $A\subseteq X$ ist aber
auch $a\in X$ und wegen $B\subseteq Y$ ist auch $b\in Y$. Daher existieren
$a\in X$ und $b\in Y$, so dass $t=(a,b)$. Gemäß Definition heißt das,
$t\in X\times Y$. Gemäß Definition ist $A\times B$ somit eine Teilmenge
von $X\times Y$.\;\qedsymbol
\end{Beweis}

\begin{Satz}
Es gilt $X\times (A\cap B) = (X\times A)\cap (X\times B)$.
\end{Satz}
\begin{Beweis}
Es ginge zu Fuß. Aber Satz \ref{Index-in-Schnitt} ermöglicht die
kurze Termumformung%
\begin{align*}
& X\times (A\cap B) = \bigcup_{x\in X}\bigcup_{y\in A\cap B} \{(x,y)\}
= \bigcup_{x\in X}\Big(\Big(\bigcup_{y\in A}\{(x,y)\}\Big)\cap \Big(\bigcup_{y\in B}\{(x,y)\}\Big)\Big)\\
&\quad = \Big(\bigcup_{x\in X}\bigcup_{y\in A}\{(x,y)\}\Big)\cap\Big(\bigcup_{x\in X}\bigcup_{y\in B}\{(x,y)\}\Big)
= (X\times A)\cap (X\times B).\,\qedsymbol
\end{align*}
\end{Beweis}

\newpage
\section{Abbildungen}

\subsection{Der Abbildungsbegriff}

Unter einer \emph{Abbildung}\index{Abbildung}, auch \emph{Funktion}%
\index{Funktion} genannt, verstehen wir ganz allgemein eine Zuordnung von
Elementen einer Definitionsmenge zu Elementen einer Zielmenge, bei der
zu \emph{jedem} Element der Definitionsmenge \emph{genau ein} Element
der Zielmenge gehört. Es hat allerdings ein wenig gedauert, bis
diese Vorstellung als die Förderliche erkannt wurde.

In der historischen Entwicklung ging ihr die speziellere Vorstellung
voran, dass eine Größe, etwa eine physikalische Größe, in einer
rechnerischen Abhängigkeit von einer anderen Größe steht. So steht
die Periodendauer der Schwingung eines Pendels in einer bestimmten
rechnerischen Abhängigkeit von der Pendellänge.

Eine recht pragmatische Vorstellung von einer Funktion vermittelt das Modell
der Black Box, das soll eine Rechenmaschine sein, deren innere Mechaniken
bzw. Elektroniken unbekannt bleiben. Speist man ein Argument $x$ in die
Black Box ein, spuckt sie daraufhin einen Wert $f(x)$ aus. Speist man
abermals dasselbe Argument ein, spuckt sie abermals denselben Wert aus.

Der Begriff der Abbildung ist für die Mathematik zentral.

\begin{Definition}[Abbildung]\newlinefirst
Eine Abbildung $f\colon X\to Y$ ist eine Relation $f=(X,Y,G)$ mit
$G\subseteq X\times Y$, die die beiden Eigenschaften
\begin{gather*}
\forall x\in X\colon\exists y\in Y\colon (x,y)\in G,\\
\forall x\in X\colon\forall y,y'\in Y\colon (x,y)\in G\land (x,y')\in G\cond y=y'
\end{gather*}
erfüllt. Man nennt $G$ den Graph, $X$ die Definitions- und
$Y$ die Zielmenge.
\end{Definition}
Genau genommen möchte man die Abbildung $f$ eigentlich nicht als mit
dem Tripel $(X,Y,G)$ identisch sehen, sondern als dasjenige Objekt,
dessen innere Information aus diesem Tripel besteht.

Zuweilen wird der Graph von $f$ auch einfach als $f$ bezeichnet.
Sollte dies verwirrend sein, schreibt man ausführlicher $G_f$ oder
$\mathrm{Graph}(f)$.

Statt $(x,y)\in G_f$ schreibt man üblicherweise $y=f(x)$. Es
wird $f(x)$ gelesen als »$f$ von $x$« oder »das Bild von $x$ unter
$f$«. Es wird $f\colon X\to Y$ gelesen als »$f$ ist eine Abbildung
von $X$ nach $Y$«. Besitzt ein $y\in Y$ ein $x\in X$ mit $y=f(x)$,
nennt man $x$ ein Urbildelement zu $y$.

\newpage
\subsection{Bild, Urbild}

Wird ein jedes Element einer Teilmenge der Definitionsmenge in eine
Abbildung geschickt, formen die Werte eine neue Menge, die
\emph{Bildmenge}\index{Bildmenge} der Teilmenge unter der Abbildung.

\begin{Definition}[Bild]\newlinefirst
Die Bildmenge einer Menge $A\subseteq X$ unter der Abbildung
$f\colon X\to Y$ ist
\[f(A) := \{f(x)\mid x\in A\} = \{y\mid\exists x\in A\colon y=f(x)\}.\]
\end{Definition}
Insofern $y=f(x)$ als äquivalent zu $y\in\{f(x)\}$ befunden wird, ergibt
sich auch die Darstellung $f(A) = \bigcup_{x\in A} \{f(x)\}$.
Insbesondere in Programmiersprachen treten endliche Mengen als Objekte
auf. Die Berechnung verläuft mithin gemäß der rekursiven Festlegung%
\[f(\emptyset) := \emptyset,\qquad f(\{x_1,\ldots,x_n\}) :=
f(\{x_1,\ldots,x_{n-1}\})\cup \{f(x_n)\},\]
für die auch die Bezeichnung $\mathrm{map}(f,A)$ gebräuchlich ist.

Die analytische Geometrie sieht Figuren als Punktmengen. Abbildungen
transformieren die Punktmengen, woraus neue Figuren hervorgehen.
Ein einfaches Beispiel bietet die Abbildung%
\[f\colon\R\to\R^2,\quad f(t) :=
\begin{pmatrix}p_x + v_x t\\ p_y + v_y t\end{pmatrix}.\]
In der Hinsicht, dass die reellen Zahlen $\R$ als eine Zahlengerade
aufgefasst werden, formt ihr Bild $f(\R)$ eine Gerade der euklidischen
Ebene. Die passende Wahl der Parameter $(p_x,p_y)$ als Basispunkt
und $(v_x,v_y)$ als Geschwindigkeitsvektor gestattet es, jede
beliebige Gerade der Ebene zu beschreiben. Man nennt in der üblichen
Terminologie aber auch $t$ den Parameter und $f$ eine Parametergerade,
wobei das Bild $f(\R)$ auch Spur genannt wird. Diese Sprechweisen
entspringen der Sichtweise, dass $f(t)$ ein durch die Zeit $t$
parametrisierter Punkt ist, der mit dem Lauf der Zeit eine Spur zieht.
Der Geschwindigkeitsvektor charakterisiert die Bewegung des Punktes,
für die Spur ist dagegen nur dessen Richtung von Bedeutung.

Das \emph{Urbild}\index{Urbild} eines Wertes gibt Auskunft, welche
Elemente der Definitionsmenge unter der Abbildung zu dem Wert führen.
Allgemeiner gibt das Urbild einer Menge von Werten Auskunft, welche
Elemente der Definitionsmenge in die Menge führen.

\begin{Definition}[Urbild]\newlinefirst
Das Urbild einer Menge $B$ unter $f\colon X\to Y$ ist
\[f^{-1}(B) := \{x\in X\mid f(x)\in B\} = \{x\in X\mid \exists y\in B\colon y=f(x)\}.\]
\end{Definition}
Dem Wesen des Abbildungsbegriffs entspringend, zeichnet sich die
Urbildoperation durch Verträglichkeit mit den Mengenoperationen aus,
was bei der Bildoperation nur zum Teil stimmt.

\begin{Satz}
Für jede Abbildung $f$ und beliebige Mengen $A,B,A_i$ gilt
\begin{align*}
f^{-1}(A\cap B) &= f^{-1}(A)\cap f^{-1}(B), &&\textstyle
f^{-1}(\bigcap_{i\in I} A_i) = \bigcap_{i\in I} f^{-1}(A_i),\\
f^{-1}(A\cup B) &= f^{-1}(A)\cup f^{-1}(B), &&\textstyle
f^{-1}(\bigcup_{i\in I} A_i) = \bigcup_{i\in I} f^{-1}(A_i).
\end{align*}
\end{Satz}
\begin{Beweis}
Den Beweis verschafft die äquivalente Umformung
\begin{align*}
&\textstyle x\in f^{-1}(\bigcap_{i\in I} A_i)
\iff f(x)\in\bigcap_{i\in I} A_i
\iff (\forall i\in I\colon f(x)\in A_i)\\
&\textstyle\quad\iff (\forall i\in I\colon x\in f^{-1}(A_i))
\iff x\in\bigcap_{i\in I} f^{-1}(A_i).
\end{align*}
Bei der Vereinigung verläuft die Umformung analog.\,\qedsymbol
\end{Beweis}

\noindent
Der Satz lehrt die wichtige Folgerung, dass die Urbilder zweier
disjunkter Mengen ebenfalls disjunkt sind. Allgemein überführt die
Urbildoperation eine disjunkte Zerlegung einer Menge in disjunkte
Urbilder. Ist die Zielmenge am feinsten zerlegt, besteht sie also
aus einelementigen Mengen, nennt man die Urbilder dieser Mengen
auch Fasern. Man beachte, dass ein Urbild oder eine Faser leer
sein kann.

Ist die Zielmenge bezüglich einer Ordnungsrelation angeordnet,
nennt man die Fasern auch \emph{Niveaumengen}\index{Niveaumenge}.
Diese Begrifflichkeit betrifft vor allem die Funktionen $f\colon X\to\R$
mit $X\subseteq\R^n$. Eine Faser $f^{-1}(\{c\})$, auch als $f(x)=c$
beschrieben, nennt man auch eine \emph{implizite Funktion}. Sie sind
in der analytischen Geometrie und der mehrdimensionalen Analysis von
großer Bedeutung.

\subsection{Komposition}

\begin{Definition}[Komposition]\newlinefirst
Sei $f\colon X\to Y$ und $g\colon Y\to Z$. Ihre Verkettung,
gelesen »$g$ nach $f$«, ist%
\[(g\circ f)\colon X\to Z,\quad (g\circ f)(x):=g(f(x)).\]
\end{Definition}

\noindent
Oft liegt die Situation $f\colon X\to Y$ und $g\colon Y'\to Z$
mit $Y\subseteq Y'$ vor. Das ist aber nicht weiter schlimm. Es
darf dann $g\circ f := g|_Y\circ f$ gesetzt werden, wobei mit $g|_Y$ die
Einschränkung des Definitionsbereichs von $g$ auf $Y$ gemeint ist.

\begin{Definition}[Einschränkung]\newlinefirst
Sei $f\colon X\to Y$ und $A\subseteq X$. Die Einschränkung von
$f$ auf $A$ ist%
\[f|_A\colon A\to Y,\quad f|_A(x):=f(x).\]
\end{Definition}

\noindent
Die Abbildung $\id_X\colon X\to X$ mit $\id_X(x):=x$ heißt
\emph{identische Abbildung}\index{identische Abbildung}. Sie verhält
sich bei der Komposition neutral, das heißt, bezüglich
$f\colon X\to Y$ gilt $f\circ\id_X = f$ und $\id_Y\circ f=f$.

\subsection{Injektionen, Surjektionen, Bijektionen}

\begin{Definition}[Injektion]\newlinefirst
Eine Abbildung $f\colon X\to Y$ heißt injektiv\index{injektiv}, wenn%
\[\forall x,x'\in X\colon f(x)=f(x')\cond x=x'.\]
\end{Definition}

\noindent
Erinnern wir uns an den Abschnitt \emph{Logik mit Gleichheit}, fällt
auf, dass die Definition der Injektivität als Umkehrung der
Ersetzungsregel betrachtbar ist. Das heißt, für eine auf den Werten
der Terme $t,t'$ definierte Injektion $f$ gilt die Äquivalenz%
\[t=t' \iff f(t)=f(t').\]
Die Injektionen vermitteln somit genau die \emph{Äquivalenzumformungen}%
\index{Aequivalenzumformung@Äquivalenzumforumung} von Gleichungen.
Wie sich aus Def. \ref{def:Mengen-Gleichheit} in Verbindung mit
Def. \ref{def:Aussonderung} ergibt, erfährt die Lösungsmenge einer
Bestimmungsgleichung durch sie keine Veränderung. Man notiert%
\[L := \{x\in G\mid t=t'\} = \{x\in G\mid f(t)=f(t')\}\]
für die Lösungsmenge $L$ der Gleichung $t=t'$ in der Variable $x$,
die die Werte der Grundmenge $G$ durchläuft. Aufgrund dessen schaffen
sie ein wesentliches Werkzeug zum Lösen von Bestimmungsgleichungen.

Es muss $X$ bei $f\colon X\to Y$ nicht notwendigerweise die Grundmenge
sein. Wichtig ist allein, dass die Terme $t,t'$ ausschließlich Werte in
$X$ annehmen können. Beispielsweise ist das Quadrieren auf den reellen
Zahlen zwar nicht injektiv, bei Einschränkung der Definitionsmenge auf
die nichtnegativen reellen Zahlen allerdings schon. So initiiert es
die äquivalente Umformung%
\[|x| = |x-2| \bicond x^2 = (x-2)^2 = x^2-4x+4 \bicond
0 = -4x+4 \bicond x = 1.\]
Außerdem darf die Variable der Bestimmungsgleichung in einer
Äquivalenzumformung als Parameter auftauchen. Die injektive Funktion%
\[f_a\colon\R\to\R,\quad f_a(\xi) := \xi + a\]
beschreibt zum Beispiel den Sachverhalt, dass eine Zahl $a$ zu beiden
Seiten einer Gleichung addiert werden darf,%
\[t = t' \iff t + a = t' + a.\]
Hier darf insbesondere auch $a:=x$ als Parameter eingesetzt werden. Bei%
\[f_a\colon\R\to\R,\quad f_a(\xi) := a\xi\]
gilt es allerdings $a\ne 0$ zu berücksichtigen. Das heißt, die Setzung
$a:=x$ liefert nur dann eine Äquivalenzumformung, wenn die Grundmenge,
die $x$ durchläuft, eine Teilmenge von $\R\setminus\{0\}$ ist. Andernfalls
müsste man diesen Umstand durch eine Fallunterscheidung künstlich herstellen.

Erwähnenswert ist weiterhin, dass die gemachten Begriffe bereits
Gleichungen in mehreren Variablen und Gleichungssysteme%
\index{Gleichungssystem} umfassen. Eine Gleichung in zwei Variablen
liegt vor, wenn die Grundmenge aus Paaren besteht. Ein System von zwei
Gleichungen liegt vor, wenn die die Terme $t,t'$ jeweils ein Paar zum
Wert haben. Diese formale Beschreibung müsste man allerdings als ein
wenig oberflächlich befinden, kämen nicht weitere Erörterungen hinzu.
Tiefersinnig wäre es an sich ohne Pointe, tauchte nur eine der Variablen in
der Gleichung auf. Gleichermaßen mag ein System in zwei Variablen
erst reizvoll sein, wenn die Gleichungen in nichttrivialer Weise
miteinander verwoben sind.

% Stellt eine Verbindung zu den Konzepten der Katengorientheorie her.
\begin{Satz}
Eine Abbildung $g$ mit $g\circ f = \id$ heißt Linksinverse von $f$.
Eine Abbildung $f\colon X\to Y$ mit nichtleerem $X$ ist genau dann
injektiv, wenn sie mindestens eine Linksinverse besitzt.
\end{Satz}
\begin{Beweis}
Sei $g$ eine Linksinverse von $f$. Seien $x,x'$ fest, aber beliebig,
und sei $f(x)=f(x')$. Mithin gilt $g(f(x))=g(f(x'))$. Weil $g$ eine
Linksinverse ist, hat man aber $g(f(x))=x$ und $g(f(x'))=x'$, womit
sich wie gewünscht $x=x'$ ergibt.

Sei $f$ injektiv. Mit $X\ne\emptyset$ liegt irgendein $a\in X$ vor.
Sofern $y\in f(X)$ ist, liegt außerdem ein $x$ mit $y=f(x)$ vor.
Es wird $g$ festgelegt per Fallunterscheidung
\[g(y) := \begin{cases}
x, & \text{wenn}\;y\in f(X),\\
a, & \text{wenn}\;y\notin f(X).
\end{cases}\]
Sie ist verhält sich so, dass $g(f(x))=x$ für jedes $x\in X$ gilt,
denn mit $x\in X$ ist $f(x)\in f(X)$. Somit existiert mit $g$ eine
Linksinverse.\,\qedsymbol
\end{Beweis}

\begin{Definition}[Surjektion]\newlinefirst
Eine Abbildung $f\colon X\to Y$ heißt surjektiv\index{surjektiv},
wenn $f(X)=Y$ ist.
\end{Definition}

\noindent
Weil $f(X)\subseteq Y$ allgemeingültig ist, genügt es generell,
$Y\subseteq f(X)$ zu zeigen.

\begin{Definition}[Bijektion]\newlinefirst
Eine Abbildung heißt bijektiv, wenn sie sowohl injektiv als auch
surjektiv ist.
\end{Definition}

\begin{Satz}
Jede Bijektion $f$ besitzt eine eindeutig bestimmte Abbildung, welche
sowohl ihre einzige Linksinverse als auch ihre einzige Rechtsinverse
ist. Man nennt sie die Umkehrabbildung $f^{-1}$.
\end{Satz}
\begin{Beweis}
Sei $f\colon X\to Y$ bijektiv. Es existiert somit mindestens eine
Linksinverse $g$ und mindestens eine Rechtsinverse $h$. Weil die
Verkettung das Assoziativgesetz erfüllt, darf man rechnen
\[g = g\circ\id_Y = g\circ (f\circ h) = (g\circ f)\circ h
= {\id_X}\circ h = h.\]
Sei nun $g'$ eine weitere Linksinverse und $h'$ eine weitere
Rechtsinverse. Wiederholt man die obige Rechnung abermals, erhält man
$g'=h$ und $g'=h'$. Ergo gilt $g = h = g' = h'$.\,\qedsymbol
\end{Beweis}

\subsection{Allgemeines Mengenprodukt}

Nachdem nun der Begriff der Abbildung bereits erklärt wurde, kann eine
Begriffsverallgemeinerung des kartesischen Produktes erörtert werden,
die Tupel mit Folgen und Funktionen in Beziehung setzt. Eine Folge
darf man einerseits als Funktion betrachten, deren Definitionsbereich
die natürlichen Zahlen sind. Andererseits ist eine Folge indizierbar wie
ein Tupel. Das bringt uns auf die Idee, Tupel wie Folgen als Funktionen
aufzufassen.

Die Ziffern der Zahl 2520 sind in Little"=Endian"=Konvention das Tupel
\[t=(0,2,5,2)\in\N\times\N\times\N\times\N.\]
Die Indexmenge $I:=\{0,1,2,3\}$ enthält die Stellenwerte der Ziffern.
Es verhält sich nun so, dass die Funktion
\[f\colon I\to\N,\quad f(0) := 0,\; f(1) := 2,\; f(2) := 5,\; f(3) := 2\]
genau die Information enthält, die das Tupel charakterisiert. Es ist
$f(i) = t_i$ die Indizierung des Tupels. So gesehen ist jedes Tupel durch
eine Funktion kodiert. Führt man diese Überlegung fort, gelangt man
schließlich zum allgemeinen kartesischen Produkt
\[\prod_{i\in I} X_i := \{f\colon I\to\bigcup_{i\in I} X_i\mid
\forall i\in I\colon f(i)\in X_i\}.\]
Der Formelsalat sagt im Wesentlichen nur, dass eine Abbildung
$f\in\prod_{i\in I} X_i$ einen in $X_i$ liegenden Funktionswert
$f(i)$ besitzt.

Diese allgemeine Produkte von Mengen treten in der Mathematik nur sehr
sporadisch in Erscheinung. Eine Rolle spielen sie in der Theorie der
Kardinalzahlen. In der abhängigen Typentheorie besitzen sie allerdings ein
direktes allgegenwärtiges Analogon, den \emph{Typ abhängiger Funktionen}.

Bei einer Gleichsetzung wie $\prod_{i\in\{1,2\}} X_i\cong X_1\times X_2$ ist
zu beachten, dass es sich eigentlich nicht um eine Gleichung handelt, denn
Abbildungen sind ja nicht dasselbe wie Paare. Allerdings gehört zu jedem
Paar wie gesagt in kanonischer Weise genau eine gleichartige Abbildung.
Es besteht eine Isomorphie, was durch die Tilde über dem Gleichheitszeichen
angedeutet wird. Damit verbunden liegt ein Isomorphismus vor, mit dem
sich ein Objekt der einen Seite in ein Objekt der anderen Seite
überführen lässt. Das ist so ähnlich wie die Übersetzung eines Textes
von der einen in die andere Sprache. Die Gestalt ändert sich, aber der
Inhalt bleibt gleich.

Intuitiv sollte $\prod_{i\in I} X_i$ nichtleer sein, sofern jedes $X_i$
nichtleer ist. Dennoch lässt sich dieser Sachverhalt nicht ohne Weiteres
ableiten. Es handelt sich um ein Axiom der Mengenlehre, das sogenannte
\emph{Auswahlaxiom}\index{Auswahlaxiom}. Es besagt, dass zu jeder
Familie $(X_i)$ nichtleerer Mengen eine Auswahlfunktion $f$ existiert,
die zu jeder Menge $X_i$ ein Element $f(i)\in X_i$ auswählt.
Eine gründliche Untersuchung des Auswahlaxioms zeigt auf, dass es sich
mit den Unendlichkeiten schwierig verhält. Unter bestimmten Umständen
ist das Axiom nicht vonnöten. Außerdem, so lässt sich zeigen, impliziert
es den Satz vom ausgeschlossenen Dritten.

\section{Relationen}

\subsection{Relationen im Allgemeinen}

Beim Erlangen von tieferliegenden Einsichten in Probleme und
Zusammenhänge spielt das Aufspüren und Klären unterschiedlicher
Beziehungen eine wesentliche Rolle. Zwei Objekte können auf
unterschiedliche Art und Weise in Beziehung stehen. In der Mathematik
beschäftigt man sich mit Beziehungen, die zu den zwei Objekten eine
Aussage trifft, der ein Wahrheitsgehalt beigemessen wird. Zwei Zahlen
$x,y$ sind gleich. Eine Zahl $x$ ist kleiner als eine Zahl $y$. Der
Betrag der Differenz zweier Zahlen $x,y$ ist kleiner als eine bstimmte
Konstante. Der Abstand der Punkte $x=(x_1,x_2)$ und $y=(y_1,y_2)$ ist
kleiner als eine bestimmte Konstante. Zwei Stühle $x,y$ eines
Stuhlkreises sind benachbart. Die Schüler $x,y$ gehen in dieselbe
Klasse. Es gibt einen Weg, der vom Ort $x$ zum Ort $y$ führt.

Die eindrückliche Vielfalt möglicher Beziehungen macht es ja
unerlässlich, eine allgemeine Auffassung von ihnen zu erhalten.
Eine zweistellige Relation $R$ schafft ein Beziehungsgefüge zwischen
zwei Mengen $X,Y$. Wir sagen, ein Objekt $x\in X$ steht bezüglich
$R$ zu einem Objekt $y\in Y$ in Beziehung, wenn die Aussage $R(x,y)$
eine wahre ist. Ein Beispiel wäre $R(x,y):=(x<y)$, wobei $X=Y$ die Menge
der ganzen Zahlen sei.

\begin{Definition}[Zweistellige Relation]\newlinefirst
Es seien $X,Y$ zwei Mengen. Jede Teilmenge $R\subseteq X\times Y$ heißt
Relation zwischen $X$ und $Y$. Zur Relation soll die Kenntnis von
$X,Y$ dazugehören. Insofern ist das durch das Tripel $(X,Y,R)$ kodierte
Objekt die Relation und $R$ ihr Graph.
\end{Definition}

\noindent
Eine Relation ist auch interpretierbar als wahrheitswertige Funktion
\[1_R\colon X\times Y\to\{0,1\},\quad 1_R(x,y):=[(x,y)\in R].\]
Sie ist die Indikatorfunktion der Teilmenge $R$ bezüglich der
Grundmenge $X\times Y$. Anstelle $(x,y)\in R$ oder $1_R(x,y)=1$
schreiben wir schlicht $R(x,y)$ für die Aussage, dass $x$ und $y$
bezüglich $R$ in Relation stehen.

Eine Relation mit $X\ne Y$ heißt \emph{heterogen}, eine mit
$X=Y$ heißt \emph{homogen}. Wir werden uns fast ausschließlich mit
homogenen Relationen beschäftigen. Man darf allerdings sagen, dass
auch die heterogenen in der Mathematik weit verbreitet sind, denn
jede Abbildung ist betrachtbar als Relation mit speziellen
Eigenschaften. Genauer ist eine Abbildung eine linkstotale und
rechtseindeutige Relation. Linkstotal heißt, dass jedes $x\in X$
zu mindestens einem $y\in Y$ in Beziehung steht. Rechtseindeutig heißt,
dass ein $y\in Y$ höchstens zu einem $x\in X$ in Beziehung steht. Es
handelt sich also lediglich um eine Sprechweise für die bereits
bekannten definierenden Eigenschaften.

Relationen lassen sich in Pfeildiagrammen darstellen. Man man zweichnet
hierbei einem Pfeil von einem $x\in X$ zu einem $y\in Y$, falls $x,y$
in Relation stehen.

\subsection{Äquivalenzrelationen}

Manchmal interessiert man sich nicht für die gänzliche Gleichheit
zweier Objekte. Die \emph{Äquivalenzrelationen}%
\index{Aequivalenzrelation@Äquivalenzrelation} verallgemeinern den
Gleichheitsbegriff dahingehend, dass zwei Objekte schon dann als
\emph{gleichartig}\index{gleichartig} angesehen werden, wenn sie in
in einer bestimmten Eigenschaft übereinstimmen. Um welche Eigenschaft
es sich dabei handelt, bestimmt die Relation.

\begin{Definition}[Äquivalenzrelation]\newlinefirst
Seien $A$ eine Menge und seien $x,y,z\in A$. Sei $R(x,y):=(x\sim y)$ eine
Relation. Man nennt $R$ Äquivalenzrelation, wenn gilt:
\[\begin{array}{ll}
x\sim x, &\text{(Reflexivität)}\\
x\sim y \implies y\sim x, & \text{(Symmetrie)}\\
x\sim y\land y\sim z\implies x\sim z.\quad & \text{(Transitivität)}
\end{array}\]
\end{Definition}

\begin{Definition}[Äquivalenzklasse]\newlinefirst
Sei $M$ eine Menge und $x\sim y$ eine Äquivalenzrelation für $x,y\in M$.
Die Menge%
\[[a] := \{x\in M\mid x\sim a\}\]
nennt man die Äquivalenzklasse zum Repräsentanten $a\in M$.
\end{Definition}

\begin{Satz}[Äquivalenzrelation induziert Zerlegung]\newlinefirst
Eine Menge wird durch eine Äquivalenzrelation in disjunkte
Äquivalenzklassen zerlegt, lat. partitioniert.
\end{Satz}
\strong{Beweis.} Sei $M$ die Menge und $x\sim y$ die Äquivalenzrelation.
Zu zeigen ist, dass kein Element von $M$ in mehr als einer
Äquivalenzklasse vorkommt. Seien $a,b,c\in M$, sei $c\in [a]$
und $c\in [b]$. Aufgrund von $c\sim a$ sowie $c\sim b$ und der
Transitivität gilt%
\[x\in [a]\iff x\sim a\iff x\sim c\iff x\sim b\iff x\in [b].\]
Man hat also
\[(\forall x\in M\colon x\in [a]\Leftrightarrow x\in [b])\iff [a]=[b].\]
Wenn also $[a]\ne [b]$ ist, kann nicht gleichzeitig $c\in [a]$ und $c\in [b]$
sein.\;\qedsymbol

\begin{Satz}[Zerlegung induziert Äquivalenzrelation]\newlinefirst
Sei $M$ eine Menge. Die Familie $(A_k)$ von Mengen $A_k\subseteq M$
bilde eine Zerlegung von $M$, d.\,h. dass die Vereinigung aller
$A_k$ die Menge $M$ überdeckt und dass paarweise $A_i\cap A_j=\emptyset$
für $i\ne j$ ist. Dann ist%
\[x\sim y\defiff \exists k\colon x\in A_k\land y\in A_k\]
eine Äquivalenzrelation auf $M$.
\end{Satz}
\strong{Beweis.} Da die $A_k$ die Menge $M$ überdecken,
muss es für ein beliebiges $x\in M$ mindestens eine Menge $A_k$
geben, so dass $x\in A_k$. Daher gilt $x\sim x$.

Die Symmetrie ergibt sich trivial.

Zur Transitivität. Voraussetzung ist $x\sim y$ und $y\sim z$.
Es gibt also ein $i$ mit $x\in A_i$ und $y\in A_i$. Außerdem gibt
es ein $j$ mit $y\in A_j$ und $z\in A_j$. Somit gilt%
\[\exists i\colon\exists j\colon x\in A_i\land y\in A_i\land y\in A_j\land z\in A_j.\]
Wegen
\[A_i\cap A_j = \emptyset \iff \forall y\colon (y\in A_i\land y\in A_j\Leftrightarrow 0)\]
für $i\ne j$ kann $y\in A_i\land y\in A_j$ aber nur erfüllt sein,
wenn $i=j$ ist. Daher ergibt sich%
\[\exists i\colon x\in A_i\land z\in A_i,\]
das heißt $x\sim z$.\;\qedsymbol

\begin{Definition}[Quotientenmenge]%
\index{Quotientenmenge}\index{Faktormenge}\newlinefirst
Für eine gegebene Äquivalenzrelation wird die aus allen
Äquivalenzklassen bestehende Menge
\[M/{\sim} := \{[x]\mid x\in M\}\]
als Quotientenmenge oder Faktormenge bezeichnet.
\end{Definition}

\begin{Definition}[Quotientenabbildung]%
\index{Quotientenabbildung}\newlinefirst
Für eine gegebene Äquivalenzrelation ist die Projektion
\[\pi\colon M\to M/{\sim},\quad \pi(x):=[x]\]
surjektiv und wird Quotientenabbildung genannt.
\end{Definition}

\begin{Definition}[Repräsentantensystem]%
\index{Repräsentantensystem}\newlinefirst
Für eine gegebene Äquivalenzrelation auf $M$ nennt man eine
Teilemenge $A\subseteq M$ ein vollständiges Repräsentantensystem,
wenn die Einschränkung $\pi|_A$ bijektiv ist, wobei mit $\pi$
die Quotientenabbildung gemeint ist.
\end{Definition}
Repräsentantensysteme ermöglichen die einfache Handhabung von
Äquivalenzklassen. Möchte man wissen, ob ein Element $x$ in der
Äquivalenklasse $[a]$ enthalten ist, dann braucht man bloß
zu überprüfen, ob $x\sim a$ ist. Außerdem besitzt die
Quotientenabbildung nun eine Darstellung $p\colon M\to A$,
dergestalt dass $\pi = \pi|_{A}\circ p$. Warum sollte das von
Bedeutung sein? Nun, Äquivalenzklassen fallen oft unendlich groß
aus. In der Kombinatorik treten zwar auch endliche Äquivalenzklassen auf,
diese werden trotzdem schnell unzugänglich groß. Die Äquivalenzklassen
und die Quotientenabbildung muss man also als abstrakte mathematische
Objekte betrachten. Abstrakte mathematische Objekte müssen wir erst
über eine Darstellung zugänglich machen, und genau dies ermöglicht
ein Repräsentantensystem.


\begin{Satz}[Charakterisierung von Äquivalenzklassen]\newlinefirst
Sei auf der Menge $M$ eine Äquivalenzrelation gegeben. Eine
Teilmenge $A\subseteq M$ ist genau dann eine Äquivalenzklasse,
wenn%
\begin{align*}
1.\;& A\ne\emptyset,\\
2.\;& x,y\in A\implies x\sim y,\\
3.\;& x\in A\land y\in M\land x\sim y\implies y\in A.
\end{align*}
\end{Satz}
\strong{Beweis.} Angenommen, $A$ ist eine Äquivalenzklasse.
Dann gibt es definitionsgemäß ein $a$ mit $A=[a]$. Daher ist
mindestens $a\in A$ und somit $A\ne\emptyset$. Mit $x,y\in A$ ergibt
sich $A=[x]=[y]$. Aufgrund von%
\[x\sim y \iff [a]=[b]\]
muss somit $x\sim y$ sein. Sei nun $x\in A$ und $y\in M$ mit
$x\sim y$. Es folgt $A=[x]=[y]$. Daher muss $y\in A$ sein.

Umgekehrt angenommen, die drei Eigenschaften sind erfüllt.
Zu zeigen ist, dass es ein $a$ gibt mit $A=[a]$. Da $A$ gemäß 1.
nichtleer ist, enthält es mindestens ein Element, dieses nennen wir
$a$. Für jedes weitere Element $x\in A$ ergibt sich
$x\sim a$, da sonst 2. verletzt sein würde. Schließlich muss man
noch wissen, ob $x\in A$, wenn $x\sim a$ und $x\in M$ ist.
Dies ist aber mit 3. gesichert. Es gibt also
tatsächlich ein $a$ mit $A=\{x\in M\mid x\sim a\}$.\;\qedsymbol

Eine große Fülle von Äquivalenzrelationen lässt sich auf die folgende
einfache Art konstruieren. Hat man eine beliebige Abbildung
$f\colon X\to Y$, dann sind die Urbilder $f^{-1}(\{y_1\})$ und
$f^{-1}(\{y_2\})$ disjunkt, sofern $y_1\ne y_2$, denn%
\[f^{-1}(\{y_1\})\cap f^{-1}(\{y_2\}) = f^{-1}(\{y_1\}\cap\{y_2\})
= f^{-1}(\emptyset) = \emptyset.\]
Demnach ist gemäß
\[Z = X/{\sim} = \{f^{-1}(\{y\})\mid y\in f(X)\}\]
eine Zerlegung des Definitionsbereichs $X$ gegeben und somit auch eine
Äquivalenzrelation. Für $x_1,x_2\in X$ gilt%
\[x_1\sim x_2 \iff f(x_1) = f(x_2).\]
Ist $f$ zudem surjektiv, dann gehört zu jedem Element von $Y$ genau eine
Äquivalenzklasse. Demnach definiert $f$ dann eine verallgemeinerte
Quotientenabbildung, da die Elemente von $Y$ die Äquivalenzklassen
charakterisieren. Die Bijektion $\varphi\colon Z\to Y$ hat dabei
die Eigenschaft $f = \varphi\circ\pi$. Sofern $Y$ für uns zugänglich
ist, resultiert hieraus auch eine verallgemeinerte Darstellung der
Quotientenabbildung, denn
\[f = \varphi\circ\pi = \varphi\circ (\pi|_A\circ p)
= (\varphi\circ\pi|_A)\circ p.\]
Nun ist $\varphi\circ\pi|_A$ auch bijektiv, weil $\varphi$
und $\pi|_A$ es sind. Somit charakterisiert $Y$ ein
vollständiges Repräsentantensystem.

Was bisher erläutert wurde, mag recht abstrakt erscheinen. Wir haben
aber eigentlich ein recht intuitives Verständnis für diese
Begrifflichkeiten. Ein Bilderbuchbeispiel für eine Quotientenmenge
bieten die Klassen einer Schule. Zwei Schüler seien genau dann
äquivalent, wenn sie in dieselbe Klasse gehen. Die Äquivalenzklasse eines
Schülers ist dann schlicht seine Schulklasse. Die Menge der Schüler der
Schule wird in disjunkte Schulklassen zerlegt. Die Menge dieser
Schulklassen bildet die Quotientenmenge. Ein vollständiges
Repräsentantensystem entsteht zum Beispiel durch die Wahl eines
Klassensprechers in jeder Klasse.

Ein weiteres typisches Beispiel für eine Äquivalenzrelation ist die
Kongruenz modulo $m$, die elementar in der Zahlentheorie
und Gruppentheorie vorkommt. Die Äquivalenzklassen sind hier die
Restklassen. Die Reste bilden ein kanonisches vollständiges
Repräsentantensystem. Das Bilden des Restes zu einer Zahl ist
eine Darstellung der Quotientenabbildung.

\subsection{Operationen auf Äquivalenzklassen}

Äquivalenzklassen werden später wichtig sein für die Formulierung von
Konstruktionen. Bei diesen Konstruktionen ist eine Abbildung
zwischen Quotientenmengen erforderlich. Weil die Äquivalenzklassen
dabei über Repräsentanten dargestellt sind, liegt es nahe, auch
die Abbildung über Repräsentanten zu definieren. Dies wirft die
Frage nach der \emph{Wohldefiniertheit}\index{wohldefiniert} auf.
Darunter versteht man, dass die Abbildung auch tatsächlich unabhängig
von den gewählten Repräsentanten ist. Was das genau bedeutet, wird im
folgenden Abschnitt erklärt.

Gegeben seien zwei Quotientenmengen $M/\sim$ und $M'/\sim'$.
Eine vorhandene Abbildung $f\colon M\to M'$ induziert dann
eventuell gemäß%
\[f\colon M/\sim\to M'/\sim',\quad f([a]):=[f(a)]\]
eine Abbildung zwischen den Quotientenmengen. Kommt es dabei nicht
zu einem Widerspruch, liegt also eine Abbildung vor, spricht man
von \emph{Wohldefiniertheit}. Hierfür darf der Funktionswert nicht
vom gewählten Repräsentant abhängen, d.\,h. die Bedingung%
\[\forall x\in [a]\colon f(x)\in [f(a)]\]
muss erfüllt sein. Anders formuliert:
\[x\sim a \implies f(x)\sim' f(a).\]
Für mehrstellige Abbildungen ist das Vorgehen analog. Eine
Abbildung $f\colon M^2\to M'$ induziert%
\[f\colon (M/\sim)^2\to M'/\sim',\quad f([a],[b]):=[f(a,b)],\]
sofern
\[x\sim a\land y\sim b\implies f(x,y)\sim' f(a,b).\]
Bei den Konstruktionen kommen in der Regel zweistellige Abbildungen
(mit $M=M'$) vor, weil die Verknüpfungen von Elementen der algebraischen
Strukturen zweistellig sind. Diese Verknüpfungen werden im nächsten
Abschnitt besprochen.

\subsection{Kongruenzrelationen}

\begin{Definition}[Kongruenzrelation]%
\index{Kongruenzrelation}\newlinefirst
Gegeben sei eine Menge $M$, auf der eine zweistellige
Verknüpfung $*\colon M^2\to M$ definiert ist. Eine Äquivalenzrelation
auf $M$ nennt man Kongruenzrelation, wenn die induzierte Verknüpfung
$[a]*[b]:=[a*b]$ wohldefiniert ist.
\end{Definition}
Bei einer Kongruenzrelation sagt man »$a$ ist kongruent zu $b$«
anstelle von »$a$ ist äquivalent zu $b$« und schreibt $a\equiv b$
anstelle von $a\sim b$. Eigentlich kann man den Begriff für eine
beliebige Stelligkeit definieren. Es besteht jedoch zunächst nur Bedarf
an zweistelligen Verknüpfungen.

Im Folgenden schreiben wir für die Verknüpfung kurz $ab$ anstelle
$a*b$. Das spart ein wenig Schreibaufwand und ist so üblich, solange
keine Verwechslungsgefahr mit einer bereits auf andere Art definierten
Multiplikation besteht.

\begin{Satz}\label{Kongruenz-Quotient}
Sei $M$ eine Struktur aus der Liste Magma, Monoid, Halbgruppe,
Gruppe, kommutatives Monoid, kommutative Gruppe. Sei $\equiv$ eine
Kongruenzrelation auf $M$ und $\varphi$ die zugehörige
Quotientenabbildung. Dann bildet die Quotienmenge $M/\equiv$
bezüglich der induzierten Verknüpfung $\varphi(a)\varphi(b):=\varphi(ab)$
ebenfalls eine Struktur derselben Art und $\varphi$ ist ein
Homomorphismus.
\end{Satz}
\strong{Beweis.} Im Folgenden seien $a',b',c'$ beliebige Elemente
der Quotientenmenge. Weil $\varphi$ surjektiv ist, gibt es immer
$a,b,c\in M$ mit $a'=\varphi(a)$, $b'=\varphi(b)$ und $c'=\varphi(c)$.

Die Verknüpfung auf $M$ sei abgeschlossen. Dann ist
\[a'b' = \varphi(a)\varphi(b) = \varphi(ab)\in M/\equiv.\]
Somit ist die Quotientenmenge bezüglich der induzierten Verknüpfung
abgeschlossen.

Die Verknüpfung auf $M$ erfülle das Assoziativgesetz. Dann gilt
\[(a'b')c' = \varphi(ab)\varphi(c) = \varphi(abc)
= \varphi(a)\varphi(bc) = a'(b'c').\]
Die induzierte Verknüpfung erfüllt somit ebenfalls das
Assoziativgesetz.

Die Verknüpfung auf $M$ habe ein neutrales Element $e$. Dann gilt
\[\varphi(a)=\varphi(ea) = \varphi(e)\varphi(a),\quad
\varphi(a)=\varphi(ae)=\varphi(a)\varphi(e).\]
Demzufolge besitzt $M/\equiv$ mit $e':=\varphi(e)$ ebenfalls
ein neutrales Element.

Zur Verknüpfung auf $M$ gebe es zu jedem Element ein inverses. Dann gilt
\[\varphi(e) = \varphi(aa^{-1}) = \varphi(a)\varphi(a^{-1}),\quad
\varphi(e) = \varphi(a^{-1}a) = \varphi(a^{-1})\varphi(a).\]
Demzufolge gibt es auf der Quotientenstruktur mit
$\varphi(a)^{-1}:=\varphi(a^{-1})$ ebenfalls zu jedem Element
ein inverses.

Die Verknüpfung auf $M$ sei kommutativ. Dann gilt
\[a'b' = \varphi(a)\varphi(b) = \varphi(ab) = \varphi(ba)
= \varphi(b)\varphi(a) = b'a'.\]
Somit ist die Verknüpfung auf der Quotientenstruktur $M/\equiv$ ebenfalls
kommutativ.\,\qedsymbol

\begin{Satz}\label{Kongruenz-Ring-Quotient}
Satz \ref{Kongruenz-Quotient} gilt auch für Ringe, unitäre Ringe,
kommutative Ringe und kommutative unitäre Ringe, sofern die
Relation eine Kongruenzrelation sowohl bezüglich der additiven
als auch der multiplikativen Verknüpfung ist.
\end{Satz}
\strong{Beweis.} Sei $(R,+,\cdot)$ der Ring und $\equiv$ die
Kongruenzrelation. Gemäß Satz \ref{Kongruenz-Quotient} ist
$(R/\equiv,+)$ eine kommutative Gruppe und $(R/\equiv,\cdot)$
eine Halbgruppe. Bei einem unitären Ring ist $(R/\equiv,\cdot)$
ein Monoid, und bei einem kommutativen unitären Ring ein
kommutatives Monoid.

Es verbleiben noch die Distributivgesetze zu prüfen.
Sei $\varphi$ die Quotientenabbildung. Man rechnet
\begin{align*}
a'(b'+c') &= \varphi(a)(\varphi(b)+\varphi(c)) = \varphi(a)(\varphi(b+c))
= \varphi(a(b+c)) = \varphi(ab+ac)\\
&= \varphi(ab)+\varphi(ac) = \varphi(a)\varphi(b)+\varphi(a)\varphi(c)
= a'b'+a'c'.
\end{align*}
Die Rechnung zum Rechtsdistributivgesetz ist analog.

Damit ist der Satz gezeigt, und ferner ist gezeigt dass $\varphi$ ein
Ringhomomorphismus ist. Und für einen unitären Ring ist $\varphi$
Eins-erhaltend, wie bereits aus Satz \ref{Kongruenz-Quotient}
hervorgeht.\,\qedsymbol

% Ordinalzahlen. Eine Dominoreihe verläuft sich in einer Schachtel
% ins Unendliche. Der Stein mit Limesordinal muss extra umfallen,
% zu ihm gibt es keinen Vorgänger.





\chapter{Algebra}
\section{Gruppentheorie}
\subsection{Grundbegriffe}
\begin{definition}[Gruppenhomomorphismus]
Sind $(G,*)$ und $(H,\bullet)$ zwei Gruppen, so
heißt $\varphi\colon G\to H$ \emdef{Gruppenhomomorphismus}%
\index{Gruppenhomomorphismus}, wenn
\begin{gather}
\forall g_1,g_2\in G\colon
  \varphi(g_1*g_2) = \varphi(g_1)\bullet\varphi(g_2)
\end{gather}
gilt. Ein \emdef{Gruppenisomorphismus}\index{Isomorphismus!zwischen Gruppen}
ist ein bijektiver Gruppenhomomorphismus, da die Umkehrabbildung
auch wieder ein Gruppenhomomorphismus ist.
\end{definition}
\begin{definition}[Direktes Produkt]
\emdef{Direktes Produkt}\index{direktes Produkt}:
\begin{gather}
G\times H := \{(g,h)\mid g\in G, h\in H\},\\
(g_1,h_1)*(g_2,h_2) := (g_1*g_2, h_1*h_2).
\end{gather}
\end{definition}
\noindent
\strong{Satz von Lagrange.} Für Gruppen $G,H$ gilt:
\begin{equation}
H\le G\implies |G| = |G/H|\cdot |H|.
\end{equation}

\subsection{Gruppenaktionen}\label{Gruppenaktion}
\begin{definition}[Gruppenaktion]
Eine Funktion $f\colon G\times X\to X$ heißt
\emdef{Gruppenaktion}\index{Gruppenaktion}, wenn
\begin{gather}
\hspace{-1em}\forall g_1,g_2{\in}G, x{\in}X\colon f(g_1,f(g_2,x)) = f(g_1 g_2,x),\\
\hspace{-1em}\forall x\in X\colon f(e,x) = x
\end{gather}
gilt, wobei mit $e$ das neutrale Element von $G$ gemeint ist.
Anstelle von $f(g,x)$ wird üblicherweise kurz $gx$ (oder
$g+x$ bei einer Gruppe $(G,+)$) geschrieben.

Anstelle von \emdef{Linksaktionen} kommen auch \emdef{Rechtsaktionen}
vor, die sich von Linksaktionen in der Reihenfolge unterscheiden.
Eine Rechtsaktion $f\colon X\times G\to X$ genügt den Regeln
\begin{gather}
\hspace{-1em}\forall g_1,g_2{\in}G, x{\in}X\colon f(f(x,g_1),g_2) = f(x,g_1 g_2),\\
\hspace{-1em}\forall x\in X\colon f(x,e) = x.
\end{gather}
\end{definition}

\begin{definition}[Orbit, Stabilisator]
Für ein $x\in X$ wird
\begin{equation}\label{eq:Orbit}
Gx := \{gx\mid g\in G\}
\end{equation}
\emdef{Bahn}\index{Bahn} oder
\emdef{Orbit}\index{Orbit!unter einer Gruppenaktion} genannt.
Die Menge
\begin{equation}
G_x := \{g\in G\mid gx=x\}
\end{equation}
wird \emdef{Fixgruppe}\index{Fixgruppe}
oder \emdef{Stabilisator}\index{Stabilisator} genannt.
Die Menge
\begin{equation}
X^g := \{x\in X\mid gx=x\}
\end{equation}
heißt \emdef{Fixpunktmenge}.
\end{definition}

\noindent
Fixgruppen sind immer Untergruppen:
\begin{equation}
\forall x\colon G_x\le G.
\end{equation}
Bahnen sind Äquivalenzklassen, die Quotientenmenge
\begin{equation}
X/G := \{Gx\mid x\in X\}
\end{equation}
wird \emdef{Bahnenraum}\index{Bahnenraum} genannt.

\strong{Bahnformel.}\index{Bahnformel}
Ist $G$ eine endliche Gruppe, so gilt:
\begin{equation}
|G| = |Gx|\cdot |G_x|.
\end{equation}
\strong{Lemma von Burnside.}\index{Lemma von Burnside}
Für eine endliche Gruppe $G$ gilt:%
\begin{equation}
|X/G| = \frac{1}{|G|}\sum_{g\in G}|X^g|.
\end{equation}

\section{Ringe}\index{Ring}
Ist $(R,+,*)$ ein Ring, so gilt für alle $a,b\in R$:
\begin{align}
0*a &= a*0 = 0,\\
(-a)*b &= a*(-b) = -(a*b),\\
(-a)*(-b) &= -(a*b).
\end{align}
\begin{definition}[Ringhomomorphismus]
Sind $(R,+,*)$ und $(R',+',*')$ Ringe, so wird
$\varphi\colon R\to R'$ als \emdef{Ringhomomorphismus}
bezeichnet, wenn
\begin{align}
\varphi(a+b) &= \varphi(a)+'\varphi(b),\\
\varphi(a*b) &= \varphi(a)*'\varphi(b),
\end{align}
für alle $a,b\in R$ gilt und $\varphi(1)=1$ ist.
\end{definition}

\subsection{Polynome}\index{Polynom}
\begin{definition}[Polynom, Polynomring, Koeffizienten]
Sei $R$ ein kommutativer unitärer Ring.
Mit $R[X]$ bezeichnen wir die Menge der unendlichen Folgen
\begin{equation}
(a_k) = (a_0,a_1,\ldots,a_n,0,0,0,\ldots)
\end{equation}
mit $a_k\in R$, bei denen ab einem Index alle Folgenglieder null sind.

Für zwei Folgen aus $R[X]$ wird nun die Addition
\begin{equation}
(a_k) + (b_k) := (a_k+b_k)
\end{equation}
und die Multiplikation
\begin{equation}\label{eq:Faltung}
(a_i)*(b_j) = \bigg(\sum_{i=0}^k a_i b_{k-i}\bigg)
\end{equation}
erklärt. In der Form \eqref{eq:Faltung} wird die Operation auch
\emdef{Faltung}\index{Faltung!von zwei Folgen}
der Folgen $(a_i)$ und $(b_j)$ genannt.

Die Menge $R[X]$ bildet mit der Addition und Multiplikation
einen kommutativen unitären Ring, den \emdef{Polynomring}
mit Koeffizienten in $R$. Ein Element von $R[X]$ wird
\emdef{Polynom} genannt.

Man definiert nun
\begin{equation}
X:=(0,1,0,0,0,\ldots),
\end{equation}
womit sich jedes Polynom in der Form
\begin{equation}\textstyle
(a_k) = \sum_{k=0}^n a_k X^k
\end{equation}
schreiben lässt. Die $a_k$ nennt man \emdef{Koeffizienten}
des Polynoms.
\end{definition}

\noindent
Die Addition bekommt nun die Form
\begin{equation}
\sum_{k=0}^m a_k X^k + \sum_{k=0}^n b_k X^k
:= \sum_{k=0}^p (a_k+b_k)X^k.
\end{equation}
mit $p=\max(m,n)$. Die Multiplikation lässt sich nun in der Form
\begin{equation}
\bigg(\sum_{i=0}^m a_i X^i\bigg)\bigg(\sum_{j=0}^n b_j X^j\bigg)
:= \sum_{k=0}^{m+n}\bigg(\sum_{i=0}^k a_i b_{k-i}\bigg) X^k.
\end{equation}
schreiben. Die Multiplikation von Polynomen ist das gewöhnlichen
Ausmultiplizieren der Polynome, wobei $X^i X^j=X^{i+j}$ gilt.

Die $X^k$ können als Vektorraumbasis betrachtet
werden und dienen dabei dazu, die $a_k$ auseinanderzuhalten.
Zwei Polynome $\sum_{k=0}^m a_k X^k$ und $\sum_{k=0}^n b_k X^k$
sind genau dann gleich, wenn $a_k=b_k$ für alle $k\le\max(m,n)$ gilt.

Da $R[X]$ wieder ein kommutativer unitärer Ring ist,
ist auch $R[X][Y]$ ein Polynomring. Man definiert
\begin{equation}
R[X,Y] := R[X][Y].
\end{equation}
Polynome aus $R[X,Y]$ lassen sich in der Form
\begin{equation}
\sum_{j=0}^n \bigg(\sum_{i=0}^m a_{ij}X^i\bigg)Y^j
= \sum_{i=0}^m\sum_{j=0}^n a_{ij} X^i Y^j
\end{equation}
mit $a_{ij}\in R$ schreiben.

Allgemein ist die Menge
\begin{equation}
R[X_1,\ldots,X_q] := X[X_1,\ldots,X_{q-1}][X_q]
\end{equation}
ein kommutativer unitärer Ring. Die Polynome lassen sich in der Form
\begin{equation}
\sum_{k\in\N_0^q} a_k X^k\quad (a_k\in R)
\end{equation}
mit
\[k=(k_1,\ldots,k_q)\quad\text{und}\quad X^k:=\prod_{i=1}^q X_i^{k_i}\]
schreiben.

\begin{definition}[Grad]
Für ein Polynom $f=\sum_{k=0}^n a_k X^k$ mit $a_n\ne 0$ wird
$n$ als \emdef{Grad} von $f$ bezeichnet. Man schreibt $n=\deg f$.

Für ein Monom $a_{ij} X^i Y^j$ mit $a_{ij}\ne 0$ heißt $i+j$
\emdef{Totalgrad}. Der \emdef {Grad} eines Polynoms
\begin{equation}
\textstyle\sum_{i=1}^m\sum_{j=1}^n a_{ij} X^i Y^j
\end{equation}
ist der maximale Totalgrad aller Monome mit $a_{ij}\ne 0$.
Für Polynome in mehr als zwei Variablen ist die Definition analog.
\end{definition}

\strong{Regeln.}\\
Für zwei Polynome $f,g\in R[X_1,\ldots,X_q]$ gilt:
\begin{align}
\deg(f+g)&\le \max(\deg f,\deg g),\\
\deg(fg)&\le (\deg f)(\deg g).
\end{align}
Für zwei Polynome $f,g$ mit $\deg f\ne\deg g$ gilt:
\begin{equation}
\deg(f+g) = \max(\deg f,\deg g).
\end{equation}
Ist $R$ ein Integritätsring, so gilt für $f,g\in R[X_1,\ldots,X_q]$:%
\begin{equation}
\deg(fg) = (\deg f)(\deg g).
\end{equation}
Jeder Körper, z.\,B. $\R$ oder $\C$ ist ein Integritätsring.
Auch die ganzen Zahlen $\Z$ bilden einen Integritätsring.
Ein Polynomring ist genau dann ein Integritätsring, wenn die
Koeffizienten aus einem Integritätsring entstammen.

\begin{definition}[Einsetzungshomomorphismus]
Seien $R,R'$ kommutative unitäre Ringe. Sei $R'$ eine Ringerweiterung
von $R$ und sei $r\in R'$. Die Abbildung $\varphi_r\colon R[X]\to R'$
mit
\begin{equation}\textstyle
\varphi_r(p) = p(r) := \sum_{k=0}^n a_k r^k 
\end{equation}
für jedes Polynom
\[\textstyle p = \sum_{k=0}^n a_k X^k\]
ist ein Ringhomomorphismus. Man bezeichnet $p(r)$ als \emdef{Einsetzung}
von $r$ in $p$ und $\varphi_r$ als
\emdef{Einsetzungshomomorphismus}\index{Einsetzungshomomorphimus}.
\end{definition}

Man kann auch $R'=R$ und $r=X$ setzen, dann gilt $p=p(X)$.
Ein Polynom stimmt also mit der Einsetzung seiner eigenen formalen
Variablen überein.

\begin{definition}[Polynomfunktion]
Für ein festes $p\in R[X]$ wird die Funktion
\begin{equation}
f\colon R'\to R',\quad f(x):=p(x)
\end{equation}
als \emdef{Polynomfunktion} bezeichnet.
\end{definition}

\noindent
In einigen Ringen können unterschiedliche Polynome zur selben
Polynomfunktion führen. Handelt es sich bei $R$ jedoch um einen
unendlichen Körper, z.\,B. $R=\R$ oder $R=\C$, dann gibt es zu jeder
Polynomfunktion nur ein einziges Polynom.

\section{Körper}
\begin{definition}[Körperhomomorphismus]
Sind $(K,+,\bullet)$ und $(K',+',\bullet')$ Körper, so
wird $\varphi\colon K\to K'$ als \emph{Körperhomomorphismus}
bezeichnet, wenn
\begin{align}
\varphi(a+b) &= \varphi(a)+'\varphi(b),\\
\varphi(a\bullet b) &= \varphi(a)\bullet'\varphi(b)
\end{align}
für alle $a,b\in K$ gilt und $\varphi(1)=1$ ist.
\end{definition}




\chapter{Kategorientheorie}

\section{Grundbegriffe}

\begin{Definition}[Kategorie]
Eine Kategorie ist ein Tripel $\category C = (\Ob,\Hom,{\circ})$,
sofern die folgenden beiden Axiome erfüllt sind:
\begin{enumerate}
\item Für $f\colon A\to B$, $g\colon B\to C$, $h\colon C\to D$ gilt
das Assoziativgesetz\\
$h\circ (g\circ f) = (h\circ g)\circ f$.
\item Für jedes Objekt $X$ existiert die Identität $\id_X\colon X\to X$,
so dass $f\circ\id_A = \id_B\circ f=f$ für alle Objekte $A,B$
und $f\colon A\to B$.
\end{enumerate}
\end{Definition}
Die Elemente der Klasse $\Ob$ nennt man Objekte. Die
Elemente der Klasse $\Hom$ nennt man Morphismen. Die
Verknüpfung $g\circ f$, sprich $g$ nach $f$, nennt man Verkettung
von $g$ und $f$.

Die Schreibweise ist $f\colon X\to Y$ gleichbedeutend mit
$f\in\Hom(X,Y)$, wobei $X,Y\in\Ob$.
Mit $\Hom(X,Y)$ ist die Teilklasse von
$\Hom$ gemeint, die alle Morphismen von $X$ nach $Y$
enthält. Man schreibt $\dom(f) = X$ und $\cod(f) = Y$.

Nun gut, man macht hier zunächst zwei Beobachtungen. Erstens
erinnern die Axiome an die Monoid"=Axiome, haben aber den Unterschied,
dass die Morphismen kompatibel sein müssen. D.\,h. um $g\circ f$
bilden zu können, muss $\cod(f)=\dom(g)$ sein.

Zweitens erinnern die Axiome an die Regeln für die Verkettung
von Abbildungen. Tatsächlich bilden die Abbildungen eine Kategorie.

\begin{Satz}[Kategorie der Mengen]\mbox{}\\*
Sei $\Omega$ das Mengenuniversum und für $A,B\in\Omega$ sei
$\Hom(A,B):=\Abb(A,B)$. Sei $g\circ f$ die Verkettung
von Abbildungen. Dann bildet $\strong{Set}:=(\Omega,\Hom,\circ)$
eine Kategorie.
\end{Satz}
\strong{Beweis.} Trivial.\;\qedsymbol

\begin{Satz}[Kategorie der Gruppen]\mbox{}\\*
Sei $\Omega$ die Klasse aller Gruppen und für $G,H\in\Omega$ sei
$\Hom(G,H)$ die Klasse der Homomorphismen von $G$ nach $H$.
Sei $g\circ f$ die Verkettung von Homomorphismen.
Dann bildet $\strong{Group}:=(\Omega,\Hom,\circ)$
eine Kategorie.
\end{Satz}
\strong{Beweis.} Homomorphismen sind Abbildungen, die Axiome
daher wie bei der Kategorie der Mengen erfüllt. Die Verkettung
zweier Homomorphismen ist ja auch ein Homomorphismus.\;\qedsymbol

Entsprechend bilden Ringe mit Ringhomomorphismen, Körper mit
Körperhomomorphismen, Vektorräume mit Vektorraumhomomorphismen
usw. Kategorien. Des Weiteren bilden die endlichen Mengen, Gruppen,
Ringe jeweils eine Kategorie.

Nun ist es so, dass Gruppen auch Mengen und Homomorphismen
auch Abbildungen sind. Die Kategorie der Gruppen ist gewissermaßen
in der Kategorie der Mengen enthalten. Um das zu präzisieren,
benötigen wir den Begriff des Vergissfunktors.

\begin{Definition}[Kovarianter Funktor]\mbox{}\\*
Sind $\category C,\category D$ Kategorien, dann nennt man
$F\colon\category C\to\category D$ einen
kovarianten Funktor, wenn jedem Objekt $X\in\Ob(\category C)$ ein Objekt
$F(X)\in\Ob(\category D)$ zugeordnet wird und jedem Morphismus
$f\in\Hom_{\category C}(X,Y)$ ein ein Morphismus
$F(f)\in\Hom_{\category D}(F(X),F(Y))$ zugeordnet wird,
so dass die folgenden beiden Verträglichkeitsaxiome erfüllt sind:%
\begin{gather*}
F(g\circ f) = F(g)\circ F(f),\\
F(\id_X) = \id_{F(X)}.
\end{gather*}
\end{Definition}
\begin{Definition}[Kontravarianter Funktor]\mbox{}\\*
Wie beim kovarianten Funktor, mit dem Unterschied
$F(g\circ f) = F(f)\circ F(g)$.
\end{Definition}
Bemerkung: Die Notation ist überladen. Nämlich ist die Zuordnung
$F\colon\Ob(\category C)\to\Ob(\category D)$ zu unterscheiden
von
\[\tilde F\colon\Hom_{\category C}(X,Y)\to\Hom_{\category D}(F(X),F(Y)).\]
Das Paar $(F,\tilde F)$ kodiert dann eigentlich den Funktor
$C\to D$.

\begin{Satz}[Vergissfunktor]\mbox{}\\*
Sei $F\colon\strong{Group}\to\strong{Set}$ mit $F((G,*,e)):=G$,
und jedem Gruppenhomomorphismus%
\[\varphi\colon (G,*,e)\to (G',*',e')\]
sei die Abbildung $F(\varphi)\colon G\to G'$ mit
$F(\varphi)(x):=\varphi(x)$ zugeordnet. Bei $F$ handelt
es sich um einen kovarianten Funktor.
\end{Satz}
\strong{Beweis.}
Es gilt $F(\id)(x) = \id(x)$, und daher $F(\id)=\id$.
Außerdem gilt%
\[
F(\varphi_2\circ\varphi_1)(x) = (\varphi_2\circ\varphi_1)(x)
= \varphi_2(\varphi_1(x))
= F(\varphi_2)(F(\varphi_1)(x))
= (F(\varphi_2)\circ F(\varphi_1))(x),
\]
und daher $F(\varphi_2\circ\varphi_1)
= F(\varphi_2)\circ F(\varphi_1)$.\;\qedsymbol

\begin{Satz} Sei $P(X)=2^X$ die Potenzmenge von $X$. Dann ist
wie folgt ein kovarianter Funktor gegeben:
\[P\colon\strong{Set}\to\strong{Set},\quad
P(X):=2^X,\quad P(f)(M):=f(M),\]
wobei $f$ eine beliebige Abbildung
und $f(M)$ die Bildmenge von $M$ unter $f$ ist.
\end{Satz}
\strong{Beweis.} Nach Satz \ref{Bild-unter-Komposition} gilt
\[P(g\circ f)(M) = (g\circ f)(M) = g(f(M))
= P(g)(P(f)(M)) = (P(g)\circ P(f))(M).\]
Daher ist $P(g\circ f)=P(g)\circ P(f)$. Außerdem ist
\[P(\id_X)(M) = \id_X(M) = M = \id_{P(X)}(M)\]
und daher $P(\id_X)=\id_{P(X)}$.\;\qedsymbol

Zum Funktor $P$ kommt noch ein weiterer Aspekt hinzu.
Für eine Abbildung $f$ kann man ganz pedantisch das
Bild $f(x)$ von der Bildmenge $f(\{x\})$ unterscheiden.
Aufgrund der Gleichung $f(\{x\})=\{f(x)\}$ verschwimmt diese
Unterscheidung aber gewissermaßen.
Die Abbildungen $f$ und $P(f)$ verhalten sich also
gewissermaßen gleich. Man kann sagen, dass $f$
auf ganz natürliche Art und Weise die Abbildung $P(f)$
zugeordnet ist. Definiert man
\[\eta(X)\colon X\to 2^X,\quad \eta(X)(x):=\{x\},\]
dann kommutiert das folgende Diagramm:
\[\xymatrix{
X \ar[r]^f \ar[d]_{\eta(X)} & Y \ar[d]^{\eta(Y)} \\
2^X \ar[r]_{P(f)} & 2^Y }\]
D.\,h. es gilt $\eta(Y)\circ f = P(f)\circ\eta(X)$.
Die Zuordnung $\eta$ ist eine sogenannte natürliche Transformation.

\begin{Definition}[Natürliche Transformation]
Seien $\category C, \category D$ Kategorien und
$F,G\colon \category C\to \category D$ Funktoren.
Dann schreibt man $\eta\colon F\to G$ und nennt $\eta$ natürliche
Transformation, wenn die folgenden beiden Axiome erfüllt sind:
\begin{enumerate}
\item Jedes Objekt $X\in\Ob(\category C)$ bekommt einen Morphismus
$\eta(X)\colon F(X)\to G(X)$.
\item Für jeden Morphismus $f\colon X\to Y$ gilt
$\eta(Y)\circ F(f)=G(f)\circ\eta(X)$.
\end{enumerate}
\end{Definition}
Die zweite Bedingung lässt sich übersichtlich als kommutierendes Diagramm
darstellen:
\[\xymatrix{
F(X) \ar[r]^{F(f)} \ar[d]_{\eta(X)} & F(Y) \ar[d]^{\eta(Y)} \\
G(X) \ar[r]_{G(f)} & G(Y)}\]
Ein weiteres Beispiel ergibt sich bezüglich Äquivalenzrelationen
in Erinnerung an \eqref{eq:induzierte-Abbildung}.
Eine Abbildung $f\colon M\to M'$ heiße \emph{induzierend}, wenn%
\[x\sim a \implies f(x)\sim' f(a).\]
\begin{Satz}
Die Paare $(M,\sim)$, bestehend aus Menge und Äquivalenzrelation,
bilden mit den induzierenden Abbildungen
als Morphismen bezüglich Verkettung eine Kategorie.
\end{Satz}
\strong{Beweis.}
Die identische Abbildung ist offensichtlich induzierend. Hat man
neben $f\colon M\to M'$ eine weitere induzierende Abbildung $g\colon M'\to M''$, dann
folgt $g(y)\sim'' g(b)$ aus $y\sim' b$. Aus $x\sim a$ folgt
mit $y:=f(x)$ und $b:=f(a)$ somit $g(f(x))\sim'' g(f(x))$.
Daher ist auch $g\circ f$ induzierend.\;\qedsymbol

Genau dann wenn $f$ induzierend ist, existiert eine induzierte
Abbildung
\[I(f)\colon M/\sim\to M'/\sim',\;\text{so dass}\;I(f)\circ\pi = \pi'\circ f,\]
wobei $\pi,\pi'$ jeweils die kanonische Projektion ist.
\begin{Satz}
Bei der Induktion $I$ handelt es sich um einen kovarianten Funktor.
\end{Satz}
\strong{Beweis.} Man betrachte das folgende kommutierende Diagramm:
\[\xymatrix{
M \ar[r]^{f} \ar[d]_{\pi}
& M' \ar[r]^{g} \ar[d]^{\pi'}
& M'' \ar[d]^{\pi''}\\
M/\sim \ar[r]_{I(f)}
& M'/\sim' \ar[r]_{I(g)}
& M''/\sim''}\]
Die Induktion $I$ besitzt die Eigenschaften
\begin{gather*}
I(f)\circ\pi = \pi'\circ f,\\
I(g)\circ\pi' = \pi''\circ g,\\
I(g\circ f)\circ\pi = \pi''\circ (g\circ f).
\end{gather*}
Damit kann man nun rechnen
\begin{equation}
I(g\circ f)\circ\pi = \pi''\circ g\circ f
= I(g)\circ\pi'\circ f = I(g)\circ I(f)\circ\pi.
\end{equation}
Infolge gilt $I(g\circ f)=I(g)\circ I(f)$, da die kanonische
Projektion $\pi$ eine Surjektion ist. Aus der Forderung $I(\id)\circ\pi
= \pi\circ\id = \pi$ ergibt sich $I(\id) = \id$,
da $\pi$ surjektiv ist.\;\qedsymbol

Die Abbildung $\eta((M,\sim)):=\pi$, die jeder Menge mit
Äquivalenzrelation ihre kanonische Projektion zuordnet,
ist eine natürliche Transformation.

\strong{Beispiel zur Vertiefung.}
Ein weiteres Beispiel berührt einen Grundbegriff der linearen Algebra.
Hat man einen Vektorraum $Y$ über dem Körper $K$, ohne dass wir jetzt
genau verstehen müssen was das bedeutet -- man stelle sich $Y:=\R$
und $K:=\R$ vor --, dann bildet für eine beliebige Menge $X\ne\emptyset$
auch $\Abb(X,Y)$ einen Vektorraum über diesem Körper bezüglich den
punktweisen Operationen%
\begin{gather*}
(\lambda f)(x) := \lambda f(x),\\
(f_1+f_2)(x) := f_1(x)+f_2(x),
\end{gather*} 
wobei $\lambda\in K$, $x\in X$, $f,f_1,f_2\in\Abb(X,Y)$. Die lineare
Algebra handelt von \emph{linearen Abbildungen}. Seien $V,W$
Vektorräume über dem Körper $K$, wobei auch $V=W$ sein darf. Eine
Abbildung $\varphi\colon V\to W$ heißt linear, falls%
\begin{gather*}
\forall v_1,v_2\in V\colon\;\varphi(v_1+v_2) = \varphi(v_1)+\varphi(v_2),\\
\forall \lambda\in K,v\in V\colon\;\varphi(\lambda v) = \lambda\varphi(v).
\end{gather*}
Man kann zeigen dass die Vektorräume mit den linearen Abbildungen
als Morphismen auch eine Kategorie bilden, aber darauf will ich an
dieser Stelle nicht hinaus. Hat man nun eine feste, aber beliebige
Abbildung $g\colon X'\to X$, dann ist%
\[\varphi\colon\Abb(X,Y)\to\Abb(X',Y),\quad \varphi(f):=f\circ g\]
eine lineare Abbildung, man spricht auch von einem
\emph{linearen Operator}, dem \emph{Kompositionsoperator}
$C_g=\varphi$. Die Bestätigung ist nicht sonderlich schwer, man
muss bloß blind die Definitionen einsetzen und dem Formalismus
folgen. Es gilt%
\begin{align*}
\varphi(\lambda f)(x) &= ((\lambda f)\circ g)(x)
= (\lambda f)(g(x)) = \lambda f(g(x))\\
&= \lambda (f\circ g)(x) = \lambda (\varphi(f))(x)
= (\lambda\varphi(f))(x),
\end{align*}
kurz $\varphi(\lambda f)=\lambda\varphi(f)$. Und es gilt
\begin{align*}
\varphi(f_1+f_2)(x) &= ((f_1+f_2)\circ g)(x)
= (f_1+f_2)(g(x)) = f_1(g(x)) + f_2(g(x))\\
&= (f_1\circ g)(x) + (f_2\circ g)(x)
= \varphi(f_1)(x)+\varphi(f_2)(x)\\
&= (\varphi(f_1)+\varphi(f_2))(x),
\end{align*}
kurz $\varphi(f_1+f_2)=\varphi(f_1)+\varphi(f_2)$.

Man bemerkt nun, dass diese Rechnungen lediglich auf der Eigenschaft
der Operationen beruhen, punktweise zu sein. Dies soll im Folgenden
präzisiert werden. Die Formulierung wollen wir allgemein für
Operationen beliebiger Stelligkeit haben. Sei also $p\colon Y^n\to Y$
eine $n$-stellige Operation, man stelle sich dabei $Y:=\R$ vor. Man
definiert nun die punktweise Anwendung von $p$ als%
\begin{equation}\label{eq:eta-punktweise}
\eta_p\colon\Abb(X,Y)^n\to\Abb(X,Y),\quad
\eta_p(f)(x) := p(f_1(x),\ldots,f_n(x)),
\end{equation}
wobei $f:=(f_1,\ldots,f_n)$ ein Tupel von Funktionen ist. Sei außerdem%
\begin{equation}
F(\varphi)(f) := (\varphi(f_1),\ldots,\varphi(f_n)).
\end{equation}
Zeigen wollen wir für $\varphi(f):=f\circ g$ nun
\begin{equation}
\varphi(\eta_p(f))(x) = \eta_p(F(\varphi)(f))(x),
\quad\text{kurz}\quad
\varphi\circ\eta_p = \eta_p\circ F(\varphi).
\end{equation}
Bei der Bestätigung folgt man wieder blind den Definitionen und dem
Formalismus. Es gilt%
\begin{align}
\varphi(\eta_p(f))(x)
&= \eta_p(f)(g(x)) = p(f_1(g(x)),\ldots,f_n(g(x)))\\
&= p(\varphi(f_1)(x),\ldots,\varphi(f_n)(x))
= \eta_p(\varphi(f_1),\ldots,\varphi(f_n))(x)\\
&= \label{eq:Ende-eta-punktweise}\eta_p(F(\varphi)(f))(x).
\end{align}
Das bedeutet, dieses Diagramm kommutiert:
\[\xymatrix{
\Abb(X,Y)^n \ar[r]^{F(\varphi)} \ar[d]_{\eta_p}
& \Abb(X',Y)^n \ar[d]^{\eta_p} \\
\Abb(X,Y) \ar[r]_{\varphi} & \Abb(X',Y)}\]
Das legt den Verdacht nahe, dass es sich bei $F$ um einen Funktor
und bei $\eta_p$ um eine natürliche Transformation handelt.

\begin{Satz}[Kategorie mit Kompositionsoperatoren als Morphismen]\mbox{}\\*
Sei $Y\ne\emptyset$. Sei $\Omega:=\{\Abb(X,Y)\mid X\;\text{ist beliebig}\}$.
Sei%
\begin{gather*}\Hom(\Abb(X,Y),\Abb(X',Y))\\
\quad := \{\varphi\mid\exists f{\in}\Abb(X,Y), g{\in}\Abb(X',X)\colon
\varphi = f\circ g\}.
\end{gather*}
Sei $\psi\circ \varphi$ die gewöhnliche Verkettung. Dann bildet
$\strong{Komp}:=(\Omega,\Hom,\circ)$ eine Kategorie.
\end{Satz}
\strong{Beweis.}
Zunächst müssen wir bestätigen, dass $\Hom$ bezüglich
$\circ$ abgeschlossen ist. Sei $\varphi_1(f):=f\circ g_1$ und
$\varphi_2(f):=f\circ g_2$ mit $\dom(\varphi_2)=\cod(\varphi_1)$, so
dass man $\varphi:=\varphi_2\circ\varphi_1$ bilden kann.
Gesucht ist ein $g$, so dass $\varphi = f\circ g$. Nun gilt%
\begin{equation}
(\varphi_2\circ\varphi_1)(f) = (f\circ g_1)\circ g_2
= f\circ g_1\circ g_2 = f\circ (g_1\circ g_2).
\end{equation}
Man kann also $g:=g_1\circ g_2$ setzen. Nun verbleibt bloß noch die
Existenz fester Identitäten $\id$ zu bestätigen. Man definiert dazu
$\id(f):=f\circ\id$. Für $\varphi(f):=f\circ g$ gilt dann%
\begin{gather}
(\varphi\circ\id)(f) = (f\circ \id)\circ g = f\circ g = \varphi(f),\\
(\id\circ\varphi)(f) = (f\circ g)\circ\id = f\circ g = \varphi(f).\;\qedsymbol
\end{gather}
\begin{Satz}
Bei $F(\varphi)(f):=(\varphi(f_1),\ldots,\varphi(f_n))$ für
$f=(f_1,\ldots,f_n)$ handelt es sich um einen kovarianten Funktor.
\end{Satz}
\strong{Beweis.} Es gilt
\begin{align}
F(\psi\circ\varphi)(f) &= (\psi(\varphi(f_1)),\ldots,\psi(\varphi(f_n)))
= F(\psi)(F(\varphi)(f))
= (F(\psi)\circ F(\varphi))(f),
\end{align}
kurz $F(\psi\circ\varphi) = F(\psi)\circ F(\varphi)$. Und es gilt
\begin{equation}
F(\id)(f) = (\id(f_1),\ldots,\id(f_n)) = (f_1,\ldots,f_n) = f = \id(f),
\end{equation}
kurz $F(\id) = \id$.\;\qedsymbol
\begin{Satz}
Die in \eqref{eq:eta-punktweise} definierte Operation $\eta_p$ ist
eine natürliche Transformation.
\end{Satz}
\strong{Beweis.} Wurde in \eqref{eq:eta-punktweise} bis
\eqref{eq:Ende-eta-punktweise} schon gezeigt.\;\qedsymbol

Jetzt haben wir so viele Funktoren kennengelernt, dass allgemeine
Regeln betreffend Funktoren gut motiviert sind. Der folgende Korollar
durchleuchtet die Funktoren ein wenig.
\begin{Korollar}
Wird ein Funktor auf einen Isomorphismus angewendet, ist das Resultat
wieder ein Isomorphismus.
\end{Korollar}
\strong{Beweis.} Dieser ist direkt aus den Definitionen zu erhalten.
Wir betrachten nur einen kovarianten Funktor $F$, weil der Beweis für
einen kontravarianten Funktor analog verläuft.

Sei $f\colon X\to Y$ ein beliebiger Isomorphismus im Definitionsbereich
des Funktors. Laut Definition existert eine Inverse, das heißt, ein
$g\colon Y\to X$ mit $g\circ f = \id_X$ und $f\circ g = \id_Y$. Gemäß
der definierenden Eigenschaft eines Funktors darf man rechnen
\begin{gather*}
\id_{F(X)} = F(\id_X) = F(g\circ f) = F(g)\circ F(f),\\
\id_{F(Y)} = F(\id_Y) = F(f\circ g) = F(f)\circ F(g).
\end{gather*}
Somit ist $F(f)$ ein Isomorphismus mit Inverse $F(g)$.\,\qedsymbol

Wird beispielsweise der Vergissfunktor von Gruppen zu Mengen auf
einen Gruppenisomorphismus angewendet, ist das Resultat zwingend
eine Bijektion, da die Bijektionen die Isomorphismen in der Kategorie
der Mengen sind.

\section{Anfangsobjekte und Endobjekte}

\begin{Definition}[Anfangsobjekt, Endobjekt, Nullobjekt]\mbox{}\\*
Es sei $\category C$ eine Kategorie. Ein
$A\in\Ob(\category C)$ heißt Anfangsobjekt, wenn
es zu jedem Objekt $X\in\Ob(\category C)$ genau einen
Morphismus $A\to X$ gibt.
Ein $E\in\Ob(\category C)$ heißt Endobjekt, wenn
es zu jedem Objekt $X\in\Ob(\category C)$ genau einen
Morphismus $X\to E$ gibt. Ein Objekt heißt Nullobjekt, wenn
es sowohl Anfangsobjekt als auch Endobjekt ist.
\end{Definition}

\noindent
\strong{Mengen.}
Wir untersuchen zunächst die Kategorie der Mengen. Anfangsobjekt
bedeutet hier eine Menge $A$, bei der es zu jeder Menge $X$ genau
eine Abbildung $A\to X$ gibt. Betrachten wir zunächst endliche
Mengen, dann ergibt sich aufgrund von Gleichung \eqref{eq:Anzahl-Abb}
ja die Bedingung $|X|^{|A|}=1$. Das geht nur, wenn $|A|=0$ ist,
und das bedeutet $A=\emptyset$. Die einzige Abbildung in $\Abb(\emptyset,X)$
ist die leere Abbildung, und dies bleibt auch dann richtig, wenn
$X$ gänzlich beliebig ist. Somit haben wir die leere Menge als
einziges Anfangsobjekt identifiziert.

Endobjekt bedeutet eine Menge $E$, so dass es zu jeder Menge $X$
genau eine Abbildung $X\to E$ gibt. Wieder beschränken wir uns
zunächst auf endliche Mengen und nutzen $\eqref{eq:Anzahl-Abb}$.
Wir erhalten die Bedingung $|E|^{|X|}=1$. Das geht nur, wenn
$|E|=1$ ist. Jede Menge mit einem Element ist also Endobjekt,
denn allgemein gibt es dann nur eine einzige Abbildung, nämlich
die konstante Abbildung. Dies bleibt auch dann richtig, wenn
$X$ gänzlich beliebig ist.

Ein Nullobjekt existiert offenbar nicht.

Benutzen wir doch $1:=\{\emptyset\}$ als kanonisches Endobjekt.
Interessant ist, dass man zu einer Menge $X$ jedes Element
$x\in X$ mit der Abbildung $x\colon 1\to X$ identifizieren kann,
für die $x(\emptyset)=x$ gilt. Zu einer Abbildung $f\colon X\to Y$
können wir die Zuordnung $f(x)=y$ bzw. $(x,y)\in f$ nun in der Form
$f\circ x = y$ beschreiben.

\section{Produkt und Koprodukt}

Ein wichtiger Begriff der Theorie ist das Produkt von
Objekten. Weil es sich dabei um eine Verallgemeinerung des
kartesischen Produktes von Mengen handelt, möchte ich die
Zusammenhänge zunächst am vertrauten Schauplatz der Mengen
betrachten.

Zu zwei Mengen $A_1, A_2$ können wir das Produkt
$A_1\times A_2$ bilden. Man definiert die Projektionen
auf die Komponenten als
\begin{align*}
&\pi_1\colon A_1\times A_2\to A_1,\quad\pi_1((x,y)) := x,\\
&\pi_2\colon A_1\times A_2\to A_2,\quad\pi_2((x,y)) := y.
\end{align*}
Nun betrachten wir Abbildungen $f_1\colon X\to A_1$ und
$f_2\colon X\to A_2$. Zunächst sei $X:=\{\emptyset\}$. Die jeweilige
Abbildung pickt dann ein Element aus der jeweiligen Menge heraus,
das sind $a_1:=f_1(\emptyset)$ und $a_2:=f_2(\emptyset)$. Nun
ist eine Abbildung $f$ gesucht, sodass das Diagramm
\[\xymatrix{
& X\ar[dl]_{f_1}\ar[dr]^{f_2}\ar[d]^f & \\
A_1 & A_1\times A_2 \ar[l]^{\pi_1}\ar[r]_{\pi_2} & A_2
}\]
kommutiert. Die Bedingungen an $f$ sind also
$\pi_i\circ f = f_i$ für $i\in\{1,2\}$. Damit ist aber eindeutig
festgelegt, dass $f(\emptyset)=(a_1,a_2)$ sein muss, denn ein
Tupel ist durch die Komponenten festgelegt, und die sind
$\pi_i(f(\emptyset)) = f_i(\emptyset) = a_i$ für $i\in\{1,2\}$.
Somit ist $f$ eindeutig bestimmt.

Die Betrachtung kann man genauso für eine allgemeine Menge $X$ führen,
weil die Argumentation dann jeweils für jedes Element von $X$ gilt.
Wieder ist $f$ eindeutig bestimmt.

Gelegentlich wird $f$ als $f=f_1\times f_2$ notiert.

\begin{Definition}[Produkt]\newlinefirst
Sei $C$ eine Kategorie und seien $Y_1,Y_2\in C$.
Ein Objekt $Y\in\mathcal C$ mit Projektionen $\pi_1\colon Y\to Y_1$ und
$\pi_2\colon Y\to Y_2$ heißt Produkt, wenn zu jedem Objekt
$X\in C$ und allen Morphismen $f_1\colon X\to Y_1$
und $f_2\colon X\to Y_2$ genau ein Morphismus $f\colon X\to Y$
existiert, so dass $f_1=\pi_1\circ f$ und $f_2=\pi_2\circ f$.
\end{Definition}

\noindent
Das kartesische Produkt $Y:=Y_1\times Y_2$ ist ein Produkt in
der Kategorie der Mengen. Wir schreiben $f(x)=y$ mit $y=(y_1,y_2)$.
Nun ist $\pi_1(y)=y_1$ und $\pi_2(y)=y_2$, laut Forderung soll also
$y_1=f_1(x)$ und $y_2=f_2(x)$ sein. Dadurch ist $f$ mit
$f(x):=(f_1(x),f_2(x))$ eindeutig festgelegt.

Zu zwei Mengen können wir weiterhin die disjunkte Vereinigung
$X_1\sqcup X_2$ bilden. Wir rekapitulieren, dass zu ihr die beiden
kanonischen Injektionen
\begin{align*}
& i_1\colon X_1\to X_1\sqcup X_2,\quad i_1(x) := (1,x),\\
& i_2\colon X_2\to X_1\sqcup X_2,\quad i_2(x) := (2,x)
\end{align*}
gehören. Man stellt sich nun die Frage, was das Wesensmerkmal
der disjunkten Vereinigung ist. Das ist doch, dass zu jedem ihrer
Elemente die Information vorliegt, ob es aus der linken oder der
rechten Menge entstammt. Das heißt, es muss eine Abbildung geben,
die auf den Tag projiziert. Betrachten wir dazu die Abbildungen
$f_1\colon X_1\to Y$ und $f_2\colon X_2\to Y$ mit $Y:=\{1,2\}$ und
$f_k(x):=k$. Mit der Abbildung $f_k$ gelangt man von $X_k$ also
direkt zum Tag $k$. Nun ist eine Abbildung $f$ gesucht, so dass das
Diagramm
\[\xymatrix{
X_1\ar[r]^{i_1}\ar[dr]_{f_1} & X_1\sqcup X_2\ar[d]^{f}
& X_2\ar[l]_{i_2}\ar[dl]^{f_2}\\
& Y &
}\]
kommutiert. Das heißt, es soll $f\circ i_k = f_k$ für $k\in\{1,2\}$ sein.
Das macht $f((1,x)) = 1$ und $f((2,x)) = 2$. Dadurch ist $f$ eindeutig
bestimmt. Es ist die gesuchte Projektion auf den Tag.

\begin{Definition}[Koprodukt]\newlinefirst
Sei $C$ eine Kategorie und seien $X_1,X_2\in C$. Ein Objekt $X$
mit Morphismen $i_1\colon X_1\to X$ und $i_2\colon X_2\to X$
heißt Koprodukt, wenn zu jedem Objekt $Y\in C$ und allen Morphismen
$f_1\colon X_1\to Y$ und $f_2\colon X_2\to Y$ genau ein Morphismus
$f\colon X\to Y$ existiert, so dass $f_1 = f\circ i_1$
und $f_2 = f\circ i_2$.
\end{Definition}

\noindent
Die disjunkte Vereinigung $X:=X_1\sqcup X_2$ mit den Injektionen
$i_1(x):=(1,x)$ und $i_2(x):=(2,x)$ ist ein Koprodukt in der
Kategorie der Mengen. Es gilt schon mal
\[f(x) = \strong{match}\; x \begin{cases}
(1,x)\mapsto y_1,\\
(2,x)\mapsto y_2.
\end{cases}\]
Laut Forderung soll außerdem $y_1 = f_1(x)$ und $y_2 = f_2(x)$ sein,
wodurch $f$ eindeutig festgelegt ist.


\chapter{Elemente der Stochastik}

\section{Grundbegriffe}

\subsection{Ereignisse}

Die Wahrscheinlichkeitstheorie beschäftigt sich mit
\emph{Zufallsexperimenten}. Darunter versteht man ein Experiment
mit zufälligem Ausgang, das, um der Wissenschaftlichkeit genüge zu tun,
unter genau definierten Versuchsbedingungen durchgeführt wird.
Der Ausgang führt immer zu einem \emph{Ergebnis}. Alle erreichbaren
Ergebnisse fasst man zur \emph{Ergebnismenge} zusammen. Allgemeiner
genügt es, wenn jedes Ergebnis in der Ergebnismenge liegt, wobei diese
aber auch Elemente enthalten darf, die das Experiment niemals abwirft.
Jede Teilmenge der Ergebnismenge nennt man ein \emph{Ereignis}\index{Ereignis}.
Die Potenzmenge der Ergebnismenge heißt \emph{Ereignisraum}, sie
besteht aus allen denkbaren Ereignissen. Man sagt, ein Ereignis sei
eingetreten, wenn das Ergebnis des Versuchs im diesem Ereignis liegt.

Zu beachten ist, dass wir dabei eine endliche oder höchstens
abzählbar unendliche Ergebnismenge voraussetzen. Bei überabzählbaren
Ergebnismengen kommt es zu Unwägbarkeiten, deren Klärung Gegenstand der
Maßtheorie ist.

Ein schlichtes Experiment bietet der Wurf des Spielwüfels, ein mit
Augenzahlen beschriftetes regelmäßiges Hexaeder. Die Ergebnismenge
wird als
\[\Omega := \{1, 2, 3, 4, 5, 6\}\]
festgelegt. Betrachten wir die drei Ereignisse
\[A := \{2\},\quad B := \{1,2\},\quad C:=\{1,3\}.\]
Ist $\omega=2$ das Ergebnis des Versuchs, sind die Ereignisse $A,B$
eingetreten. Zwei Ereignisse, die niemals gleichzeitig eintreten,
heißen \emph{disjunkt}. So sind die $A,C$ disjunkt, weil ihre
Schnittmenge leer ist.

\subsection{Wahrscheinlichkeiten}

Man kann nicht voraussagen, wie ein Experiment ausgehen wird. Das
Wahrscheinlichkeitsmaß liefert dennoch ein Maß dafür, wie sicher der
Eintritt eines Ereignisses ist. Wahrscheinlichkeit wird tiefergründig
verständlich, wenn dasselbe Zufallsexperiment abermals wiederholt wird.
Wir zählen, wie häufig ein Elementarereignis eingetreten ist.

Es sei ein Versuch $n$ mal durchgeführt worden, was zu den Ergebnissen
$a_i$ für $i=1$ bis $i=n$ geführt hat. Wir definieren die \emph{relative
Häufigkeit} des Ereignisses $A$ als die Zahl
\[r_{n,a}(A) := \frac{1}{n}|\{i\in\{1,\ldots,n\}\mid a_i\in A\}|.\]
Relative Häufigkeiten bieten bei hinreichend großem $n$ eine Näherung
für die Wahrscheinlichkeit. Zur Vermessung eines Würfels wird man
diesen also möglichst oft werfen wollen. Man erhält so die relativen
Häufigkeiten der Elementarereignisse, und damit näherungsweise auch
ihre Wahrscheinlichkeiten. So lässt sich feststellen, ob ein
Würfel gezinkt wurde.

Fassen wir $a$ als Funktion $i\mapsto a_i$ auf, können wir schreiben
\[\{i\mid a_i\in A\} = \{i\mid i\in a^{-1}(A)\} = a^{-1}(A).\]
Für disjunkte Ereignisse $A,B$ erhält man nun
\begin{align*}
r_{n,a}(A\cup B) &= \tfrac{1}{n}|a^{-1}(A\cup B)|
= \tfrac{1}{n}|a^{-1}(A)\cup a^{-1}(B)|\\
&= \tfrac{1}{n}|a^{-1}(A)| + \tfrac{1}{n}|a^{-1}(B)|
= r_{n,a}(A) + r_{n,a}(B).
\end{align*}

\subsection{Zufallsgrößen}

Eine \emph{Zufallsgröße}\index{Zufallsgroesse@Zufallsgröße} darf man
sich als eine Funktion $X\colon\Omega\to\Omega'$
vorstellen, die eine kausale Verbindung zwischen den Ergebnismengen
$\Omega,\Omega'$ schafft. Ein Ergebnis $\omega\in\Omega$ führt
zu $X(\omega)$. Ursächlich für ein $x\in\Omega'$ sind daher all die
$\omega$ mit $x=X(\omega)$. Das heißt, ursächlich für das
Elementarereignis $\{x\}$ ist dessen Urbild $X^{-1}(\{x\})$.
Infolge muss die Wahrscheinlichkeit von $\{x\}$ die es Urbildes sein.
Insofern definiert man auf $\Omega'$ das Wahrscheinlichkeitsmaß%
\[P_X\colon\mathcal P(\Omega')\to [0,1],\quad P_X(A) := P(X^{-1}(A)).\]
Man nennt $P_X$ die \emph{Verteilung}\index{Verteilung} von $X$. Mit der
identischen Zufallsgröße
\[\id\colon\Omega\to\Omega,\quad \id(\omega) := \omega\]
versteht sich auch das ursprüngliche Maß $P$ als die Verteilung $P=P_{\id}$.

Geläufig sind die Schreibweisen
\begin{align*}
P(X=x) &:= P(X^{-1}(\{x\})), & \{X=x\} &:= X^{-1}(\{x\}),\\
P(X\in A) &:= P(X^{-1}(A)), & \{X\in A\} &:= X^{-1}(A).
\end{align*}
Es ist $\{X=x\}$ dasselbe wie $\{X\in\{x\}\}$. 
Ist $P$ die Gleichverteilung auf $\Omega$, ergibt sich
\[P(X\in A) = \frac{|\{X\in A\}|}{|\Omega|}.\]
Standardbeispiel. Wir werfen zwei Spielwürfel.
Die Ergebnismenge sei%
\[\Omega := \{1,\ldots,6\}\times\{1,\ldots,6\},\]
und jedes der 36 elementaren Ereignisse sei gleich wahrscheinlich, habe
also die Wahrscheinlichkeit $\tfrac{1}{36}$.
Es bezeichne $\omega_1$ das Ergebnis des ersten, und $\omega_2$
das des zweiten Wurfs. Wir betrachten die Zufallsgröße%
\[X\colon\Omega\to\{2,\ldots,12\},\quad
X(\omega_1,\omega_2) := \omega_1+\omega_2.\]
Gesucht sei $P(X=4)$. Man ermittelt
\[\{X=4\} = \{(1,3), (2,2), (3,1)\},\quad\text{ergo}\;P(X=4) = \tfrac{3}{36}.\]
Allgemein zerfällt ein Ereignis $A$ ja in seine disjunkten Elementarereignisse
$\{x\}$, so dass $A = \bigcup_{x\in A} \{x\}$ gilt. Weil nun die
Fasern $X^{-1}(\{x\})$ ebenfalls disjunkt sind, muss $P(X\in A)$ die
Summe der $P(X=x)$ mit $x\in A$ sein. Das heißt, man rechnet%
\[P(X\in A) = P(X^{-1}(\bigcup_{x\in A} \{x\})) = P(\bigcup_{x\in A} X^{-1}(\{x\}))
= \sum_{x\in A} P(X=x).\]
Die Verteilung $P_X$ ist demzufolge bereits eindeutig bestimmt,
sobald $P(X=x)$ für jedes $x\in\Omega'$ vorliegt. Dies motiviert
uns, die Funktion
\[p_X\colon\Omega\to[0,1],\quad p_X(x) := P(X=x)\]
zu definieren, genannt die \emph{Wahrscheinlichkeitsfunktion} der
Zufallsgröße $X$.

\newpage
\section{Mehrstufige Experimente}

\subsection{Bedingte Wahrscheinlichkeiten}

Es findet ein zweistufiges Experiment statt, welches sich sich
aus einem ersten und einem zweiten Wurf eines Spielwürfels
zusammensetzt. Bei jedem der Würfe bestehe eine Gleichverteilung.
Zur Frage steht, wie wahrscheinlich das Ereignis $\{(6,6)\}$ ist.
Ein Paar $(\omega_1,\omega_2)$ fasse hierbei das Ergebnis $\omega_1$
des ersten und $\omega_2$ des zweiten Wurfs zusammen.

Die Wahrscheinlichkeit der ersten Sechs beträgt $\tfrac{1}{6}$,
die der zweiten ebenfalls $\tfrac{1}{6}$. Sie multiplizieren sich
zu zu $\tfrac{1}{36}$, richtig?

Es wäre doch möglich, dass zwischen den beiden Würfen eine,
sagen wir, geisterhafte Beziehung besteht, dergestalt dass
der zweite Wurf niemals in einer Sechs resultiert, sofern das
Ergebnis des ersten eine war. Trotzdem sind die Wahrscheinlichkeiten bei
jedem der Würfe für sich allein gesehen gleichverteilt. Dafür muss man
nicht unbedingt die Wirklichkeit manipulieren. Das Phänomen ist bereits
bei der Erzeugung von Zufallszahlen im Computer beobachtbar. War die
erste Zufallszahl eine Sechs, braucht der Generator die zweite lediglich
solange zu verwerfen, wie sie eine Sechs sein sollte. Umstände dieser
Art stellen nicht nur ein Gedankenspiel dar, so dass wir uns notgedrungen
mit ihnen auseinandersetzen müssen. Sie führen zum Begriff der
\emph{bedingten Wahrscheinlichkeit}.

Bisher wurde immer nur die Verteilung der Wahrscheinlichkeiten eines
Würfels für sich allein betrachtet. Das war modelliert durch die Größe
\[X_0\colon\Omega\to\Omega,\quad X_0(\omega):=\omega,\quad \Omega := \{1,\ldots,6\},\]
mit der Gleichverteilung $P_0$, so dass $P_0(X=6)=\tfrac{1}{6}$.

Wir modellieren das zweistufige Experiment durch die Zufallsgröße
\[X\colon\Omega^2\to\Omega^2,\quad X(\omega) := (X_1,X_2)(\omega)
= (X_1(\omega), X_2(\omega)),\]
die sich mit $\omega = (\omega_1,\omega_2)$ aus den zwei Zufallsgrößen
\begin{gather*}
X_1\colon\Omega^2\to\Omega,\quad X_1(\omega_1,\omega_2) := \omega_1,\\
X_2\colon\Omega^2\to\Omega,\quad X_2(\omega_1,\omega_2) := \omega_2
\end{gather*}
zusammensetzt. Es stellt $X_1(\omega)$ das Ergebnis des ersten und
$X_2(\omega)$ das des zweiten Wurfs dar. Wie gewünscht gilt
\[(X_1(\omega),X_2(\omega)) = (X_0(\omega_1),X_0(\omega_2)) = (\omega_1,\omega_2).\]
Es bezeichne $P$ die Verteilung auf $\Omega^2$. Wir wissen hier allerdings
lediglich
\begin{gather*}
P(X_1=\omega_1) = P_0(X_0=\omega_1) = \tfrac{1}{6},\\
P(X_2=\omega_2) = P_0(X_0=\omega_2) = \tfrac{1}{6}.
\end{gather*}
Die Fehlannahme besteht nun darin, dass per se
\[P(\{X_1=\omega_1\}\cap\{X_2=\omega_2\}) = P(X_1=\omega_1)P(X_2=\omega_2)\]
gelten müsse. Ist diese Gleichung erfüllt, nennt man die
Zufallsgrößen $X_1,X_2$ \emph{unabhängig}. In der bisherigen Sichtweise,
wo wir nur $X_0$ mit $P_0$ gesehen haben, war es uns nicht möglich,
stochastische Abhängigkeit zu beschreiben. Man notiert allgemein%
\[P(X=x,Y=y) := P(\{X=x\}\cap\{X=y\}) = P(X=x)P(Y=y\mid X=x).\]
Der letzte Faktor bezeichne hierbei die bedingte Wahrscheinlichkeit
für das Ereignis $\{Y=y\}$, unter der Bedingung, dass $\{X=x\}$
bereits eingetreten ist.
\begin{Definition}[Bedingte Wahrscheinlichkeit]\newlinefirst
Die bedingte Wahrscheinlichkeit für den Eintritt von $A$ unter der
Bedingung $B$ ist für $P(B)\ne 0$ definiert gemäß
\[P(A\mid B) := \frac{P(A\cap B)}{P(B)}.\]
\end{Definition}
Wir setzen speziell $B:=\{X=x\}$ und $A:=\{Y=y\}$ ein,
das macht
\[P(Y=y\mid X=x) = \frac{P(X=x,Y=y)}{P(X=x)}.\]
Sind $X,Y$ unabhängig, gilt also
\[P(Y=y\mid X=x) = P(Y=y).\]
Mit der geisterhaften Beziehung zwischen den Würfeln wäre allerdings
\[0 = P(X_2=6\mid X_1=6) \ne P(X_1=6) = \tfrac{1}{6}.\]


\begin{thebibliography}{00}

\bibitem{Hoffmann}
Dirk W. Hoffmann: »Grenzen der Mathematik«.
Springer"=Verlag, Berlin 2011.

\bibitem{Klein}
Felix Klein: »Elementarmathematik vom höheren Standpunkte aus«.\\
Springer"=Verlag, Berlin 1933.

\bibitem{Rubin}
Jean E. Rubin: »Set Theory for the Mathematician«.
Holden"=Day, San Francisco 1967.

\bibitem{Companion}
Timothy Gowers (ed.): »The Princeton Companion to Mathematics«.
Princeton University Press, Princeton 2008.

\end{thebibliography}


\printindex

\end{document}


