
\begin{thebibliography}{00}

\bibitem{Gentzen1935}
Gerhard Gentzen: \emph{Untersuchungen über das logische Schließen}.
In: \emph{Mathematische Zeitschrift}. Band~39, 1935, S.~176--210,
S.~405--431.

\bibitem{Gentzen1936}
Gerhard Gentzen: \emph{Die Widerspruchsfreiheit der reinen
Zahlentheorie}. In: \emph{Mathematische Annalen}. Band~112,
1936, S.~493--565.

\bibitem{Gentzen1938}
Gerhard Gentzen: \emph{Die gegenwärtige Lage in der mathematischen
Grundlagenforschung. Neue Fassung des Widerspruchsfreiheitsbeweises für
die reine Zahlentheorie}. In: \emph{Forschungen zur Logik und zur
Grundlegung der exakten Wissenschaften}. Heft~4, S.~Hirzel,
Leipzig 1938.

\bibitem{Menzler-Trott}
Eckart Menzler-Trott: \emph{Gentzens Problem. Mathematische Logik
im nationalsozialistischen Deutschland}. Birkhäuser, Basel 2001.

\bibitem{Johansson}
Ingebrigt Johansson: \emph{Der Minimalkalkül, ein reduzierter
intuitionistischer Formalismus}. In: \emph{Compositio Mathematica}.
Band~4, 1937, S.~119–136.

\bibitem{Diener}
Hannes Diener, Maarten McKubre-Jordens: \emph{Classifying Material
Implications over Minimal Logic}. In: \emph{Archive for Mathematical
Logic}. Band~59, 2020, S.~905--924.
\href{https://doi.org/10.1007/s00153-020-00722-x}%
{doi:10.1007/s00153-020-00722-x}.

\bibitem{von-Plato-Reasoning}
Jan von Plato: \emph{Elements of Logical Reasoning}.
Cambridge University Press, Cambridge 2013.

\bibitem{von-Plato}
Jan von Plato: \emph{Gentzen's Logic}. In: \emph{Handbook of The
History of Logic}. Band~5, North-Holland, 2009.

\bibitem{Hazen}
Francis Jeffry Pelletier, Allen P. Hazen: \emph{A History of Natural
Deduction}.\\ In: \emph{Handbook of The History of Logic}.
Band 11, North-Holland, 2012.

\bibitem{Hazen-online} Francis Jeffry Pelletier, Allen Hazen:
\href{https://plato.stanford.edu/entries/natural-deduction/}%
{\emph{Natural Deduction Systems in Logic}}.\\
In: \emph{The Stanford Encyclopedia of Philosophy}.

\bibitem{Indrzejczak}
Andrzej Indrzejczak:
\href{https://iep.utm.edu/natural-deduction/}{\emph{Natural Deduction}}.
In: \emph{The Internet Encyclopedia of Philosophy}.

\bibitem{Mimram}
Samuel Mimram:
\emph{\href{https://www.lix.polytechnique.fr/Labo/Samuel.Mimram/publications/}%
{Program = Proof}}.
Laboratoire d'Informatique de l'Ecole polytechnique, Palaiseau 2020.

\bibitem{Hoffmann}
Dirk W. Hoffmann: \emph{Grenzen der Mathematik}.
Springer, Berlin 2011.

\bibitem{Ebbinghaus-Logik}
Heinz-Dieter Ebbinghaus, Jörg Flum, Wolfgang Thomas:
\emph{Einführung in die mathematische Logik}.
Springer Spektrum, 1978, 6. Auflage 2018.\\
\href{https://doi.org/10.1007/978-3-662-58029-5}%
{doi:10.1007/978-3-662-58029-5}.

\bibitem{OpenLogic} Open Logic Project:
\emph{The Open Logic Text}. Complete Build, Oktober 2022.

\bibitem{Avigad} Jeremy Avigad:
\emph{Mathematical Logic and Computation}.\\
Cambridge University Press, 2023.

\bibitem{Avigad-Foundations} Jeremy Avigad: \emph{Foundations}.
September 2021. \href{https://arxiv.org/abs/2009.09541v4}{arxiv:2009.09541v4}.

\bibitem{Wansing} Heinrich Wansing (Hrsg.):
\emph{Dag Prawitz on Proofs and Meaning}.
In: \emph{Outstanding Contributions to Logic}. Springer, 2015.
\href{https://doi.org/10.1007/978-3-319-11041-7}{doi:10.1007/978-3-319-11041-7}.

\bibitem{Tarski} Alfred Tarski:
\emph{Einführung in die mathematische Logik}. Springer, Wien 1937.\\
\href{https://doi.org/10.1007/978-3-7091-5928-6}{doi:10.1007/978-3-7091-5928-6}.

\bibitem{Enderton} Herbert B. Enderton:
\emph{A Mathematical Introduction to Logic}.
Academic Press, New York 1972, 2. Auflage 2001.

\bibitem{Blackburn} Patrick Blackburn, Johan van Benthem:
\emph{Modal Logic: A Semantic Perspective}.
In: \emph{Studies in Logic and Practical Reasoning}.
Band 3, 2007, S. 1--84.

\bibitem{Oberschelp} Arnold Oberschelp:
\emph{Logik für Philosophen}. Metzler, Stuttgart 1997.

\bibitem{Oberschelp-Sorten} Arnold Oberschelp:
\emph{Untersuchungen zur mehrsortigen Quantorenlogik}.\\
In: \emph{Mathematische Annalen}. Band 145, 1962, S. 297--333.

\bibitem{Taylor} Paul Taylor:
\emph{Practical Foundations of Mathematics}.\\
Cambridge University Press, 1999.

\bibitem{Pierce} Benjamin C. Piercce:
\emph{Basic Category Theory for Computer Scientists}.
The MIT Press, Cambridge (Massachusetts) 1991.

\bibitem{Lambek-Scott} Joachim Lambek, Philip J. Scott:
\emph{Introduction to higher order categorical logic}.
Cambridge University Press, 1986.

\bibitem{Tabatabai} Amirhossein Akbar Tabatabai:
\emph{An Introduction to Categorical Proof Theory}.\\
August 2024. \href{https://arxiv.org/abs/2408.09488}{arxiv:2408.09488}.

\bibitem{Coquand-Type-Theory} Thierry Coquand:
\emph{Type Theory}. In: \emph{The Stanford Encyclopedia of Philosophy}.

\bibitem{HoTT}
\emph{Homotopy Type Theory. Univalent Foundations of Mathematics}.
The Univalent Foundations Program, 2013.

\bibitem{Paulin-Mohring} Christine Paulin-Mohring:
\emph{Introduction to the Calculus of Inductive Constructions}. In:
Bruno Woltzenlogel Paleo, David Delahaye: \emph{All about Proofs, Proofs for All}.
In: \emph{Studies in Logic}. Band 55, 2015.

\bibitem{Seldin} Jonathan P. Seldin:
\emph{Coquand's Calculus of Constructions: A Mathematical Foundation
for a Proof Development System}. In: \emph{Formal Aspects of Computing}.
Band 4, 1992, S. 425--441.
\href{https://doi.org/10.1007/BF01211392}{doi:10.1007/BF01211392}.

\bibitem{Deiser-Grundbegriffe}
Oliver Deiser:
\emph{\href{https://www.aleph1.info/?call=Puc&permalink=grundbegriffe}%
{Grundbegriffe der Mathematik}. Sprache, Zahlen und erste
Erkundungen}. März 2021, letzte Version Oktober 2022.

\bibitem{Cantor} Georg Cantor:
\emph{Beiträge zur Begründung der transfiniten Mengenlehre}.\\
In: \emph{Mathematische Annalen}. Band~46, 1895, S.~481.

\bibitem{Fraenkel} Abraham Adolf Fraenkel:
\emph{Einleitung in die Mengenlehre}.\\
Springer, Berlin 1919, 3. Auflage 1928.

\bibitem{Deiser} Oliver Deiser:
\emph{Einführung in die Mengenlehre}.
Springer, 2002, 3. Auflage 2010.

\bibitem{Ebbinghaus-Mengen} Heinz-Dieter Ebbinghaus:
\emph{Einführung in die Mengenlehre}.\\
Springer Spektrum, 5. Auflage 2021.

\bibitem{Enderton-sets} Herbert B. Enderton:
\emph{Elements of Set Theory}. Academic Press, New York 1977.

\bibitem{Shulman} Michael A. Shulman:
\emph{Set theory for category theory}.
Oktober 2008.\\
\href{https://arxiv.org/abs/0810.1279}{arxiv:0810.1279}.

\bibitem{Jech} Thomas Jech: \emph{Set Theory: The Third Millennium
Edition, revised and expanded}. Springer, 2002.
\href{https://doi.org/10.1007/3-540-44761-X}{doi:10.1007/3-540-44761-X}.

\bibitem{Siraphob} Ben Siraphob:
\href{https://siraben.dev/2021/06/27/classical-math-coq.html}{%
\emph{A non-trivial trivial theorem: doing classical mathematics in Coq}}.
Blogpost, 27. Juni 2021.

\bibitem{Stach} Stefan Müller-Stach:
\emph{Richard Dedekind: Was sind und was sollen die Zahlen?
Stetigkeit und Irrationale Zahlen}. Springer, 2017.

\bibitem{Lorenzen} Paul Lorenzen:
\emph{Die Definition durch vollständige Induktion}.
In: \emph{Monatshefte für Mathematik und Physik}.
Band~47, 1939, S.~356--358.

\bibitem{Halmos} Paul R. Halmos:
\emph{Naive Set Theory}. Springer, 1974.

\bibitem{Mainzer} Klaus Mainzer:
\emph{Natürliche, ganze und rationale Zahlen}.
In: Heinz-Dieter Ebbinghaus u. a.: \emph{Zahlen}.
Springer, 1983, 3. Auflage 1992.

\bibitem{Lamm} Christoph Lamm:
\emph{Karl Grandjot und der Dedekindsche Rekursionssatz}.\\
In: \emph{Mitteilungen der DMV}. Band~24, 2016, S.~37--45.

\bibitem{Glosauer} Tobias Glosauer:
\emph{Elementar(st)e Gruppentheorie}.
Springer, 2016.

\bibitem{Stanley} Richard P. Stanley:
\emph{Enumerative Combinatorics}. Band 1.\\
Cambridge University Press, 2012.

\bibitem{Henze} Norbert Henze: \emph{Stochastik für Einsteiger}.
Springer, 1997, 12. Auflage 2018.

\bibitem{Winskel} Glynn Winskel:
\emph{The Formal Semantics of Programming Languages}.
The MIT Press, Cambridge (Massachusetts) 1993.

\bibitem{Harel} David Harel, Dexter Kozen, Jerzy Tiuryn:
\emph{Dynamic Logic}.
The MIT Press, Cambridge (Massachusetts) 2000.

\bibitem{Hansson} Sven Ove Hansson (Ed.), Vincent F. Hendricks (Ed.):
\emph{Introduction to Formal Philosophy}. Springer, 2018.
\href{https://doi.org/10.1007/978-3-319-77434-3}{doi:10.1007/978-3-319-77434-3}.

\end{thebibliography}
