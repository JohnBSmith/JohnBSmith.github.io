
\chapter{Logisches Schließen}

\section{Grundbegriffe}

\subsection{Was ist eine Schlussregel?}

% Alles Schlussregeln und Axiome müssen unmittelbar einsichtig sein,
% müssen unbedenklich sein. Normalerweise will man in der Wissenschaft
% nicht so informell sein, aber hier sind wir am basalen Grund.
% Selbst in der Semantik wird der hier aufgestellte Formalismus
% benötigt. Der Korrektheitsbeweis würde somit zirkulär.

% Am Anfang ist noch nicht allzu viel zu rechnen. Wir müssen uns
% erst um die Schlussregeln und Axiome bemühen, bevor das erste Theorem
% bewiesen werden kann.

% Zur Syntax. Rangfolge der Junktoren. Abstrakte Syntaxbäume.
% Kurzschreibweisen, Verlockung nach Steno.

Logisches Schließen findet in einzelnen Schritten statt.
Ein Schritt stellt hierbei immer die Ableitung
einer Schlussfolgerung aus einer oder mehreren Voraussetzungen dar.
Die Voraussetzungen heißen \emph{Prämissen}, die Schlussfolgerung
\emph{Konklusion}. Darstellen wollen wir den Schritt durch eine
waagerechte Linie, wobei die Prämissen oberhalb befindlich sein sollen,
und die Konklusion unterhalb. Der Schritt
\[\dfrac{\text{wenn es regnet, wird die Straße nass}\qquad\text{es regnet}}
{\text{die Straße wird nass}}\]
beschreibt beispielsweise, dass aus den Prämissen »wenn es regnet, wird
die Straße nass« und »es regnet« die Konklusion »die Straße wird nass«
gefolgert wird.

Schlüsse wie der Obige treten in der Mathematik ständig auf. Ihnen allen
liegt ein bestimmtes Muster zugrunde, welches
sich durch eine als \emph{Modus ponens}
oder \emph{Abtrennungsregel} bezeichnete schematische \emph{Schlussregel}
beschreiben lässt. Es bezeichne hierzu $A\cond B$ die Implikation
»wenn $A$, dann $B$«. Es dürfen nun in
\[\dfrac{A\cond B\qquad A}{B}\]
für $A,B$ beliebige Aussagen eingesetzt werden. So darf »es regnet»
für $A$ und »die Straße wird nass« für $B$ eingesetzt werden.

\subsection{Sequenzen}

Das Schließen von Aussagen allein genügt nicht. Um freier argumentieren
zu können, würden wir gerne den Umstand beschreiben können, dass eine
Aussage unter bestimmten Annahmen abgeleitet werden konnte. Diese
Annahmen $A_k$ sind selbst Aussagen. Wir fassen sie zu einer endlichen
Ansammlung
\[\Gamma := [A_1,A_2,\ldots,A_n]\]
zusammen, worunter wir eine endliche Liste, oder auch eine endliche
Menge verstehen wollen, denn man soll mit dieser Liste umgehen können
wie mit einer Menge. Das heißt, es ist nicht von Bedeutung, wie
oft eine Aussage vorkommt oder in welcher Reihenfolge die Aussagen
stehen. Man bezeichnet $\Gamma$ als die \emph{Antezedenz} oder die
Liste der \emph{Antezedenzen}. Es wird $\Gamma$ auch der \emph{Kontext}
oder die \emph{Umgebung} genannt, das sind auf die Typentheorie
zurückzuführende Sprechweisen, die einen ganz ähnlichen Formalismus
besitzt. Wir bezeichnen die Symbolik \[\Gamma\vdash A\]
als \emph{Sequenz}. Sie drückt das \emph{Urteil} aus, dass die
Aussage $A$ aus den Annahmen vermittels Schlussregeln ableitbar ist.
Der Modus ponens wird nun in der allgemeinen Form
\[\dfrac{\Gamma\vdash A\cond B\qquad\Gamma\vdash A}{\Gamma\vdash B}\]
beschrieben. Wir argumentieren beim Schließen also ab jetzt nicht mehr
mit den Aussagen selbst, sondern mit den Sequenzen. Dies hat einen wichtigen
Grund, nämlich dass die Berücksichtigung der Abhängigkeit von Annahmen
expliziter Teil des Schließens wird.

Ein Kontext kann auch eine leere Liste sein.
Besitzt eine vermittels Schlussregeln ableitbare Sequenz einen
leeren Kontext, so bezeichnet man die Antezedenz als ein
\emph{Theorem} im engeren Sinne. Ein Theorem ist also eine
Aussage, die für sich allein gilt, ohne dass dafür irgendwelche
Annahmen getroffen werden müssen.

Für Sequenzen gilt die \emph{Abschwächungsregel}. Sie besagt, dass
falls die Aussage $A$ bereits aus $\Gamma$ ableitbar ist, diese
Aussage erst recht ableitbar ist, wenn zu $\Gamma$ weitere Annahmen
$\Gamma'$ hinzugefügt werden. Kurzum gilt die Regel
\[\dfrac{\Gamma\vdash A}{\Gamma,\Gamma'\vdash A}.\]
Hierbei bedeutet $\Gamma,\Gamma'$ die Konkatenation der Listen
$\Gamma$ und $\Gamma'$, also im Wesentlichen dasselbe wie die
Vereinigung $\Gamma\cup\Gamma'$, insofern man die Kontexte als
Mengen betrachtet.

\subsection{Zulässige Schlussregeln}

Wiewohl die Regeln des Schließens den Mechanismus zum Beweis
von Aussagen bilden, ist ihre Rolle sogar noch ein wenig tiefgreifender.
Wir können sie nämlich ebenfalls zur Ableitung \emph{weiterer Regeln}
nutzen. Das heißt, wir können sie dazu nutzen, den logischen Kalkül
selbst zu erweitern. Erweiterungen dieser Art nennen wir
\emph{zulässige Schlussregeln}.

Mit den bisherigen Regeln ist bereits die zulässige Regel
\[\dfrac{\Gamma\vdash A\cond B\qquad\Gamma'\vdash A}
{\Gamma,\Gamma'\vdash B}\]
ableitbar, die eine allgemeinere Form des Modus ponens darstellt. Man
erhält sie kurzerhand, indem den Prämissen des Modus
ponens jeweils die Abschwächungsregel vorgesetzt wird:
\[\infer{\Gamma,\Gamma'\vdash B}{
  \infer{\Gamma,\Gamma'\vdash A\cond B}{\Gamma\vdash A\cond B}
& \infer{\Gamma,\Gamma'\vdash A}{\Gamma'\vdash A}}
\]
Die einfache Form des Modus ponens erhält man mit $\Gamma':=\Gamma$ als
Spezialfall unter Anwendung der Kontraktionsregel.

\subsection{Implikationseinführung}

Ich möchte mich nun der Frage zuwenden, wie eine Implikation $A\cond B$
bewiesen wird. Intuitiv ist hierzu aus der Annahme $A$ die
Aussage $B$ zu folgern. Das heißt, es genügt die Ableitung
der Sequenz $A\vdash B$. Ein weiteres Mal gilt es noch zu
berücksichtigen, dass ein Beweis auch auf einen vorausgesetzten
Kontext $\Gamma$ beschränkt sein dürfen soll. Reflektiert man darüber
eine Weile, dürfte es der Überlegung nach wohl genügen, dass $A$
einfach dem Kontext $\Gamma$ hinzugefügt wird, woraus $B$ zu folgern
ist. Man gelangt zur Regel
\[\dfrac{\Gamma,A\vdash B}{\Gamma\vdash A\cond B}.\]
Wer diese Regel nicht so leicht fassbar findet, insbesondere nicht
direkt plausibel, ob sie bedenkenlos eingesetzt werden darf, der
ist nicht allein. Es gibt auch logische Kalküle, die diese Regel nicht
explizit enthalten. Sie tritt dennoch als \emph{Deduktionstheorem} in
Erscheinung, ein metalogisches Theorem, dessen Beweis erst erbracht
werden muss. Ich möchte diesen Weg allerdings aus einem bestimmten Grund
nicht gehen. Nämlich ist beim Beweis eigentlich natürliches Schließen
auf der metalogischen Ebene zu verwenden, wenn dies auch in informaler
Weise stattfinden mag. Aber nicht jeder Leser weiß zu diesem Zeitpunkt,
wie akkurates logisches Schließen geht. Der Leser benötigt am Anfang
etwas, um sich an den eigenen Haaren aus dem Sumpf zu ziehen.

\subsection{Axiome}

Zur Komplettierung des Kalküls gesellen sich schließlich auch noch
\emph{Axiome} hinzu, das sind gemachte Grundannahmen, die nicht weiter
bewiesen werden müssen. Sie sollten daher möglichst plausibel,
oder besser noch zweifelsfrei einsichtig sein. Für die Logik selbst
genügt das Axiom
\[A\vdash A.\]
Der Kalkül funktioniert dergestalt, dass für $A$ eine beliebige Aussage
eingesetzt werden darf, worunter auch zusammengesetzte Aussagen
wie fallen. Eine gern gewählter Weg der Definition des logischen
Kalküls sieht $A$ als eine metasprachliche Variable, für die eine
beliebige Formel eingesetzt werden darf. Unter dieser Sichtweise
spricht man von einem \emph{Axiomenschema}. Wie eine Schablone
produziert es für jede Einsetzung einer konkreten logischen Formel ein
eigenes Axiom.

Anstelle $A,B,C$ werden für metasprachliche Variablen zuweilen auch
die griechischen Buchstaben $\varphi,\psi,\chi$ benutzt. Man muss sie
von atomaren logischen Variablen unterscheiden, für die wir in diesem
Buch, um Missverständnissen aus dem Weg zu gehen, kleine Buchstaben
$a,b,c$ oder $p,q,r$ verwenden werden. Sprachlich suggestiv steht
$\varphi$ für \emph{Formel} oder \emph{formula}, $a$ für
\emph{Aussage} und $p$ für \emph{proposition}.

In diesem Sinne sind auch die Schlussregeln Schemata. Sofern man
Schlussregeln mit null Prämissen gestattet, lässt sich das
Axiomenschema auch als Regel
\[\dfrac{}{A\vdash A}\]
auffassen. In dieser Weise wollen wir die Anwendung von Axiomen in den
Beweisbäumen darstellen.

Axiome in der Form von Sequenzen heißen auch \emph{Grundsequenzen}.

Wir haben nun die Mittel in der Hand, um erste Theoreme beweisen
zu können. Es ist $A\cond A$ ein Theorem. Der Beweisbaum ist:
\[
\infer[\infernote{Implikationseinführung}]{\vdash A\cond A}{
  \infer[\infernote{Axiom}]{A\vdash A}{}}
\]
Unter der Lesung, dass $A$ eine Metavariable ist, handelt es
eigentlich nicht nur um ein Theorem, sondern um ein Schema von
Theoremen. Setzt man für $A$ bspw. die konkrete Formel $p\cond q$
ein, bekommt man das konkrete Theorem
\[\vdash (p\cond q)\cond (p\cond q).\]
% Es ist zu bemerken, dass die Unterscheidung zwischen Metavariablen
% und atomaren logischen Variablen später durch die Einsetzungsregel
% mehr oder weniger hinfällig wird. Hierbei handelt es sich aber um eine
% höhere Überlegung, deren Beweis der Logiker erbringen will. In der
% Wissenschaft, insbesondere in der Logik, will man den Dingen auf den
% Grund gehen, will alles genau auseinandernehmen. Da möchte man bestimmte
% Regeln nicht einfach so als gegeben voraussetzen.

\subsection{Junktoren}

Bisher traten zusammengesetzte Aussagen alleinig in Form einer
Implikation auf. Will man logische Zusammenhänge beschreiben können,
muss die logische Sprache um weitere Junktoren bereichert werden.
Unter einem \emph{Junktor} versteht man eine logische Verknüpfung von
Aussagen zu einer zusammengesetzten Aussage.

Wir werden einen Junktor durch \emph{Einführungsregeln} und
\emph{Beseitigungsregeln} charakterisieren. Die Regeln der Implikation
wurden bereits beschrieben; die Einführung geschieht per
Implikationseinführung, die Beseitigung per Modus ponens.
Für die restlichen Junktoren der Aussagenlogik lassen sich die Regeln
wahlweise in Form von Axiomenschemata oder in Form von Schlussregeln
darstellen. Ich möchte das per Schemata machen, weil diese ein wenig
kompakter sind, was sie hoffentlich ein wenig leichter einsichtig macht.
Die entsprechenden Schlussregeln leiten wir anschließend als zulässige
Regeln ab.

Die Konjunktion $A\land B$, auch Und"=Verknüpfung genannt,
sprich »$A$ und $B$«, ist charakterisiert durch die Sequenzen
\[A,B\vdash A\land B;\qquad A\land B\vdash A;\qquad A\land B\vdash B.\]
Aus dem Fall von sowohl Regen als auch Schnee ist der Fall von
Schneeregen ableitbar. Aus dem Fall von Schneeregen ist der Fall
von Regen ableitbar. Aus dem Fall von Schneeregen ist der Fall
von Schnee ableitbar. So sind diese Sequenzen zu verstehen.

Die Einführung der Konjunktion geschieht mit der Regel
\[\dfrac{\Gamma\vdash A\qquad\Gamma'\vdash B}{\Gamma,\Gamma'\vdash A\land B}.\]
Denn es findet sich der Beweisbaum:
\[
\infer[\infernote{MP}]{\Gamma,\Gamma'\vdash A\land B}{
  \infer[\infernote{MP}]{\Gamma\vdash B\cond A\land B}{
    \infer[\infernote{Impl-Einf.}]{\vdash A\cond (B\cond A\land B)}{
      \infer[\infernote{Impl-Einf.}]{A\vdash B\cond A\land B}{
        \infer[\infernote{Axiom}]{A,B\vdash A\land B}{}}}
  & \Gamma\vdash A}
& \Gamma'\vdash B}
\]
Es steht MP als Abkürzung für Modus ponens, und Impl-Einf. für
Implikationseinführung. Man schreibt alternativ auch das Kürzel
$\cond$E anstelle Impl-Einf. und das Kürzel $\cond$B anstelle
von MP. Hierbei steht E offenkundig für \emph{Einführung} und
B für \emph{Beseitigung}. Aber Vorsicht, in der englischsprachigen
Literatur sind das I für \emph{introduction} und E für
\emph{elimination}.


