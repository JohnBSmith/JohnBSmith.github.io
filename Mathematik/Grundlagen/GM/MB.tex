
\chapter{Maschinengestütztes Beweisen}

\section{Terme und Typen}

\subsection{Zur Aussagenlogik}

\begin{table}
\begin{center}
\caption{Funktionen für die Konjunktion und die Disjunktion}
\label{tab:Funktionen-Konjunktion-Disjunktion}
\begin{tabular}{@{}cll@{}}
\toprule
\textbf{Junktor} & \textbf{Einführung} & \textbf{Beseitigung}\\
\midrule[\heavyrulewidth]
Konjunktion & $\code{conj}\,A\;B\colon A \to B \to A\land B$
& $\code{proj1}\;A\;B\colon A\land B\to A$\\
& & $\code{proj2}\;A\;B\colon A\land B\to B$\\
\midrule[\heavyrulewidth]
Disjunktion & $\code{or\_introl}\,A\;B\colon A\to A\lor B$
& $\code{or\_elim}\,A\;B\colon (A\to C)\to (A\to C)$\\
& $\code{or\_intror}\,A\;B\colon A\to B\lor A$
& $\phantom{\code{or\_elim}\,A\;B\colon}\to (A\lor B\to C)$\\
\bottomrule
\end{tabular}
\end{center}
\end{table}

Tabelle \ref{tab:Funktionen-Konjunktion-Disjunktion} zeigt die Funktionen,
mit denen Terme für die Konjunktion und die Disjunktion konstruiert
werden können.

Ein kurzes Beispiel. Mit der Herleitung
\[\infer{\Gamma\vdash B\land A}{
  \infer{\Gamma\vdash B}{\Gamma\vdash A\land B}
& \infer{\Gamma\vdash A}{\Gamma\vdash A\land B}}
\;\rightsquigarrow\quad
\infer{\Gamma\vdash\code{conj} (\code{proj2}(h)) (\code{proj1}(h))\colon B\land A}{
  \infer{\Gamma\vdash\code{proj2}(h)\colon B}{\Gamma\vdash h\colon A\land B}
& \infer{\Gamma\vdash\code{proj1}(h)\colon A}{\Gamma\vdash h\colon A\land B}}\]
kommt man bezüglich $\Gamma:=[h\colon A\land B]$ zum Term
\[(h\mapsto\code{conj} (\code{proj2}(h)) (\code{proj1}(h)))\colon A\land B\to B\land A.\]
Der Quelltext hierzu:

\noindent
\begin{lstlisting}[escapechar=`]
`\kw{Theorem}` conj_commutes A B:
  A `$\land$` B `$\to$` B `$\land$` A.
`\kw{Proof.}`
  exact (`\kw{fun}` h => conj (proj2 h) (proj1 h)).
`\kw{Qed.}`
\end{lstlisting}

\section{Taktiken}

Statt Terme zu schreiben, nutzt man lieber Taktiken. Sie bieten den
Vorteil, dass die Schlussregeln mit ihnen rückwerts angewendet werden
können. Man gibt dabei das zu beweisende Theorem als \emph{Ziel}
vor. Vermittels den Taktiken, die zu den Schlussregeln des natürlichen
Schließens gehören, kann das Ziel auf ein oder mehrere
\emph{Unterziele} zurückgeführt werden. Das geht bis zum Erreichen
von Grundsequenzen so weiter.
