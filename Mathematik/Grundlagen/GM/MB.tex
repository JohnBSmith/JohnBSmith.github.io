
\chapter{Maschinengestütztes Beweisen}

\section{Terme und Typen}

\subsection{Zur Aussagenlogik}

Der Einführung einer Subjunktion\index{Subjunktion} entspricht die
Einführung einer $\lambda$"=Abstraktion,\index{Abstraktion}%
\index{Lambda-Abstraktion} also einer anonymen Funktion.%
\index{anonyme Funktion} Der aus der Herleitung
\[\begin{prooftree}
    \infer0{A,B\vdash A}
  \infer1{A\vdash B\cond A}
\infer1{\vdash A\cond B\cond A}
\end{prooftree}
\;\rightsquigarrow\;\;
\begin{prooftree}
    \infer0{\Gamma, a\colon A, b\colon B\vdash a\colon A}
  \infer1{\Gamma, a\colon A\vdash (b\mapsto a)\colon B\to A}
\infer1{\Gamma\vdash a\mapsto (b\mapsto a)\colon A\to B\to A}
\end{prooftree}\]
mit $\Gamma:=[A\colon\mathrm{Prop}, B\colon\mathrm{Prop}]$ entstandene
Term wird zum Beispiel abgefasst als
\begin{lstlisting}[escapechar=`, xleftmargin=\mathindent]
`\kw{Theorem}` Beispiel1 (A B: Prop):
  A -> B -> A.
`\kw{Proof}`
  `\kw{fun}` a => `\kw{fun}` b => a.
\end{lstlisting}
Der Beseitigung der Subjunktion entspricht die Applikation einer
Funktion. Betrachten wir dazu die Herleitung
\[\begin{prooftree}
      \infer0{A\cond B\vdash A\cond B}
      \infer0{A\vdash A}
    \infer2{A,A\cond B\vdash B}
  \infer1{A\vdash (A\cond B)\cond B}
\infer1{\vdash A\cond (A\cond B)\cond B}
\end{prooftree}
\;\rightsquigarrow\;\;
\begin{prooftree}
      \infer0{\Gamma,f\colon A\to B\vdash f\colon A\to B}
      \infer0{\Gamma,a\colon A\vdash a\colon A}
    \infer2{\Gamma,a\colon A, f\colon A\to B\vdash f(a)\colon B}
  \infer1{\Gamma,a\colon A\vdash (f\mapsto f(a))\colon (A\to B)\to B}
\infer1{\Gamma\vdash (a\mapsto f\mapsto f(a))\colon A\to (A\to B)\to B}
\end{prooftree}\]
mit $\Gamma:=[A\colon\mathrm{Prop}, B\colon\mathrm{Prop}]$.
Der Beweis wird demnach implementiert als
\begin{lstlisting}[escapechar=`, xleftmargin=\mathindent]
`\kw{Theorem}` Beispiel2 (A B: Prop):
  A -> (A -> B) -> B.
`\kw{Proof}`
  `\kw{fun}` a => `\kw{fun}` f => f a.
\end{lstlisting}

\begin{table}
\begin{center}
\caption{Funktionen für die Konjunktion und die Disjunktion}
\label{tab:Funktionen-Konjunktion-Disjunktion}
\begin{tabular}{@{}cll@{}}
\toprule
\textbf{Junktor} & \textbf{Einführung} & \textbf{Beseitigung}\\
\midrule[\heavyrulewidth]
Konjunktion & $\code{conj}\,A\;B\colon A \to B \to A\land B$
& $\code{proj1}\;A\;B\colon A\land B\to A$\\
& & $\code{proj2}\;A\;B\colon A\land B\to B$\\
\midrule[\heavyrulewidth]
Disjunktion & $\code{or\_introl}\,A\;B\colon A\to A\lor B$
& $\code{or\_elim}\,A\;B\colon (A\to C)\to (A\to C)$\\
& $\code{or\_intror}\,A\;B\colon A\to B\lor A$
& $\phantom{\code{or\_elim}\,A\;B\colon}\to (A\lor B\to C)$\\
\bottomrule
\end{tabular}
\end{center}
\end{table}

\noindent
Tabelle \ref{tab:Funktionen-Konjunktion-Disjunktion} zeigt die Funktionen,
mit denen Terme für die Konjunktion und die Disjunktion konstruiert
werden können.\index{Konjunktion}\index{Disjunktion}

Ein kurzes Beispiel. Mit der Herleitung
\[\begin{prooftree}
    \hypo{\Gamma\vdash A\land B}
  \infer1{\Gamma\vdash B}
    \hypo{\Gamma\vdash A\land B}
  \infer1{\Gamma\vdash A}
\infer2{\Gamma\vdash B\land A}
\end{prooftree}
\;\rightsquigarrow\;\;
\begin{prooftree}
    \hypo{\Gamma\vdash h\colon A\land B}
  \infer1{\Gamma\vdash\code{proj2}(h)\colon B}
    \hypo{\Gamma\vdash h\colon A\land B}
  \infer1{\Gamma\vdash\code{proj1}(h)\colon A}
\infer2{\Gamma\vdash\code{conj} (\code{proj2}(h)) (\code{proj1}(h))\colon B\land A}
\end{prooftree}\]
kommt man bezüglich $\Gamma:=[h\colon A\land B]$ zum Term
\[(h\mapsto\code{conj} (\code{proj2}(h)) (\code{proj1}(h)))\colon A\land B\to B\land A.\]
Der Quelltext hierzu:

\noindent
\begin{lstlisting}[escapechar=`, xleftmargin=\mathindent]
`\kw{Theorem}` conj_commutativity (A B: Prop):
  A /\ B -> B /\ A.
`\kw{Proof}`
  `\kw{fun}` h => conj (proj2 h) (proj1 h).
\end{lstlisting}
Dieser Beweis ist allerdings für \texttt{Prop} spezifisch. Für Typen
allgemeiner Art findet sich aber ein analoger Beweis. Der
Curry"=Howard"=Korrespondenz entsprechend betrifft dies die Produkttypen
$A\times B$, die \verb|prod A B| oder \verb|A*B| geschrieben werden. Die
zweite Schreibweise wird aber nur gestattet, wenn keine Zweideutigkeit
mit der Multiplikation von Zahlen besteht. Die Analoga zu
\texttt{proj1}, \texttt{proj2} und \texttt{conj} sind
\texttt{fst}, \texttt{scd} und \texttt{pair}, wobei dies für
first, second steht und man \verb|(x, y)| statt \verb|pair x y|
schreiben darf. Es ergibt sich also der folgende Quelltext:
\begin{lstlisting}[escapechar=`, xleftmargin=\mathindent]
`\kw{Theorem}` prod_commutativity (A B: Type):
  A * B -> B * A.
`\kw{Proof}`
  `\kw{fun}` h => (snd h, fst h).
\end{lstlisting}
Statt die Projektionen zu nutzen, kann der Term auch via
Musterabgleich als
\begin{lstlisting}[escapechar=`, xleftmargin=\mathindent]
`\kw{fun}` h => `\kw{match}` h `\kw{with}` (a, b) => (b, a) `\kw{end}`
\end{lstlisting}
abgefasst werden, wobei hierfür zudem die Kurzschreibweise
\begin{lstlisting}[escapechar=`, xleftmargin=\mathindent]
`\kw{fun}` '(a, b) => (b, a) `\textrm{bzw.}` `\kw{fun}` '(pair a b) => (pair b a)
\end{lstlisting}
existiert.

Für den Beweis von $A\lor B\cond B\lor A$ findet sich
\begin{lstlisting}[escapechar=`, xleftmargin=\mathindent]
`\kw{Theorem}` disj_commutativity (A B: Prop):
  A \/ B -> B \/ A.
`\kw{Proof}`
  or_elim
    (`\kw{fun}` a => or_intror a)
    (`\kw{fun}` b => or_introl b).
\end{lstlisting}
Alternativ wird der Term via Musterabgleich abgefasst:
\begin{lstlisting}[escapechar=`, xleftmargin=\mathindent]
`\kw{fun}` h => `\kw{match}` h `\kw{with}`
| or_introl a => or_intror a
| or_intror b => or_introl b
`\kw{end}`
\end{lstlisting}

\section{Taktiken}

Statt Terme zu schreiben, nutzt man lieber Taktiken. Sie bieten den
Vorteil, dass die Schlussregeln mit ihnen rückwerts angewendet werden
können. Man gibt dabei das zu beweisende Theorem als \emph{Ziel}\index{Ziel}
vor. Vermittels den Taktiken, die zu den Schlussregeln des natürlichen
Schließens gehören, kann das Ziel auf ein oder mehrere \emph{Unterziele}%
\index{Unterziele} zurückgeführt werden. Das geht bis zum Erreichen
von Grundsequenzen so weiter.

Mit $\Gamma:=[A\colon\mathrm{Prop},B\colon\mathrm{Prop}]$ gilt
\[\begin{prooftree}
        \infer0{\Gamma,g\colon\lnot B\vdash g\colon\lnot B}
          \infer0{\Gamma,f\colon A\to B\vdash f\colon A\to B}
          \infer0[exact $a$]{\Gamma,a\colon A\vdash a\colon A}
        \infer2[apply $f$]{\Gamma,f\colon A\to B, a\colon A\vdash f(a)\colon B}
      \infer2[apply $g$]{\Gamma, f\colon A\to B, g\colon\lnot B, a\colon A\vdash g(f(a))\colon\bot}
    \infer1[intro $a$]{\Gamma, f\colon A\to B, g\colon\lnot B\vdash a\mapsto g(f(a))\colon \lnot A}
  \infer1[intro $g$]{\Gamma, f\colon A\to B\vdash g\mapsto a\mapsto g(f(a))\colon \lnot B\to\lnot A}
\infer1[intro $f${\normalsize .}]{\Gamma\vdash f\mapsto g\mapsto a\mapsto g(f(a))\colon (A\to B)\to (\lnot B\to\lnot A)}
\end{prooftree}\]
Man beachte, dass der konstruierte Term $f\mapsto g\mapsto a\mapsto g(f(a))$ am Anfang noch
unbekannt ist. Bekannt sind im jeweiligen Schritt lediglich die zur
Verfügung stehenden Variablen und der Typ, dessen Term zu konstruieren
ist. Erst nachdem man sich rückwärts von der Wurzel aus zu den
Blättern hin durchgearbeitet hat, kann man den Weg zur Wurzel hin
zurücklaufen und dabei schrittweise den Terme konstruieren.
In der Praxis gibt man sich in der Regel zufrieden, bei den Blättern
angekommen zu sein. Der tatsächliche Verlauf sieht also so aus:
\[\begin{prooftree}
        \infer0{\Gamma,g\colon\lnot B\vdash g\colon\lnot B}
          \infer0{\Gamma,f\colon A\to B\vdash f\colon A\to B}
          \infer0[exact $a$]{\Gamma,a\colon A\vdash {?}\colon A}
        \infer2[apply $f$]{\Gamma,f\colon A\to B, a\colon A\vdash {?}\colon B}
      \infer2[apply $g$]{\Gamma, f\colon A\to B, g\colon\lnot B, a\colon A\vdash {?}\colon\bot}
    \infer1[intro $a$]{\Gamma, f\colon A\to B, g\colon\lnot B\vdash {?}\colon \lnot A}
  \infer1[intro $g$]{\Gamma, f\colon A\to B\vdash {?}\colon \lnot B\to\lnot A}
\infer1[intro $f$]{\Gamma\vdash {?}\colon (A\to B)\to (\lnot B\to\lnot A)}
\end{prooftree}\]
Der Quelltext hierzu:
\begin{lstlisting}[escapechar=`, xleftmargin=\mathindent]
`\kw{Theorem}` contraposition (A B: Prop):
  (A -> B) -> (~B -> ~A).
`\kw{Proof.}`
  intro f. intro g. intro a.
  apply g. apply f. exact a.
`\kw{Qed.}`
\end{lstlisting}
Die Entwicklungsumgebung ermöglicht es hierbei, auf dem Verlauf der
Taktiken zu wandern. Zwischen den Schritten wird das aktuelle Ziel und
dessen Kontext gezeigt, so dass man die Übersicht darüber behält,
was zu beweisen verbleibt und welche Mittel dafür verfügbar sind.
