
\chapter{Maschinengestütztes Beweisen}

\section{Terme und Typen}

\subsection{Zur Aussagenlogik}

\begin{table}
\begin{center}
\caption{Funktionen für die Konjunktion und die Disjunktion}
\label{tab:Funktionen-Konjunktion-Disjunktion}
\begin{tabular}{@{}cll@{}}
\toprule
\textbf{Junktor} & \textbf{Einführung} & \textbf{Beseitigung}\\
\midrule[\heavyrulewidth]
Konjunktion & $\code{conj}\,A\;B\colon A \to B \to A\land B$
& $\code{proj1}\;A\;B\colon A\land B\to A$\\
& & $\code{proj2}\;A\;B\colon A\land B\to B$\\
\midrule[\heavyrulewidth]
Disjunktion & $\code{or\_introl}\,A\;B\colon A\to A\lor B$
& $\code{or\_elim}\,A\;B\colon (A\to C)\to (A\to C)$\\
& $\code{or\_intror}\,A\;B\colon A\to B\lor A$
& $\phantom{\code{or\_elim}\,A\;B\colon}\to (A\lor B\to C)$\\
\bottomrule
\end{tabular}
\end{center}
\end{table}

Tabelle \ref{tab:Funktionen-Konjunktion-Disjunktion} zeigt die Funktionen,
mit denen Terme für die Konjunktion und die Disjunktion konstruiert
werden können.

Ein kurzes Beispiel. Mit der Herleitung
\[\begin{prooftree}
    \hypo{\Gamma\vdash A\land B}
  \infer1{\Gamma\vdash B}
    \hypo{\Gamma\vdash A\land B}
  \infer1{\Gamma\vdash A}
\infer2{\Gamma\vdash B\land A}
\end{prooftree}
\;\rightsquigarrow\;\;
\begin{prooftree}
    \hypo{\Gamma\vdash h\colon A\land B}
  \infer1{\Gamma\vdash\code{proj2}(h)\colon B}
    \hypo{\Gamma\vdash h\colon A\land B}
  \infer1{\Gamma\vdash\code{proj1}(h)\colon A}
\infer2{\Gamma\vdash\code{conj} (\code{proj2}(h)) (\code{proj1}(h))\colon B\land A}
\end{prooftree}\]
kommt man bezüglich $\Gamma:=[h\colon A\land B]$ zum Term
\[(h\mapsto\code{conj} (\code{proj2}(h)) (\code{proj1}(h)))\colon A\land B\to B\land A.\]
Der Quelltext hierzu:

\noindent
\begin{lstlisting}[escapechar=`]
`\kw{Theorem}` conj_commutes A B:
  A /\ B -> B /\ A.
`\kw{Proof.}`
  exact (`\kw{fun}` h => conj (proj2 h) (proj1 h)).
`\kw{Qed.}`
\end{lstlisting}

\section{Taktiken}

Statt Terme zu schreiben, nutzt man lieber Taktiken. Sie bieten den
Vorteil, dass die Schlussregeln mit ihnen rückwerts angewendet werden
können. Man gibt dabei das zu beweisende Theorem als \emph{Ziel}
vor. Vermittels den Taktiken, die zu den Schlussregeln des natürlichen
Schließens gehören, kann das Ziel auf ein oder mehrere
\emph{Unterziele} zurückgeführt werden. Das geht bis zum Erreichen
von Grundsequenzen so weiter.

Mit $\Gamma:=[A\colon\mathrm{Prop},B\colon\mathrm{Prop}]$ gilt
\[\begin{prooftree}
        \infer0{\Gamma,g\colon\lnot B\vdash g\colon\lnot B}
          \infer0{\Gamma,f\colon A\to B\vdash f\colon A\to B}
          \infer0[exact $a$]{\Gamma,a\colon A\vdash a\colon A}
        \infer2[apply $f$]{\Gamma,f\colon A\to B, a\colon A\vdash f(a)\colon B}
      \infer2[apply $g$]{\Gamma, f\colon A\to B, g\colon\lnot B, a\colon A\vdash g(f(a))\colon\bot}
    \infer1[intro $a$]{\Gamma, f\colon A\to B, g\colon\lnot B\vdash a\mapsto g(f(a))\colon \lnot A}
  \infer1[intro $g$]{\Gamma, f\colon A\to B\vdash g\mapsto a\mapsto g(f(a))\colon \lnot B\to\lnot A}
\infer1[intro $f${\normalsize .}]{\Gamma\vdash f\mapsto g\mapsto a\mapsto g(f(a))\colon (A\to B)\to (\lnot B\to\lnot A)}
\end{prooftree}\]
Man beachte, dass der konstruierte Term $f\mapsto g\mapsto a\mapsto g(f(a))$ am Anfang noch
unbekannt ist. Bekannt sind im jeweiligen Schritt lediglich die zur
Verfügung stehenden Variablen und der Typ, dessen Term zu konstruieren
ist. Erst nachdem man sich rückwärts von der Wurzel aus zu den
Blättern hin durchgearbeitet hat, kann man den Weg zur Wurzel hin
zurücklaufen und dabei schrittweise den Terme konstruieren.
In der Praxis gibt man sich in der Regel zufrieden, bei den Blättern
angekommen zu sein. Der tatsächliche Verlauf sieht also so aus:
\[\begin{prooftree}
        \infer0{\Gamma,g\colon\lnot B\vdash g\colon\lnot B}
          \infer0{\Gamma,f\colon A\to B\vdash f\colon A\to B}
          \infer0[exact $a$]{\Gamma,a\colon A\vdash {?}\colon A}
        \infer2[apply $f$]{\Gamma,f\colon A\to B, a\colon A\vdash {?}\colon B}
      \infer2[apply $g$]{\Gamma, f\colon A\to B, g\colon\lnot B, a\colon A\vdash {?}\colon\bot}
    \infer1[intro $a$]{\Gamma, f\colon A\to B, g\colon\lnot B\vdash {?}\colon \lnot A}
  \infer1[intro $g$]{\Gamma, f\colon A\to B\vdash {?}\colon \lnot B\to\lnot A}
\infer1[intro $f$]{\Gamma\vdash {?}\colon (A\to B)\to (\lnot B\to\lnot A)}
\end{prooftree}\]
Der Quelltext hierzu:
\begin{lstlisting}[escapechar=`]
`\kw{Theorem}` contraposition (A B: Prop):
  (A -> B) -> (~B -> ~A).
`\kw{Proof.}`
  intro f. intro g. intro a.
  apply g. apply f. exact a.
`\kw{Qed.}`
\end{lstlisting}
