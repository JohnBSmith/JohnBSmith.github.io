
\chapter{Kategorielle Blicke auf die Logik}

\section{Grundbegriffe}

\subsection{Kategorien}

Logisches Schließen weist enge Bezüge zu Typentheorien und zu
Konstruktionen in der Mengenlehre auf. Im Folgenden wird eine
Abstraktion erklärt, die bei der tieferen Klärung dieser
Bezüge Hilfe leistet.

\begin{Definition}[Kategorie]
Eine \emph{Kategorie} ist ein Tripel $\category C = (\Ob,\Hom,{\circ})$,
sofern die folgenden beiden Axiome erfüllt sind:
\begin{enumerate}
\item Für $f\colon A\to B$, $g\colon B\to C$, $h\colon C\to D$ gilt
das Assoziativgesetz\\
$h\circ (g\circ f) = (h\circ g)\circ f$.
\item Für jedes Objekt $X$ existiert die Identität $\id_X\colon X\to X$,
so dass $f\circ\id_A = \id_B\circ f=f$ für alle Objekte $A,B$
und $f\colon A\to B$.
\end{enumerate}
\end{Definition}
Die Elemente der Klasse $\Ob$ nennt man \emph{Objekte}. Die Elemente der
Klasse $\Hom$ nennt man \emph{Morphismen}.  Die Schreibweise
$f\colon X\to Y$ ist gleichbedeutend mit $f\in\Hom(X,Y)$, wobei $X,Y\in\Ob$.
Mit $\Hom(X,Y)$ ist die Teilklasse von $\Hom$ gemeint, die alle
Morphismen von $X$ nach $Y$ enthält. Man schreibt $\dom(f) = X$ und
$\cod(f) = Y$. Die Verknüpfung $g\circ f$, gelesen »$g$ nach $f$«, ist
definiert für $\cod(f)=\dom(g)$. Man nennt sie die \emph{Verkettung}
von $g$ und $f$.

Nun gut, man macht hier zunächst zwei Beobachtungen. Erstens erinnern
die Axiome an die Regeln für die Verkettung von Abbildungen.
In der Tat bilden die Mengen mit den Abbildungen als Morphismen
eine Kategorie.

Zweitens erinnern die Axiome an die Monoid"=Axiome, haben aber den
Unterschied, dass die Morphismen, die verkettet werden sollen, kompatibel
sein müssen. Wie sich unschwer herausstellt, bilden die Monoide genau
die Kategorien, die ein einziges Objekt besitzen. Die Morphismen werden
dabei als die Elemente des Objektes $X$ gedeutet, die Verkettung als deren
Verknüpfung. Zwei Morphismen $f,g$ müssen hierbei zwingend kompatibel sein,
denn $\cod(f)=X$ und $\dom(g)=X$. Hiermit liegt ein erstes Beispiel vor,
in dem die Morphismen nicht notwendigerweise Abbildungen darstellen.

\begin{Satz}[Kategorie der Mengen]\newlinefirst
Sei $\Omega$ das Mengenuniversum und für $A,B\in\Omega$ sei
$\Hom(A,B):=\Abb(A,B)$. Sei $g\circ f$ die Verkettung
von Abbildungen. Dann bildet $\category{Set}:=(\Omega,\Hom,\circ)$
eine Kategorie.
\end{Satz}
\strong{Beweis.} Trivial.\;\qedsymbol

\begin{Satz}[Kategorie der Gruppen]\newlinefirst
Sei $\Omega$ die Klasse aller Gruppen und für $G,H\in\Omega$ sei
$\Hom(G,H)$ die Klasse der Homomorphismen von $G$ nach $H$.
Sei $g\circ f$ die Verkettung von Homomorphismen.
Dann bildet $\category{Group}:=(\Omega,\Hom,\circ)$
eine Kategorie.
\end{Satz}
\strong{Beweis.} Homomorphismen sind Abbildungen, die Axiome
daher wie bei der Kategorie der Mengen erfüllt. Die Verkettung
zweier Homomorphismen ist ja auch ein Homomorphismus.\;\qedsymbol

Entsprechend bilden Ringe mit Ringhomomorphismen, Körper mit
Körperhomomorphismen, Vektorräume mit Vektorraumhomomorphismen
usw. Kategorien. Des Weiteren bilden die endlichen Mengen, Gruppen,
Ringe jeweils eine Kategorie.

\subsection{Funktoren}

Nun ist es so, dass Gruppen auch Mengen und Homomorphismen
auch Abbildungen sind. Die Kategorie der Gruppen ist gewissermaßen
in der Kategorie der Mengen enthalten. Um das zu präzisieren,
benötigen wir den Begriff des Vergissfunktors.

\begin{Definition}[Kovarianter Funktor]\newlinefirst
Sind $\category C,\category D$ Kategorien, dann nennt man
$F\colon\category C\to\category D$ einen
\emph{kovarianten Funktor}, wenn jedem Objekt $X$ von $\category C$ ein Objekt
$F(X)$ von $\category D$ zugeordnet wird und jedem Morphismus
$f\in\Hom_{\category C}(X,Y)$ ein ein Morphismus
$F(f)\in\Hom_{\category D}(F(X),F(Y))$ zugeordnet wird,
so dass die folgenden beiden Verträglichkeitsaxiome erfüllt sind:%
\begin{gather*}
F(g\circ f) = F(g)\circ F(f),\\
F(\id_X) = \id_{F(X)}.
\end{gather*}
\end{Definition}
\begin{Definition}[Kontravarianter Funktor]\mbox{}\\*
Wie beim kovarianten Funktor, mit dem Unterschied
$F(g\circ f) = F(f)\circ F(g)$.
\end{Definition}
\strong{Bemerkung.} Die Notation ist überladen. Nämlich ist die Zuordnung
$F\colon\Ob(\category C)\to\Ob(\category D)$ zu unterscheiden
von
\[\tilde F\colon\Hom_{\category C}(X,Y)\to\Hom_{\category D}(F(X),F(Y)).\]
Das Paar $(F,\tilde F)$ kodiert dann eigentlich den Funktor
$\category C\to\category D$.

\begin{Satz}[Vergissfunktor]\newlinefirst
Sei $F\colon\category{Group}\to\category{Set}$ mit $F((G,*,e)):=G$,
und jedem Gruppenhomomorphismus%
\[\varphi\colon (G,*,e)\to (G',*',e')\]
sei die Abbildung $F(\varphi)\colon G\to G'$ mit
$F(\varphi)(x):=\varphi(x)$ zugeordnet. Bei $F$ handelt
es sich um einen kovarianten Funktor.
\end{Satz}
\strong{Beweis.}
Es gilt $F(\id)(x) = \id(x)$, und daher $F(\id)=\id$.
Außerdem gilt%
\[
F(\varphi_2\circ\varphi_1)(x) = (\varphi_2\circ\varphi_1)(x)
= \varphi_2(\varphi_1(x))
= F(\varphi_2)(F(\varphi_1)(x))
= (F(\varphi_2)\circ F(\varphi_1))(x),
\]
und daher $F(\varphi_2\circ\varphi_1)
= F(\varphi_2)\circ F(\varphi_1)$.\;\qedsymbol

\begin{Satz} Sei $P(X)=2^X$ die Potenzmenge von $X$. Dann ist
wie folgt ein kovarianter Funktor gegeben:
\[P\colon\category{Set}\to\category{Set},\quad
P(X):=2^X,\quad P(f)(M):=f(M),\]
wobei $f$ eine beliebige Abbildung
und $f(M)$ die Bildmenge von $M$ unter $f$ ist.
\end{Satz}
\strong{Beweis.} Nach Satz \ref{Bild-unter-Komposition} gilt
\[P(g\circ f)(M) = (g\circ f)(M) = g(f(M))
= P(g)(P(f)(M)) = (P(g)\circ P(f))(M).\]
Daher ist $P(g\circ f)=P(g)\circ P(f)$. Außerdem ist
\[P(\id_X)(M) = \id_X(M) = M = \id_{P(X)}(M)\]
und daher $P(\id_X)=\id_{P(X)}$.\;\qedsymbol

Zum Funktor $P$ kommt noch ein weiterer Aspekt hinzu.
Für eine Abbildung $f$ kann man ganz pedantisch das
Bild $f(x)$ von der Bildmenge $f(\{x\})$ unterscheiden.
Aufgrund der Gleichung $f(\{x\})=\{f(x)\}$ verschwimmt diese
Unterscheidung aber gewissermaßen.
Die Abbildungen $f$ und $P(f)$ verhalten sich also
gewissermaßen gleich. Man kann sagen, dass $f$
auf ganz natürliche Art und Weise die Abbildung $P(f)$
zugeordnet ist. Definiert man
\[\eta(X)\colon X\to 2^X,\quad \eta(X)(x):=\{x\},\]
dann kommutiert das folgende Diagramm:
\[\xymatrix{
X \ar[r]^f \ar[d]_{\eta(X)} & Y \ar[d]^{\eta(Y)} \\
2^X \ar[r]_{P(f)} & 2^Y }\]
D.\,h. es gilt $\eta(Y)\circ f = P(f)\circ\eta(X)$.
Die Zuordnung $\eta$ ist eine sogenannte natürliche Transformation.

\begin{Definition}[Natürliche Transformation]\newlinefirst
Seien $\category C, \category D$ Kategorien und
$F,G\colon \category C\to \category D$ Funktoren.
Dann schreibt man $\eta\colon F\to G$ und nennt $\eta$ \emph{natürliche
Transformation}, wenn die folgenden beiden Axiome erfüllt sind:
\begin{enumerate}
\item Jedes Objekt $X$ von $\category C$ bekommt einen Morphismus\\
$\eta(X)\colon F(X)\to G(X)$.
\item Für jeden Morphismus $f\colon X\to Y$ gilt
$\eta(Y)\circ F(f)=G(f)\circ\eta(X)$.
\end{enumerate}
\end{Definition}
Die zweite Bedingung lässt sich übersichtlich als kommutierendes Diagramm
darstellen:
\[\xymatrix{
F(X) \ar[r]^{F(f)} \ar[d]_{\eta(X)} & F(Y) \ar[d]^{\eta(Y)} \\
G(X) \ar[r]_{G(f)} & G(Y)}\]
Ein weiteres Beispiel ergibt sich bezüglich Äquivalenzrelationen
in Erinnerung an die auf S.~\pageref{wohldefiniert} erklärten
wohldefinierten Abbildungen. Eine Abbildung $f\colon M\to M'$ heiße
\emph{induzierend} bezüglich $(M,\sim),(M,\sim')$, wenn für alle
$a,b\in M$ gilt%
\[a\sim b \Rightarrow f(a)\sim' f(b).\]

\begin{Satz}
Die Paare $(M,\sim)$, bestehend aus Menge und Äquivalenzrelation,
bilden mit den induzierenden Abbildungen
als Morphismen bezüglich Verkettung eine Kategorie.
\end{Satz}
\strong{Beweis.}
Die identische Abbildung ist offensichtlich induzierend. Hat man
neben $f\colon M\to M'$ eine weitere induzierende Abbildung $g\colon M'\to M''$, dann
folgt $g(y)\sim'' g(b)$ aus $y\sim' b$. Aus $x\sim a$ folgt
mit $y:=f(x)$ und $b:=f(a)$ somit $g(f(x))\sim'' g(f(x))$.
Daher ist auch $g\circ f$ induzierend.\;\qedsymbol

Genau dann wenn $f$ induzierend ist, existiert eine induzierte
Abbildung
\[I(f)\colon M/\sim\to M'/\sim',\;\text{so dass}\;I(f)\circ\pi = \pi'\circ f,\]
wobei $\pi,\pi'$ jeweils die kanonische Projektion ist.
\begin{Satz}
Bei der Induktion $I$ handelt es sich um einen kovarianten Funktor.
\end{Satz}
\strong{Beweis.} Man betrachte das folgende kommutierende Diagramm:
\[\xymatrix{
M \ar[r]^{f} \ar[d]_{\pi}
& M' \ar[r]^{g} \ar[d]^{\pi'}
& M'' \ar[d]^{\pi''}\\
M/\sim \ar[r]_{I(f)}
& M'/\sim' \ar[r]_{I(g)}
& M''/\sim''}\]
Die Induktion $I$ besitzt die Eigenschaften
\begin{gather*}
I(f)\circ\pi = \pi'\circ f,\\
I(g)\circ\pi' = \pi''\circ g,\\
I(g\circ f)\circ\pi = \pi''\circ (g\circ f).
\end{gather*}
Damit kann man nun rechnen
\begin{equation}
I(g\circ f)\circ\pi = \pi''\circ g\circ f
= I(g)\circ\pi'\circ f = I(g)\circ I(f)\circ\pi.
\end{equation}
Infolge gilt $I(g\circ f)=I(g)\circ I(f)$, da die kanonische
Projektion $\pi$ eine Surjektion ist. Aus der Forderung $I(\id)\circ\pi
= \pi\circ\id = \pi$ ergibt sich $I(\id) = \id$,
da $\pi$ surjektiv ist.\;\qedsymbol

Die Abbildung $\eta((M,\sim)):=\pi$, die jeder Menge mit
Äquivalenzrelation ihre kanonische Projektion zuordnet,
ist eine natürliche Transformation.

Funktoren haben einen allgemeine Eigenschaft.
\begin{Satz}
Wird ein Funktor auf einen Isomorphismus angewendet, ist das Resultat
wieder ein Isomorphismus.
\end{Satz}
\strong{Beweis.} Dieser ist direkt aus den Definitionen zu erhalten.
Wir betrachten nur einen kovarianten Funktor $F$. Der Beweis für
einen kontravarianten Funktor verläuft analog.

Sei $f\colon X\to Y$ ein beliebiger Isomorphismus im Definitionsbereich
des Funktors. Laut Definition existert eine Inverse, das heißt, ein
$g\colon Y\to X$ mit $g\circ f = \id_X$ und $f\circ g = \id_Y$. Gemäß
der definierenden Eigenschaft eines Funktors darf man rechnen
\begin{gather*}
\id_{F(X)} = F(\id_X) = F(g\circ f) = F(g)\circ F(f),\\
\id_{F(Y)} = F(\id_Y) = F(f\circ g) = F(f)\circ F(g).
\end{gather*}
Somit ist $F(f)$ ein Isomorphismus mit Inverse $F(g)$.\,\qedsymbol

Wird beispielsweise der Vergissfunktor von Gruppen zu Mengen auf
einen Gruppenisomorphismus angewendet, ist das Resultat zwingend
eine Bijektion, da die Bijektionen die Isomorphismen in der Kategorie
der Mengen sind.

\subsection{Anfangs- und Endobjekte}

\begin{Definition}[Anfangsobjekt, Endobjekt, Nullobjekt]\newlinefirst
Es sei $\category C$ eine Kategorie. Ein
$A\in\Ob(\category C)$ heißt \emph{Anfangsobjekt}, wenn
es zu jedem Objekt $X\in\Ob(\category C)$ genau einen
Morphismus $A\to X$ gibt.
Ein $E\in\Ob(\category C)$ heißt \emph{Endobjekt}, wenn
es zu jedem Objekt $X\in\Ob(\category C)$ genau einen
Morphismus $X\to E$ gibt. Ein Objekt heißt \emph{Nullobjekt}, wenn
es sowohl Anfangsobjekt als auch Endobjekt ist.
\end{Definition}

\noindent
\strong{Mengen.}
Wir untersuchen zunächst die Kategorie der Mengen. Anfangsobjekt
bedeutet hier eine Menge $A$, bei der es zu jeder Menge $X$ genau
eine Abbildung $A\to X$ gibt. Betrachten wir zunächst endliche
Mengen, dann ergibt sich aufgrund von Satz \ref{Anzahl-Abb} auf
S.~\pageref{Anzahl-Abb} ja die Bedingung $|X|^{|A|}=1$. Das geht nur,
wenn $|A|=0$ ist,
und das bedeutet $A=\emptyset$. Die einzige Abbildung in $\Abb(\emptyset,X)$
ist die leere Abbildung, und dies bleibt auch dann richtig, wenn
$X$ gänzlich beliebig ist. Somit haben wir die leere Menge als
einziges Anfangsobjekt identifiziert.

Endobjekt bedeutet eine Menge $E$, so dass es zu jeder Menge $X$
genau eine Abbildung $X\to E$ gibt. Wieder beschränken wir uns
zunächst auf endliche Mengen und nutzen Satz $\ref{Anzahl-Abb}$.
Wir erhalten die Bedingung $|E|^{|X|}=1$. Das geht nur, wenn
$|E|=1$ ist. Jede Menge mit einem Element ist also Endobjekt,
denn allgemein gibt es dann nur eine einzige Abbildung, nämlich
die konstante Abbildung. Dies bleibt auch dann richtig, wenn
$X$ gänzlich beliebig ist.

Ein Nullobjekt existiert offenbar nicht.

Benutzen wir doch $1:=\{\emptyset\}$ als kanonisches Endobjekt.
Interessant ist, dass man zu einer Menge $X$ jedes Element
$x\in X$ mit der Abbildung $x\colon 1\to X$ identifizieren kann,
für die $x(\emptyset)=x$ gilt. Zu einer Abbildung $f\colon X\to Y$
können wir die Zuordnung $f(x)=y$ bzw. $(x,y)\in f$ nun in der Form
$f\circ x = y$ beschreiben.

\subsection{Produkt und Koprodukt}

Ein wichtiger Begriff der Theorie ist das Produkt von
Objekten. Weil es sich dabei um eine Verallgemeinerung des
kartesischen Produktes von Mengen handelt, möchte ich die
Zusammenhänge zunächst am vertrauten Schauplatz der Mengen
betrachten.

Zu zwei Mengen $A_1, A_2$ können wir das Produkt
$A_1\times A_2$ bilden. Man definiert die Projektionen
auf die Komponenten als
\begin{align*}
&\pi_1\colon A_1\times A_2\to A_1,\quad\pi_1((x,y)) := x,\\
&\pi_2\colon A_1\times A_2\to A_2,\quad\pi_2((x,y)) := y.
\end{align*}
Nun betrachten wir Abbildungen $f_1\colon X\to A_1$ und
$f_2\colon X\to A_2$. Zunächst sei $X:=\{\emptyset\}$. Die jeweilige
Abbildung pickt dann ein Element aus der jeweiligen Menge heraus,
das sind $a_1:=f_1(\emptyset)$ und $a_2:=f_2(\emptyset)$. Nun
ist eine Abbildung $f$ gesucht, sodass das Diagramm
\[\xymatrix{
& X\ar[dl]_{f_1}\ar[dr]^{f_2}\ar[d]^f & \\
A_1 & A_1\times A_2 \ar[l]^{\pi_1}\ar[r]_{\pi_2} & A_2
}\]
kommutiert. Die Bedingungen an $f$ sind also
$\pi_i\circ f = f_i$ für $i\in\{1,2\}$. Damit ist aber eindeutig
festgelegt, dass $f(\emptyset)=(a_1,a_2)$ sein muss, denn ein
Tupel ist durch die Komponenten festgelegt, und die sind
$\pi_i(f(\emptyset)) = f_i(\emptyset) = a_i$ für $i\in\{1,2\}$.
Somit ist $f$ eindeutig bestimmt.

Die Betrachtung kann man genauso für eine allgemeine Menge $X$ führen,
weil die Argumentation dann jeweils für jedes Element von $X$ gilt.
Wieder ist $f$ eindeutig bestimmt.

Gelegentlich wird $f$ als $f=f_1\times f_2$ notiert.

\begin{Definition}[Produkt]\newlinefirst
Sei $\category C$ eine Kategorie und seien $Y_1,Y_2$ Objekte von
$\category C$. Ein Objekt $Y$ von $\category C$ mit Projektionen
$\pi_1\colon Y\to Y_1$ und $\pi_2\colon Y\to Y_2$ heißt \emph{Produkt},
wenn zu jedem Objekt $X$ von $\category C$ und allen Morphismen
$f_1\colon X\to Y_1$ und $f_2\colon X\to Y_2$ genau ein Morphismus
$f\colon X\to Y$ existiert, so dass $f_1=\pi_1\circ f$ und
$f_2=\pi_2\circ f$.
\end{Definition}

\noindent
Das kartesische Produkt $Y:=Y_1\times Y_2$ ist ein Produkt in
der Kategorie der Mengen. Wir schreiben $f(x)=y$ mit $y=(y_1,y_2)$.
Nun ist $\pi_1(y)=y_1$ und $\pi_2(y)=y_2$, laut Forderung soll also
$y_1=f_1(x)$ und $y_2=f_2(x)$ sein. Dadurch ist $f$ mit
$f(x):=(f_1(x),f_2(x))$ eindeutig festgelegt.

Zu zwei Mengen können wir weiterhin die disjunkte Vereinigung
$X_1\sqcup X_2$ bilden. Wir rekapitulieren, dass zu ihr die beiden
kanonischen Injektionen
\begin{align*}
& i_1\colon X_1\to X_1\sqcup X_2,\quad i_1(x) := (1,x),\\
& i_2\colon X_2\to X_1\sqcup X_2,\quad i_2(x) := (2,x)
\end{align*}
gehören. Man stellt sich nun die Frage, was das Wesensmerkmal
der disjunkten Vereinigung ist. Das ist doch, dass zu jedem ihrer
Elemente die Information vorliegt, ob es aus der linken oder der
rechten Menge entstammt. Das heißt, es muss eine Abbildung geben,
die auf den Tag projiziert. Betrachten wir dazu die Abbildungen
$f_1\colon X_1\to Y$ und $f_2\colon X_2\to Y$ mit $Y:=\{1,2\}$ und
$f_k(x):=k$. Mit der Abbildung $f_k$ gelangt man von $X_k$ also
direkt zum Tag $k$. Nun ist eine Abbildung $f$ gesucht, so dass das
Diagramm
\[\xymatrix{
X_1\ar[r]^{i_1}\ar[dr]_{f_1} & X_1\sqcup X_2\ar[d]^{f}
& X_2\ar[l]_{i_2}\ar[dl]^{f_2}\\
& Y &
}\]
kommutiert. Das heißt, es soll $f\circ i_k = f_k$ für $k\in\{1,2\}$ sein.
Das macht $f((1,x)) = 1$ und $f((2,x)) = 2$. Dadurch ist $f$ eindeutig
bestimmt. Es ist die gesuchte Projektion auf den Tag.

\begin{Definition}[Koprodukt]\newlinefirst
Sei $\category C$ eine Kategorie und seien $X_1,X_2$ Objekte von
$\category C$. Ein Objekt $X$ von $\category C$ mit Morphismen
$i_1\colon X_1\to X$ und $i_2\colon X_2\to X$ heißt \emph{Koprodukt},
wenn zu jedem Objekt $Y$ von $\category C$ und allen
Morphismen $f_1\colon X_1\to Y$ und $f_2\colon X_2\to Y$ genau ein
Morphismus $f\colon X\to Y$ existiert, so dass $f_1 = f\circ i_1$
und $f_2 = f\circ i_2$.
\end{Definition}

\noindent
Die disjunkte Vereinigung $X:=X_1\sqcup X_2$ mit den Injektionen
$i_1(x):=(1,x)$ und $i_2(x):=(2,x)$ ist ein Koprodukt in der
Kategorie der Mengen. Es gilt schon mal
\[f(x) = \strong{match}\; x \begin{cases}
(1,x)\mapsto y_1,\\
(2,x)\mapsto y_2.
\end{cases}\]
Laut Forderung soll außerdem $y_1 = f_1(x)$ und $y_2 = f_2(x)$ sein,
wodurch $f$ eindeutig festgelegt ist.

\subsection{Exponentialobjekte}

Die Notation $B^A$ stehe für die Menge der Abbildungen $A\to B$.
Es soll nun die Applikation einer Abbildung auf ein Argument
als eigenständige Operation
\[\varepsilon\colon B^A\times A \to B,\quad
\varepsilon(f,a) := f(a)\]
gedacht werden. Einer zweistelligen Abbildung
$g\colon X\times A\to B$ lässt sich die Abbildung
\[\hat g\colon X\to B^A,\quad \hat g(x)(a) := g(x,a).\]
zuordnen. Diesen Vorgang nennen wir Currying. Man findet nun
\[g(x,a) = \hat g(x)(a) = \varepsilon(\hat g(x),a)
= (\varepsilon\circ (\hat g\times\id_A))(x,a).\]
Die Gleichung $g = \varepsilon\circ(\hat g\times\id_A)$ ist also
für jede Abbildung $g$ erfüllt, was bedeutet, dass das Diagramm
\[\xymatrix{
X\times A\ar[dr]^g\ar[d]_{\hat g\times\id_A} & \\
B^A\times A \ar[r]_{\varepsilon} & B
}\]
kommutiert.

\begin{Definition}[Exponentialobjekt]%
\label{def:Exponentialobjekt}\newlinefirst
Sei $\category C$ eine Kagegorie, in der das Produkt je zweier Objekte
existiert. Zu zwei Objekten $A,B$ von $\category C$ heißt ein
Objekt $B^A$ von $\category C$ zusammen mit einem Morphismus
$\varepsilon\colon B^A\times A\to B$ \emph{Exponentialobjekt},
wenn es zu jedem Objekt $X$ von $\category C$ und Morphismus
$g\colon X\times A\to B$ genau einen Morphismus $\hat g\colon X\to B^A$
gibt, so dass $\varepsilon\circ (\hat g\times\id_A) = g$ gilt.
\end{Definition}

\begin{Satz}\label{Exp-Isomorphie}
Es besteht die Isomorphie
$\Hom_{\category C}(X\times A, B)\cong\Hom_{\category C}(X,B^A)$.
\end{Satz}

\noindent\strong{Beweis.} Es sei $\lambda(g):=\hat g$ bezüglich Def.
\ref{def:Exponentialobjekt}. Zu zeigen ist, dass es sich bei $\lambda$
um einen Isomorphismus handelt. Für jedes $h\colon X\to B^A$ sei dazu
\[\lambda'(h) := \varepsilon\circ (h\times\id_{A})\colon X\times A\to B.\]
Zu zeigen ist, dass $\lambda'$ der inverse Morphismus zu $\lambda$ ist.
Def. \ref{def:Exponentialobjekt} sichert nun direkt zu, dass
$\lambda'(\lambda(g)) = g$ gelten muss. Zu bestätigen verbleibt
$\lambda(\lambda'(h)) = h$. Die Allaussage in Def. \ref{def:Exponentialobjekt}
wird hierzu spezialisiert mit $g:=\varepsilon\circ (h\times\id_A)$.
Nun wissen wir aber nicht nur, dass $\hat g$ existieren muss, wir
können es mit der Setzung $\hat g:=h$ angeben, denn dieses erfüllt die
Gleichung
\[\varepsilon\circ (\hat g\times\id_A) = \varepsilon\circ (h\times\id_A).\]
Infolge gilt
\[\lambda(\lambda'(h)) = \lambda(\varepsilon\circ (h\times\id_A))
= \lambda(\varepsilon\circ (\hat g\times\id_A))
= \lambda(g) = \hat g = h.\,\qedsymbol\]

\newpage
\section{Beweise als Terme}

\subsection{Kartesisch abgeschlossene Kategorien}

\begin{Definition}[Kartesisch abgeschlossene Kategorie]\newlinefirst
Eine Kategorie $\category C$ heißt \emph{kartesisch abgeschlossen},
wenn
\begin{enumerate}[nosep]
\item sie ein Terminalobjekt enthält,
\item je zwei Objekte $A,B$ von $\category C$ ein Produkt
  $A\times B$ in $\category C$ besitzen,
\item je zwei Objekte $A,B$ von $\category C$ ein Exponential
  $B^A$ in $\category C$ besitzen.
\end{enumerate}
\end{Definition}

\noindent
Dass die Kategorie der Mengen kartesisch abgeschlossen ist, wurde
bereits während der Diskussion der drei Begrifflichkeiten
nachgerechnet. Die Kategorie der endlichen Mengen ist ebenfalls
kartesisch abgeschlossen. Sind $A,B$ endlich, ist ja $A\times B$
und $B^A$ ebenfalls eine endliche Menge.

\subsection{Zur BHK-Interpretation}

Im Bestreben, die Struktur der Schlussregeln zu ergründen, tritt
mit der Zeit das trübe Muster zutage, dass der Aufbau von Beweisen
gleichermaßen abläuft, wie der von Elementen bestimmter geläufiger
Mengenkonstruktionen. Ein Beweis der Konjunktion $A\land B$ ist
zum Beispiel erbracht, wenn ein Beweis $a$ der Aussage $A$ und
ein Beweis $b$ der Aussage $B$ gefunden wurde. Dies gilt natürlich
nur unter der Voraussetzung, dass der Beweis der Konjunktion durch ihre
Einführungsregel zustande gekommen ist. Das heißt, wenn überhaupt,
kann es sich nur um die Normalform der Beweise handeln. Wird nun die
Evidenz für $A\land B$ vorgelegt, muss aus dieser sowohl $a$ als auch
$b$ extrahiert werden können. Der Beweis von $A\land B$ ist demnach
das Paar $(a,b)$. Man erkennt nun, dass sich die Einführungsregel
\[\dfrac{\vdash A\quad\;\vdash B}{\vdash A\land B}\quad\text{analog zu}\quad
\dfrac{a\in [A]\quad\; b\in [B]}{(a,b)\in [A]\times [B]}\]
verhält. Hierbei bezeichne $[A]$ die Menge der Beweise der Aussage $A$,
wobei die Gleichheit $[A\land B] = [A]\times [B]$ gefordert wird.

Die gerade geschilderte Überlegung folgt der \emph{BHK"=Interpretation
der intuitionistischen Logik}, benannt nach Luitzen Brouwer, Arend
Heyting und Andrei Kolmogorow. Ihre ursprüngliche Form ist nicht
mathematisch präzisiert sondern wird wie folgt abgefasst.

Ein Beweis von $A\land B$ ist erbracht, wenn sowohl ein Beweis
von $A$ als auch ein Beweis von $B$ vorliegt.

Ein Beweis von $A\lor B$ ist erbracht, wenn zu mindestens einer der Aussagen
$A,B$ ein Beweis vorliegt.

Ein Beweis von $A\cond B$ ist erbracht, wenn eine Konstruktion
gefunden wurde, mit der man einen Beweis von $B$ aus einem Beweis
von $A$ erhält.

Ein Beweis von $\lnot A$ ist erbracht, wenn eine Konstruktion gefunden wurde,
mit der man einen Beweis von $\bot$ aus einem Beweis von $A$ erhält.

Ich will hier aber weiter den Bezug zur Mengenlehre herstellen, und damit
einer mathematischen Präzisierung näher kommen.

Bei der Subjunktion tritt nun die Analogie zwischen
\[\dfrac{\vdash A\cond B\quad\;\vdash A}{\vdash B}\quad\text{und}\quad
\dfrac{f\in\mathrm{Abb}([A],[B])\quad\; a\in [A]}{f(a)\in [B]}\]
in Erscheinung, wobei $[A\cond B] = \mathrm{Abb}([A],[B])$ gefordert
wird.

Ein Beweis der Disjunktion $A\lor B$ gilt bereits als erbracht,
wenn ein Beweis zumindest einer der beiden Aussagen $A,B$ gefunden
wurde. Wird nun die Evidenz von $A\lor B$ zur Prüfung vorgelegt, muss
man erfahren dürfen, zu welcher der beiden Aussagen der Beweis gefunden
wurde. Diese Unterscheidung schafft die disjunkte Vereinigung
$[A\lor B] = [A]\sqcup [B]$. Bei der Disjunktion besteht die Analogie also zwischen
\[\dfrac{\vdash A}{\vdash A\lor B},\;\,\dfrac{\vdash B}{\vdash A\lor B}\quad\text{und}\quad
\dfrac{a\in [A]}{(1,a)\in [A]\sqcup [B]},\;\,\dfrac{b\in [B]}{(2,b)\in [A]\sqcup [B]}.\]
Solange dies die einzigen Regeln zur Einführung der Disjunktion
bleiben, wird eine Aussage auch nur dann als wahr angesehen werden
dürfen, wenn ein Beweis zumindest einer der beiden Aussagen $A,B$
vorgelegt wurde. Infolge stellt sich der Satz vom ausgeschlossenen
Dritten unzulässig heraus, die Logik somit als intuitionistisch.

Ein gern herangezogenes Beispiel, das erfahrbar macht, wie es sich mit
der intuitionistischen Logik verhält, bietet die anschließende Überlegung
zu Rechtssystemen. Betrachten wir den folgenden Fall.
Aus dem Kühlschrank wurde ein Kuchen entwendet. Außerdem liegen
belastbare Beweismittel vor, die belegen, dass Alice oder Bob
dafür verantwortlich ist. Allerdings fanden sich weder Indizien,
die eine der beiden Personen belasten, noch fand sich ein Alibi, das
eine der beiden entlasten würde. Somit bleibt für beide die
Unschuldsvermutung erhalten.
Würde man nun von den beiden je einen halben Kuchen als Schadensersatz
fordern, käme dies einer Kollektivstrafe gleich. Wird das Konzept
von Kollektivstrafen verworfen, führt dies insofern dazu, dass dem
Rechtssystem die intuitionistische Logik innewohnen muss.

Zur Berücksichtigung von Hypothesen wird ein Urteil der Form
$H\vdash A$ als $\vdash H\cond A$ gedacht. Zufolge dieser Überlegung
nimmt die Regel
\[\dfrac{H\vdash A\qquad H\vdash B}
{H\vdash A\land B}\quad\text{die Form}\quad
\dfrac{a\colon [H]\to [A]\qquad b\colon [H]\to [B]}
{(x\mapsto (a(x),b(x)))\colon [H]\to [A]\times [B]}\]
an. Mit der Einbeziehung von Hypothesen wird schließlich
die Subjunktionseinführung fassbar. Zu
\[\dfrac{H,A\vdash B}
{H\vdash A\cond B}\quad\text{findet sich}\quad
\dfrac{f\colon [H]\times [A]\to [B]}
{(x\mapsto a\mapsto f(x, a))\colon [H]\to [A]\to [B]}\]
Mit der Notation $x\mapsto a\mapsto f(x, a)$ ist die Funktion $g$
gemeint, für die gilt
\[g(x)(a) = f(x, a).\]
Der Funktionswert von $g$ an der Stelle $x$ soll also wiederum eine
Funktion sein, dessen Funktionswert an der Stelle $a$ der Wert
$f(x, a)$ ist. Der Übergang von $f$ zu $g$ wird \emph{Schönfinkeln}, im
Englischen \emph{currying}, genannt, benannt nach Moses Schönfinkel und
Haskell Brooks Curry. Das logische Gegenstück lässt sich auch als das
Theoremschema
\[(H\land A\cond B)\cond (H\cond (A\cond B)),\]
darstellen, das Schema der \emph{Exportation}.

\subsection{Sequenzen als Morphismenklassen}

Bislang waren Abbildungen zwischen Mengen Gegenstand der Betrachtung.
Uns hält aber nichts davon ab, diese als Morphismen zwischen
Objekten zu sehen. Dieser Abstraktionsschritt soll nun als nächstes
vorgenommen werden.

Mit der Idee, das Urteil $A\vdash B$ als Vorhandensein eines
Morphismus $A\to B$ zu interpretieren, gelangt man zur kategoriellen
Semantik. Dabei werden nur solche Sequenzen betrachtet, die eine einzige
Vorderformel besitzen, was aber keine wesentliche Einschränkung
darstellt, da die Vorderformeln immer zu einer Konjunktion
zusammengefasst werden können. Ist die Ansammlung der Vorderformeln
leer, erhält man als leere Konjunktion die tautologische Formel.
Das heißt, ein Urteil der Form $\vdash A$ wird als $\top\vdash A$
betrachtet.

Jede Aussage wird als ein Objekt interpretiert, und Urteile sind
Morphismen zwischen den Aussagen. Die Konjunktion $A\land B$
wird interpretiert als Produkt $A\times B$, die Disjunktion $A\lor B$
als Koprodukt $A\sqcup B$. Die kontradiktorische Formel $\bot$ wird als
Anfangsobjekt $0$ interpretiert, die tautologische Formel $\top$ als
ein Terminalobjekt $1$. Bei der Subjunktion $A\cond B$ muss man nun ein
wenig achtsam sein. Ihr entspricht nicht etwa die Morphismenklasse
$A\to B$, sondern wie bei allen Formeln ein Objekt, das Exponentialobjekt
$B^A$. Die Negation $\lnot A$ wird als $A\cond\bot$ betrachtet, also
durch $0^A$ interpretiert.

Der Überlegung nach darf das Urteil $\vdash A$ genau dann gefällt
werden, wenn ein Morphismus $1\to A$ existiert. Dieser Morphismus
wählt ein Element aus $A$ aus, womit $A$ bewohnt sein muss.
Allgemein stellen wir Aussagen gemäß Tabelle \ref{tab:Aussagen-zu-Objekten}
als Objekte dar.

\begin{table}
\begin{center}
\caption{Übersetzung von Aussagen in Objekte}%
\label{tab:Aussagen-zu-Objekten}
\begin{tabular}{l}
\toprule
$\begin{aligned}
{}[A\land B] &:= [A]\times [B] & [\bot] &:= 0\\
[A\lor B] &:= [A]\sqcup [B] &  [\top] &:= 1\\
[A\cond B] &:= [B]^{[A]} & [A\bicond B] &:= [B]^{[A]}\times [A]^{[B]}\\
[\lnot A] &:= 0^{[A]} & [\{A_1,\ldots,A_n\}] &:= [A_1]\times\ldots\times [A_n]
\end{aligned}$\\
\bottomrule
\end{tabular}
\end{center}
\end{table}

\begin{Satz}\label{Urteil-impliziert-Morphismus}
Bildet die Übersetzung aus Tabelle \ref{tab:Aussagen-zu-Objekten} in
eine bikartesisch abgeschlossene Kategorie ab, so gilt für die
intuitionistische Logik die Folgerung
\[(\Gamma\vdash A) \,\cond\, (\exists f\colon [\Gamma]\to [A]).\]
\end{Satz}
\begin{Beweis}
Es wird eine strukturelle Induktion über den Aufbau von Beweisen
durchgeführt. Zum Induktionsanfang. Die Regel
\[\dfrac{}{A\vdash A}\quad\text{fordert}\quad\dfrac{}{\exists f\colon [A]\to [A]}.\]
Dies zeigt sich mit der Setzung $f:=\id$. Die Abschwächungsregel
\[\dfrac{\Gamma\vdash B}{\Gamma,A\vdash B}\quad\text{fordert}\quad
\dfrac{\exists f\colon [\Gamma]\to [B]}{\exists g\colon [\Gamma]\times [A]\to [B]}.\]
Mit der Prämisse, die Induktionsvoraussetzung ist, muss man zur
Konklusion gelangen, was kurzum mit der Setzung $g:=f\circ\pi_1$
erreicht wird, wobei $\pi_1$ gemäß der universellen Eigenschaft des
Produktes zur Verfügung steht.

Die Konjunktionseinführung
\[\dfrac{\Gamma\vdash A\qquad\Gamma\vdash B}{\Gamma\vdash A\land B}
\quad\text{fordert}\quad
\dfrac{\exists f_1\colon [\Gamma]\to [A]\qquad\exists f_2\colon [\Gamma]\to [B]}
{\exists f\colon [\Gamma]\to [A]\times [B]}.\]
Man legt $f:=(f_1, f_2)$ vor, was gemäß der universellen
Eigenschaft des Produktes gebildet werden darf. Die Beseitigungsregel
\[\dfrac{\Gamma\vdash A\land B}{\Gamma\vdash A}\quad\text{fordert}\quad
\dfrac{\exists f\colon [\Gamma]\to [A]\times [B]}
{\exists g\colon [\Gamma]\to [A]}.\]
Man setzt $g:=\pi_1\circ f$, wobei $\pi_1$ gemäß der universellen
Eigenschaft des Produktes zur Verfügung steht.

Die Subjunktionseinführung
\[\dfrac{\Gamma,A\vdash B}{\Gamma\vdash A\to B}\quad\text{fordert}\quad
\dfrac{\exists f\colon [\Gamma]\times [A]\to [B]}{\exists g\colon [\Gamma]\to [B]^{[A]}}.\]
Man hat $g:=\lambda f$, wobei $\lambda$ der angegebene Isomorphismus zu
Satz \ref{Exp-Isomorphie} ist. Die Beseitigungsregel
\[\dfrac{\Gamma\vdash A\cond B\qquad\Gamma\vdash A}{\Gamma\vdash B}\quad\text{fordert}\quad
\dfrac{\exists f\colon [\Gamma]\to [B]^{[A]}\qquad\exists g\colon [\Gamma]\to [A]}
{\exists h\colon [\Gamma]\to [B]}.\]
Es ist $\lambda^{-1} f\colon [\Gamma]\times [A]\to [B]$. In der Kategorie
der Mengen ginge die Setzung $h(x) := (\lambda^{-1} f)(x, g(x))$. Das
gesuchte Abstraktum hierzu ist $h := (\lambda^{-1} f)\circ (\id, g)$.

Da die Negation $\lnot A$ als Subjunktion $A\cond\bot$ definiert wird,
brauchen wir diese nicht explizit zu betrachten. Dasselbe gilt für
die Bijunktion $A\bicond B$, die als $(A\cond B)\land (B\cond A)$
definiert wird.

Zuletzt verbleibt noch ex falso quodlibet $\bot\vdash A$ als
ein weiterer Induktionsanfang zu bestätigen. Dazu muss ein Morphismus
$f\colon 0\to [A]$ existent sein, was aber gerade die Eigenschaft des
Anfangsobjektes $0$ darstellt.\,\qedsymbol
\end{Beweis}

\noindent
Weiterhin stellt sich heraus, dass die Schnittregel der schlichten
Komposition von Morphismen entspricht. Genauer wird die Übersetzung des
Schlusses
\[\dfrac{A\vdash B\qquad B\vdash C}{A\vdash C}\;\;\text{gesichert durch}\;\;
\dfrac{f\colon A\to B\qquad g\colon B\to C}{g\circ f\colon A\to C}.\]
Die übersetzte allgemeine Form der Schnittregel braucht nicht mehr
bestätigt werden, dies ist bereits als abschließender Schritt in der
Bestätigung der übersetzten Subjunktionsbeseitigung zu finden.

Die eigentliche BHK"=Interpretation soll die Struktur von Beweisen der
intuitionistischen Logik erklären. Dafür sollte man aber am besten
ausgehend von einem Morphismus zum zugehörigen Beweis zurückgelangen
können. Insbesondere sollte die Umkehrung von Satz
\ref{Urteil-impliziert-Morphismus} ebenfalls gelten.
Es mag allerdings sein, dass zwischen den Objekten Morphismen bestehen,
die keinem Beweis entsprechen. Betrachten wir Objekte wieder
speziell als Mengen, kann dies ausgeschlossen werden, indem als Menge
der erreichbaren Konstrukte, heißt formulierbaren Terme, die kleinste
induktive Menge gewählt wird, die unter den verfügbaren
Konstruktionsschritten abgeschlossen ist. Weil die Schritte zu
verschiedenartigen Konstrukten führen, erhält man zudem eine frei
erzeugte induktive Menge. Man kann nun per struktureller Rekursion über
den Term eine Rückübersetzung zum ursprünglichen Beweis definieren.
Die kleinste induktive Menge verhält sich demnach isomorph zur
ursprünglichen Menge der formulierbaren Beweise.

Was ist dann aber gewonnen, wenn Beweise durch die Übersetzung
lediglich in eine andere Gestalt gebracht werden? Zum einen ermöglicht
die Übersetzung offenbar, Beweise mit Konzepten der Kategorientheorie
in Verbindung zu bringen. Zum anderen wurden in der gerade gemachten
Überlegung die Beweise zusätzlich in die Übersetzung miteinbezogen.
Das Beschriebene System zur Konstruktion von Termen stellt sich später
als der einfach getypte Lambda"=Kalkül heraus, dem in der Informatik
eine grundlegende Bedeutung zukommt. Der Lambda"=Kalkül verhält sich so
ähnlich wie ein Termersetzungssystem und formalisiert im Wesentlichen
die Konstruktion und Applikation von Funktionen.

Von der BHK"=Interpretation ausgehend gelangt man somit zum
\emph{Curry"=Howard"=Isomorphismus}, der die Übersetzung von Beweisen
in die entsprechenden Terme des einfach getypten Lambda"=Kalküls
beschreibt. Aussagen werden hierbei in Typen übersetzt.
