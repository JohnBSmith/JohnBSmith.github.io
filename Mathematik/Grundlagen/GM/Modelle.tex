
\chapter{Elemente der Modelltheorie}

\section{Semantik der klassischen Aussagenlogik}

\subsection{Die Erfüllungsrelation}

Bislang trat die Logik in der Form eines formalen Systems in
Erscheinung. Gegenstand eines solchen Systems sind im Allgemeinen
\emph{Wörter} einer formalen Sprache; im natürlichen Schließen sind das
die Sequenzen. Einige Wörter, die \emph{Axiome}, werden als gegeben
vorausgesetzt. Unter Anwendung von \emph{Ableitungsregeln}, auch
\emph{Inferenzregeln} genannt, das sind die Schlussregeln, leitet man
aus bereits abgeleiteten Wörtern weitere Wörter der Sprache ab.
In diesem Sinne handelt es sich um ein rein syntaktisches System.

Zum tieferen Verständnis muss man sich im Fortgang damit beschäftigen,
welche inhaltliche Bedeutung den logischen Aussagen beigemessen
wird. Der hierfür wesentliche Schritt besteht in der Definition
einer passenden \emph{Semantik}\index{Semantik}.

Gegenstand der Semantik der Logik ist der Wahrheitsgehalt von Aussagen.
Man hat gefunden, dass es sich mit der Frage nach dem Wesen der Wahrheit
schwierig verhält. Wir wollen daher an dieser Stelle gar nicht erst
versuchen, sie zu ergründen. Stattdessen tritt Wahrheit für uns zunächst
lediglich im leicht fassbaren Rahmen der zweiwertigen booleschen
Algebra auf.

In der klassischen Semantik der Aussagenlogik herrscht das
\emph{Bivalenzprinzip}\index{Bivalenzprinzip}, das besagt, dass jede
Aussage entweder \emph{wahr} oder \emph{falsch} sein muss, also einen
von zwei Wahrheitswerten haben muss. Eine Aussage kann nicht
\emph{ein wenig wahr} oder \emph{halbwegs wahr} sein, noch kann sie
eine von mehreren unterschiedlichen gleichwertigen Wahrheiten haben.
Wir schreiben kurz $0$ für falsch und $1$ für wahr. Enthält eine Formel
logische Variablen, kommt ihr ein Wahrheitswert zu, sobald alle
Variablen durch eine Interpretation\index{Interpretation} mit
einem Wahrheitswert belegt wurden.

Die Art und Weise, wie einer Formel ein Wahrheitswert zukommt,
präzisiert die Erfüllungsrelation. Sie wird als Rekursion über den
Formelaufbau definiert. Der Wahrheitswert einer Formel ist hierbei
einzig und allein durch die Wahrheitswerte ihrer Teilformeln bestimmt.

\begin{Definition}[Erfüllungsrelation]\newlinefirst
Eine Interpretation $I$ ist eine Funktion, die jede atomare logische
Variable $v$ mit einem Wahrheitswert $I(v)\in\{0,1\}$ belegt.
Man definiert $I\models A$, sprich »$I$ erfüllt $A$«, rekursiv als
\begin{align*}
(I\models\bot) &\iff 0,\\
(I\models\top) &\iff 1,\\
(I\models v) &\iff I(v),\\
(I\models\lnot A) &\iff \lnot (I\models A),\\
(I\models A\land B) &\iff (I\models A)\land (I\models B),\\
(I\models A\lor B) &\iff (I\models A)\lor (I\models B),\\
(I\models A\cond B) &\iff (I\models A)\cond (I\models B),\\
(I\models A\bicond B) &\iff (I\models A)\bicond (I\models B).
\end{align*}
\end{Definition}
Die rechte Seite ist jeweils metalogisch zu verstehen und per
Wahrheitstafel definiert, siehe Tabelle \ref{tab:Junktoren}.
Die Schreibweise $I\nvDash A$ ist gleichbedeutend mit
$\lnot (I\models A)$. Eine Interpretation wird auch als \emph{Modell}
bezeichnet. Man nennt sie \emph{Modell} einer Formel, falls sie die
Formel erfüllt. Andernfalls spricht man von einem \emph{Kontramodell}
oder \emph{Gegenmodell} der Formel.

\begin{table}
\caption{Wahrheitstafel der Junktoren}
\label{tab:Junktoren}
\centering
\begin{tabular}{cc@{\qquad}c@{\qquad}c@{\qquad}c@{\qquad}c@{\qquad}c}
\toprule
$A$ & $B$ & $\lnot A$ & $A\land B$ & $A\lor B$ & $A\cond B$ & $A\bicond B$\\
\midrule
$0$ & $0$ & $1$ & $0$ & $0$ & $1$ & $1$\\
$1$ & $0$ & $0$ & $0$ & $1$ & $0$ & $0$\\
$0$ & $1$ & $1$ & $0$ & $1$ & $1$ & $0$\\
$1$ & $1$ & $0$ & $1$ & $1$ & $1$ & $1$\\
\bottomrule
\end{tabular}
\end{table}

\begin{Definition}
Für einen Kontext $\Gamma = \{A_1,\ldots,A_n\}$ setzt man
\[(I\models\Gamma)\defiff (I\models A_1)\land\ldots\land (I\models A_n).\]
\end{Definition}

\subsection{Gültigkeit einer Formel}

Eine wichtige Rolle spielen \emph{allgemeingültige} Formeln, die man
in der Aussagenlogik auch als \emph{Tautologien}\index{Tautologie}
bezeichnet. Sie sind immer wahr, unabhängig davon, mit welchem
Wahrheitswert ihre logischen Variablen belegt werden.

Als allgemeinere Begrifflichkeit wollen auf einen Kontext $\Gamma$
bezogen gültige Formeln $A$ betrachten. Die Idee hierbei ist,
dass wenn die Formeln des Kontextes als wahr angenommen werden, die
Formel $A$ ebenfalls wahr sein muss. Trifft dies auf $A$ zu,
schreibt man $\Gamma\models A$, sprich »im Kontext $\Gamma$ ist
$A$ gültig«, oder auch »$\Gamma$ zieht $A$ nach sich«. Die Bezeichnung
\emph{logische Folgerung} oder \emph{logische Konsequenz} ist
ebenfalls verbreitet.

\begin{Definition}[Gültige Formel]\newlinefirst
Eine Formel $A$ heißt genau dann gültig im Kontext $\Gamma$,
wenn jede Interpretation, die sämtliche Formeln von $\Gamma$ erfüllt,
auch $A$ erfüllt. Metalogisch
\[(\Gamma\models A)\defiff \forall I\colon (I\models\Gamma)\cond (I\models A).\]
\end{Definition}
Eine im leeren Kontext gültige aussagenlogische Formel $A$ nennt man
wie gesagt Tautologie. Statt $\emptyset\models A$ schreibt man auch
kurz $\models A$. Wie bei Sequenzen schreibt man auch $\Gamma,A,B\models C$
anstelle von $\Gamma\cup\{A,B\}\models C$.

\subsection{Wahrheitstafeln}

Obgleich der Variablenvorrat unendlich groß sein darf, enthält ein
eine Formel von den Variablen nur endlich viele. Insofern sind für eine
Formel in einem Kontext auch nur endlich viele Interpretationen relevant.
Sind insgesamt $n$ Variablen vorhanden, sind es $2^n$ Interpretationen.

Eine Interpretation $I$ mit der Auswertung $I\models A$ ist nichts
anderes als eine Zeile der Wahrheitstafel der Formel $A$. Eine Formel
ist genau dann tautologisch, wenn in der Ergebnisspalte in jeder
Zeile eine~1 steht.

Die Wahrheitstafel \ref{tab:Tautologie-zur-Kontraposition} bestätigt
\[\models (a\cond b)\bicond (\lnot b\cond\lnot a).\]

\begin{table}
\caption{Wahrheitstafel der Tautologie zur Kontraposition}
\label{tab:Tautologie-zur-Kontraposition}
\centering
\begin{tabular}{cc@{\quad\;\;}c@{\quad\;\;}c@{\quad\;\;}c@{\quad\;\;}c@{\quad\;\;}c}
\toprule
$a$ & $b$ & $\lnot b$ & $\lnot a$ & $a\cond b$ & $\lnot b\cond\lnot a$
& $(a\cond b)\bicond (\lnot b\cond\lnot a)$\\
\midrule
$0$ & $0$ & $1$ & $1$ & $1$ & $1$ & $1$ \\
$1$ & $0$ & $1$ & $0$ & $0$ & $0$ & $1$ \\
$0$ & $1$ & $0$ & $1$ & $1$ & $1$ & $1$ \\
$1$ & $1$ & $0$ & $0$ & $1$ & $1$ & $1$ \\
\bottomrule
\end{tabular}
\end{table}

