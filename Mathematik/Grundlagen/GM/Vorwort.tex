
Manchmal werden Begriffe benutzt, die erst später erklärt werden.
An diesen Stellen wird dann auf den eklärenden Abschnitt verwiesen.
Das Buch muss somit unter Umständen nichtlinear gelesen werden. Einige
Autoren verfolgen das didaktische Prinzip, dass ihr Buch linear vom Anfang
bis zum Ende gelesen werden kann. Ich empfinde die nichtlineare Lesung
allerdings als reizvoller.

Manche Abschnitte stellen Vertiefungen dar, die für das Voranschreiten
entbehrlich sind. Es wird erwähnt, wenn ein Abschnitt unverfänglich
übersprungen werden kann.

Manche lernen Mathematik lieber durch Rechnen von Aufgaben. Das
vorliegende Buch umfasst allerdings lediglich die Theorie, da
die Hinzufügung von Aufgaben meines Erachtens die Darstellung zu sehr
aufblähen würde. Mir scheint, eine umfassende Sammlung von Aufgaben
mit ausführlichen Lösungen ist in einem separaten Buch besser
aufgehoben. Zumindest wird aber zu fast jedem Satz der Beweis
ausgeführt, was man immer auch als Übungsaufgabe samt Lösung sehen kann.
