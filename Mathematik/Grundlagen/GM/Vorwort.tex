
Manchmal werden Begriffe benutzt, die erst später erklärt werden.
An diesen Stellen wird dann auf den eklärenden Abschnitt verwiesen.
Das Buch muss somit unter Umständen nichtlinear gelesen werden. Einige
Autoren verfolgen das didaktische Prinzip, dass ihr Buch linear vom Anfang
bis zum Ende gelesen werden kann. Ich empfinde die nichtlineare Lesung
allerdings als reizvoller.

Manche Abschnitte stellen Vertiefungen dar, die für das Voranschreiten
entbehrlich sind. Es wird erwähnt, wenn ein Abschnitt unverfänglich
übersprungen werden kann.

