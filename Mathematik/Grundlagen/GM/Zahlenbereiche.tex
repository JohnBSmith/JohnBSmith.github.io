
\chapter{Zahlenbereiche}

\section{Die natürlichen Zahlen}

\subsection{Modelle der natürlichen Zahlen}

Zahlen spielen in den Mathematik eine maßgebliche Rolle, und dies nicht
nur bei der quantitativen Erfassung von Größen, sondern auch bei der
logischen Klärung der Gegenstände. Zum Beispiel
setzen wesentliche Teile der Analysis, der linearen Algebra und der
Stochastik den Begriff der reellen Zahlen voraus. Und dies sind die
Kerngebiete der Mathematik, die das Fundament für die Natur- und
Ingenieurwissenschaften bilden.

Den auf den Zahlenbereichen definierten arithmetischen Operationen
wohnen bestimmte Rechenregeln inne. Anfangs mag man diese Regeln im
Rahmen einer Theoriefindung klären und daraufhin zu Grundregeln gelangen,
die man axiomatisch voraussetzt. Stattdessen will ich die Zahlenbereiche
aber sogleich vermittels Mengenlehre modellieren und aufzeigen, dass
die Modelle die Grundregeln erfüllen. Die Zahlenbereiche werden dabei
schrittweise aufeinander aufgebaut.

Wir starten mit den natürlichen Zahlen.

\begin{Definition}[Peano-Axiome]\newlinefirst
Eine Menge $\N$ ist als Struktur $(\N,0,s)$ mit $0\in\N$ und $s\colon\N\to\N$
ein Modell der \emph{natürlichen Zahlen}, wenn gilt
\[\begin{array}{@{}ll@{}}
\text{(P1)} & \text{$s$ ist injektiv},\\[3pt]
\text{(P2)} & \forall n\in\N\colon s(n)\ne 0,\\[3pt]
\text{(P3)} & A(0)\land (\forall n\in\N\colon A(n)\cond A(s(n)))\cond (\forall n\in\N\colon A(n)).
\end{array}\]
\end{Definition}
Zusätzlich muss, wie bereits diskutiert, der Rekursionssatz vorausgesetzt
werden, um die arithmetischen Operationen definieren zu können. Das
Axiom (P3), das das Prinzip der Induktion beschreibt, habe ich als Schema
gefasst, um in der Logik erster Stufe verbleiben zu können.
Es vermittelt zu jeder Aussageform $A(n)$ ein Axiom.
Die ursprüngliche Fassung der Peano"=Axiome substituiert $A$ allerdings
gegen eine Prädikatvariable und setzt somit die Logik zweiter
Stufe voraus.

\begin{Definition}[Addition natürlicher Zahlen]%
\label{def:N-Addition}\newlinefirst
Die \emph{Addition} zweier natürlicher Zahlen ist rekursiv definiert als
\[a+0 := a,\qquad a+s(b) := s(a+b).\]
\end{Definition}
\begin{Definition}[Multiplikation natürlicher Zahlen]\newlinefirst
Die \emph{Multiplikation} zweier natürlicher Zahlen ist rekursiv definiert als
\[a\cdot 0 := 0,\qquad a\cdot s(b) := a\cdot b + a.\]
\end{Definition}

\noindent
Diese rekursiven Bildungsvorschriften lassen sich direkt so im Computer
implementieren. Gleichwohl ist das Prozedere außerordentlich
ineffizient und somit praktisch nur für sehr kleine Zahlen brauchbar.

\begin{Definition}[Ordnung der natürlichen Zahlen]\newlinefirst
Die \emph{Ordnung} der natürlichen Zahlen ist definiert als
\[a\le b\,:\bicond\,\exists n\in\N\colon a+n = b.\]
\end{Definition}

\noindent
Werden die natürlichen Zahlen als formales System dargestellt, in dem
gerechnet werden kann, entfällt der Rekursionssatz vorläufig aus der
Betrachtung. Dafür werden Addition und Multiplikation axiomatisch erklärt.
Man nennt dieses System die \emph{Peano"=Arithmetik}, kurz PA. Beschränkt
man sich dabei auf die intuitionistische Logik, spricht man von der
\emph{Heyting"=Arithmetik}, kurz HA.
\begin{Definition}[Peano-Arithmetik]\newlinefirst
Die Struktur $(\N,0,s,+,\cdot,\le)$ sei ein Modell der
\emph{Peano"=Arithmetik}, wenn $(\N,0,s)$ die Peano"=Axiome
erfüllt und zusätzlich für alle $a,b\in\N$ gilt
\[\begin{array}{@{}l@{\quad\;\;}l@{\quad\;\;}l@{}}
a+0 = a, & a+s(b) = s(a+b), &
a\le b \bicond \exists n\in\N\colon a+n = b.\\[3pt]
a\cdot 0 = 0, & a\cdot s(b) = a\cdot b + a.
\end{array}\]
\end{Definition}

\noindent
Für die natürlichen Zahlen finden sich diverse Modelle, darunter jenes
bereits diskutierte, das sie als endliche Ordinalzahlen darstellt.
Dahingehend tut sich die Frage auf, ob jedes dieser Modelle
gleichberechtigt ist.

\begin{Definition}[Isomorphismus zwischen Peano-Strukturen]%
\label{def:Isomorphismus-Peano}\newlinefirst
Eine Bijektion $f\colon\N\to\N'$ ist ein Isomorphismus zwischen
zwei Peano"=Strukturen $(\N,0,s)$ und $(\N',0',s')$, wenn
$f(0)=0'$ und $f(s(n))=s'(f(n))$ gilt.
\end{Definition}

\noindent
Man rechnet unschwer nach, dass die Umkehrabbildung eines Isomorphismus'
ebenfalls ein Isomorphismus ist.

\begin{Satz}[Isomorphiesatz von Dedekind]\newlinefirst
Je zwei Modelle der natürlichen Zahlen sind isomorph.
\end{Satz}
\begin{Beweis}
Es seien $(\N,0,s)$, $(\N',0',s')$ zwei Modelle der natürlichen Zahlen.
Vermittels des Rekursionssatzes definiert man
\[\begin{array}{lr@{\;}lr@{\;}l}
f\colon\N\to\N', & f(0) &:= 0', & f(s(n)) &:= s'(f(n)),\\[3pt]
g\colon\N'\to\N, & g(0') &:= 0, & g(s'(n)) &:= s(g(n)),\\[3pt]
h\colon\N\to\N, & h(0) &:= 0, & h(s(n)) &:= s(h(n)).
\end{array}\]
Man rechnet unschwer nach, dass $g\circ f = h$ und $h = \id$ gilt.
Analog erhält man $f\circ g = \id$, womit $f,g$ bijektiv sind.
Und per Definition sind $f,g$ Isomorphismen zwischen Peano"=Strukturen.\,\qedsymbol
\end{Beweis}

\noindent
Man macht sich außerdem klar, dass die rekursive Definition des
Isomorphismus in Def. \ref{def:Isomorphismus-Peano} enthalten ist.
Das heißt, der Isomorphismus ist zudem noch eindeutig bestimmt.
Insofern es also nur einen, und somit einen ausgezeichneten Isomorphismus
gibt, kann man sagen, dass je zwei Modelle der natürlichen Zahlen
kanonisch isomorph sind.

\begin{Satz}
Mit $1:=s(0)$ gilt $s(a)=a+1$.
\end{Satz}
\begin{Beweis}
Es findet sich $a+1 = a+s(0) = s(a+0) = s(a)$.\,\qedsymbol
\end{Beweis}

\begin{Satz}[Assoziativgesetz der Addition]\newlinefirst
Für alle $a,b,c\in\N$ gilt $(a+b)+c = a+(b+c)$.
\end{Satz}
\begin{Beweis}
Induktion über $c$. Im Anfang $c=0$ gilt
\[(a+b)+c = (a+b)+0 = a+b = a+(b+0) = a+(b+c).\]
Mit der Induktionsvoraussetzung $(a+b)+c = a+(b+c)$ findet sich
\[(a+b)+s(c) = s((a+b)+c) \stackrel{\mathrm{IV}}= s(a+(b+c))
= a+s(b+c) = a+(b+s(c)).\,\qedsymbol\]
\end{Beweis}

\begin{Satz}[Neutrales Element der Addition]%
\label{nat-zero}\newlinefirst
Für alle $a\in\N$ gilt $0+a = a+0 = a$.
\end{Satz}
\begin{Beweis}
Per Definition gilt $a+0=a$. Zu $0+a$ per Induktion über $a$.
Im Anfang $a=0$ ist $0+a = 0$ per Definition. Zum Schritt.
Induktionsvoraussetzung sei $0+a=a$. Man findet
\[0+s(a) = s(0+a) \stackrel{\mathrm{IV}}= s(a).\,\qedsymbol\]
\end{Beweis}

\begin{Satz}[Kommutativgesetz der Addition]\newlinefirst
Für alle $a,b\in\N$ gilt $a+b = b+a$.
\end{Satz}
\begin{Beweis}
Zunächst $a+1 = 1+a$ per Induktion über $a$. Im Anfang $a=0$ gilt
die Aussage gemäß Satz \ref{nat-zero}. Zum Schritt. Man findet
\[1+s(a) = s(1+a) \stackrel{\mathrm{IV}}= s(a+1) = a+s(1).\]
Nun $a+b=b+a$ per Induktion über $b$. Der Fall $b=0$
gilt gemäß Satz \ref{nat-zero}. Anfang sei $b=1$. Dieser
wurde zuvor gezeigt. Zum Schritt findet sich
\begin{align*}
s(b) + a &= (b+1)+a = b+(1+a) = b+(a+1) = b+s(a)\\
&= s(b+a)\stackrel{\mathrm{IV}}= s(a+b) = a+s(b).\,\qedsymbol
\end{align*}
\end{Beweis}

\begin{Satz}[Distributivgesetz der Multiplikation]\newlinefirst
Für alle $a,b,c\in\N$ gilt $(a+b)c = ac+bc$.
\end{Satz}
\begin{Beweis} Induktion über $c$. Im Anfang $c=0$ resultieren
beide Seiten der Gleichung im Wert~$0$. Zum Schritt findet sich
\[(a+b)s(c) = (a+b)c+(a+b) \stackrel{\mathrm{IV}}= ac+bc+a+b = ac+a+bc+b
= as(c) + bs(c).\qedsymbol\]
\end{Beweis}

\begin{Satz}\label{nat-mul-zero}
Für alle $a\in\N$ gilt $0a=0$.
\end{Satz}
\begin{Beweis}
Induktion über $a$. Im Anfang $a=0$ ist per Definition $0a=0$.
Der Schritt ist
\[0s(a) = 0a+0 \stackrel{\mathrm{IV}}= 0+0 = 0.\]
\end{Beweis}

\begin{Satz}[Neutrales Element der Multiplikation]%
\label{nat-mul-one}\newlinefirst
Für alle $a\in\N$ gilt $a\cdot 1=a$ und $1\cdot a=a$.
\end{Satz}
\begin{Beweis}
Die Formel $a\cdot 1 = a$ folgt unmittelbar aus der Definition
und bereits bekannten Regeln.

Die Formel $1\cdot a = a$ per Induktion über $a$. Im Anfang $a=0$ folgt
die Regel unmittelbar aus der Definition. Der Schritt ist
\[1s(a) = 1a+1 \stackrel{\mathrm{IV}}= a+1 = s(a).\,\qedsymbol\]
\end{Beweis}

\begin{Satz}[Kommutativgesetz der Multiplikation]\newlinefirst
Für alle $a,b\in\N$ gilt $ab=ba$.
\end{Satz}
\begin{Beweis}
Induktion über $b$. Im Anfang $b=0$ gilt die Regel gemäß
Lemma \ref{nat-mul-zero}. Der Schritt ist
\[as(b) = ab+a \stackrel{\mathrm{IV}}= ba+a = ba + 1a = (b+1)a = s(b)a.\,\qedsymbol\]
\end{Beweis}

\begin{Satz}
Aus $a\le b$ folgt $a+c\le b+c$.
\end{Satz}
\begin{Beweis}
Mit der Prämisse liegt ein $n$ mit $a+n = b$ vor. Somit folgt
$a+c+n = b+c$, womit $n$ auch ein Zeuge für $a+c\le b+c$ ist.\,\qedsymbol
\end{Beweis}

\section{Die ganzen Zahlen}

\subsection{Konstruktion}

\begin{Definition}[Ganze Zahlen]\newlinefirst
Auf $\N_0\times\N_0$ wird die folgende Äquivalenzrelation definiert:
\[(x,y)\sim (x',y')\,:\bicond\, x+y' = x'+y.\]
Die Quotientenmenge $\Z := (\N_0\times\N_0)/{\sim}$ nennt man
die \emph{ganzen Zahlen}.
\end{Definition}

\begin{Satz}[Ring der ganzen Zahlen]\newlinefirst
Die Operationen
\begin{align*}
[(x,y)]+[(x',y')] &:= [(x+x',y+y')],\\
[(x,y)]\cdot [(x',y')] &:= [(xx'+yy',xy'+x'y)]
\end{align*}
sind auf $\Z$ wohldefiniert und $(\Z,+,\cdot)$ bildet einen
kommutativen unitären Ring.
\end{Satz}
\strong{Beweis.} Wohldefiniert heißt, dass das Ergebnis der Operationen
nicht von den gewählten Repräsentanten der Argumente abhängig ist.
Sei dazu $(x,y)\sim(a,b)$ und
$(x',y')\sim (a',b')$. Zu zeigen ist nun
\[(x+x',y+y')\sim (a+a',b+b'),\]
was laut Definition zu
\[(x+x')+(b+b') = (a+a')+(y+y').\]
äquivalent ist. Gemäß Voraussetzung ist $x+b=a+y$ und $x'+b'=a'+y'$.
Man bekommt damit auf der linken Seite
\[x+x'+b+b' = x+b+x'+b' = a+y+a'+y',\]
was wiederum mit der rechten Seite übereinstimmt.

Mit der Multiplikation verhält es sich etwas komplizierter.
Zu Vereinfachung wird zunächst gezeigt:
\begin{gather*}
[(x,y)]\cdot [(x',y')] = [(a,b)]\cdot [(x',y')],\\
\iff (xx'+yy',xy'+yx')\sim (ax'+by',ay'+bx')\\
\iff xx'+yy' + ay'+bx' = ax'+by' + xy'+yx'\\
\iff (x+b)x' + (a+y)y' = (a+y)x' + (x+b)y'.
\end{gather*}
Diese Gleichung ist gemäß Voraussetzung $(x,y)\sim (a,b)$
bzw. $x+b=a+y$ erfüllt.

Analog bestätigt man
\[[(a,b)]\cdot [(x',y')] = [(a,b)]\cdot [(a',b')].\]
Gemäß Transitivität ergibt sich somit
\[[(x,y)]\cdot [(x',y')] = [(a,b)]\cdot [(a',b')].\]
Es ist nun zu bestätigen, dass $(\Z,+)$ eine kommutative Gruppe ist.
Das Assoziativgesetz:
\begin{gather*}
([(x,y)]+[(x',y')])+[(x'',y'')]
= [(x+x',y+y')] + [(x'',y'')]\\
= [(x+x'+x'',y+y'+y'')]
= [(x,y)]+[(x'+x'',y'+y'')]\\
= [(x,y)]+([(x',y')]+[(x'',y'')]).
\end{gather*}
Das neutrale Element ist $[(0,0)]$:
\[[(x,y)]+[(0,0)] = [(x+0,y+0)] = [(x,y)].\]
Das inverse Element zu $[(x,y)]$ ist $[(y,x)]$, denn es gilt
\begin{gather*}
[(x,y)]+[(y,x)] = [(x+y,y+x)] = [(0,0)]\\
\iff (x+y,y+x)\sim (0,0)\iff x+y+0 = y+x+0.
\end{gather*}
Das Kommutativgesetz:
\[[(x,y)]+[(x',y')] = [(x+x',y+y')] = [(x'+x,y'+y)]
= [(x',y')]+[(x,y)].\]
Es ist nun zu bestätigen, dass $(\Z,\cdot)$ ein kommutatives
Monoid bildet. Das Assoziativgesetz:
\begin{gather*}
([(x,y)]\cdot [(x',y')])\cdot [(x'',y'')]
= [(xx'+yy',xy'+x'y)]\cdot [(x'',y'')]\\
= [(xx'x''+x''yy'+xy'y''+x'yy'',\;
xx'y''+yy'y''+xx''y'+x'x''y)]\\
= [(x,y)]\cdot [(x'x''+y'y'',x'y''+x''y')]
= [(x,y)]\cdot ([(x',y')]\cdot [(x'',y'')]).
\end{gather*}
Das Kommutativgesetz:
\begin{gather*}
[(x,y)]\cdot [(x',y')] = [(xx'+yy',\;xy'+yx')]\\
= [(x'x+y'y,\;x'y+xy')] = [(x',y')]\cdot [(x,y)].
\end{gather*}
Das neutrale Element ist $[(1,0)]$, denn es gilt
\[[(x,y)]\cdot [(1,0)] = [(x\cdot 1+y\cdot 0,\;1\cdot y+x\cdot 0)]
= [(x,y)].\]
Schließlich ist noch das Distributivgesetz zu bestätigen.
Man findet
\begin{gather*}
[(a,b)]\cdot ([(x,y)]+[(x',y')])
= [(a,b)]\cdot [(x+x',y+y')]\\
= [(ax+ax'+by+by',\;ay+ay'+bx+bx')]\\
= [(ax+by,ay+bx)]+[(ax'+by',ay'+bx')]\\
= [(a,b)]\cdot [(x,y)] + [(a,b)]\cdot [(x',y')].
\end{gather*}
Somit sind alle Axiome bestätigt.\;\qedsymbol

\begin{Definition}[Monoidhomomorphismus]\newlinefirst
Seien $(M,+)$ und $(M',+')$ zwei Monoide. Eine Abbildung
$\varphi\colon M\to M'$ heißt Monoidhomomorphismus, wenn
für alle $a,b\in M$ gilt
\[\varphi(a+b) = \varphi(a)+\varphi(b)\]
und $\varphi(0)=0'$ ist.
\end{Definition}
Einen injektiven Homomorphismus nennt man Monomorphismus. Ein
Monomorphismus charakterisiert eine Einbettung einer Struktur als
Unterstruktur einer anderen.

\begin{Satz}[Einbettung der natürlichen Zahlen in die ganzen]%
\label{embedding-nat-int}\mbox{}\\*
Die Abbildung $\varphi\colon\N_0\to\Z$ mit $\varphi(n):=[(n,0)]$
ist ein Monoidmonomorphismus.
\end{Satz}
\strong{Beweis.} Es ergibt sich
\[\varphi(a+b) = [(a+b,0)] = [(a,0)]+[(b,0)] = \varphi(a)+\varphi(b).\]
Außerdem ist $\varphi(0)=[(0,0)]$, und $[(0,0)]$
ist das neutrale Element von $(\Z,+)$.

Schließlich ist noch die Injektivität zu prüfen:
\begin{gather*}
[(a,0)] = \varphi(a) = \varphi(b)  = [(b,0)]\iff (a,0)\sim (b,0)\\
\iff a+0 = b+0 \iff a=b.\;\qedsymbol
\end{gather*}
Anstelle von $\varphi(n)=[(n,0)]$ darf man daher einfach schreiben
$n=[(n,0)]$. Außerdem definiert man $a-b:=a+(-b)$. Daraus
ergibt sich nun
\[[(x,y)] = [(x,0)]+[(0,y)] = [(x,0)] - [(y,0)] = x-y.\]
Die umständliche Schreibweise $[(x,y)]$ wird ab jetzt nicht
mehr benötigt.

\begin{Definition}[Totalordnung der ganzen Zahlen]\newlinefirst
Auf $\Z$ wird die folgende strenge Totalordnung definiert:
\[[(x,y)] < [(x',y')]\,:\bicond\, x+y'<x'+y.\]
\end{Definition}

\begin{Satz}[Einbettung der Totalordnung]\newlinefirst
Die Abbildung $\varphi$ aus Satz \ref{embedding-nat-int}
genügt der Forderung
\[a<b\,\cond\, \varphi(a)<\varphi(b).\]
\end{Satz}
\strong{Beweis.} Nach den Definitionen ist
\[\varphi(a)<\varphi(b)\iff [(a,0)]<[(b,0)]\iff a+0<0+b\iff a<b.\;\qedsymbol\]


\section{Die rationalen Zahlen}

\subsection{Konstruktion}

\begin{Definition}[Rationale Zahlen]\newlinefirst
Auf $\Z\times\N_1$ wird die folgende Äquivalenzrelation definiert:
\[(x,y)\sim (x',y')\,:\bicond\, xy' = x'y.\]
Die Quotientenmenge $\Q := (\Z\times\N_1)/{\sim}$ nennt man
die \emph{rationalen Zahlen}.
\end{Definition}
Für die Äquivalenzklasse $[(x,y)]$ schreibt man $\frac{x}{y}$.

\begin{Satz}[Körper der rationalen Zahlen]\newlinefirst
Die Operationen
\[\frac{x}{y}+\frac{x'}{y'} := \frac{xy'+x'y}{yy'},
\qquad\frac{x}{y}\cdot \frac{x'}{y'} := \frac{xx'}{yy'}\]
sind auf $\Q$ wohldefiniert und $(\Q,+,\cdot)$ bildet einen Körper.
\end{Satz}
\strong{Beweis.} Wohldefiniert bedeutet, dass das Ergebnis der
Operationen nicht von den gewählten Repräsentanten der Argumente
abhängig ist. Sei dazu $(a,b)\sim (x,y)$ und
$(a',b')\sim (x',y')$. Zu zeigen ist nun
\begin{align*}
&(ab'+a'b,bb')\sim (xy'+x'y,yy'),\\
&\iff (ab'+a'b)(yy') = (xy'+x'y)(bb')\\
&\iff ab' yy' + a'byy' = xy'bb'+x'ybb'.
\end{align*}
Substituiert man $ay=xb$ und $a'y'=x'b'$ auf
der linken Seite der Gleichung, ergibt sich die rechte Seite.
Zu zeigen ist weiterhin
\begin{align*}
(aa',bb')\sim (xx',yy')
\iff aa'yy' = xx'bb'.
\end{align*}
Wieder wird linke Seite der Gleichung über $ay=xb$
und $a'y'=x'b'$ in die rechte Seite überführt.
Die Wohldefiniertheit der Operationen ist damit bestätigt.

Es bleibt zu prüfen, dass $(\Q,+,\cdot)$ allen Körperaxiomen genügt.
Das neutrale Element der Addition ist $0/1$, denn es gilt
\[\frac{x}{y}+\frac{0}{1} = \frac{x\cdot 1+0\cdot y}{y\cdot 1} = \frac{x}{y}.\]
Das neutrale Element der Multiplikation ist $1/1$, denn es gilt
\[\frac{x}{y}\cdot\frac{1}{1} = \frac{x\cdot 1}{y\cdot 1} = \frac{x}{y}.\]
Die Assoziativität der Addition ergibt sich ohne größere Umstände:
\begin{align*}
\bigg(\frac{x}{y}+\frac{x'}{y'}\bigg)+\frac{x''}{y''}
&= \frac{xy'+x'y}{yy'} + \frac{x''}{y''}
= \frac{xy'y''+x'yy''+x''yy'}{yy'y''},\\
\frac{x}{y}+\bigg(\frac{x'}{y'}+\frac{x''}{y''}\bigg)
&= \frac{x}{y}+\frac{x'y''+x''y'}{y'y''}
= \frac{xy'y''+x'yy''+x''yy'}{yy'y''}.
\end{align*}
Die Assoziativität der Multiplikation ist etwas einfacher:
\[\bigg(\frac{x}{y}\cdot\frac{x'}{y'}\bigg)\cdot\frac{x''}{y''}
= \frac{xx'}{yy'}\cdot\frac{x''}{y''} = \frac{xx'x''}{yy'y''}
= \frac{x}{y}\cdot\frac{x'x''}{y'y''}
= \frac{x}{y}\cdot\bigg(\frac{x'}{y'}\cdot\frac{x''}{y''}\bigg).\]
Das Kommutativgesetz der Addition:
\[\frac{x}{y}+\frac{x'}{y'} = \frac{xy'+x'y}{yy'}
= \frac{x'y+xy'}{y'y}
= \frac{x'}{y'}+\frac{x}{y}.\]
Das Kommutativgesetz der Multiplikation:
\[\frac{x}{y'}\cdot\frac{x'}{y'}
= \frac{xx'}{yy'} = \frac{x'x}{y'y}
= \frac{x'}{y'}\cdot\frac{x}{y}.\]
Das additiv inverse Element zu $x/y$ ist $(-x)/y$, denn es gilt
\[\frac{x}{y}+\frac{-x}{y} = \frac{xy+(-x)y}{y^2}
= \frac{0}{y^2} = \frac{0}{1}.\]
Das multiplikativ inverse Element zu $x/y$ mit $x\ne 0$
ist $y/x$, denn es gilt
\[\frac{x}{y}\cdot\frac{y}{x} = \frac{xy}{xy} = \frac{1}{1}.\]
Schließlich findet bestätigt man noch das Distributivgesetz:
\begin{align*}
&\frac{a}{b}\cdot\bigg(\frac{x}{y}+\frac{x'}{y'}\bigg)
= \frac{a}{b}\cdot\frac{xy'+x'y}{yy'}
= \frac{axy'+ax'y}{byy'},\\
&\frac{ax}{by}+\frac{ax'}{by'}
= \frac{axby'+ax'by}{byby'}
= \frac{b}{b}\cdot\frac{axy'+ax'y}{byy'}.
\end{align*}
Hierbei beachtet man, dass $b/b=1/1$ das multiplikativ
neutrale Element ist.\;\qedsymbol

\begin{Definition}[Ringhomomorphismus]\newlinefirst
Seien $(R,+,*)$ und $(R',+',*')$ zwei Ringe. Die Abbildung
$\varphi\colon R\to R'$ heißt \emph{Ringhomomorphismus}, wenn für alle
$a,b\in R$ gilt:
\begin{align*}
\varphi(a+b) = \varphi(a)+'\varphi(b),\qquad
\varphi(a*b) = \varphi(a)*'\varphi(b).
\end{align*}
Besitzt $R$ ein Einselement $1$ und $R'$ ein Einselement $1'$,
dann nennt man $\varphi$ \emph{Eins"=erhaltend}, wenn $\varphi(1)=1'$ ist.
\end{Definition}
Einen injektiver Homomorphismus wird Monomorphismus genannt. Ein
Monomorphismus charakterisiert eine Einbettung einer Unterstruktur
in eine andere Struktur.

\newpage
\begin{Satz}[Einbettung der ganzen Zahlen in die rationalen]\newlinefirst
Sei $\varphi\colon\Z\to\Q$ mit $\varphi(z):=z/1$. Die
Abbildung $\varphi$ ist ein Eins"=erhaltender Ringmonomorphismus.
\end{Satz}
\strong{Beweis.} Die Erhaltung des Einselements ergibt sich
trivial. Ferner findet man
\[\varphi(a+b) = \frac{a+b}{1} = \frac{a\cdot 1+b\cdot 1}{1\cdot 1}
= \frac{a}{1}+\frac{b}{1} = \varphi(a)+\varphi(b)\]
und
\[\varphi(ab) = \frac{ab}{1} = \frac{ab}{1\cdot 1} = \frac{a}{1}\cdot\frac{b}{1}
= \varphi(a)\cdot\varphi(b).\;\qedsymbol\]
Gemäß der Einbettung können wir die ganze Zahl $z$ ab jetzt
mit der rationalen Zahl $z/1$ identifizieren. Das heißt, man schreibt
einfach $z=z/1$ statt $\varphi(z)=z/1$.

\begin{Definition}[Division rationaler Zahlen]\newlinefirst
Wie in jedem Körper ist die Division für $a,b\in\Q$
definiert als $a/b := ab^{-1}$.
\end{Definition}
Die Division ist also gerade die Multiplikation des Kehrwertes
des Nenners:
\[\frac{x}{y}/\frac{x'}{y'} = \frac{x}{y}\cdot\frac{y'}{x'}.\]
Die Division muss natürlich mit der Notation für rationale Zahlen
kompatibel sein, sonst dürfte man nicht die gleiche Schreibweise
verwenden. Zur Unterscheidung schreiben wir Division für einen
Augenblick mit Doppelstrich als $a\doubleslash b$. Man findet
\[\frac{x}{y} = \frac{x}{1}\cdot\frac{1}{y}
= \frac{x}{1}\doubleslash\frac{y}{1} = x\doubleslash y.\]
Tatsächlich führt beides zum gleichen Ergebnis.

Da die rationalen Zahlen einen Körper bilden, gilt $a/a=aa^{-1}=1$
für jede rationale Zahl $a$.

\begin{Satz}[Addition, Subtraktion, Multiplikation von Brüchen]\newlinefirst
Seien $a,b,c,d$ rationale Zahlen mit $b\ne 0$ und $d\ne 0$. Es gilt
\[\frac{a}{b}+\frac{c}{d} = \frac{ad+bc}{bd},
\qquad \frac{a}{b}-\frac{c}{d} = \frac{ad-bc}{bd},
\qquad \frac{a}{b}\cdot\frac{c}{d} = \frac{ac}{bd}.\]
\end{Satz}
Der Beweis wird dem Leser überlassen.

\newpage
\subsection{Beträge rationaler Zahlen}

\begin{Definition}[Signum]\newlinefirst
Für jede rationale Zahl $x$ setzt man
\[\sgn(x) := \left\{\begin{array}{@{}rl@{}}
1, & \mathrm{wenn}\; x>0,\\
0, & \mathrm{wenn}\; x=0,\\
-1, & \mathrm{wenn}\;x<0.
\end{array}\right.\]
\end{Definition}
Vermittels Fallunterscheidung rechnet man unschwer nach,%
\[\sgn(xy) = \sgn(x)\sgn(y).\]
Für $x\ne 0$ ist außerdem $\sgn(x)^2=1$ unmittelbar einsichtig,
allgemein gilt demnach $\sgn(x)^2=[x\ne 0]$. Infolge
erhält man $\tfrac{1}{\sgn(x)}=\sgn(x)$ für $x\ne 0$, und somit%
\[\sgn(\tfrac{1}{x}) = \tfrac{1}{\sgn(x)} = \sgn(x),
\;\text{denn}\;1 = \sgn(1) = \sgn(x\cdot\tfrac{1}{x})
= \sgn(x)\sgn(\tfrac{1}{x}).\]

\begin{Definition}[Betrag]\newlinefirst
Man setzt $|x| := \sgn(x)\cdot x$ für jede rationale Zahl $x$.
\end{Definition}

\noindent
Aus der Definition leiten sich einige elementare Rechenregeln ab. Es folgt%
\[|xy| = \sgn(xy)xy = \sgn(x)\sgn(y)xy
= \sgn(x)x\sgn(y)y = |x||y|.\]
Multipliziert man die definierende Gleichung im Fall $x\ne 0$ auf
beiden Seiten mit $\sgn(x)$, erhält man außerdem%
\[\sgn(x)|x| = \sgn(x)^2\cdot x = 1\cdot x = x.\]
Im Fall $x=0$ ist $x=\sgn(x)|x|$ aber auch erfüllt. Dies bedeutet, dass
sich jede Zahl in ihren Betrag und ihr Vorzeichen zerlegen lässt, wobei
der Null kein Vorzeichen zukommen braucht.


\begin{Satz}[Dreiecksungleichung]\newlinefirst
Die Ungleichung $|x+y| \le |x| + |y|$ gilt für alle rationalen Zahlen $x,y$.
\end{Satz}
\begin{Beweis}
Für jede rationale Zahl $q$ gilt $q\le |q|$. Mit der Setzung $q:=xy$ erhält man
$xy\le |xy| = |x||y|$. Zusammen mit $|x+y|^2 = (x+y)^2$ sowie $|x|^2 = x^2$
und $|y|^2 = y^2$ ergibt sich somit
\begin{align*}
|x+y|^2 = x^2+2xy+y^2 \le x^2 + 2|xy| + y^2 = (|x|+|y|)^2.
\end{align*}
Weil Quadrieren auf den nichtnegativen rationalen Zahlen eine monoton
steigende Funktion ist, und somit die Ordnung erhält, gelangt man zur
gesuchten Ungleichung letztlich mit der Äquivalenz%
\[|x+y| \le |x| + |y| \,\Leftrightarrow\, |x+y|^2 \le (|x|+|y|)^2.\,\qedsymbol\]
\end{Beweis}

\newpage
\section{Die reellen Zahlen}

\subsection{Konstruktion}

\begin{Definition}[Offene Kugel]\newlinefirst
Zu $\varepsilon\in\Q_{\ge 0}$ und $q\in\Q$ nennt man
$U_\varepsilon(q):=\{p\in\Q\mid |p-q|<\varepsilon\}$
die \emph{offene Kugel} vom Radius $\varepsilon$ mit Zentrum $q$ oder
kurz die $\varepsilon$-Umgebung von $q$.
\end{Definition}

\begin{Definition}[Konvergente rationale Folge]\newlinefirst
Eine Folge $x\colon\N\to\Q$ heißt \emph{konvergent} gegen $L$, falls
\[\forall\varepsilon\in\Q_{>0}\colon\exists n_0\in\N\colon\forall n\ge n_0\colon |x_n-L| < \varepsilon.\]
\end{Definition}

\begin{Satz}
Eine Folge rationaler Zahlen konvergiert gegen höchstens einen Grenzwert.
\end{Satz}
\begin{Beweis}
Sei $x\colon\N\to\Q$ konvergent gegen $L_1,L_2$. Angenommen, es wäre $L_1\ne L_2$.
Wir wählen $\varepsilon$ so klein, dass die Umgebungen $U_\varepsilon(L_1)$
und $U_\varepsilon(L_2)$ disjunkt sind. Nun liegt ein $n_1$ vor,
so dass $x_n\in U_\varepsilon(L_1)$ für $n\ge n_1$. Entsprechend ein
$n_2$, so dass $x_n\in U_\varepsilon(L_2)$ für $n\ge n_2$. Für
$n:=\max(n_1,n_2)$ wäre $x_n$ absurderweise in beiden Umgebungen.
Also muss die Annahme falsch sein.\,\qedsymbol
\end{Beweis}

\begin{Definition}[Grenzwert]\newlinefirst
Konvergiert die Folge $x$ gegen $L$, dann definieren wir ihren \emph{Grenzwert} als
\[\lim_{n\to\infty} x_n := L.\]
\end{Definition}

\begin{Definition}[Cauchy-Folge]\newlinefirst
Eine Folge $x\colon\N\to\Q$ heißt \emph{Cauchy"=Folge}, falls
\[\forall\varepsilon\in\Q_{>0}\colon\exists n_0\in\N\colon
\forall m\ge n_0\colon\forall n\ge n_0\colon |x_m - x_n| < \varepsilon.\]
\end{Definition}

\begin{Definition}[Reelle Zahlen]\newlinefirst
Auf $C_\Q$, der Menge aller Cauchy"=Folgen auf $\Q$, definiert man
die \emph{reellen Zahlen} als die Quotientenmenge $\R:=C_\Q/{\sim}$
bezüglich der Äquivalenzrelation
\[x\sim y \;:\Leftrightarrow\; \lim_{n\to\infty}(x_n-y_n) = 0\]
und legt Addition und Multiplikation reeller Zahlen fest als
\[[x]+[y] := [x+y],\quad\;\; [x]\cdot [y] := [x\cdot y].\]
\end{Definition}

\begin{Satz}
Sind $x,y$ Cauchy"=Folgen, so ist $(x+y)_n := x_n+y_n$ ebenfalls eine.
\end{Satz}
\begin{Beweis}
Die beiden Voraussetzungen sind
\begin{gather*}
\forall\varepsilon_1\in\Q_{>0}\colon\exists n_1\in\N\colon
  \forall m\ge n_1\colon\forall n\ge n_1\colon |x_m-x_n|<\varepsilon_1,\\
\forall\varepsilon_2\in\Q_{>0}\colon\exists n_2\in\N\colon
  \forall m\ge n_2\colon\forall n\ge n_2\colon |y_m-y_n|<\varepsilon_2.
\end{gather*}
Es sei $\varepsilon\in\Q_{>0}$ fest, aber beliebig. Damit wird mit
$\varepsilon_1:=\tfrac{1}{2}\varepsilon$ und $\varepsilon_2:=\tfrac{1}{2}\varepsilon$
spezialisiert. Man setze $n_0:=\max(n_1,n_2)$, daraufhin gilt für $m\ge n_0$ und $n\ge n_0$ sowohl
$|x_m-x_n|<\tfrac{1}{2}\varepsilon$ als auch $|y_m-y_n|<\tfrac{1}{2}\varepsilon$.
Vermittels der Dreiecksungleichung findet sich nun die gesuchte Ungleichung
\[|(x+y)_m - (x+y)_n| = |(x_m-x_n) + (y_m-y_n)| \le |x_m-x_n| + |y_m-y_n| < \varepsilon.\,\qedsymbol\]
\end{Beweis}

\begin{Satz}
Jede Cauchy"=Folge ist beschränkt.
\end{Satz}
\begin{Beweis}
Sei $x$ eine Cauchy"=Folge. Spezialisiert man diese Voraussetzung mit
$\varepsilon:=1$, erhält man ein $n_0$, so dass $|x_m-x_n|<1$ für
$m\ge n_0$ und $n\ge n_0$ gelten muss. Vermittels der umgekehrten
Dreiecksungleichung folgt daraus
\[|x_m|-|x_n| \le |x_m-x_n| < 1,\;\text{also}\; |x_m| < |x_{n_0}| + 1.\]
Damit ist die Folge zumindest ab $n_0$ beschränkt. Aber für
$m\le n_0$ ist das Maximum $M:=\max_{m\le n_0} |x_m|$ schlicht eine
Schranke. Somit ist $x$ mit $|x_m| < M + 1$ für jedes $m\in\N$
eine beschränkte Folge.\,\qedsymbol
\end{Beweis}

\begin{Satz}
Sind $x,y$ Cauchy"=Folgen, so ist $(x\cdot y)_n := x_n\cdot y_n$ ebenfalls eine.
\end{Satz}
\begin{Beweis}
Mit dem Ansatz
\[(xy)_m - (xy)_n = x_m y_m - x_n y_n = (x_m y_m - x_m y_n) + (x_m y_n - x_n y_n)\]
und der Dreiecksungleichung erhält man
\[|(xy)_m - (xy)_n| \le |x_m y_m - x_m y_n| + |x_m y_n - x_n y_n|
= |x_m||y_m-y_n| + |y_n||x_m-x_n|.\]
Wir setzen wieder $n_0:=\max(n_1,n_2)$, so dass sowohl $|x_m-x_n|<\varepsilon$
als auch $|y_m-y_n|<\varepsilon$ für $m\ge n_0$ und $n\ge n_0$. Weil $x,y$
beschränkt sind, existiert außerdem eine Schranke $s$ mit $|x_m|\le s$
für jedes $m$ und $|y_n|\le s$ für jedes $n$. Ergo hat man
\[|x_m||y_m-y_n| + |y_n||x_m-x_n| < |x_m|\varepsilon + |y_n|\varepsilon
\le s\varepsilon + s\varepsilon = 2s\varepsilon.\]
Ist $\varepsilon'$ nun fest, aber beliebig, kann man mit
$\varepsilon:=\frac{\varepsilon'}{2s}$ spezialisieren. Für jedes $\varepsilon'\in\Q_{>0}$
existiert also $n_0\in\N$ mit $|(xy)_m - (xy)_n| < \varepsilon'$
für $m\ge n_0$ und $n\ge n_0$.\,\qedsymbol
\end{Beweis}

\begin{Satz}
Die Addition und Multiplikation reeller Zahlen ist wohldefiniert.
\end{Satz}
\begin{Beweis}
Es gelte $x\sim x'$ und $y\sim y'$. Zu zeigen ist $x+y\sim x'+y'$
und $x\cdot y\sim x'\cdot y'$. Zu jedem $\varepsilon$ existiert laut
den Voraussetzungen ein $n_0$ mit $|x_n-x_n'| < \varepsilon$
und $|y_n-y_n'|<\varepsilon$ für jedes $n\ge n_0$. Vermittels der
Dreiecksungleichung erhält man somit die Ungleichung
\[|(x+y)_n - (x'+y')_n| = |x_n-x_n' + y_n-y'_n|
\le |x_n-x_n'| + |y_n-y_n'| < 2\varepsilon.\]
Also unterbietet die Folge $(x+y)-(x'+y')$ jedes $\varepsilon'\in\Q_{>0}$
bezüglich $\varepsilon:=\tfrac{1}{2}\varepsilon'$, womit sie ebenfalls
gegen null konvergiert.

Zu den Produktfolgen macht man wieder den Ansatz
\[(xy)_n - (x'y')_n = x_n y_n - x_n' y_n' = (x_n y_n - x_n' y_n) + (x_n' y_n - x_n' y_n').\]
Mit der Dreiecksungleichung ergibt sich daher
\[|(xy)_n - (x'y')_n| \le |y_n||x_n-x_n'| + |x_n'||y_n-y_n'|
< |y_n|\varepsilon + |x_n'|\varepsilon \le 2s\varepsilon,\]
wobei $s$ wieder eine Schranke ist, so dass $|y_n|<s$ und $|x_n'|<s$ für jedes $n\in\N$.
Sie existiert, weil $y_n$ und $x_n'$ als Cauchy"=Folgen beschränkt sind.
Also unterbietet die Folge $xy-x'y'$ jedes $\varepsilon'\in\Q_{>0}$
bezüglich $\varepsilon:=\tfrac{\varepsilon'}{2s}$, womit sie ebenfalls
gegen null konvergiert.\,\qedsymbol
\end{Beweis}

\noindent
Im Weiteren gilt es zu prüfen, dass die gemachte Konstruktion der
reellen Zahlen die Körperaxiome erfüllt. Hierzu muss gezeigt werden,
dass jede reelle Zahl ungleich null einen Kehrwert besitzt.

\begin{Satz}
Jede reelle Zahl ungleich null besitzt ein multiplikatives Inverses.
\end{Satz}
\begin{Beweis}
Es sei $a\in\R$ mit $a\ne 0$. Also existiert eine Cauchy"=Folge $x$
mit $[x] = a$. Wegen $a\ne 0$ muss $x_n\ne 0$ ab einem $n_0$ gelten.
Wir würden nun gerne die Folge der Kehrwerte bezüglich $x$ betrachten.
Allerdings kann $x$ Nullstellen haben. Wir setzen daher
\[u_n := \begin{cases}
1, & \text{wenn $n<n_0$},\\
x_n, & \text{sonst}.
\end{cases}\]
Es gilt $u_n\ne 0$ für jedes $n$.
Wir setzen $b:=[y]$ mit $y_n := \tfrac{1}{u_n}$. Es ergibt sich
\[x_n y_n = \tfrac{x_n}{1}[n<n_0] + \tfrac{x_n}{x_n}[n\ge n_0]
= 1[n\ge n_0].\]
Ergo gilt $ab = [xy] = 1$. Also ist $b$ ein multiplikatives Inverses
von $a$. Es verbleibt zu bestätigen, dass $y$ eine Cauchy"=Folge ist.
Es sei $\varepsilon\in\Q_{>0}$ beliebig. Wegen $u_n\ne 0$ und
$\lim_{n\to\infty}u_n \ne 0$ existiert ein $\delta>0$ mit $|u_n|>\delta$.
Weil $u$ eine Cauchy"=Folge ist, existiert $k$ mit
$|u_m-u_n|<\delta^2\varepsilon$ für alle $m,n\ge k$. Wir wählen dieses
$k$ für die Existenzaussage. Mit $\frac{1}{|u_m|}<\frac{1}{\delta}$ und
$\frac{1}{|u_n|}<\frac{1}{\delta}$ erhält man nämlich
\[|y_m-y_n| = \Big|\frac{1}{u_m}-\frac{1}{u_n}\Big|
= \frac{|u_m-u_n|}{|u_m u_n|}
< \frac{\delta^2\varepsilon}{\delta^2} = \varepsilon.\,\qedsymbol\]
\end{Beweis}
