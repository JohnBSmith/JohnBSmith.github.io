
\chapter{Mengenlehre}

\section{Grundbegriffe}

\subsection{Der Mengenbegriff}

Eine \emph{Menge} darf man sich wie einen Beutel vorstellen, der
einzelne Objekte enthält. Die Objekte heißen \emph{Elemente} der Menge.
Jedoch gilt es hierbei zu beachten, dass es sich mit einer Menge nicht
gänzlich wie mit einem Beutel verhält, in dem dasselbe Objekt mehrmals
zu finden sein kann.
\begin{quote}
»\emph{Unter einer Menge verstehen wir jede Zusammenfassung von
bestimmten wohlunterschiedenen Objekten unserer Anschauung oder
unseres Denkens zu einem Ganzen.}«\\
--- Georg Cantor, 1895 (redigiert aus \cite{Cantor})
\end{quote}
Obwohl blumig anmutend, fassen diese Worte das Konzept recht gut
auf den Punkt. Wichtig ist hier das Wort \emph{wohlunterschieden},
das uns zu verstehen gibt, dass ein Element nicht mehrmals in einer
Menge enthalten sein kann. \emph{Eine Menge ist genau dadurch festgelegt,
welche Elemente sie enthält. Sie enthält Elemente weder mehrmals, noch
in einer bestimmten Reihenfolge.}

Es gibt die \emph{leere Menge}, notiert als $\emptyset$ oder $\{\}$.
Man darf sie sich wie einen leeren Beutel vorstellen. Dagegen enthält
die Menge $\{\emptyset\}$ genau ein Element. Es ist ein Beutel, 
der den leeren Beutel enthält.

Wir schreiben kurz $x\in A$ für »$x$ ist ein Element von $A$«,
auch »$x$ gehört zu $A$« oder »$x$ liegt in $A$«.
Es steht $x\notin A$ für $\lnot x\in A$.

Für eine endliche Menge definiert man
\[x\in\{x_1,\ldots,x_n\}\defiff x=x_1\lor\ldots\lor x=x_n.\]

\subsection{Gleichheit von Mengen}

Es wurde gesagt, eine Menge ist dadurch charakterisiert, welche
Elemente sie enthält. Um diesen Gedanke näher zu erfassen, sollten
wir klären, wie es sich mit der Gleichheit von Mengen verhält.
Insofern Mengen durch ihre Elemente bestimmt sind, darf man doch sagen,
zwei Mengen $A,B$ sind genau dann gleich, falls $A,B$ ein Objekt
gemeinsam enthalten oder gemeinsam nicht enthalten. Das heißt, betrachtet
man ein beliebiges Objekt $x$, so ist $x$ genau dann in $A$ enthalten,
wenn $x$ in $B$ enthalten ist.
\begin{Definition}[Gleichheit von Mengen]\newlinefirst
Für zwei Mengen $A,B$ definiert man
\[A = B \defiff (\forall x\colon x\in A\bicond x\in B).\]
\end{Definition}
Der Leser wird mühelos bestätigen, dass die Gleichheit die Axiome
einer Äquivalenzrelation erfüllt. Wobei der Begriff der Relation
noch zu definieren wäre. Weil die Gesamtheit aller Mengen eine
sogenannte echte Klasse ist, handelt es sich nicht um eine
Äquivalenzrelation im engeren Sinne.

Es ist zulässig, ein Element bei der aufzählenden Angabe einer Menge
mehrmals aufzuführen. Dies ändert allerdings nichts daran, dass ein
Element stets nur einmal in einer Menge enthalten ist. Zum Beispiel
gilt $\{\emptyset,\emptyset\}=\{\emptyset\}$. Mit der Definition der
aufzählenden Angabe und dem Idempotenzgesetz
der Aussagenlogik findet sich nämlich die äquivalente Umformung
\[x\in\{\emptyset,\emptyset\}\iff x=\emptyset\lor x=\emptyset
\iff x=\emptyset\iff x\in\{\emptyset\}.\]

\subsection{Beschränkte Quantifizierung}

In der Mathematik erstreckt sich die Quantifizierung meist nicht
über das gesamte Diskursuniversum, sondern bleibt auf eine bestimmte
Menge beschränkt. Eine extra Notation macht dies ergonomisch, wobei
eine Erweiterung der logischen Sprache hierfür nicht nötig ist. Die
beschränkte Quantifizierung wird logisch auf eine unbeschränkte
zurückgeführt.

\begin{Definition}[Beschränkte Quantifizierung]\newlinefirst
Für jede Menge $M$ und jede Aussageform $A(x)$ setzt man
\begin{gather*}
(\forall x\in M\colon A(x))\defiff (\forall x\colon x\in M\cond A(x)),\\
(\exists x\in M\colon A(x))\defiff (\exists x\colon x\in M\land A(x)).
\end{gather*}
\end{Definition}
Die Aussage $\forall x\in\emptyset\colon A(x)$ ist allgemeingültig, man
spricht von der \emph{leeren Wahrheit}, engl. \emph{vacuous truth}.
Via Ex falso quodlibet erhält man nämlich:
\[
\infer{\vdash\forall x\colon x\in\emptyset\cond A(x)}{
  \infer[\infernote{EFQ}]{x\in\emptyset\vdash A(x)}{
    \infer{x\in\emptyset\vdash\bot}{
      \infer{\vdash\lnot x\in\emptyset}{}
    & \infer{x\in\emptyset\vdash x\in\emptyset}{}}}}
\]
Viele Regeln zur beschränkten Quantifizierung sind analog zu den
Regeln der unbeschränkten. Beispielsweise gilt
\[(\forall x\in M\colon A(x)\land B(x)) \iff (\forall x\in M\colon A(x))
\land (\forall x\in M\colon B(x)).\]
Man muss allerdings Vorsicht walten lassen. Nicht bei jeder Äquivalenz
liegt eine direkte Analogie vor. Zwar besteht für eine Formel $A$,
in der $x$ nicht frei vorkommt, die Äquivalenz
\[(\exists x\colon A) \iff A.\]
Die Analogie ist jedoch von der ein klein wenig intrikateren Form
\[(\exists x\in M\colon A) \iff M\ne\emptyset\land A.\]
Diese Beziehung erklärt sich durch die Umformung
\[(\exists x\in M\colon A)\bicond (\exists x\colon x\in M\land A)
\bicond (\exists x\colon x\in M)\land A \bicond M\ne\emptyset\land A.\]
Die letzte Umformung gilt, weil $M\ne\emptyset$ gleichbedeutend mit
$\exists x\colon x\in M$ ist.

\subsection{Teilmengen}

\begin{Definition}[Teilmengenbeziehung]\newlinefirst
Man definiert $A\subseteq B$, sprich »$A$ ist eine Teilmenge von $B$«, als
\[A\subseteq B\defiff (\forall x\colon x\in A\cond x\in B).\]
\end{Definition}

\begin{Definition}[Potenzmenge]\newlinefirst
Die Potenzmenge einer Menge $M$ ist definiert als
\[\mathcal P(M) := \{A\mid A\subseteq M\}.\]
\end{Definition}
Zum Beispiel ist $\mathcal P(\emptyset)=\{\emptyset\}$ und
$\mathcal P(\{0\}) = \{\emptyset, \{0\}\}$. Des Weiteren
\begin{gather*}
\mathcal P(\{0,1\}) = \{\emptyset, \{0\}, \{1\}, \{0,1\}\},\\
\mathcal P(\{0,1,2\}) = \{\emptyset, \{0\}, \{1\}, \{2\}, \{0,1\}, \{0,2\}, \{1,2\}, \{0,1,2\}\}.
\end{gather*}
% Eine Potenzmenge einer Grundmenge bildet mit der Teilmengenbeziehung
% eine Poset, eine halbgeordnete Menge.

\subsection{Mengenoperationen}

\begin{Definition}[Schnitt, Vereinigung, Differenz]\newlinefirst
Zu zwei Mengen $A,B$ definiert man
\begin{align*}
A\cap B &:= \{x\mid x\in A\land x\in B\}, && \text{(Schnittmenge)}\\
A\cup B &:= \{x\mid x\in A\lor x\in B\}, && \text{(Vereinigungsmenge)}\\
A\setminus B &:= \{x\mid x\in A\land x\notin B\}. && \text{(Differenzmenge)}
\end{align*}
\end{Definition}
Sind $A,B$ Teilmengen einer Grundmenge $G$, so sind auch
$A\cap B$, $A\cup B$ und $A\setminus B$ Teilmengen der Grundmenge.


