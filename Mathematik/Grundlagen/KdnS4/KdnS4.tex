\documentclass[8pt]{beamer}
\usetheme{Antibes}
\useinnertheme{rectangles}
\useoutertheme{infolines}
\usepackage[utf8]{inputenc}
\usepackage[T1]{fontenc}
\usepackage[ngerman]{babel}

% Patch the look of +, = in arev
\usefonttheme{serif}

\usepackage{arev}
% Patch punctuation to be upright
\DeclareMathSymbol{.}{\mathpunct}{operators}{`.}
\DeclareMathSymbol{,}{\mathpunct}{operators}{`,}

\usepackage{amsmath}
\usepackage{amssymb}
\usepackage{proof}
\usepackage{booktabs}

\setbeamertemplate{footline}{%
\begin{beamercolorbox}[ht=3.0ex,dp=1ex]{title in head/foot}
\hfill\footnotesize\insertpagenumber\enspace\enspace\end{beamercolorbox}}

\definecolor{bluegreen1}{rgb}{0.0,0.20,0.28}
\definecolor{bluegreen2}{rgb}{0.0,0.20,0.28}
\setbeamercolor*{palette primary}{fg=white,bg=bluegreen1}
\setbeamercolor*{palette secondary}{fg=white,bg=bluegreen2}
\setbeamercolor*{palette tertiary}{fg=white,bg=bluegreen2}
\setbeamercolor{itemize item}{fg=black}
\setbeamercolor{block title}{bg=bluegreen2}
\newcommand{\modest}[1]{{\small\color{gray}#1}}
\hypersetup{colorlinks,urlcolor=magenta}

\newcommand{\unit}[1]{\mathrm{#1}}
\newcommand{\strong}[1]{\textsf{\textbf{#1}}}
\newcommand{\defiff}{\quad:\Longleftrightarrow\quad}
\newcommand{\infernote}[1]{\!\text{\footnotesize #1}}
\renewcommand{\qedsymbol}{\ensuremath{\Box}}
\newcommand{\discharge}[1]{$\sim$#1}
\newcommand{\centerheadline}[1]{%
  \begin{center}\strong{#1}\end{center}}
\newcommand{\parspace}{\vspace{0.8em}}
\newcommand{\cond}{\rightarrow}
\newcommand{\Z}{\mathbb Z}
\newcommand{\R}{\mathbb R}
\newcommand{\kw}[1]{\textbf{#1}}
\newcommand{\unres}{{?}}

\title{Natürliches Schließen}
\subtitle{Teil 4: Programme als Beweise}
\date{}

\begin{document}

\begin{frame}
\maketitle
\end{frame}

\begin{frame}
\centerheadline{Die Curry-Howard-Korrespondenz}
\end{frame}

\begin{frame}
Die Curry-Howard-Korrespondenz ist eine Beziehung zwischen Aussagen und
Typen. Jedem Junktor entspricht hierbei eine ganz bestimmte Typkonstruktion.\pause
\begin{center}
\begin{tabular}{rl}
\toprule
\strong{Aussage} & \strong{Typ}\\
\midrule
Konjunktion --- $A\land B$ & $A\times B$ --- Produkt\\
Disjunktion --- $A\lor B$ & $A+B$ --- Summe\\
Implikation --- $A\cond B$ & $A\to B$ --- Funktionentyp\\
Kontradiktion --- $\bot$ & $0$ --- leerer Typ\\
Tautologie --- $\top$ & $1$ --- Einheitstyp\\
\bottomrule
\end{tabular}
\end{center}\pause
Darüber hinaus stellt sie eine Beziehung zwischen Beweisen und Termen
dar. Zu jedem Beweis gehört ein bestimmter Term. Die Schlussregeln des
natürlichen Schließens spiegeln die Regeln zur Konstruktion von
Termen wider.
\end{frame}

\begin{frame}[t]
\strong{Typen}

\vspace{3em}
Die Schreibweise $a\colon A$ drückt das Urteil aus, dass $a$ ein Term
vom Typ $A$ sein muss.\pause

\parspace
Analog zur Logik ist ein Kontext eine Liste\footnote{Technisches Detail:
Ein Kontext soll wohlgeformt sein. Beispielsweise ist
$[s\colon\mathrm{string}, P\colon (\mathrm{nat}\to\mathrm{Type}), x\colon P(s)]$
irrsinnig. In der einfachen Typentheorie treten solche Komplikationen
allerdings noch nicht auf; erst in der abhängigen.}
\[\Gamma = [(a_1\colon A_1),\ldots,(a_n\colon A_n)],\]
wobei die $a_k$ Variablen sind. Ist nun die Sequenz
\[\Gamma\vdash (b\colon B)\]
eine ableitbare, liegt das Urteil $b\colon B$ vor, sofern die Variablen
in $\Gamma$ vorausgesetzt werden dürfen. Das ist so zu verstehen,
dass man mit den Variablen aus $\Gamma$ den Term $b$ zusammenbasteln kann.
Eine Variable darf hierbei mehrmals benutzt werden.

\parspace
Ein Objekt $b\colon B$ stellt einen Zeugen dafür dar, dass der Typ
$B$ von mindestens einem Objekt bewohnt ist. Anders ausgedrückt ist
$b$ ein Beweis für die Aussage $B$.

\parspace
\begin{small}
Die runden Klammern schreibt man nicht mit. Ich wollte nur einmal
verdeutlichen, wie die Formeln zu lesen sind.
\end{small}
\end{frame}

\begin{frame}
Zur Konstruktion von Termen finden sich nun Schlussregeln, die die
logischen widerspiegeln.\pause

\begin{block}{Axiom (Einführung von Grundsequenzen)}
\begin{center}
\begin{tabular}{c@{\qquad\qquad}c}
\strong{Aussagen} & \strong{Typurteile}\\[10pt]
$\dfrac{}{A\vdash A}$ & $\dfrac{}{a\colon A\vdash a\colon A}$
\end{tabular}
\end{center}
\end{block}
\end{frame}

\begin{frame}
\begin{block}{Schlussregeln}
\begin{center}
\begin{tabular}{c@{\qquad\qquad}c}
\strong{Konjunktion}
& \strong{Paar}\\[10pt]
$\dfrac{\Gamma\vdash A\qquad\Gamma'\vdash B}{\Gamma,\Gamma'\vdash A\land B}$
&
$\dfrac{\Gamma\vdash a\colon A\qquad\Gamma'\vdash b\colon B}{\Gamma,\Gamma'\vdash (a,b)\colon A\times B}$\\[18pt]
$\dfrac{\Gamma\vdash A\land B}{\Gamma\vdash A}$
& $\dfrac{\Gamma\vdash t\colon A\times B}{\Gamma\vdash \pi_l(t)\colon A}$\\[18pt]
$\dfrac{\Gamma\vdash A\land B}{\Gamma\vdash B}$
& $\dfrac{\Gamma\vdash t\colon A\times B}{\Gamma\vdash \pi_r(t)\colon B}$
\end{tabular}
\end{center}
\end{block}\pause

\parspace
In Worten: Die Einführung der Konjunktion entspricht der Konstruktion
des Paares. Die Beseitigung der Konjunktion entspricht der Projektion
auf eine der Komponenten.
\end{frame}

\begin{frame}
\begin{block}{Schlussregeln}
\begin{center}
\begin{tabular}{c@{\qquad\qquad}c}
\strong{Implikation}
& \strong{Funktion}\\[10pt]
$\dfrac{\Gamma, A\vdash B}{\Gamma\vdash A\cond B}$
&
$\dfrac{\Gamma, a\colon A\vdash b\colon B}{\Gamma\vdash (a\mapsto b)\colon A\to B}$\\[18pt]
$\dfrac{\Gamma\vdash A\cond B\qquad\Gamma'\vdash A}{\Gamma,\Gamma'\vdash B}$
& $\dfrac{\Gamma\vdash f\colon A\to B\qquad\Gamma'\vdash a\colon A}{\Gamma,\Gamma'\vdash f(a)\colon B}$
\end{tabular}
\end{center}
\end{block}\pause

\parspace
In Worten: Die Einführung der Implikation entspricht der Einführung
einer anonymen Funktion (Abstraktion). Die Beseitigung der Implikation
entspricht der Applikation einer Funktion.\pause

\parspace
\begin{small}
Alonzo Church schrieb $\lambda x.t$ anstelle von $x\mapsto t$.
Für $f(x)$ schrieb er $(f\,x)$ oder kurz $f\,x$.%
\footnote{Alonzo Church: \emph{The Calculi of Lambda-Conversion}.
In: \emph{Annals of Mathematics Studies}. Nr.~6.\\
Princeton University Press, Princeton 1941.}
Im Laufe der Zeit verbreitete sich diese Notation in der Informatik.
Die Regeln, die wir hier aufstellen, bilden den einfach
typisierten $\lambda$-Kalkül mit Erweiterung um Produkte von zwei
Faktoren und Summen von zwei Summanden.
\end{small}
\end{frame}

\begin{frame}
Eine kurze Ableitung:

\vspace{1em}
\begin{center}
\begin{tabular}{c@{\qquad\qquad}c}
\strong{Aussagen} & \strong{Programmterme}\\[8pt]
$\infer{\vdash A\land B\cond A}{
  \infer{A\land B\vdash A}{
    \infer{A\land B\vdash A\land B}{}}}$
&
$\infer{\vdash (t\mapsto \pi_l(t))\colon A\times B\to A}{
  \infer{t\colon A\times B\vdash \pi_l(t)\colon A}{
    \infer{t\colon A\times B\vdash t\colon A\times B}{}}}$
\end{tabular}
\end{center}\pause

\parspace
\begin{footnotesize}
\strong{Bemerkung.}
Anstelle von $t\mapsto\pi_l(t)$ kann man auch $(a,b)\mapsto a$
schreiben. Streng genommen müsste man hierfür allerdings ein erweitertes
formales System definieren. Technisch wird ein unabweisbarer
Musterabgleich durchgeführt. Ist $t$ das Argument, wird $t$
mit $(a,b)$ abgeglichen, weshalb $a=\pi_l(t)$ sein muss.
\end{footnotesize}
\end{frame}

\begin{frame}[t]
\strong{Implementierung der Konstruktion}

\vspace{4em}
Der konstruierte Programmterm liefert den Beweis, dass der Typ
$A\times B\to A$ ein bewohnter ist. Man kann den Term nun in
Gallina formulieren und durch den Beweisassistent Coq prüfen lassen,
ob die Konstruktion fehlerfrei durchgeführt wurde:\pause

\begin{block}{Gallina}
\texttt{\kw{Definition} proof1 (A B: Type): A*B -> A\\
\ \ := \kw{fun} t => fst t.}
\end{block}\pause

Es ginge auch so:
\begin{block}{Gallina}
\texttt{\kw{Definition} proof2 (A B: Type): A*B -> A\\
\ \ := \kw{fun} t => \kw{match} t \kw{with} (a, b) => a \kw{end}.}
\end{block}
Oder kurz:
\begin{block}{Gallina}
\texttt{\kw{Definition} proof3 (A B: Type): A*B -> A\\
\ \ := \kw{fun} '(a, b) => a.}
\end{block}
\end{frame}

\begin{frame}
Aussagen bekommen eigentlich den speziellen Typ \texttt{Prop}.
Infolgedessen verändern sich Syntax und genutzte Funktionen,
wobei die Mechanismen aber äquivalent bleiben.
\begin{block}{Gallina}
\texttt{\kw{Definition} proof4 (A B: Prop): A /{\textbackslash} B -> A\\
\ \ := \kw{fun} h => proj1 h.}
\end{block}\pause
Auch hier ist eine Zerlegung per Musterabgleich durchführbar:
\begin{block}{Gallina}
\texttt{\kw{Definition} proof5 (A B: Prop): A /{\textbackslash} B -> A\\
\ \ := \kw{fun} h => \kw{match} h \kw{with} conj a b => a \kw{end}.}
\end{block}\pause
Äquivalent zum Musterabgleich ist das Induktionsprinzip der Konjunktion:
\begin{block}{Gallina}
\texttt{\kw{Definition} proof6 (A B: Prop): A /{\textbackslash} B -> A\\
\ \ := \kw{fun} h => and\_ind (\kw{fun} a b => a) h.}
\end{block}
Der Beweis einer Konjunktion wird hier in seine beiden Teile zerlegt,
um einen oder beide nutzen zu können. Die Zerlegung wird durch eine
Rückruffunktion dargestellt, die \texttt{and\_ind} als erstes Argument
bekommt.
\end{frame}

\begin{frame}
Zum Verständnis der Summe $A+B$ zweier Typen $A,B$ betrachten wir
zunächst die Analogie für Mengen. Seien $\mathrm L,\mathrm R$
zwei Konstanten, als \emph{Tags} oder \emph{Diskriminatoren} bezeichnet.
Man definiert die disjunkte Vereinigung als
\[A+B := (\{\mathrm L\}\times A)\cup (\{\mathrm R\}\times B).\]\pause
Nun definiert man die beiden Injektionen
\begin{align*}
&\iota_l\colon A\to A+B,\quad\iota_l(a) := (\mathrm L, a),\\
&\iota_r\colon B\to A+B,\quad\iota_r(b) := (\mathrm R, b).
\end{align*}
Weil es sich um Injektionen handelt, kann zu einem Wert aus $A+B$
eine Fallunterscheidung per Musterabgleich vorgenommen werden.
Es wird dabei schlicht geprüft, welches der beiden Tags vorliegt.\pause{}
Sinnloses Beispiel:
\[f\colon\Z+\Z\to\Z,\quad f(n) := \strong{match}\;\,n\,\begin{cases}
\iota_l(a)\mapsto a,\\
\iota_r(b)\mapsto b^2.
\end{cases}\]
\end{frame}

\begin{frame}
\begin{block}{Schlussregeln}
\begin{center}
\begin{tabular}{c@{\qquad\qquad}c}
\strong{Disjunktion}
& \strong{Term der Summe}\\[10pt]
$\dfrac{\Gamma\vdash A}{\Gamma\vdash A\lor B}$
& $\dfrac{\Gamma\vdash a\colon A}{\Gamma\vdash\iota_l(a)\colon A+B}$\\[18pt]
$\dfrac{\Gamma\vdash B}{\Gamma\vdash A\lor B}$
& $\dfrac{\Gamma\vdash b\colon B}{\Gamma\vdash\iota_r(b)\colon A+B}$\\[18pt]
$\dfrac{\Gamma\vdash A\lor B\quad\Gamma,A\vdash C\quad\Gamma,B\vdash C}{\Gamma\vdash C}$
& $\dfrac{\Gamma\vdash t\colon A+B\quad\Gamma,a\colon A\vdash c\colon C
  \quad\Gamma,b\colon B\vdash c'\colon C}{\Gamma\vdash\strong{match}\;\,t\;\{\iota_l(a)\mapsto c, \iota_r(b)\mapsto c'\}\colon C}$
\end{tabular}
\end{center}
\end{block}
\end{frame}

\begin{frame}
Einen Term vom Typ $A+B\to B+A$ erhält man beispielsweise wie folgt:\pause
\[t\mapsto\strong{match}\;\,t\,\begin{cases}
\iota_l(a)\mapsto\iota_r(a),\\
\iota_r(b)\mapsto\iota_l(b).
\end{cases}\]\pause
\begin{block}{Gallina}
\texttt{\kw{Definition} proof7 (A B: Type): A + B -> B + A\\
\ \ := \kw{fun} t =>\\
\ \ \ \ \kw{match} t \kw{with}\\
\ \ \ \ | inl(a) => inr(a)\\
\ \ \ \ | inr(b) => inl(b)\\
\ \ \ \ \kw{end}.}\\
\mbox{}\\
\texttt{\kw{Definition} proof8 (A B: Prop): A {\textbackslash}/ B -> B {\textbackslash}/ A\\
\ \ := \kw{fun} t =>\\
\ \ \ \ \kw{match} t \kw{with}\\
\ \ \ \ | or\_introl(a) => or\_intror(a)\\
\ \ \ \ | or\_intror(b) => or\_introl(b)\\
\ \ \ \ \kw{end}.}
\end{block}
\end{frame}

\begin{frame}
\centerheadline{Abhängige Typen}
\end{frame}

\begin{frame}
Zur Ausweitung der Korrespondenz auf die Prädikatenlogik
muss die Typentheorie um \emph{abhängige Typen} erweitert werden.
Ein Typ darf hier von einem Term abhängen, analog dazu wie eine
Menge als indizierte Familie von einem Index abhängt. Die logische
Entsprechung eines solchen Typs ist ein Prädikat. Ist $P$ ein
Prädikat, dann kann je $x$ die Aussage $P(x)$ eine wahre oder
eine falsche sein. Ein abhängiger Typ kann gleichermaßen bewohnt oder
unbewohnt sein.

\parspace
Im Fortgang vermitteln zwei neue Konstruktionen
die Kodierung der Quantoren:
\begin{itemize}
\item \emph{Pi-Typen} entsprechen Universalquantifizierungen,
\item \emph{Sigma-Typen} entsprechen Existenzquantifizierungen.
\end{itemize}
\end{frame}

\begin{frame}[t]
\strong{Pi-Typen}

\vspace{3em}
Erklärung anhand der Analogie in der Mengenlehre. Ein Produkt $A\times B$
enthält Paare $(a,b)$ mit $a\in A$ und $b\in B$. Nun fasst man das
Paar als endliche Folge auf, also als Abbildung
\[f\colon\{0,1\}\to A\cup B, \quad f(0):=a,\quad f(1):=b.\]\pause
Allgemein definiert man insofern
\[\prod_{x\in X} A_x := \{f\colon X\to\bigcup_{x\in X}A_x\mid
\forall x\in X\colon f(x)\in A_x\}.\]\pause
Bei Typen verhält es sich analog.

\parspace
Eine Term $f$ vom Typ $\prod_{x\colon X} A_x$ heißt \emph{abhängige Funktion}.
Ihr Wert $f(x)$ ist vom Typ $A_x$. Der Typ des Wertes hängt also vom Argument
der Funktion ab.\pause

\parspace
So wie das zweistellige Produkt die Konjunktion kodiert, kodiert
das allgemeine Produkt die Allquantifizierung. Wurde die Funktion $f$
konstruiert, erbringt dies für jedes $x$ einen Zeugen bzw. Beweis $f(x)$,
dass der Typ $A_x$ bewohnt ist.
\end{frame}

\begin{frame}[t]
\strong{Sigma-Typen}

\vspace{3em}
Betrachten wir zunächst die disjunkte Vereinigung von Mengen.
Für $t\in A+B$ ist entweder $t=(\mathrm L,a)$ mit $a\in A$ oder
$t=(\mathrm R,b)$ mit $b\in B$. Möchte man den Begriff nun auf
beliebig viele Summanden ausdehnen, findet sich
\[\sum_{x\in X} A_x := \bigcup_{x\in X} (\{x\}\times A_x).\]\pause
Analog verhält es sich bei der Summe von Typen. Ein Term $t$ vom
Typ $\sum_{x\colon X} A_x$ heißt \emph{abhängiges Paar}.
Für $t=(x,a)$ ist $a$ vom Typ $A_x$. Der Typ der rechten Komponente
des Paars hängt also von der linken Komponente ab.\pause

\parspace
So wie die zweistellige Summe die Disjunktion kodiert, kodiert
die allgemeine Summe die Existenzquantifizierung. Ein konstruiertes
Paar $(x,a)$ bezeugt, dass die Summe bewohnt ist. Man bezeichnet auch
$x$ selbst als den Zeugen, wobei zugehörige Beweis $a$ belegt, dass der
Summand $A_x$ bewohnt ist. Der Zeuge besteht also eigentlich nicht nur
in $x$, sondern in der gesamten Information $(x,a)$.
\end{frame}

\begin{frame}
Wir schreiben ab jetzt $\forall x.P(x)$ und $\exists x.P(x)$ anstelle
von $\forall x\colon P(x)$ und $\exists x\colon P(x)$, um Verwirrung
zu vermeiden.

\parspace
Der Aussage
\[A\land(\forall x. P(x))\cond\forall x.A\land P(x)\]
entspricht
\[A\times(\prod_x P(x))\to\prod_x (A\times P(x)).\]
Es ist also ein Paar $(a,f)$ mit $a\colon A$ und $f\colon\prod_x P(x)$
gegeben. Zu konstruieren ist eine abhängige Funktion, die jedem
$x$ ein Paar $(a,y)$ mit $a\colon A$ und $y\colon P(x)$ zuordnet.\pause

\parspace
Schlicht $y:=f(x)$ setzen. Die gesuchte Funktion ist also $x\mapsto (a,f(x))$.
\begin{block}{Gallina}
\texttt{\kw{Definition} proof9 (A: Type) (X: Type) (P: X -> Type):\\
\ \ A*(forall x, P x) -> forall x, A*(P x)\\
\ \ := \kw{fun} t => \kw{match} t \kw{with} (a, f) => \kw{fun} x => (a, f x) \kw{end}.}
\end{block}
\end{frame}

\begin{frame}
\centerheadline{Taktiken}
\end{frame}

\begin{frame}
Als Alternative zur direkten Formulierung von Termen bietet Coq
außerdem noch Taktiken an, darunter versteht man Prozeduren zur
automatischen Erstellung von Teilbeweisen.\pause

\parspace
Die Erstellung des Beweisbaums verläuft hierbei rückwärts.
Man arbeitet sich also von der Wurzel aus bis zu den Blättern durch.
Ein noch unbestätigter Knoten stellt in diesem Zusammenhang ein zu
erreichendes \emph{Ziel} dar, dessen Kindknoten die \emph{Unterziele}.

\parspace
Es stehen Basistaktiken zur Verfügung, die im Prinzip die Schlussregeln
des natürlichen Schließens widerspiegeln. Dies darf allerdings nicht
immer allzu genau verstanden werden. Man kann ja zu einigen Regeln
alternative Formulierungen in Form von zulässigen Schlussregeln herleiten.
Im gleichen Sinne steht ein Sammelsurium von Taktiken zur Verfügung.
\end{frame}

\begin{frame}
\begin{block}{Taktik zum Axiom}
\begin{center}
\begin{tabular}{c@{\qquad\quad}c}
\strong{Regel} & \strong{Taktik}\\[6pt]
$\dfrac{}{\Gamma, h\colon A, \Gamma'\vdash h\colon A}$
& \texttt{exact h}
\end{tabular}
\end{center}
\end{block}
\begin{footnotesize}
\strong{Bemerkung.} Es gestattet \texttt{exact} nicht nur Variablen
für die Blätter des Baums. Die Erstellung
beliebiger korrekt konstruierter Terme ist ebenfalls möglich.
\end{footnotesize}
\end{frame}

\begin{frame}
\begin{block}{Taktiken zu den Einführungsregeln}
\begin{center}
\begin{tabular}{c@{\qquad\quad}c}
\strong{Regel} & \strong{Taktik}\\[6pt]
$\dfrac{\Gamma, h\colon A\vdash\unres\colon B}{\Gamma\vdash\unres\colon A\to B}$
& \texttt{intro h}\\[14pt]
$\dfrac{\Gamma\vdash\unres\colon A\qquad\Gamma\vdash\unres\colon B}
  {\Gamma\vdash\unres\colon A\land B}$
& \texttt{split}\\[14pt]
$\dfrac{\Gamma\vdash\unres\colon A}{\Gamma\vdash\unres\colon A\lor B}$
& \texttt{left}\\[14pt]
$\dfrac{\Gamma\vdash\unres\colon B}{\Gamma\vdash\unres\colon A\lor B}$
& \texttt{right}
\end{tabular}
\end{center}
\end{block}
\end{frame}

\begin{frame}
\begin{block}{Taktiken zu den Beseitigungsregeln}
\begin{center}
\begin{tabular}{c@{\qquad\quad}c}
\strong{Regel} & \strong{Taktik}\\[6pt]
$\dfrac{\Gamma\vdash h\colon A\to B\qquad\Gamma\vdash\unres\colon A}{\Gamma\vdash\unres\colon B}$
& \texttt{apply h}\\[14pt]
$\dfrac{\Gamma,a\colon A,b\colon B,\Gamma'\vdash\unres\colon C}
  {\Gamma,h\colon A\land B,\Gamma'\vdash\unres\colon C}$
& \texttt{destruct h as (a, b)}
\end{tabular}
\end{center}
\end{block}
\end{frame}

\begin{frame}
Um $A\land B\cond A$ zu zeigen, können wir also wie folgt vorgehen:
\[\infer[\texttt{intro h}]{\vdash (h\mapsto a)\colon A\land B\cond A}{
  \infer[\texttt{destruct h as (a, b)}]{h\colon A\land B\vdash a\colon A}{
    \infer[\texttt{exact a}]{a\colon A, b\colon B\vdash a\colon A}{}
  }
}\]\pause
Die Implementierung:
\begin{block}{Gallina}
\texttt{\kw{Theorem} projl (A B: Prop): A /{\textbackslash} B -> A.\\
\kw{Proof}.\\
\ \ intro h.\\
\ \ destruct h as (a, b).\\
\ \ exact a.\\
\kw{Qed}.}
\end{block}
\end{frame}

\begin{frame}
Um $(A\cond B)\land A\cond B$ zu zeigen, können wir wie folgt vorgehen:
\[\infer[\texttt{intro h}]{\vdash (h\mapsto f(a))\colon (A\to B)\land A\to B}{
  \infer[\texttt{destruct h as (f, a)}]{h\colon (A\to B)\land A\vdash f(a)\colon B}{
    \infer[\texttt{apply f}]{f\colon A\to B, a\colon A\vdash f(a)\colon B}{
      \infer{f\colon A\to B\vdash f\colon A\to B}{}
    & \infer[\texttt{exact a}]{a\colon A\vdash a\colon A}{}
    }
  }
}\]\pause
Die Implementierung:
\begin{block}{Gallina}
\texttt{\kw{Theorem} mp (A B: Prop): (A -> B) /{\textbackslash} A -> B.\\
\kw{Proof}.\\
\ \ intro h.\\
\ \ destruct h as (f, a).\\
\ \ apply f.\\
\ \ exact a.\\
\kw{Qed}.}
\end{block}
\end{frame}

\begin{frame}
Um $A\land B\cond B\land A$ zu zeigen, können wir wie folgt vorgehen:
\[\infer[\texttt{intro h}]{\vdash (h\mapsto \mathrm{conj}(b)(a))\colon A\land B\to B\land A}{
  \infer[\texttt{destruct h as (a, b)}]{h\colon A\land B\vdash\mathrm{conj}(b)(a)\colon B\land A}{
    \infer[\texttt{split}]{a\colon A, b\colon B\vdash\mathrm{conj}(b)(a)\colon B\land A}{
      \infer[\texttt{exact b}]{b\colon B\vdash b\colon B}{}
    & \infer[\texttt{exact a}]{a\colon A\vdash a\colon A}{}
    }
  }
}\]\pause

Die Implementierung:
\begin{block}{Gallina}
\texttt{\kw{Theorem} conjunction\_commutes (A B: Prop): A /{\textbackslash} B -> B /{\textbackslash} A.\\
\kw{Proof}.\\
\ \ intro h.\\
\ \ destruct h as (a, b).\\
\ \ split.\\
\ \ - exact b.\\
\ \ - exact a.\\
\kw{Qed}.}
\end{block}
\end{frame}

\begin{frame}
Um $A\lor B\cond B\lor A$ zu zeigen, können wir wie folgt vorgehen:
\begin{small}
\[
\infer[\texttt{intro h}]{\vdash (h\mapsto s)\colon A\lor B\to B\lor A}{
  \infer[\texttt{destruct h as [a | b]}]{h\colon A\lor B\vdash s\colon B\lor A}{
    \infer{h\colon A\lor B\vdash h\colon A\lor B}{}
  & \infer{a\colon A\vdash r(a)\colon B\lor A}{
      \infer{a\colon A\vdash a\colon A}{}}
  & \infer[\texttt{left}]{b\colon B\vdash l(b)\colon B\lor A}{
      \infer[\texttt{exact b}]{b\colon B\vdash b\colon B}{}}}}
\]
\end{small}
mit $l := \mathrm{or\_introl}$, $r := \mathrm{or\_intror}$ und
\[s := \strong{match}\;h\begin{cases}
l(a)\mapsto r(a),\\
r(b)\mapsto l(b).
\end{cases}
\]\pause
Die Implementierung:
\begin{block}{Gallina}
\texttt{\kw{Theorem} disjunction\_commutes (A B: Prop): A {\textbackslash}/ B -> B {\textbackslash}/ A.\\
\kw{Proof}.\\
\ \ intro h.\\
\ \ destruct h as [a | b].\\
\ \ - right. exact a.\\
\ \ - left.  exact b.\\
\kw{Qed}.}
\end{block}
\end{frame}

\begin{frame}
Nun wird man irgendwann höhere kognitive Sprünge vollführen und
mühseligen Kleinkram auslassen wollen. Hierfür stehen höhere Taktiken
zur Verfügung, die einfache Beweise gänzlich automatisch konstruieren.
Ein Beispiel hierfür ist \texttt{tauto}.\pause
\begin{block}{Gallina}
\texttt{\kw{Theorem} projl (A B: Prop): A /{\textbackslash} B -> A.\\
\kw{Proof}.\\
\ \ tauto.\\
\kw{Qed}.}
\end{block}\pause
Gut, das wäre im ersten Semester nicht gestattet. Nun kann man trotzdem
Schummeln, indem man sich den erzeugten Beweis einfach anschaut:
\begin{block}{Gallina}
\texttt{\kw{Print} projl.\\
\mbox{}\\
(* Ausgabe: *)\\
projl = \\
\kw{fun} (A B: Prop) (H: A /{\textbackslash} B) => and\_ind (\kw{fun} (H0: A) (\_: B) => H0) H\\
\ \ \ \ \ : \kw{forall} A B: Prop, A /{\textbackslash} B -> A}
\end{block}
\end{frame}

\begin{frame}
Für $(A\land\forall x. P(x))\to\forall x. A\land P(x)$ findet sich:\pause
\[
\infer[\texttt{intro h}]{\vdash (h\mapsto x\mapsto y)\colon (A\land\forall x. P(x))\to\forall x. A\land P(x)}{
  \infer[\texttt{destruct h as (a, f)}]{h\colon A\land\forall x. P(x)\vdash (x\mapsto y)\colon\forall x.A\land P(x)}{
    \infer[\texttt{intro x}]{a\colon A, f\colon \forall x.P(x)\vdash (x\mapsto y)\colon\forall x.A\land P(x)}{
      \infer[\texttt{split}]{a\colon A, f\colon \forall x.P(x), x\colon X\vdash y\colon A\land P(x)}{
        \infer[\texttt{exact a}]{a\colon A\vdash a\colon A}{}
      & \infer[\texttt{apply f}]{f\colon\forall x.P(x), x\colon X\vdash f(x)}{}}}}}
\]
Hierbei gilt $y:=\mathrm{conj}(a)(f(x))$. Die Implementierung:\pause

\begin{block}{Gallina}
\texttt{\kw{Theorem} const\_factor (A: Prop) (X: Type) (P: X -> Prop):\\
\ \ A /{\textbackslash} (\kw{forall} x, P x) -> \kw{forall} x, A /{\textbackslash} P x.\\
\kw{Proof}.\\
\ \ intro h.\\
\ \ destruct h as (a, f).\\
\ \ intro x.\\
\ \ split.\\
\ \ - exact a.\\
\ \ - apply f.\\
\kw{Qed}.}
\end{block}
\end{frame}

\begin{frame}
Für $A\land (\exists x. P(x))\to\exists x. A\land P(x)$ findet sich:\pause
\begin{small}
\[
\infer[\texttt{intro h}]{\vdash (h\mapsto s')\colon A\land (\exists x. P(x))\to\exists x. A\land P(x)}{
  \infer[\texttt{destruct h as (a, s)}]{h\colon A\land\exists x. P(x)\vdash s'\colon\exists x. A\land P(x)}{
    \infer[\texttt{destruct s as (x, p)}]{a\colon A,s\colon\exists x.P(x)\vdash s'\colon\exists x.A\land P(x)}{
      \infer{s\colon\exists x.P(x)\vdash s\colon\exists x.P(x)}{}
    & \infer[\texttt{exists x}]{a\colon A,x\colon X,p\colon P(x)\vdash s'\colon\exists x.A\land P(x)}{
        \infer[\texttt{split}]{a\colon A,x\colon X,p\colon P(x)\vdash \mathrm{conj}(a)(p)\colon A\land P(x)}{
          \infer[\texttt{exact a}]{a\colon A\vdash a\colon A}{}
        & \infer[\texttt{exact p}]{x\colon X,p\colon P(x)\vdash p\colon P(x)}{}}}}}}
\]
\end{small}\pause
Die Implementierung:
\begin{block}{Gallina}
\texttt{\kw{Theorem} distribute (A: Prop) (X: Type) (P: X -> Prop):\\
\ \ A /{\textbackslash} (\kw{exists} x, P x) -> \kw{exists} x, A /{\textbackslash} P x.\\
\kw{Proof}.\\
\ \ intro h.\\
\ \ destruct h as (a, s).\\
\ \ destruct s as (x, p).\\
\ \ exists x.\\
\ \ split.\\
\ \ - exact a.\\
\ \ - exact p.\\
\kw{Qed}.}
\end{block}
\end{frame}

\begin{frame}
\centerheadline{Zum Abschluss nochmals Zahlentheorie}
\end{frame}

\begin{frame}
Zum Abschluss soll der Beweis des Satzes
\emph{Das Quadrat einer geraden Zahl ist ebenfalls gerade}
implementiert werden. Gleichungen werden durch Typen ausgedrückt.
Wir schreiben $p\colon (n = 2k)$ für das Urteil, dass $p$ vom Typ
$n = 2k$ ist. Es stellt also $p$ einen Beweis für die Gleichheit
$n = 2k$ dar.\pause

\parspace
Nun darf die Ersetzungsregel zur Anwendung kommen. Man schreibt dies als
\[\dfrac{\Gamma\vdash p\colon n = 2k}{\Gamma\vdash y\colon f(n) = f(2k)}
\quad\text{mit}\;y:=\mathrm{f\_equal}(f)(p).\]\pause
Hiermit findet sich:
\begin{small}
\[
\infer[\texttt{intro h}]{(h\mapsto y'')\colon(\exists k. n = 2k)\to (\exists l. n^2 = 2l)}{
  \infer[\texttt{destruct h as (k, p)}]{h\colon\exists k. n = 2k\vdash y''\colon\exists l. n^2 = 2l}{
    \infer{h\colon (\exists k. n = 2k)\vdash h\colon (\exists k. n = 2k)}{}
  & \infer[\texttt{exists ($kn$)}]{\Gamma\vdash y''\colon\exists l. n^2 = 2l}{
      \infer[\texttt{rewrite Nat.mul\_assoc}]{\Gamma\vdash y'\colon n^2 = 2(kn)}{
        \infer[\texttt{apply (f\_equal ($x\mapsto xn$) $p$)}]{\Gamma\vdash y\colon n^2 = (2k)n}{
          \infer{\Gamma\vdash p\colon n = 2k}{}}}}}}
\]
\end{small}
mit $\Gamma:=[k\colon\mathrm{nat}, p\colon n = 2k]$.
\end{frame}

\begin{frame}
Die Implementierung:
\begin{block}{Gallina}
\texttt{\kw{Require} \kw{Import} Coq.Arith.PeanoNat.\\
\mbox{}\\
\kw{Theorem} even\_square (n: nat):\\
\ \ (\kw{exists} k, n = 2*k) -> (\kw{exists} l, n*n = 2*l).\\
\kw{Proof}.\\
\ \ intro h.\\
\ \ destruct h as (k, p).\\
\ \ exists (k*n).\\
\ \ rewrite Nat.mul\_assoc.\\
\ \ apply (f\_equal (\kw{fun} x => x*n) p).\\
\kw{Qed}.}
\end{block}
\end{frame}

\begin{frame}
Möchte man den Beweisterm von Hand schreiben, erhält man zunächst
separat $n^2 = (2k)n$ per Ersetzungsregel und $2(kn) = (2k)n$
per Assoziativgesetz. Um aus den beiden Gleichungen nun
$n^2 = 2(kn)$ zu gewinnen, muss noch explizit das
Transitivgesetz angewendet werden, wofür die Funktion
\texttt{eq\_trans\_r} zur Verfügung steht.\pause

\begin{block}{Gallina}
\texttt{\kw{Require} \kw{Import} Coq.Arith.PeanoNat.\\
\mbox{}\\
\kw{Theorem} even\_square (n: nat):\\
\ \ (\kw{exists} k, n = 2*k) -> (\kw{exists} l, n*n = 2*l).\\
\kw{Proof}.\\
\ \ exact (\kw{fun} t =>\\
\ \ \ \ \kw{match} t \kw{with} ex\_intro \_ k p => ex\_intro \_ (k*n)\\
\ \ \ \ \ \ (\kw{let} y := f\_equal (\kw{fun} x => x*n) p \kw{in}\\
\ \ \ \ \ \ \ \ eq\_trans\_r y (Nat.mul\_assoc 2 k n))\\
\ \ \ \ \kw{end}).\\
\kw{Qed}.}
\end{block}
\end{frame}

\begin{frame}
Alternativ kann man sich aber auch vom Fluss der verbleibenden
Ziele leiten lassen und Taktiken zur Termersetzung nutzen:
\begin{block}{Gallina}
\texttt{\kw{Require} \kw{Import} Coq.Arith.PeanoNat.\\
\mbox{}\\
\kw{Theorem} even\_square (n: nat):\\
\ \ (\kw{exists} k, n = 2*k) -> (\kw{exists} l, n*n = 2*l).\\
\kw{Proof}.\\
\ \ intro h.\\
\ \ destruct h as (k, p).\\
\ \ exists (k*n).\\
\ \ rewrite Nat.mul\_assoc.\\
\ \ replace (2*k) with n.\\
\ \ reflexivity.\\
\kw{Qed}.}
\end{block}
\end{frame}

\begin{frame}
\strong{Literatur}
\begin{itemize}
\item Philip Wadler: \emph{Propositions as Types}.
In: \emph{Communications of the ACM}. Band~58, Nr.~12, 2015. S.~75--84.
\href{https://doi.org/10.1145/2699407}{doi:10.1145/2699407}.
---Der klassische Einführungsartikel in die Thematik.
\item Samuel Mimram: \emph{Program = Proof}.
\href{https://www.lix.polytechnique.fr/Labo/Samuel.Mimram/publications/}{Link (Open Access)}.
\item Christine Paulin-Mohring: \emph{Introduction to the Coq
proof-assistant for practical software verification}. Laboratoire de
recherche en informatique, Université Paris-Saclay, 2011.
\href{https://www.lri.fr/~paulin/}{Link (Open Access)}.
\item Benjamin C. Pierce u.\,a.:
\emph{Software Foundations. Volume 1: Logical Foundations.}
University of Pennsylvania, 2022.
\href{https://softwarefoundations.cis.upenn.edu/}{Link (Open Access)}.
\end{itemize}
\end{frame}

\begin{frame}
Ende.
\vfill\hfill\modest{November 2022}\\
\hfill\modest{Creative Commons CC0 1.0}
\end{frame}


\end{document}


