\documentclass[8pt]{beamer}
\usetheme{Antibes}
\useinnertheme{rectangles}
\useoutertheme{infolines}
\usepackage[utf8]{inputenc}
\usepackage[T1]{fontenc}
\usepackage[ngerman]{babel}

% Patch the look of +, = in arev
\usefonttheme{serif}

\usepackage{arev}
% Patch punctuation to be upright
\DeclareMathSymbol{.}{\mathpunct}{operators}{`.}
\DeclareMathSymbol{,}{\mathpunct}{operators}{`,}

\usepackage{amsmath}
\usepackage{amssymb}
\usepackage{proof}
\usepackage{booktabs}

\setbeamertemplate{footline}{%
\begin{beamercolorbox}[ht=3.0ex,dp=1ex]{title in head/foot}
\hfill\footnotesize\insertpagenumber\enspace\enspace\end{beamercolorbox}}

\definecolor{bluegreen1}{rgb}{0.0,0.20,0.28}
\definecolor{bluegreen2}{rgb}{0.0,0.20,0.28}
\setbeamercolor*{palette primary}{fg=white,bg=bluegreen1}
\setbeamercolor*{palette secondary}{fg=white,bg=bluegreen2}
\setbeamercolor*{palette tertiary}{fg=white,bg=bluegreen2}
\setbeamercolor{itemize item}{fg=black}
\setbeamercolor{block title}{bg=bluegreen2}
\newcommand{\modest}[1]{{\small\color{gray}#1}}
\hypersetup{colorlinks,urlcolor=magenta}

\newcommand{\unit}[1]{\mathrm{#1}}
\newcommand{\strong}[1]{\textsf{\textbf{#1}}}
\newcommand{\defiff}{\quad:\Longleftrightarrow\quad}
\newcommand{\infernote}[1]{\!\text{\footnotesize #1}}
\renewcommand{\qedsymbol}{\ensuremath{\Box}}
\newcommand{\discharge}[1]{$\sim$#1}
\newcommand{\centerheadline}[1]{%
  \begin{center}\strong{#1}\end{center}}
\newcommand{\parspace}{\vspace{0.8em}}
\newcommand{\cond}{\rightarrow}
\newcommand{\Z}{\mathbb Z}
\newcommand{\R}{\mathbb R}
\newcommand{\kw}[1]{\textbf{#1}}

\title{Der Kalkül des natürlichen Schließens}
\subtitle{Teil 4: Programme als Beweise}
\date{}

\begin{document}

\begin{frame}
\maketitle
\end{frame}

\begin{frame}
\centerheadline{Die Curry-Howard-Korrespondenz}
\end{frame}

\begin{frame}
Die Schreibweise $a\colon A$ drückt das Urteil aus, dass $a$ ein Term
vom Typ $A$ sein muss.\pause{} Analog zur Logik ist ein Kontext eine Liste
\[\Gamma = [a_1\colon A_1,\ldots,a_n\colon A_n].\]
Ist nun die Sequenz
\[\Gamma\vdash (b\colon B)\]
eine ableitbare, liegt das Urteil $b\colon B$ vor, sofern die Terme
in $\Gamma$ vorausgesetzt werden dürfen. Das ist so zu verstehen,
dass man mit dem Termen aus $\Gamma$ den Term $b$ zusammenbasteln kann.
Ein Term darf hierbei mehrmals benutzt werden.\pause
\end{frame}

\begin{frame}
Zur Konstruktion von Termen finden sich nun Schlussregeln, die die
logischen widerspiegeln.\pause

\begin{block}{Axiom (Einführung von Grundsequenzen)}
\begin{center}
\begin{tabular}{c@{\qquad\qquad}c}
\strong{Aussagen} & \strong{Typurteile}\\[10pt]
$\dfrac{}{A\vdash A}$ & $\dfrac{}{(a\colon A)\vdash (a\colon A)}$
\end{tabular}
\end{center}
\end{block}
\end{frame}

\begin{frame}
\begin{block}{Schlussregeln}
\begin{center}
\begin{tabular}{c@{\qquad\qquad}c}
\strong{Konjunktion}
& \strong{Paar}\\[10pt]
$\dfrac{\Gamma\vdash A\qquad\Gamma'\vdash B}{\Gamma,\Gamma'\vdash A\land B}$
&
$\dfrac{\Gamma\vdash a\colon A\qquad\Gamma'\vdash b\colon B}{\Gamma,\Gamma'\vdash (a,b)\colon A\times B}$\\[18pt]
$\dfrac{\Gamma\vdash A\land B}{\Gamma\vdash A}$
& $\dfrac{\Gamma\vdash t\colon A\times B}{\Gamma\vdash \pi_l(t)\colon A}$\\[18pt]
$\dfrac{\Gamma\vdash A\land B}{\Gamma\vdash B}$
& $\dfrac{\Gamma\vdash t\colon A\times B}{\Gamma\vdash \pi_r(t)\colon B}$
\end{tabular}
\end{center}
\end{block}\pause

\parspace
In Worten: Die Einführung der Konjunktion entspricht der Konstruktion
des Paares. Die Beseitigung der Konjunktion entspricht der Projektion
auf eine der Komponenten.
\end{frame}

\begin{frame}
\begin{block}{Schlussregeln}
\begin{center}
\begin{tabular}{c@{\qquad\qquad}c}
\strong{Implikation}
& \strong{Funktion}\\[10pt]
$\dfrac{\Gamma, A\vdash B}{\Gamma\vdash A\cond B}$
&
$\dfrac{\Gamma, a\colon A\vdash b\colon B}{\Gamma\vdash (a\mapsto b)\colon A\to B}$\\[18pt]
$\dfrac{\Gamma\vdash A\cond B\qquad\Gamma'\vdash A}{\Gamma,\Gamma'\vdash B}$
& $\dfrac{\Gamma\vdash f\colon A\to B\qquad\Gamma'\vdash a\colon A}{\Gamma,\Gamma'\vdash f(a)\colon B}$
\end{tabular}
\end{center}
\end{block}\pause

\parspace
In Worten: Die Einführung der Implikation entspricht der Einführung
einer anonymen Funktion (Abstraktion). Die Beseitigung der Implikation
entspricht der Applikation der Funktion.\pause

\parspace
Alonzo Church schrieb $\lambda x.t$ anstelle von $x\mapsto t$.
Die Applikation $f(x)$ wird oft zu $fx$ verkürzt.
Churchs Notation hat sich im Laufe der Zeit irgendwie in die Informatik
eingebürgert. Die Regeln, die wir hier aufstellen, bilden den einfach
typisierten $\lambda$-Kalkül mit Erweiterungen um Produkte von zwei
Faktoren und Summen von zwei Summanden.
\end{frame}

\begin{frame}
\begin{center}
\begin{tabular}{c@{\qquad\qquad}c}
\strong{Aussagen} & \strong{Programmterme}\\[8pt]
$\infer{\vdash A\land B\cond A}{
  \infer{A\land B\vdash A}{
    \infer{A\land B\vdash A\land B}{}}}$
&
$\infer{\vdash (t\mapsto \pi_l(t))\colon A\times B\to A}{
  \infer{t\colon A\times B\vdash \pi_l(t)\colon A}{
    \infer{t\colon A\times B\vdash t\colon A\times B}{}}}$
\end{tabular}
\end{center}\pause

\parspace
\strong{Bemerkung.}
Anstelle von $t\mapsto\pi_l(t)$ kann man auch $(a,b)\mapsto a$
schreiben. Streng genommen muss hierbei allerdings ein unabweisbarer
Musterabgleich durchgeführt werden. Ist $t$ das Argument, wird $t$
mit $(a,b)$ abgeglichen, weshalb $a=\pi_l(t)$ sein muss.
\end{frame}

\begin{frame}[t]
\strong{Implementierung der Konstruktion}

\vspace{4em}
Der konstruierte Programmterm liefert den Beweis, dass der Typ
$A\times B\to A$ ein bewohnter ist. Man kann den Term nun in
Gallina formulieren und durch den Beweisassistent Coq prüfen lassen,
ob die Konstruktion fehlerfrei durchgeführt wurde:\pause

\begin{block}{Gallina}
\texttt{\kw{Definition} proof1(A B: Type): A*B -> A\\
\ \ := \kw{fun} t => fst t.}
\end{block}\pause

\parspace
Es ginge auch so:
\begin{block}{Gallina}
\texttt{\kw{Definition} proof2(A B: Type): A*B -> A\\
\ \ := \kw{fun} t => \kw{match} t \kw{with} (a, b) => a \kw{end}.}
\end{block}
\end{frame}

\begin{frame}
Ende.
\vfill\hfill\modest{November 2022}\\
\hfill\modest{Creative Commons CC0 1.0}
\end{frame}


\end{document}


