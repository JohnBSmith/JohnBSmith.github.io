\documentclass[9pt]{beamer}
\usetheme{Antibes}
\useinnertheme{rectangles}
\useoutertheme{infolines}
\usepackage[utf8]{inputenc}
\usepackage[T1]{fontenc}
\usepackage[ngerman]{babel}

% patch the look of +, = in arev
\usefonttheme{serif} 

\usepackage{arev}
% Patch punctuation to be upright
\DeclareMathSymbol{.}{\mathpunct}{operators}{`.}
\DeclareMathSymbol{,}{\mathpunct}{operators}{`,}

\usepackage{amsmath}
\usepackage{amssymb}
\usepackage[all]{xy}

\setbeamertemplate{footline}{%
\begin{beamercolorbox}[ht=3.0ex,dp=1ex]{title in head/foot}
\hfill\footnotesize\insertpagenumber\enspace\enspace\end{beamercolorbox}}

\definecolor{brown1}{rgb}{0.26,0.14,0}
\definecolor{brown2}{rgb}{0.20,0.10,0}
\setbeamercolor*{palette primary}{fg=white,bg=brown1}
\setbeamercolor*{palette secondary}{fg=white,bg=brown2}
\setbeamercolor*{palette tertiary}{fg=white,bg=brown2}
\setbeamercolor{itemize item}{fg=black,bg=white}
\newcommand{\modest}[1]{{\small\color{gray}#1}}

\newcommand{\ee}{\mathrm e}
\newcommand{\ui}{\mathrm i}
\newcommand{\real}{\operatorname{Re}}
\newcommand{\imag}{\operatorname{Im}}
\newcommand{\uv}[1]{\underline{#1}}
\newcommand{\bv}[1]{\mathbf{#1}}

\newcommand{\N}{\mathbb N}
\newcommand{\Z}{\mathbb Z}
\newcommand{\Q}{\mathbb Q}
\newcommand{\R}{\mathbb R}
\newcommand{\C}{\mathbb C}

\newcommand{\id}{\operatorname{id}}
\newcommand{\sgn}{\operatorname{sgn}}
\newcommand{\Abb}{\operatorname{Abb}}
\newcommand{\ob}{\operatorname{ob}}
\newcommand{\unit}[1]{\mathrm{#1}}
\newcommand{\chem}[1]{\mathrm{#1}}
\newcommand{\strong}[1]{\textsf{\textbf{#1}}}
\newcommand{\defiff}{\quad:\Longleftrightarrow\quad}
\renewcommand{\qedsymbol}{\ensuremath{\square}}

\title{Kategorientheorie}
\date{}

\begin{document}

\begin{frame}
\maketitle
\end{frame}

\begin{frame}
Was ist eine Kategorie?
\end{frame}

\begin{frame}
\begin{itemize}
\item Ein ganz normales Axiomensystem das auf der Prädikatenlogik aufbaut.
\item Man verkettet Morphismen, ganz analog wie Abbildungen
  verkettet werden.
\item Die Verkettung von Morphismen hat Axiome analog zu denen
  eines Monoids.
\end{itemize}
\end{frame}

\begin{frame}
Also nichts neues. Alles schon gewohnt.
\end{frame}

\begin{frame}
Ist das nicht zu einfach? Ein Monoid $(M,\circ)$ hat nur die Axiome
\begin{itemize}
\item Assoziativität: $\forall_{a,b,c{\in}M}((a\circ b)\circ c = a\circ (b\circ c))$,
\item neutrales Element existiert: $\exists_{e\in M}\forall_{a\in M}(a\circ e = e\circ a = a)$.
\end{itemize}
\end{frame}

\begin{frame}
Eine Kategorie ist ein Tripel $C=(\operatorname{ob},\hom,\circ)$.
Die Teilstruktur $(\hom,\circ)$ hat die »Monoid"=Axiome«.
Man bezeichnet $g\circ f$ für $f,g\in\hom$ als Verkettung.

\vspace{1em}
Die Elemente der Klasse $\ob$ nennt man Objekte.
Wenn die Klasse bezüglich der Morphismen und Verkettung dieser
eine Kategorie bildet, spricht man von der Kategorie dieser Objekte.
\end{frame}

\begin{frame}
Die Klasse $\hom$ besteht aus allen Morphismen zwischen Objekten.
Speziell für zwei Objekte $A,B\in\ob$ ist $\hom(A,B)$ die Klasse
aller Morphismen $f\colon A\to B$.
\end{frame}

\begin{frame}
Wie gesagt gilt für $f\colon A\to B$, $g\colon B\to C$ und
$h\colon C\to D$ das Assoziativgesetz:
\[h\circ (g\circ f) = (h\circ g)\circ f.\]
Für jedes Objekt $X$ existiert die Identität $\id_X\colon X\to X$, so
dass
\[f\circ\id_A = \id_B\circ f = f.\]
\end{frame}

\begin{frame}
Man kann zeigen dass die Identität für jedes Objekt eindeutig
bestimmt ist. Wir wollen uns damit jetzt nicht aufhalten.
\end{frame}

\begin{frame}
Das einfachste Beispiel für eine Kategorie ist die Kategorie der
Mengen mit den Abbildungen als Morphismen und der gewöhnlichen
Verkettung von Abbildungen.
Sei $\ob=\Omega$ die Klasse aller Mengen, sei $\hom=\Abb$ die Klasse
aller Abbildungen und $\hom(A,B)=\Abb(A,B)$. Dann ist
\[\mathbf{Set} = (\Omega,\Abb,\circ)\]
eine Kategorie.
\end{frame}

\begin{frame}
Sei $\Omega$ die Klasse aller Gruppen. Für $G,H\in\Omega$ sei $\hom(G,H)$
die Klasse der Gruppenhomomorphismen. Die Verkettung sei die
gewöhnliche Verkettung. Dann ist
\[\mathbf{Group} = (\Omega,\hom,\circ)\]
eine Kategorie.
\end{frame}

\begin{frame}
Sei $K$ ein Körper und $\Omega_K$ die Klasse aller Vektorräume
über dem Körper. Für $V,W\in\Omega_K$ sei $\hom(V,W)$ die Klasse
der linearen Abbildungen von $V$ nach $W$, d.\,h. der
Vektorraum"=Homomorphismen. Sei die Verkettung die gewöhnliche
Verkettung. Dann ist
\[\mathbf{Vect}_K = (\Omega_K,\hom,\circ)\]
eine Kategorie.
\end{frame}

\begin{frame}
Es ist so, dass in den Vektorräumen die Struktur von Gruppen
enthalten ist und in den Gruppen die Struktur von Mengen. Um das
zu präzisieren muss der Begriff \emph{Funktor} eingeführt werden. Wir
brauchen einen sogenannten Vergissfunktor, um Struktur zu vergessen
bzw. zu entfernen.
\end{frame}

\begin{frame}
Sind $C,D$ Kategorien, dann nennt man $F\colon C\to D$ kovarianten
Funktor, wenn
\begin{itemize}
\item jedem Objekt $X\in\ob(C)$ ein Objekt $F(X)\in\ob(D)$ zugeordent wird,
\item jedem Morphismus $f\in\hom_C(X,Y)$ ein Morphismus $F(f)\in\hom_D(F(X),F(Y))$
  zugeordnet wird,
\end{itemize}
so dass die folgenden beiden Verträglichkeitsaxiome erfüllt sind:
\begin{itemize}
\item $F(g\circ f) = F(g)\circ F(f),$
\item $F(\id_X) = \id_{F(X)}.$
\end{itemize}
\end{frame}

\begin{frame}
Bemerkung: Die Notation ist überladen. Die Zuordnung
\[F\colon\ob(C)\to\ob(D)\]
ist eigentlich zu unterscheiden von
\[\tilde F\colon\hom_C(X,Y)\to\hom_D(F(X),F(Y)).\]
Das Paar $(F,\tilde F)$ kodiert dann eigentlich den Funktor $C\to D$.

\vspace{1em}
Ab nun schreiben wir aber $F=\tilde F=(F,\tilde F)$, da der Leser
weiß was gemeint ist, wenn er das Wort Funktor liest.
\end{frame}

\begin{frame}
Noch eine Bemerkung: Die Verträglichkeitsaxiome lassen $F$ ausschauen wie einen
Homomorphismus zwischen Monoiden. Es kann gut möglich
sein, dass die Klasse der Morphismen mit den Funktoren als
Morphismen selbst wieder eine Kategorie bildet.
\end{frame}


\begin{frame}
Der Vergissfunktor von der Gruppenkategorie zur Mengenkategorie wird
wie folgt definiert:
\[F\colon\mathbf{Group}\to\mathbf{Set},\quad F((G,*,e)):=G,\]
und jedem Gruppenhomomorphismus
\[\varphi\colon (G,*,e)\to (G',*',e')\]
wird die Abbildung
\[F(\varphi)\colon G\to G',\quad F(\varphi)(x):=\varphi(x)\]
zugeordnet.
\end{frame}

\begin{frame}
Der Vergissfunktor ist tatsächlich ein kovarianter Funktor.

\vspace{1em}
\strong{Beweis.}
Es gilt $F(\id)(x)=\id(x)$, und daher $F(\id)=\id$. Außerdem gilt
\begin{align*}
F(\varphi_2\circ\varphi_1)(x) &= (\varphi_2\circ\varphi_1)(x)
= \varphi_2(\varphi_1(x))\\
&= F(\varphi_2)(F(\varphi_1)(x))\\
&= (F(\varphi_2)\circ F(\varphi_1))(x),
\end{align*}
und daher
\[F(\varphi_2\circ\varphi_1) = F(\varphi_2)\circ F(\varphi_1).\;\qedsymbol\]
\end{frame}

\begin{frame}
Es ergibt sich das folgende Diagramm:
\[\xymatrix{(G,*,e) \ar[r]^\varphi \ar@{|->}[d]^F & \ar@{|->}[d]^F (G',*',e')\\
G \ar[r]^{F(\varphi)} & G'}
\]
Das ist aber nur eine Übersicht und kein kommutatives Diagramm, da
die Pfeile von unterschiedlichem Typ sind.
\end{frame}

\begin{frame}
Ein Funktor ordnet einem kommutierenden Diagramm ein
kommutierendes Diagramm zu. Hier ergibt sich:
\[
\begin{matrix}\xymatrix{(G,*,e)\ar[d]_{\varphi_1} \ar[dr]^{\varphi_2\circ\varphi_1}\\
  (G',*',e')\ar[r]_{\varphi_2} & (G'',*'',e'')}\end{matrix}
\quad\mapsto\quad\begin{matrix}
\xymatrix{G\ar[d]_{F(\varphi_1)}\ar[dr]^{F(\varphi_2\circ\varphi_1)}\\
  G'\ar[r]_{F(\varphi_2)} & G''}\end{matrix}
\]
\end{frame}

\begin{frame}
Ein zweites Beispiel. Sei $P(X)=2^X$ die Potenzmenge von $X$.
Dann ist wie folgt ein kovarianter Funktor gegeben:
\[P\colon\mathbf{Set}\to\mathbf{Set},\quad P(X):=2^X,\]
mit
\[P(f)(M) := f(M).\]
Dabei ist $f$ eine Abbildung und $f(M)$ die Bildmenge von $M$
unter $f$.
\end{frame}

\begin{frame}
Wir setzen $(g\circ f)(M)=g(f(M))$ als bekannt voraus und rechnen
damit wieder nach:
\begin{align*}
P(g\circ f)(M) &= (g\circ f)(M) = g(f(M))\\
&= P(g)(P(f)(M)) = (P(g)\circ P(f))(M),
\end{align*}
also gilt
\[P(g\circ f) = P(g)\circ P(f).\]
\end{frame}

\begin{frame}
Die Abbildung
\[P\colon \Abb(X,Y)\to\Abb(2^X,2^Y)\]
scheint dabei ein Einbettungsmonomorphismus zu sein.
\end{frame}

\begin{frame}
Außerdem gilt $f(\{x\})=\{f(x)\}$ für alle Abbildungen. Daher definiert
man $\eta(X)\colon X\to 2^X$ mit $\eta(X)(x):=\{x\}$. Dann
kommutiert das folgende Diagramm:
\[\xymatrix{
X \ar[r]^f \ar[d]_{\eta(X)} & Y \ar[d]^{\eta(Y)} \\
2^X \ar[r]_{P(f)} & 2^Y }\]
\end{frame}

\begin{frame}
Hier ordnet $\eta$ jedem Objekt $X\in\ob(\mathbf{Set})$ einen Morphismus
\[\eta(X)\colon \id(X)\to P(X)\]
zu, wobei $\id(X)=X$ und $P(X)=2^X$ gilt. Man schreibt dann auch
$\eta\colon \id\to P$ und bezeichnet $\eta$ als natürliche
Transformation.
\end{frame}

\begin{frame}
Seien $C,D$ Kategorien und $F,G\colon C\to D$ Funktoren. Dann schreibt
man $\eta\colon F\to G$ und nennt $\eta$ natürliche Transformation,
wenn die folgenden beiden Bedingungen erfüllt sind:
\begin{itemize}
\item jedes Objekt $X\in\ob(C)$ bekommt einen Morphismus\\
  $\eta(X)\colon F(X)\to G(X)$,
\item für jeden Morphismus $f\colon X\to Y$ gilt\\
  $\eta(Y)\circ F(f) = G(f)\circ\eta(X)$.
\end{itemize}
\end{frame}

\begin{frame}
Die zweite Bedingung lässt sich übersichtlich als kommutierendes
Diagramm darstellen:
\[\xymatrix{
F(X) \ar[r]^{F(f)} \ar[d]_{\eta(X)} & F(Y) \ar[d]^{\eta(Y)} \\
G(X) \ar[r]_{G(f)} & G(Y)}\]
\end{frame}

\begin{frame}
Im Beispiel war $F=\id$ und $G=P$.
\end{frame}

\begin{frame}
Ende.

\vfill
\modest{Dieser Text steht unter der Lizenz\\
Creative Commons CC0 1.0}
\end{frame}

\end{document}


