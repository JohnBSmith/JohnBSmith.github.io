\documentclass[8pt]{beamer}
\usetheme{Antibes}
\useinnertheme{rectangles}
\useoutertheme{infolines}
\usepackage[utf8]{inputenc}
\usepackage[T1]{fontenc}
\usepackage[ngerman]{babel}

% Patch the look of +, = in arev
\usefonttheme{serif}

\usepackage{arev}
% Patch punctuation to be upright
\DeclareMathSymbol{.}{\mathpunct}{operators}{`.}
\DeclareMathSymbol{,}{\mathpunct}{operators}{`,}

\usepackage{amsmath}
\usepackage{amssymb}
\usepackage{proof}
\usepackage{booktabs}

\setbeamertemplate{footline}{%
\begin{beamercolorbox}[ht=3.0ex,dp=1ex]{title in head/foot}
\hfill\footnotesize\insertpagenumber\enspace\enspace\end{beamercolorbox}}

\definecolor{bluegreen1}{rgb}{0.0,0.20,0.28}
\definecolor{bluegreen2}{rgb}{0.0,0.20,0.28}
\setbeamercolor*{palette primary}{fg=white,bg=bluegreen1}
\setbeamercolor*{palette secondary}{fg=white,bg=bluegreen2}
\setbeamercolor*{palette tertiary}{fg=white,bg=bluegreen2}
\setbeamercolor{itemize item}{fg=black}
\setbeamercolor{block title}{bg=bluegreen2}
\newcommand{\modest}[1]{{\small\color{gray}#1}}
\hypersetup{colorlinks,urlcolor=magenta}

\newcommand{\unit}[1]{\mathrm{#1}}
\newcommand{\strong}[1]{\textsf{\textbf{#1}}}
\newcommand{\defiff}{\quad:\Longleftrightarrow\quad}
\newcommand{\infernote}[1]{\!\text{\footnotesize #1}}
\renewcommand{\qedsymbol}{\ensuremath{\Box}}
\newcommand{\discharge}[1]{$\sim$#1}
\newcommand{\centerheadline}[1]{%
  \begin{center}\strong{#1}\end{center}}
\newcommand{\parspace}{\vspace{0.8em}}
\newcommand{\cond}{\rightarrow}
\newcommand{\bicond}{\leftrightarrow}
\newcommand{\Z}{\mathbb Z}
\newcommand{\R}{\mathbb R}

\title{Natürliches Schließen}
\subtitle{Teil 3: Theorien mit Gleichheit}
\date{}

\begin{document}

\begin{frame}
\maketitle
\end{frame}

\begin{frame}
Um mathematische Strukturen beforschen zu können, müssen den
logischen Axiomen auch noch mathematische Axiome hinzugefügt
werden.

\parspace
Beispielsweise die Axiome der Mengenlehre. Daneben gibt es aber auch
viele weniger umfängliche Theorien.

\parspace
Viele Theorien beinhalten einen Begriff der Gleichheit.
\end{frame}

\begin{frame}
\begin{block}{Axiom (Reflexivität)}
\[\vdash\forall x\colon x=x\]
\end{block}\pause

\begin{block}{Axiom (Symmetrie)}
\[\vdash\forall x\colon\forall y\colon x=y\cond y=x\]
\end{block}\pause

\begin{block}{Axiom (Transitivität)}
\[\vdash\forall x\colon\forall y\colon\forall z\colon x=y\land y=z\cond x=z\]
\end{block}
\end{frame}

\begin{frame}
Die Axiome induzieren Schlussregeln. Beispielsweise gilt
\[\dfrac{\Gamma\vdash t=t'}{\Gamma\vdash t'=t}\]
für beliebige Terme $t,t'$, denn:

\[
\infer{\Gamma\vdash t'=t}{
  \infer{\vdash t=t'\cond t'=t}{
    \infer{\vdash\forall y\colon t=y\cond y=t}{
      \infer{\vdash\forall x\colon\forall y\colon x=y\cond y=x}{}}}
& \Gamma\vdash t=t'}
\]
\end{frame}

\begin{frame}
\begin{block}{Schlussregel zur Reflexivität}
\[\dfrac{}{\Gamma\vdash t=t}\]
\end{block}\pause

\begin{block}{Schlussregel zur Symmetrie}
\[\dfrac{\Gamma\vdash t=t'}{\Gamma\vdash t'=t}\]
\end{block}\pause

\begin{block}{Schlussregel zur Transitivität}
\[\dfrac{\Gamma\vdash t=t'\qquad\Gamma'\vdash t'=t''}{\Gamma,\Gamma'\vdash t=t''}\]
\end{block}
\end{frame}

\begin{frame}
\centerheadline{Ersetzungsregeln}
\end{frame}

\begin{frame}
Im Folgenden sei $f(x)$ ein Funktionsterm in der Variable $x$,
beispielsweise $f(x):=2x$. Die Applikation ist als $f(t):=f(x)[x:=t]$
definiert.\pause

\parspace
In gleichartiger Weise sei $P(x)$ eine Formel in der Variable $x$,
beispielsweise $P(x):=(x>0)$. Die Applikation ist als $P(t):=P(x)[x:=t]$
definiert.\pause

\parspace
Wichtig: Hier ist mit $f$ keine allgemeine Funktion gemeint, sondern
eine, deren Funktionsterm auf ein Blatt Papier geschrieben werden kann.
Gleichermaßen ist mit $P$ kein allgemeines Prädikat gemeint, sondern
eines, dessen Formel auf ein Blatt Papier geschrieben werden kann.
\end{frame}

\begin{frame}
\begin{block}{Ersetzungsregel für Terme}
\[\dfrac{\Gamma\vdash t=t'}{\Gamma\vdash f(t)=f(t')}\]
\end{block}
Bzw. als Axiom
\[\vdash \forall x\colon\forall y\colon x=y\cond f(x)=f(y).\]
\end{frame}

\begin{frame}
\begin{block}{Ersetzungsregel für Formeln}
\[\dfrac{\Gamma\vdash t=t'}{\Gamma\vdash P(t)\bicond P(t')}\]
\end{block}\pause
Alternativ ginge:
\begin{block}{Ersetzungsregel für Formeln}
\[\dfrac{\Gamma\vdash t=t'\qquad\Gamma'\vdash P(t)}{\Gamma,\Gamma'\vdash P(t')}\]
\end{block}\pause
Die beiden Regeln sind äquivalent, denn (setze $x:=t$ und $y:=t'$):
\begin{small}
\[
\begin{array}{@{}l@{\qquad}l}
\infer{\Gamma,\Gamma'\vdash P(y)}{
  \infer{\Gamma\vdash P(x)\cond P(y)}{
    \infer{\Gamma\vdash P(x)\bicond P(y)}{\Gamma\vdash x=y}}
& \Gamma'\vdash P(x)}
&
\infer{\Gamma\vdash P(x)\bicond P(y)}{
  \infer{\Gamma\vdash P(x)\cond P(y)}{
    \infer{\Gamma, P(x)\vdash P(y)}{
      \Gamma\vdash x=y & \infer{P(x)\vdash P(x)}{}}}
& \infer{\Gamma\vdash P(y)\cond P(x)}{
    \infer{\Gamma,P(y)\vdash P(x)}{
      \Gamma\vdash x=y & \infer{P(y)\vdash P(y)}{}}}}
\end{array}
\]
\end{small}
\end{frame}

\begin{frame}
Außerdem gibt es in der Logik noch eine weitere besonders nützliche
Ersetzungsregel, die allerdings nicht vorausgesetzt werden muss.
Sie folgt aus den übrigen logischen Schlussregeln.

\parspace
Es bezeichne hierzu $P(X)$ eine Formel in der logischen Variablen $X$,
beispielsweise $P(X) := (A\land X)$. Zu einer weiteren Formel
$B$ verstehen wir abermals $P(B):=P(X)[X:=B]$ als Applikation.\pause

\begin{block}{Zulässige Ersetzungsregel}
\[\dfrac{\Gamma\vdash A\leftrightarrow B}{\Gamma\vdash P(A)\leftrightarrow P(B)}\]
\end{block}\pause
Bewiesen werden kann sie per struktureller Induktion über den
Formelaufbau.
\end{frame}

\begin{frame}
\centerheadline{Zahlentheorie}
\end{frame}

\begin{frame}
Beispiel einer Ableitung in der Zahlentheorie.
Das Diskursuniversum sei $U=\Z$. Wir schreiben kurz $m\mid n$
für »$m$ teilt $n$«. Man kann dies so definieren:
\[\vdash m\mid n\bicond\exists k\colon n=km.\]\pause
\strong{Satz.} \emph{Ist eine Zahl durch zwei teilbar, so
ist ihr Quadrat ebenfalls durch zwei teilbar.}\pause

\parspace
Zunächst die Beweisführung in Worten.
Es gelte $2\mid n$. Dann gibt es einen Zeugen $k$ mit $n=2k$.
Somit gilt $n^2 = (2k)n = 2(kn)$. Mit $k':=kn$ hat
man also einen Zeugen für $\exists k'\colon n^2=2k'$, womit
laut Definition $2\mid n^2$ gilt.\pause

\parspace
Die Ableitung von Gleichungen belassen wir informal.
Es findet sich:
\[
\infer{\vdash 2\mid n \cond 2\mid n^2}{
  \infer{2\mid n\vdash 2\mid n^2}{
    \infer{2\mid n\vdash\exists k\colon n=2k}{
      \infer{2\mid n\vdash 2\mid n}{}}
  & \infer{n=2k\vdash 2\mid n^2}{
      \infer{n=2k\vdash\exists k'\colon n^2=2k'}{
        \infer{n=2k\vdash n^2 = 2kn}{
          \infer{n=2k\vdash n=2k}{}}}}}}
\]
\end{frame}

\begin{frame}
Die Ableitung der Gleichung nochmals formal betrachtet:
\[
\infer[\infernote{Transitivgesetz}]{n=2k\vdash n^2 = 2(kn)}{
  \infer[\infernote{Ersetzungsregel}]{n=2k\vdash n^2 = (2k)n}{
    \infer{n=2k\vdash n=2k}{}
  }
& \infer[\infernote{Assoziativgesetz}]{\vdash (2k)n = 2(kn)}{}}
\]
\begin{footnotesize}
\strong{Bemerkung.} Das Assoziativgesetz ist aus den
Peano"=Axiomen herleitbar.
\end{footnotesize}
\end{frame}

\begin{frame}
Ende.
\vfill\hfill\modest{November 2022}\\
\hfill\modest{Creative Commons CC0 1.0}
\end{frame}


\end{document}


