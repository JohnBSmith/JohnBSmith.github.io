\documentclass[9pt,fleqn,aspectratio=169]{beamer}
\usetheme{Antibes}
\useinnertheme{rectangles}
\useoutertheme{infolines}
\usepackage[utf8]{inputenc}
\usepackage[T1]{fontenc}
\usepackage[ngerman]{babel}

% \usepackage{arev}
\usepackage{fouriernc}
\usefonttheme{serif}

% Patch punctuation to be upright for arev (Bitstream Vera Sans)
% \DeclareMathSymbol{.}{\mathpunct}{operators}{`.}
% \DeclareMathSymbol{,}{\mathpunct}{operators}{`,}

\usepackage{amsmath}
\usepackage{amssymb}
\usepackage{ebproof}

\setbeamertemplate{footline}{%
\begin{beamercolorbox}[ht=3.0ex,dp=1ex]{title in head/foot}
\hfill\footnotesize\insertpagenumber\enspace\enspace\end{beamercolorbox}}

\definecolor{fgcolor0}{rgb}{1.0,1.0,1.0}
\definecolor{bgcolor0}{rgb}{0.05,0.1,0.05}
\setbeamercolor{palette primary}{fg=fgcolor0,bg=bgcolor0}
\setbeamercolor{palette secondary}{fg=fgcolor0,bg=bgcolor0}
\setbeamercolor{palette tertiary}{fg=fgcolor0,bg=bgcolor0}
\setbeamercolor{palette quaternary}{fg=fgcolor0,bg=bgcolor0}
\setbeamercolor{itemize item}{fg=fgcolor0}
\setbeamercolor{background canvas}{bg=bgcolor0}
\setbeamercolor{normal text}{fg=fgcolor0}
% \hypersetup{colorlinks,urlcolor=magenta}

\beamertemplatenavigationsymbolsempty
\setbeamersize{text margin left=6em}
\setbeamersize{text margin right=6em}

\newcommand{\strong}[1]{\textbf{#1}}
\newcommand{\parspace}{\vspace{0.8em}}
\newcommand{\headline}[1]{\begin{center}\strong{#1}\end{center}}
\newcommand{\lnec}{\Box}
\newcommand{\inote}[1]{{\!\footnotesize #1}}

\ebproofset{rule thickness=0.4608pt}
% Thickness of fraction rules, i.e.
% \the\fontdimen8\textfont3

\begin{document}

\begin{frame}
\begin{center}
{\LARGE Natürliches Metaschließen}

\vspace{1em}
{\large Juli 2024}
\end{center}
\end{frame}

\begin{frame}
\headline{Motivation}
\end{frame}

\begin{frame}[t]
\vspace{0.5em}
Das Beweisen von Theoremen geschieht vermittels Schlussregeln. Darüber
hinaus erweisen sie sich aber auch zur Ableitung neuer Schlussregeln
als dienlich.\pause

\parspace
\strong{Beispiel.} Als gegeben vorausgesetzt seien die beiden Regeln
\[\dfrac{\Gamma\vdash A\quad\;\;\Gamma\!,\,A\vdash B}{\Gamma\vdash B}
\;\text{(Schnitt)},\qquad
\dfrac{}{A,B\vdash A\land B}.
\]\pause
Nun wird man die Regel
\[\dfrac{\Gamma\vdash A\quad\;\;\Gamma\vdash B}{\Gamma\vdash A\land B}\]
unschwer als zulässig befinden, denn\pause{} es findet sich der Herleitungsbaum
\only<4-5>{%
\[\begin{prooftree}
  \hypo{\Gamma\vdash A}
    \hypo{\Gamma\vdash B}\infer0{A,B\vdash A\land B}
  \infer2[Schnitt]{\Gamma,\,A\vdash A\land B}
\infer2[Schnitt.]{\Gamma\vdash A\land B}
\end{prooftree}\]}%
\only<6>{%
\[\begin{prooftree}
  \hypo{\color{cyan}\Gamma\vdash A}
    \hypo{\color{cyan}\Gamma\vdash B}\infer0{A,B\vdash A\land B}
  \infer2[Schnitt]{\Gamma,\,A\vdash A\land B}
\infer2[Schnitt.]{\color{magenta}\Gamma\vdash A\land B}
\end{prooftree}\]}%
\only<5-6>{%
Allerdings besteht hierbei ein technisches Problem:\\
\emph{Es wurde aus {\color{cyan}Prämissen} eine {\color{magenta}Konklusion} hergeleitet.}}
\end{frame}

\begin{frame}
\emph{Die Herleitung findet eigentlich innerhalb der Metalogik statt.}\pause

\parspace
Warum dies ein Problem darstellen sollte?\pause{}
\emph{Wenn wir einen Theorembeweisverifizierer verwenden, der lediglich die Gegenstandslogik
versteht, bleibt uns die metalogische Ebene verwehrt.}\pause

\parspace
Klären wir zunächst kurz ab, was bislang unausgesprochen blieb.
\end{frame}

\begin{frame}
Um die Metalogik als natürliches Schließen zu formalisieren,
notieren wir
\[S_1,\ldots,S_n\Vdash S\]
als Metasequenz, womit das Urteil gemeint sei, dass die Sequenz $S$ aus
den Sequenzen $S_1$ bis $S_n$ ableitbar ist.\pause

\parspace
Die eigentliche Gestalt der Herleitung klärt sich nun als
\[\begin{prooftree}
  \infer0{{\color{cyan}(\Gamma\vdash A)}\Vdash (\Gamma\vdash A)}
    \infer0{{\color{cyan}(\Gamma\vdash B)}\Vdash (\Gamma\vdash B)}
    \infer0{\Vdash (A,B\vdash A\land B)}
  \infer2[Schnitt]{{\color{cyan}(\Gamma\vdash B)}\Vdash (\Gamma,\,A\vdash A\land B)}
\infer2[Schnitt.]{{\color{cyan}(\Gamma\vdash A)},{\color{cyan}(\Gamma\vdash B)}\Vdash
  \color{magenta}(\Gamma\vdash A\land B)}
\end{prooftree}\]
\end{frame}

\begin{frame}
Infolgedessen muss aber jede der Schlussregeln eine Entsprechung
in der Metalogik induzieren. Das heißt, liegt eine Schlussregel
\[\dfrac{S_1\qquad S_2\qquad\ldots\qquad S_n}{S}\]
vor, muss in der Metalogik die entsprechende Regel
\[\dfrac{\Delta\Vdash S_1\qquad\Delta\Vdash S_2\qquad\ldots\qquad\Delta\Vdash S_n}
{\Delta\Vdash S}\]
zur Verfügung stehen.\pause

\parspace
\begin{footnotesize}
\strong{Bemerkung.} Bei Einbeziehung der Abschwächungsregel nimmt sie die Form
\[\dfrac{\Delta_1\Vdash S_1\qquad\Delta_2\Vdash S_2\qquad\ldots\qquad\Delta_n\Vdash S_n}
{\Delta_1,\Delta_2,\ldots,\Delta_n\Vdash S}\]
an, wie sie gerade genutzt wurde.
\end{footnotesize}
\end{frame}

\begin{frame}
\headline{Ein logischer Kalkül}
\end{frame}

\begin{frame}
Folgende Idee: Angenommen, das metalogische System beschreibt dieselbe
Logik wie die Gegenstandslogik, dann erscheint es zielführend,
das metalogische System in die Gegenstandslogik zu verschieben.\pause{}

\parspace
Zu den herkömmlichen Junktoren wird $\lnec_\Gamma A$ hinzugefügt,
mit der Bedeutung $\Gamma\vdash A$. Die klassische Semantik wird daraufhin
entsprechend erweitert um
\[\begin{array}{r@{}l}
(\mathcal I\models\lnec_\Gamma A)\;: &\Leftrightarrow\;
(\forall\mathcal I'\colon (\mathcal I'\models\Gamma)\to (\mathcal I'\models A))\\[2pt]
\; &\Leftrightarrow\; (\Gamma\models A).
\end{array}\]\pause
Die einzigen beiden Schlussregeln seien
\[\dfrac{\vdash A\to B\quad\;\;\vdash A}{\vdash B},\qquad\dfrac{\vdash A}{\vdash\lnec_\Gamma A}.\]
\end{frame}

\begin{frame}
Wir notieren nun die herkömmlichen Schlussregeln als Axiomenschemata. Das wären
\[\begin{array}{@{}l@{\qquad}l}
\lnec_{\Gamma,\,A} B \to \lnec_\Gamma (A\to B), & \text{(SE: Subjunktionseinführung)}\\[2pt]
\lnec_\Gamma(A\to B) \to \lnec_\Gamma A\to \lnec_\Gamma B. & \text{(SB: Subjunktionsbeseitigung)}
\end{array}\]\pause
Außerdem zur steht zur Verfügung,
\[\begin{array}{@{}l@{\qquad}l}
\lnec_A A, & \text{(GS: Grundsequenz)}\\[2pt]
\lnec_{\Gamma} A\to\lnec_{\Gamma,\Gamma'} A, & \text{(Abs: Abschwächung)}\\[2pt]
\lnec_\emptyset A\to A. & \text{(Lift)}
\end{array}\]\pause
Zunächst einmal benötigen wir die metalogischen Entsprechungen der
»Schlussregeln«. Ist $A\to B$ eine »Schlussregel«, benötigen wir
$\lnec_\Delta A\to\lnec_\Delta B$. Dies bereitet keine großen
Schwierigkeiten. Die Herleitung ist\pause{}
\[\begin{prooftree}
\hypo{\vdash A\to B}
\infer1{\vdash\lnec_\Delta (A\to B)}
\infer0{\vdash\lnec_\Delta (A\to B)\to\lnec_\Delta A\to\lnec_\Delta B}
\infer2[\!\!.]{\vdash\lnec_\Delta A\to\lnec_\Delta B}
\end{prooftree}\]
\end{frame}

\begin{frame}
Für »Schlussregeln« mit zwei Prämissen würden wir aber gerne das Theorem
\[\lnec_\Gamma (A\to B\to C)\to\lnec_\Gamma A\to\lnec_\Gamma B\to\lnec_\Gamma C
\qquad\text{(SBB)}\]
zur Verfügung haben.\pause{} Dazu wird zunächst der Hilfssatz
\[\lnec_\Delta \lnec_\Gamma (A\to B) \to \lnec_\Delta (\lnec_\Gamma A\to \lnec_\Gamma B)
\qquad\text{(HSB)}\]
abgeleitet. Die Herleitung ist\pause{}
\[\begin{prooftree}
    \infer0{\vdash\lnec_\Gamma (A\to B)\to \lnec_\Gamma A\to \lnec_\Gamma B}
  \infer1{\vdash\lnec_\Delta (\lnec_\Gamma (A\to B)\to \lnec_\Gamma A\to \lnec_\Gamma B)}
  \infer0{\vdash\lnec_\Delta (A'\to B')\to \lnec_\Delta A'\to \lnec_\Delta B'}
\infer2{\vdash\lnec_\Delta \lnec_\Gamma (A\to B) \to \lnec_\Delta (\lnec_\Gamma A\to \lnec_\Gamma B)}
\end{prooftree}\]
mit
\begin{gather*}
A' := \lnec_\Gamma (A\to B),\\
B' := (\lnec_\Gamma A\to \lnec_\Gamma B).
\end{gather*}
\end{frame}

\begin{frame}
Insofern $A,B$ ausschweifend werden, bietet es sich an, die Schreibweise
\[\dfrac{\vdash A}{\vdash B}\;\text{\footnotesize T}
\quad\text{als Abkürzung für}\quad
\dfrac{\begin{prooftree}[center=false]\infer0{\vdash A\to B}\end{prooftree}\quad\;\;\vdash A}{\vdash B}\]
zu nutzen, wobei T den Name des Theoremschemas meint, aus dem das Theorem
$A\to B$ hervorgeht.\pause{} Analog
\[\dfrac{\vdash A\quad\;\;\vdash B}{\vdash C}\;\text{\footnotesize T}\quad
\text{für}\quad
\begin{prooftree}
    \infer0{\vdash A\to B\to C}
    \hypo{\vdash A}
  \infer2{\vdash B\to C}
  \hypo{\vdash B}
\infer2{\vdash C}
\end{prooftree}\]
wobei T das Schema meint, aus dem $A\to B\to C$ hervorgeht.
\end{frame}

\begin{frame}
Es findet sich
\[\begin{prooftree}
                \infer0[\inote{GS}]{\lnec_{A'} \lnec_\Gamma (A\to B\to C)}
              \infer1[\inote{Abs}]{\lnec_{A'\!,\,B'} \lnec_\Gamma (A\to B\to C)}
            \infer1[\inote{HSB}]{\lnec_{A'\!,\,B'} (\lnec_\Gamma A\to\lnec_\Gamma (B\to C))}
          \infer1[\inote{SB}]{\lnec_{A'\!,\,B'} \lnec_\Gamma A\to\lnec_{A'\!,\,B'}\lnec_\Gamma (B\to C)}
            \infer0[\inote{GS}]{\lnec_{B'} \lnec_\Gamma A}
          \infer1[\inote{Abs}]{\lnec_{A'\!,\,B'} \lnec_\Gamma A}
        \infer2{\vdash\lnec_{A'\!,\,B'}\lnec_\Gamma (B\to C)}
      \infer1[\inote{HSB}]{\vdash\lnec_{A'\!,\,B'}(\lnec_\Gamma B\to\lnec_\Gamma C)}
    \infer1[\inote{SE}]{\vdash\lnec_{A'}(\lnec_\Gamma A\to\lnec_\Gamma B\to\lnec_\Gamma C)}
  \infer1[\inote{SE}]{\vdash\lnec_\emptyset(\lnec_\Gamma (A\to B\to C)\to\lnec_\Gamma A\to\lnec_\Gamma B\to\lnec_\Gamma C)}
\infer1[\inote{Lift}]{\vdash\lnec_\Gamma (A\to B\to C)\to\lnec_\Gamma A\to\lnec_\Gamma B\to\lnec_\Gamma C}
\end{prooftree}\]
mit
\begin{gather*}
A' := \lnec_\Gamma (A\to B\to C),\\
B' := \lnec_\Gamma A.
\end{gather*}
\end{frame}

\begin{frame}
Zu jeder »Schlussregel« der Form $A\to B\to C$ gelingt nun die Herleitung
\[\begin{prooftree}
    \hypo{\vdash A\to B\to C}
  \infer1{\vdash\lnec_\Delta (A\to B\to C)}
\infer1[\inote{SBB}.]{\vdash\lnec_\Delta A\to \lnec_\Delta B\to \lnec_\Delta C}
\end{prooftree}\]\pause
Folgende Konvention. Ist T der Name des als Schlussregel aufgefassten
Theoremschemas $A\to B$, so sei MT der Name des Theoremschemas
$\lnec_\Delta A\to\lnec_\Delta B$.

\parspace
Ist T entsprechend der Name für $A\to B\to C$, so sei MT der Name für
$\lnec_\Delta A\to\lnec_\Delta B\to\lnec_\Delta C$.
\end{frame}

\begin{frame}
Übrigens können wir diese zulässigen Regeln
\[\dfrac{\vdash A\to B}{\vdash\lnec_\Delta A\to\lnec_\Delta B},\qquad
\dfrac{\vdash A\to B\to C}{\vdash\lnec_\Delta A\to\lnec_\Delta B\to\lnec_\Delta C}\]
nun ebenfalls als Theoremschemata ausdrücken. Hierzu findet sich 
\[\begin{array}{@{}l@{\quad}l@{}}
\lnec_\emptyset (A\to B) \to\lnec_\Delta A\to\lnec_\Delta B, & \text{(ME1: Metaregeleinführung, 1 Prämisse)}\\
\lnec_\emptyset (A\to B\to C) \to\lnec_\Delta A\to\lnec_\Delta B\to\lnec_\Delta C. & \text{(ME2: Metaregeleinführung, 2 Prämissen)}
\end{array}\]\pause
Die Herleitung von ME1 ist zum Beispiel\pause
\[\begin{prooftree}
        \infer0[\!\footnotesize GS\!\!]{\vdash\lnec_{\lnec_\emptyset (A\to B)}\lnec_\emptyset (A\to B)}
      \infer1[\inote{MAbs}]{\vdash\lnec_{\lnec_\emptyset (A\to B)}\lnec_\Delta (A\to B)}
    \infer1[\inote{MSB}]{\vdash\lnec_{\lnec_\emptyset (A\to B)}(\lnec_\Delta A\to\lnec_\Delta B)}
  \infer1[\inote{SE}]{\vdash\lnec_\emptyset(\lnec_\emptyset (A\to B)\to\lnec_\Delta A\to\lnec_\Delta B)}
\infer1[\inote{Lift}.]{\vdash\lnec_\emptyset (A\to B)\to\lnec_\Delta A\to\lnec_\Delta B}
\end{prooftree}\]
\end{frame}

\begin{frame}
Wir können nun die Schnittregel als Theoremschema
\[\lnec_{\Gamma\!,\,\,A}B\to\lnec_\Gamma A\to\lnec_\Gamma B\]
herleiten. Es findet sich\pause{}
\[\begin{prooftree}
        \infer0[\!\footnotesize GS\!\!]{\vdash\lnec_{\lnec_{\Gamma\!,\,\,A}B}\lnec_{\Gamma\!,\,\,A}B}
      \infer1[\inote{MSE}]{\vdash\lnec_{\lnec_{\Gamma\!,\,\,A}B}\lnec_\Gamma (A\to B)}
    \infer1[\inote{MSB}]{\vdash\lnec_{\lnec_{\Gamma\!,\,\,A}B}(\lnec_\Gamma A\to\lnec_\Gamma B)}
  \infer1[\inote{SE}]{\vdash\lnec_\emptyset(\lnec_{\Gamma\!,\,\,A}B\to\lnec_\Gamma A\to\lnec_\Gamma B)}
\infer1[\inote{Lift}.]{\vdash\lnec_{\Gamma\!,\,\,A}B\to\lnec_\Gamma A\to\lnec_\Gamma B}
\end{prooftree}\]
\end{frame}

\begin{frame}
Nun sind wir befähigt, die am Anfang beschriebene Herleitung zu formalisieren.
Aus dem Axiom $\lnec_{A,B} (A\land B)$ soll mithilfe der »Schnittregel« die »Regel«
\[\lnec_\Gamma A\to\lnec_\Gamma B\to\lnec_\Gamma (A\land B)\]
abgeleitet werden. Es findet sich der Herleitungsbaum\pause
\[\begin{prooftree}
              \infer0{\vdash\lnec_{A,B}(A\land B)}
            \infer1[\inote{Abs}]{\vdash\lnec_{\Gamma,A,B}(A\land B)}
          \infer1{\vdash\lnec_\Delta\lnec_{\Gamma,A,B}(A\land B)}
              \infer0[\inote{GS}]{\vdash\lnec_{\lnec_\Gamma B}\lnec_\Gamma B}
            \infer1[\inote{MAbs}]{\vdash\lnec_{\lnec_\Gamma B}\lnec_{\Gamma,A} B}
          \infer1[\inote{Abs}]{\vdash\lnec_\Delta\lnec_{\Gamma,A} B}
        \infer2[\inote{MSchnitt}]{\vdash\lnec_\Delta\lnec_{\Gamma\!,\,\,A}(A\land B)}
          \infer0[\inote{GS}]{\vdash\lnec_{\lnec_\Gamma A}\lnec_\Gamma A}
        \infer1[\inote{Abs}]{\vdash\lnec_\Delta\lnec_\Gamma A}
      \infer2[\inote{MSchnitt}]{\vdash\lnec_\Delta\lnec_\Gamma (A\land B)}
    \infer1[\inote{SE}]{\vdash\lnec_{\lnec_\Gamma A}(\lnec_\Gamma B\to\lnec_\Gamma (A\land B))}
  \infer1[\inote{SE}]{\vdash\lnec_\emptyset(\lnec_\Gamma A\to\lnec_\Gamma B\to\lnec_\Gamma (A\land B))}
\infer1[\inote{Lift}]{\vdash\lnec_\Gamma A\to\lnec_\Gamma B\to\lnec_\Gamma (A\land B)}
\end{prooftree}\]
mit $\Delta:=\{\lnec_\Gamma A,\lnec_\Gamma B\}$.
\end{frame}

\begin{frame}
Da ist schon ein wenig umständlich. Wo die restlichen »Regeln« des
natürlichen Schließens zur Verfügung stehen, kann man alternativ
\[\lnec_\emptyset(A\to B\to A\land B)\]
herleiten, und gelangt damit via ME2 kurzerhand zu
\[\lnec_\Gamma A\to\lnec_\Gamma B\to\lnec_\Gamma (A\land B).\]
\end{frame}

\begin{frame}
\headline{Eine Vereinfachung}
\end{frame}

\begin{frame}
Zu $\Gamma=\{H_1,\ldots,H_n\}$ setzen wir $H:=H_1\land\ldots\land H_n$
und dürfen $\lnec_\Gamma A$ überall gegen $\lnec_H A$ ersetzen,
da diese beiden Formeln äquivalent sind.\pause

\parspace
Weiterhin sind $\lnec_H A$ und $\lnec_\emptyset (H\to A)$ äquivalent,
womit $\lnec_H A$ überall gegen $\lnec_\emptyset (H\to A)$ ersetzt
werden darf.\pause

\parspace
Es wird $\lnec A:=\lnec_\emptyset A$ geschrieben, da dies weniger
umständlich ist.
\end{frame}

\begin{frame}
Die Einführung und Beseitigung der Subjunktion wird
nun beschrieben durch
\begin{gather*}
\lnec (H\land A\to B) \to \lnec (H\to A\to B),\\
\lnec (H\to A\to B) \to \lnec(H\to A) \to \lnec(H\to B).
\end{gather*}\pause
Sie geben sich nun als nezessisierte Formen der Schemata
\begin{gather*}
(H\land A\to B) \to (H\to A\to B),\\
(H\to A\to B) \to (H\to A) \to (H\to B)
\end{gather*}
zu erkennen.
\end{frame}

\begin{frame}
Außerdem bestehen die Schemata
\begin{gather*}
\lnec (A\to B) \to (\lnec A \to \lnec B),\\
\lnec A \to A,\\
\lnec A \to \lnec\lnec A.
\end{gather*}\pause
Dies sind die Axiome der Modallogik S4.
\end{frame}

\begin{frame}
Das letzte Schema darin ist neu, aber intuitiv verständlich: Wenn $A$ ein
Theorem ist, ist es auch ein Theorem, dass $A$ ein Theorem ist.\pause

\parspace
Wir bleiben skeptisch, rechnen $\models (\lnec A\to \lnec\lnec A)$ lieber nach.\pause

\parspace
Sei $\mathcal I$ fest, aber beliebig. Aus $\mathcal I\models\lnec A$
ist $\mathcal I\models\lnec\lnec A$ zu folgern, also $\mathcal I'\models\lnec A$
für jedes $\mathcal I'$, also $\mathcal I''\models A$ für jedes $\mathcal I''$.
Laut Prämisse gilt $\mathcal J\models A$ für jedes $\mathcal J$. Wir
spezialisieren bezüglich $\mathcal J := \mathcal I''$ und sind fertig.
\end{frame}

\begin{frame}
\headline{Bilanz}
\end{frame}

\begin{frame}
Ist dies nicht ein wenig umständlich?\pause{} Es mag auf den ersten
Blick so erscheinen. Dagegen spricht allerdings folgendes.
\begin{itemize}
\item Der Aufwand konzentriert sich eigentlich nur auf die Herleitung
  einer Reihe von Hilfssätzen. Der Rest läuft ab wie zuvor im
  herkömmlichen natürlichen Schließen.
\item Der Verifizierer wird eleganter, da die Zahl der Schlussregeln
  auf zwei reduziert wird. Gleichzeitig wird ermöglicht, durch die
  Wahl unterschiedlicher Axiomenschemata alternative Systeme von
  »Schlussregeln« aufzustellen.
\end{itemize}\pause
Allerdings ist diese Metalogik recht schwach. Es können nur ableitbare
Schlussregeln hergeleitet werden, nicht aber andere denkbare zulässige.
Eine Regel wird zulässig genannt, wenn es ein Verfahren gibt, dass
diese auf die Anwendung der verfügbaren Regeln zurückführt. Zur Konstruktion
so eines Verfahrens müsste aber strukturelle Induktion über den Aufbau von
Formeln und Beweisen zur Verfügung stehen. Damit einhergehend wäre wohl
zudem ein Rattenschwanz an Hilfsbegriffen und Mechanismen erforderlich.
\end{frame}

\end{document}
