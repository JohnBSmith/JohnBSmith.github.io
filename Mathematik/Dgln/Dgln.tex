\documentclass[a4paper,10pt,fleqn,twocolumn,twoside,openany]{article}
\usepackage[utf8]{inputenc}
\usepackage[T1]{fontenc}
\usepackage{lmodern}
\usepackage{ngerman}
\usepackage{amsmath}
\usepackage{amssymb}
\usepackage{amsthm}
\usepackage{textcomp}
\usepackage{microtype}

\usepackage{color}
\definecolor{c1}{RGB}{00,40,80}
\definecolor{c2}{RGB}{20,60,100}
\definecolor{c3}{RGB}{80,120,180}
\usepackage[colorlinks=true,linkcolor=c1]{hyperref}
\usepackage{geometry}
\geometry{a4paper,left=25mm,right=10mm,top=24mm,bottom=34mm}
\setlength{\columnsep}{4mm}
\usepackage{lipsum}
\usepackage{multicol}

\setcounter{secnumdepth}{4}
\setcounter{tocdepth}{3}
\usepackage{tocloft}
\setlength{\cftsecindent}{0pt}
\setlength{\cftsubsecindent}{15pt}
\setlength{\cftsubsubsecindent}{23pt}
\renewcommand{\cftsecfont}{\normalfont\sffamily\bfseries}
\renewcommand{\cftsubsecfont}{\normalfont\sffamily}
\renewcommand{\cftsubsubsecfont}{\normalfont\sffamily}
\renewcommand\cftsecpagefont{\normalfont\sffamily\bfseries}
\renewcommand\cftsubsecpagefont{\normalfont\sffamily}
\renewcommand\cftsubsubsecpagefont{\normalfont\sffamily}

\usepackage{titlesec}
\titleformat{\section}[block]
  {\normalfont\sffamily\LARGE\bfseries}{\thesection}{1em}{\LARGE}
\titleformat{\subsection}[block]
  {\normalfont\sffamily\Large\bfseries}{\thesubsection}{1em}{\Large}
\titleformat{\subsubsection}[block]
  {\normalfont\sffamily\large\bfseries}{\thesubsubsection}{1em}{\large}

\titlespacing*{\section}{0pt}{10pt}{4pt}
\titlespacing*{\subsection}{0pt}{2pt}{2pt}
\titlespacing*{\subsubsection}{0pt}{2pt}{2pt}

\usepackage[justification=RaggedRight,singlelinecheck=off]{caption}

\numberwithin{equation}{section}

\renewcommand{\baselinestretch}{1.0}

\newcommand{\strong}[1]{{\sffamily\bfseries #1}}

% \ui: imaginäre Einheit
% \ue: Einheitsvektor
% \ue: eulersche Zahl
% \uv{x}: unterstrichener Vektor

\newcommand{\ui}{\mathrm i}
\newcommand{\ee}{\mathrm e}
\newcommand{\uv}[1]{\underline{#1}}

\newcommand{\N}{\mathbb N}
\newcommand{\Z}{\mathbb Z}
\newcommand{\R}{\mathbb R}
\newcommand{\C}{\mathbb C}

\DeclareMathOperator*{\sgn}{sgn}
\DeclareMathOperator*{\Real}{Re}
\DeclareMathOperator*{\Imag}{Im}
\DeclareMathOperator*{\rg}{rg}
\DeclareMathOperator*{\diag}{diag}
\DeclareMathOperator*{\Eig}{Eig}

\theoremstyle{definition}
\newtheorem{Definition}{Definition}

\begin{document}
\setlength{\abovedisplayskip}{6pt}
\setlength{\belowdisplayskip}{6pt}
\setlength{\abovedisplayshortskip}{6pt}
\setlength{\belowdisplayshortskip}{6pt}

\begin{titlepage}
\centering
\phantom{x}

\vspace{20em}
{\noindent\Huge\sffamily\textbf{Differentialgleichungen}}

\vspace{2em}
{\Large Dezember 2016}\\
\end{titlepage}

\thispagestyle{empty}

\renewcommand{\contentsname}{
\normalfont\sffamily\bfseries\LARGE
Inhaltsverzeichnis}
\tableofcontents

\section{Vorbereitungen}
\subsection{Notation}
Manchmal werden aus pragmatischen Gründen unpräzise Schreibweisen
gewählt. Wenn eine Variable beim bestimmten Integral sowohl als Grenze
als auch als Integrationsvariable auftaucht, so definiert man
\begin{equation}
\int_a^x f(x)\,\mathrm dx := \int_a^x f(t)\,\mathrm dt.
\end{equation}
Es gibt auch Ausdrücke, wo eine solche ungenaue Schreibweise gar nicht
möglich ist. Z.\,B. bei
\begin{equation}
\int_a^x f(xt)\,\mathrm dt.
\end{equation}
Beim Lösen von Differentialgleichungen
wird an vielen Stellen die Substitutionsregel
\begin{equation}
\int_a^b f(g(x))\,g'(x)\,\mathrm dx
= \int_{g(a)}^{g(b)} f(u)\,\mathrm du
\end{equation}
verwendet. Oft schreibt man dabei auch
\begin{equation}
g'(x)\,\mathrm dx = u'(x)\,\mathrm dx
= \frac{\mathrm du}{\mathrm dx}\,\mathrm dx
= \mathrm du.
\end{equation}
Bei dieser ungenauen Schreibweise wird $\mathrm dx$ einfach
{\glqq}gekürzt{\grqq}. Hier darf die Substitution bei den
Integralgrenzen nicht vergessen werden. Wird $u=g(x)$ substituiert,
so ist $g^{-1}(u)=x$ und man schreibt präziser
\begin{equation}
\int_a^b = \int_{x=a}^{x=b} = \int_{g^{-1}(u)=a}^{g^{-1}(u)=b}
= \int_{u=g(a)}^{u=g(b)}
\end{equation}
Die Substitutionsregel verlangt übrigens (anders als der
Transformationssatz) nicht, dass $g$ invertierbar ist.

Noch etwas: Normalerweise schreibt man den Hauptsatz als
\begin{equation}
\int_a^b y'\,\mathrm dx = y(b)-y(a).
\end{equation}
Man kann aber auch die Differentiale {\glqq}kürzen{\grqq},
vergisst nicht die Grenzen zu substituieren, und schreibt dann
\begin{equation}
\int_a^b \frac{\mathrm dy}{\mathrm dx}\,\mathrm dx
= \int_{y(a)}^{y(b)}\mathrm dy.
\end{equation}
Auch in der Theorie gewöhnlicher Differentialgleichungen werden
partielle Ableitungen auftreten. Für partielle Ableitungen soll
auch die eulersche Schreibweise Verwendung finden. Man definiert
\begin{equation}
D_x := \frac{\partial}{\partial x},\quad
D_y := \frac{\partial}{\partial y}.
\end{equation}
Wenn die Variablen mit einem Index nummeriert werden, so
tut man das bei den partiellen Ableitungen ebenfalls und schreibt
\begin{equation}
D_k := \frac{\partial}{\partial x_k}.
\end{equation}

\subsection{Begriffe}
Was ist eine Differentialgleichung? Um diesen Begriff besser in den
allgemeinen mathematischen Rahmen einordnen zu können, wollen wir eine
Differentialgleichung (Dgl.) als Spezialfall einer Funktionalgleichung
beschreiben. Eine Funktionalgleichung ist nun informell eine Gleichung,
deren Lösungsmenge nicht aus Zahlen besteht, sondern aus Funktionen.

\begin{Definition}
Sei $f\colon D\to Z$. Eine \emph{Funktionalgleichung} ist
eine Bestimmungsgleichung der Form
\begin{equation}\label{eq:Funktionalgleichung}
\forall x{\in}D\colon\;H(x,f)=0
\end{equation}
mit $g\colon D\times\operatorname{Abb}(D,Z)\to G^m$ wobei $(G,+)$
eine abelsche Gruppe ist. Gefunden werden soll die
Lösungsmenge%
\begin{equation}
L = \{f\mid\forall x\colon H(x,f)=0\}.\;\Box
\end{equation}
\end{Definition}
\noindent
Man beachte, dass in dieser Definition auch Systeme
von Funktionalgleichungen
mit eingeschlossen sind. Außerdem kann z.\,B. $D=\R^n$ sein, womit
sich Funktionen in mehreren Variablen beschreiben lassen.

Gleichung \eqref{eq:Funktionalgleichung} ist sehr abstrakt und
beschreibt eine unheimliche Vielfalt an Funktionen.
Wir wollen uns daher zunächst auf reelle Funktionen beschränken.

An dieser Stelle sei noch angemerkt, dass
\eqref{eq:Funktionalgleichung} auch eine Verallgemeinerung der
impliziten Beschreibung einer Funktion ist. Man spezifiziert dazu
\begin{equation}
H(x,f) := F(x,f(x)),\quad F\colon D\times Z\to G^m.
\end{equation}
Auch Rekursionsgleichungen sind Spezialfälle von
Funktionalgleichungen. Wähle dazu $f\colon\N\to Z$, d.\,h. $f$ soll
eine Folge sein. Eine Rekursionsgleichung erster Ordnung erhält man
nun durch die Festlegung
\begin{equation}
H(x,f) := f(x)-g(x,f(x-1)).
\end{equation}
Es ist nun so, dass irgendwo in der Berechnung von $H(x,f)$
Differentialoperatoren enthalten sein können. Wir wollen nun,
dass ausschließlich solche Operatoren verwendet werden
und schränken die Struktur von \eqref{eq:Funktionalgleichung}
durch die Festlegung
\begin{equation}
H(x,f) := g(x,(Df)(x),(D^2 f)(x),\ldots,(D^n f)(x))
\end{equation}
stark ein, wobei $D$ der gewöhnliche Ableitungsoperator sein soll,
der einer Funktion ihre erste Ableitung zuordnet. Dann ist $D^2$ die
zweite Ableitung usw. Eine solche Gleichung wird dann als
\emph{Differentialgleichung} bezeichnet. Wichtig ist hierbei,
dass nur die Ableitungen an der Stelle $x$ betrachtet werden.
Z.\,B. sind Funktionalgleichungen der Form
\begin{equation}
H(x,f) := g(x,(Df)(x),(Df)(x-1))
\end{equation}
darin nicht enthalten.
\begin{Definition}
Eine \emph{implizite Differentialgleichung} der Ordnung $n$
ist eine Gleichung der Form
\begin{equation}
\forall x\colon g(x,y_0,y_1,y_2,\ldots,y_n)=0.
\end{equation}
mit $y_k=(D^k f)(x)$. Dabei sollen zunächst nur solche Funktionen $f$
in Frage kommen, die auf einem offenen Definitionsbereich
definiert und dort differenzierbar sind. $\Box$
\end{Definition}
\noindent
Manchmal lässt sich eine implizite Differentialgleichung
in eine explizite Form bringen.
\begin{Definition}
Eine \emph{explizite Differentialgleichung} der Ordnung $n$
ist eine Gleichung der Form
\begin{equation}
\forall x\colon y_n = g(x,y_0,y_1,\ldots,y_{n-1})
\end{equation}
mit $y_k:=(D^k f)(x)$. $\Box$
\end{Definition}

\section{Lösungsmethoden}
Man betrachte das Anfangswertproblem
\begin{equation}\label{eq:AWP-exp}
f'(x) = f(x), \quad f(0)=1,
\end{equation}
dessen Lösung gefunden werden soll. Die Gleichung lässt sich
beschreiben durch
\begin{equation}
y' = g(x,y)
\end{equation}
mit $g(x,y):=y$. Die Lipschitz-Bedingung führt hier auf%
\begin{equation}
|g(x,y_2)-g(x,y_1)| = |y_2-y_1|\le L|y_2-y_1|,
\end{equation}
also $L\ge 1$. Somit ist $g$ global Lipschitz-stetig. Daher muss
es eine eindeutige Lösung der Dgl. geben.

\subsection{Klassische Methode}
Wir fordern zunächst, dass $f(x)$ keine Nullstellen habe. Somit
lässt sich die Gleichung \eqref{eq:AWP-exp} in die Form
\begin{equation}\label{eq:AWP-exp-Umformung}
\frac{f'(x)}{f(x)}=1
\end{equation}
bringen. Man verwendet nun die Substitutionsregel in der Form
\begin{equation}
\int_a^b h(f(x))\,f'(x)\,\mathrm dx
= \int_{f(a)}^{f(b)} h(u)\,\mathrm du.
\end{equation}
Hier ist $h(u)=1/u$. Somit ergibt sich
\begin{equation}
\int_0^x \frac{f'(x)}{f(x)}\,\mathrm dx
= \int_{f(0)}^{f(x)} \frac{1}{u}\,\mathrm du
= \ln|f(x)|-\ln|f(0)|.
\end{equation}
Diese Rechnung wird auch als \emph{logarithmische Integration}
bezeichnet. Es ist an dieser Stelle praktisch, die Abkürzungen
$y:=f(x)$ und $y_0:=f(0)$ zu verwenden.

Auf der anderen Seite von \eqref{eq:AWP-exp-Umformung} erhalten wir
\begin{equation}
\int_0^x 1\,\mathrm dx = x.
\end{equation}
Integriert man Gleichung \eqref{eq:AWP-exp-Umformung} auf beiden
Seiten, so erhält man nun
\begin{equation}
\ln|y|-\ln|y_0| = x.
\end{equation}
Umformung dieser Gleichung bringt
\begin{equation}
|y| = |y_0|\,\ee^x.
\end{equation}
Verwendet man nun die allgemeine Regel
\begin{equation}
|y|=y\sgn(y),
\end{equation}
so ergibt sich
\begin{equation}
y\sgn(y) = y_0\sgn(y_0)\,\ee^x.
\end{equation}
Wenn $y$ nun differenzierbar sein soll, so muss $y$ auch stetig sein.
Weil wir fordern, dass $y$ keine Nullstellen hat, muss aus
Stetigkeitsgründen entweder $y>0$ für alle $x$ oder $y<0$ für alle
$x$ sein. Daher ist $\sgn(y)=\sgn(y_0)$. Man erhält
\begin{equation}
y = y_0\,\ee^x.
\end{equation}
Was ist, wenn die Funktion $y$ eine Nullstelle besitzt? Wenn $a$
eine Nullstelle ist, so wird dort wegen der Dgl. \eqref{eq:AWP-exp}
auch $y^{(n)}(a)=0$ sein. Setzt man voraus, dass $y$ analytisch ist,
so kann die Taylorreihe an der Stelle $a$ gebildet werden und man
erhält die Funktion
\begin{equation}
\forall x\colon y(x)=0.
\end{equation}
Das Anfangswertproblem kann hiermit nicht erfüllt werden, d.\,h.
die Lösungsmenge ist leer.

\subsection{Picard-Iteration}

Als Beispiel soll wieder die Dgl. $y'=y$ dienen.
Diese integriert man auf beiden Seiten und erhält
\begin{equation}
y=y(0)+\int_0^x y\,\mathrm dx.
\end{equation}
Diese Umformung bringt uns eigentlich nicht weiter.
Der Ansatz ist nun, zu erkennen, dass es sich um eine
Fixpunktgleichung handelt. Sei $K:=y(0)$.
Wir wählen nun $y_1:=0$ und iterieren gemäß
\begin{equation}
y_{n+1} = K+\int_0^x y_n\,\mathrm dx.
\end{equation}
Der Anfang dieser Iteration ist:
\begin{gather}
y_2 = K,\\
y_3 = K+Kx,\\
y_4 = K+Kx+\frac{1}{2}Kx^2,\\
y_5 = K+Kx+\frac{1}{2}Kx^2+\frac{1}{6}Kx^3.
\end{gather}
Das geht immer so weiter, der Grenzwert ist
\begin{equation}
y = K\sum_{k=0}^{\infty} \frac{x^k}{k!} = K\ee^x.
\end{equation}

\subsection{Potenzreihenmethode}

Als Beispiel soll wieder die Dgl. $y'=y$ dienen.
Wenn sich die Funktion $y(x)$ als Potenzreihe darstellen lässt, so ist
\begin{equation}
y=\sum_{k=0}^\infty a_k(x-x_0)^k.
\end{equation}
Wir wählen den Entwicklungspunkt $x_0=0$. Die Potenzreihe setzt man
nun in die Dgl. ein und erhält
\begin{equation}
\sum_{k=1}^\infty ka_kx^k = \sum_{k=0}^\infty a_kx^k.
\end{equation}
Mit einem Koeffizientenvergleich erhält man $ka_k=a_{k-1}$.
Über die Rekursionsformel für die Fakultät erhält man $k!a_k=a_0$.
Einsetzen von $a_k$ in den allgemeinen Potenzreihenansatz bringt dann
\begin{equation}
y=\sum_{k=0}^\infty a_0\frac{x^k}{k!} = a_0\ee^x.
\end{equation}
Alternativ kann man auch nach folgender Methode rechnen.
Wenn sich die Funktion $y$ an der Stelle $x_0$ in eine Potenzreihe
entwickeln lässt, so ist nach dem Satz von Taylor
\begin{equation}
y = \sum_{k=0}^\infty \frac{1}{k!}y^{(k)}(x_0)(x-x_0)^k.
\end{equation}
Entwickelt man nun an der Stelle $x_0=0$ und verwendet die Dgl.,
so erhält man
\begin{equation}
y = \sum_{k=0}^\infty \frac{1}{k!}y(0)x^k
= y(0)\sum_{k=0}^\infty \frac{x^k}{k!}
= y(0)\ee^x.
\end{equation}

\subsection{Laplace-Transformation}

Bei linearen Dgln. mit konstanten Koeffizienten kann man die
Laplace-Transformation zur Lösung verwenden. Man Transformiert
erst die Dgl., wobei eine algebraische Gleichung entsteht. Diese
wird gelöst. Die Lösung kann man anschließend zurück transformieren.

Als Beispiel soll wieder die Dgl. $y'=y$ dienen. Mit der Rechenregel
$L\{y'\} = pL\{y\}-y_0$ erhält man
\begin{equation}
pL\{y\}-y_0 = L\{y\}.
\end{equation}
Umformen bringt
\begin{equation}
L\{y\} = \frac{y_0}{p-1}.
\end{equation}
Es ist $L\{x\} = 1/p$. Zusammen mit dem Dämpfungssatz erhält man
\begin{equation}
L\{\ee^x\} = \frac{1}{p-1}.
\end{equation}
Man rechnet nun
\begin{equation}
L\{y_0\ee^x\} = y_0L\{\ee^x\} = \frac{y_0}{p-1}.
\end{equation}
Die Lösung $y=y_0\ee^x$ lässt sich ablesen.

\subsection{Eigenwertgleichung}

Die Gleichung $y'=y$ schreibt man auch $Dy=y$. Mit $\lambda=1$
kann man $Dy=\lambda y$ schreiben. Das ist analog zum Eigenwertproblem
$Av=\lambda v$ in der linearen Algebra. Die Rolle der Matrix übernimmt
der Differentialoperator $D$. Wie findet man die Eigenvektoren?
Man macht den Ansatz $y_b=\ee^{\lambda x}$. In diesem Fall ist
also $y_b=\ee^x$.

Der Eigenvektor $y_b$ bildet nun die Basis eines eindimensionalen
Vektorraums. Alle Lösungen lassen sich als Linearkombination von $y_b$
darstellen. Linearkombinationen sind z.\,B. $2y_b$ oder $y_b+3y_b$.
Allgemein erhält man%
\begin{equation}
y = Ky_b = K\ee^x.
\end{equation}
Nehmen wir als zweites Beispiel die Gleichung $y''=-4y$. Bezüglich
des Differentialoperators $D'=D^2$ ist der Eigenwert $\lambda'=-4$.
Zunächst macht man aber bei linearen Dgln. mit konstanten
Koeffizienten immer den Ansatz $y_b=\ee^{\lambda x}$. Dieser
wird in die Dgl. eingesetzt und man erhält
$\lambda^2 \ee^{\lambda x} = -4\ee^{\lambda x}$.
Division durch $\ee^{\lambda x}$ bringt dann $\lambda^2 = -4$.
Man erhält die Lösungen $\lambda_1 = -2i$ und $\lambda_2 = 2i$.
Die Linearkombinationen sind
\begin{equation}
y=K_1\ee^{-2ix}+K_2\ee^{2ix}.
\end{equation}
Die Eigenwerte $\lambda_k$ aus dem Ansatz muss man vom
Eigenwert $\lambda'$ des Differentialoperators $D'$ unterscheiden,
wenn es sich nicht um den gewöhnlichen Differentialoperator $D$
handelt.

\subsection{Tricks}
Spezielle Typen von Differentialgleichungen lassen sich mit
speziellen Tricks Lösen. Dazu gehören z.\,B. Substitutionen.

\section{Lineare GDG mit konstanten Koeffizienten}
\subsection{Erste Ordnung, homogen}
Zu lösen ist die Gleichung
\begin{equation}
y' = ay.
\end{equation}
Lösungsmethoden für diesen Typ wurden schon ausführlich besprochen.
Es ergibt sich
\begin{equation}
y = K\ee^{ax}.
\end{equation}
Das AWP $y'(x_0) = y_0$ führt auf
\begin{equation}
K=\frac{y_0}{a\ee^{ax_0}}.
\end{equation}

\subsection{Erste Ordnung, inhomogen}
Zu lösen ist die Gleichung
\begin{equation}
y' = ay+b.\quad (a\ne 0)
\end{equation}
Wähle die Substitution $y=u+c$, wobei
$c$ eine Konstante ist. Nun gilt
\begin{align}
y' &= (u+c)' = u',\\
ay+b &= a\cdot (u+c)+b = au+ac+b.
\end{align}
Wir wollen nun $ac+b$ verschwinden lassen, d.\,h. wir setzen
$ac+b=0$. Somit ergibt sich $c=-b/a$. Als neue Dgl. ergibt sich
$u' = au$. Diese Gleichung lässt sich aber über die Trennung
der Variablen lösen, wobei sich $u=K\ee^{ax}$ ergibt. Resubstitution
führt zu
\begin{equation}
y = K\ee^{ax}-\tfrac{b}{a}.
\end{equation}
Die Probe durch Einsetzen in die Dgl. bestätigt die Lösung.
Für das AWP $y'(x_0)=y_0$ ergibt sich
\begin{equation}
K = \frac{y_0}{a\ee^{ax_0}}.
\end{equation}

\begin{thebibliography}{x}
\bibitem{Heuser} Harro Heuser: »\emph{Gewöhnliche
Differentialgleichungen}«. Teubner, Wiesbaden 1989, 4. Auflage 2004.

\bibitem{Strehmel} Karl Strehmel, Rüdiger Weiner, Helmut Podhaisky:
»\emph{Numerik gewöhnlicher Differentialgleichungen}«. Springer
Fachmedien, Wiesbaden 1995, 2. Auflage 2012.

\bibitem{Dieudonne} J. Dieudonné:
»\emph{Grundzüge der modernen Analysis}«, Band 4.
Vieweg, Braunschweig 1976.
\end{thebibliography}

\vfill
\noindent
\texttt{Dieses Heft ist unter der Creative-\\
Commons-Lizenz CC0 veröffentlicht.}

\end{document}


