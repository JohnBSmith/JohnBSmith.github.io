\documentclass{beamer}
\usetheme{Antibes}
\useinnertheme{rectangles}
\useoutertheme{infolines}
\usepackage[utf8]{inputenc}
\usepackage[T1]{fontenc}

% patch the look of +, = in arev
\usefonttheme{serif} 

\usepackage{arev}
\usepackage{amsmath}
\usepackage{amssymb}

\setbeamertemplate{footline}{%
\begin{beamercolorbox}[ht=3.0ex,dp=1ex]{title in head/foot}
\hfill\footnotesize\insertpagenumber\enspace\enspace\end{beamercolorbox}}

\definecolor{brown1}{rgb}{0.20,0.10,0}
\definecolor{brown2}{rgb}{0.20,0.10,0}
\setbeamercolor*{palette primary}{fg=white,bg=brown1}
\setbeamercolor*{palette secondary}{fg=white,bg=brown2}
\setbeamercolor*{palette tertiary}{fg=white,bg=brown2}
\newcommand{\modest}[1]{{\small\color{gray}#1}}

\newcommand{\ee}{\mathrm e}
\newcommand{\ui}{\mathrm i}
\newcommand{\real}{\operatorname{Re}}
\newcommand{\imag}{\operatorname{Im}}
\newcommand{\uv}[1]{\underline{#1}}
\newcommand{\bv}[1]{\mathbf{#1}}

\newcommand{\N}{\mathbb N}
\newcommand{\Z}{\mathbb Z}
\newcommand{\Q}{\mathbb Q}
\newcommand{\R}{\mathbb R}
\newcommand{\C}{\mathbb C}

\newcommand{\id}{\operatorname{id}}
\newcommand{\sgn}{\operatorname{sgn}}
\newcommand{\Abb}{\operatorname{Abb}}
\newcommand{\unit}[1]{\mathrm{#1}}
\newcommand{\chem}[1]{\mathrm{#1}}
\newcommand{\strong}[1]{\textsf{\textbf{#1}}}

\title{Lineare gewöhnliche Dgln.}
\date{}

\begin{document}
\maketitle

\begin{frame}
Gegeben ist das Anfangswertproblem
\[y'' = y'-y,\quad y(0)=0, y'(0)=1.\]
\end{frame}

\begin{frame}
Wie löst man das?
\end{frame}

\begin{frame}
In der Hochschule wird der Exponentialansatz $y=\ee^{\lambda x}$
unterrichtet. Einsetzen in die Dgl. führt zu
\[\lambda^2\ee^{\lambda x} = \lambda\ee^{\lambda x}-\ee^{\lambda x}.\]
Da $\ee^{\lambda x}\ne 0$ ist, ist die Division durch diesen Faktor
eine Äquivalenzumformung der Gleichung. Es ergibt sich
\[\lambda^2 = \lambda-1.\]
\end{frame}

\begin{frame}
Man bezeichnet
\[P(\lambda) = \lambda^2-\lambda+1\]
als \emph{charakteristisches Polynom}. Wir suchen demnach die
Nullstellen von $P(\lambda)$.
\end{frame}

\begin{frame}
Das sind
\[\lambda_1 = \frac{1}{2}+\frac{\sqrt{3}}{2}\ui\]
und
\[\lambda_2 = \frac{1}{2}-\frac{\sqrt{3}}{2}\ui.\]
\end{frame}

\begin{frame}
Die Lösung hat demnach die Gestalt
\[y(x) = K_1\ee^{\lambda_1 x}+K_2\ee^{\lambda_2 x}.\]
Einsetzen in das Anfangswertproblem bringt
\[0=y(0)=K_1+K_2,\qquad 1=y'(0)=K_1\lambda_1+K_2\lambda_2.\]
\end{frame}

\begin{frame}
Die Lösung lässt sich zu
\[y(x) = K_1\ee^{x/2}(\ee^{\sqrt{3}x\ui/2}+\ee^{-\sqrt{3}x\ui/2})\]
umformen. Unter Heranziehung der eulerschen Formel ergibt sich
\[y(x) = 2\ui K_1\ee^{x/2}\sin(\tfrac{\sqrt{3}}{2} x).\]
\end{frame}

\begin{frame}
Mit
\[K_1=\frac{1}{\lambda_1-\lambda_2} = \frac{1}{\sqrt{3}\ui}\]
ergibt sich schließlich
\[y(x) = \tfrac{2}{\sqrt{3}}\ee^{x/2}\sin(\tfrac{\sqrt{3}}{2} x).\]
Die Lösung lässt sich über eine Probe durch Einsetzen und numerisch
durch das Euler-Verfahren verifizieren.
\end{frame}

\begin{frame}
Das Ende unserer intellektuellen Betrachtungen? 
\end{frame}

\begin{frame}
Wir definieren zunächst $v=y'$ und Formen die Dgl. zweiter Ordnung
in ein System von zwei Dgln. erster Ordnung um:
\begin{align*}
v'&=v-y,\\
y'&=v.
\end{align*}
Das System lässt sich zu einem Vergleich von Tupeln zusammenfassen:
\[\begin{bmatrix}
v'\\
y'
\end{bmatrix}
= \begin{bmatrix}
v-y\\
v
\end{bmatrix}.\]
\end{frame}

\begin{frame}
Die rechte Seite lässt sich als Produkt aus einer Matrix und
einem Vektor interpretieren:
\[\begin{bmatrix}
v'\\
y'
\end{bmatrix}
= \begin{bmatrix}
1 & -1\\
1 & 0
\end{bmatrix}
\begin{bmatrix}
v\\
y
\end{bmatrix}.\]
Kurz:
\[\mathbf y' = A\mathbf y.\]
\end{frame}

\begin{frame}
Die Gleichung $y'=ay$ besitzt doch die
Lösung
\[y=y_0\ee^{(x-x_0)a}.\]
In Analogie wäre
\[\mathbf y=\mathbf y_0\ee^{(x-x_0)A}.\]
\end{frame}

\begin{frame}
Aber man kann doch nicht die Exponentialfunktion
auf eine Matrix $X=(x-x_0)A$ anwenden?
\end{frame}

\begin{frame}
Doch!
\end{frame}

\begin{frame}
Angenommen, $X$ besitzt die Diagonalisierung $X=TST^{-1}$, wobei
$S$ eine Diagonalmatrix mit den Eigenwerten von $X$ ist. Dann
gilt
\[X^2 = TS\underbrace{T^{-1}T}_{=E}ST^{-1} = TS^2T^{-1}.\]
Allgemein:
\[X^k = TS^k T^{-1}.\]
\end{frame}

\begin{frame}
Jetzt ergibt sich
\[\begin{split}\exp(X) &= \sum_{k=0}^\infty \frac{X^k}{k!}
= \sum_{k=0}^\infty T\frac{S^k}{k!}T^{-1}
= T\Bigg(\sum_{k=0}^\infty \frac{S^k}{k!}\Bigg)T^{-1}\\
&= T\exp(S)T^{-1}.
\end{split}\]
\end{frame}

\begin{frame}
Das Exponential einer Diagonalmatrix ist ganz einfach:
\[\exp(\begin{bmatrix}
\lambda_1 & 0\\
0 & \lambda_2
\end{bmatrix})
= \begin{bmatrix}
\ee^{\lambda_1} & 0\\
0 & \ee^{\lambda_2}
\end{bmatrix}.\]
\end{frame}

\begin{frame}
Wir müssen zunächst die Eigenwerte von $A$ bestimmen.
Der Ansatz dafür ist $P(\lambda)=\det(A-\lambda E)=0$. In unserem
Fall ist
\[P(\lambda) = \det(
\begin{bmatrix}
1 & -1\\
1 & 0
\end{bmatrix}-\lambda\begin{bmatrix}
1 & 0\\
0 & 1
\end{bmatrix})
= \det\begin{bmatrix}
1-\lambda & -1\\
1 & -\lambda
\end{bmatrix}.\]
Es ergibt sich
\[P(\lambda) = \lambda^2-\lambda+1.\]
\end{frame}

\begin{frame}
Huch. Das kennen wir ja schon.
\end{frame}

\begin{frame}
Warum?
\end{frame}

\begin{frame}
Machen wir den Ansatz $\mathbf y = \ee^{\lambda x}\mathbf v$.
Einsetzen in $\mathbf y'=A\mathbf y$ bringt
\[\lambda\ee^{\lambda x}\mathbf v = A\ee^{\lambda x}\mathbf v.\]
Nach Division durch $\ee^{\lambda x}$ ergibt sich
\[\lambda\mathbf v = A\mathbf v,\]
das ist das Eigenwertproblem.
\end{frame}

\end{document}


