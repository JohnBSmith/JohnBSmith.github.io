\documentclass[a4paper,11pt,fleqn,twocolumn,twoside,dvipdfmx]{scrartcl}
\usepackage[utf8]{inputenc}
\usepackage[T1]{fontenc}
\usepackage{libertine}
\usepackage[libertine]{newtxmath}
\usepackage{ngerman}
\usepackage{microtype}

\usepackage{amsmath}
\usepackage{amssymb}

\usepackage{color}
\definecolor{c1}{RGB}{0,40,60}
\usepackage[colorlinks=true,linkcolor=c1]{hyperref}
\usepackage{geometry}
\geometry{a4paper,left=26mm,right=10mm,top=24mm,bottom=28mm}
\setlength{\columnsep}{5mm}

\newcommand{\Z}{\mathbb Z}

\begin{document}
\thispagestyle{empty}

\begin{huge}
\noindent
{\sffamily\bfseries Algebra}
\par
\end{huge}

\tableofcontents
% \newpage

\section{Zyklische Gruppen}
\subsection{Restklassen}

Das Zifferblatt einer Uhr geht von eins bis zwölf. Dreht sich der
Uhrzeiger von zwölf aus um eine Stunde weiter, so landet er wieder bei
der Zahl eins. Die Stunde zwölf können wir auch als Stunde null
bezeichnen. Die Menge der Zahlen ist dann%
\[Z_{12} = \{0,1,2,3,4,5,6,7,8,9,10,11\}.\]
Dreht sich der Uhrzeiger bei der Zahl elf um eine Stunde weiter, so
landet er bei der Zahl null. D.\,h. $11+1=0$. Da das etwas seltsam
aussieht, schreiben wir $11+1\equiv 0$ und sagen, $11+1$ ist kongruent
zu null.

Wir machen nun die grundlegende Beobachtung, dass sich die Zahl nicht
ändert, wenn wir 12 hinzu addieren. Die Zahl wird sich auch nicht
ändern, wenn wir 12 subtrahieren. Wenn wir die Zahl als $h$ bezeichnen,
dann ist also%
\[h\equiv h+12\equiv h+24\equiv h-12\quad\mathrm{usw}.\]
Die Zahl 12 nennen wir den Modul.

Wie überprüft man, ob zwei Zahlen kongruent sind? Man kann doch
solange 12 subtrahieren oder addieren, bis man bei einer Zahl in der
Menge $Z_{12}$ landet. Z.\,B. sind 86 und 182 kongruent, denn es ist%
\begin{gather*}
86 = 2+7\cdot 12,\\
182 = 2+15\cdot 12.
\end{gather*}
Das ist natürlich eine etwas unpraktische Methode. Ich will nicht bis
zum Abend 12 subtrahieren, wenn die Zahlen sehr groß sind. Wenn man
aber die Differenz $182-86$ betrachtet, so erkennt man%
\begin{gather*}
182-86 = (2+15\cdot 12)-(2+7\cdot 12)\\
= (15-7)\cdot 12.
\end{gather*}
Die Differenz ist ein Vielfaches von 12 und daher durch 12 teilbar.
Das geht auch im Allgemeinen. Sind $a_1,a_2$ zwei kongruente Zahlen,
so ist%
\begin{gather*}
a_2-a_1 = (r+12q_2)-(r+12q_1)\\
= 12(q_2-q_1).
\end{gather*}
Wenn $a_1,a_2$ nicht kongruent sind, dann ist $a_1=r_1+12q_1$
und $a_2=r_2+12q_2$ mit $r_1\ne r_2$. Wir können ohne Beschränkung
der Allgmeinheit $r_1<r_2$ annehmen. Dann ist%
\begin{gather*}
a_2-a_1 = (r_2+12q_2)-(r_1+12q_1)\\
= (r_2-r_1)+12(q_2-q_1).
\end{gather*}
Die Differenz lässt bei Division den Rest $r_2-r_1$ und ist daher nicht
durch 12 teilbar. Zwei Zahlen sind also genau dann kongruent modulo 12,
wenn die Differenz durch 12 teilbar ist. Offenbar ist diese Aussage
auch für andere Moduln als 12 gültig. Wir wollen den Modul allgemein
mit dem Buchstaben $m$ bezeichnen. Man erhält den

\textbf{Satz.} Zwei Zahlen sind genau dann kongruent modulo $m$,
wenn ihre Differenz durch $m$ teilbar ist.

Wenn man sagt, $a$ ist kongruent zu $b$ modulo $m$, dann schreibt man
\[a\equiv b \pmod{m}\quad\text{oder kurz}\quad a\equiv b\quad (m).\]
%
Jede Zahl können wir ja in der Form $a=r+12q$ mit $0\le r\le 11$
darstellen. Dabei haben $r$ und $q$ eine besondere Bedeutung.
Es handelt sich bei $r$ um den Rest bei der Division durch 12
und $q$ ist der Quotient.

Es ist $-10\equiv 2\equiv 14\equiv 26$ usw. modulo 12. Alle diese
Zahlen sind kongruent und lassen sich in der Form $a=2+12k$
mit $k\in\mathbb Z$ schreiben. Die Menge $M$ dieser Zahlen wollen
wir daher als Restklasse zum Rest zwei bezeichnen und schreiben
kurz $M=2+12\mathbb Z$ in Anlehnung daran, dass sich ein Element
dieser Menge als $a=2+12k$ schreiben lässt.

Jeder Zahl $r$ aus $Z_{12}$ kann man genau eine Restklasse zuordnen.
Wir ordnen $r$ einfach mal $r+12\mathbb Z$ zu. Diese Zuordnung ist
bijektiv. Zwei Restklassen kann man nun addieren, indem man die
entsprechenden Zahlen aus $Z_{12}$ addiert und die Summe per Konruenz
wieder in $Z_{12}$ hineinbringt.

Beispiel:
\[8+9 = 17\equiv 5\quad (12)\]
Man kann es auch so schreiben:
\begin{gather*}
8+12\mathbb Z+9+12\mathbb Z = (8+9)+12\mathbb Z\\
= 17+12\mathbb Z = 5+12+12\mathbb Z = 5+12\mathbb Z
\end{gather*}
In dieser Notation kann $12\mathbb Z$ die Zahl 12 verschlucken
und ausspucken.

Die Summe liegt per Kongruenz immer wieder in $Z_{12}$. Außerdem ist
null das neutrale Element. Zu jeder Zahl $a$ in $Z_{12}$ gibt es eine
inverse Zahl $a'$, so dass $a+a'\equiv 0\,\,(12)$ ist.

Ich will zeigen, wie man eine Zahl invertiert. Die inverse Zahl
zu 2 ist z.\,B. $-2$. Per Kongruenz addiert man noch 12 und erhält 10.
Es ist also%
\[a' = -a+12 = 12-a.\]
Die Menge $Z_{12}$ zusammen mit der Addition besitzt damit die Struktur
einer Gruppe. Man bezeichnet diese Gruppe als Gruppe der Restklassen
zum Modul 12 oder kurz als Restklassengruppe modulo 12.

Die Restklassengruppe modulo $m$ wurde hier mit $Z_m$ bezeichnet.
Eine allgemein übliche und eindeutige Schreibweise für diese Gruppe ist
$\Z/m\Z$.

\subsection{Erzeuger}

Aus $Z_{12}$ kann man ein Element $a$ nehmen und es wiederholt zu sich
selbst addieren. Das inverse Element von $a$ soll man auch addieren
dürfen. Summen sind z.\,B. $a+a=2a$, $a+a+a=3a$ oder $a+a'=0a$.

Mit $\langle a\rangle$ bezeichnet man die Hülle von $a$. Das ist die
Menge aller Summen, welche per Kongruenz wieder auf $Z_{12}$
eingeschränkt werden. Es ist%
\[\langle a\rangle
:= \{g|\,\,g=ka,\,\,k\in\mathbb Z\}\quad\mathrm{modulo}\,12.\]
Ein Element $a$ mit der Eigenschaft $\langle a\rangle = Z_{12}$ heißt
Erzeuger von $Z_{12}$. Z.\,B. ist drei kein Erzeuger von $Z_{12}$,
denn es ist%
\[\langle 3\rangle = \{0,3,6,9\}.\]
Bei fünf handelt es sich um einen Erzeuger, denn es ist%
\[\langle 5\rangle = \{0,5,10,3,8,1,6,11,4,9,2,7\}.\]
Warum ist fünf ein Erzeuger und drei nicht? Wir überprüfen alle Zahlen
in $Z_{12}$, um einen Überblick darüber zu bekommen, welche der Zahlen
Erzeuger sind. Die Erzeuger sind $1,5,7,11$. Das sind Primzahlen. Aber
zwei und drei sind keine Erzeuger, obwohl sie auch Primzahlen sind.
Wir machen nun die grundlegende Beobachtung, dass zwei und drei in der
Primfaktorzerlegung von 12 vorkommen, die Zahlen $1,5,7,11$
jedoch nicht.

Wir sehen uns $Z_{35}$ an. Dort sind unter anderem 1,2,3,4,6,8,9
Erzeuger. Unsere Vermutung scheint sich zu bestätigen. Eine Zahl ist
Erzeuger der Gruppe $Z_m$, wenn sie teilerfremd zum Modul $m$ ist.

\textbf{Satz.} $\langle a\rangle = Z_m$ gdw. $\mathrm{ggT}(a,m)=1$.

Beweis der Implikation. Bei den Produkten $ka$ soll für $k$
gelten $1\le k\le m-1$. Alle diese Produkte müssen (für ein festes $a$)
ungleich einem Vielfachen von $m$ sein, damit es überhaupt möglich ist,
genügend unterschiedliche Elemente zu erzeugen. Wenn $a$ aber
Primfaktoren von $m$ enthält, so kann man immer ein $k$ finden,
so dass das Produkt $ka$ ein Vielfaches von $m$ ist. Daher muss $a$
teilerfremd zu $m$ sein. qed.

Beweis der Gegenimplikation. Alle Produkte müssen (für ein festes $a$)
inkongruent sein. Sei ohne Beschränkung der Allgemeinheit $k_1<k_2$.
Es darf $k_2a-k_1a = (k_2-k_1)a$ kein Vielfaches von $m$ sein.
Die Differenz $k=k_2-k_1$ ist wieder ein $k$ mit $1\le k\le m-1$.
Dass $ka$ kein Vielfaches von $m$ ist, ist aber durch die
Voraussetzung $\mathrm{ggT}(a,m)=1$ abgesichert. qed.

Mit $\mathrm{ord}(a)$ wird die Ordnung eines Elements von $Z_m$
bezeichnet. Das ist die kleinste natürliche Zahl $k$ für die
$ka\equiv 0$ ist.

Mit $\mathrm{ord}(G)=|G|$ wird die Ordnung der Gruppe $G$ bezeichnet.
Das ist die Anzahl der Elemente von $G$. Z.\,B. ist $|Z_m|=m$.

Wir beobachten nun, dass die Ordnung eines Erzeugers von $Z_m$ mit der
Ordnung von $Z_m$ übereinstimmt. Weiterhin beobachten wir%
\[\mathrm{ord}(a) = \mathrm{ord}(\langle a\rangle).\]
Tatsächlich bildet $\langle a\rangle$ wieder eine Gruppe,
auch wenn $a$ kein Erzeuger von $Z_m$ ist. Die Gruppe
$\langle a\rangle$ ist immer eine Untergruppe von $Z_m$.


\subsection{Untergruppen}

Eine Teilmenge von $Z_m$, die wieder eine Gruppe ist, wird Untergruppe
von $Z_m$ genannt. Z.\,B. ist $A=\{0,3,6,9\}$ eine Untergruppe
von $Z_{12}$. Jedoch ist $B=\{0,4\}$ keine Untergruppe von $Z_{12}$,
weil z.\,B. $4+4=8$ ist. Die Summe muss aber per Kongruenz modulo 12
wieder in der gleichen Menge liegen. Das Axiom der Abgeschlossenheit
trifft nicht zu, und so kann $B$ keine Gruppe sein.

Wenn $H$ eine Untergruppe von $G$ ist, dann schreibt man $H\le G$ in
Anlehnung an $H\subseteq G$. Wenn $H$ eine echte Untergruppe von $G$
ist, dann schreibt man $H<G$.

Ein Isomorphismus ist eine bijektive Funktion $\varphi$ zwischen zwei
Gruppen, die die Eigenschaft der Verträglichkeit hat. Die Eigenschaft
der Verträglichkeit ist%
\[\varphi(a+b) = \varphi(a)+\varphi(b).\]
%
Wenn man für zwei Gruppen $G,H$ einen Isomorphimus finden kann, dann
sind diese beiden Gruppen isomorph und man schreibt $G\simeq H$.
Zwei isomorphe Gruppen haben die gleiche Struktur. Zwei isomorphe
Gruppen sind in Wirklichkeit ein und dieselbe Gruppe, bloß in einer
anderen Gestalt. 

Z.\,B. ist $Z_4\simeq \{0,3,6,9\}$. Durch die folgende Wertetabelle
ist ein Isomorphismus gegeben.%
\begin{gather*}
\begin{array}{c|c|c|c|c}
g & 0 & 1 & 2 & 3\\
\hline
\varphi(g) & 0 & 3 & 6 & 9
\end{array}
\end{gather*}
%
Mit $\varphi^{-1}$ soll die Umkehrfunktion von $\varphi$ bezeichnet
werden. Um $a+b$ zu berechnen kann man doch auch%
\begin{gather*}
a+b = \varphi(\varphi^{-1}(a+b)) = \varphi(\varphi^{-1}(b)+\varphi^{-1}(a))\\
= \varphi(\varphi^{-1}(a)+\varphi^{-1}(b))
\end{gather*}
rechnen. Z.\,B. ist
\[3+6 = \varphi(1+2) = \varphi(3) = 9.\]
Man muss aber beachten, dass man modulo vier rechnet, wenn man sich
in $Z_4$ befindet. Nach Anwendung des Isomorphismus befindet man sich
in $\{0,3,6,9\}$, und dort muss man wieder modulo 12 rechnen.

Die Untergruppen von $Z_{12}$ sind alle wieder zyklisch, d.\,h. man
kann für jede Untergruppe einen Erzeuger finden. Tatsächlich gilt
diese Aussage auch allgemein für die Gruppen $Z_m$. Alle Untergruppen
einer zyklischen Gruppe sind auch wieder zyklisch.


\subsection{Produkte von Gruppen}

Wenn man zwei Gruppen $G, H$ hat, dann kann man das kartesische
Produkt $G\times H$ von den Mengen $G,H$ bilden, das aus allen
Tupeln $(g,h)$ mit $g\in G$ und $h\in H$ besteht. Mit
\[(g_1,h_1)+(g_2,h_2) := (g_1+g_2,h_1+h_2)\]
wird das Produkt wieder zu einer Gruppe, die direktes Produkt
(oder einfach Produkt) der Gruppen $G$ und $H$ genannt wird.

Natürlich ist es unproblematisch das direkte Produkt aus mehr als zwei
Gruppen zu bilden. Wenn das direkte Produkt $n$ Faktoren hat, so hat
man halt Tupel mit $n$ Komponenten.

Die Produkte von zyklischen Gruppen können in bestimmten Fällen selbst
wieder zyklisch sein. Es gilt der

\textbf{Satz.} Das Produkt von $Z_m\times Z_n$ ist genau dann zyklisch,
wenn $\mathrm{ggT}(m,n)=1$ ist. Wenn das Produkt zyklisch ist, dann ist
$Z_m\times Z_n\simeq Z_{mn}$.

Das ist ein sehr interessanter Satz. Nach diesem Satz ist z.\,B.%
\[Z_{12}\simeq Z_4\times Z_3 = Z[2^2]\times Z[3].\]
Was Gruppen jetzt von Zahlen unterscheidet ist z.\,B.%
\[(Z_2)^2 = Z_2\times Z_2 \not\simeq Z_4 = Z[2^2].\]
Über die Gruppe $Z_2^2 = (Z_2)^2$ wissen wir mit dem Satz über
Produkte von zyklischen Gruppen, dass sie nicht zyklisch sein kann.
Diese Gruppe ist isomorph zur kleinschen Vierergruppe. Drei Elemente
dieser Gruppe erzeugen Untergruppen, die jeweils isomorph zu $Z_2$
sind. Das vierte Element ist das neutrale Element.

Auf jeden Fall gilt aber immer
\[|G\times H| = |G|\,|H|.\]
Diese Formel hilft, die Ordnung einer Gruppe zu bestimmen,
wenn eine Zerlegung in einfachere Gruppen bekannt ist.
Z.\,B. ist die Ordnung der kleinschen Vierergruppe%
\[|Z_2\times Z_2| = |Z_2|\,|Z_2|
= 2\times 2 = 4.\]


\subsection{Gruppentafeln}

Die Gruppentafel dient zur Veranschaulichung kleiner endlicher Gruppen,
sie enthält alle Ergebnisse $a+b$. Dabei steht $a$ in der
Eingangsspalte und $b$ in der Kopfzeile. Das Ergebnis $a+b$ steht in
der Zelle, die durch Zeilennummer und Spaltennummer bestimmt wird.
Für die zyklische Gruppe $Z_4$ lautet die Gruppentafel z.\,B.%
\begin{gather*}
\begin{array}{c|cccc}
+ & 0 & 1 & 2 & 3\\
\hline
0 & 0 & 1 & 2 & 3\\
1 & 1 & 2 & 3 & 0\\
2 & 2 & 3 & 0 & 1\\
3 & 3 & 0 & 1 & 2
\end{array}
\end{gather*}
%
An der Gruppentafel kann man bestimmte Dinge ablesen.
Wenn eine Gruppe kommutativ ist, dann ist die Gruppentafel bezüglich
der Diagonalen symmetrisch. Das neutrale Element hat eine Kopie der
Kopfzeile als Zeile. Um ein inverses Element bezüglich eines Elements
zu suchen, geht man in der Zeile bis zum neutralen Element,
und dann nach oben.


\subsection{Prime Restklassengruppen}

Die Elemente von $Z_{12}$ kann man ja auch multiplizieren. Nicht jede
Zahl aus $Z_{12}$ hat auch eine inverse Zahl. Wir suchen eine Zahl
$a'$, so dass $aa'\equiv 1$ ist. Z.\,B. ist $5\times 5\equiv 1$. D.\,h.
fünf ist zu sich selbst invers. Zu vier kann man allerdings keine
inverse Zahl finden.

Eine Zahl $a$ aus $Z_m$ hat nur dann eine inverse Zahl,
wenn $\mathrm{ggT}(a,m)=1$ ist. Die Menge der multiplikativ
invertierbaren Zahlen aus $Z_m$ bildet mit der Multiplikation
eine Gruppe. Diese Gruppe wird Gruppe der primen Restklassen
oder kurz prime Restklassengruppe genannt. Für die prime
Restklassengruppe von $Z_m$ verwendet man die Notation $Z_m^\ast$.
Z.\,B. ist%
\[Z_{12}^\ast = \{1,5,7,11\}.\]
Diese Gruppe ist nicht zyklisch. Die Gruppe ist isomorph zur
kleinschen Vierergruppe. Die Gruppe%
\[Z_{10}^\ast = \{1,3,7,9\}\]
ist jedoch zyklisch und isomorph zu $Z_4$.
Zu beachten ist, dass man in $Z_{10}^\ast$ multipliziert und
modulo 10 rechnet. In $Z_4$ addiert man jedoch und rechnet modulo vier.

Bei der folgenden Funktion handelt es sich nicht um einen
Isomorphismus. Die Funktion wird nur zufällig mit dem gleichen
Buchstaben $\varphi$ bezeichnet, der auch für Isomorphismen gewählt
wurde. Mit $\varphi$ wird die eulersche Phi-Funktion bezeichnet.
Wenn $m$ und $n$ teilerfremd sind, dann ist
$\varphi(mn) = \varphi(m)\varphi(n)$. Wenn $p$ eine Primzahl ist,
dann ist $\varphi(p^k)=p^{k-1}(p-1)$. Hat man eine Primfaktorzerlegung
von einer Zahl $a$, so kann man also auch leicht $\varphi(a)$
ausrechnen.

Der Zusammenhang zwischen der Phi-Funktion und primen
Restklassengruppen ist die Gleichung%
\[\mathrm{ord}(Z_m^\ast) = \varphi(m).\]
Wir kennen ja schon $\mathrm{ord}(Z_{12}^\ast)=4$
und $\mathrm{ord}(Z_{10}^\ast)=4$. Das müsste man auch mit der
Phi-Funktion erhalten. Wir rechnen nach.%
\begin{gather*}
\varphi(12) = \varphi(2^2\times 3) = \varphi(2^2)\varphi(3)\\
= 2(2-1)(3-1) = 4\\
\varphi(10) = \varphi(2\times 5) = \varphi(2)\varphi(5)\\
= (2-1)(5-1) = 4
\end{gather*}
%
Die Erzeuger einer zyklischen primen Restklassengruppe werden
auch Primitivwurzeln genannt.


\section{Gruppen allgemein}

\subsection{Äquivalenzklassen}

Man denke sich eine Schule mit Schülern. Zwei Schüler $a,b$ sind
äquivalent, wenn sie in die gleiche Klasse gehen. Man schreibt
dann $a\sim b$. Eine Relation heißt Äquivalenzrelation, wenn sie
reflexiv, symmetrisch und transitiv ist. Das heißt es ist%
\begin{gather*}
a\sim a,\\
a\sim b \Rightarrow b\sim a,\\
a\sim b \wedge b\sim c \Rightarrow a\sim c.
\end{gather*}
Wenn $a$ ein Schüler ist, dann ist $[a]$ die Äquivalenzklasse von $a$.
Das ist die Menge aller Schüler, die zu $a$ äquivalent sind. Mit $M$
soll die Menge aller Schüler der Schule bezeichnet werden. Die Menge
der Schüler kann in disjunkte Äquivalenzklassen zerlegt werden. Damit
ist gemeint, dass kein Schüler der Schule gleichzeitig in zwei
unterschiedlichen Klassen sein kann. Ein Schüler einer bestimmten
Klasse wird Repräsentant dieser Klasse genannt.

Was haben wir jetzt davon? Angenommen man hat zwei Mengen $A,B$ mit
sehr vielen Elementen und will überprüfen, ob diese beiden Mengen
gleich sind. Dann müsste man jedes Element aus $A$ nehmen und
überprüfen, ob es auch in $B$ vorhanden ist und jedes Element aus
$B$ nehmen und überprüfen, ob es in $A$ vorhanden ist. Wenn diese
beiden Mengen $A,B$ aber Äquivalenzklassen sind, so braucht man sich
nur zwei Repräsentanten $a\in A$ und $b\in B$ auszusuchen und
überprüft, ob sie Äquivalent sind.

Man kann z.\,B. alle Geraden im $\mathbf R^2$ betrachten,
die durch den Koordinatenursprung gehen. Man will jetzt überprüfen,
ob zwei Geraden $g_1,g_2$ gleich sind. Dann müsste man zu jedem
Punkt $(x,y)$ auf $g_1$ schauen, ob dieser Punkt auch in der
Menge $g_2$ liegt. Für $g_2$ müsste man dasselbe tun.

Die Menge $g_1$ hat unendlich viele Punkte, so dass man sie nicht
einmal auf ein Blatt Papier schreiben kann. Die Lösung ist jetzt,
die Geraden als Äquivalenzklassen anzusehen, wenn wir uns den
Koordinatenursprung einmal wegdenken. Wir suchen uns nun
Repräsentanten $(x_1,y_1)\in g_1$ und $(x_2,y_2)\in g_2$ aus.
Welche das sind, das ist unerheblich. Man darf jedoch nicht den
Koordinatenursprung wählen. Die Äquivalenzrelation lautet nun%
\[(x_1,y_1)\sim (x_2,y_2) \Leftrightarrow
\exists r{:}\, (rx_1,ry_1)=(x_2,y_2).\]
%
Die Menge aller Äquivalenzklassen wird auch als Faktormenge oder
Quotientenmenge bezeichnet. Wenn $M$ eine Menge ist, dann ist mit
$M/{\sim}$ die Faktormenge bezüglich der Äquivalenzrelation $\sim$ gemeint.

Mit $\mathbf R^2/{\sim}$ ist also die Menge aller Ursprungsgeraden
gemeint. Diese Menge ist von der Ebene $\mathbf R^2$ zu unterscheiden.


\subsection{Präsentation einer Gruppe}

Für die Operation, bezüglich der die Menge $G$ eine Gruppe bildet,
wurde bisher immer ein Pluszeichen verwendet. Man kann auch ein
Malzeichen benutzen. Anstelle von $a+b$ schreibt man also $ab$.
Anstelle von $ka$ schreibt man dann $a^k$. Das neutrale Element
ist dann nicht null, sondern eins.

Es gibt eine erweiterte Notation für die Hülle $\langle g\rangle$.
Man schreibt hinter das Element $g$ noch einen senkrechten Strich.
Hinter den Strich kommen Bedingungen, die erfüllt sind.
Die von $g$ erzeugte zyklische Gruppe ist z.\,B.%
\[Z_m = \langle g|\,g^m=1\rangle.\]
Diese Notation heißt Präsentation einer Gruppe. Alternativ kann man
auch die additative Schreibweise verwenden. Dann schreibt man%
\[Z_m = \langle g|\,mg=0\rangle.\]
Im Allgemeinen hat man
\[\langle g_1,\ldots\,g_n|\,R_1,\ldots\,R_p\rangle.\]
Man bildet beliebige Worte aus den Symbolen $g_1,\ldots,g_n$,
wobei die inversen Elemente auch als Symbole zugelassen sind.
Dann wendet man die Relationen $R_1,\ldots,R_p$ an, um die Menge
der Worte zu reduzieren. Allgemeine Eigenschaften wie $g^{-1}g=1$
oder $(ab)^{-1}=b^{-1}a^{-1}$ dürfen natürlich auch verwendet werden.

Die kleinsche Vierergruppe hat z.\,B. die Präsentation%
\[Z_2^2 = \langle a,b|\,a^2=1, b^2=1, ab=ba\rangle.\]
Wir können z.\,B. das Wort $a^5 b^{-2} a^3 a^2 b^2$
bilden. Aber das kann auf folgende Weise reduziert werden:%
\begin{gather*}
a^5 b^{-1} a^3 a^2 b^2 = a^5 b^{-1} a^3
= a a^2 a^2 b^{-1} a a^2\\
= ab^{-1}a
= ab^{-1}bba\\
= aba = aab = b.
\end{gather*}
Da die Multiplikation durch $ab=ba$ kommutativ wird, kann man
die Potenzen auch einfach zusammenfassen und macht folgende
einfachere Rechnung:%
\begin{gather*}
a^5 b^{-1} a^3 a^2 b^2 = a^{10} b\\
= (a^2)^5 b = 1^5 b = b.
\end{gather*}
Egal welches Wort man nimmt, es lässt sich immer auf ein Element aus%
\[Z_2^2 = \{1,a,b,ab\}\]
reduzieren. Eine etwas interessantere Gruppe ist die Diedergruppe
$D_4$. Die Diedergruppe $D_4$ hat z.\,B. die Präsentation%
\[D_4 = \langle a,b|\,a^2=b^2=(ab)^4=1\rangle.\]
Die Gruppe ist nicht kommutativ und hat die Elemente%
\[D_4 = \{1,a,b,aba,bab,ab,ba,(ab)^2\}.\]
Jedes andere Wort lässt sich wieder auf eines in dieser Menge
reduzieren. Z.\,B. ist%
\begin{gather*}
(ab)^3 = (ab)^{-1}(ab)^4 = (ab)^{-1}\\
= b^{-1} a^{-1} = b^{-1}bb a^{-1}aa = ba.
\end{gather*}
Woher ich weiß, dass es keine weiteren Elemente gibt? Nun ja, $D_4$
hat vier Elemente, die jeweils das neutrale Element als Quadrat haben.
Das sind im Einzelnen%
\[a^2=b^2=(aba)^2=(bab)^2=1.\]
Hinzu kommt noch ein Zyklus der Länge vier. Der Zyklus ist%
\[1,ab,(ab)^2,(ab)^3.\]
Man braucht sich bloß den Zykelgraph von $D_4$ ansehen.

Bei der Diedergruppe $D_4$ handelt es sich um die Symmetriegruppe
des Quadrats. Es gibt vier Spiegelungen, drei Drehungen und die
identische Operation. Man sieht, dass $D_4$ die Symmetrie des
Quadrats kodiert.

Was ist mit dem gleichseitigen Dreieck? Die Diedergruppe $D_3$ kodiert
die Symmetrie des gleichseitigen Dreiecks. Es sind drei Spiegelungen,
zwei Drehungen und die identische Operation. Ob es zwei oder drei
Drehungen gibt, ist Ansichtssache, denn die Drehung um $360^\circ$ ist
auch immer die identische Operation.

Wir stellen jetzt natürlich die Frage, was Gruppen mit Symmetrien
zu tun haben. Bei $D_3$ und $D_4$ besteht die Gruppe anscheinend aus
den Symmetrieabbildungen. Verfolgt man diese Überlegung weiter,
so erkennt man, dass es sich bei der Multiplikation von Elementen um
die Komposition der entsprechenden Symmetrieabbildungen handelt.

Von der Präsentation von $D_4$ wissen wir jetzt auch, dass man mit
zwei Spiegelungen alle anderen Symmetrieabbildungen erzeugen kann.


\subsection{Permutationen}

Was ist eine Permutation? Man nehme das Tupel $(1,2,3,4)$.
Die Komponenten des Tupels kann man vertauschen. Z.\,B. zu $(3,2,4,1)$.
Das Vertauschen bezeichnet man als Permutation des Tupels.

Sei $M=\{1,\ldots,n\}$. Eine Permutation wird definiert als bijektive
Funktion $p$ mit $M$ als Definitionsbereich und Bildmenge.

Beim Tupel ordnet die Permutation einem Komponentenindex einen anderen
Komponentenindex zu. Schreiben wir das Tupel kurz als%
\[(a_k) = (a_1,\ldots,a_n).\]
Durch die Permutation $p$ erhält man das Tupel%
\[(a_{p(k)}) = (a_{p(1)},\ldots,a_{p(n)}).\]
Eine Permutation kann durch eine Wertetablle angegeben werden.
Z.\,B.%
\[p =
\begin{pmatrix}
1 & 2 & 3 & 4\\
p(1) & p(2) & p(3) & p(4)
\end{pmatrix}
=\begin{pmatrix}
1 & 2 & 3 & 4\\
3 & 2 & 4 & 1
\end{pmatrix}.\]
%
Eine Transposition ist eine spezielle Permutation, die zwei Komponenten
vertauscht. Jede Permutation lässt sich als Komposition von
Transpositionen zusammensetzen.

Die Menge aller Permutationen der Menge $M$ bildet zusammen mit der
Komposition von Permutationen eine Gruppe, die als symmetrische
Gruppe $S_n$ bezeichnet wird. Die Untergruppen der symmetrischen Gruppe
bezeichnet man als Permutationsgruppen. Von grundlegender Bedeutung
ist nun der

\textbf{Satz von Cayley.} Jede Gruppe ist zu einer Permutationsgruppe
isomorph.

\subsection{Nebenklassen}

Sei $H$ eine Untergruppe von $G$. Man kann nun ein $g\in G$ nehmen,
und es von links auf jedes $h\in H$ anwenden. Die Menge, welche dabei
entsteht, wird Linksnebenklasse von $H$ genannt und mit $gH$
bezeichnet. Es ist also%
\[gH:=\{gh|\,h\in H\}.\]
In additativer Schreibweise ist
\[g+H:=\{g+h|\, h\in H\}.\]
Nehmen wir die Gruppe $Z_{12}$ und die Untergruppe $H=\{0,3,6,9\}$.
Es ergibt sich z.\,B.%
\[2+H = 2+\{0,3,6,9\} = \{2,5,8,11\}.\]
Mit $G/H$ wird die Menge aller Linksnebenklassen von $H$ bezeichnet.
Die Anzahl der Elemente von $G/H$ wird mit $|G/H|$ bezeichnet.
Von grundlegender Bedeutung ist der

\textbf{Satz von Lagrange.} $|G|=|H|\,|G/H|$.

Dieser Satz sagt aus, dass der Quotient $|G|/|H|$ eine natürliche Zahl
ist, dass die Ordnung einer Untergruppe also ein Teiler der
Gruppenordnung ist. Eine Untergruppe kann also gar nicht jede
beliebige Ordnung haben.

Die Gruppe $Z_{12}$ kann z.\,B. nur Untergruppen mit der Ordnung
$1,2,3,4,6,12$ haben. Die Gruppe $Z_{11}$ kann nur $\{1\}$ und sich
selbst als Untergruppe haben. Mit dem Satz von Lagrange kann man
also Gruppen als Untergruppen sofort ausschließen.

Die Diedergruppe $D_4$ hat die Ordnung acht. Für Untergruppen sind
also nur die Ordnungen $1,2,4,8$ erlaubt. Tatsächlich sind die
nichttrivialen Untergruppen $Z_2$ und $Z_4$.


\subsection{Gruppenaktionen}

Sei $G$ eine Gruppe und $X$ eine Menge. Eine Funktion
$f{:}\,G\times X\rightarrow X$ heißt Gruppenaktion, wenn sie die
folgenden Eigenschaften erfüllt:%
\begin{gather*}
f(g_1g_2,x) = f(g_1,f(g_2,x)),\\
f(e,x) = x.
\end{gather*}
Dabei sollen $g_1,g_2\in G$ sein. Mit $e$ ist das neutrale Element
von $G$ gemeint. Anstelle von $f(g,x)$ schreibt man auch kürzer $gx$.
Wenn man additative Schreibweise verwendet, dann schreibt man wieder
$g_1+g_2$ anstelle von $g_1g_2$.

Nehmen wir z.\,B. ein Quadrat und nummerieren die Ecken mit $A,B,C,D$
durch. Die Gruppe $Z_4$ kann man nun auf der Menge $X=\{A,B,C,D\}$
agieren lassen. Wir definieren die Gruppenaktion durch%
\begin{gather*}
f(1,A)=B,\quad f(1,B)=C,\\
f(1,C)=D,\quad f(1,D)=A.
\end{gather*}
%
Z.\,B. ist
\begin{gather*}
f(3,B) = f(1+1+1,B) = f(1,f(1,f(1,B)))\\
= f(1,f(1,C)) = f(1,D) = A.
\end{gather*}
%
Die Bahn bzw. der Orbit von $x$ ist definiert als%
\[Gx := \{gx|\,g\in G\}.\]
Z.\,B. berechnet man den Orbit von $A$ zu%
\begin{gather*}
f(Z_4,A) = \{f(0,A),\,f(1,A),\,f(2,A),\,f(3,A)\}\\
= \{A,B,C,D\}.
\end{gather*}
Bei \{0,2\} handelt es sich um eine Untergruppe von $Z_4$.
Man erhält den Orbit%
\begin{gather*}
f(\{0,2\},A) = \{f(0,A),\,f(2,A)\} = \{A,C\}.
\end{gather*}
%
Was das soll? Wir sind hier beim zentralen Werkzeug angelangt,
was die Untersuchung von Symmetrien angeht. Nehmen wir z.\,B. das
Quadrat und legen die Mitte auf den Koordinatenursprung. Der Rand des
Quadrats ist eine Menge $X$ aus Punkten $(x,y)$. Die Gruppenaktion
definieren wir durch%
\[f(1,(x,y)) = (y,-x).\]
Eine Objekt ist nur dann symmetrisch, wenn die Anwendung von
Gruppenaktionen nicht aus der Menge $X$ herausführt.

Außerdem haben wir jetzt die Möglichkeit, ein kompliziertes Objekt
aus einem Keim heraus zu erzeugen. Das Quadrat ist z.\,B. der Orbit
einer der Kanten. Wir gehen einen Schritt weiter und bilden Figuren
nach dem Motto%
\[\text{symmetrische Figur} = \text{Orbit von Tintenklecks}.\]
Mit dem bilden des Orbits $GM$ von einer Figur $M$ wird diese Figur
vervollständigt, so dass sie symmetrisch bezüglich der Gruppenaktion
$gx$ wird. Eigentlich haben wir nur Orbits von Punkten definiert und
nicht von Mengen. Intuitiv kann man aber für jeden Punkt aus $M$ den
Orbit bilden. Der Orbit von $M$ soll dann einfach die Vereinigungsmenge
aller Orbits sein.

Die Punkte eines Orbits sind äquivalent. Eine symmetrische Figur kann
daher in disjunkte Orbits zerlegt werden.

Punkte $x$ mit $gx=x$ heißen Fixpunkte der Gruppenaktion.
Der Einzige Fixpunkt der Drehung mit $Z_4$ ist der Punkt $(0,0)$.

Eine Symmetriegruppe besteht ja aus den Symmetrieabbildungen,
wobei die Gruppenmultiplikation der Komposition von Abbildungen
entspricht. Die Symmetrieabbildungen bilden auf natürliche Weise
eine solche Gruppe, denn die Verkettung ist ja assoziativ und zu
jeder Abbildung gibt es auch eine Umkehrabbildung.

Man fragt sich nun, warum man die Gruppenaktion braucht, wenn doch
die Symmetrieabbildungen schon die Symmetriegruppe bilden. Der Vorteil
ist der folgende. Bei der Symmetriegruppe ist die Menge festgelegt,
auf der die Symmetrieabbildungen agieren. Bei der Gruppenaktion kann
man ein und dieselbe Gruppe jedoch für unterschiedliche Mengen
verwenden. Z.\,B. haben die Ecken eines Quadrates dieselbe Symmetrie,
wie der Rand des Quadrates und wie die gesamte Quadratfläche. Mit der
Gruppenaktion kann man diese drei Symmetriegruppen durch die gleiche
Gruppe beschreiben.

Der Zusammenhang zwischen Gruppenaktion und Symmetrieabbildung ist der
folgende. Wenn man bei der Gruppenaktion $f(g,x)$ das Gruppenelement
$g$ konstant hält, so erhält man die entsprechende Symmetrieabbildung
$f(x):=f(g,x)$. Es kann aber sein, dass unterschiedliche
Gruppenelemente die gleiche Symmetrieabbildung produzieren. Nehmen wir
die Menge $X=\{a,b\}$ und $G=Z_4$. Die Gruppenaktion legen wir mit
$f(1,a)=b$ und $f(1,b)=a$ fest. Damit ergibt sich $f(0,x)=f(2,x)$
und $f(1,x)=f(3,x)$. Die auf $X$ agierende Gruppe $Z_4$ hat vier
Elemente, die Symmetriegruppe jedoch nur zwei.

Man definiert daher den Begriff der treuen Gruppenaktion.
Eine Gruppenaktion heißt treu, wenn zu zwei unterschiedlichen
Gruppenelementen $g,h$ auch immer die Gruppenaktionen unterschiedlich
sind. Dafür muss jedes mal $f(g,x)\ne f(h,x)$ für mindestens ein $x$
sein.


\subsection{Kontinuierliche Gruppen}

Die Rotationssymmetrie eines regelmäßigen Polygons mit $n$ Ecken
wird durch die Gruppe $Z_n$ beschrieben. Ein Kreis hat jedoch eine
Rotationssymmetrie, bei der man um einen beliebig kleinen Winkel
drehen kann. Man benötigt eine kontinuierliche Gruppe. Diese Gruppe
heißt spezielle orthogonale Gruppe der Drehungen in der Ebene und
wird  mit $\mathrm{SO}(2)$ abgekürzt. Eine Drehung kann durch die
Drehmatrix%
\[D(\varphi) = \begin{bmatrix}
\cos\varphi & -\sin\varphi\\
\sin\varphi & \cos\varphi
\end{bmatrix}\]
beschrieben werden. Es gilt dann
$D(\varphi_1)D(\varphi_2)=D(\varphi_1+\varphi_2)$.
Die Summe der Winkel berechnet man modulo $2\pi$. Man hat
also gewissermaßen immernoch eine Restklassengruppe, der Rest liegt
jedoch im kontinuierlichen Intervall $0\le x<2\pi$.

Ebenso gibt es eine Gruppe der Translationen $\mathrm T(2)$, welche
aus allen Translationen der Ebene zusammengesetzt ist. Eine
Translation $t$ kann man durch $t(x)=x+v$ beschreiben. Dabei wird zum
Punkt $x$ der Verschiebungsvektor $v$ addiert.

Diese Gruppen haben eine besondere Eigenschaft. Es handelt sich um
Isometriegruppen. Eine Isometrieabbildung erhält Abstände. Das heißt%
\[d(x_1,x_2)=d(f(x_1),f(x_2)),\]
wenn $d$ die Abstandsfunktion und $f$ die Isometrieabbildung ist.
Eine Gruppe, die aus Isometrieabbildungen besteht, heißt
Isometriegruppe. Die Gruppen $\mathrm{SO}(2), \mathrm T(2)$ und alle
ihre diskreten Untergruppen sind Isometriegruppen.

Alle Isometrieabbildungen des euklidischen Raumes bilden eine Gruppe.
Diese Gruppe wird euklidische Gruppe $\mathrm E(2)$ genannt.
Die Gruppen $\mathrm{SO}(2)$ und $\mathrm T(2)$ sind Untergruppen
von $\mathrm E(2)$.

Isometrieabbildungen sind spezielle Symmetrieabbildungen.
Es gibt auch Symmetrieabbildungen, die keine Isometrieabbildungen sind,
z.\,B. die Skalierung $f(k,x) = 2^k x$ mit $k\in\mathbb Z$.
Eine Bezüglich dieser Skalierung symmetrische Figur ist z.\,B. die
Menge der Kreise mit Radius $r=2^k$ und Mittelpunkt im
Koordinatenursprung.

\end{document}


