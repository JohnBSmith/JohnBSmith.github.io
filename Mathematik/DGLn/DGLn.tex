\documentclass[a4paper,10pt,fleqn,twocolumn,twoside,openany]{article}
\usepackage[utf8]{inputenc}
\usepackage[T1]{fontenc}
\usepackage{lmodern}
\usepackage{ngerman}
\usepackage{amsmath}
\usepackage{amssymb}
\usepackage{amsthm}
\usepackage{textcomp}

\usepackage{color}
\definecolor{c1}{RGB}{00,40,80}
\definecolor{c2}{RGB}{20,60,100}
\definecolor{c3}{RGB}{80,120,180}
\usepackage[colorlinks=true,linkcolor=c1]{hyperref}
\usepackage{geometry}
\geometry{a4paper,left=25mm,right=10mm,top=20mm,bottom=28mm}
\setlength{\columnsep}{4mm}
\usepackage{lipsum}
\usepackage{multicol}

\setcounter{secnumdepth}{4}
\setcounter{tocdepth}{3}
\usepackage{tocloft}
\setlength{\cftsecindent}{0pt}
\setlength{\cftsubsecindent}{15pt}
\setlength{\cftsubsubsecindent}{23pt}
\renewcommand{\cftsecfont}{\normalfont\sffamily\bfseries}
\renewcommand{\cftsubsecfont}{\normalfont\sffamily}
\renewcommand{\cftsubsubsecfont}{\normalfont\sffamily}
\renewcommand\cftsecpagefont{\normalfont\sffamily\bfseries}
\renewcommand\cftsubsecpagefont{\normalfont\sffamily}
\renewcommand\cftsubsubsecpagefont{\normalfont\sffamily}

\usepackage{titlesec}
\titleformat{\section}[block]
  {\normalfont\sffamily\LARGE\bfseries}{\thesection}{1em}{\LARGE}
\titleformat{\subsection}[block]
  {\normalfont\sffamily\Large\bfseries}{\thesubsection}{1em}{\Large}
\titleformat{\subsubsection}[block]
  {\normalfont\sffamily\large\bfseries}{\thesubsubsection}{1em}{\large}

\titlespacing*{\section}{0pt}{10pt}{4pt}
\titlespacing*{\subsection}{0pt}{2pt}{2pt}
\titlespacing*{\subsubsection}{0pt}{2pt}{2pt}

\usepackage[justification=RaggedRight,singlelinecheck=off]{caption}

\numberwithin{equation}{section}

\renewcommand{\baselinestretch}{1.0}

\newcommand{\strong}[1]{{\sffamily\bfseries #1}}

% \ui: imaginäre Einheit
% \ue: Einheitsvektor
% \ue: eulersche Zahl
% \uv{x}: unterstrichener Vektor

\newcommand{\ui}{\mathrm i}
\newcommand{\ee}{\mathrm e}
\newcommand{\uv}[1]{\underline{#1}}

\newcommand{\N}{\mathbb N}
\newcommand{\Z}{\mathbb Z}
\newcommand{\R}{\mathbb R}
\newcommand{\C}{\mathbb C}

\DeclareMathOperator*{\sgn}{sgn}
\DeclareMathOperator*{\Real}{Re}
\DeclareMathOperator*{\Imag}{Im}
\DeclareMathOperator*{\rg}{rg}
\DeclareMathOperator*{\diag}{diag}
\DeclareMathOperator*{\Eig}{Eig}

\theoremstyle{definition}
\newtheorem{Definition}{Definition}

\begin{document}
\setlength{\abovedisplayskip}{6pt}
\setlength{\belowdisplayskip}{6pt}
\setlength{\abovedisplayshortskip}{6pt}
\setlength{\belowdisplayshortskip}{6pt}

\begin{titlepage}
\centering
\phantom{x}

\vspace{20em}
{\noindent\Huge\sffamily\textbf{Differentialgleichungen}}

\vspace{2em}
{\Large Dezember 2016}\\
\end{titlepage}

\thispagestyle{empty}

\noindent
Dieses Heft ist unter der Lizenz\\
Creative Commons CC0 veröffentlicht.

\renewcommand{\contentsname}{
\normalfont\sffamily\bfseries\LARGE
Inhaltsverzeichnis}
\tableofcontents

\section{Vorbereitungen}
\subsection{Begriffe}
Was ist eine Differentialgleichung? Um diesen Begriff besser in den
allgemeinen mathematischen Rahmen einordnen zu können, wollen wir eine
Differentialgleichung (DGL) als Spezialfall einer Funktionalgleichung
beschreiben. Eine Funktionalgleichung ist nun informell eine Gleichung,
deren Lösungsmenge nicht aus Zahlen besteht, sondern aus Funktionen.

\begin{Definition}
Sei $f\colon D\to Z$. Eine \emph{Funktionalgleichung} ist
eine Bestimmungsgleichung der Form
\begin{equation}\label{eq:Funktionalgleichung}
\forall x{\in}D\colon\;H(x,f)=0
\end{equation}
mit $g\colon D\times\operatorname{Abb}(D,Z)\to G^m$ wobei $(G,+)$
eine abelsche Gruppe ist. Gefunden werden soll die
Lösungsmenge%
\begin{equation}
L = \{f\mid\forall x\colon H(x,f)=0\}.\;\Box
\end{equation}
\end{Definition}
\noindent
Man beachte, dass in dieser Definition auch Systeme
von Funktionalgleichungen
mit eingeschlossen sind. Außerdem kann z.\,B. $D=\R^n$ sein, womit
sich Funktionen in mehreren Variablen beschreiben lassen.

Gleichung \eqref{eq:Funktionalgleichung} ist sehr abstrakt und
beschreibt eine unheimliche Vielfalt an Funktionen.
Wir wollen uns daher zunächst auf reelle Funktionen beschränken.

An dieser Stelle sei noch angemerkt, dass
\eqref{eq:Funktionalgleichung} auch eine Verallgemeinerung der
impliziten Beschreibung einer Funktion ist. Man spezifiziert dazu
\begin{equation}
H(x,f) := F(x,f(x)),\quad F\colon D\times Z\to G^m.
\end{equation}
Auch Rekursionsgleichungen sind Spezialfälle von
Funktionalgleichungen. Wähle dazu $f\colon\N\to Z$, d.\,h. $f$ soll
eine Folge sein. Eine Rekursionsgleichung erster Ordnung erhält man
nun durch die Festlegung
\begin{equation}
H(x,f) := f(x)-g(x,f(x-1)).
\end{equation}
Es ist nun so, dass irgendwo in der Berechnung von $H(x,f)$
Differentialoperatoren enthalten sein können. Wir wollen nun,
dass ausschließlich solche Operatoren verwendet werden
und schränken die Struktur von \eqref{eq:Funktionalgleichung}
durch die Festlegung
\begin{equation}
H(x,f) := g(x,(Df)(x),(D^2 f)(x),\ldots,(D^n f)(x))
\end{equation}
stark ein, wobei $D$ der gewöhnliche Ableitungsoperator sein soll,
der einer Funktion ihre erste Ableitung zuordnet. Dann ist $D^2$ die
zweite Ableitung usw. Eine solche Gleichung wird dann als
\emph{Differentialgleichung} bezeichnet. Wichtig ist hierbei,
dass nur die Ableitungen an der Stelle $x$ betrachtet werden.
Z.\,B. sind Funktionalgleichungen der Form
\begin{equation}
H(x,f) := g(x,(Df)(x),(Df)(x-1))
\end{equation}
darin nicht enthalten.
\begin{Definition}
Eine \emph{implizite Differentialgleichung} der Ordnung $n$
ist eine Gleichung der Form
\begin{equation}
\forall x\colon g(x,y_0,y_1,y_2,\ldots,y_n)=0.
\end{equation}
mit $y_k=(D^k f)(x)$. Dabei sollen zunächst nur solche Funktionen $f$
in Frage kommen, die auf einem offenen Definitionsbereich
definiert und dort differenzierbar sind. $\Box$
\end{Definition}
\noindent
Manchmal lässt sich eine implizite Differentialgleichung
in eine explizite Form bringen.
\begin{Definition}
Eine \emph{explizite Differentialgleichung} der Ordnung $n$
ist eine Gleichung der Form
\begin{equation}
\forall x\colon y_n = g(x,y_0,y_1,\ldots,y_{n-1})
\end{equation}
mit $y_k:=(D^k f)(x)$. $\Box$
\end{Definition}


\end{document}


