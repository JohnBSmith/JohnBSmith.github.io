\documentclass[a4paper,11pt,fleqn]{article}
\usepackage[utf8]{inputenc}
\usepackage{ngerman}
\usepackage{amsmath}
\usepackage{amssymb}
\usepackage{color}
\definecolor{c1}{RGB}{00,40,80}
\usepackage[colorlinks=true,linkcolor=c1]{hyperref}
%\usepackage{geometry}
%\geometry{a4paper,left=25mm,right=15mm,top=20mm,bottom=28mm}
%\setlength{\columnsep}{6mm}
\begin{document}

\noindent
{\huge\textbf{Schneller rechnen}}

\tableofcontents
\vspace{4mm}

\noindent
\textbf{Zusammenfassung}. Rechnen geschah anfangs mühevoll und
langsam mittels Logarithmentafel und Rechenschieber. Dann gab
es elektronische Taschenrechner. Der nächste Schritt waren
Taschrechner, die auch mit komplexen Zahlen umgehen und
Funktionsgraphen zeichnen konnten.
Bei vielen mehrdimensionalen Aufgaben ergibt sich nun die Situation,
dass man sich etwa um das Lösen linearer Gleichungssysteme, die
Berechnung von Determinanten oder das Finden von Eigenwerten
bemühen muss. Auch diese Rechnungen sollen nun durch die
Verwendung eines Matrizenrechners in den Hintergrund geschoben
werden. Stattdessen können wir uns auf algebraische und analytische
Zusammenhänge konzentrieren.

\section{Lineare Algebra}
\subsection{Lineare Gleichungssysteme}
Ein lineares Gleichungssystem ist von der Form
\begin{align*}
a_{11}x_1+\ldots+a_{1n}x_n &= b_1,\\
\ldots\\
a_{m1}x_1+\ldots+a_{mn}x_n &= b_m.
\end{align*}
Dieses können wir in der Form
\begin{gather*}
\begin{bmatrix}
a_{11} &\ldots & a_{1n}\\
\ldots &\ldots & \ldots\\
a_{m1} &\ldots & a_{mn}
\end{bmatrix}
\begin{bmatrix}
x_1\\ \ldots\\ x_n
\end{bmatrix}
= \begin{bmatrix}
b_1\\ \ldots\\ b_n
\end{bmatrix}
\end{gather*}
aufschreiben. Also kurz \(Ax=b\). Nehmen wir nun erstmal
an, \(A\) sei eine quadratische Matrix. Nun ist \(A\) genau
dann invertierbar, wenn die Determinante \(\det(A)\ne 0\) ist. In diesem Fall
können wir auf beiden Seiten von links mit \(A^{-1}\) multiplizieren.
Somit ergibt sich \(A^{-1}Ax=A^{-1}b\). Wegen \(A^{-1}A=E_n\) und
\(E_n x=x\) ergibt sich
\[x=A^{-1}b.\]
Das ist ganz analog zu \(ax=b\). Wenn hier \(a\ne 0\) ist, so kann
man durch \(a\) dividieren und erhält \(x=a^{-1}b\).

Ein Beispiel hierzu ist
\begin{align*}
x_1+4x_2 &= 2,\\
3x_2+2x_2 &=1.
\end{align*}
Die Matrixform ist
\begin{gather*}
\begin{bmatrix}
1 & 4\\
3 & 2
\end{bmatrix}
\begin{bmatrix}
x_1\\ x_2
\end{bmatrix}
= \begin{bmatrix}
2\\ 1
\end{bmatrix}.
\end{gather*}
In Octave macht man nun folgende Eingaben.
\begin{verbatim}
> A=[1,4; 3,2];
> b=[2; 1];
> det(A)
-10
> A^(-1)*b
[0.0; 0.5]
\end{verbatim}

\subsection{Nilpotente Matrizen}

Betrachten wir die Matrix
\[N=\begin{bmatrix}
-15 & -27 & -18\\
 -7 & -17 & -10\\
 26 &  49 &  32
\end{bmatrix}.\]
Diese Matrix ist nilpotent. Jede nilpotente Matrix ist singulär
(nicht invertierbar). Daher ist auch \(\det(N)=0\). Wir wollen
den Nilpotenzgrad von \(N\) bestimmen.
\begin{verbatim}
> N=[-15,-27,-18; -7,-17,-10; 26,49,32];
> det(N)
1e-13
> all(all(N^2 < 1e-10))
0
> all(all(N^3 < 1e-10))
1
\end{verbatim}
Also hat \(N\) den Nilpotenzgrad drei.

Die Matrix \(N\) kann nun als lineare Abbildung \(f\) von
\(V=\mathbb R^3\) nach \(W=\mathbb R^3\) betrachtet werden. Wir
wollen jetzt den Rang von \(N\) bestimmen.
\begin{verbatim}
> rank(N,1e-10)
2
\end{verbatim}
Damit hat \(N\) den Defekt 1. Der Bildraum ist also eine Ebene in
\(W\), die durch den Koordinatenurspung von \(W\) verläuft. Der
Nullraum (Kern) ist eine Gerade in \(V\), die durch den
Koordinatenursprung von \(V\) verläuft. Angenommen wir haben nun das
Gleichungssystem
\[Nx=b.\]
Wegen \(\det(N)=0\) kann es für kein \(b\) eine eindeutige Lösung
geben. Wenn man wahllos verschiedene \(x\) einsetzt, liegen die
\(Nx\) alle in einer Ebene. Falls es Lösungen geben soll, muss
\(b\) auch in dieser Ebene liegen. Sucht man sich ein zufälligen
Vektor \(b\) aus, so wird dieser mit hoher Wahrscheinlichkeit nicht
in der Ebene \(\mathrm{Bild}(f)=\{Nx|x\in V\}\) liegen. Setzt man
\(b=0\), so ergibt sich \(\mathrm{Kern}(f)\) als Lösung. 

Rechnerisch gesehen braucht man ein \(b\) sodass
\[\mathrm{Rang}((N|b))=\mathrm{Rang}(N)=2\]
ist. Mit \((N|b)\) ist die erweiterte Koeffizientenmatrix von \(N\)
gemeint. Setzt man \(b=0\), so ist das natürlich der Fall.

Wir nehmen nun \(x=[1,0,0]^T\) und erhalten
\[b=Nx = [-15,-7,26]^T.\]
\begin{verbatim}
> b = N*[1;0;0];
> rank([N,b],1e-10)
2
\end{verbatim}
Wie gibt man nun die Lösungen von \(Nx=b\) an? Nun, da die
Lösungsmenge ein affiner Raum ist, genügt es einen Einhängepunkt
\(P\) und eine Basis \(B=(e_k)\) anzugeben. Dann ist Lösungsmenge von
der Form
\[L = \{P+t_1e_1+\ldots t_d e_d\;|\; t\in\mathbb R^d\}.\]
Nun ist aber auch
\[L = P+\mathrm{Kern}(f).\]
Daher müssen wir einen Punkt \(P\) und eine Basis von
\(\mathrm{Kern}(f)\) angeben.
\begin{verbatim}
# null berechnet eine ONB für den Nullraum (Kern)
> B = null(N)
[-0.45617; -0.30411; 0.83631]
# Die ONB besteht in diesem Fall nur aus einem Vektor,
# weil ja dim(Kern(f))=Defekt(f)=1 ist.
# Dieser Basisvektor ist normiert.
> norm(B)
1
\end{verbatim}
Jetzt braucht man noch einen Punkt \(P\). Wir könnten jetzt einfach
\(x=[1,0,0]^T\) benutzen. Aber das wäre ja gemogelt. Wir müssen
davon ausgehen, dass wir noch keine spezielle Lösung kennen.
Dafür benutzt man \texttt{linsolve}.
\begin{verbatim}
> x = linsolve(N,b)
[0.79191; -0.13873; 0.38150]
> all(N*x-b < 1e-10)
1
\end{verbatim}
Wenn man ein eindeutig lösbares System hat, dann bestimmt
natürlich \texttt{linsolve} genau diese Lösung. Bei \(Nx=b\)
bekommt man nur eine Lösung von vielen. Die Funktion \(g(t)=x+tB\)
für die Lösungsgerade erhält man wie folgt.
\begin{verbatim}
> g = @(t) x+t*B;
\end{verbatim}
Nun wollen wir noch das \(t\) für \(g(t)=[1,0,0]^T\) bestimmen.
Daher muss die Gleichung \(x_1+tB_1=1\) gelöst werden.
\begin{verbatim}
> t=(1-x(1))/B(1)
-0.45617
\end{verbatim}

\subsection{Matrixexponential}
Bei Systemen von linearen Differentialgleichungen mit konstanten
Koeffizienten gibt es ein Lösungsverfahren wo man das
Matrixexponential
\[\exp(A)=\sum_{k=0}^\infty\frac{A^k}{k!}\]
berechnen muss.

Betrachten wir die Matrix
\[A=\begin{bmatrix}
2 & 4\\
0 & 1
\end{bmatrix}\]
Das Matrixexponential lässt sich nun mit \texttt{expm} berechnen.
\begin{verbatim}
> A = [2,4; 0,1];
> A^0+A^1+A^2/2+A^3/6+A^4/24+A^5/120
[7.26667, 18.20000;
 0.00000,  2.71667]
> expm(A)
[7.38906, 18.68310;
 0.00000,  2.71828]
\end{verbatim}
Dieses Exponential können wir auch ohne Partialsumme selbst berechnen
wenn eine Eigenzerlegung \(A=TDT^{-1}\) möglich ist. Es ist nämlich z.B.
\[A^3 = TDT^{-1}TDT^{-1}TDT^{-1} = TD^3T^{-1}\]
weil Matrizenmultiplikation assoziativ ist und sich die \(T^{-1}T\)
zur Einheitsmatrix \(E_2\) aufheben. Allgemein gilt \(A^m=TD^mT^{-1}\).
Für die Diagonalmatrix \(D\) ist außerdem
\[D^m = \begin{bmatrix}
d_1 & 0\\
0 & d_2
\end{bmatrix}^m
= \begin{bmatrix}
d_1^m & 0\\
0 & d_2^m
\end{bmatrix}.\]
wie man durch Multiplizieren leicht überprüfen kann.
Ist \(f\) allgemein eine Funktion, welche auf dem Spektrum von
\(A\) definiert ist, so gilt \(f(A)=Tf(D)T^{-1}\) und
\[f(D) = f(\mathrm{diag}(d_1,\ldots,d_n))
= \mathrm{diag}(f(d_1),\ldots,f(d_n)).\]
Daher ist
\[\exp(A) = T\exp(D)T^{-1}.\]
Die Diagonalmatrix \(D\) enthält die Eigenwerte von \(A\).
Das lässt sich mit Octave ganz leicht bewerkstelligen.
\begin{verbatim}
> [T,D] = eig(A)
T = [1.00000, -0.97014;
     0.00000,  0.24254]
D = [2, 0;
     0, 1]
> T*D*T^(-1)
[2, 4;
 0, 1]
> T*[e^2, 0; 0, e^1]*T^(-1)
[7.38906, 18.68310;
 0.00000,  2.71828]
\end{verbatim}
In diesem Fall lässt sich \(\exp(A)\) sogar exakt angeben.
Wegen
\[\begin{bmatrix}
a & c\\
0 & b
\end{bmatrix}^m
=\begin{bmatrix}
a^m & \frac{a^m-b^m}{a-b}c\\
0 & b^m
\end{bmatrix}
\]
gilt nämlich
\[\exp\bigg(\begin{bmatrix}
a & c\\
0 & b
\end{bmatrix}\bigg)
=\begin{bmatrix}
e^a & \frac{e^a-e^b}{a-b}c\\
0 & e^b
\end{bmatrix}.
\]
Mit dieser Formel erhält man
\[\exp\bigg(\begin{bmatrix}
2 & 4\\
0 & 1
\end{bmatrix}\bigg)
=\begin{bmatrix}
e^2 & 4e^2-4e\\
0 & e
\end{bmatrix}.
\]

\subsection{Simultane LGS}

Angenommen man hat für die selbe Matrix \(A\) mehrere lineare
Gleichungssysteme \(Ax=b_k\). Dann lassen sich
die Vektoren \(b_k\) zu einer Matrix \(B\) zusammenfassen.
Für drei Vektoren in der Ebene setzt man z.B.
\[B=(b_1,b_2,b_3)
= (\begin{bmatrix}b_{11}\\ b_{21}\end{bmatrix},
\begin{bmatrix}b_{12}\\ b_{22}\end{bmatrix},
\begin{bmatrix}b_{13}\\ b_{23}\end{bmatrix})
= \begin{bmatrix}
b_{11} & b_{12} & b_{13}\\
b_{21} & b_{22} & b_{23}
\end{bmatrix}.\]
Jetzt lassen sich die drei Gleichungen \(Ax=b_1\), \(Ax=b_2\) und
\(Ax=b_3\) zu einer Gleichung \(AX=B\) zusammenfassen.
Gehen wir nun davon aus, dass \(A\) nicht singulär ist. Dann können
alle drei LGS eindeutig gelöst werden. Man erhält die Lösung
\[X = A^{-1}B.\]
Beispielsystem:
\begin{gather*}
\begin{bmatrix}
1 & 4\\
3 & 2
\end{bmatrix}
\begin{bmatrix}
x_1 & x_1' & x_1''\\
x_2 & x_2' & x_2''
\end{bmatrix}
= \begin{bmatrix}
2 & 3 & 8\\
1 & 4 & 5
\end{bmatrix}.
\end{gather*}
Lösung:
\begin{verbatim}
> A = [1,4; 3,2];
> B = [2,3,8; 1,4,5];
> X = A^(-1)*B
[0.0, 1.0, 0.4;
 0.5, 0.5, 1.9]
\end{verbatim}
Die Lösungsvektoren lassen sich auch aus \(X\) herausnehmen.
Den ersten Vektor bekommt man z.B. mit \texttt{X(1:2,1)}.


\subsection{Reelle Matrixpotenzen}

Man betrachte eine quadratische Matrix \(A\). Dann ist
\(f(x)=Ax\) ein linearer Endomorphismus vom Koordinatenraum
\(\mathbb R^n\) in den Koordinatenraum \(\mathbb R^n\). Die Matrix
\(A^2\) lässt sich als zweimalige Anwendung von \(f\) interpretieren.
Die Matrixpotenz \(A^m\) entspricht somit einer \(m\)-fachen Iteration
von \(f\). Eine reelle Matrixpotenz lässt sich als fraktionale
Iteration interpretieren.

Da das Thema im Allgemeinen recht kompliziert ist, beschränken wir
uns auf Rotationsmatrizen für den \(\mathbb R^2\). Diese sind von der
Form
\[R(\varphi) = \begin{bmatrix}
\cos\varphi & -\sin\varphi\\
\sin\varphi & \cos\varphi
\end{bmatrix}.\]
Eine solche Rotationsmatrix dreht einen Vektor \(x\) um den
Winkel \(\varphi\) gegen den Uhrzeigersinn. Hier gilt für
die Multiplikation von Matrizen
\[R(\varphi_1)R(\varphi_2) = R(\varphi_1+\varphi_2),\]
was man mit den Additionstheoremen bestätigen kann.
Für eine \(m\)-fache Applikation erhält man daher
\[R(\varphi)^m x = R(m\varphi)x.\]
Für eine reelle Zahl \(r\) ist dann aber wohl
\[R(\varphi)^r = R(r\varphi).\]
Will man allgemein \(A^r\) für eine quadratische Matrix \(A\)
berechnen, so sollte \(A\) zunächst diagonalisierbar sein und
ausschließlich positive Eigenwerte haben. Ander ausgedrückt:
Die Potenzfunktionen sollen auf dem Spektrum von \(A\) definiert
sein.

Mit Octave lassen sich reelle Matrixpotenzen berechnen. Die hier
dargestellten Formeln lassen sich numerisch verifizieren.
Außerdem lässt sich mit \(\exp(A)=e^A\) das Matrixexponential
und mit \(\ln(A)\) der Matrixlogarithmus berechnen.
Ich wir erklären, was es in diesem Fall damit auf sich hat.

Für den Matrixlogarithmus einer Rotationsmatrix gilt zunächst einmal
\[\ln(R(\varphi))
= \begin{bmatrix}
0 & -\varphi\\
\varphi & 0
\end{bmatrix}
= \varphi\begin{bmatrix}
0 & -1\\
1 & 0
\end{bmatrix} = \varphi R(\pi/2).
\]
Mit der Formel \(A=\exp(\ln(A))\) ergibt sich dann
\[R(\varphi) = \exp(\varphi R(\pi/2)).\]
Bei der Menge \(\{\varphi R(\pi/2)|\varphi\in\mathbb R\}\)
handelt es sich um eine Lie-Algebra. Diese ist aber langweilig,
weil die Matrizen der Lie-Algebra in diesem Fall kommutativ sind.
Das Matrixexponential ordnet den Elementen der Lie-Algebra Elemente
der Lie-Gruppe zu. In diesem Fall besteht die Lie-Gruppe aus allen
Rotationsmatrizen. Wegen der Kommutativität kann man auf die
BCH-Formel verzichten, sodass gilt
\[e^{X+Y} = e^X e^Y.\]
Nun kann man auch schreiben
\[R(\varphi)^r = \big(e^{\varphi R(\pi/2)}\big)^r
= e^{r\varphi R(\pi/2)} = R(r\varphi).\]

\end{document}

