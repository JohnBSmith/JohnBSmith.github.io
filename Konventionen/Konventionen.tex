\documentclass[a4paper,11pt,fleqn]{article}
\usepackage[utf8]{inputenc}
\usepackage{ngerman}
\usepackage{amsmath}
\usepackage{amssymb}
\usepackage{color}
\definecolor{c1}{RGB}{00,40,60}
\usepackage[colorlinks=true,linkcolor=c1]{hyperref}

\newcommand{\ee}{\mathrm e}
\newcommand{\ui}{\mathrm i}

\begin{document}
\thispagestyle{empty}

\begin{huge}
\noindent
\textbf{Empfehlungen zum mathematischen Sprachgebrauch}
\par
\end{huge}

\tableofcontents

\subsection*{Vorwort}
Dieses Dokument beschreibt Empfehlungen zum mathematischen
Sprachgebrauch. Darin enthalten sind sowohl Schreibweisen als
auch inhaltliche Definitionen. Die Empfehlungen stehen niemals
in der Luft, sondern werden immer vollständig begründet.
Das Dokument ist nicht dogmatisch zu verstehen.

\section{Geometrie und lineare Algebra}
\subsection{Notation für Quadranten}
Im ebenen kartesischen Koordinatensystem werden Quadranten für
gewöhnlich gegen den Uhrzeitsinn mit den römischen Zahlen
I, II, III, IV nummeriert. Man startet bei $x>0$, $y>0$.

Diese Praxis erscheint mir äußert fragwürdig, weil sie in höheren
Dimensionen sehr unübersichtlich wird. Außerdem ist nicht von
vornherein klar, ob im oder gegen den Uhrzeitsinn nummeriert wird.
Weiterhin ist nicht von vornherein klar, in welchem Quadrant
die Nummerierung gestartet wird.

Ich schlage deshalb vor, die Quadranten durch PP, NP, NN, PN
zu identifizieren. Hierbei ist P als Abkürzung für \emph{positiv}
und N als Abkürzung für \emph{negativ} gemeint. Diese Abkürzungen sind
auch im Englischen und anderen europäischen Sprachen gültig.
Die Stellen in der Identifikation stehen dabei für die Stellen
im Koordinatentupel. Bei Oktanten hat man dementsprechend PPP, PPN
usw.

Weiterhin ergibt sich jetzt der Vorteil, dass die Halbebenen durch
PX, NX, XP, XN dargestellt werden können.

\subsection{Notation für Skalarprodukte}
Für das Skalarprodukt zweier Vektoren $v,w$ gibt es eine Vielzahl von
Schreibweisen, die Verwendung finden. Darunter sind $vw$, $v\circ w$,
$v\cdot w$, $v\bullet w$, $v*w$ und $\langle v,w\rangle$,
$(v,w)$, $[v,w]$. Außerdem gibt es noch $v|w$, $\langle v|w\rangle$,
$(v|w)$, $[v|w]$.

Ich schlage vor, $\langle v,w\rangle$ (einschließlich
$\langle v|w\rangle$) als einzige Schreibweise zu verwenden.

Das Skalarprodukt ist nicht assoziativ, und $\langle v,w\rangle$
hebt diesen Mangel im Gegensatz zu Schreibweisen mit Infixoperator
hervor.

Gegen die Schreibweisen $vw$ und $v\cdot w$ spricht, dass sie
bei Funktionenräumen mit dem punktweisen Produkt verwechselt
werden können. Außerdem wird mit $vw$ auch das Produkt einer
Clifford-Algebra bezeichnet.

Gegen die Schreibweise $v\circ w$ spricht, dass sie bei
Funktionenräumen mit der Komposition verwechselt werden kann.

Bei $(v,w)$ denkt man zuerst an ein Tupel, bei $[v,w]$
an ein Tupel oder ein geschlossenes Intervall. Würde man diese
Schreibweisen allgemein für Skalarprodukte verwenden, so wären sie
stark überladen.

\subsection{Adjungierte Matrix}
Für die adjungierte Matrix von A wird manchmal die Notation $A^*$ ver-
wendet. Bei dieser Notation besteht jedoch Verwechslungsgefahr mit
der konjugierten Matrix $\overline{A}$ wo manchmal ebenfalls die
Notation $A^*$ verwendet wird. 

Die Notation mit dem Dolch, $A^\dagger$, finde ich nicht so schön,
weil einige $A^t$ anstelle von $A^T$ für die transponierte Matrix
benutzen. Im Drucksatz kann das noch unterschieden werden,
aber bei Handschrift kann es schlimm sein.

Ich finde die Notation $A^H$ für die adjungierte Matrix daher
am besten. Die Notationen $A^H$ und $\overline{A}$ sind m.\,E.
unmissverständlich.

\subsection{Standardskalarprodukt}
Ich würde das Standardskalarprodukt für $v,w\in\mathbb C^n$
am besten semilinear im ersten Argument definieren:
\begin{equation}
\langle v,w\rangle := \sum_{k=1}^{n} \overline{v_k}\,w_k.
\end{equation}
Diese Variante ist kompatibel mit der Bra-Ket-Notation.
Nachteile sind mir keine bewusst.

Das Standardskalarprodukt für die Fourieranalysis würde ich
am besten durch
\begin{equation}\label{eq:FourierSP}
\langle f,g\rangle := \frac{1}{T}\int_{t_0}^{t_0+T}
\overline{f(t)}\,g(t)\,\mathrm dt
\end{equation}
definieren. Mit $T$ ist die Periodendauer gemeint.

Würde man den Normierungsfaktor $1/T$ weglassen, so würde
$\|f\|:=\sqrt{\langle f,f\rangle}$ nicht mehr mit Formel
für den klassischen Effektivwert übereinstimmen.

Würde man den Normierungsfaktor $1/T$ weglassen, so könnte man
nicht einfach schreiben:
\begin{equation}
f(t) = \sum_{k=-\infty}^\infty \langle b_k,f\rangle b_k(t).
\qquad (b_k(t):=\ee^{k\ui t})
\end{equation}
Man müsste dann schreiben:
\begin{equation}
f(t) = \sum_{k=-\infty}^{\infty} \langle b_k,f\rangle b_k(t).
\qquad(b_k(t):=\tfrac{1}{\sqrt{T}}\ee^{k\ui t})
\end{equation}
Das Problem ist hier jetzt, dass es sich bei $c_k:=\langle b_k,f\rangle$
nicht mehr um den klassischen Fourierkoeffizienten handelt, weil
der Faktor $\sqrt{T}$ herumgegeben wird.

Eine weitere Alternative ist
\begin{equation}
\langle f,g\rangle := \int_0^1 \overline{f(x)}\,g(x)\,\mathrm dx.
\end{equation}
Hier müsste man $b_k(x):=\ee^{k\ui Tx}$ verwenden. Diese Variante
scheint unüblich zu sein. Sie ist aber kompatibel zu
\eqref{eq:FourierSP}, wenn man alle beteiligten Funktionen
und Operatoren in $x$ mit $t=Tx$ darstellt.

Semilinear im ersten Argument ist das Skalarprodukt deshalb, weil
diese Variante kompatibel zur Bra-Ket-Notation ist. Nachteile sind
mir keine bekannt.

\section{Analysis}
\subsection{Differenz von Funktionswerten}
Für Differenz $F(b)-F(a)$ finde ich die Notation $[F(x)]_a^b$
am besten. Wenn man ganz pedantisch ist, so bemerkt man, dass
$x$ im Ausdruck eine gebundene Variable ist und schreibt besser
$[F(x)]_{x=a}^{x=b}$.

Die Notation $F(x)|_a^b$ finde ich ambivalent. Man muss z.B.
\[[2+F(x)]_a^b = (2+F(b))-(2+F(a)) = F(b)-F(a)\]
von
\[2+[F(x)]_a^b = 2+F(b)-F(a)\]
unterscheiden können. Bei der Notation $F(x)|_a^b$ müsste
man dafür extra ein Paar Klammern setzen.

\end{document}


