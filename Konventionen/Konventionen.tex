\documentclass[a4paper,11pt,fleqn]{article}
\usepackage[utf8]{inputenc}
\usepackage{ngerman}
\usepackage{amsmath}
\usepackage{amssymb}
\usepackage{color}
\definecolor{c1}{RGB}{00,40,60}
\usepackage[colorlinks=true,linkcolor=c1]{hyperref}

\newcommand{\ee}{\mathrm e}
\newcommand{\ui}{\mathrm i}
\renewcommand*\ttdefault{pcr}

\begin{document}
\thispagestyle{empty}

\begin{huge}
\noindent
\textbf{Empfehlungen zum mathematischen Sprachgebrauch}
\par
\end{huge}

\tableofcontents

\subsection*{Vorwort}
Dieses Dokument beschreibt Empfehlungen zum mathematischen
Sprachgebrauch. Darin enthalten sind sowohl Schreibweisen als
auch inhaltliche Definitionen. Die Empfehlungen stehen niemals
in der Luft, sondern werden immer vollständig begründet.
Das Dokument ist nicht dogmatisch zu verstehen.

\section{Geometrie und lineare Algebra}
\subsection{Notation für Quadranten}
Im ebenen kartesischen Koordinatensystem werden Quadranten für
gewöhnlich gegen den Uhrzeigersinn mit den römischen Zahlen
I, II, III, IV nummeriert. Man startet bei $x>0$, $y>0$.

Diese Praxis erscheint mir äußert fragwürdig, weil sie in höheren
Dimensionen sehr unübersichtlich wird. Außerdem ist nicht von
vornherein klar, ob im oder gegen den Uhrzeigersinn nummeriert wird.
Weiterhin ist nicht von vornherein klar, in welchem Quadrant
die Nummerierung gestartet wird.

Ich schlage deshalb vor, die Quadranten durch PP, NP, NN, PN
zu identifizieren. Hierbei ist P als Abkürzung für \emph{positiv}
und N als Abkürzung für \emph{negativ} gemeint. Diese Abkürzungen sind
auch im Englischen und anderen europäischen Sprachen gültig.
Die Stellen in der Identifikation stehen dabei für die Stellen
im Koordinatentupel. Bei Oktanten hat man dementsprechend PPP, PPN
usw.

Weiterhin ergibt sich jetzt der Vorteil, dass die Halbebenen durch
PX, NX, XP, XN dargestellt werden können.

\subsection{Polarkoordinaten}
Seien $(x,y)$ die kartesischen Koordinaten und $(r,\varphi)$
die Polarkoordinaten. Die Berechnung von $\varphi$ in
Abhängigkeit von $x,y$ geschieht m.\,E. entweder nach
einem umständlichen Algorithmus oder fehlerhaft.
Folgende Formel lässt sich jedoch leicht merken und deckt
alle Fälle ab:
\begin{equation}
\varphi = s(y)\,\arccos\Big(\frac{x}{r}\Big).
\end{equation}
Hierbei ist $s(y)$ die rechtsstetige
Signumfunktion. Sie ist definiert durch
\begin{equation}
s(y):=
\begin{cases}
y{\ge}0\colon\,1,\\
y{<}0\colon\,{-}1.
\end{cases}
\end{equation}
Oder alternativ via Iverson-Klammern:
\begin{equation}
s(y):=[y{\ge}0]-[y{<}0].
\end{equation}
Oder alternativ durch einen Trick:
\begin{equation}
s(y) := \operatorname{sgn}(y)+1-|\operatorname{sgn}(y)|.
\end{equation}
Steht ein Computer zur Verfügung, und das ist der Normalfall geworden,
so wird man direkt oder indirekt \verb|atan2(y,x)| verwenden.
Man kann sich darauf verlassen dass \verb|atan2| möglichst präzise
implementiert ist.

Wenn ein positiver Winkel gefordert wird, muss man
anschließend nur $2\pi$ zu $\varphi$ addieren.


\subsection{Kugelkoordinaten}

Kugelkoordinaten können verwirrend sein, weil es verschiedene
Bezeichnungs- und Darstellungskonventionen gibt. Anstelle sich auf eine
Darstellungskonvention festzulegen, will ich hier alle Konventionen
miteinander verbinden. Das hört sich jetzt kompliziert an. Aber in
diesem Fall kommt nach kompliziert tatsächlich einfach.

Mit $\lambda$ bezeichnen wir die Länge (in der Astronomie der
Azimutwinkel). Man definiert für gewöhnlich $0\le\lambda<2\pi$ oder
$-\pi<\lambda\le\pi$. Welche der beiden Festlegungen man verwendet,
ist eigentlich belanglos, weil Winkel eigentlich nur modulo $2\pi$
bestimmt sind und man bei negativen Winkeln daher $2\pi$ addieren
kann, wenn man will.

Mit $\beta$ bezeichnen wir die Breite. Man definiert
$-\pi/2\le\beta\le\pi/2$.
Außerdem bezeichnen wir mit $\theta$ die Kobreite (auch Polarwinkel
genannt). Es gilt $\theta=\pi/2-\beta$ und $0\le\theta\le\pi$.

Man geht also zuerst mit $\lambda$ auf dem Äquator entlang und
bewegt sich dann mit $\beta$ nach oben oder unten. Alternativ
kommt man mit $\theta$ vom Nordpol (in der Astronomie der Zenit)
herunter.

Mit $r$ wird der Radius der Kugel bezeichnet. Mit $r_{xy}$ wird
der Radius bezeichnet, welcher sich bei der Projektion eines Punktes
der Kugeloberfläche auf die $xy$-Ebene ergibt. Das ist eine
Hilfsgröße, die gleich gebraucht wird.

Für das Dreieck in der $xy$-Ebene gilt nun
\begin{equation}
  \begin{split}
  x &= r_{xy}\cos\lambda,\\
  y &= r_{xy}\sin\lambda.
  \end{split}
\end{equation}
Die $z$-Achse und der Radiusvektor zu $r$ spannen ein
Dreieck auf. Für dieses gilt
\begin{equation}
  \begin{split}
  z &= r\cos\theta,\\
  r_{xy} &= r\sin\theta.
  \end{split}
\end{equation}
Wenn man beide Formeln kombiniert, so erhält man
\begin{equation}
  \begin{split}
  x &= r\cos\lambda\sin\theta,\\
  y &= r\sin\lambda\sin\theta,\\
  z &= r\cos\theta.
  \end{split}
\end{equation}
Man sollte sich die Formeln $\sin(\pi/2-x)=\cos x$
und $\cos(\pi/2-x)=\sin x$ merken. Wenn man diese benutzt, so
ergibt sich sofort die alternative Darstellung
\begin{equation}
  \begin{split}
  x &= r\cos\lambda\cos\beta,\\
  y &= r\sin\lambda\cos\beta,\\
  z &= r\sin\beta.
  \end{split}
\end{equation}
Für $\lambda=\mathrm{const}$ erhält man Längenhalbkreise, für
$\beta=\mathrm{const}$ Breitenkreise.

Ich erachte es als besser, die unterschiedlichen Konventionen so
darzustellen, dass sie ohne Konflikte verbunden werden können.
Anstelle von zwei Variablen wurden daher die drei Variablen
$\lambda,\beta,\theta$ definiert.

Oft wird auch die Bezeichnung $\varphi=\lambda$ verwendet.
Das kommt daher, dass in der $xy$-Ebene die Polarkoordinaten
mit der Transformation
\begin{equation}
  \begin{split}
  x &= r_{xy}\cos\varphi,\\
  y &= r_{xy}\sin\varphi
  \end{split}
\end{equation}
enthalten sind. Das $\varphi$ steht hierbei für \emph{Phase}.

Es stellt sich noch die Frage, welche Reihenfolge man für die
Funktionsargumente verwendet. Sollte man die oft in
der Physik anzutreffende Reihenfolge $f(r,\theta,\varphi)$
mit $\varphi=\lambda$ übernehmen? Wesentlich vernünftiger ist m.\,E.
die Reihenfolge $f(r,\varphi,\theta)$. Das harmoniert besser mit
den Hyperkugelkoordinaten. Im $n$-dimensionalen Raum ergibt sich
nämlich
\begin{equation}\label{eq:Hyperkugel}
f(r,\varphi,\theta_1,\theta_2,\ldots,\theta_{n-2}),
\end{equation}
wobei sich das Intervall für $\varphi$ über einen Vollkreis
erstreckt, die Intervalle für die $\theta_k$ aber nur jeweils
über einen Halbkreis.

Außerdem hat man bei Zylinderkoordinaten auch immer die
Reihenfolge%
\begin{equation}\label{eq:Zylinder}
f(r,\varphi,z)
\end{equation}
und nicht $f(r,z,\varphi)$.

Bei ebenen Polarkoordinaten ergibt sich $f(r,\varphi)$
außerdem als Spezialfall von \eqref{eq:Hyperkugel}
und als Bestandteil von \eqref{eq:Zylinder}.


\subsection{Notation für Skalarprodukte}
Für das Skalarprodukt zweier Vektoren $v,w$ gibt es eine Vielzahl von
Schreibweisen, die Verwendung finden. Darunter sind $vw$, $v\circ w$,
$v\cdot w$, $v\bullet w$, $v*w$ und $\langle v,w\rangle$,
$(v,w)$, $[v,w]$. Außerdem gibt es noch $v|w$, $\langle v|w\rangle$,
$(v|w)$, $[v|w]$.

Ich schlage vor, $\langle v,w\rangle$ (einschließlich
$\langle v|w\rangle$) als einzige Schreibweise zu verwenden.
Hat man nur Plain-Text zur Verfügung (z.B. im Chat), so
kann man \verb|<v,w>| schreiben.

Das Skalarprodukt ist nicht assoziativ, und $\langle v,w\rangle$
hebt diesen Mangel im Gegensatz zu Schreibweisen mit Infixoperator
hervor.

Gegen die Schreibweisen $vw$ und $v\cdot w$ spricht, dass sie
bei Funktionenräumen mit dem punktweisen Produkt verwechselt
werden können. Außerdem wird mit $vw$ auch das Produkt einer
Clifford-Algebra bezeichnet.

Gegen die Schreibweise $v\circ w$ spricht, dass sie bei
Funktionenräumen mit der Komposition verwechselt werden kann.

Bei $(v,w)$ denkt man zuerst an ein Tupel, bei $[v,w]$
an ein Tupel oder ein geschlossenes Intervall. Würde man diese
Schreibweisen allgemein für Skalarprodukte verwenden, so wären sie
stark überladen.

\subsection{Notation für adjungierte Matrizen}
Für die adjungierte Matrix von A wird manchmal die Notation $A^*$
verwendet. Bei dieser Notation besteht jedoch Verwechslungsgefahr mit
der konjugierten Matrix $\overline{A}$ wo manchmal ebenfalls die
Notation $A^*$ verwendet wird. 

Die Notation mit dem Dolch, $A^\dagger$, finde ich nicht so schön,
weil einige $A^t$ anstelle von $A^T$ für die transponierte Matrix
benutzen. Im Drucksatz kann das noch unterschieden werden,
aber bei Handschrift kann es schlimm sein.
Gegen den Dolch spricht weiter, dass diese Notation nicht
verwendet werden kann, wenn man nur Plain-Text zur Verfügung
hat.

Ich finde die Notation $A^H$ für die adjungierte Matrix daher
am besten. Die Notationen $A^H$ und $\overline{A}$ sind m.\,E.
unmissverständlich. Hat man nur Plain-Text zur Verfügung
(z.B. im Chat), so kann man \verb|A^H| und \verb|conj(A)|
schreiben.

\subsection{Standardskalarprodukt}
Ich würde das Standardskalarprodukt für $v,w\in\mathbb C^n$
am besten semilinear im ersten Argument definieren:
\begin{equation}
\langle v,w\rangle := \sum_{k=1}^{n} \overline{v_k}\,w_k.
\end{equation}
Diese Variante ist kompatibel mit der Bra-Ket-Notation.
Nachteile sind mir keine bewusst.

Das Standardskalarprodukt für die Fourieranalysis würde ich
am besten durch
\begin{equation}\label{eq:FourierSP}
\langle f,g\rangle := \frac{1}{T}\int_{t_0}^{t_0+T}
\overline{f(t)}\,g(t)\,\mathrm dt
\end{equation}
definieren. Mit $T$ ist die Periodendauer gemeint.

Würde man den Normierungsfaktor $1/T$ weglassen, so würde
$\|f\|:=\sqrt{\langle f,f\rangle}$ nicht mehr mit der Formel
für den klassischen Effektivwert übereinstimmen.

Würde man den Normierungsfaktor $1/T$ weglassen, so könnte man
nicht einfach schreiben:
\begin{equation}
f(t) = \sum_{k=-\infty}^\infty \langle b_k,f\rangle b_k(t).
\qquad (b_k(t):=\ee^{k\ui \omega t})
\end{equation}
Man müsste dann schreiben:
\begin{equation}
f(t) = \sum_{k=-\infty}^{\infty} \langle b_k,f\rangle b_k(t).
\qquad(b_k(t):=\tfrac{1}{\sqrt{T}}\ee^{k\ui \omega t})
\end{equation}
Das Problem ist hier jetzt, dass es sich bei $c_k:=\langle b_k,f\rangle$
nicht mehr um den klassischen Fourierkoeffizienten handelt, weil
der Faktor $\sqrt{T}$ herumgegeben wird.

Semilinear im ersten Argument ist das Skalarprodukt deshalb, weil
diese Variante kompatibel zur Bra-Ket-Notation ist. Nachteile sind
mir keine bekannt.

\section{Analysis}
\subsection{Differenz von Funktionswerten}
Für die Differenz $F(b)-F(a)$ finde ich die Notation $[F(x)]_a^b$
am besten. Wenn man ganz pedantisch ist, so bemerkt man, dass
$x$ im Ausdruck eine gebundene Variable ist und schreibt besser
$[F(x)]_{x=a}^{x=b}$.

Die Notation $F(x)|_a^b$ finde ich ambivalent. Man muss z.B.
\[[2+F(x)]_a^b = (2+F(b))-(2+F(a)) = F(b)-F(a)\]
von
\[2+[F(x)]_a^b = 2+F(b)-F(a)\]
unterscheiden können. Bei der Notation $F(x)|_a^b$ müsste
man dafür extra ein Paar Klammern setzen.

\subsection{Notation für Kettenbrüche}
Man schreibt für gewöhnlich
\begin{equation}
b_0 + \frac{a_1|}{|b_1} + \frac{a_2|}{|b_2} + \ldots
\quad:=\quad b_0+\frac{a_1}{b_1+\frac{a_2}{b_2+\ldots}}.
\end{equation}
Diese Notation ist der Beschränkung unterworfen, dass sich damit
nur Kettenbrüche, jedoch keine anderen Kettenausdrücke formulieren
lassen. Man würde daher lieber notieren:
\begin{equation}
b_0+\Big(x\mapsto\frac{a_1}{b_1+x}\Big)
\circ\Big(x\mapsto\frac{a_2}{b_2+x}\Big)
\circ\ldots\circ\Big(x\mapsto\frac{a_n}{b_n+x}\Big)(x_0).
\end{equation}
Das ist sehr langatmig. Deshalb schlage ich die Kurznotation
\begin{equation}
[x{=}]\quad b_0+\frac{a_1}{b_1+x},\,\frac{a_2}{b_2+x},\,\ldots,\,
\frac{a_n}{b_n+x},\,x_0
\end{equation}
vor. Sind nun die Funktionen $T_k\colon X\to X$ gegeben, so
kann man schreiben
\begin{equation}
\begin{split}
& [x{=}]\quad T_1(x),\,T_2(x),\,\ldots,\,T_n(x),\,x_0\\
&= (T_1\circ T_2\circ\ldots\circ T_n)(x_0).
\end{split}
\end{equation}
Alternativ gibt es noch
\begin{equation}
\begin{split}
& [{=}x]\quad x_0,\, T_1(x),\, T_2(x),\,\ldots,\, T_n(x)\\
&= (T_n\circ\ldots\circ T_2\circ T_1)(x_0).
\end{split}
\end{equation}


\section{Algebra}
\subsection{Notation für Körpererweiterungen}

Ich schlage für Körpererweiterungen die Notation $L/\!/K$
vor.

Körpererweiterungen werden zuweilen durch $L/K$ notiert, um
auszudrücken, dass $L$ ein Erweiterungskörper von $K$ ist.
Man denkt sich dabei, dass $L$ über $K$ steht. Die Notation
$M/A$ ist aber eigentlich schon für Quotientenmengen vergeben,
wenn durch $A$ auf irgendeine Art eine Äquivalenzrelation
gegeben ist. Sowohl Quotientenmengen also auch Körpererweiterungen
kommen ausgerechnet in der Algebra sehr häufig vor.
Um die Dringlichkeit deutlich zu machen, hier ein Beispiel
wo beides in einer Formel vorkommt:
\begin{equation}
\mathbb R[X]/(X^2+1)/\!/\mathbb R.
\end{equation}
Man könnte auch auf die Idee kommen, einen Körpererweiterung
durch $K\le L$ zu symbolisieren, weil ja $K\subseteq L$ ist.
Das ist aber eine schlechte Idee, weil hiermit schon
Untergruppenbeziehungen symbolisiert werden und Körper ja
additive Gruppenstruktur enthalten.

\section{Notation}
\subsection{Bindungsstärke von Operatoren}
Die Schreibweisen $\forall x\colon P(x)$ und $\exists x\colon P(x)$
für die Quantoren binden ihr Prädikat schwächer als alle anderen
Operatoren, so dass alles nach dem Doppelpunkt zum Prädikat gehört.
Diese Regel ist aber nicht allen bewusst. M.\,E. ist die Notation
\begin{equation}
\forall x[P(x)]
\end{equation}
besser geeignet, da hier explizit sichtbar ist, wo das Prädikat endet.

\end{document}


