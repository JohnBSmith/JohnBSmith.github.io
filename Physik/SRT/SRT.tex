\documentclass[a4paper,11pt,fleqn,twocolumn,twoside,dvipdfmx]{scrartcl}
\usepackage[utf8]{inputenc}
\usepackage[T1]{fontenc}
\usepackage[ngerman]{babel}
\usepackage{libertine}
\usepackage[libertine,smallerops]{newtxmath}
\usepackage[scaled=0.80]{DejaVuSans}
\usepackage[scaled=0.80]{DejaVuSansMono}
\usepackage{microtype}

\usepackage{amsmath}
\usepackage{amssymb}
\usepackage{color}
\definecolor{c1}{RGB}{00,40,80}
\usepackage[colorlinks=true,linkcolor=c1]{hyperref}
\usepackage{geometry}
\geometry{a4paper,left=25mm,right=10mm,top=24mm,bottom=28mm}
\setlength{\columnsep}{5mm}
\begin{document}

\begin{huge}
\noindent\normalfont\bfseries\sffamily
SRT
\par
\end{huge}

\vspace{1em}\noindent
Mai 2019

\tableofcontents

\section{Grundlagen}

\subsection{Die Kraft}

Bisher war die Gleichung für die Kraft in der Newtonschen Mechanik
durch $F=ma$ gegeben. Die Differentialgleichung $F=p'(t)$ erfüllt dies
auch, da $p'=mv'=ma$ ist. Sei $m$ die Ruhemasse.
Der Impuls ist nun aber $p=\gamma mv$. Weil sich an der Gültigkeit
der Differentialgleichung nichts ändert, berechnet man mit der
Produktregel%
\[p' = m(\gamma 'v + \gamma a).\]
Dabei ist der Lorentzfaktor
\[\gamma = \frac{1}{\sqrt{1-(v/c)^2}}.\]
Dieser Faktor ist von der Geschwindigkeit ab"-hängig.
Falls die Geschwindigkeit von der Zeit abhängig ist, so ist es der
Faktor also ebenfalls. Außerdem hat er die Eigenschaft $\gamma\ge 1$.
Der Fall $\gamma=1$ ergibt sich, falls sich das Raumschiff in der
Ruhe befindet.

Der Lorentzfaktor wird unter doppelter Verwendung der Kettenregel
abgeleitet zu%
\[\gamma' = \frac{va}{c^2}\Big(1-\frac{v^2}{c^2}\Big)^{-3/2}
= \frac{va}{c^2} \gamma^3.\]
Die Kraft ist damit
\[F = ma\,\Big(\frac{v^2}{c^2}\gamma^3 + \gamma\Big).\]
Bildet man den Hauptnenner durch Erweitern mit der Wurzel und fasst
zusammen, so reduziert sich der Ausdruck auf%
\[F = ma\gamma^3.\]

\subsection{Die Energie}

Die Arbeit lässt sich aus der Kraft berechnen. Man forme um:%
\[a\,\mathrm ds = \frac{\mathrm dv}{\mathrm dt} v\,\mathrm dt
= v\,\mathrm dv.\]
Die Ableitung des Lorentzfaktors nach der Geschwindigkeit wird
benötigt. Sie ist Analog zur Zeitableitung. Man lässt nur den letzten
Schritt aus. Man erhält%
\[\frac{\mathrm d\gamma}{\mathrm dv} = \frac{v}{c^2} \gamma^3.\]
Damit kann das Differential ausgetauscht werden und Integration durch
Substitution ist möglich. Dies beides verwendet man für die
Stammfunktion%
\begin{gather*}
\int F\,\mathrm ds = \int m\gamma^3 v\,\mathrm dv
= \int mc^2\,\mathrm d\gamma\\
= \gamma mc^2 + K.
\end{gather*}
Damit ist die Arbeit nach dem Hauptsatz%
\[\int_0^v F\,\mathrm ds = \gamma mc^2 - mc^2.\]
Für den Lorentzfaktor ist die Gleichung%
\[\gamma^2=\gamma^2\frac{v^2}{c^2}+1\]
gültig. Dies kann man durch Umformen nach $\gamma$ überprüfen.
Einsetzen bringt nun%
\[E^2 = \gamma^2m^2c^4
= \Big(\gamma^2\frac{v^2}{c^2}+1\Big)\,m^2c^4.\]
Man erhält die Energie"=Impuls"=Beziehung%
\[E^2 = p^2c^2+m^2c^4.\]


\subsection{Die Eigenzeit}

Mit der Zeit ist es in der SRT problematisch. Wir führen deshalb die
Eigenzeit ein. Die Eigenzeit $\tau$ ist die Zeit, welche eine Uhr
anzeigt, welche man mit auf das Raumschiff nimmt. Bewegt sich das
Raumschiff, so wird die Uhr die Zeit an Bord des Raumschiffes
anzeigen. Das ist die Eigenzeit des Raumschiffes.

Man fragt sich nun natürlich, wie der Zusammenhang zwischen der
Eigenzeit $\tau$ und der Zeit $t$ eines Inertialsystems ist.
Wäre das Raumschiff unbeschleunigt, so könnte man die Formel für die
Zeitdilatation benutzen. Aus Sicht der Erde geht die Uhr an Bord
langsamer, es ist
$\Delta\tau<\Delta t$.
Die Formel lautet dann $\Delta t = \gamma\Delta\tau.$

Wenn die Zeit aber in ganz kleine Abschnitte unterteilt wird,
dann kann das Raumschiff in jedem Abschnitt als unbeschleunigt
angesehen werden, denn die Geschwindigkeit wird sich in dieser
kurzen Zeitspanne nicht merklich ändern. Das heißt, es ist%
\[\mathrm dt = \gamma(t)\,\mathrm d\tau.\]
Man erhält somit
\[\tau = \int_0^t \frac{1}{\gamma(t)}\,\mathrm dt.\]
Wenn man die Bewegung des Raumschiffes aus Sicht der Erde kennt,
so kann man damit $\gamma(t)$ und somit auch $\tau$ ausrechnen.
Die Eigenzeit kann also als eine Funktion der Zeit dargestellt werden.

Hiermit gelingt es auch, das seltsame Zwillingsparadoxon aufzulösen.
Aus Sicht des Raumschiffes wird sich die Erde doch auch bewegen, und
die Uhren der Erde werden auch langsamer gehen. Wieso wird die
Besatzung des Schiffes dann langsamer gealtert sein, wenn es wieder
bei der Raumstation der Erde eintrifft?

Wir denken uns dazu zwei Raumstationen, die zueinander in Ruhe sind.
Es wird nun ein Raumschiff losgeschickt. Das Raumschiff beschleunigt
gleich"-mäßig, bewegt sich eine Weile gleichförmig und bremst dann
wieder gleich"-mäßig ab, bis es sich bei der anderen Raumstation
in Ruhe befindet.

Wir gehen nun davon aus, dass auf das Raumschiff eine konstante
Kraft $F$ wirkt, die das Raumschiff aus seiner Sicht gleichmäßig
beschleunigt. Aus Sicht der Raumstation ist $F=ma\gamma^3$. Daher
wird $a$ zunehmend kleiner, damit $F$ konstant bleibt. Um die
Eigenzeit zu berechnen, benötigt man $\gamma$, was aber wiederum
von $v(t)$ abhängig ist.

Es ist daher einfacher, die Gleichung $p'=F$ zu nehmem, und sie auf
beiden Seiten zu integrieren. Man rechnet also%
\[\int_0^t (\gamma mv)'\,\mathrm dt = \int_0^t F\,\mathrm dt.\]
Mit dem Hauptsatz erhält man%
\[\gamma mv-\gamma_0mv_0 = Ft.\]
Wir wählen die Abkürzung%
\[A:=\frac{Ft}{mc}+\frac{\gamma_0v_0}{c}.\]
Damit ist $\gamma v = Ac$. Umstellen nach $v$ bringt%
\[v = \frac{Ac}{\sqrt{1+A^2}}.\]
Nun können wir $v$ in $\gamma$ einsetzen und damit die Eigenzeit
berechnen. Besser ist es jedoch, gleich nach $\gamma$ umzustellen.
Man stellt erstmal den Lorentzfaktor um und erhält%
\[v = \frac{c}{\gamma}\sqrt{\gamma^2-1}.\]
Einsetzen bringt dann $A=\sqrt{\gamma^2-1}$. Das formt man nach
$\gamma$ um. Man erhält%
\[\gamma=\sqrt{1+A^2}.\]
Eigentlich hätte man $\gamma$ oben auch durch scharfes Hinsehen
ablesen können.

Wir unterteilen die Zeit nun in die Beschleunigungsphase $[0,t_1]$,
die gleichförmige Phase $[t_1,t_2]$ und die Bremsphase $[t_2,t_3]$.
Für alle drei Phasen muss die Eigenzeit berechnet werden. Aus einer
Integraltafel entnimmt man%
\[\int \frac{\mathrm dx}{\sqrt{1+x^2}} = \mathrm{arsinh}\,x+K.\]
Mit linearer Substitution erhält man%
\begin{gather*}
\tau = \int_a^b \frac{\mathrm dt}{\sqrt{1+A^2}}\\
= \frac{mc}{F}[\mathrm{arsinh}\,A(b)-\mathrm{arsinh}\,A(a)].
\end{gather*}
%
Das folgende Beispiel kann man numerisch durchrechnen.
\begin{gather*}
m=1,\,\,c=1,\\
F = 10[0<t<1]-10[2<t<3]
\end{gather*}
Man erhält
\begin{gather*}
A = 10t[0<t<1]+10[1<t<2]\\
-10(t-3)[2<t<3],\\
\tau = \frac{1}{10}\mathrm{arsinh}(10t)[0<t<1]\\
+ \frac{1}{10}(\mathrm{arsinh}(10)+t-1)[1<t<2]\\
+ \frac{1}{10}(\mathrm{arsinh}(10t-30)\\
+2\,\mathrm{arsinh}(10)+1)[2<t<3].
\end{gather*}
Wenn man nun die Beschleunigungsphase und die Bremsphase sehr kurz
sein lässt, so erhält man%
\[\tau = \frac{t}{\gamma}.\]
Das ist natürlich seltsam. Nach dem Beschleunigen sollte sich das
Raumschiff wieder in einem Inertialsystem befinden und dort sollte
man die Uhr der Raumstation als langsamer wahrnehmen. Man kann auch
die ganze Strecke zwischen beiden Raumstationen mit synchronisierten
Uhren versehen. Das Argument, die Uhren wären weit entfernt,
zählt also nicht.

Wichtig ist nun, dass die letzte Formel nicht die Sicht des
Raumschiffes beschreibt.

Die einzige Erklärung muss die folgende sein. Aus der Sicht des
Raumschiffes treten starke Zeitverschiebungen während der
Beschleunigungsphase oder der Bremsphase auf, welche die Verlangsamung
der Umgebung mehr als kompensieren.


\subsection{Vierervektoren}

Die drei Ortskoordinaten kann man mit der Zeit zu einem Vierervektor
zusammenfassen. Die nullte Koordinate soll dabei die mit der
Lichtgeschwindigkeit multiplizierte Zeit sein. Den Ortsvektor wollen
wir mit $\vec r$ bezeichnen. Es sei also
\[(x^k) = (x^0,x^1,x^2,x^3) := (ct,r_1,r_2,r_3).\]
Man schreibt auch kürzer $(x^k)=(ct,\vec r)$.
Der metrische Tensor lautet%
\begin{gather*}
g_{ij} = \begin{bmatrix}
1 & 0 & 0 & 0\\
0 & -1 & 0 & 0\\
0 & 0 & -1 & 0\\
0 & 0 & 0 & -1
\end{bmatrix}\\
= \mathrm{diag}(1,-1,-1,-1).
\end{gather*}
Man kann den Index mit dem metrischen Tensor senken. Es ist%
\[x_k = \sum_{i=0}^3 g_{ki}x^i = \sum_{i} g_{ki}x^i = g_{ki}x^i.\]
Die letzte Schreibweise ist die einsteinsche Summenkonvention. Diese
besagt, dass über Indizes, die einmal unten und einmal oben auftreten,
summiert wird. Wir wollen diese Konvention aus stilistischen Gründen
gelegentlich verwenden. Durch das Senken erhält man
\[(x_k) = (x_0,x_1,x_2,x_3) = (ct,-x^1,-x^2,-x^3).\]
Die Multiplikation mit der Diagonalmatrix $g_{ij}$ ist natürlich
besonders einfach, weil $g_{ij}=0$ ist, wenn die Indizes nicht
übereinstimmen. Man hat daher einfacher, und ohne eine Summe zu bilden%
\[x_k = g_{kk}x^k.\]
Außerdem ist der metrische Tensor zu sich selbst invers. Das heißt,
es ist $g^{-1}=g$, oder in Koordinaten $g^{ij}=g_{ij}$. Das kann man
leicht einsehen, weil ja $g^{kk}g^{kk}=1$ ist. Daher ist%
\[x^k = g^{kk}x_k.\]
Manchmal wird der metrische Tensor der flachen Raumzeit auch mit
$\eta$ bezeichnet. Weiterhin gibt es auch die Konventionen%
\begin{align*}
g&=\mathrm{diag}(-1,1,1,1),& (x^0=ct)\\
g&=\mathrm{diag}(1,1,1,-1),& (x^4=ct)\\
g&=\mathrm{diag}(-1,-1,-1,1).& (x^4=ct)
\end{align*}
Ich kann nicht sagen, ob eine der vier Konventionen einen
wesentlichen Vorteil bringt.

Ableiten des Ortsvektors nach der Eigenzeit bringt%
\begin{gather*}
u = \frac{\mathrm dx}{\mathrm d\tau}
= \gamma(t)\frac{\mathrm dx}{\mathrm dt}\\
= \gamma(t)(c,v_1,v_2,v_3) = \gamma(c,\vec v).
\end{gather*}
Die folgende Rechnung bringt uns ein merkwürdiges Resultat.
Das Quadrat der Geschwindigkeit ist%
\begin{gather*}
u^k u_k = \gamma^2 (c^2-v_1^2-v_2^2-v_3^2)
= \gamma^2 (c^2-v^2)\\
= \frac{c^2-v^2}{1-(v/c)^2}
= c^2 \frac{c^2-v^2}{c^2-v^2}
= c^2.
\end{gather*}
Man erhält $|u|=c$. Das heißt, alles bewegt sich mit
Lichtgeschwindigkeit. Mit $\vec p=\gamma m\vec v$
erhält man%
\[u^k = \frac{1}{m} (\gamma mc,p_1,p_2,p_3).\]
Es drängt sich auf, $\gamma mc$ als Komponente des Viererimpulses
zu sehen. Man definiert daher $P = mu.$
Wegen $E/c=\gamma mc$ ist $P=(E/c,\vec p)$. Man erhält%
\[P^k P_k = (E/c)^2-p^2.\]
Außerdem ist
\[P^k P_k = (mu^k)(mu_k) = m^2 u^k u_k = m^2 c^2.\]
Setzt man gleich und formt um, so erhält man wieder die
Energie"=Impuls"=Beziehung%
\[E^2 = p^2c^2 + m^2c^4.\]
Die Viererbeschleunigung $A$ wird als die Ableitung der
Vierergeschwindigkeit $u$ nach der Eigenzeit definiert.
Die Viererkraft $K$ definiert man als Eigenzeitableitung des
Viererimpulses $P$. Durch diese Festlegung hat man $K=mA$.
Die Beschleunigung ist%
\begin{gather*}
A = \frac{\mathrm du}{\mathrm d\tau}
= \gamma(t)\frac{\mathrm du}{\mathrm dt}
= \gamma\frac{\mathrm d}{\mathrm dt}(\gamma c,\gamma\vec v)\\
= \gamma (\gamma' c,\gamma'\vec v+\gamma\vec a).
\end{gather*}

\section{Die Raumzeit}
\subsection{\texorpdfstring{Metrischer Tensor und\newline
Poincaré-Gruppe}{Metrischer Tensor und Poincare-Gruppe}}

Wir wissen, dass die Poincaré"=Gruppe die Geometrie der Raumzeit
kodiert. Andererseits enthält der metrische Tensor die Information
über die Geometrie der Raumzeit. Es stellt sich die Frage, was denn
nun der Zusammenhang zwischen metrischem Tensor und Poincaré"=Gruppe
ist. Um diese Frage zu klären, schieben wir den Minkowskiraum zunächst
beiseite und betrachten stattdessen die einfacher handhabbare
euklidische Ebene. Der metrische Tensor hat hier die besonders
einfache Gestalt%
\[g=\begin{bmatrix}
1 & 0\\
0 & 1
\end{bmatrix}.\]
Hiermit kann der Betrag berechnet werden. Es ist%
\[|x| = \sqrt{\sum_{i,j}g_{ij}x^ix^j}.\]
Der Betrag induziert die Abstandsfunktion%
\[d(x,y) = |x-y|.\]
Die Abstandsfunktion (Metrik) bestimmt aber nun darüber, welche
Abbildungen isometrisch sind, und welche nicht. Aber dann legt die
Metrik doch auch implizit fest, wie die Isometriegruppe auszusehen hat.
Für eine gegebene Grundmenge filtert die Metrik alle Isometrien aus
der Grundmenge heraus. Wie wählt man nun die Grundmenge? Zunächst muss
die Definitionsmenge festgelegt werden. Der metrische Tensor kann man
aus den Basisvektoren berechnet haben. Hat man stattdessen den
metrischen Tensor gegeben, so legt er in diesem Fall eine
Orthonormalbasis fest. Man kann sich nun eine Orthonormalbasis
aussuchen, für jede Richtung gibt es genau eine. Dann ist die
Definitionsmenge aber doch%
\[V = \mathrm{Span}(e_1,e_2).\]
Die Grundmenge ist dann $\mathrm{Abb}(V,V)$. Vergisst man nun den
Ursprung, so erhält man den affinen Punktraum. Dieser hat die
euklidische Gruppe als Isometriegruppe.

Da die Metrik die Geometrie des affinen Raumes schon komplett bestimmt,
muss man die Isometriegruppe also aus der Metrik herleiten können.

Die Lorentztransformationen müssten sich also aus der Minkowski"=Metrik
ergeben.

\newpage
\mbox{}
\vfill\noindent
\texttt{Dieser Text steht unter der Lizenz\\
Creative Commons CC0.}

\end{document}


