\documentclass[a4paper,11pt,fleqn,twocolumn,twoside,dvipdfmx]{scrartcl}
\usepackage[utf8]{inputenc}
\usepackage[T1]{fontenc}
\usepackage[ngerman]{babel}
\usepackage{microtype}
\usepackage{libertine}
\usepackage[libertine]{newtxmath}
\usepackage[scaled=0.80]{DejaVuSans}

\usepackage{amsmath}
\usepackage{amssymb}
\usepackage{color}
\definecolor{c1}{RGB}{00,40,80}
\usepackage[colorlinks=true,linkcolor=c1]{hyperref}
\usepackage{geometry}
\geometry{a4paper,left=26mm,right=10mm,top=24mm,bottom=28mm}
\setlength{\columnsep}{5mm}
\begin{document}
% \thispagestyle{empty}

\begin{huge}
\noindent
\textbf{\textsf{Fluidmechanik}}
\par
\end{huge}

\tableofcontents

\section{Hydrostatik}
\subsection{Hydrostatische Grundgleichung}

Man stelle sich ein senkrecht gehaltenes Zylinderrohr vor, das
oben und unten offen ist. Wenn sich Wasser mit der Masse $m$ in
dem Rohr befindet, so wird es von der Kraft $F=mg$ nach unten
gezogen und wird sich nach unten bewegen. Ein Kolben, der am unteren
Ende eingeführt wird, muss also die betragsmäßig gleich große
Gegenkraft aufbringen, um die Wassersäule in der Schwebe zu halten.
Da die Kraft normal (senkrecht) auf der Kolbenfläche steht, lässt sich
die Definition des Druckes direkt Verwenden. Es ist%
\[p = \frac{F}{A} = \frac{mg}{A} = \frac{\varrho Vg}{A}
= \frac{\varrho Ahg}{A} = \varrho gh.\]
Mit $h$ ist dabei die Höhe der Wassersäule gemeint.

Auch der Luftdruck wirkt. Jedoch wirkt er sowohl von unten als
auch von oben. Das gleicht sich aus, sodass er keinen Einfluss hat.
Wenn man den Kolben entnimmt und das Rohr stattdessen unten
verschließt, dann muss der Luftdruck, welcher von oben auf die
Wassersäule einwirkt, mit einbezogen werden. Bezeichnen wir diesen
Umgebungsdruck mit $p_0$. Damit erhält man also die hydrostatische
Grundgleichung%
\[p = p_0+\varrho gh.\]

\subsection{Hydrostatisches Paradoxon}

Der Druck scheint nur von der Höhe $h$ der Wassersäule
abhängig zu sein. Was ist, wenn ein Taucher unter Wasser unter einen
Überhang taucht. Über ihm befindet sich dann doch eine kleinere
Wassersäule. Wie groß ist der Druck, der auf ihn wirkt?

Dazu denkt man sich einen Zylinder, der in der Mitte um $90^\circ$
umgebogen und am Ende verschlossen ist. Der Querschnitt soll bezüglich
der Höhe der Wassersäule näherungsweise vernachlässigbar sein. Gegen
die Kraft der Wassersäule wirkt die Zwangskraft vom Boden. Die Kraft
wirkt aber auch auf die Seiten. Sie wird waagerecht bis zum Ende
des Zylinders wirken. Man stellt also fest, dass der Druck in einer
bestimmten Höhe $h$ überall gleich sein muss.

Das hat natürlich Konsequenzen, die recht paradox anmuten.
Zum Beispiel lässt sich ein geschlossenes Fass zum platzen bringen.
Man lässt oben aus dem Fass ein recht dünnes Rohr in die Höhe führen
und füllt Wasser hinein. Ist das Rohr hoch genug, so wird der Druck
im Fass so groß, dass es platzt. Zumindest wird irgendwann Wasser
durch die Ritzen spritzen. Dieses Experiment wurde bereits von
Blaise Pascal durchgeführt.

Das Paradoxon wird dadurch aufgelöst, dass es sich hier nicht um
eine Energiebetrachtung handelt, sondern um eine Kraftbetrachtung.
Durch das dünne hohe Rohr lässt sich eine hohe Kraft aufbringen.
Da die Wassersäule im dünnen Rohr nicht viel Volumen hat, ergibt sich
trotzdem eine geringe potentielle Energie. Wenn das Fass geöffnet
wird, so wird die Wassersäule schnell sinken. Es ist damit nicht
möglich, mit dieser hohen Kraft besonders viel zu bewerkstelligen.
Das Prinzip ist also das gleiche, wie beim Hebel und beim Flaschenzug.

\section{Hydrodynamik}
\subsection{Kontinuitätsgleichung}
Man denke sich ein waagerechtes Rohr. Das Volumen an Wasser, welches
auf der einen Seite hereinströmt, muss auf der anderen Seite des
Rohres auch wieder hinausströmen. Allgemeiner wird durch jeden
Querschnitt während der Zeit $\Delta t$ das gleiche Volumen
$\Delta V$ strömen. Wenn die Strömungsgeschwindigkeit zeitlich
konstant ist, dann hat man also%
\[\frac{\Delta V}{\Delta t} = \mathrm{const.}\]
Für sehr kleine Zeiträume ist die Strömungsgeschwindigkeit
näherungsweise konstant. Man erhält im Grenzübergang $\Delta t$
gegen null somit das Kontinuitätsgesetz%
\[V'(t) = \mathrm{const.}\]
Da Wasser inkompressibel ist, ergibt sich über $m=\varrho V$ auch%
\[m'(t) = \mathrm{const.}\]
Volumenstrom und Massenstrom sind also örtlich konstant.
Man setzt nun $\mathrm dV=A\,\mathrm ds$, wobei $A$ die
Querschnittsfläche an einer betrachteten Stelle sein soll.
Es ergibt sich $V'(t) = As'(t) = Ac$. Für zwei verschiedene
Stellen gilt daher $A_1c_1 = A_2c_2$.

Gehen wir nun davon aus, dass die Strömungsgeschwindigkeit $c$
zeitlich konstant ist. Man kann nun den örtlich \textit{und zeitlich}
konstanten Volumenstrom messen. Damit lässt sich dann mit der
Gleichung $V'(t) = Ac$ die Strömungsgeschwindigkeit an einer
beliebigen Stelle des Rohres bestimmen. Bei einer Verengung des
Rohres erhöht sich natürlich auch die Strömungsgeschwindigkeit.
Bei Erweiterung des Querschnitts wird die Strömungsgeschwindigkeit
dementsprechend absinken. Diese Aussagen sind Teilaspekte des
Venturi"=Effekts.

Das Volumen, welches die Querschnittsfläche $A$ durchströmt hat,
kann als Funktion $V(t,x)$ mit zwei Argumenten aufgefasst werden.
Das erste Argument ist die Zeit, die vergangen ist,
seitdem man angefangen hat, das Volumen zu messen, welches durch
$A$ geflossen ist. Das zweite Argument ist der Ort, an welchem
$A$ angeheftet ist. Man kann jetzt schreiben%
\[\frac{\partial}{\partial x}\frac{\partial V}{\partial t}=0.\]
Für die partiellen Ableitungen soll an dieser Stelle die eulersche
Kurzschreibweise eingeführt werden:%
\[D_t := \frac{\partial}{\partial t},\quad
D_x := \frac{\partial}{\partial x}.\]
Nach Vornahme der Substitution $Q:=D_tV$ lautet die
Gleichung nunmehr $D_x Q=0$. Die Lösung dieser partiellen
Differentialgleichung ist die Funktion $Q(t)$, der vom
Zeitpunkt abhängige Volumenstrom.

Man stelle sich nun ein Volumenelement mit den Kantenlängen
$\mathrm dx_1,\mathrm dx_2,\mathrm dx_3$ vor. So viel Volumen, wie
in dieses Volumenelement hineinströmt, muss auch wieder hinausströmen.
Innerhalb eines solch kleinen Quaders ist außerdem der
Geschwindigkeitsvektor $u$ der Strömung örtlich konstant.
Mit der Produktregel ergibt sich%
\begin{gather*}\frac{\mathrm dV}{\mathrm dt}
= \frac{1}{\mathrm dt}
(\mathrm dx_1\wedge\mathrm dx_2\wedge\mathrm dx_3)\\
= u_1 \mathrm dx_2\wedge\mathrm dx_3 +
u_2\mathrm dx_1\wedge\mathrm dx_3\\
+ u_3\mathrm dx_1\wedge\mathrm dx_2.
\end{gather*}
Beim ersten Summand wurde gerechnet
\begin{gather*}
\frac{\mathrm dx_1}{\mathrm dt}\wedge
\mathrm dx_2\wedge\mathrm dx_3\\
= u_1\wedge\mathrm dx_2\wedge\mathrm dx_3
= u_1\mathrm dx_2\wedge\mathrm dx_3.
\end{gather*}
Wir können das als Skalarprodukt des Geschwindigkeitsvektors mit dem
Flächenvektor auffassen. Es ist%
\[\frac{\mathrm dV}{\mathrm dt} = \langle u,\mathrm dA\rangle
= \langle\begin{bmatrix} u_1\\u_2\\ u_3\end{bmatrix},
\begin{bmatrix}\mathrm dx_2\mathrm dx_3\\
\mathrm dx_1\mathrm dx_3\\
\mathrm dx_1\mathrm dx_2
\end{bmatrix}\rangle.\]
Für eine größere Oberfläche ist der Volumenstrom%
\[Q=\int_A \langle u,\mathrm dA\rangle.\]
Für ein Gebiet mit einer geschlossene Oberfläche muss $Q=0$ sein.
Sonst würde es im Gebiet ja Wasserhähne oder Abflüsse geben, was wir
ausschließen wollen. Mit dem Integralsatz von Gauß im Raum erhält man%
\[Q=\int_V \mathrm{div}\,u\,\mathrm dV=0.\]
Das Gebiet kann man nun beliebig klein machen. Als Grenzwert ergibt
sich die Kontinuitätsgleichung $\mathrm{div}\,u=0$.

\subsection{Bernoulli-Gleichung}

Man stelle sich zwei waagerechte Rohrabschnitte vor, zwischen denen
sich ein Steigrohr befindet. Eine inkompressible und reibungsfreie
Flüssigkeit soll durch das Rohr strömen. Ein dünner Volumenabschnitt
wird beim Steigen kinetische Energie verlieren und potenzielle Energie
gewinnen. Die Summe an Energie bleibt aber nach dem Energieerhaltungssatz
konstant. Für einen Volumenabschnitt mit der Masse $m$ gilt also%
\[E=E_\mathrm{pot}+E_\mathrm{kin}
=mgh+\frac{1}{2}mc^2=\mathrm{const}.\]
Nach der Substitution $m=\varrho V$ dividiert
man auf beiden Seiten der Gleichung durch $V$ und erhält%
\[\varrho gh+\frac{1}{2}\varrho c^2=\frac{E}{V}=\mathrm{const}.\]
Diese Gleichung ist jedoch nur gültig, wenn es zu keiner
Verengung des Rohrquerschnittes kommt, da in diesem Fall die
Strömungsgeschwindigkeit $c$ ohne einen Höhenverlust ansteigt.
Woher kommt die Energie für die Beschleunigung?

Die Flüssigkeit betrachtet man dazu mikroskopisch.
Die Flüssigkeitsmoleküle bewegen sich ja ungeordnet und stoßen
fortlaufend ungeordnet aufeinander. Dadurch ergibt sich ein
statischer Druck. Während der Querschnitt beim Strömen abnimmt,
stoßen Moleküle aber öfter gegen die Rohrwand. Da es nur
elastische Stöße gibt, kommt es dabei nicht zu einem Energieverlust.
Die Stöße sind zwar chaotisch, klar ist jedoch, dass die ungeordnete
Bewegung zum Teil in eine geordnete überführt wird. Sonst würde
sich die Strömungsgeschwindigkeit nicht erhöhen.

Die Energie wird also aus dem statischen Druck $p_s$ gesaugt,
welcher somit kleiner werden muss. Die Summe aus statischem
Druck und dynamischen Druck wird jedoch konstant bleiben.

Damit ergibt sich die Bernoulli"=Gleichung%
\[p_s+\varrho gh+\frac{1}{2}\varrho c^2 = \mathrm{const}.\]

\subsection{Torricelli-Ausflussformel}
Man stelle sich ein Zylinderfass der Höhe $h$ vor. Am Boden des
Zylinderfasses befindet sich ein Ablauf mit Ventil. Mit der
Bernoulli"=Gleichung ergibt sich nun%
\[p_s+\varrho gh = p_s+\frac{1}{2}\varrho c^2.\]
Somit erhält man die Torricelli"=Ausflussformel%
\[c^2=2gh.\]
Wir wollen nun wissen, wie lange es dauert, bis das Wasser aus dem
Zylinderfass abgelaufen ist. Das Fass soll die Querschnittsfläche $A_1$
haben. Die Querschnittsfläche des Ablaufes bezeichnen wir mit $A_2$.
Es ist nun $A_1c_1=A_2c$ und $c_1=h'(t)$. Damit ergibt sich%
\[\Big(\frac{A_1}{A_2}\Big)^2 h'(t)^2 = 2gh.\]
Diese Differentialgleichung muss für den Anfangswert $h(0)=h_0$
gelöst werden. Bei dieser DGL handelt es sich um eine autonome
DGL erster Ordnung, und solche sind separabel. Da $h'(t)<0$ ist,
wählt man die negative Wurzel. Es ist also%
\[h'(t) = -\frac{A_2}{A_1} \sqrt{2gh} = -k\sqrt{h}.\]
Damit ergibt sich
\[\int_0^t \frac{h'(t)}{\sqrt{h}} \mathrm dt
= -k\int_0^t \mathrm dt.\]
Anwendung der Substitutionsregel bringt nun%
\[\int_0^t \frac{h'(t)}{\sqrt{h}}\mathrm dt
= \int_{h_0}^h \frac{1}{\sqrt{h}}\mathrm dh=2\sqrt{h}-2\sqrt{h_0}.\]
Die Höhe ist somit
\[h(t) = \bigg[\sqrt{h_0}-\frac{kt}{2}\bigg]^2.\]

\end{document}


