\documentclass[a4paper,10pt,fleqn,twocolumn,twoside,dvipdfmx]{scrartcl}
\usepackage[utf8]{inputenc}
\usepackage[T1]{fontenc}
\usepackage[ngerman]{babel}
\usepackage{microtype}

% Bugfix. Siehe https://komascript.de/node/2295
\unsettoc{toc}{onecolumn}

\usepackage{amsmath}
\usepackage{amssymb}

\usepackage{libertine}
\usepackage[libertine,smallerops]{newtxmath}
% \usepackage[scaled=0.8]{DejaVuSansMono}
% \usepackage[scaled=0.8]{DejaVuSans}
\setkomafont{sectioning}{\normalfont\normalcolor\bfseries}

\usepackage{color}
\definecolor{c1}{RGB}{00,00,00}
\usepackage[colorlinks=true,linkcolor=c1]{hyperref}
\usepackage{geometry}
\geometry{a4paper,left=24mm,right=14mm,top=24mm,bottom=30mm}
\setlength{\columnsep}{5mm}
\numberwithin{equation}{section}

\usepackage{bm}
\newcommand{\bvec}[1]{\mathbf{#1}}
\newcommand{\R}{\mathbb R}

\begin{document}

\thispagestyle{empty}

\begin{huge}
\noindent
\textbf{Analytische Mechanik}
\par
\end{huge}
\vspace{1em}

\tableofcontents

\section{Lagrange"=Formalismus}

\subsection{Bewegungen}

Im Raum $\R^3$ befinden sich $N$ Punktmassen mit Koordinaten
$\bvec x_k$ für $k\in\{1,\ldots,N\}$. Eine Bewegung liegt vor,
wenn jede Punktmasse jeweils eine stetige Funktion der Zeit ist:
\begin{equation}
\bvec x_k\colon T\to\R^3, \quad\text{$T\subseteq\R$ ein Zeitintervall}.
\end{equation}
Die Koordinaten der Punktmassen lassen sich zusammenfassen zu einem
Koordinatenpunkt im $\R^{3N}$, dem \emph{Ortsraum}. Die Bewegung des
Systems von Punktmassen ist also beschrieben durch eine stetige
Funktion
\begin{equation}
\bvec x\colon T\to\R^{3N}.
\end{equation}

\subsection{Zwangsbedingungen}

Eine Einschränkung der Bewegungsfreiheit lässt sich formulieren
als implizite Funktion. Sei dazu $f\colon\R^{3N}\to\R^p$
differenzierbar und $0$ ein regulärer Wert. Dann ist die Lösungsmenge
der Gleichung $f(\bvec x)=0$ eine Untermannigfaltigkeit der Dimension
$3N-p$. Man bezeichnet dies als holonom"=skleronome Zwangsbedingung.

Nun kann eine Zwangsbedingung aber auch zeitabhängig sein. Sei dazu
$f\colon\R^{3N+1}\to\R^p$ differenzierbar und $0$ ein regulärer Wert.
Dann ist die Lösungsmenge $f(t,\bvec x)=0$ eine Untermannigfaltigkeit
des $\R^{3N+1}$ mit der Dimension $3N+1-p$. Man spricht von einer
holonom"=rheonomen Zwangsbedingung.

\subsection{Zwangskräfte}

Die Wahl eines lokalen Koordinatensystem $\Phi\colon U\to\R^{3N}$ mit
$U\subseteq\R^{3N-p}$, so dass $\bvec x := \Phi(q)$ die Gleichung
$f(\bvec x)=0$ löst, nennt man \emph{generalisierte Koordinaten}.

Betrachten wir zunächst den Fall $N=1$ und $p=1$. Nun kann sich die
Punktmasse nicht mehr frei bewegen, sondern ist auf die
Mannigfaltigkeit $M$ eingeschränkt. Befindet man sich innerhalb
der Mannigfaltigkeit und weiß nichts über den umliegenden Raum,
wird die Punktmasse durch eine Kraft $\bvec F$ beschleunigt,
die im Tangentialraum $T_{\bvec x} M$ liegen muss.

Die Kraft wird gemäß $\bvec F = \bvec K+\bvec Z$ in zwei
Anteile zerlegt. Der Anteil $\bvec K$ ist die von außen wirkende
Kraft, auch eingeprägte Kraft genannt. Dies ist die Kraft, wenn keine
Zwangsbedingungen vorhanden wären, z.\,B. die gewöhnliche
Gewichtskraft. Der Anteil $\bvec Z$
ist die Zwangskraft, welche aus zwei Anteilen besteht. Der erste
Anteil von $\bvec Z$ kompensiert den zum Tangentialraum normalen
Anteil von $\bvec K$, denn sonst würde $\bvec F$ nicht im
Tangentialraum liegen. Der zweite Anteil von $\bvec Z$ wirkt
der lokalen Zentrifugalkraft \emph{aus der Mannigfaltigkeit heraus}
entgegen, ist also ein Anteil der lokalen Zentripetalkraft, dieser
steht auch normal auf dem Tangentialraum. Somit steht $\bvec Z$
insgesamt normal auf $T_{\bvec x} M$.

Die Zentrifugalkraft ist von der Geschwindigkeit $\bvec x'(t)$
abhängig, daher auch $\bvec Z$. Das ist problematisch.
Zum Aufstellen von Bewegungsgleichungen muss $\bvec Z$ irgendwie
herausgerechnet werden.

Bekannt ist außerdem, dass der Gradient $\nabla f$ aus
mathematischen Gründen normal auf $T_{\bvec x} M$ stehen muss.
Demnach sind Zwangskraft und Gradient kollinear:%
\begin{equation}
\bvec Z(t) = \lambda(t)\nabla f(\bvec x(t)).
\end{equation}
Allgemein kann $\lambda$ bei skleronomen Zwangsbedingungen
auch dargestellt werden als Funktion
$\lambda(\bvec x(t),\bvec x'(t))$.

Weil $\bvec Z(t)$ rechtwinklig zu $\bvec x'(t)\in T_{\bvec x(t)} M$
steht, ist
\begin{equation}
\langle\bvec Z(t),\bvec x'(t)\rangle = 0.
\end{equation}
Die Zwangskraft kann daher keine Arbeit verrichten.

Weil innerhalb von $M$ nur die tangentiale Kraft einen Einfluss auf
die Bewegung haben kann, ergibt sich die Bewegungsgleichung
\begin{equation}
m\bvec x''(t) = \bvec F = \bvec K+\bvec Z = \bvec K+\lambda\nabla f.
\end{equation}
Diese Gleichung wird \emph{Lagrange"=Gleichung erster Art} genannt.

\begin{thebibliography}{00}
\bibitem{Fischer} Helmut Fischer, Helmut Kaul:
\emph{Mathematik für Physiker Band 3}. Springer, Wiesbaden 2003, 3. Aufl. 2013.
\bibitem{Arnold} Vladimir I. Arnold: \emph{Mathematical Methods of Classical
Mechanics}. Springer, New York 1978, 2. Aufl. 1989.
\end{thebibliography}

\vfill\noindent
Dieser Text steht unter der Lizenz\\
Creative Commons CC0.

\end{document}


