\documentclass[a4paper,10pt,fleqn,twocolumn,twoside,dvipdfmx]{scrartcl}
\usepackage[utf8]{inputenc}
\usepackage[T1]{fontenc}
\usepackage[ngerman]{babel}
\usepackage{microtype}

% Bugfix. Siehe https://komascript.de/node/2295
\unsettoc{toc}{onecolumn}

\usepackage{amsmath}
\usepackage{amssymb}

\usepackage{libertine}
\usepackage[libertine,smallerops]{newtxmath}
% \usepackage[scaled=0.8]{DejaVuSansMono}
% \usepackage[scaled=0.8]{DejaVuSans}
\setkomafont{sectioning}{\normalfont\normalcolor\bfseries}

\usepackage{color}
\definecolor{c1}{RGB}{00,00,00}
\usepackage[colorlinks=true,linkcolor=c1]{hyperref}
\usepackage{geometry}
\geometry{a4paper,left=24mm,right=14mm,top=24mm,bottom=30mm}
\setlength{\columnsep}{5mm}
\numberwithin{equation}{section}

\usepackage{bm}
\newcommand{\bv}[1]{\mathbf{#1}}
\newcommand{\R}{\mathbb R}

\begin{document}

\thispagestyle{empty}

\begin{huge}
\noindent
\textbf{Analytische Mechanik}
\par
\end{huge}
\vspace{1em}

\tableofcontents

\section{Lagrange"=Formalismus}

\subsection{Bewegungen}

Im Raum $\R^3$ seien $N$ Punktmassen mit Koordinaten
$\bv x_k$ für $k\in\{1,\ldots,N\}$ befindlich. Eine \emph{Bewegung}
liegt vor, wenn jede Punktmasse jeweils eine stetige Funktion der Zeit ist:%
\[\bv x_k\colon T\to\R^3, \quad\text{$T\subseteq\R$ ein Zeitintervall}.\]
Die Koordinaten der Punktmassen lassen sich zusammenfassen zu einem
Koordinatenpunkt im $\R^{3N}$, dem \emph{Ortsraum}. Die Bewegung des
Systems von Punktmassen ist also beschrieben durch eine stetige
Funktion
\[\bv x\colon T\to\R^{3N}.\]
Bei Bewegungen wird meist die Differentialrechnung eine Rolle spielen,
weshalb man glatte oder zumindest hinreichend oft differenzierbare
Bewegungen betrachten wird.

\subsection{Zwangsbedingungen}

Kurze Rekapitulation. Sei $f\colon\R^n\to\R^m$ eine differenzierbare
Abbildung. Man nennt $y\in\R^m$ einen regulären Wert von $f$, wenn
$\mathrm df(x)$ für jedes $x\in f^{-1}(\{y\})$ surjektiv ist, oder
äquivalent, wenn $\mathrm df$ den konstanten Rang $m$ hat.
Unter diesem Umstand ist $M:=f^{-1}(\{y\})$ eine Untermannigfaltigkeit
von $\R^n$ mit der Dimension $\dim M = n-m$. Man bezeichnet die
Einschränkung $f|M$ als Submersion. Außerdem gilt
$T_x M = \operatorname{Kern}(\mathrm df(x))$, wobei mit $T_x M$
der Tangentialraum von $M$ am Punkt $x$ gemein ist.

Eine Einschränkung der Bewegungsfreiheit lässt sich
als implizite Funktion formulieren. Sei dazu $f\colon\R^{3N}\to\R^p$
differenzierbar und 0 ein regulärer Wert von $f$. Dann ist die Lösungsmenge
der Gleichung $f(\bv x)=0$ eine Untermannigfaltigkeit der Dimension
$3N-p$. Man bezeichnet dies als holonom"=skleronome Zwangsbedingung.

Nun kann eine Zwangsbedingung aber auch zeitabhängig sein. Sei dazu
\[f\colon\R\times\R^{3N}\to\R^p\]
differenzierbar und sei $0$ zu jedem festen Parameter $t$ ein
regulärer Wert von $\bv x\mapsto f(t,\bv x)$. Dann ist die
Lösungsmenge $f(t,\bv x)=0$ zu jedem $t$ eine Untermannigfaltigkeit
des $\R^{3N}$ mit der Dimension $3N-p$. Man spricht von einer
holonom"=rheonomen Zwangsbedingung.

\subsection{Generalisierte Koordinaten}

Eine solche als Lösungsmenge einer holonomen Zwangsbedingung $f(t,\bv x)=0$
beschriebene Untermannigfaltigkeit $M$ bezeichnet man als
\emph{Konfigurationsmannigfaltigkeit}.

Orte auf $M$ lassen sich durch ein lokales Koordinatensystems
$\Phi\colon U\to M$ mit $U\subseteq\R^S$ mit $S=3N-p$ beschreiben, so
dass $\bv x = \Phi(q)$ für jedes $q\in U$ die Gleichung $f(t,\bv x)=0$
löst. Man bezeichnet $q=(q_1,\ldots,q_S)$ als \emph{generalisierte Koordinaten}.

Die Elemente aus $T_q U\cong\R^S$ bezeichnet man als
\emph{virtuelle Verrückungen}. Die Tangentialvektoren aus
$T_{\bv x} M$ bezüglich $\bv x = \Phi(q)$ tragen ebenfalls diese
Bezeichnung, da der lineare Isomorphismus%
\[\mathrm d\Phi(q)\colon T_q U\to T_{\bv x} M\]
besteht.

\subsection{Zwangskräfte}

Betrachten wir zunächst den Fall $N=1$ und $p=1$. Nun kann sich die
Punktmasse nicht mehr frei bewegen, sondern ist auf die
Mannigfaltigkeit $M$ eingeschränkt. Die Bewegung wollen wir aber
weiterhin mit dem zweiten newtonschen Gesetz als
\[m\bv x''(t) = \bv F\]
beschrieben wissen. Wäre bei Abhandensein äußerer Kräfte $\bv F=0$,
resultiert die Anfangswertaufgabe in einer geradlinigen Bewegung, was
aber bei gekrümmtem $M$ absurd ist, da die Bewegung innerhalb $M$
verlaufen soll. Ergo muss eine weitere Kraft
existieren, die wir als \emph{Zwangskraft} bezeichnen. Die Kraft
$\bv F$ wird gemäß $\bv F = \bv K+\bv Z$ in zwei Anteile zerlegt.
Der Anteil $\bv K$ ist die von außen wirkende Kraft, auch
\emph{eingeprägte Kraft} genannt. Dies ist die bei Abhandensein
von Zwangsbedingungen bestehende Kraft, z.\,B. die gewöhnliche
Gewichtskraft. Der Anteil $\bv Z$ ist die Zwangskraft.

Betrachten wir die Situation $\bv K = 0$ unter einer skleronomen
Zwangsbedingung genauer, in der sich die
Punktmasse mit konstanter Geschwindigkeit bewegt. Die Konstanz
der Geschwindigkeit $|\bv x'|$ ist gleichbedeutend mit der Konstanz
der kinetischen Energie $T = \frac{1}{2}m|\bv x'|^2$. Infolge gilt
\[0=\frac{\mathrm dT}{\mathrm dt}
= \frac{1}{2}m\frac{\mathrm d}{\mathrm dt}\langle\bv x',\bv x'\rangle
= \langle m\bv x'',\bv x'\rangle = \langle\bv Z,\bv x'\rangle.\]
Es steht also $\bv Z$ normal auf $\bv x'$.

Die Betrachtung ist zudem auch für ein allgemeines $\bv K$
unter einer skleronomen Zwangsbedingung durchführbar.
Mit $m\bv x''=\bv K+\bv Z$ findet sich
\[\frac{\mathrm dT}{\mathrm dt} = \langle\bv K+\bv Z,\bv x'\rangle
= \langle\bv K,\bv x'\rangle + \langle\bv Z,\bv x'\rangle.\]
Unsere Auffassung von der Zwangskraft ist die, dass sie allein
dafür da ist, die Punktmasse in $M$ zu halten, aber keine
Änderung der kinetischen Energie bewirkt. Ergo muss
$\langle\bv Z,\bv x'\rangle = 0$ gelten. Es steht also $\bv Z$
ganz allgemein normal auf $\bv x'$. Anders ausgedrückt darf die
Zwangskraft keine Arbeit verrichten. Als Nebenresultat ergibt sich die
Beziehung
$\frac{\mathrm dT}{\mathrm dt} = \langle\bv K,\bv x'\rangle$.

Wir Postulieren an dieser Stelle, dass $\bv Z$ allgemein keinen
tangentialen Einfluss haben darf, auch bei rheonomen Zwangsbedingungen.
Demzufolge steht $\bv Z$ an jedem Punkt $\bv x$ rechtwinklig auf
$T_{\bv x} M$.

Weil $M$ als Niveaumenge definiert wurde, auf der $f$ konstant ist,
verschwindet die Richtungsableitung am Punkt $\bv x$ für jeden
Tangentialvektor aus $T_{\bv x} M$. Ergo steht der räumliche Gradient
$\nabla f(t,\bv x)$ im rechten Winkel zum Tangentialraum. Mit räumlich
ist hierbei gemeint, dass der Gradient nicht die Zeitableitung enthalten
soll.

\subsection{Lagrange-Gleichung erster Art}

Demnach sind Zwangskraft und Gradient kollinear, womit zu jedem
Zeitpunkt $t$ ein Skalar $\lambda$ existiert, so dass
\[\bv Z(t) = \lambda\nabla f(t,\bv x(t))\]
gilt. Es findet sich die Bewegungsgleichung
\[m\bv x'' = \bv F = \bv K+\bv Z = \bv K+\lambda\nabla f,\]
die \emph{Lagrange"=Gleichung erster Art} genannt wird. Man stellt
$\lambda$ als Funktion $\lambda(t,\bv x(t),\bv x'(t))$ dar, da es
ein Teilterm einer Dgl. zweiter Ordnung ist.

\newpage
\begin{thebibliography}{00}
\bibitem{Fischer} Helmut Fischer, Helmut Kaul:
\emph{Mathematik für Physiker Band 3}. Springer, Wiesbaden 2003, 3. Aufl. 2013.

\bibitem{Arnold} Vladimir I. Arnold: \emph{Mathematical Methods of Classical
Mechanics}. Springer, New York 1978, 2. Aufl. 1989.

\bibitem{Bartelmann} Matthias Bartelmann u. a.:
\emph{Theoretische Physik}. Springer, Berlin \& Heidelberg 2015.

\bibitem{Lee} John M. Lee: \emph{Introduction to Smooth Manifolds}.
Springer, New York 2003, 2. Auflage 2013.

\bibitem{Jaenich} Klaus Jänich: \emph{Mathematik 2: Geschrieben für
Physiker}. Springer, Heidelberg 2002, 2. Auflage 2011.
\end{thebibliography}

\vfill\noindent
Dieser Text steht unter der Lizenz\\
Creative Commons CC0.

\end{document}


