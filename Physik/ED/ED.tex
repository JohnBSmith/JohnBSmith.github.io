\documentclass[a4paper,11pt,fleqn,twocolumn,twoside,dvipdfmx]{scrartcl}
\usepackage[utf8]{inputenc}
\usepackage[T1]{fontenc}
\usepackage[ngerman]{babel}
\usepackage{microtype}
\usepackage{libertine}
\usepackage[libertine,smallerops]{newtxmath}
\addtokomafont{disposition}{\rmfamily}

\usepackage{amsmath}
\usepackage{amssymb}
\usepackage{color}
\definecolor{c1}{RGB}{00,40,60}
\usepackage[colorlinks=true,linkcolor=c1]{hyperref}
\usepackage{geometry}
\geometry{a4paper,left=25mm,right=10mm,top=24mm,bottom=30mm}
\setlength{\columnsep}{4.8mm}

\newcommand{\ui}{\mathrm i}

\begin{document}
\thispagestyle{empty}

\begin{huge}
\noindent
\textbf{Elektrodynamik}
\par
\end{huge}


\tableofcontents
%\newpage

\section{Elektrostatisches Feld}
\subsection{Coulombsches Gesetz}

Man denke sich den dreidimensionalen euklidischen Punktraum.
Weiterhin wird im Modell gefordert, dass jeder Punkt von Vakuum oder
Luft umgeben ist, ohne näher zu beschreiben, was damit gemeint sein
soll.

Legt man zwei Punktladungen in den Raum, so wirkt auf eine der
Ladungen gemäß dem coulombschen Gesetz eine Kraft.
Der Betrag der Kraft ist
\begin{equation}\label{Coulomb}
|F|=\frac{1}{4\pi\varepsilon_0}\frac{|q_1q_2|}{r^2}.
\end{equation}
und die Wirkung ist in Richtung der jeweils anderen Punktladung,
falls beide ein unterschiedliches Vorzeichen haben. Bei gleichem
Vorzeichen zeigt die Kraft genau in die entgegengesetzte Richtung.
Hierbei sind $q_1,q_2$ die Werte der Punktladungen, $r$ der
Abstand beider und $\varepsilon_0$ eine experimentell bestimmbare
Konstante. Dieses experimentell überprüfbare Gesetz soll unsere
axiomatische Grundlage für alle weiteren Betrachtungen bilden.

Nach Newton wirken auf beide Punktladungen Kräfte, die gleich groß
und genau entgegengesetzt sind. Das heißt es gilt $F_1=-F_2$.

Mit $x_1,x_2$ sollen nun die beiden Punkte bezeichnet werden,
bei denen die Punktladungen liegen. Nach der Wahl eines
Koordinatenursprungs und einer Orthonormalbasis können die
Punkte auch mit den Koordinatentupeln $x_k=(x_{k1},x_{k2},x_{k3})$
identifiziert werden.

Sei $r_{ij}:=x_i-x_j$. Bei $r_{ij}$ handelt es sich um ein
Element des euklidischen Vektorraumes, und von einem solchen lässt
sich der Betrag bilden. Das gibt Motivation zu folgender
Definition: sei $\hat r_{ij}:=\frac{r_{ij}}{|r_{ij}|}$.
Außerdem ist $r=|r_{12}|=|r_{21}|$.

Bei $r_{21}$ handelt es sich nun um den Verschiebungsvektor,
welcher von $q_1$ aus nach $q_2$ verschiebt. Entsprechend
ist $\hat r_{21}$ der Richtungsvektor, welcher von $q_1$ aus
nach $q_2$ zeigt. Kraft ist nun das Produkt aus Richtung und
Betrag. Daher gilt
\[F_1=-\mathrm{sgn}(q_1q_2)\,\hat r_{21}|F_1|.\]
Rechnet man $\mathrm{sgn}(q_1q_2)\,|q_1q_2|=q_1q_2$ so ergibt sich
also
\[F_1=-\frac{q_1q_2}{4\pi\varepsilon_0}\frac{\hat r_{21}}{r^2}
=-\frac{q_1q_2}{4\pi\varepsilon_0}\frac{r_{21}}{|r_{21}|^3}.\]
Bei mehreren Ladungen werden sich die Kräfte überlagern. Gibt es
also mehrere Ladungen $q_1$ bis $q_n$, so ergibt sich
\[F_j = \sum_{i\ne j} F_{ij}.\]
wobei $F_{ij}$ die Kraft ist, welche von $q_j$ aus auf
$q_i$ zeigt. Damit ergibt sich
\[F_j = -\sum_{i\ne j} \frac{q_iq_j}{4\pi\varepsilon_0}
\frac{r_{ij}}{|r_{ij}|^3}.\]
Alle Faktoren, die nicht vom Laufindex $i$ abhängig sind, lassen
sich aus der Summe herausbringen. Man erhält
\[F_j = -\frac{q_j}{4\pi\varepsilon_0}
\sum_{i\ne j} q_i\frac{r_{ij}}{|r_{ij}|^3}.\]
Nun hängt der Rest in der Summe aber nicht von $q_j$ ab.
Das gibt Motivation zur Definition der \emph{elektrischen Feldstärke}.
Ist $q$ eine Probepunktladung und $F$ die Kraft auf diese Ladung,
so gilt
\[E := \frac{F}{q}.\]
Jeder Punkt $x$ im Punktraum erhält damit eine elektrische
Feldstärke $E(x)$. Dieses Vektorfeld bezeichnen wir als das
\emph{elektrostatische Feld}.

Für eine diskrete Verteilung von Punktladungen gilt demnach
\begin{equation}\label{eq:ESumme}
E(x) = \frac{1}{4\pi\varepsilon_0}
\sum_{k=1}^n q_k\frac{x-x_k}{|x-x_k|^3}.
\end{equation}
Diese Formel ist mathematisch praktischer, aber immer noch äquivalent
zum coulombschen Gesetz. Wenn $E(x)$ berechnet ist, so ist
damit auch $F=qE(x)$ bekannt. Mit $E(x)$ als Ausgangspunkt
soll nun die mathematische Struktur des elektrostatischen Feldes
untersucht werden. Dazu beschränken wir uns zur Vereinfachung
zunächst auf eine einzige Punktladung $q_0$. Demnach gilt nun
\begin{equation}\label{eq:E-Feld}
E(x) = \frac{q_0}{4\pi\varepsilon_0}\frac{(x-x_0)}{|x-x_0|^3}.
\end{equation}
Erkenntnisbringend ist nun die Feststellung, dass $E(x)$ --
dem Anschein der Formel nach -- an allen Stellen außer $x=x_0$
differenzierbar ist. D.\,h. alle partiellen Ableitungen sind
für $x\ne x_0$ stetig.

Somit können wir $E(x)$ mit Hilfe von Differentialoperatoren
untersuchen. Wir wählen nun ein orthonormales Koordinatensystem
mit dem Ursprung bei $x_0$. Somit ist $x_0=0$. Die Rotation
ist
\begin{gather*}
\nabla\wedge E(x)
= \frac{q_0}{4\pi\varepsilon_0}\nabla\wedge\frac{x}{|x|^3}\\
= \sum_{i<j} \Big(D_i\frac{x_j}{|x|^3}-D_j\frac{x_i}{|x|^3}\Big)
e_i\wedge e_j.
\end{gather*}
Mit $\nabla$ ist der Nabla-Operator gemeint, mit $a\wedge b$
das äußere Produkt und $D_k$ ist die partielle Ableitung nach
Variable Nr. $k$. Nun gilt mit der Produktregel
\[D_i \frac{x_j}{|x|^3}
= \frac{1}{|x|^3}D_i x_j+x_jD_i\frac{1}{|x|^3}.\]
Wegen $i\ne j$ verschwindet der erste Summand aber.
Für den zweiten Summand gilt
\[D_i\frac{1}{|x|^3} = -\frac{3}{|x|^4}D_i|x|\]
und
\[D_i|x| = \frac{1}{2|x|}D_i\langle x,x\rangle
= \frac{1}{2|x|} 2x_i.\]
Damit ergibt sich insgesamt
\[D_i \frac{x_j}{|x|^3}
= -\frac{3x_ix_j}{|x|^5}.\]
Der Ausdruck ist nun aber symmetrisch bezüglich Vertauschung
von $i$ und $j$. Daher verschwindet die Differenz
$D_i\frac{x_j}{|x|^3}-D_j\frac{x_i}{|x|^3}$
und es gilt
\begin{equation}
\nabla\wedge E(x)=0.\qquad (x\ne 0)
\end{equation}
Das elektrische Feld einer Punktladung ist also rotationsfrei
und damit nach dem Poincaré"=Lemma ein Potentialfeld, sofern es
denn einen einfach zusammenhängenden Definitionsbereich besitzt. Dieses
Resultat ist wesentlich, denn es ermöglicht die Applikation
weitreichender mathematischer Hilfsmittel aus der Potentialtheorie.
Für mehrere Punktladungen ist $E(x)$ eine Überlagerung der einzelnen
Felder. Es gilt
\[E(x) = \sum_{i=1}^n E_i(x).\]
Somit gilt an allen Stellen außer den Quellen:
\[\nabla\wedge E(x) = \sum_{i=1}^n \nabla\wedge E_i(x)
= \sum_{i=1}^n 0 = 0.\]
Das elektrostatische Feld ist also sogar immer ein
Potentialfeld. Als nächste mathematische Abstraktion wollen wir
also nicht mehr mit dem elektrischen Vektorfeld, sondern mit
dem zugrundeliegenden skalaren Potential $\varphi$ arbeiten.
Wir definieren $\varphi(x)$ implizit, indem wir verlangen, dass die
Gleichung
\begin{equation}\label{eq:phi-implizit}
E(x)=-\nabla\varphi(x)
\end{equation}
gültig ist. Daher lässt sich $\varphi$ vertikal verschieben, da
jeder konstante Summand beim bilden des Gradienten entfällt.
Man wählt das konventionell so, dass
\[\lim_{|x|\rightarrow\infty}\varphi(x)=0\]
gilt. Man hat dann einen Potentialtrichter bzw.
ein Potentialgewölbe mit Nullniveau im unendlichen.

Wir brauchen nun eine Funktion dessen Ableitung
$x/|x|^3$ ist. Das eindimensionale Analogon
ist $1/x^2$ und davon ist $-1/x$ eine Stammfunktion.
Daher kommt man auf die Idee zunächst den Ansatz
$-1/|x|$ zu probieren. In der Tat ist
unter Verwendung der Kettenregel
\begin{gather*}
\nabla\frac{1}{|x|} = -\frac{1}{|x|^2}\nabla |x|
= -\frac{1}{|x|^2}\frac{1}{2|x|}\nabla\langle x,x\rangle\\
= -\frac{1}{|x|^2}\frac{1}{2|x|}2x
= -\frac{x}{|x|^3}.
\end{gather*}
Das Feld um eine Punktladung $q_0$ ist damit
\begin{equation}\label{eq:Potential-Raum}
\varphi(x) = \frac{q_0}{4\pi\varepsilon_0|x|}.
\end{equation}
Wenn die Ladung nicht im Koordinatenursprung liegt, so ergibt
sich allgemein
\[\varphi(x) = \frac{q_0}{4\pi\varepsilon_0|x-x_0|}.\]
Mehrere Potentialfelder überlagern sich, da man summandenweise
ableiten kann und dabei das elektrische Feld erhält.
Damit ist
\[\varphi(x) = \frac{1}{4\pi\varepsilon_0}
\sum_{k=1}^n \frac{q_k}{|x-x_k|}.\]
Die Isoflächen (Niveauflächen)
des Potentials werden \emph{Äquipotentialflächen} genannt.
Beschränkt man sich auf die Ebene, so hat man Isolinien die
entsprechend \emph{Äquipotentiallinien} genannt werden.

Für das elektrostatische Feld ergeben sich Feldlinien, so dass
die elektrische Feldstärke $E(x)$ der Tangentialvektor
einer Feldlinie am Punkt $x$ ist. Wenn die Feldlinie als
Ortskurve $x(s)$ gegeben ist, gilt also
\[x'(s)=E(x(s)).\]
Sei $\hat E:=\frac{E}{|E|}$. Wenn $s$ eine Parametrisierung nach
der Bogenlänge sein soll, dann muss $|x'(s)|=1$ für alle $s$ sein.
Man setzt dann also
\[x'(s)=\hat E(x(s)).\]
Die Feldlinien stehen überall senkrecht zu den Äquipotentialflächen,
da $E(x)$ proportional zum Gradienten von $\varphi(x)$ ist.

Zur Bestimmung der Divergenz des elektrostatischen Feldes macht man
nun folgende Rechnung:
\begin{gather*}
\langle\nabla,\frac{x}{|x|^3}\rangle
= \sum_{k=1}^n D_k\frac{x_k}{|x|^3}
= \sum_{k=1}^n \bigg(\frac{1}{|x|^3}+x_kD_k\frac{1}{|x|^3}\bigg)\\
= \sum_{k=1}^n \bigg(\frac{1}{|x|^3}-3\frac{x_k^2}{|x|^5}\bigg)
= \frac{n}{|x|^3}-3\frac{|x|^2}{|x|^5}
= \frac{n-3}{|x|^3}.
\end{gather*}
Für $n=3$ ergibt sich nun
\begin{equation}\label{eq:quellenfrei}
\langle\nabla,E(x)\rangle = 0.\qquad (x\ne 0)
\end{equation}
Das elektrische Feld im Raum ist also quellenfrei.
Die Bedingung $n=3$ impliziert interessanterweise, dass elektrische
Felder in der Ebene entweder nicht quellenfrei sind, oder dass
die Gleichungen \eqref{eq:E-Feld} und \eqref{eq:Potential-Raum} für
ebene Felder nicht stimmen. Da das erstgenannte absurd erscheint,
müsste letzteres der Fall sein.

Ebene Felder erscheinen für den Leser zunächst schlecht realisierbar,
treten aber tatsächlich z.\,B. im dünnen Elektrolyt auf einem
Isolator und bei Querschnitten von parallelen Linienladungen auf.
Prinzipiell sind auch Felder auf einer gekrümmten Oberfläche
denkbar. Man könnte z.\,B. eine dünne Schicht leitfähiges Gel
auf einen Torus schmieren. Hierzu müsste die Laplace-Gleichung
\eqref{eq:Laplace-Gleichung} für ein gekrümmtes Koordinatensystem
auf dem Torus betrachtet werden.

Man betrachte zunächst das Potential.
Als Abkürzung definiert man nun den sogenannten \emph{Laplace-Operator}
\begin{equation}
\Delta\varphi := \langle\nabla,\nabla\varphi\rangle.
\end{equation}
Gleichung
\eqref{eq:phi-implizit} in Verbindung mit Gleichung
\eqref{eq:quellenfrei} bringt die sogenannte \emph{Laplace-Gleichung}
\begin{equation}\label{eq:Laplace-Gleichung}
\Delta\varphi = 0.
\end{equation}
Diese partielle Differentialgleichung wird für alle weiteren
Betrachtungen der Elektrostatik wesentlich sein.

Man betrachte die Laplace-Gleichung in der Ebene.
Speziell in dieser zweiten Dimension ist es nun möglich,
Mittel der Funktionentheorie zu verwenden,
was sich als die virtuoseste Beschreibung von ebenen Feldern
herausstellen wird.

Die elektrische Feldstärke soll ab jetzt allgemeiner durch
\eqref{eq:phi-implizit} definiert sein.

Aus der Laplace-Gleichung ergibt sich nun umgekehrt, dass $E(x,y)$
ein quellenfreies Feld ist. Weiterhin lässt sich die
informelle Rechnung
\begin{equation}
\nabla\wedge E
= -\nabla\wedge\nabla\varphi
= -(\underbrace{\nabla\wedge\nabla}_{=0})\varphi =0
\end{equation}
machen. Diese lässt sich mathematisch verifizieren und bedeutet,
dass $E$ ein Potentialfeld sein muss, also rotationsfrei ist.
Quellen- und Rotationsfreiheit zusammen ist nun aber
äquivalent zu den Cauchy-Riemannschen Differentialgleichungen
(CRDG). Sei dazu
\begin{equation}
z:=x+\ui y=(x,y)
\end{equation}
und
\begin{equation}
f(z)=f(x,y):=E_x(x,y)-\mathrm{i}E_y(x,y).
\end{equation}
Sei nun $u:=E_x$ und $v:=-E_y$.

Quellenfreiheit:
\begin{equation}
D_x E_x+D_y E_y=0\implies D_x u=D_y v.
\end{equation}

Rotationsfreiheit:
\begin{equation}
\begin{split}
&(D_x E_y-D_y E_x)(e_x\wedge e_y)=0\\
&\implies D_x E_y-D_y E_x=0\\
&\implies D_x v = -D_y u.
\end{split}
\end{equation}
Somit ist $f(z)$ eine holomorphe Funktion.

Weiterhin ist $\varphi$ eine harmonische Funktion, da
$\varphi$ also Lösung der Laplace-Gleichung
$\Delta\varphi=0$ ist. Denkt man sich die ursächlichen Punktladungen
weg, so kann man $\varphi$ als Lösung eines Randwertproblems
erhalten. Dabei wird Information über $\varphi$ auf dem Rand
des Teilraumes vorausgesetzt um $\varphi$ insgesamt zu bestimmen.


\subsection{Die Energiedichte}

Um Ladungen aus dem unendlichen zusammen zu bringen, muss man Arbeit
aufwenden. Daher speichert eine Konstellation von Punktladungen
Energie. Aber prinzipiell könnte man eine Punktladung als zwei
Punktladungen betrachten, die sehr dicht beieinander sind.
Dort steckt noch Energie drin die prinzipiell beliebig groß ist.
Das ist natürlich problematisch. Stattdessen wollen wir uns
vorstellen, die Energie ist überall im elektrostatischen Feld
gespeichert. Dazu machen wir folgenden Ansatz.

In einem kleinen Volumenelement ist das elektrostatische Feld
homogen. Man stelle sich nun vor, die Ladungen wurden aus
dem Vakuum erzeugt (da sich positive und negative Ladungen ja
aufheben). Dann werden die Ladungen getrennt und mit diesen
zwei Kondensatorplatten in einem kleinen Abstand $d$ geladen.
Dazu bedarf es der Energie
\[W=\frac{1}{2}CU^2.\]
Für den Plattenkondensator gilt nun
$C=\varepsilon_0 A/d$. Zusammen mit $U=Ed$ ergibt sich dann
\[W=\frac{1}{2}\varepsilon_0 E^2\frac{A}d.\]
Mit $V=Ad$ erhält man die Energiedichte
\[w:=\frac{W}{V} = \frac{1}{2}\varepsilon_0 E^2.\]
Betrachtet man das Feld nun allgemein, so ist jedem
Punkt $x$ die Energiedichte
\[w=\frac{1}{2}\varepsilon_0|E(x)|^2
=\frac{1}{2}\varepsilon_0|\nabla\varphi(x)|^2\]
zugeordnet. Ein stabiles elektrostatisches Feld sollte ein Minimum
an Energie haben. Das heißt, es sollte
\[W = \int_V w\,\mathrm dV = \mathrm{min}\]
sein. Damit kann die Energiedichte als Lagrangedichte aufgefasst
werden. Sei $D[a]:=\frac{\partial}{\partial a}$ und $D_k:=D[x_k]$.
Für eine stationäre Energiedichte gilt
\[D[\varphi]w-\sum_{k=1}^3 D_k D[D_k\varphi]w=0.\]
Setzt man für $w$ ein, so ergibt sich
\begin{gather*}
D[D_k\varphi]w=\frac{1}{2}\varepsilon_0 D[D_k\varphi]|\nabla\varphi|^2\\
= \frac{1}{2}\varepsilon_0 D[D_k\varphi]\sum_{i=1}^3 (D_i\varphi)^2
= \varepsilon_0 D_k\varphi.
\end{gather*}
Weiterhin ist $D[\varphi]w=0$.
Damit ergibt sich
\[-\varepsilon_0(D_1^2\varphi+D_2^2\varphi+D_3^2\varphi)=0.\]
Nach dem Kürzen und Zusammenfassen ergibt sich die Laplace-Gleichung
\[\Delta\varphi=0.\]

\subsection{Verpflanzung}

Kommen wir zurück auf ebene Felder in einem ladungsfreien Gebiet
$G$ der Ebene. Den Rand von $G$ wollen wir mit $R(G)$ bezeichnen.
Solche Felder sind über $f(z)=E_x(z)-\ui E_y(z)$ durch holomorphe
Funktionen beschreibbar. Aus praktischen Gründen soll aber
die Darstellung $f(z)=E_x(z)+\ui E_y(z)$ gewählt werden.

Da das Addieren von Komponenten dem Addieren von Realteilen und
Imaginärteilen entspricht, überträgt sich das Superpositionsprinzip
sofort auf die Darstellung als komplexe Funktion.

Meist ist das Feld $f(z)$ am Rand $R(G)$. bekannt.
Man denke sich z.\,B. ein Metallstück. Dort laufen die Feldlinien
rechtwinklig rein, und das Potential kann man (in Bezug zu einem
beliebig gewählten Nullpotential) messen. Das Potential ist dabei
homogen für das gesamte Metallstück.

Ist das Gebiet nun einfach zusammenhängend und das Feld auf dem
Rand bekannt, so ist das Feld damit auf dem gesamten Gebiet bekannt.
Nehmen wir nun an, dass diese Umformung des Randes $T(z)$ bijektiv
und holomorph ist. Die Verkettung von zwei
holomorphen Funktionen ist nun auch holomorph. Nach der
Transformation ist das Randfeld auf den neuen Rand
$T(R(G))$ verlegt. Und damit ist das Feld auch im gesamten neuen
Gebiet $T(G)$ bekannt.

Damit lässt sich für ein Feld $f(z)$ die Zerlegung
\[f(z) = g(T(z))\]
formulieren. Damit lassen sich aus einem bekannten Feld $g(z)$
mittels Transformationen $T(z)$ viele neue Felder $f(z)$ bauen.

Die umgekehrte Aufgabe ist schwieriger. Falls $f(z)$ auf dem
Rand $T(R(G))$ bekannt ist, so muss ein $g$ und eine passende
Transformation $T$ gefunden werden. Als Rand kann und wird man eine
Äquipotentiallinie wählen. Folgendes bleibt aber schwierig: Der
Feldstärkevektor steht zwar senkrecht zur Äquipotentiallinie,
aber sein Betrag muss beim durchlaufen der Äquipotentiallinie
nicht konstant bleiben.

Holomorphe Funktionen sind zur allgemeinen Beschreibung von
elektrostatischen Feldern noch zu speziell. Das Feld $f(z)$ hat
an dem Ort, wo eine Punktladung angeheftet ist, ja eine Polstelle.
Aber das ist genau die Eigenschaft einer meromorphen Funktion. Solche
sind überall dort holomorph, wo sie keine Polstelle haben.


\subsection{Komplexes Potential}

Da das elektrische Potenzial $\varphi(x,y)$ eine harmonische
Funktion ist, kann es als Realteil einer komplexen Funktion $f(z)$
betrachtet werden, die dann als komplexes Potential bezeichnet wird.
Wir machen also den Ansatz
\[f(z) = \varphi(z)+\ui\psi(z).\]
Aus der Theorie der harmonischen Funktionen weiß man, dass die
Funktion $\psi(z)$ ebenfalls ein Potential und bis auf eine
Konstante eindeutig ist. Wir nennen $\psi(z)$ das konjugiert
harmonische Potential. Außerdem stehen die Linien $\varphi(z)=\mathrm{const}$
und $\psi(z)=\mathrm{const}$, falls sie sich treffen,
rechtwinklig aufeinander.

Für komplexe Funktionen allgemein gilt die Gleichung
\[f'(z) = D_x\varphi(z)+D_x\psi(z).\]
Mit den Cauchy-Riemannschen Gleichungen erhält man
\begin{gather*}
-E(z) = \nabla\varphi = D_x\varphi+\ui D_y\varphi\\
= D_x\varphi-\ui D_x\psi = \overline{f'(z)}.
\end{gather*}
Was bringt uns diese Beschreibung nun? Es ist so, dass
$\psi(z)=\mathrm{const}$ die Feldlinien beschreibt. D.h. die
Äquipotentiallinien des konjugiert harmonischen Potentials sind
die Feldlinien.

Mit folgender Technik lassen sich Feldlinienbilder mit wenig
Berechnungsaufwand visualisieren: Die Umkehrfunktion von $f(z)$
formt das kartesische Koordinatensystem mit waagerechten Feldlinien
und senkrechten Äquipotentiallinien in das Feldlinienbild
von $f(z)$ um.

Was man nun leider noch sagen muss, ist, dass sich elektrostatische
Felder in der Ebene von denen im Raum unterscheiden. Macht man
einen Schnitt im Raum durch die Punktladung, so wird sich das
Potential der Punktladung
in der Schnittebene vom Potential einer Punktladung in der Ebene
unterscheiden. Das Analogon im Raum zur Punktladung in der Ebene ist
vielmehr das Feld einer unendlich langen Linienladung.

Unter anderem erfüllt das Feld in der Schnittebene nicht mehr die
ebene Laplace-Gleichung. Damit können holomorphe Funktionen nicht mehr
für das Feld in der Schnittebene benutzt werden. Das ist leider ein
deutlicher Abschlag zur bisher so eleganten Theorie.

Wie sieht nun das komplexe Potential einer Punktladung in der Ebene
aus? Bei einer negativen Punktladungen laufen die Feldlinien
strahlenförmig in ein Zentrum hinein. Dann muss das Potential
im Kreis von $0$ bis $2\pi$ linear zunehmen. Damit
gilt
\[\psi(z) = \psi(1)\,\mathrm{arg}(z).\]

\subsection{Ladungsdichten}

Betrachten wir eine Ladung $Q$ auf einer Geraden. Auf dieser wählt
man ein eindimensionales Koordinatensystem. Für das elektrische
Feld gilt dann
\begin{equation}
E(x) = \frac{1}{4\pi\varepsilon_0} Q\frac{x-x_0}{|x-x_0|^3}.
\end{equation}
Nun wird die Ladung auf einem Intervall der Geraden verschmiert.
Die Gesamtladung soll dabei die Summe aller Teilladungen sein.
Wenn es bei der Verschmierung aber unendlich viele unendlich
kleine Teilladungen gibt, so muss man eine Ladungsdichte
einführen. Es gilt dann
\begin{equation}
Q = \int_a^b \lambda(x)\,\mathrm dx,
\end{equation}
wobei mit $\lambda(x)$ die Linienladungsdichte am Ort $x$ gemeint
ist. Konzentriert man die gesamte Ladung am Ort $x_0$, so muss
man eine Deltadistribution benutzen. Es ist dann
\begin{equation}
\lambda(x) = Q\delta(x-x_0).
\end{equation}
Beim Summieren der elektrischen Feldstärken ergibt sich
die gleiche Situation. Jede infinitesimale Ladung trägt
einen infinitesimalen Teil zum Feld bei. Hier ergibt sich
\begin{equation}
E(x) = \frac{1}{4\pi\varepsilon_0}\int_a^b
\lambda(x')\frac{x-x'}{|x-x'|^3}\,\mathrm dx'.
\end{equation}
Verwendet man Deltadistributionen, so lässt sich diese Formel
mit \eqref{eq:ESumme} vereinheitlichen. Die Linienladungsdichte teilt
sich dabei in einen diskreten und einen kontinuierlichen Teil
auf, was durch die Formel
\begin{equation}
\lambda(x) = \lambda_c(x)+\sum_{k=1}^n q_k\delta(x-x_k)
\end{equation}
beschrieben werden kann. Analoge Formeln gelten bei der
Flächenladungsdichte $\sigma(x)$ und der Raumladungsdichte $\rho(x)$.
Bei dieser ist dann
\begin{equation}
E(x) = \frac{1}{4\pi\varepsilon_0}\iiint_V
\rho(x')\frac{x-x'}{|x-x'|^3}\,\mathrm dx_1'
\mathrm dx_2'\mathrm dx_3'.
\end{equation}
Mit dem Satz von Fubini als Begründung können wir das Integral
abstrakter schreiben, so dass es die Form
\begin{equation}
E(x) = \frac{1}{4\pi\varepsilon_0}\int_V
\rho(x')\frac{x-x'}{|x-x'|^3}\,\mathrm dx'.
\end{equation}
annimmt.

\subsection{Gaußsche Größen}
Neben den hier verwendeten Größen werden in der Elektrodynamik
auch gaußsche Größen benutzt. Damit ist gemeint, dass ein Teil
der Größen anders definiert ist. Um die Unterschiede präzise
herausstellen zu können, bezeichnen wir mit $X^g$ die gaußsche Größe,
wenn mit $X$ die normale Größe gemeint ist.
Für die elektrische Feldstärke gilt z.B.
$E^g := \sqrt{4\pi\varepsilon_0}\,E$.
Die gaußsche Ladung ist durch
$q^g:=q/\sqrt{4\pi\varepsilon_0}$ definiert. Hiermit lassen
sich einige Gleichungen kürzer schreiben, z.B. bekommt
(\ref{Coulomb}) die Form
\begin{equation}
|F|r^2 = |q_1^g q_2^g|.
\end{equation}

\vfill
\noindent
Dieser Text steht unter der Lizenz\\
Creative Commons CC0.

\end{document}


