\documentclass[a4paper,11pt,fleqn,twocolumn,twoside,dvipdfmx]{scrartcl}
\usepackage[utf8]{inputenc}
\usepackage[T1]{fontenc}
\usepackage[ngerman]{babel}
\usepackage{microtype}
\usepackage{libertine}
\usepackage[libertine,smallerops]{newtxmath}
% \usepackage[scaled=0.8]{DejaVuSansMono}
% \usepackage[scaled=0.8]{DejaVuSans}
\setkomafont{sectioning}{\normalfont\normalcolor\bfseries}

\usepackage{amsmath}
\usepackage{amssymb}

\usepackage{color}
\definecolor{c1}{RGB}{00,40,80}
\usepackage[colorlinks=true,linkcolor=c1]{hyperref}
\usepackage{geometry}
\geometry{a4paper,left=25mm,right=12mm,top=20mm,bottom=30mm}
\setlength{\columnsep}{6mm}

\usepackage{bm}
\newcommand{\bvec}[1]{\bm{\mathsf{#1}}}

\begin{document}

\thispagestyle{empty}

\begin{huge}
\noindent
\textbf{Aufzeichnungen\\
zur Mechanik}
\par
\end{huge}
\vspace{1em}

\tableofcontents

\section{Bewegungen}

\subsection{Vorbereitungen}

Wie kann man eine Bewegung in der Ebene oder im Raum beschreiben?
Ein Punkt lässt sich ja durch einen Ortsvektor beschreiben. In der
Ebene wird ein Punkt beschrieben durch%
\[\bvec x
= \begin{bmatrix}x_1\\ x_2\end{bmatrix}
= x_1e_1+x_2e_2.\]
Man wählt also ein Koordinatensystem mit zwei Achsen, in dem der
Punkt die Koordinaten $(x_1,x_2)$ hat. Im Raum ist dementsprechend%
\[\bvec x = x_1e_1+x_2e_2+x_3e_3.\]
Wenn sich der Punkt nun bewegen soll, so muss jedem Zeitpunkt $t$
ein Ortsvektor zugeordnet werden. Um das zu erreichen, kann man die
Komponenten $x_1,x_2$ von der Zeit abhängig machen. Jede Komponente
kann dann als eine Funktion betrachtet werden, welche die Zeit als
Argument hat. Es ist also%
\[\bvec x = f_1(t)e_1+f_2(t)e_2.\]
In der Mathematik bezeichnet man einen solchen, durch $t$
parametrisierten Vektor als Parameterkurve.

Eine solche Funktion kann man auch ableiten. Man wendet einfach die
Summenregel der Differenzialrechnung an. Da $e_1,e_2$ konstant sind,
kann man sie aus der Ableitung herausziehen. Es ist%
\begin{gather*}
\bvec x' = (f_1(t)e_1+f_2(t)e_2)'\\
= (f_1(t)e_1)'+(f_2(t)e_2)'\\
= f_1'(t)e_1+f_2'(t)e_2.
\end{gather*}
In der eulerschen Schreibweise ist das noch arithmetischer.
Man multipliziert einfach mit dem Differntialoperator aus.
Man rechnet dann%
\begin{gather*}
D\bvec x = D(f_1(t)e_1+f_2(t)e_2)\\
= Df_1(t)e_1+Df_2(t)e_2.
\end{gather*}
Die Ableitung einer Parameterkurve an der Stelle $t_0$ hat eine
anschauliche Bedeutung. Es ist der Tangentialvektor an die Kurve
am Punkt $\bvec x(t_0)$. Der Betrag dieses Tangentialvektors
ist die Geschwindigkeit, mit der sich der Massepunkt zur Zeit $t_0$
auf der Kurve bewegt. Deshalb führt man die Bezeichnung
$\bvec v:=\bvec x'$ ein. Der Betrag ist%
\[|\bvec v| = \sqrt{f_1'(t)^2+f_2'(t)^2}.\]

\subsection[Umlenkung der Erdbeschleunigung]
{Umlenkung der\newline Erdbeschleunigung}

Die Folgenden Vorgänge sind nicht auf die Erde beschränkt. Sollen
sie zum Beispiel auf dem Mond stattfinden, so muss man nur die
Fallbeschleunigung $g$ des Mondes verwenden.

Ein kleiner Rollwagen wird durch einen Faden mit einem Gewicht
befestigt und auf einen Tisch gestellt. Das Gewicht wird über die
Tischkante gehängt und soll den Wagen beschleunigen. Um Reibung zu
vermindern, kann an der Tischkante eine Rolle aufgestellt werden, die
den Faden nach unten umlenkt. Wie stark wird der Wagen beschleunigt?
Ist seine Beschleunigung gleichförmig?

Es gilt also zu zeigen, dass es wirklich eine konstante Beschleunigung
gibt. Man gebe dem Gewicht die Nummer~1 und dem Wagen die Nummer~2.
Zu welchem Körper eine Größe gehört kann nun anhand der Indexnummer
entschieden werden.

Die Kraft ${m_1}g$ die durch das Gewicht und die Erdbeschleunigung
erzeugt wird, bewegt nach dem Loslassen beide Massen, den Wagen und das
Gewicht. Das heißt, diese Kraft kann durch zwei Gleichungen ausgedrückt
werden. Es ist%
\begin{gather*}
F = {m_1}g,\\
F = (m_1+m_2)a.
\end{gather*}
Nach dem Gleichsetzen und Umformen nach $a$ erhält man die Formel%
\[a = \frac{m_1}{m_1+m_2} g.\]
Da die Massen konstant sind und als Faktor mal Erdbeschleunigung
wirken, ist auch die Beschleunigung des Wagens konstant. Mit Hilfe
der Formel kann berechnet werden, dass $a=0$ aus $m_1=0$ folgt.
Weiterhin folgt $a=g$ aus $m_2=0$.

Das stimmt auch mit der Wirklichkeit überein. Die Massen der Körper
können dabei natürlich auch jeweils in Bezug zueinander verschwindend
gering sein. Sie müssen nicht exakt null sein.

\subsection{Fallmaschine}

Zwei Massen sind durch ein Seil verbunden, das über eine Rolle gehängt
wird. Mit dieser Vorrichtung lässt sich die Erdbeschleunigung beliebig
verkleinern. Wie groß ist die Beschleunigung?

Die Kräfte sind $F_1 = {m_1}g$ und $F_2 = {m_2}g$. Dass die Kräfte
umgelenkt werden, ändert nichts daran, dass sie entgegengesetzt
gerichtet sind. Eine Umlenkrolle kann eine Kraft in jede Richtung
lenken. Man muss nur die Beschleunigung konstant halten. Die
resultierende Kraft ist also die Differenz $F=F_1-F_2$.
Dabei ist die zweite Kraft die kleinere. Das heißt die zweite
Masse ist kleiner als die erste.

Die resultierende Kraft zieht beide Massen $F=(m_1+m_2)a$.
Gleichsetzen des Gleichungssystems liefert%
\[a = \frac{m_1-m_2}{m_1+m_2}g.\]
Man will nun erst für beide Enden die Gleiche Masse $M$ verwenden,
und dann an einem Ende eine kleine Masse $m$ hinzufügen. Einsetzen
von $m_1=M+m$ und $m_2=M$ liefert%
\[a = \frac{m}{2M+m} g.\]

\subsection{Kreisbewegungen}

Eine Kreisbewegung in der Ebene kann parameterisiert werden durch%
\[\bvec x = r \begin{bmatrix}
\cos\omega t\\
\sin\omega t
\end{bmatrix}.\]
Dabei ist $r$ der Radius, $\omega$ die Kreisfrequenz und $t$
die Zeit. Um die Geschwindigkeit zu erhalten wird der Weg abgeleitet.
Man erhält%
\[\bvec v = \bvec s'
= \omega r \begin{bmatrix}
-\sin\omega t\\
\cos\omega t
\end{bmatrix}.\]
Der Betrag der des Geschwindigkeitsvektors müsste unseren bisherigen
Überlegungen nach konstant sein. Wir bilden ihn und überprüfen das.
Dabei macht man sich die Identität $\sin^2 x+\cos^2 x=1$ zum Nutzen.
Das Minuszeichen verschwindet beim Quadrieren. Man erhält%
\[v = \omega r.\]
Die Geschwindigkeit hängt also tatsächlich nicht von der Zeit ab.
Dadurch ist auch die kinetische Energie konstant. Um eine Gleichung
für die Beschleunigung zu bekommen, leiten wir wieder ab und erhalten%
\[\bvec a = \bvec v' = - {\omega^2}r
\begin{bmatrix}
\cos\omega t\\
\sin\omega t\end{bmatrix}.
\]
Der Betrag des Beschleunigungsvektors ist%
\[a = {\omega^2}r.\]
Formt man die Gleichung für $v$ nach $\omega$ um, und setzt dann
in die Gleichung für $a$ ein, so ist%
\[a = \frac{v^2}{r}.\]
Die Kraft $\bvec F=m\bvec a$, die durch Trägheit entsteht,
wird \textit{Zentrifugalkraft} genannt und zieht den Massepunkt
aus der Kreisbahn. Eine Gegenkraft muss den Massepunkt also auf der
Kreisbahn halten. Diese Kraft wird \textit{Zentripetalkraft} genannt.
Sie ist dem bekannten Gesetz von Kraft und Gegenkraft zufolge
$\bvec F_z = -\bvec F$. Teilen durch Masse liefert die Beziehung
für die Beschleunigung. An den Beträgen ändert die Richtung nichts.

\subsection{Senkrechter Wurf}

Die Gleichung für die Kräfte lautet $ma=-mg$. Nach dem Teilen durch
die Masse und Integrieren erhält man eine Gleichung für die
Geschwindigkeit:%
\[v = -gt + v_0.\]
Nochmaliges Integrieren bringt die Gleichung für den Weg:%
\[s = -\frac{1}{2}gt^2 + {v_0}t + s_0.\]
Befindet sich der Massepunkt in Ruhe, so entfällt die
Anfangsgeschwindigkeit. Wird der Massepunkt in die Höhe geworfen,
so ist sie positiv. Wird er nach unten geworfen so ist sie negativ.
Der Anfangsweg stellt die Höhe am Anfang dar. Die Gleichung gilt
natürlich überall. Sie kann nach belieben in ein Koordinatensystem
eingebettet werden.

\subsection{Waagerechter Wurf}

Eine Kugel rollt mit einer Geschwindigkeit $v_0$ auf einem Tisch
und fällt anschließend über den Rand. Wie lässt sich die Flugbahn
berechnen?

Die Bewegung kann in eine horizontale und eine vertikale Teilbewegung
aufgeteilt werden. Horizontal bewegt sich die Kugel ja gleichförmig
(das heißt unbeschleunigt) als wie, wenn sie auf dem Tisch weiter
rollen würde. Vertikal wird sie Beschleunigt, als wie, wenn man eine
Kugel aus der Hand fallen lässt. Damit ergeben sich zwei Gleichungen
zu einem System%
\begin{gather*}
x = {v_0}t,\\
y = -\frac{1}{2}gt^2.
\end{gather*}
Das Minuszeichen ist nicht zwingend. Es verdeutlicht nur eine
Bewegung nach unten. Stellt man die Gleichung für $x$ nach $t$
um und setzt in die Gleichung für $y$ ein, so erhält man%
\[y = -\frac{g}{2v_0^2} x^2.\]
Man kann das Gleichungssystem als Parameter{}gleichung für den Ort
in Abhängigkeit der Zeit ansehen. Dann lässt sich durch Ableiten
die Parameter{}gleichung für die Geschwindigkeit ausrechnen:%
\[\bvec v = \bvec s'
= \begin{bmatrix}v_0\\ -gt
\end{bmatrix}\]
Der Betrag des Geschwindigkeitsvektors ist%
\[v = \sqrt{v_0^2 + g^2t^2}.\]

\subsection{Schräger Wurf}

Die Gleichungen für die Teilbewegungen sind aus den vorherigen
Rechnungen bekannt:%
\begin{gather*}
x = v_1t\\
y = v_2t - \frac{g}{2}t^2.
\end{gather*}
Setzt man nun für die Zeit ein, so ist%
\[y = \frac{v_2}{v_1}x - \frac{g}{2v_1^2} x^2.\]
Gegeben sei eine Anfangsgeschwindigkeit $v_0$ und ein
Winkel $\alpha$. Klar ist, dass $v_0^2=v_1^2+v_2^2$ ist,
was man auch durch Ableiten zum Tangentialvektor bei $t=0$
erkennen kann. Dadurch ist%
\begin{gather*}
v_1 = v_0\cos\alpha,\\
v_2 = v_0\sin\alpha.
\end{gather*}
Setzt man für die Teilgeschwindigkeiten ein, so wird die Gleichung
für den Weg zu%
\[y = x\tan\alpha - \frac{gx^2}{2v_0^2\cos^2\alpha}.\]
Ableiten der Gleichung, bevor man die trigonometrischen Funktionen
einsetzt, und null setzen bringt $gx = v_1v_2$.
Setzt man $x=v_1t$, so ist%
\[t = \frac{v_0\sin\alpha}{g}.\]
Einsetzen in die Parametergleichung $y(t)$ bringt die Wurfhöhe%
\[h = \frac{v_0^2\sin^2\alpha}{2g}.\]

\subsection{Echter senkrechter Wurf}

Die Gleichungen lassen sich auf dem Mond auch für weite Würfe gut
verwenden. Man beachte aber auch, dass sich die Fall{}beschleunigung
mit dem Abstand zum Mond ändert. Auf der Erde herrscht Luftwiderstand,
der die Wurfbahn verändert und berücksichtigt werden muss.

Die Kraft vom Luftwiderstand wird von der Kraft durch die
Erdbeschleunigung abgezogen. Die Kraftgleichung ist $ma = -mg + F_w$.
Man teile durch die Masse. Um die Gleichungen kurz zu halten habe
ich eine neue Konstante definiert:%
\[F_w = \frac{1}{2}c_w\varrho Av^2,\quad B = \frac{c_w\varrho A}{2m}.\]
Die Beschleunigung ist die Ableitung der Geschwindigkeit.
Dadurch ergibt sich eine Differentialgleichung%
\[v' = -g + Bv^2.\]

\subsection{Impuls}

\textbf{Definition.} Der Impuls ist durch die Gleichung $p=mv$
definiert. Die Einheit ist dem entsprechend Ns.

Die Herleitung für einen wichtigen Erhaltungssatz: Beim Stoß gibt es
Kraft und Gegenkraft. Auch, wenn sie sehr schlecht zu
veranschaulichen sind, es ist%
\begin{align*}
F_1 &= -F_2\\
m_1(u_1-v_1) &= m_2(v_2-u_2)\\
m_1u_1+m_2u_2 &= m_1v_1+m_2v_2\\
p_u &= p_v
\end{align*}
Wir rechnen hier nicht mit Kraftbeträgen, sondern mit
gerichteten Kräften in einer Dimension. Positiv ist in die
eine Richtung, negativ in die andere. In mehreren Dimensionen nennt
man sie auch Vektoren.

\textbf{Impulserhaltungssatz}: Der Gesamtimpuls ist konstant
und bleibt auch bei Zusammenstößen erhalten. Für den Stoß zweier
Körper ist $p_u=p_v$.

\textbf{Elastischer Stoß}: Beispiele sind der Flummi, Kugeln und
aufgepumpte Bälle. Die kinetische Energie bleibt erhalten.
Der Impuls bleibt erhalten.

\textbf{Unelastischer Stoß}: Beispiele sind Sandsäcke und nicht
aufgepumpte Bälle. Die kinetische Energie wird umgewandelt
(Verformung, Reibung, Wärme). Der Impuls bleibt erhalten.


\subsection{Ballistisches Pendel}

Bestimmung der Geschossgeschwindigkeit mit dem ballistischen Pendel.
Durch eine Vorrichtung wird ein Projektil auf ein an zwei Sehnen
(bifilar) aufgehängtes ballistisches Pendel geschossen. Das feste
Projektil trifft die Oberfläche geringer Festigkeit orthogonal und
dringt ein. Daraufhin wird das Pendel -- durch die waagerechte
Auslenkung -- senkrecht um eine Höhe $h$ angehoben. Bekannt sind
außerdem ja die Masse des Geschosses und die des Pendels. Wie findet
man die Geschwindigkeit des Geschosses?

Es erfolgt ein unelastischer Stoß, da das Projektil im Pendel stecken
bleibt. Sei das Projektil Körper~1 und das Pendel der zweite.
Mit gegebenen $m_1, m_2$ und $v_2=0$ ist, weil der
Impulserhaltungssatz gilt:%
\begin{gather*}
u = \frac{m_1 v_1 + m_2 v_2}{m_1 + m_2} = \frac{m_1 v_1}{m_1 + m_2},\\
v_1 = \frac{m_1 + m_2}{m_1} u.
\end{gather*}
%
Nun fehlt noch die Geschwindigkeit $u$ des Pendels direkt nach
dem Stoß. Diese Geschwindigkeit nimmt langsam ab, die kinetische
Energie wandelt sich in potenzielle um und das Pendel gewinnt an
Höhe. Nachdem es durch den unelastischen Stoß $u$ erhielt -- wobei
es keine Energieerhaltung gab -- gibt es nun sie wieder.
Es ist $E_\text{kin} = E_\text{pot}$.%
\begin{align*}
\frac{1}{2}(m_1+m_2) u^2 &= (m_1+m_2)gh\\
u^2 &= 2gh\\
u &= \sqrt{2gh}
\end{align*}
Da es sich bei Geschwindigkeiten um Beträge handelt, gibt es keine
negativen, so dass Quadrieren und Radizieren Äquivalenzumformungen
sind, wie bei Addition und Subtraktion. Also entfällt die negative
Wurzel. Somit ist%
\[v_1 = \frac{m_1+m_2}{m_1} \sqrt{2gh}\]

\subsection{Kreisbewegungen}

% Unterschied zwischen allgemeiner Kreisbewegung,
% gleichförmiger Kreisbewegung und Rotation

Eine Rotation (Drehbewegung) ist die \textit{Drehung eines Körpers
um einen Punkt}. Wo dieser sich dieser Punkt befindet ist egal.
Der Körper darf sich dabei natürlich nicht verformen, was alles
sehr viel komplizierter machen würde. Ein Spezialfall ist die
Kreisbewegung eines Massepunktes. Alle Formeln die hier gefunden
werden können gelten also auch dafür.

Da jeder Punkt des Körpers mit einem anderem Abstand $r$ vom
Zentrum einen anderen Weg in der gleichen Zeit zurücklegt, wenn
sich der Winkel ändert, können mit dem Weg $s$ nur die Gleichungen
für den jeweiligen Abstand aufgestellt werden und nie für den
ganzen Körper. Die Gleichungen sollten sich also auf den Winkel
beziehen. Diesen Drehwinkel nennen wir nun klein sigma $\sigma$
und führen damit eine neue Größe ein. Wie bei jedem Winkel
üblich ist er im Bogenmaß dimensionslos.

Gibt es eine Beziehung zwischen dem Weg und dem Winkel? Ja,
denn der Weg ist nichts anderes als die Länge des Kreisbogens
mit dem Radius $r$. Da die Vergrößerung eines Kreises eine
zentrische Streckung ist und sich damit zwei Kreise ähnlich
sind, bleiben die Verhältnisse der Strecken erhalten.
Das bedeutet $s$ ist proportional zu $\sigma$ also $s=c\sigma$.
Fehlt also noch der Streckungsfaktor. Diesen gewinnt man am besten
aus der Erkenntnis, dass im Vollkreis $s=u=2\pi r$ und
$\sigma = 2\pi$ ist. Man rechnet%
\[\frac{s}{\sigma} = \frac{2\pi r}{2\pi} = r.\]
%
Damit erhält man $s=r\sigma$. Diese Formel kann auch direkt
aus der Definition des Kreises, ein paar Tricks und Integralrechnung
nur mit Formeln gewonnen werden.

\section{Gravitation}

\subsection{Das Gravitationsgesetz}

Der Mond bewegt sich näherungsweise auf einer Kreisbewegung um die
Erde. Der Mond habe die noch unbekannte Masse $m$ und die Erde habe
die noch unbekannte größere Masse $M$. Durch die Trägheit wirkt die
Zentrifugalkraft $F=mr\omega^2$ auf den Mond. Es muss eine
Zentripedalkraft umgekehrter Richtung aber gleichen Betrags geben,
die den Mond auf seiner Bahn um die Erde hält.
Diese wird als \textit{Gravitationskraft} bezeichnet.

Man kann vermuten, dass diese Kraft durch die Massen selbst entsteht.
Wenn dem so wäre würden sie sich gegenseitig beeinflussen. Dann
bewegen sich beide um einen Schwerpunkt,
das sogenannte \textit{Gravizentrum}.

Nun können wir uns das dritte der drei Gesetze der Planetenbewegung
von Johannis Kepler zunutze machen. Zuerst muss $\omega=2\pi/T$
ersetzt werden. Das dritte Gesetz besagt, dass für eine
Planetenbahn $T^2/a^3=C$ konstant ist. Die Große Halbachse
$a$ der Ellipse ist gleich dem Radius, weil wir es mit einer
Kreisbewegung zu tun haben und der Kreis ein Spezialfall der
Ellipse ist. Dadurch ist $T^2=Cr^3$ und damit%
\[F = \frac{4\pi^2 m}{Cr^2}.\]
Diese Kraft wirkt auch auf die Erde, aber die Masse ist $M$
und die Konstante ist anders. Gleichsetzen der Kräfte
bringt $mC_2=MC$. Das Produkt von Planetenmasse und der Konstante
aus Kepler drei scheint nun eine echte Konstante zu sein, die überall
gleich ist. Wir wählen dafür
die Bezeichnung $G$ mit $4\pi^2/G=MC$, was zunächst etwas komisch
wirkt. Aber dafür ergibt sich eine Vereinfachung in der Schreibweise.
Einsetzen für $C$ bringt schließlich das

\textbf{Gravitationsgesetz}
\[F = G \frac{Mm}{r^2}.\]
Hierbei ist $G$ ein Wert, der immer und überall gleich ist, eine
universelle Naturkonstante. Die Gleichungen gelten aber zunächst
nur für Kreisbewegungen. Was bei den Ellipsenbahnen passiert,
ist noch unklar.

Dummerweise sind diese Konstante und die Massen immer noch unbekannt.
Deshalb sollten wir nun danach suchen. Die Kraft wirkt auch auf
uns auf der Erdoberfläche. Aber da $F=mg$ ist erhält man durch
Gleichsetzen $gr^2=GM$. Mit der Masse der Erde haben wir also auch
die Gravitationskonstante und umgekehrt.

Es ist vorteilhaft, das Gravitationsgesetz in Vektorform zu
formulieren. Wir haben zwei Massen $m_1, m_2$.
Mit $\bvec x_1, \bvec x_2$ wollen wir die Positionen
der Massen bezeichnen. Auf $m_1$ wirkt ein Kraftvektor, der in
Richtung $m_2$ zeigt. Der Einheitsvektor in diese Richtung ist%
\[\frac{\bvec r_1}{r_1} = \frac{\bvec x_2-\bvec x_1}
{|\bvec x_2-\bvec x_1|}.\]
Der Kraftvektor ist dann
\[\bvec F_1 = G \frac{m_1m_2}{r_1^2}\frac{\bvec r_1}{r_1}
= Gm_1m_2\frac{\bvec r_1}{r_1^3}.\]
Damit erhält man
\[\bvec F_1 = Gm_1m_2\frac{\bvec x_2-\bvec x_1}
{|\bvec x_2-\bvec x_1|^3}.\]
Wenn auf eine Masse mehrere Kräfte wirken, so addiert man die
Kraftvektoren. Befinden sich im Weltraum mehrere Massen,
so erhält man%
\[\bvec F_k = Gm_k\sum_{i\ne k}m_i\frac{\bvec x_i-\bvec x_k}
{|\bvec x_i-\bvec x_k|^3}.\]
Die Gravitation $\bvec g(\bvec x)$ bei einem
Punkt $\bvec x$ im Weltraum ist%
\[\bvec g(\bvec x)
= G\sum_{i=1}^n m_i\frac{\bvec x_i-\bvec x}
{|\bvec x_i-\bvec x|^3}.\]
Auf eine kleine Probemasse, etwa ein Raumschiff, wirkt die Kraft%
\[\bvec F = m\bvec g(\bvec x).\]

\subsection{Die Energie im Gravitationsfeld}

Die Richtung des Radiusvektors und die Richtung der Kraft die
Aufgebracht werden muss, um der Gravitationskraft entgegen zu wirken
sind gleich. Dadurch kann eine einfachere Definition der Arbeit
verwendet werden (die allgemeine geht über ein Kurvenintegral). Da
sie nicht konstant ist, sondern vom Weg abhängig, ist die Arbeit der
Flächeninhalt%
\begin{gather*}
W = \int_{r_1}^{r_2} F\,\mathrm dr
= GMm\int_{r_1}^{r_2} r^{-2}\,\mathrm dr\\
= GMm\Big(\frac{1}{r_1}-\frac{1}{r_2}\Big).
\end{gather*}
Diese Arbeit müsste irgendwie wieder in die Hubarbeit übergehen.
Dazu sei $r_1$ der Abstand vom Erdmittelpunkt zur Erdoberfläche.
Mit der Höhe über der Erdoberfläche $h=r_2-r_1$
und dem Hauptnenner ist%
\[\frac{1}{r_1}-\frac{1}{r_2} = \frac{h}{r_1r_2}\]
und weiterhin gilt ja auch $GM=gr_1^2$ und damit%
\[W = g{r_1^2}m \frac{h}{r_1r_2} = mgh\frac{r_1}{r_2}.\]
Da sich $r_1$ und $r_2$ nur sehr geringfügig unterscheiden,
ist in guter Näherung $r_1/r_2\approx 1$. Man erhält also%
\[W \approx mgh.\]

\subsection{Geostationäre Satelliten}

Setzt man die Zentrifugalkraft mit der Gravitationskraft gleich,
so ergibt sich die Beziehung $r^3\omega^2=GM$. Die
Winkelgeschwindigkeit eines Satelliten auf einer Kreisbahn hängt also
vom Abstand zum Erdmittelpunkt ab und umgekehrt. Setzt man für einen
geostationären Satelliten die Winkelgeschwindigkeit der Erde
$\omega=2\pi/T$ ein, so erhält man den Abstand. Die Rotation der
Erde hat eine Periodendauer von 24~h.

\newpage
\subsection{Gravitation und Potential}

Das Gravitationsfeld ist der negative Gradient des Potentialfeldes.
Es ist%
\[\bvec g(\bvec r) = -\nabla u(\bvec r).\]
Das folgende Potentialfeld kann die Gravitation im Weltraum
beschreiben:%
\[u(\bvec r) = \frac{Gm}{r}.\]
Man bilde davon also den negativen Gradient. Die Gravitationskonstante
und die Masse sind konstante Faktoren, die vor den Vektor der
partiellen Ableitungen geschrieben werden können. Man wende die
Kettenregel an. Das Minus entfernt sich durch eines aus der
Ableitung der Wurzel im Nenner. Man rechnet%
\begin{gather*}
\bvec g(\bvec r)
= -Gm\nabla\frac{1}{r}
= -Gm\sum_{i=1}^3 \frac{\partial}{\partial r_i}
e_i \frac{1}{\sqrt{\bvec r^2}}\\
= \frac{1}{2}Gm (\bvec r^2)^{-3/2}
\sum_{i=1}^3 \frac{\partial}{\partial r_i} e_i (r_1^2+r_2^2+r_3^2)\\
= \frac{Gm}{2r^3}(2r_1e_1 + 2r_2e_2 + 2r_3e_3).
\end{gather*}
Damit ist die Gleichung für die Gravitation%
\[\bvec g(\bvec r) = \frac{Gm}{r^3} \bvec r.\]


\vfill\noindent
Dieser Text steht unter der Lizenz\\
Creative Commons CC0.

\end{document}


