\documentclass[a4paper,10pt,fleqn,twocolumn,twoside]{scrartcl}
\usepackage[utf8]{inputenc}
\usepackage{ngerman}
\usepackage{amsmath}
\usepackage{amssymb}
\usepackage{color}
\definecolor{c1}{RGB}{00,40,60}
\usepackage[colorlinks=true,linkcolor=c1]{hyperref}
\usepackage{geometry}
\geometry{a4paper,left=25mm,right=12mm,top=20mm,bottom=28mm}
\setlength{\columnsep}{4mm}
\numberwithin{equation}{section}
\newcommand{\ui}{\mathrm i}
\newcommand{\ee}{\mathrm e}
\begin{document}

\noindent
{\LARGE\sffamily\bfseries Aufzeichnungen zur Physik}
\tableofcontents

\section{Quantenmechanik}
\subsection{Schrödingergleichung}

Man geht davon aus, dass eine Materiewelle der Wellengleichung
\begin{equation}
D_x^2\Psi = \frac{1}{c^2}D_t^2\Psi
\end{equation}
genügt. Nun wählt man eine stehende harmonische Welle als Ansatz.
Es ist dann
\begin{equation}
\Psi(x,t) = \Psi(x)\,\ee^{-\ui\omega t}.
\end{equation}
Die Ableitung nach der Zeit ist
\begin{equation}
D_t\Psi(x,t) = -\ui\omega\Psi(x)\,\ee^{-\ui\omega t}
= -\ui\omega\Psi(x,t).
\end{equation}
Für die Materiewelle gilt nun $\lambda=h/p=h/(mc)$. Damit rechnet
man nun
\begin{equation}
\frac{1}{c^2} = \frac{1}{(\lambda f)^2}
= \frac{m^2 c^2}{h^2 f^2}.
\end{equation}
Nun gilt aber
\begin{equation}
E=T+V=\frac{1}{2}mc^2+V
\end{equation}
und somit $2m(E-V)=m^2 c^2$. Setzt man noch $E=hf=\hbar\omega$,
so ergibt sich
\begin{equation}
\frac{1}{c^2} = \frac{2m(\hbar\omega-V)}{(hf)^2}.
\end{equation}
Damit ist dann
\begin{equation}
D_x^2\Psi
= -\ui\omega\frac{2m(\hbar\omega-V)}{(\hbar\omega)^2}D_t\Psi.
\end{equation}
Multipliziert man aus und benutzt $VD_t\Psi=-\ui\omega V\Psi$,
so ergibt sich
\begin{equation}
\frac{D_x^2\Psi}{2m}
= \frac{D_t\Psi}{\ui\hbar}+\frac{1}{\hbar^2}V\Psi.
\end{equation}
Man formt noch in die Standardform um und erhält die
\emph{Schrödingergleichung}
\begin{equation}
D_t\Psi = \frac{1}{i\hbar}\Big(-\frac{\hbar^2}{2m}D_x^2+V\Big)\Psi.
\end{equation}
Man definiert nun
\begin{equation}
\hat H := -\frac{\hbar^2}{2m}D_x^2+V.
\end{equation}
Somit kürzt sich die Gleichung ab zu
\begin{equation}
D_t\Psi = \frac{1}{\ui\hbar}\hat H\Psi.
\end{equation}
Der Satz von Taylor lautet in Operatorschreibweise
\begin{equation}
\Psi(x,t)=\exp(tD_t)\Psi(x,0).
\end{equation}
Verwendet man nun die Schrödingergleichung, so ergibt sich
der Zeitentwicklungsoperator zu
\begin{equation}
\hat U = \exp(tD_t) = \exp\Big(\frac{t\hat H}{\ui\hbar}\Big).
\end{equation}

\subsection[Zeitunabhängige Schrödingergleichung]
{Zeitunabhängige\\
Schrödingergleichung}

Man geht wieder von der Wellengleichung
\begin{equation}
D_x^2\Psi=\frac{1}{c^2}D_t^2\Psi
\end{equation}
aus und nimmt den entkoppelten Ansatz
\begin{equation}
\Psi(x,t)=\Psi(x)\sin(\omega t),
\end{equation}
der sich für eine stehende Welle ergibt. Eingesetzt in
die Wellengleichung ergibt sich zunächst
\begin{equation}
\sin(\omega t)\Psi''(x)=-\frac{\omega^2}{c^2}\sin(\omega t)\Psi(x).
\end{equation}
Nach dem Kürzen bleibt
\begin{equation}
\Psi''(x) + \frac{\omega^2}{c^2}\Psi(x)
\end{equation}
übrig. Man nutzt nun wieder
\begin{equation}
\frac{1}{c^2} = \frac{m^2 c^2}{h^2 f^2}
= \frac{2m(E-V)}{h^2 f^2}
= \frac{2m(E-V)}{\hbar^2 \omega^2}.
\end{equation}
Somit ergibt sich
\begin{equation}
\Psi''(x) + \frac{2m}{\hbar^2}(E-V)\Psi(x) = 0.
\end{equation}
Das ist die \emph{zeitunabhängige Schrödingergleichung}.

\newpage
\section{Metrologie}
\subsection{Abweichungen}

Physikalische Größen gibt man als Produkt von Zahlenwert und Einheit
an. Nun wird man sich fragen, wie viele Dezimalstellen man überhaupt
bei der Messung einer Größe aufschreiben muss. Dazu muss ersteinmal
präzise definiert werden, was man unter
\textit{signifikanten Stellen} überhaupt versteht. Dazu schreibt man
den Zahlenwert in der Darstellung
\begin{equation}
z = m\times 10^{e}
\end{equation}
wobei $m$ die Mantisse und $e$ der Exponent ist. Der Exponent
ist nun so zu wählen, dass für die Mantisse $m<1$ und $m\ge 0.1$
gilt. Die Anzahl der signifikanten Stellen ist dann einfach die
Anzahl der Nachkomma"-stellen.

Die signifikanten Stellen sind nun die Dezimalstellen, welche bei
wiederholter Messung nicht oder nur sehr gering schwanken.

Das ist natürlich eine recht schwammige Definition. Aufgrund der
Angabe der signifikanten Stellen kennt man nur die Zehnerpotenz
der Fehlergrenze. Man kann also nicht sagen, ob eine Stelle schwach
geschwankt hat oder gar nicht. Dieser Umstand ist recht unzulänglich
und wir führen daher die Begriffe \emph{Abweichung} und
\emph{Grenzabweichung} ein. Synonyme dafür sind \emph{Fehler} und
\emph{Fehlergrenze}.

Wir nehmen nun ein Lineal und werden damit versuchen, eine Länge
von $100\,\mathrm{mm}$
zwischen zwei Punkten auf einem Blatt Papier zu messen. Wir gehen
außerdem davon aus, dass die Skala des Lineals bei der Herstellung
hinreichend genau kalibriert wurde.

Bezeichnen wir mit $w=100\,\mathrm{mm}$ den wahren Wert der Länge.
Den Messwert bezeichnen wir mit $x$. Mit dem Lineal können wir
die Länge schätzungsweise auf etwa $\frac{1}{2}\mathrm{mm}$ genau
messen. Diese Abweichung bezeichnen wir als Grenzabweichung $G$,
es ist
\begin{equation}
G = 0.5\,\mathrm{mm}.
\end{equation}
Der Messerwert $x$ wird um den wahren Wert herum schwanken. Es ist
\begin{equation}
x\in [w-G,w+G].
\end{equation}
Den Wert $\Delta x := x-w$ bezeichnet man als Abweichung. Da wir
den wahren Wert nicht kennen, können wir weder das Intervall noch
die Abweichung angeben. Das Intervall kann man jedoch als
\begin{equation}
w-G\le x\le w+G.
\end{equation}
schreiben. Durch Umformen dieser Ungleichungen gelangt man zu
$x-G\le w\le x+G.$

Damit ist $w\in [x-G,x+G]$. Das Notiert man in der Kurzschreibweise
\begin{equation}
w = x\pm G.
\end{equation}
Mit dem Lineal werden wir z.\,B. eine Länge von $9.8\,\mathrm{mm}$
messen. Die Angabe ist dann
\begin{equation}
L = 9.8\,\mathrm{mm}\pm 0.5\,\mathrm{mm}
= (99.8\pm 0.5)\,\mathrm{mm}.
\end{equation}
Die tatsächliche Abweichung, die wir nicht ermitteln können, ist
\begin{equation}
\Delta x = 99.8\,\mathrm{mm}-100.0\,\mathrm{mm}
= -0.2\,\mathrm{mm}.
\end{equation}

\subsection{Rechnen mit Messwerten}

Beim Rechnen mit Messwerten wird man auf die Frage stoßen, auf wie viele
signifikante Stellen man das Ergebnis angibt. Angenommen wir haben
eine Masse von $10\,\mathrm g$ mit der Grenzabweichung
$1\,\mathrm g$ gemessen. Bei Division durch drei
erhält man das Ergebnis $3\,\mathrm g$. Ausführen der
Umkehroperation bringt $9\,\mathrm g$. Die Abweichung durch das
Runden ist also genauso groß geworden, wie die Grenzabweichung selbst.
Diese Unzulänglichkeit beheben wir sofort, indem wir das Ergebnis
der Rechnung auf eine signifikante Stelle mehr als den Messwert
angeben.

Bei Funktionen mit sehr flachem Anstieg ist der Effekt noch stärker.
Die Umkehrfunktion hat dann einen sehr hohen Anstieg. Kleine
Abweichungen der Argumente führen dann zu großen Abweichungen
der Funktionswerte. Nehmen wir die Zahl vier und die Grenz
abweichung eins. Es ist z.\,B.
\begin{equation}
4^{1/10} = 1.14869 \approx 1.15 \approx 1.1 \approx 1.
\end{equation}
Wendet man die Umkehrfunktion auf die ungenauste Rundung an, so erhält
man den Wert 1. Der nächst bessere ist 2.59 und dann 4.05. Für ein
zuverlässiges Rechenergebnis muss man also zwei Stellen mehr angeben,
als gemessen wurden.

Wenn man bei einer beliebigen Rechnung eine bzw. zwei Stellen mehr
angibt, so kann das Ergebnis der Rechnung unter Umständen scheingenau werden. Man
steht dann vor dem Dilemma: eventuelle Scheingenauigkeit oder eventuell großer
Rundungsfehler. Daher sollte die Grenzabweichung immer mit angegeben
werden.

Beim Rechnen mit Fehlerfortpflanzungsgesetzen besteht das gleiche
Problem. Daher sollte man für die Grenzabweichung, welche sich ergibt,
auch eine Stelle mehr angeben. Die Grenzabweichung kann in den
allermeisten Situationen also mit zwei signifikanten Stellen angegeben
werden.

Die Stellenwerte der minderwertigsten Stellen von Wert und
Grenzabweichung sollten natürlich übereinstimmen.
Wenn man z.\,B. eine Spannung von $9.24\,\mathrm V$ mit der
Grenzabweichung $0.4\,\mathrm V$ errechnet hat, so kann man
anstelle von $(9.2\pm 0.4)\mathrm V$ auch angeben
\begin{equation}
(9.24\pm 0.40)\mathrm V.
\end{equation}
Diese Angabe ist \textit{nicht} scheingenau, da die Grenzabweichung
mit angegeben ist.

\end{document}

