\documentclass[a4paper,11pt,fleqn,twocolumn,twoside]{scrartcl}
\usepackage[utf8]{inputenc}
\usepackage[T1]{fontenc}
\usepackage[ngerman]{babel}
\usepackage{amsmath}
\usepackage{amssymb}
\usepackage{booktabs}
\usepackage{microtype}
\usepackage{lipsum}

\usepackage{libertine}
% \usepackage[scaled=0.80]{DejaVuSans}
\usepackage[scaled=0.76]{DejaVuSansMono}
% \renewcommand\ttdefault{lmvtt}
\usepackage[libertine]{newtxmath}

% Links ein großer Rand wegen Bindekorrektur für Schnellhefter.
\usepackage{geometry}
\geometry{a4paper,left=25mm,right=10mm,top=20mm,bottom=30mm}
\setlength{\columnsep}{5mm}

\newcommand{\ee}{\mathrm e}
\newcommand{\ui}{\mathrm i}
\newcommand{\real}{\operatorname{Re}}
\newcommand{\imag}{\operatorname{Im}}
\newcommand{\uv}[1]{\underline{#1}}
\newcommand{\bv}[1]{\mathbf{#1}}

\newcommand{\N}{\mathbb N}
\newcommand{\Z}{\mathbb Z}
\newcommand{\Q}{\mathbb Q}
\newcommand{\R}{\mathbb R}
\newcommand{\C}{\mathbb C}

\newcommand{\id}{\operatorname{id}}
\newcommand{\sgn}{\operatorname{sgn}}
\newcommand{\Abb}{\operatorname{Abb}}
\newcommand{\unit}[1]{\mathrm{#1}}
\newcommand{\chem}[1]{\mathrm{#1}}
\newcommand{\strong}[1]{\textsf{\textbf{#1}}}
\newcommand{\smallstrong}[1]{{\footnotesize\strong{#1}}}

\usepackage{caption}
\captionsetup{justification=raggedright,singlelinecheck=false}

\begin{document}
\thispagestyle{empty}

\noindent
{\LARGE\sffamily\bfseries
Grundwissen zur\\
Datensicherheit\par}

\vspace{1em}\noindent
September 2019

\tableofcontents

\section{Einführung}
\subsection{Wichtige Prinzipien}

Bei der gewährleistung der Datensicherheit spielen
vier grundlegende Prinzipien eine maßgebliche Rolle.
Das sind die \emph{Redundanz},
die \emph{Integrität}, die \emph{Privatheit} und die
\emph{Authentizität}.

\strong{Redundanz.}
Datenträger haben eine begrenzte Haltbarkeit, können beschädigt werden
oder verloren gehen. Aus diesem Grund möchte man von wichtigen Daten
mehrere Kopien lagern, möglichst an unterschiedlichen Orten.
Dies nennt man Redundanz der Daten.

\strong{Integrität.}
Es kann vorkommen, dass gespeicherte Daten auf einem Datenträger
beschädigt werden. Oder dass jemand oder ein Schadprogramm
unbemerkte Veränderungen an den Daten vornimmt. Den Schutz davor
bezeichnet man als Bewahrung der Integrität.

\strong{Privatheit.}
Daten können von jemanden oder einem Schadprogramm unerlaubt
gelesen werden. Den Schutz davor bezeichnet man als
Bewahrung der Privatheit.

\strong{Authentizität.}
Jemand kann behaupten, Author von bestimmten Daten zu sein
oder nicht zu sein, oder bezichtigt jemanden, Author zu sein
oder nicht zu sein. Den Schutz vor absichtlichen Fehlinformationen
bezeichnet man als Bewahrung der Authentizität.

\section{Datensicherung}
\subsection{Verzeichnisstruktur}

Daten sollten ihrer Speichergröße nach in unterschiedliche Ordner
eingeteilt werden. Z.\,B. sind Ordner wie \emph{Texte}, \emph{Bilder},
\emph{Videos} sinnvoll. Dies bietet mehrere Vorteile. Erstens brauchen
Texte\footnote{Quelltexte, LaTeX, HTML anstelle von Exporten wie PDF}
vergleichsweise verschwindend gering Speicherplatz und lassen
sich sogar noch gut komprimieren. Aus diesem Grund können sehr oft
Sicherheitskopien von Texten gemacht werden, ohne jemals
überfüllten Speicher befürchten zu müssen. Zweitens lässt sich mit
Programmen wie \texttt{grep} bzw. \texttt{findstr} eine Volltextsuche
auf das Verzeichnis anwenden, wobei große Binärdateien aber stören
würden.

\subsection{Versionsgeschichte}

Bei Texten in Bearbeitung sollte man nicht nur eine Kopie
des aktuellen Textes zu speichern, sondern auch dessen frühere
Bearbeitungszustände, dies nennt man
\emph{Versionsgeschichte}. Warum ist das so wichtig?
Nun, es können immer mal Fehler auftauchen, die zu einem
unbemerkten Verlust von Textpassagen führen. Oder aber, jemand
oder ein Schadprogramm greift unerlaubt auf den Text zu und nimmmt
subtile Veränderungen vor. Hat man die Versionsgeschichte zu einem
Text, lassen sich die Versionen des Textes im Nachhinein zur Sicherheit
vergleichen.

\subsection{Deduplikation}

Mit solchen Versionsgeschichten ergibt sich aber noch ein technisches
Problem. Große Dateien wie Bilder oder Videos würde man immer
wieder speichern, auch wenn sich diese nicht mehr verändern.
Dies bezeichnet man als ungewollte Duplikation von Daten.
Zwar wird die Redundanz der Daten damit beträchtlich erhöht,
eigentlich etwas gutes, aber dies findet in einem unkontrollierten
Ausmaß statt und sorgt schnell für über jedes Maß
steigenden Datenverbrauch.

Zur Lösung dieses Problem bieten sich zwei Ansätze an.
Zum einen unterlässt man es, diese Daten thematisch zu ordnen.
Stattdessen bietet sich eine Ordnung nach dem Datum an, z.\,B.
nach dem Jahr oder dem Monat des Jahres. Hat man diese Daten
einmal archiviert, braucht man sie nicht ständig erneut speichern.
Zur thematischen Sortierung schreibt man stattdessen ein Verzeichnis,
das nur auf die Dateien verweist, sie aber nicht selbst speichert.

Zum zweiten gibt es auch extra Backup"=Software, die in der Lage ist,
Daten bei der Sicherung zu deduplizieren. Z.\,B. besitzt jede
Datei einen Hashwert. Hat sich dieser nicht verändert, braucht die
Software nur auf den Inhalt der Datei zu verweisen. Ob die Software
das kann, lässt sich auch leicht experimentell überprüfen.

\subsection{Speichermedien}

\begin{table}
\begin{tabular}{@{}lrrr@{}}
\toprule
\smallstrong{Datenträger} & \smallstrong{Einzelpreis}
& \smallstrong{Speicherplatz} & \smallstrong{Kosten je GB}\\
\midrule
Cloud    &  0,00\,€ &  1\,GB & 0,00\,€\\ 
BD       &  0,75\,€ & 25\,GB & 0,03\,€\\
HDD      & 53,00\,€ &  1\,TB & 0,05\,€\\
\midrule
DVD-R    &  0,32\,€ &4,7\,GB & 0,07\,€\\
DVD-RW   &  1,27\,€ &4,7\,GB & 0,27\,€\\
CD-R     &  0,28\,€ &700\,MB & 0,40\,€\\
\midrule
USB-Stick&  7,00\,€ & 16\,GB & 0,44\,€\\
SD-Karte &  8,99\,€ & 16\,GB & 0,56\,€\\
CD-RW    &  0,70\,€ &700\,MB & 0,99\,€\\
\bottomrule
\end{tabular}
\caption{Preise für Datenspeicher}\label{Datenspeicher}
Quelle: Kizoa, 2017.
\end{table}

Tabelle~\ref{Datenspeicher} zeigt die Kosten für unterschiedliche
Datenspeicher. Hierbei ist zunächst zu beachten, dass Datenträger
durch Beschädigung ab und zu unbrauchbar werden. Solches kommt auch
bei allergrößter Sorgfalt vor. Aus diesem Grund möchte man von
wichtigen Daten immer mehrere Replikate haben, es wäre also gefährlich,
alle Daten auf eine einzelne HDD zu speichern. Ich würde empfehlen,
wichtige Daten redundant auf drei USB-Sticks und drei DVD-R zu
speichern. Zwei bis drei Replikate sollte man an einen anderen
Ort bringen, damit diese im Falle eines Brands erhalten bleiben.

Bei hohen Speichergrößen wird man entsprechend mehrere
redundante Replikate von BD bis HDD wählen.

\section{Kryptografische Methoden}

\subsection{Hashfunktionen}

Zur Gewährleistung der Integrität bedient man sich kryptografischen
Hashfunktionen. Jede Datei besitzt einen eindeutigen Fingerabdruck,
den man Hashwert nennt. Zu einer Datei lässt sich dieser Hashwert
sehr schnell berechnen. Allerdings ist es aufgrund astronomisch großen
Rechenaufwands faktisch unmöglich, zwei Dateien mit dem selben Hashwert
zu finden (ein sogenannter \emph{Kollisionsangriff}), geschweige denn
zu einem gegebenen Hashwert eine Datei zu finden (ein sogenannter
\emph{Urbildangriff}).

Das empfohlene moderne Verfahren zur Erzeugung eines solchen
Hashwerts ist SHA-3-256. Auch SHA-2-256 kann man noch
benutzen. Das ältere Verfahren MD5 konnte man als unsicher
nachweisen, da Kollisionsangriffe gefunden wurden. Allerdings
ist das für uns nicht besonders problematisch, da es auch für MD5
noch keinen praktischen Urbildangriff gibt.

\subsection{Verschlüsselung}

Die Privatheit von Daten lässt sich mit Verschlüsselung
gewährleisten. Mittels eines Schlüssels (eine Zeichenkette)
wandelt man hierbei Klardaten in ein Chiffrat um. Die Klardaten lassen
sich aus dem Chiffrat nur wieder zurückgewinnen, wenn der Schlüssel
bekannt ist.

Moderne Verschlüsselung gibt es in zwei Konzepten, der
\emph{symmetrischen} und der \emph{asymmetrischen} Verschlüsselung.
An dieser Stelle ist zunächst nur die symmetrische Verschlüsselung
von Bedeutung.

Das empfohlene moderne Verfahren zur Verschlüsselung
ist die Blockverschlüsselung AES-128. Eine Blockverschlüsselung
lässt sich in verschiedenen Betriesmodi benutzen. Einige
Betriebsmodi gewährleisten leider nicht die Integrität. Dies lässt
sich experimentell überprüfen, indem man mit einem Hexeditor Bytes
einer verschlüsselten Datei manipuliert. Man könnte Integrität
erzwingen, indem man ein Hashverfahren oder eine Datenkompression
vorschaltet. Eine bessere Möglichkeit ist allerdings der Betriebsmodus
EAX, der extra dafür geschaffen worden ist.

Verschlüsselung ist kein Kinderspielzeug. Kommt der Schlüssel
abhanden, ist die Entschlüsselung der Chiffrats unmöglich.
Es gibt keinen Schlüsseldienst. Man kann auch nicht irgendwo
seinen Personalausweis vorlegen und bekommt dann einen neuen
Schlüssel zugeschickt. Mit dem Schlüssel gehen auch die Klardaten
für immer verloren. Aus diesem Grund sollte man von den Klardaten
besser irgendwo unverschlüsselte Kopien aufbewahren. Auch sollte
man die Kopien der Klardaten nicht wegwerfen, bevor man sich sicher
ist, den Schlüssel korrekt eingegeben zu haben.

\end{document}
