\documentclass[8pt,fleqn,aspectratio=169]{beamer}
\usetheme{Antibes}
\useinnertheme{rectangles}
\useoutertheme{infolines}
\usepackage[utf8]{inputenc}
\usepackage[T1]{fontenc}
\usepackage[ngerman]{babel}

% Patch the look of +, = in arev
\usefonttheme{serif}

\usepackage{arev}
% Patch punctuation to be upright
\DeclareMathSymbol{.}{\mathpunct}{operators}{`.}
\DeclareMathSymbol{,}{\mathpunct}{operators}{`,}

\usepackage{amsmath}
\usepackage{amssymb}
\usepackage{booktabs}

\setbeamertemplate{footline}{%
\begin{beamercolorbox}[ht=3.0ex,dp=1ex]{title in head/foot}
\hfill\footnotesize\insertpagenumber\enspace\enspace\end{beamercolorbox}}

% \documentclass[8pt,aspectratio=169]{beamer}
\setbeamersize{text margin left=6em}
\setbeamersize{text margin right=6em}

\definecolor{bluegreen1}{rgb}{0.0,0.20,0.28}
\definecolor{bluegreen2}{rgb}{0.0,0.20,0.28}
\setbeamercolor*{palette primary}{fg=white,bg=bluegreen1}
\setbeamercolor*{palette secondary}{fg=white,bg=bluegreen2}
\setbeamercolor*{palette tertiary}{fg=white,bg=bluegreen2}
\setbeamercolor{itemize item}{fg=black}
\setbeamercolor{block title}{bg=bluegreen2}
\newcommand{\modest}[1]{{\small\color{gray}#1}}
\hypersetup{colorlinks,urlcolor=magenta}

\usepackage{listings}
\lstset{basicstyle=\ttfamily}
\lstdefinelanguage{IMP}{sensitive=true, keywords={
  true, false, skip, if, then, else, end, while, do, not, and, or}}

\newcommand{\inferrulewidth}{0.4608pt}
\usepackage{ebproof}
\ebproofset{rule margin = 0.5ex}
\ebproofset{label separation = 0.3em}
\ebproofset{right label template={\scriptsize\inserttext}}
\ebproofset{rule thickness=\inferrulewidth}

\newcommand{\unit}[1]{\mathrm{#1}}
\newcommand{\strong}[1]{\textsf{\textbf{#1}}}
\newcommand{\defiff}{\quad:\Longleftrightarrow\quad}
\newcommand{\infernote}[1]{\!\text{\footnotesize #1}}
\renewcommand{\qedsymbol}{\ensuremath{\Box}}
\newcommand{\discharge}[1]{$\sim$#1}
\newcommand{\centerheadline}[1]{%
  \begin{center}\strong{#1}\end{center}}
\newcommand{\parspace}{\vspace{0.8em}}
\newcommand{\cond}{\rightarrow}
\newcommand{\kw}[1]{\textbf{\texttt{#1}}}
\newcommand{\code}[1]{{\texttt{#1}}}
\newcommand{\qb}[1]{[\!\![#1]\!\!]}
\newcommand{\Bool}{\mathrm{Bool}}
\newcommand{\Int}{\mathrm{Int}}
\newcommand{\Loc}{\mathrm{Loc}}
\newcommand{\Aexp}{\mathrm{Aexp}}
\newcommand{\Bexp}{\mathrm{Bexp}}
\newcommand{\Com}{\mathrm{Com}}
\newcommand{\N}{\mathbb N}
\newcommand{\Z}{\mathbb Z}
\newcommand{\R}{\mathbb R}
\newcommand{\Abb}{\mathrm{Abb}}

\title{Verifikation probalistischer Programme}
% \subtitle{Untertitel}
\date{}

\begin{document}

\begin{frame}
\maketitle
\end{frame}

\begin{frame}[fragile]
Im Folgenden sei $\mathtt{random}(a,b)$ ein Zufallszahlengenerator, 
der auf $\{a,\ldots,b\}$ gleichverteilte Zufallszahlen liefert.\pause

\parspace
Betrachten wir hierzu nun das Programm:
\begin{lstlisting}[language=IMP, xleftmargin=\mathindent]
while not x <= 2 do
   x := random(1, 6)
end
\end{lstlisting}
Bei Terminierung des Programms haben wir $P(\mathtt x=1)=\tfrac{1}{2}$
und $P(\mathtt x=2)=\tfrac{1}{2}$.

\parspace
\emph{Aber wie wird dies formal bewiesen?} Es mag ja Programme geben,
die so kompliziert sind, dass nicht mehr klar ersichtlich ist, welche
Verteilung ihre Ausgaben haben.
\end{frame}

\begin{frame}
\centerheadline{Probalistische denotationelle Semantik}
\end{frame}

\begin{frame}
Bei einem gewöhnlichen Kommando $c$ gab es einen deterministischen
Übergang vom Zustand $s\in S$ in den Zustand $s'=C\qb{c}(s)$. Bei der
Zuweisung einer zufällig gewählten Zahl trennt sich der Pfad nun aber
auf, wobei den erreichbaren Zielzuständen eine positive Wahrscheinlichkeit
zukommt. Ziel des Übergangs ist so gesehen eine Zufallsgröße $Y$, wobei
ein Zustand $s'$ mit der Wahrscheinlichkeit $P(Y=s')$ erreicht wird.\pause

\parspace
Sagen wir, bereits der Startzustand $X_0$ sei eine Zufallsgröße, die zunächst
die Einheitsmasse in einem Zustand $s\in S$ haben kann. Zu einem gewöhnlichen
Kommando $c$ gehört die Funktion $f\colon S\to S$ mit $f(s):=C\qb{c}(s)$,
das die Transformation $Y=f(X)$ für Zufallsgrößen $X,Y$ induziert.\pause

\parspace
Wir definieren daher die Menge der Massefunktionen als
\[M := \{p\colon S\cup\{\bot\}\to\R\mid \sum_{s\in S\cup\{\bot\}} p(s) = 1\}.\]
Der probalistische Übergang zu einem Kommando erklärt sich nun als Funktion
\[D\colon\mathrm{Com}\to (M\to M),\]
die die Transformation der Massefunktion durch das Kommando beschreibt.
\end{frame}

\begin{frame}
Zur Transformation einer Zufallsgröße $X$ gilt die allgemeine Beziehung
\[P(f(X)=y) = \sum_{x\in f^{-1}(\{y\})} P(X=x) = \sum_{x\colon f(x)=y} P(X=x).\]\pause
Demnach ergibt sich für jedes Kommando $c$ mit deterministischer Übergangsfunktion
$C\qb{c}$ bezüglich $p(s)=P(X=s)$ der induzierte probalistische Übergang
\[D\qb{c}(p)(s') = P(C\qb{c}(X)=s') = \sum_{s\colon C\qb{c}(s)=s'} P(X=s)
= \sum_{s\colon C\qb{c}(s)=s'} p(s).\]\pause
Die Beziehung gilt, wie gesagt, allerdings nur für die Kommandos,
die \code{random} nicht enthalten. Wir müssen daher noch den
Übergang bei allgemeinen Kommandos klären.
\end{frame}

\begin{frame}
Zu einer Sequenz $c;c'$ gilt offenbar analog
\[D\qb{c;c'}(p) = D\qb{c'}(D\qb{c}(p)).\]\pause
Wir können nachrechnen, dass dies nicht der zuvor gemachten
Feststellung im Fall deterministischer Kommandos $c,c'$ widerspricht.

\parspace
Dazu sei $f := C\qb{c}$ und $g := C\qb{c'}$. Zu zeigen gilt
\[D\qb{c;c'}(p)(s'') \stackrel{\text{def}}=
\hspace{-1.4em}\sum_{s\in (g\circ f)^{-1}(\{s''\})}\hspace{-1.4em} p(s) = 
\!\!\sum_{s'\in g^{-1}(\{s''\})}\;\sum_{s\in f^{-1}(\{s'\})}\hspace{-0.5em} p(s)
\stackrel{\text{def}}= D\qb{c'}(D\qb{c}(p))(s'').\]
\end{frame}

\begin{frame}
Es findet sich
\begin{gather*}
\sum_{s\in (g\circ f)^{-1}(\{s''\})}\hspace{-1em} p(s) = P((g\circ f)(X)=s'')
= P_X((g\circ f)^{-1}(\{s''\})) = P_X(f^{-1}(g^{-1}(\{s''\})))\\
= P_X(f^{-1}(\bigcup\nolimits_{s'\in g^{-1}(\{s''\})}\{s'\}))
= P_X(\bigcup\nolimits_{s'\in g^{-1}(\{s''\})}f^{-1}(\{s'\}))
= \hspace{-1em}\sum_{s'\in g^{-1}(\{s''\})}\hspace{-1em} P_X(f^{-1}(\{s'\})),
\end{gather*}
weil Urbilder disjunkter Zerlegungen wieder solche sind. Und weiter
\[P_X(f^{-1}(\{s'\})) = P_X(\bigcup\nolimits_{s\in f^{-1}(\{s'\})}\{s\})
= \sum_{s\in f^{-1}(\{s'\})} P_X(\{s\})\]
mit $P_X(\{s\}) = P(X^{-1}(\{s\})) = P(X=s) = p(s)$.
\end{frame}

\begin{frame}
Bezüglich einem booleschen Ausdruck $b$ sei
\[(b?p)(s) := [B\qb{b}(s)]p(s) = \begin{cases}
p(s), & \text{wenn $B\qb{b}(s)=\kw{true}$},\\
0 & \text{sonst}.
\end{cases}\]
Es sollte gelten
\[D\qb{\kw{if}\; b\;\kw{then}\;c\;\kw{else}\;c'\;\kw{end}}(p)
= D\qb{c}(b?p) + D\qb{c'}(\lnot b?p).\]\pause
Rechnen wir wieder nach, dass dies nicht dem Sachverhalt für deterministische
Kommandos $c,c'$ widerspricht. Zu zeigen ist also
\[\sum_{s\in f^{-1}(\{s'\})}p(s) = \sum_{s\in C\qb{c}^{-1}(\{s'\})}[B\qb{b}(s)]p(s)
+ \sum_{s\in C\qb{c'}^{-1}(\{s'\})}[B\qb{\lnot b}(s)]p(s)\]
bezüglich $f(s) := C\qb{\kw{if}\; b\;\kw{then}\;c\;\kw{else}\;c'\;\kw{end}}(s)$.
\end{frame}

\begin{frame}
Es findet sich
\begin{align*}
&\sum_{s\in f^{-1}(s')}p(s) = \sum_s [f(s)=s']p(s)\\
= &\sum_s \bigg[(C\qb{c}(s)=s'\land B\qb{b}(s))\lor (C\qb{c'}(s)=s'\land\lnot B\qb{b}(s))\bigg]p(s)\\
= &\sum_s \bigg([C\qb{c}(s)=s'][B\qb{b}(s)] + [C\qb{c'}(s)=s'][\lnot B\qb{b}(s)]\bigg)p(s)\\
= &\sum_s [C\qb{c}(s)=s'][B\qb{b}(s)]p(s) + \sum_s [C\qb{c'}(s)=s'][\lnot B\qb{b}(s)]p(s)\\
= &\sum_{s\in C\qb{c}^{-1}(\{s'\})}[B\qb{b}(s)]p(s) + \sum_{s\in C\qb{c'}^{-1}(\{s'\})} [\lnot B\qb{b}(s)]p(s).
\end{align*}
\end{frame}

\begin{frame}
Zur randomisierten Zuweisung überlegt man sich
\[D\qb{X:=\mathtt{random}(a,b)}(p) = \frac{1}{b-a+1}\sum_{k=a}^b D\qb{X:=k}(p).\]\pause
Wir prüfen, ob dies plausibel ist. Ist $p$ die Einheitsmasse im Zustand
$s_0$, also $p(s):=[s=s_0]$, muss sich die Massefunktion der
Gleichverteilung auf den möglichen Zuweisungen ergeben.\pause{} Es findet sich
\begin{align*}
&\frac{1}{b-a+1}\sum_{k=a}^b D\qb{X:=k}(p)(s')
= \frac{1}{b-a+1}\sum_{k=a}^b\sum_s [C\qb{X:=k}(s)=s']p(s)\\
=& \frac{1}{b-a+1}\sum_{k=a}^b [C\qb{X:=k}(s_0)=s']
= \frac{1}{b-a+1}\sum_{k=a}^b [s_0[X:=k]=s']\\
=& \frac{1}{b-a+1}[s'\in \{s_0[X:=k]\mid a\le k\le b\}].
\end{align*}
\end{frame}

\begin{frame}[fragile]
Die while-Schleife mag man als iterierte if-Anweisung auffassen, also
\[\kw{while}\; b\; \kw{do}\; c\; \kw{end}\]
als Lösung $c'$ der Rekurrenz
\[c' = \kw{if}\; b\; \kw{then}\; c; c'\;
\kw{else}\;\; \kw{skip}\;\; \kw{end}.\]\pause
Diesbezüglich erhält man die Rekurrenz
\begin{align*}
D\qb{c'}(p) &= D\qb{c; c'}(b?p) + D\qb{\kw{skip}}(\lnot b?p)\\
&= D\qb{c'}(D\qb{c}(b?p)) + (\lnot b?p).
\end{align*}
\end{frame}

\begin{frame}
Ende.
\vfill\hfill\modest{Januar 2025}\\
\hfill\modest{Creative Commons CC0 1.0}
\end{frame}

\end{document}
