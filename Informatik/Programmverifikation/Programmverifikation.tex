\documentclass[8pt,fleqn,aspectratio=169]{beamer}
\usetheme{Antibes}
\useinnertheme{rectangles}
\useoutertheme{infolines}
\usepackage[utf8]{inputenc}
\usepackage[T1]{fontenc}
\usepackage[ngerman]{babel}

% Patch the look of +, = in arev
\usefonttheme{serif}

\usepackage{arev}
% Patch punctuation to be upright
\DeclareMathSymbol{.}{\mathpunct}{operators}{`.}
\DeclareMathSymbol{,}{\mathpunct}{operators}{`,}

\usepackage{amsmath}
\usepackage{amssymb}
\usepackage{booktabs}

\setbeamertemplate{footline}{%
\begin{beamercolorbox}[ht=3.0ex,dp=1ex]{title in head/foot}
\hfill\footnotesize\insertpagenumber\enspace\enspace\end{beamercolorbox}}

% \documentclass[8pt,aspectratio=169]{beamer}
\setbeamersize{text margin left=6em}
\setbeamersize{text margin right=6em}

\definecolor{bluegreen1}{rgb}{0.0,0.20,0.28}
\definecolor{bluegreen2}{rgb}{0.0,0.20,0.28}
\setbeamercolor*{palette primary}{fg=white,bg=bluegreen1}
\setbeamercolor*{palette secondary}{fg=white,bg=bluegreen2}
\setbeamercolor*{palette tertiary}{fg=white,bg=bluegreen2}
\setbeamercolor{itemize item}{fg=black}
\setbeamercolor{block title}{bg=bluegreen2}
\newcommand{\modest}[1]{{\small\color{gray}#1}}
\hypersetup{colorlinks,urlcolor=magenta}

\usepackage{listings}
\lstset{basicstyle=\ttfamily}
\lstdefinelanguage{IMP}{sensitive=true, keywords={
  true, false, skip, if, then, else, end, while, do, not, and, or}}

\newcommand{\unit}[1]{\mathrm{#1}}
\newcommand{\strong}[1]{\textsf{\textbf{#1}}}
\newcommand{\defiff}{\quad:\Longleftrightarrow\quad}
\newcommand{\infernote}[1]{\!\text{\footnotesize #1}}
\renewcommand{\qedsymbol}{\ensuremath{\Box}}
\newcommand{\discharge}[1]{$\sim$#1}
\newcommand{\centerheadline}[1]{%
  \begin{center}\strong{#1}\end{center}}
\newcommand{\parspace}{\vspace{0.8em}}
\newcommand{\cond}{\rightarrow}
\newcommand{\kw}[1]{\textbf{\texttt{#1}}}
\newcommand{\code}[1]{\texttt{#1}}
\newcommand{\qb}[1]{[\!\![#1]\!\!]}
\newcommand{\Bool}{\mathrm{Bool}}
\newcommand{\Int}{\mathrm{Int}}
\newcommand{\Loc}{\mathrm{Loc}}
\newcommand{\Aexp}{\mathrm{Aexp}}
\newcommand{\Bexp}{\mathrm{Bexp}}
\newcommand{\Com}{\mathrm{Com}}
\newcommand{\N}{\mathbb N}
\newcommand{\Z}{\mathbb Z}
\newcommand{\Abb}{\mathrm{Abb}}

\title{Programmverifikation}
% \subtitle{Untertitel}
\date{}

\begin{document}

\begin{frame}
\maketitle
\end{frame}

\begin{frame}
\centerheadline{Die Sprache IMP}
\end{frame}

\begin{frame}
Grundlegende Aspekte der Semantik von Programmiersprachen klären wir
anhand von IMP, einer minimalistischen imperativen Programmiersprache,
die uns als Studienobjekt dienen wird.
\end{frame}

\begin{frame}[fragile]

Beispiel für ein Programm in IMP:
\begin{lstlisting}[language=IMP, xleftmargin=\mathindent]
y := 1;
while not n = 0 do
   y := y*x;
   n := n - 1
end
\end{lstlisting}\pause
Es berechnet zu ganzen Zahlen $\code{x},\code{n}$ mit $\code{n}\ge 0$
die Potenz $\code{y} = \code{x}^{\code{n}}$. Klar ersichtlich.\pause

\parspace
Das \emph{schnelle Exponentiation} genannte Programm
\begin{lstlisting}[language=IMP, xleftmargin=\mathindent]
y := 1;
while 2 <= n do
   q := n/2; r := n - 2*q;
   if r = 1 then y := y*x else skip end;
   n := q; x := x*x
end;
if n = 1 then y := y*x else skip end
\end{lstlisting}
berechnet die Potenz ebenfalls. Immer noch klar ersichtlich?

\parspace
Wir würden dies gern \emph{beweisen}.
\end{frame}

\begin{frame}
Zunächst wird eine formale Definition der Sprache IMP unternommen.\pause

\parspace
Es sei $\Int$ die Menge der ganzen Zahlen im Dezimalsystem. Während also
$\Z$ die Menge der ganzen Zahlen bezeichnet, besteht $\Int$ aus konkreten
syntaktischen Darstellungen der Zahlen aus $\Z$.\pause

\parspace
Entsprechend sei $\Bool:=\{\kw{false},\kw{true}\}$, wobei die
Symbole \kw{false} und \kw{true} syntaktische Darstellungen der
Wahrheitswerte seien.
\end{frame}

\begin{frame}
Ein \strong{arithmetischer Ausdruck} $a$, kurz \strong{Term}, sei durch die
folgenden Produktionsregeln festgelegt.
\begin{itemize}
\item Eine ganze Zahl aus $\Int$ ist ein Term.
\item Eine Variable aus $\Loc$ ist ein Term.
\item Sind $a,a'$ Terme, so sind auch $a+a'$, $a-a'$, $a*a'$, $a\,/\,a'$ Terme.
\item Nichts anderes ist ein Term.
\end{itemize}
{\footnotesize Wir fassen die Terme hierbei als abstrakte Syntaxbäume auf.
Auf die genaue Grammatik und die Konstruktion eines Parsers will ich
an dieser Stelle nicht näher eingehen, damit der Fokus auf die
eigentliche Problemstellung nicht verloren geht. Die Operationen sollen
auf die gewöhnliche Art geschrieben werden, die Operatoren dabei die
gewöhnliche Rangfolge besitzen.}
\end{frame}

\begin{frame}
Ein \strong{boolescher Ausdruck} $b$, kurz \strong{Ausdruck}, sei durch
die folgenden Produktionsregeln festgelegt.
\begin{itemize}
\item Die Symbole \kw{false} und \kw{true} sind Ausdrücke.
\item Sind $a,a'$ Terme, so sind $a=a'$ und $a <= a'$ Ausdrücke.
\item Ist $b$ ein Ausdruck, so ist auch $\kw{not}\;b$ ein Ausdruck.
\item Sind $b,b'$ Ausdrücke, so sind auch $b\;\kw{and}\;b$ und $b\;\kw{or}\;b$ Ausdrücke.
\item Nichts anderes ist ein Ausdruck.
\end{itemize}
\end{frame}

\begin{frame}
Ein \strong{Kommando} $c$, auch \strong{Programm} genannt, sei durch die
folgenden Produktionsregeln festgelegt.
\begin{itemize}
\item Das Symbol \kw{skip} ist ein Kommando.
\item Ist $X$ eine Variable und $a$ ein Term, so ist $X:=a$ ein Kommando.
\item Sind $c,c'$ Kommandos, so ist auch $c$; $c'$ ein Kommando.
\item Ist $b$ ein Ausdruck und sind $c,c'$ Kommandos, so ist auch\\
  \kw{if} $b$ \kw{then} $c_1$ \kw{else} $c_2$ \kw{end} ein Kommando.
\item Ist $b$ ein Ausdruck und $c$ ein Kommando, so ist auch\\
  \kw{while} $b$ \kw{do} $c$ \kw{end} ein Kommando.
\item Nichts anderes ist ein Kommando.
\end{itemize}
\end{frame}

\begin{frame}
\centerheadline{Denotationelle Semantik}
\end{frame}

\begin{frame}
Wir bezeichnen
\begin{itemize}
\item mit $\Aexp$ die Menge der arithmetischen Ausdrücke,
\item mit $\Bexp$ die Menge der booleschen Ausdrücke
\item mit $\Com$ die Menge der Kommandos.
\end{itemize}\pause
Will man nun einen arithmetischen Ausdruck auswerten, denkt
man sich dafür eine Funktion $A\colon\Aexp\to\Int$. Es ist
\[A\qb{(1+2)} = 3\]
usw. Nun kann ein arithmetischer Ausdruck aber auch Variablen enthalten,
womit bspw. auch der Wert $A\qb{(\code{x}+1)}$ bezüglich $\code{x}\in\Loc$
bestimmt sein muss.
\end{frame}

\begin{frame}
Die Auswertungsfunktion $A$ wird daher parametrisiert durch den aktuellen
Zustand $s$. Wir haben also $A\colon\Aexp\to (S\to\Int)$
bzw. $A\colon\Aexp\times S\to Z$.\pause

\parspace
Einen \strong{Zustand} $s$ modellieren wir schlicht als die Belegung
der verfügbaren Variablen, also als eine Funktion $s\in S$ mit
$S:=\Abb(\Loc,\Int)$.\pause

\parspace
Zum Beispiel
\[s(X) := \begin{cases}
1, & \text{wenn $X=\code{x}$},\\
2, & \text{wenn $X=\code{y}$},\\
0 & \text{sonst}. 
\end{cases}\]
\end{frame}

\begin{frame}
Wir legen $A\colon\Aexp\to (S\to\Int)$ rekursiv fest gemäß
\begin{gather*}
A\qb{n}(s) := n,\\
A\qb{X}(s) := s(X),\\
A\qb{(a+a')}(s) := A\qb{a}(s) + A\qb{a'}(s),\\
A\qb{(a-a')}(s) := A\qb{a}(s) - A\qb{a'}(s),\\
A\qb{(a*a')}(s) := A\qb{a}(s) \cdot A\qb{a'}(s),\\
A\qb{(a\,/\,a')}(s) := A\qb{a}(s)\,/\, A\qb{a'}(s).
\end{gather*}
bezüglich $n\in\Int$, $X\in\Loc$ und $a,a'\in\Aexp$.
Die Operationen auf der rechten Seite sind hierbei auf die übliche
Art und Weise berechnet. Wir verwenden euklidische Ganzzahldivision
und setzen $n\,/\,0:=0$ für jedes $n\in\Int$.
\end{frame}

\begin{frame}
Wir legen $B\colon\Bexp\to (S\to\Bool)$ rekursiv fest gemäß
\begin{gather*}
B\qb{\kw{false}}(s) := \kw{false},\\
B\qb{\kw{true}}(s) := \kw{true},\\
B\qb{a \;\code{=}\; a'}(s) := (A\qb{a}(s) = A\qb{a'}(s)),\\
B\qb{a \;\code{<=}\; a'}(s) := (A\qb{a}(s)\le A\qb{a'}(s)),\\
B\qb{(\kw{not}\;b)} := \lnot B\qb{b}(s),\\
B\qb{(b\;\kw{and}\;b')} := B\qb{b}(s)\land B\qb{b'}(s),\\
B\qb{(b\;\kw{or}\;b')} := B\qb{b}(s)\lor B\qb{b'}(s),
\end{gather*}
bezüglich $a,a'\in\Aexp$ und $b,b'\in\Bexp$. Die Relationen bzw. Verknüpfungen
auf der rechten Seite werden hierbei auf die übliche Art und Weise berechnet.
\end{frame}

\begin{frame}
Wir legen $C\colon\Com\to (S\rightharpoonup S)$ rekursiv fest gemäß
\begin{gather*}
C\qb{\kw{skip}}(s) := s,\\
C\qb{X:=a}(s) := s[X:=A\qb{a}(s)],\\
C\qb{c; c'}(s) := C\qb{c'}(C\qb{c}(s)) = (C\qb{c'}\circ C\qb{c})(s),\\
C\qb{\text{\kw{if} $b$ \kw{then} $c$ \kw{else} $c'$ \kw{end}}}(s) := \begin{cases}
C\qb{c}(s),& \text{wenn $B\qb{b}(s)=\kw{true}$},\\
C\qb{c'}(s)& \text{sonst}.
\end{cases}\\
C\qb{\text{\kw{while} $b$ \kw{do} $c$ \kw{end}}}(s) := \begin{cases}
\varphi_{b,c}(C\qb{c}(s)),& \text{wenn $B\qb{b}(s)=\kw{true}$},\\
s & \text{sonst}
\end{cases}
\end{gather*}
mit $\varphi_{b,c}(s) := C\qb{\text{\kw{while} $b$ \kw{do} $c$ \kw{end}}}(s)$.\pause

\parspace
Bei $C\qb{c}$ handelt es sich um eine partielle Funktion, da die rekursive Auswertung
der while"=Schleife unter Umständen nicht terminiert -- zum Beispiel bei
\[\texttt{\kw{while} \kw{true} \kw{do} \kw{skip} \kw{end}}.\]
\end{frame}

\begin{frame}
\centerheadline{Die Zusicherungssprache}
\end{frame}

\begin{frame}
\emph{Zusicherungen} sind Aussagen, die man in einem bestimmten Zustand als
erfüllt sehen will. Zum Beispiel ist nach der Ausführung des Kommandos
$\code{y := x*x}$ die Zusicherung $\code{y}\ge 0$
erfüllt.\pause

\parspace
Zusicherungen sind logische Formeln, die der Sprache der einsortigen Logik
erster Stufe enstammen sollen. Die logische Sprache wird passend zur
Programmiersprache IMP definiert, dergestalt dass das Diskursuniversum
$\Int$ sei und die Variablen aus $\Loc$ in den Termen auftauchen dürfen.
Benötigte Funktionssymbole der Signatur $\Int^n\to\Int$ und
Relationssymbole der Signatur $\Int^n\to\Bool$ zu $n\in\N_{\ge 0}$ kann man
je nach Bedarf in ihrer üblichen Bedeutung hinzufügen; die Operatoren
von IMP sollen dabei aber mindestens verfügbar sein.\pause

\parspace
Wir müssen nun allerdings zwischen Variablen $x,y,z$ und Variablen
$\code{x},\code{y},\code{z}\in\Loc$ unterscheiden. Die kursiven tauchen
als freie und gebundene Variablen in den Formeln auf. Die aufrechten
verhalten sich dagegen wie Konstantensymbole, über sie kann nicht
quantifiziert werden.
\end{frame}

\begin{frame}
Die Notation $I,s\models A$ stehe für die Aussage, dass
die Interpretation $I=(\mathcal M,\beta)$ und der Zustand $s$
die Formel $A$ erfüllen. Hierbei ist $\mathcal M$ eine Struktur, die
die Bedeutung der Funktions- und Relationssymbole festlegt,
und $\beta$ eine Belegung der kursiven Variablen. Der Zustand $s$
belegt die aufrechten Variablen. {\footnotesize Die Definition der
Erfüllung geschieht analog zur gewöhnlichen Logik erster Stufe, weshalb
ich sie hier nicht näher ausführen will.}\pause

\parspace
Wir betrachten nur das Modell $\mathcal M_0$, das die Symbole mit ihrer üblichen
Bedeutung versieht. Daher sei $\mathcal I_0:=\{(\mathcal M,\beta)\mid\mathcal M=\mathcal M_0\}$,
das heißt, $\mathcal I_0$ sei die Menge der Interpretationen, deren Struktur
$\mathcal M_0$ ist.\pause

\parspace
Wir schreiben später $s\models A$ als Abkürzung für $I,s\models A$,
sofern sich dadurch keine Zweideutigkeiten ergeben.
\end{frame}

\begin{frame}
\centerheadline{Der Hoare-Kalkül}
\end{frame}

\begin{frame}[fragile]
An ein Kommando $c$ können wir Zusicherungen machen. Wir notieren
$\{A\}c\{B\}$ für die Aussage, dass die Aussage $B$ unter allen Umständen
nach der Ausführung von $c$ erfüllt ist, sofern die Aussage $A$ zuvor
erfüllt war. Man nennt $A$ diesbezüglich eine \emph{Vorbedingung}
und $B$ eine \emph{Nachbedingung} von $c$.\pause

\parspace
Beispiel für ein allgemeingültiges Tripel:
\begin{lstlisting}[language=IMP, xleftmargin=\mathindent, mathescape]
$\{\kw{true}\}$
y := x*x
$\{\code{y}\ge 0\}$
\end{lstlisting}
\end{frame}

\begin{frame}
Die Allgemeingültigkeit des Tripels $\{A\}c\{B\}$ wird dahingehend
definiert als
\[(\models\{A\}c \{B\}) \;:\Leftrightarrow\; \forall I\in\mathcal I_0\colon\forall s\in S\colon
(I,s\models A)\Rightarrow\forall s'\in S\colon C\qb{c}(s)=s'\Rightarrow (I,s'\models B).\]\pause
{\footnotesize In Bezug auf IMP ist der Wert $s'$,
sofern $C\qb{c}(s)$ existiert, eindeutig bestimmt.
Denken wir uns nun den ungültigen Zustand $\bot$, der sich ergeben soll,
wenn $c$ eine nicht terminierende Schleife ist, bekommt man eine totale Funktion
$C\qb{c}\colon S\cup\{\bot\}\to S\cup\{\bot\}$, wobei $C\qb{c}(\bot):=\bot$
gesetzt wird. Diesbezüglich verkürzt sich die Allgemeingültigkeit zu
\[(\models\{A\}c \{B\}) \;\Leftrightarrow\; \forall I\in\mathcal I_0\colon\forall s\in S\colon
(I,s\models A)\Rightarrow (I,C\qb{c}(s)\models B).\]
Hierbeit verlangt man, dass $I,\bot\models A$ für jede Formel $A$ gilt.}
\end{frame}

\begin{frame}
\strong{Bemerkung.}
Wir können die Zusicherungssprache erweitern um eine modale Operation
$\Box_c B$ je Kommando $c$, üblicherweise $[c]B$ geschrieben.\pause{} Deren Semantik sei
\[(I,s\models [c]B) \;:\Leftrightarrow\; \forall s'\in S\colon R_c(s,s')\Rightarrow (I,s'\models B)\]
mit der Zugänglichkeitsrelation $R_c(s,s')\,:\Leftrightarrow\, C\qb{c}(s)=s'$.\pause

\parspace
Diesbezüglich sind $\{A\}c\{B\}$ und $A\to [c]B$ semantisch äquivalent.\pause

\parspace
Die so erweiterte Logik nennt man die \emph{dynamische Logik} von IMP.
Die denotationelle Semantik nimmt hierbei die Rolle einer Kripke"=Semantik
ein, wobei die Zustände die Kripke"=Welten sind.
\end{frame}

\begin{frame}[fragile]
Wir diskutieren nun die \strong{Schlussregeln} des Kalküls.\pause

\parspace
Für Zuweisungen gilt
\[\dfrac{}{\vdash\{A[X:=a]\}X:=a\{A\}},\]
eine Schlussregel ohne Prämissen. Mit $A[X:=a]$ ist hierbei die
Formel gemeint, die aus $A$ hervorgeht, indem jedes Vorkommen von
$X$ in $A$ durch den Term $a$ ersetzt wird.\pause

\parspace
Zum Beispiel erkennt man das Tripel
\begin{lstlisting}[language=IMP, xleftmargin=\mathindent, mathescape]
$\{\code{x}\ge 0\}$
x := x + 1
$\{\code{x}\ge 1\}$
\end{lstlisting}
unschwer als allgemeingültig. Dieses fällt unter die Fittiche der
Regel, indem $X:=\code{x}$, $a:=\code{x}+1$ und $A:=(\code{x}\ge 1)$
gesetzt wird. Damit ergibt sich $A[X:=a]$ zu $\code{x} + 1\ge 1$,
was logisch äquivalent zu $\code{x}\ge 0$ ist.
\end{frame}

\begin{frame}
Zur Gültigkeit der Regel. Wir wollen also
\[\models\{A[X:=a]\}X:=a\{A\}\]
zeigen.\pause{} Sei dazu $s$ fest, aber beliebig. Es gelte $s\models A[X:=a]$.
Des Weiteren gelte $C\qb{X:=a}(s)=s'$. Zu zeigen ist $s'\models A$.\pause

\parspace
Gemäß der denotationellen Semantik gilt $C\qb{X:=a}(s)=s[X:=a]$, womit
wir $s'=s[X:=a]$ haben. Schließlich folgt die Behauptung vermittels
der Äquivalenz
\[(s\models A[X:=a])\;\Leftrightarrow\; (s[X:=a]\models A),\]
die man unschwer als richtig erkennt oder pedantisch per struktureller
Induktion über den Aufbau von $A$ beweisen kann.
\end{frame}

\begin{frame}
Die Regel zu \kw{skip} lautet
\[\dfrac{}{\vdash\{A\}\kw{skip}\{A\}}.\]\pause
Zur Gültigkeit der Regel. Wir wollen $\models\{A\}\kw{skip}\{B\}$ zeigen.
Dazu sei $s$ fest, aber beliebig. Es gelte $s\models A$. Des Weiteren
gelte $C\qb{\kw{skip}}(s)=s'$. Zu zeigen ist $s'\models A$.\pause

\parspace
Gemäß der denotationellen Semantik gilt $C\qb{\kw{skip}}(s)=s$, womit
wir $s'=s$ und somit bereits die Behauptung haben.
\end{frame}

\begin{frame}
Die Regel für Sequenzen von Kommandos lautet
\[\dfrac{\vdash\{A\}c\{B\}\qquad \vdash\{B\}c'\{C\}}{\vdash\{A\}c; c'\{C\}}.\]\pause
Zur Gültigkeit der Regel. Wir wollen $\models\{A\} c; c'\{C\}$ zeigen.
Dazu sei $s$ fest, aber beliebig. Es gelte $s\models A$. Des Weiteren
gelte $C\qb{c;c'}(s)=s''$. Zu zeigen ist $s''\models C$.\\
{\footnotesize Die letzten beiden C's stehen für Unterschiedliches; aber eine
Verwechslung ist ausgeschlossen, denke ich.}\pause

\parspace
Gemäß der denotationellen Semantik existiert $s'=C\qb{c}(s)$ mit
\[C\qb{c;c'}(s) = C\qb{c'}(s')=s''.\]
Aus $s\models A$ und $\models\{A\}c\{B\}$ folgt nun zunächst $s'\models B$.
Mit $\models\{B\}c'\{C\}$ folgt daraufhin $s''\models C$ aus $s'\models B$.
\end{frame}

\begin{frame}
\strong{Literatur}
\begin{itemize}
\item Glynn Winskel:
  \emph{The Formal Semantics of Programming Languages: An Introduction}.
  The MIT Press, 1993.
\item David Harel, Dexter Kozen, Jerzy Tiuryn:
  \emph{Dynamic Logic}. The MIT Press, 2000.
\item Benjamin C. Pierce u.\,a.:
  \href{https://softwarefoundations.cis.upenn.edu/}{\emph{Software Foundations}}.
\end{itemize}
\end{frame}

\begin{frame}
\strong{Anlage}

\parspace
\href{https://github.com/JohnBSmith/JohnBSmith.github.io/tree/master/Informatik/Programmverifikation/imp}{%
IMP-Interpreter} -- Führt ein IMP-Programm gemäß der denotationellen
Semantik aus. In Python verfasst, unter 300 Zeilen Quelltext.
\end{frame}

\begin{frame}
Ende.
\vfill\hfill\modest{Dezember 2024}\\
\hfill\modest{Creative Commons CC0 1.0}
\end{frame}

\end{document}
