\documentclass[9pt]{beamer}
\usetheme{Antibes}
\useinnertheme{rectangles}
\useoutertheme{infolines}
\usepackage[utf8]{inputenc}
\usepackage[T1]{fontenc}
\usepackage[ngerman]{babel}

% patch the look of +, = in arev
\usefonttheme{serif} 

\usepackage{arev}
\usepackage{amsmath}
\usepackage{amssymb}

\setbeamertemplate{footline}{%
\begin{beamercolorbox}[ht=3.0ex,dp=1ex]{title in head/foot}
\hfill\footnotesize\insertpagenumber\enspace\enspace\end{beamercolorbox}}

\definecolor{brown1}{rgb}{0.26,0.14,0}
\definecolor{brown2}{rgb}{0.20,0.10,0}
\definecolor{urlcolor}{rgb}{0,0.4,0.6}
\setbeamercolor*{palette primary}{fg=white,bg=brown1}
\setbeamercolor*{palette secondary}{fg=white,bg=brown2}
\setbeamercolor*{palette tertiary}{fg=white,bg=brown2}
\setbeamercolor{itemize item}{fg=black}

\newcommand{\modest}[1]{{\small\color{gray}#1}}

\newcommand{\unit}[1]{\mathrm{#1}}
\newcommand{\strong}[1]{\textsf{\textbf{#1}}}

\title{Datensicherung, ja. Aber wie?}
\date{Juni 2020}

\begin{document}

\begin{frame}
\maketitle
\end{frame}

\begin{frame}
In die Cloud?\pause

\vspace{1em}
Wohl nicht unbedingt eine gute Idee. Fallstudie:

\vspace{1em}
»Apple bewahrt iCloud-Backups nur sechs Monate auf«.\\
{\footnotesize »iPhone- und iPad-Backups werden 180 Tage nach der letzten
Datensicherung still aus iCloud gelöscht -- auch bei zahlenden Abonnenten.«}\\
In: Heise online (12. Juni 2020).\\
{\tiny\href{https://www.heise.de/news/Apple-bewahrt-iCloud-Backups-nur-sechs-Monate-auf-4782483.html}%
{\color{urlcolor}{https://www.heise.de/news/Apple-bewahrt-iCloud-Backups-nur-sechs-Monate-auf-4782483.html}}}
\end{frame}

\begin{frame}
Ein Backup-System lässt sich einteilen in eine niedere und eine
höhere Ebene.\pause

\vspace{1em}
\strong{Niedere Ebene} -- Umgang mit den eigentlichen Daten:
\begin{itemize}
\item Verpackung, ggf. Datenkompression
\item Schutz vor Verlust durch Redundanz (Duplikate)
\item Schutz der Integrität
\item Wahrung der Vertraulichkeit
\end{itemize}\pause

\vspace{1em}
\strong{Höhere Ebene} -- höhere logische Prozesse:
\begin{itemize}
\item Sicherungsarten, Deduplikation
\item Versionsverwaltung
\item Integrität der Verwaltung
\end{itemize}
\end{frame}

\begin{frame}
Wir wollen uns hier auf die niedere Ebene beschränken.\pause

\vspace{1em}
Klassisches Anliegen: private Langzeitarchivierung.
\end{frame}

\begin{frame}
\begin{center}
\strong{Auftakt --- Poor Man's Backup}
\end{center}
\end{frame}

\begin{frame}
Naiver Ansatz: Schreibe Verzeichnis auf einen Datenträger.\pause

\vspace{1em}
\strong{Probleme:}
\begin{itemize}
\item Fremdes Dateisystem soll Metadaten nicht anfassen.
\item Einige Dateisysteme unterstützen keine langen Dateinamen
  bzw. kein Unicode.
\item Bestimmte Programme erfordern die Verpackung zu einem Paket.
\end{itemize}
\end{frame}

\begin{frame}
Naive Lösung: Verpacke Verzeichnis als »\texttt{Paket.tar.xz}«.\pause

\vspace{2em}
\strong{Problem:}
\begin{itemize}
\item Schon ein beschädigtes Bit kann dazu führen, dass sich
  das Archiv nicht mehr entpacken lässt.
\end{itemize}
\end{frame}

\begin{frame}
Verpacke ohne Kompression als »\texttt{Paket.tar}«.\pause

\vspace{2em}
\strong{Problem:}
\begin{itemize}
\item Beschädigung fällt nicht auf.
\end{itemize}
\end{frame}

\begin{frame}
Berechne kryptografischen Hashwert des Pakets.\pause

\vspace{2em}
\strong{Problem:}
\begin{itemize}
\item Erlaubt keine Rückschlüsse auf Ort und Umfang der
  Beschädigungen.
\end{itemize}
\end{frame}

\begin{frame}
Verpacke als »Paket.zip«. 

\vspace{1em}
Das ZIP-Format schützt vor Datenverlust, indem es
\begin{itemize}
\item für jede Datei einen CRC-32-Prüfwert speichert,
\item das Inhaltsverzeichnis zweimal enthält.
\end{itemize}\pause

\vspace{1em}
\strong{Probleme:}
\begin{itemize}
\item Prüfwert nur pro Datei, daher ungenau.
\item Leider kein Prüfwert der komprimierten Daten.
\item Eventuell Komplikationen, falls der Teil mit dem
  Inhaltsverzeichnis verloren geht.
\end{itemize}
\end{frame}

\begin{frame}
Ansatz: Das Tar-Format benutzen und die Hashwerte aller Dateien
\emph{vor} dem Verpacken berechnen lassen und anbeifügen.\pause

\vspace{1em}
\strong{Verbleibendes Problem:}
\begin{itemize}
\item Lokalisierung von Beschädigungen nicht genau genug.
Betrachte z.\,B. sehr große Dateien.
\end{itemize}
\end{frame}

\begin{frame}
\begin{center}
\strong{Kurze Pause}
\end{center}
\end{frame}

\begin{frame}
Was für Beschädigungen können eigentlich auftreten?\pause

\vspace{1em}
Das sind:
\begin{itemize}
\item Uniformer Bit-Rot: Verlust einzelner Bits oder Bytes,
  gleichverteilt über den gesamten Datenträger. Thermisches Rauschen,
  allgegenwärtige Ladungslecks, etc.
\item Burst-Fehler: Verlust mehrerer Bits oder Bytes hintereinander.
\item Wegfall ganzer Blöcke.
\item Zerstörung des gesamten Datenträgers, z.\,B. durch Wegfall
  zentraler Steuerungseinheiten.
\item Verschiebung: Daten an falscher Stelle, z.\,B. durch
  Synchronisationsfehler.
\end{itemize}
\end{frame}

\begin{frame}
Betrachten wir zunächst den uniformen Bit-Verlust. Angenommen, ein Byte
ist nach 10 Jahren mit einer sehr hohen Wahrscheinlichkeit $p$, sagen
wir $p=0.999999$, noch intakt, d.\,h. das Risiko für Beschädigung beträgt
1 zu 1\,Mio.\pause

\vspace{1em}
Bei 1\,MB Daten hat man $N=10^6$ Bytes und nur noch eine
Wahrscheinlichkeit $p' = p^N \approx 0.37$ für die Intaktheit aller Daten.
Das ist schon ein Risiko von ca. 2/3.\pause

\vspace{1em}
Man beachte: Komprimierte Bilder können z.\,B. schon bei einem einzigen
Bitfehler unlesbar werden.
\end{frame}

\begin{frame}
Angenommen wir haben nun zwei Duplikate einer Datei,
beide sind jedoch beschädigt.\pause

\vspace{1em}
Die Hashwerte nützen uns zunächst nichts.\pause

\vspace{1em}
Mit einer gewissen Wahrscheinlichkeit kann man zumindest alle
Beschädigungen lokalisieren, -- nämlich dann wenn die Daten an den
Stellen nicht mehr übereinstimmen. Allerdings gibt es selbst dann
irgendwann eine kombinatorische Explosion an Möglichkeiten für
die Fehlerkorrektur.
\end{frame}

\begin{frame}
Dieser Problematik kann man leicht ausweichen. Wir fügen jedem
Datenblock einen Paritätswert hinzu. Eignen tut sich dafür ein
CRC"=32"=Prüftwert (32 Bit lang) oder ein kryptografischer
Hashwert (128 bis 256 Bit lang).\pause

\vspace{1em}
Pro 256 Byte ein CRC"=32"=Prüfwert vergrößert den Speicherbedarf
lediglich um $4/256 \approx 1.6\,\%$.\pause

\vspace{1em}
Ermöglicht bei geringen Beschädigungen sogar eine
Fehlerkorrektur ohne ein Duplikat.
\end{frame}

\begin{frame}
Man beachte: CRC-32 erkennt einen größeren Fehler nur mit einer
Wahrscheinlichkeit von höchstens $1-\tfrac{1}{2^{32}}$.\pause

\vspace{1em}
$\rightarrow$ Fasse eine feste Zahl an Blöcken zu einem Bündel
zusammen und berechne für dieses zusätzlich einen kryptografischen
Hashwert. Bei einer idealen Hashfunktion wird ein Fehler dann mit einer
Wahrscheinlichkeit $p\approx 1-\tfrac{1}{2^{256}}$ erkannt. Man darf
nun $p=1$ setzen.
\end{frame}

\begin{frame}
Die Integrität ist nun mehr oder weniger geschützt, die
Vertraulichkeit jedoch nicht. Verteilt man Datenträger mit Duplikaten
auf unterschiedliche Orte, nimmt die Wahrscheinlichkeit für einen
Diebstahl zu.\pause

\vspace{1em}
Schlimmer: Man bemerkt nicht, wenn jemand in Besitz der Daten
gelangt ist.\pause

\vspace{1em}
$\rightarrow$ Daten vor dem Speichern verschlüsseln.\pause

\vspace{1em}
\strong{Problem:}
\begin{itemize}
\item Blockchiffre verträgt sich in einigen Betriebsmodi
  nicht mit Fehlerkorrektur.
\end{itemize}
\end{frame}

\begin{frame}
\strong{Lösung:}
\begin{itemize}
\item Benutze eine Stromchiffre,
\item oder Blockchiffre im Counter-Mode.
\end{itemize}\pause
\vspace{1em}
$\rightarrow$ Schlüsselstrom wird mit den Klardaten bitweise
XOR-verknüpft. Folglich führen fehlerhafte Bits im Chiffrat
lediglich zu entsprechenden fehlerhaften Bits in den Klardaten.
Verträgt sich mit der Fehlerkorrektur. Verträgt sich auch mit
der Korrektur von Verschiebungsfehlern: Schlüsselstrom z.\,B. einfach
solange verschieben, bis der CRC-32-Prüfwert eines Datenblocks
stimmt.
\end{frame}

\begin{frame}
Vorsicht: Die Nonce darf nicht abhanden kommen.\pause

\vspace{1em}
Idee: Einmalschlüssel anstelle von Nonce benutzen.\pause

\vspace{1em}
Leider heikel. Idee: Einmalschlüssel bei der Verschlüsselung
von einem kryptografischen Zufallszahlengenerator erzeugen lassen.
\end{frame}

\begin{frame}
\strong{Welche Chiffre konkret?}

\vspace{1em}
Z.\,B. die Stromchiffre ChaCha20 von Daniel
Bernstein. Ein ziemlich minimalistisches Verfahren (der eigentliche
Algorithmus beträgt weniger als 100 Zeilen Rust oder Python).
\end{frame}

\begin{frame}
Ende.
\vfill\hfill
\modest{Creative Commons CC0 1.0}
\end{frame}



\end{document}


