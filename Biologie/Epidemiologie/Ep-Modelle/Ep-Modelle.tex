\documentclass[a4paper,10pt,fleqn,twocolumn,dvipdfmx]{scrartcl}
\usepackage[utf8]{inputenc}
\usepackage[T1]{fontenc}
\usepackage[ngerman]{babel}
\usepackage{microtype}

% \usepackage{mathptmx}
\usepackage{libertine}
\usepackage[libertine,smallerops]{newtxmath}
\addtokomafont{disposition}{\rmfamily}
% \renewcommand\ttdefault{lmtt}

\usepackage{amsmath}
\usepackage{amssymb}
\usepackage{booktabs}
\usepackage{color}
\definecolor{c1}{RGB}{60,0,40}

\usepackage{geometry}
\geometry{a4paper,left=25mm,right=12mm,top=20mm,bottom=28mm}
\setlength{\columnsep}{6mm}

\numberwithin{equation}{section}

\usepackage[colorlinks=true,linkcolor=black,citecolor=black]{hyperref}
\newcommand{\ee}{\mathrm e}
\newcommand{\unit}[1]{\mathrm{#1}}
\newcommand{\strong}[1]{{\sffamily\bfseries #1}}

\begin{document}

\thispagestyle{empty}

\section*{Epidemiologische Grundmodelle}

\tableofcontents

\section{Das klassische SIR-Modell}
\subsection{Modelldefinition}

Die Population von $N$ Individuen wird aufgespalten in die Anteile
$S,I,R$ mit%
\begin{equation}
S+I+R=N.
\end{equation}
Die Ausbreitung der Krankheit verläuft nun gemäß%
\begin{align}
S' &= -\tfrac{1}{N} \beta SI,\\
I' &= \tfrac{1}{N} \beta SI - \gamma I,\\
R' &= \gamma I.
\end{align}
Hierbei handelt es sich um ein autonomes System von gewöhnlichen
Differentialgleichungen für $S(t)$, $I(t)$, $R(t)$.
Wie bei jedem autonomen System ist durch die Gleichungen ein
dynamisches System beschrieben.

Weil dieses Modell noch keine demografische Dynamik enthält,
ist $N$ eine Konstante. Günstig ist daher die Verwendung der
relativen Größen $s:=S/N$, $i:=I/N$, $r:=R/N$. Das System nimmt
damit die Gestalt%
\begin{align}
\label{eq:sir-s} s' &= -\beta si,\\
\label{eq:sir-i} i' &= \beta si - \gamma i,\\
\label{eq:sir-r} r' &= \gamma i
\end{align}
an.

\subsection{Beziehung zur Reproduktionszahl}

Die \emph{effektive Reproduktionszahl} ist definiert gemäß
$R_q := R_0 s$, wobei $R_0$ die \emph{Basisreproduktionszahl} ist.
Hierbei ist $q:=1-s$, so dass man $R_q=R_0$ für $q=0$ erhält. Man nennt
$q$ den immunen Anteil der Population.

Die Reproduktionszahl steht natürlich im Zusammenhang mit dem weiteren
Verlauf der Epidemie. Zur Herleitung fragen wir, unter welchem Umstand
sich die Epidemie bei $i\ne 0$ nicht weiter ausbreitet. Dazu muss
$i'=0$ sein. Eingesetzt in \eqref{eq:sir-i} bedeutet das%
\begin{equation}
0 = (\beta s - \gamma) i \;\Leftrightarrow\; 0 = \beta s - \gamma
\;\Leftrightarrow\; s = \gamma/\beta.
\end{equation}
Nun bedeutet $i'=0$ aber auch $R_q=1$, und daher %
\begin{equation}
1 = R_q = R_0 s = R_0 \tfrac{\gamma}{\beta}.
\end{equation}
Wir finden die Beziehung
\begin{equation}
R_0 = \beta/\gamma.
\end{equation}

\subsection{Exponentielle Anfangsphase}

Ist am Anfang der Epidemie $s\approx 1$, verläuft die Ausbreitung
der Krankheit exponentiell. Dies lässt sich leicht einsehen. Setzt
man $s=1$ in \eqref{eq:sir-i} ein, ergibt sich nämlich%
\begin{equation}\label{eq:i-ode-exp}
i' = \lambda i,\quad \lambda := \beta-\gamma.
\end{equation}
Das ist die Dgl. von Exponentialfunktionen, d.\,h. der Lösungen
$i(t) = i(0)\,\ee^{\lambda t}$.

Mathematisch kann man das noch ein wenig genauer herausarbeiten.
Dazu werden die Größen zum Zustandsvektor $x=(s,i,r)$ zusammengefasst.
Das System ist dann abstrakt beschrieben in der Form $x'=f(x)$. Ist
nun $x_0$ eine Ruhelage des dynamischen Systems, genügt die Dynamik
in der Nähe dieser Ruhelage unter gewissen Voraussetzungen
näherungsweise dem linearen System $x' = J_0 x$. Hierbei ist
$J_0:=Df(x_0)$ die Jacobimatrix von $f$ an der Stelle $x_0$.

Eine Ruhelage ist definiert durch $x'(t)=0$ und wird bei den
epidemiologischen Modellen auch als Gleichgewicht (engl.
\emph{equilibrium}) bezeichnet. Dieses kann stabil oder instabil sein.

Der Funktionswert $f(x)$ ist hier die Zusammenfassung der rechten
Seiten der Gleichungen \eqref{eq:sir-s} bis \eqref{eq:sir-r} zu einem
Tupel. Darin darf \eqref{eq:sir-r} entfallen, redundant weil von
den anderen beiden Gleichungen entkoppelt. Das macht%
\begin{equation}
f(\begin{pmatrix}s\\ i\end{pmatrix})
:= \begin{pmatrix}-\beta s i\\ \beta si - \gamma i\end{pmatrix}.
\end{equation}
Somit ergibt sich
\begin{equation}
Df = \begin{pmatrix}
-\beta i & -\beta s\\
\beta i & \beta s - \gamma
\end{pmatrix}.
\end{equation}
Auswertung der Matrix an der Ruhelage $(s,i)=(1,0)$ führt zum linearen
System%
\begin{equation}
\begin{pmatrix}
s'\\ i'
\end{pmatrix}
= \begin{pmatrix}
0 & -\beta\\
0 & \beta-\gamma
\end{pmatrix}
\begin{pmatrix}
s\\ i
\end{pmatrix}.
\end{equation}
Das System enthält \eqref{eq:i-ode-exp} wie gewünscht.

Es gibt noch eine zweite Ruhelage, nämlich $(s,i)=(0,0)$. Die
Auswertung der Matrix bei dieser führt zum System
\begin{equation}
\begin{pmatrix}
s'\\ i'
\end{pmatrix}
= \begin{pmatrix}
0 & 0\\
0 & -\gamma
\end{pmatrix}
\begin{pmatrix}
s\\ i
\end{pmatrix}.
\end{equation}
Demnach verläuft das Abklingen der Epidemie näherungsweise exponentiell
gemäß $i' = -\gamma i$. Zum gleichen Ergebnis gelangt man durch
Einsetzen von $s\approx 0$ in \eqref{eq:sir-i}.

\subsection{Peakhöhe der Infektiösen}

Mit \eqref{eq:sir-i} und \eqref{eq:sir-s} gewinnt man die Dgl.%
\begin{equation}
\frac{\mathrm di}{\mathrm ds} = \frac{i'(t)}{s'(t)}
= \frac{\beta si - \gamma i}{-\beta si} = -1 + \frac{1}{R_0 s},
\end{equation}
deren Lösung
\begin{equation}
i-i_0 = - (s-s_0) + \tfrac{1}{R_0}(\ln s - \ln s_0)
\end{equation}
man durch Separation der Variablen erhält. Umformung dieser
Gleichung liefert%
\begin{equation}\label{eq:const-of-motion}
(i+s)R_0-\ln s = (i_0+s_0)R_0-\ln s_0 = \mathrm{const}.
\end{equation}
D.\,h. die Funktion
\begin{equation}
F(t,(s,i)) := (i+s)R_0 - \ln s
\end{equation}
ist eine erstes Integral der Bewegung. Eine nichtkonstante, stetig
differenzierbare, skalarwertige Funktion $F(t,x)$
heißt \emph{erstes Integral der Bewegung} eines Systems
von Differentialgleichungen erster Ordnung, $x'=f(t,x)$,
wenn $F$ lokal konstant für jede Lösung $x(t)$ ist, d.\,h.
$\tfrac{\mathrm d}{\mathrm dt}F(t,x(t))=0$.

Da hier ein dynamisches System vorliegt, bedeutet dies, dass
das erste Integral der Bewegung auf den
Phasenraum"=Trajektorien $x(t):=(s(t),i(t))$ konstant ist.

Bei $i_\mathrm{max}$ muss $R_0 s = 1$ bzw. $s = 1/R_0$ sein, womit wir
\begin{equation}
i_\mathrm{max} = i_0 + s_0 - \tfrac{1}{R_0} - \tfrac{1}{R_0}\ln(R_0 s_0)
\end{equation}
aus \eqref{eq:const-of-motion} erhalten.

\begin{table}
\begin{tabular}{cll}
\toprule
\strong{Größe} & \strong{Einheit} & \strong{Erklärung}\\
\midrule
$t$ & $\unit{d}$ & Zeit in Tagen\\
$N$ & $\unit{indv}$ & Population\\
$S$ & $\unit{indv}$ & Anfällige, engl. \emph{susceptibles}\\
$E$ & $\unit{indv}$ & Exponierte, engl. \emph{exposed}\\
$I$ & $\unit{indv}$ & Infektiöse, engl. \emph{infectious}\\
$R$ & $\unit{indv}$ & Erholte, engl. \emph{recovered}\\
$\alpha$ & $1/\unit d$ & Kehrwert der Latenzzeit\\
$\beta$ & $1/\unit d$ & Transmissionsrate\\
$\gamma$ & $1/\unit d$ & Erholungsrate\\
$\mu$ & $1/\unit d$ & Sterberate\\
\bottomrule
\end{tabular}
\caption{Größen der Modelle SIR und SEIR.}
\end{table}

\mbox{}
\vfill

\noindent
{\small Dieser Text steht unter der Lizenz\\
Creative Commons CC0.}

\end{document}
