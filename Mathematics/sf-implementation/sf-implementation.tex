\documentclass[a4paper,10pt,fleqn]{scrartcl}
\usepackage[utf8]{inputenc}
\usepackage[T1]{fontenc}
\usepackage[english]{babel}
\usepackage{amsmath}
\usepackage{amssymb}
\usepackage{lipsum}
\usepackage{listings}
\lstset{basicstyle=\ttfamily}

\usepackage{color}
\definecolor{c1}{RGB}{00,40,60}
\usepackage[colorlinks=true,linkcolor=c1,citecolor=c1]{hyperref}

\usepackage{arev}

\usepackage{geometry}
\geometry{a4paper,left=34mm,right=34mm,top=30mm,bottom=40mm}

\numberwithin{equation}{section}

\newcommand{\ee}{\mathrm e}
\newcommand{\ui}{\mathrm i}
\newcommand{\real}{\operatorname{Re}}
\newcommand{\imag}{\operatorname{Im}}
\newcommand{\uv}[1]{\underline{#1}}
\newcommand{\bv}[1]{\mathbf{#1}}

\newcommand{\N}{\mathbb N}
\newcommand{\Z}{\mathbb Z}
\newcommand{\Q}{\mathbb Q}
\newcommand{\R}{\mathbb R}
\newcommand{\C}{\mathbb C}

\newcommand{\id}{\operatorname{id}}
\newcommand{\sgn}{\operatorname{sgn}}
\newcommand{\Abb}{\operatorname{Abb}}
\newcommand{\unit}[1]{\mathrm{#1}}
\newcommand{\chem}[1]{\mathrm{#1}}
\newcommand{\strong}[1]{\textsf{\textbf{#1}}}

\begin{document}
\thispagestyle{empty}

\section*{Implementation of special functions}

\tableofcontents

\section{Elliptic integrals and related}
\subsection{Complete elliptic integral of the first kind}
The arithmetic geometric mean $\operatorname{agm}(x,y)$
is defined and calculated as the limit of the iteration:
\begin{align}
\begin{bmatrix}
a_0\\
g_0
\end{bmatrix}
:=
\begin{bmatrix}
x\\
y
\end{bmatrix},
\quad
\begin{bmatrix}
a_{n+1}\\
g_{n+1}
\end{bmatrix}
:=
\begin{bmatrix}
\tfrac{1}{2}(a_n+g_n)\\
\sqrt{a_n g_n}
\end{bmatrix}.
\end{align}
The iteration can be stopped if $a_n$ and $g_n$ are sufficiently
close to each other. If this condition fails for some reason,
to have a more stable algorithm, a maximum number $n_{\mathrm{max}}$ of iterations
should be specified. Numerical experiments show that $n_{\mathrm{max}}=14$
is enough for 64-bit floating point arithmetic with
\begin{equation}
(x,y)\in [10^{-307},10^{308}]\times [10^{-307},10^{308}].
\end{equation}
The complete elliptic integral of the first kind is defined as
\begin{equation}\label{eq:definitionK}
K(m) := \int_0^{\pi/2} \frac{\mathrm d\theta}{\sqrt{1-m\sin^2\theta}}.
\end{equation}
It is calculated by the arithmetic
geometric mean:
\begin{equation}
K(m) = \frac{\pi}{2\operatorname{agm}(1,\sqrt{1-m})}.
\end{equation}
The domain of $K(m)$ is $m<1$, but \eqref{eq:definitionK} allows more generally
\begin{equation}
m\in\C\setminus\{x\in\R\mid x\ge 1\}.
\end{equation}
The relation between the arithmetic geometric mean and $K(m)$ holds
even for complex numbers, but one has to take care of the branch
cut of the square root.

\newpage
\subsection{Complete elliptic integral of the second Kind}

The complete elliptic integral of the second kind is defined as
\begin{equation}\label{eq:definition-E}
E(m) := \int_0^{\pi/2} \sqrt{1-m\sin^2\theta}\;\mathrm d\theta.
\end{equation}
It is calculated by
\begin{equation}
E(m) = \frac{\pi}{2}\lim_{n\to\infty}\frac{a_n}{x_n},
\end{equation}
where
\begin{equation}
\begin{bmatrix}
x_{n+1}\\ y_{n+1}\\ a_{n+1}\\ b_{n+1}
\end{bmatrix}
= \begin{bmatrix}
\tfrac{1}{2}(x_n+y_n)\\
\sqrt{x_n y_n}\\
\tfrac{1}{2}(a_n+b_n)\\
\tfrac{a_n y_n+b_n x_n}{x_n+y_n}
\end{bmatrix},\quad
\begin{bmatrix}
x_0\\ y_0\\ a_0\\ b_0
\end{bmatrix}
= \begin{bmatrix}
1\\ \sqrt{1-m}\\ 1\\ 1-m
\end{bmatrix}.
\end{equation}


\newpage
\section{Polynomials and related}
\subsection{Associated Legendre functions}

The associated Legendre functions $P_n^m(x)$ are solutions
of the general Legendre equation%
\begin{equation}
(1-x^2) \frac{\mathrm d^2}{\mathrm dx^2} P_n^m(x)
- 2x \frac{\mathrm d}{\mathrm dx} P_n^m(x)
+ \bigg[(n+1)n-\frac{m^2}{1-x^2}\bigg] P_n^m(x)=0.
\end{equation}
In case of $m=n$ one has the recurrence
\begin{equation}
P_0^0(x)=1, \quad P_n^n(x) = -(2n-1)\sqrt{1-x^2}P_{n-1}^{n-1}(x),
\end{equation}
which has the solution
\begin{equation}
P_n^n(x) = (-1)^n (2n-1)!! (1-x^2)^{n/2}.
\end{equation}
By
\begin{equation}
(2n-1)!! = (-2)^n \frac{\sqrt{\pi}}{\Gamma(\tfrac{1}{2}-n)}
\end{equation}
we obtain
\begin{equation}
P_n^n(x) = \frac{\sqrt{\pi}}{\Gamma(\tfrac{1}{2}-n)} (2\sqrt{1-x^2})^n.
\end{equation}
In case of $m=n-1$ one has
\begin{equation}
P_n^{n-1}(x) = (2n-1)xP_{n-1}^{n-1}(x).
\end{equation}
Now we use the recurrence
\begin{equation}
(n-m) P_n^m(x) = (2n-1)xP_{n-1}^m(x)-(n-1+m)P_{n-2}^m(x)
\end{equation}
to get $n\ge m$ down to $m$. The recurrence will be
converted into a bottom up iteration like in the calculation of the Fibonacci
sequence. We can remove quadratic complexity by this trick.

This leads us to the following algorithm:
\begin{lstlisting}[mathescape=true]
function $P_n^m(x)$
    if $n=m$
        return $\frac{\sqrt{\pi}}{\Gamma(\tfrac{1}{2}-n)} (2\sqrt{1-x^2})^n$
    else if $n-1=m$
        return $(2n-1)x P_m^m(x)$
    else
        let mut $a := P_m^m(x)$
        let mut $b := P_{m+1}^m(x)$
        for $k$ in $[m+2..n]$
            let $h := \frac{(2k-1)xb-(k-1+m)a}{k-m}$
            $a:=b$; $b:=h$
        end
        return $b$
    end
end
\end{lstlisting}

\section{Gamma function and related}
\subsection{Gamma function}

The easiest way to compute an approximate value of the gamma function
is Stirling's approximation
\begin{equation}\label{eq:Stirling}
\Gamma(x+1) \approx \sqrt{2\pi x}\,\bigg(\frac{x}{e}\bigg)^x
= \sqrt{2\pi}\,x^{x+1/2}\ee^{-x}.
\end{equation}
This approximation is also an asymptotic formula. That means, the
relative error gets smaller as $x\to\infty$. We can profit from
this property if me make $x$ larger by the functional equation
\begin{equation}
\Gamma(x+1) = x\,\Gamma(x).
\end{equation}
Performing the functional equation one time yields again an easy
formula, because in this case a simplification is possible:
\begin{equation}\label{eq:Stirling-a}
\Gamma(x) \approx \sqrt{2\pi}\,x^{x-1/2}\ee^{-x}.
\end{equation}
Formula \eqref{eq:Stirling-a} is more
precise than \eqref{eq:Stirling}. Iteration of this technique yields
\begin{equation}
\Gamma(x) \approx \sqrt{2\pi}\,\frac{(x+n)^{x+n-1/2}}{\ee^{x+n}}
\prod_{k=0}^{n-1} \frac{1}{x+k}.
\end{equation}
We can obtain more and more precise values, but convergence
as $n\to\infty$ is very slow.

More precise than \eqref{eq:Stirling-a} is
\begin{equation}\label{eq:Stirling-b}
\Gamma(x) \approx \sqrt{2\pi}\,x^{x-1/2}\exp\bigg(-x+\frac{1}{12x}\bigg).
\end{equation}
More precise than \eqref{eq:Stirling-b} is
\begin{equation}\label{eq:Nemes}
\Gamma(x) \approx \sqrt{\frac{2\pi}{x} }
\bigg(\frac{1}{e} \bigg(x + \frac{1}{12x - \frac{1}{10x}}\bigg)\bigg)^x,
\end{equation}
found by G. Nemes. More formulas are discussed in \cite{Luschny}.

The standard algorithm for 64-bit floating point numbers is
Lanczos approximation, see \cite{Pugh}.

\begin{thebibliography}{x}
\bibitem{Luschny}
Peter Luschny: \emph{Approximation Formulas for the Factorial Function}.
\bibitem{Pugh}
Glendon Pugh (2004): \emph{An analysis of the Lanczos Gamma approximation}.
\end{thebibliography}

\end{document}
