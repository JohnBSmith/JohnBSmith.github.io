\documentclass[a4paper,10pt,fleqn]{scrartcl}
\usepackage[utf8]{inputenc}
\usepackage[T1]{fontenc}
\usepackage[english]{babel}
\usepackage{amsmath}
\usepackage{amssymb}
\usepackage{lipsum}

\usepackage{color}
\definecolor{c1}{RGB}{00,40,60}
\usepackage[colorlinks=true,linkcolor=c1]{hyperref}

\usepackage{arev}

\usepackage{geometry}
\geometry{a4paper,left=34mm,right=34mm,top=30mm,bottom=40mm}

\newcommand{\ee}{\mathrm e}
\newcommand{\ui}{\mathrm i}
\newcommand{\real}{\operatorname{Re}}
\newcommand{\imag}{\operatorname{Im}}
\newcommand{\uv}[1]{\underline{#1}}
\newcommand{\bv}[1]{\mathbf{#1}}

\newcommand{\N}{\mathbb N}
\newcommand{\Z}{\mathbb Z}
\newcommand{\Q}{\mathbb Q}
\newcommand{\R}{\mathbb R}
\newcommand{\C}{\mathbb C}

\newcommand{\id}{\operatorname{id}}
\newcommand{\sgn}{\operatorname{sgn}}
\newcommand{\Abb}{\operatorname{Abb}}
\newcommand{\unit}[1]{\mathrm{#1}}
\newcommand{\chem}[1]{\mathrm{#1}}
\newcommand{\strong}[1]{\textsf{\textbf{#1}}}

\begin{document}
\thispagestyle{empty}

\section*{Implementation of special functions}

\tableofcontents

\section{Elliptic integrals and related}
\subsection{Complete elliptic integral of the first kind}
The arithmetic geometric mean $\operatorname{agm}(x,y)$
is defined and calculated as the limit of the iteration:
\begin{align}
\begin{bmatrix}
a_0\\
g_0
\end{bmatrix}
:=
\begin{bmatrix}
x\\
y
\end{bmatrix},
\quad
\begin{bmatrix}
a_{n+1}\\
g_{n+1}
\end{bmatrix}
:=
\begin{bmatrix}
\tfrac{1}{2}(a_n+g_n)\\
\sqrt{a_n g_n}
\end{bmatrix}.
\end{align}
The iteration can be stopped if $a_n$ and $g_n$ are sufficiently
close to each other. If this condition fails for some reason,
to have a more stable algorithm, a maximum number $n_{\mathrm{max}}$ of iterations
should be specified. Numerical experiments show that $n_{\mathrm{max}}=14$
is enough for 64-bit floating point arithmetic with
\begin{equation}
(x,y)\in [10^{-307},10^{308}]\times [10^{-307},10^{308}].
\end{equation}
The complete elliptic integral of the first kind is defined as
\begin{equation}\label{eq:definitionK}
K(m) := \int_0^{\pi/2} \frac{\mathrm d\theta}{\sqrt{1-m\sin^2\theta}}.
\end{equation}
It is calculated by the arithmetic
geometric mean:
\begin{equation}
K(m) = \frac{\pi}{2\operatorname{agm}(1,\sqrt{1-m})}.
\end{equation}
The domain of $K(m)$ is $m<1$, but \eqref{eq:definitionK} allows more generally
\begin{equation}
m\in\C\setminus\{x\in\R\mid x\ge 1\}.
\end{equation}
The relation between the arithmetic geometric mean and $K(m)$ holds
even for complex numbers, but one has to take care of the branch
cut of the square root.

\end{document}
