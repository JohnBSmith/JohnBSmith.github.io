\documentclass[a4paper,10pt,fleqn,onecolumn,twoside]{book}
\usepackage[utf8]{inputenc}
\usepackage[T1]{fontenc}
\usepackage{lmodern}
\usepackage{ngerman}
\usepackage{amsmath}
\usepackage{amssymb}
\usepackage{color}
\definecolor{c1}{RGB}{00,40,80}
\usepackage[colorlinks=true,linkcolor=c1]{hyperref}
\usepackage{geometry}
\geometry{a4paper,left=25mm,right=12mm,top=20mm,bottom=28mm}
\setlength{\columnsep}{6mm}
\usepackage{lipsum}
\usepackage{multicol}
\usepackage[toc]{multitoc}
\setcounter{secnumdepth}{4}
\usepackage{titlesec}
\titleformat{\chapter}[block]
  {\normalfont\huge\bfseries}{\thechapter}{1em}{\Huge}
\titlespacing*{\chapter}{0pt}{0pt}{10pt}
\titleformat{\section}[block]
  {\normalfont\Large\bfseries}{\thesection}{1em}{\Large}
\titleformat{\subsection}[block]
  {\normalfont\large\bfseries}{\thesubsection}{1em}{\large}
\titleformat{\subsubsection}[block]
  {\normalfont\large\bfseries}{\thesubsubsection}{1em}{\large}

\numberwithin{equation}{chapter}
\newcommand{\strong}[1]{\textbf{#1}}

% \ui: imaginäre Einheit
% \ue: Einheitsvektor
\newcommand{\ui}{\mathrm i}
\newcommand{\ue}{e}

\begin{document}

\section*{Formelsammlung\\
Mathematik}
Rumil, 2016\\
Lizenz: CC0

\tableofcontents

\begin{multicols}{2}
\chapter{Grundlagen}
\section{Komplexe Zahlen}
\subsection{Rechenoperationen}

\begin{gather}
z_1+z_2 = (a_1+a_2)+(b_1+b_2)\ui\\
z_1-z_2 = (a_1-a_2)+(b_1-b_2)\ui\\
z_1 z_2 = (a_1 a_2 - b_1 b_2)+(a_1 b_2+a_2 b_1)\ui\\
\frac{z_1}{z_2}
= \frac{a_1 a_2 + b_1 b_2}{a_2^2+b_2^2}
+ \frac{a_2 b_1 - a_1 b_2}{a_2^2+b_2^2}\ui\\
\frac{1}{z} = \frac{a}{a^2+b^2}-\frac{b}{a^2+b^2}\ui
\end{gather}

\end{multicols}
\section{Mengenlehre}
\subsection{Boolesche Algebra}
\begin{tabular}{l|l|l}
\strong{Vereinigung} & \strong{Schnitt} &\\
  $A\cup A = A$
& $A\cap A = A$
& Idempotenzgesetze\\
  $A\cup \{\} = A$
& $A\cap G = A$
& Neutralitätsgesetze\\
  $A\cup G = G$
& $A\cap \{\} = \{\}$
& Extremalgesetze\\
  $A\cup \overline A = G$
& $A\cap \overline A = \{\}$
& Komplementärgesetze\\
\noalign{\vspace{1em}}
  $A\cup B = B\cup A$
& $A\cap B = B\cap A$
& Kommutativgesetze\\
  $(A\cup B)\cup C = A\cup (B\cup C)$
& $(A\cap B)\cap C = A\cap (B\cap C)$
& Assoziativgesetze\\
  $\overline{A\cup B} = \overline A\cap\overline B$
& $\overline{A\cap B} = \overline A\cup\overline B$
& De Morgansche Regeln\\
  $A\cup (A\cap B) = A$
& $A\cap (A\cup B) = A$
& Absorptionsgesetze\\
\end{tabular}
\begin{multicols}{2}
\noindent
$G$: Grundmenge\\

\noindent
\strong{Distributivgesetze}:
\begin{gather}
M\cup (A\cap B) = (M\cup A)\cap (M\cup B)\\
M\cap (A\cup B) = (M\cap A)\cup (M\cap B)
\end{gather}
\subsection{Teilmengenrelation}
Zerlegung der Gleichheit:
\begin{equation}
A=B \iff A\subseteq B \land B\subseteq A
\end{equation}
Umschreibung der Teilmengenrelation:
\begin{equation}
\begin{split}
A\subseteq B &\iff A\cap B=A\\
& \iff A\cup B=B\\
& \iff A\setminus B=\{\}
\end{split}
\end{equation}
Kontraposition:
\begin{equation}
A\subseteq B = \overline B\subseteq \overline A
\end{equation}
\end{multicols}

\end{document}


