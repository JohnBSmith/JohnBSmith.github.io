\documentclass[a4paper,10pt,fleqn,twocolumn,twoside,openany]{book}
\usepackage[utf8]{inputenc}
\usepackage[T1]{fontenc}
\usepackage{lmodern}
\usepackage{ngerman}
\usepackage{amsmath}
\usepackage{amssymb}
\usepackage{color}
\definecolor{c1}{RGB}{00,40,80}
\usepackage[colorlinks=true,linkcolor=c1]{hyperref}
\usepackage{geometry}
\geometry{a4paper,left=25mm,right=10mm,top=20mm,bottom=28mm}
\setlength{\columnsep}{4mm}
\usepackage{lipsum}
\usepackage{multicol}

\usepackage[toc]{multitoc}
\setcounter{secnumdepth}{4}
\setcounter{tocdepth}{2}
\usepackage{tocloft}
\setlength{\cftsecindent}{0pt}
\setlength{\cftsubsecindent}{23pt}
\setlength{\cftsubsubsecindent}{55pt}
\renewcommand{\cftchapfont}{\normalfont\sffamily\bfseries}
\renewcommand{\cftsecfont}{\normalfont\sffamily}
\renewcommand{\cftsubsecfont}{\normalfont\sffamily}
\renewcommand\cftchappagefont{\normalfont\sffamily\bfseries}
\renewcommand\cftsecpagefont{\normalfont\sffamily}
\renewcommand\cftsubsecpagefont{\normalfont\sffamily}

\usepackage{titlesec}
\titleformat{\chapter}[block]
  {\normalfont\sffamily\huge\bfseries}{\thechapter}{1em}{\Huge}
\titleformat{\section}[block]
  {\normalfont\sffamily\Large\bfseries}{\thesection}{1em}{\Large}
\titleformat{\subsection}[block]
  {\normalfont\sffamily\large\bfseries}{\thesubsection}{1em}{\large}
\titleformat{\subsubsection}[block]
  {\normalfont\sffamily\large\bfseries}{\thesubsubsection}{1em}{\large}

\titlespacing*{\chapter}{0pt}{0pt}{10pt}
\titlespacing*{\section}{0pt}{0pt}{4pt}
\titlespacing*{\subsection}{0pt}{2pt}{2pt}
\titlespacing*{\subsubsection}{0pt}{2pt}{2pt}

\usepackage[justification=RaggedRight,singlelinecheck=off]{caption}

\numberwithin{equation}{chapter}

\renewcommand{\baselinestretch}{0.9}

\newenvironment{ttsection}{\ttfamily}{\par}
\newcommand{\strong}[1]{{\sffamily\bfseries #1}}

% table header bold font
\newcommand{\thbf}[1]{{\sffamily\bfseries #1}}

% \ui: imaginäre Einheit
% \ue: Einheitsvektor
% \ue: eulersche Zahl
\newcommand{\ui}{\mathrm i}
\newcommand{\ue}{e}
\newcommand{\unit}[1]{\mathrm{#1}}
\newcommand{\ee}{\mathrm e}

\begin{document}
\setlength{\abovedisplayskip}{6pt}
\setlength{\belowdisplayskip}{6pt}
\setlength{\abovedisplayshortskip}{6pt}
\setlength{\belowdisplayshortskip}{6pt}

\begin{titlepage}
\centering
\phantom{x}

\vspace{20em}
{\noindent\Huge\sffamily\textbf{Formelsammlung\\
Mathematik}}

\vspace{2em}
{\Large November 2016}\\
\end{titlepage}

\thispagestyle{empty}

\noindent
Dieses Buch ist unter der Lizenz\\
Creative Commons CC0 veröffentlicht.
\vspace{8em}

\noindent
\begin{ttsection}
\begin{tabular}{r|r|r|r}
 0 & 0000 & 0 &  0\\
 1 & 0001 & 1 &  1\\
 2 & 0010 & 2 &  2\\
 3 & 0011 & 3 &  3\\
\noalign{\vspace{1em}}
 4 & 0100 & 4 &  4\\
 5 & 0101 & 5 &  5\\
 6 & 0110 & 6 &  6\\
 7 & 0111 & 7 &  7\\
\noalign{\vspace{1em}}
 8 & 1000 & 8 & 10\\
 9 & 1001 & 9 & 11\\
10 & 1010 & A & 12\\
11 & 1011 & B & 13\\
\noalign{\vspace{1em}}
12 & 1100 & C & 14\\
13 & 1101 & D & 15\\
14 & 1110 & E & 16\\
15 & 1111 & F & 17
\end{tabular}
\end{ttsection}

\newpage
\noindent
$\sin(-x) = -\sin x$\\
$\cos(-x) = \cos x$
\vspace{1em}

\noindent
$\sin(x+y) = \sin x\cos y + \cos x\sin y$\\
$\sin(x-y) = \sin x\cos y - \cos x\sin y$\\
$\cos(x+y) = \cos x\cos y - \sin x\sin y$\\
$\cos(x-y) = \cos x\cos y + \sin x\sin y$
\vspace{1em}

\noindent
$\ee^{\ui\varphi}=\cos\varphi+\ui\sin\varphi$
\vspace{2em}

\noindent
\strong{Polarkoordinaten}\\
$x=r\cos\varphi$\\
$y=r\sin\varphi$\\
$\varphi\in(-\pi,\pi]$\\
$\det J=r$
\vspace{1em}

\noindent
\strong{Zylinderkoordinaten}\\
$x=r_{xy}\cos\varphi$\\
$y=r_{xy}\sin\varphi$\\
$z=z$\\
$\det J=r_{xy}$
\vspace{1em}

\noindent
\strong{Kugelkoordinaten}\\
$x=r\sin\theta\,\cos\varphi$\\
$y=r\sin\theta\,\sin\varphi$\\
$z=r\cos\theta$\\
$\varphi\in(-\pi,\pi],\;\theta\in[0,\pi]$\\
$\det J=r^2\sin\theta$
\vspace{1em}

\noindent
$\theta=\beta-\pi/2$\\
$\beta\in[-\pi/2,\pi/2]$\\
$\cos\theta=\sin\beta$\\
$\sin\theta=\cos\beta$

\renewcommand{\contentsname}{\sffamily Inhaltsverzeichnis}
\tableofcontents


\chapter{Grundlagen}
\section{Arithmetik}
\subsection{Zahlenbereiche}
Natürliche Zahlen ab null:
\begin{equation}
\N_0 := \{0,1,2,3,4,\ldots\}.
\end{equation}
Natürliche Zahlen ab eins:
\begin{equation}
\N_1 := \{1,2,3,4,5,\ldots\}.
\end{equation}
Natürliche Zahlen:
\begin{equation}
\begin{split}
&\text{$\N$, wenn es keine Rolle spielt,}\\
&\text{ob $\N:=\N_0$ oder $\N:=\N_1$}.
\end{split}
\end{equation}
Ganze Zahlen:
\begin{equation}
\Z := \{\ldots,-3,-2,-1,0,1,2,3,\ldots\}.
\end{equation}
Rationale Zahlen:
\begin{equation}
\Q := \{\tfrac{z}{n}\mid z\in\Z,n\in\N_0\}.
\end{equation}
Reelle Zahlen:
\begin{equation}
\R := \overline{\Q}\enspace\text{bezüglich}\; d(x,y)=|x-y|.
\end{equation}
Positive reelle Zahlen:
\begin{equation}
\R^+ := \{x\in\R\mid x>0\}.
\end{equation}
Nichtnegative reelle Zahlen:
\begin{equation}
\R_0^+ := \{x\in\R\mid x\ge 0\}.
\end{equation}
Negative reelle Zahlen:
\begin{equation}
\R^- := \{x\in\R\mid x<0\}.
\end{equation}
Nichtpositive reelle Zahlen:
\begin{equation}
\R_0^- := \{x\in\R\mid x\le 0\}.
\end{equation}
Komplexe Zahlen:
\begin{equation}
\C := \{a+b\ui\mid a,b\in\R\}.
\end{equation}
Quaternionen:
\begin{equation}
\mathbb H := \{a+b\ui+c\uj+d\uk\mid a,b,c,d\in\R\}.
\end{equation}
Algebraische Zahlen:
\begin{equation}
\mathbb A := \{a\in\C\mid \exists P{\in}\Q[X]\colon P(a)=0\}.
\end{equation}
Irrationale Zahlen:
\begin{equation}
\R\setminus\Q = \{\sqrt{2},\sqrt{3},\pi,\ee,\ldots\}.
\end{equation}
Transzendente Zahlen:
\begin{equation}
\R\setminus\mathbb A = \{\pi,\ee,\ldots\}.
\end{equation}
Es gelten die folgenden Teilmengenbeziehungen:
\begin{equation}
\N\subset\Z\subset\Q\subset\R\subset\C\subset\mathbb H.
\end{equation}
Es gilt die folgende Abstufung der Mächtigkeit:
\begin{equation}
|\N| = |\Z| = |\Q| = |\mathbb A| < |\R| = |\C|.
\end{equation}

\newpage
\subsection{Intervalle}
Abgeschlossene Intervalle:
\begin{equation}
[a,b] := \{x\in\R\mid a\le x\le b\}.
\end{equation}
Offene Intervalle:
\begin{equation}
(a,b) := \{x\in\R\mid a<x<b\}.
\end{equation}
Halboffene Intervalle:
\begin{align}
(a,b] &:= \{x\in\R\mid a<x\le b\},\\
[a,b) &:= \{x\in\R\mid a\le x<b\}.
\end{align}
Unbeschränkte Intervalle:
\begin{align}
[a,\infty) &:= \{x\in\R\mid a\le x\},\\
(a,\infty) &:= \{x\in\R\mid a<x\},\\
(-\infty,b] &:= \{x\in\R\mid x\le b\},\\
(-\infty,b) &:= \{x\in\R\mid x<b\}.
\end{align}

\subsection{Summen}
\begin{definition}[Summe]
Für eine Folge $(a_n)$:
\begin{align}
\sum_{k=m}^{m-1} a_k &:= 0,\qquad(\text{leere Summe})\\
\sum_{k=m}^n a_k &:= a_n+\sum_{k=m}^{n-1} a_k.\qquad(n\ge m)
\end{align}
\end{definition}
\noindent
Für eine Konstante $c$ gilt:
\begin{equation}
\sum_{k=m}^n c = (n-m+1)\,c.
\end{equation}
Der Summierungsoperator ist linear:
\begin{align}
\sum_{k=m}^n (a_k+b_k) &= \sum_{k=m}^n a_k + \sum_{k=m}^n b_k,\\
\sum_{k=m}^n ca_k &= c\sum_{k=m}^n a_k.
\end{align}
Indexverschiebung ist möglich:
\begin{equation}
\sum_{k=m}^n a_k = \sum_{k=m-j}^{n-j} a_{k+j} = \sum_{k=m+j}^{n+j} a_{k-j}.
\end{equation}
Aufspaltung ist möglich:
\begin{equation}
\sum_{k=m}^n a_k = \sum_{k=m}^p a_k + \sum_{k=p+1}^n a_k.
\end{equation}
Vertauschung der Reihenfolge bei Doppelsummen:
\begin{equation}
\sum_{i=p}^m \sum_{j=q}^n a_{ij} = \sum_{j=q}^n \sum_{i=p}^m a_{ij}.
\end{equation}

\subsection{Produkte}
\begin{definition}[Produkt]
Für eine Folge $(a_n)$:
\begin{align}
\prod_{k=m}^{m-1} a_k &:= 1,\qquad(\text{leeres Produkt})\\
\prod_{k=m}^n a_k &:= a_n\prod_{k=m}^{n-1} a_k.\qquad(n\ge m)
\end{align}
\end{definition}

\noindent
Für eine Konstante $c$ gilt:
\begin{equation}
\prod_{k=m}^n c = c^{n-m+1}.
\end{equation}
Unter Voraussetzung des Kommutativgesetzes gilt
\begin{align}
\label{eq:product-product}
\prod_{k=m}^n (a_k b_k) &= \bigg(\prod_{k=m}^n a_k\bigg)\bigg(\prod_{k=m}^n b_k\bigg),\\
\label{eq:product-power}
\prod_{k=m}^n a_k^c &= \bigg(\prod_{k=m}^n a_k\bigg)\Big.^c.\qquad (c\in\N_0)
\end{align}

Formel \eqref{eq:product-power} gilt auch für $a_k\in\R^+$ und $c\in\C$.

Formel \eqref{eq:product-product} ist ein Spezialfall von
\begin{equation}
\prod_{i=p}^m \prod_{j=q}^n a_{ij} = \prod_{j=q}^n \prod_{i=p}^m a_{ij}.
\end{equation}
Indexverschiebung ist möglich:
\begin{equation}
\prod_{k=m}^n a_k = \prod_{k=m-j}^{n-j} a_{k+j} = \prod_{k=m+j}^{n+j} a_{k-j}.
\end{equation}
Aufspaltung ist möglich:
\begin{equation}
\prod_{k=m}^n a_k = \bigg(\prod_{k=m}^p a_k\bigg)\bigg(\prod_{k=p+1}^n a_k\bigg).
\end{equation}
Für $a_k\in\R^+$ gilt
\begin{equation}
\prod_{k=m}^n a_k = \exp\bigg(\sum_{k=m}^n \ln(a_k)\bigg).
\end{equation}

\subsection{Binomischer Lehrsatz}\index{binomischer Lehrsatz}
Sei $R$ ein unitärer Ring, z.\,B. $R=\R$ oder $R=\C$.\\
Für $a,b\in R$ mit $ab=ba$ gilt:%
\begin{equation}
(a+b)^n = \sum_{k=0}^n \binom{n}{k} a^{n-k} b^k
\end{equation}
und
\begin{equation}
(a-b)^n = \sum_{k=0}^n \binom{n}{k} (-1)^k a^{n-k} b^k.
\end{equation}
Die ersten Formeln sind:\index{binomische Formeln}
\begin{gather}
(a+b)^2 = a^2+2ab+b^2,\\
(a-b)^2 = a^2-2ab+b^2,\\
(a+b)^3 = a^3+3a^2 b+3ab^2+b^3,\\
(a-b)^3 = a^3-3a^2 b+3ab^2-b^3,\\
(a+b)^4 = a^4+4a^3 b+6a^2 b^2+4ab^3+b^4,\\
(a-b)^4 = a^4-4a^3 b+6a^2 b^2-4ab^3+b^4.
\end{gather}
\subsection{Potenzgesetze}
\begin{definition}[Potenz]
Für $a$ aus einem Monoid und $n\in\Z,n\ge 1$:
\begin{align}
a^0 &:= 1,\\
a^n &:= a^{n-1}\cdot a.
\end{align}

Für $a\in\R, a>0$ und $x\in\C$:
\begin{equation}
a^x := \exp(\ln(a)\,x).
\end{equation}
\end{definition}
\noindent
Für $a\in\R, a>0$ und $x,y\in\C$ gilt:
\begin{gather}
a^{x+y} = a^x a^y,\quad a^{x-y} = \frac{a^x}{a^y},
\quad a^{-x} = \frac{1}{a^x}.
\end{gather}

\section{Gleichungen}
\begin{definition}[Bestimmungsgleichung]
Sind $f,g$ auf der Grundmenge $G$ definierte Funktionen, so nennt
man
\begin{equation}
f(x) = g(x)\\
\end{equation}
eine \emdef{Bestimmungsgleichung}\index{Bestimmungsgleichung},
wenn die Lösungemenge
\begin{equation}
L = \{x\in G\mid f(x)=g(x)\}
\end{equation}
gesucht ist.
\end{definition}
Bei den $x\in G$ kann es sich auch um Tupel $x=(x_1,x_2)$ oder
$x=(x_1,x_2,x_3)$ usw. handeln. Man spricht in diesem Fall
von einer Gleichung \emdef{in mehreren Variablen}.

Handelt es sich bei den Funktionswerten von $f,g$ um Tupel,
dann spricht man von einem
\emdef{Gleichungssystem}\index{Gleichungssystem}.

\subsection{Äquivalenzumformungen}

Äquivalenzumformungen lassen die Lösungsmenge einer Gleichung
unverändert. Seien $A(x),B(x)$ zwei Aussageformen bzw. zwei
Gleichungen. Aus
\begin{equation}
\forall x{\in}G\,[A(x)\Longleftrightarrow B(x)]
\end{equation}
folgt
\begin{equation}
\{x\in G\mid A(x)\} = \{x\in G\mid B(x)\}.
\end{equation}
Aus
\begin{equation}
\forall x{\in} G\,[A(x)\Longrightarrow B(x)]
\end{equation}
folgt jedoch nur noch
\begin{equation}
\{x\in G\mid A(x)\}\subseteq\{x\in G\mid B(x)\}.
\end{equation}
Seien $f,g,h$ Funktionen mit Definitionsmenge $G$ und
Zielmenge $Z=\R$ oder $Z=\C$.

Für alle $x$ gilt:
\begin{align}
f(x)=g(x) &\Longleftrightarrow f(x){+}h(x)=g(x){+}h(x),\\
f(x)=g(x) &\Longleftrightarrow f(x){-}h(x)=g(x){-}h(x).
\end{align}
Besitzt $h(x)$ keine Nullstellen, dann gilt für alle $x$:
\begin{align}
f(x)=g(x) &\iff f(x)h(x)=g(x)h(x),\\
f(x)=g(x) &\iff \frac{f(x)}{h(x)}=\frac{f(x)}{h(x)}.
\end{align}
Besitzt $h(x)$ aber Nullstellen, dann gilt immerhin noch für alle $x$:
\begin{equation}
f(x)=g(x) \implies f(x)h(x)=g(x)h(x).
\end{equation}
Sei $f,g\colon G\to Z$. Sei $\varphi_x\colon Z\to Z'$ eine Injektion
für jedes $x\in G$. Es gilt
\begin{equation}
f(x)=g(x) \iff \varphi_x(f(x))=\varphi_x(g(x))
\end{equation}
für alle $x\in G$.

Bei einer Kette von Äquivalenzumformungen wird links das
Äquivalenzzeichen geschrieben, in der Mitte die Gleichung
und rechts hinter einem senkrechten Strich die Operation
$\varphi_x(w)$, welche als nächstes auf beide Seiten der Gleichung
angwendet werden soll.

Beispiel:
\begin{equation*}\setlength{\arraycolsep}{2pt}
\begin{array}{rrl@{\qquad}l}
& 2x+4 &= 2x^2-8x+2 &\mid w/2\\[2pt]
\Longleftrightarrow& x+2 &= x^2-4x+1 &\mid w-2\\[2pt]
\Longleftrightarrow& x &= x^2-4x-1 &\mid w-x\\[2pt]
\Longleftrightarrow& 0 &= x^2-7x-1.
\end{array}
\end{equation*}
Am Anfang befinden sich eventuell Bedingungen für $x$.
Bei Fallunterscheidungen wird eine Verschärfung der Bedingungen
vorgenommen, so dass es zur Verkleinerung der Grundmenge kommt.
Nach einer Fallunterscheidung ergeben sich unter Umständen neue
Injektionen.

\subsection{Quadratische Gleichungen}
\begin{definition}[Quadratische Gleichung]
Eine Gleichung der Form $ax^2+bx+c=0$ mit $a\ne 0$ heißt
\emdef{quadratische Gleichung}.
\end{definition}

Wegen $a\ne 0$ lässt sich die Gleichung durch $a$ dividieren
und es ensteht die äquivalente Normalform $x^2+px+q=0$
mit $p:=b/a$ und $q:=c/a$.

\strong{Lösung.}
Seien nun die $a,b,c$ reelle Zahlen. Die Zahl
\begin{equation}
D = p^2-4q
\end{equation}
heißt \emdef{Diskriminante}. Für $D>0$ gibt es zwei reelle Lösungen:
\begin{align}
x_1 &= \frac{-p-\sqrt{D}}{2} = \frac{-b-\sqrt{b^2-4ac}}{2a},\\
x_2 &= \frac{-p+\sqrt{D}}{2} = \frac{-b+\sqrt{b^2-4ac}}{2a}.
\end{align}
Für $D=0$ fallen beiden Lösungen zu einer \emdef{doppelten Lösung}
zusammen:
\begin{equation}
x_1 = x_2 = -\frac{p}{2} = -\frac{b}{2a}.
\end{equation}
Für $D<0$ gibt es keine reelle Lösung. Aber es gibt zwei komplexe
Lösungen, die zueinander konjugiert sind:
\begin{equation}
x_1 = \frac{-p-\ui\sqrt{|D|}}{2},\quad
x_2 = \frac{-p+\ui\sqrt{|D|}}{2}.
\end{equation}
In jedem Fall gelten die Formeln von Vieta:
\begin{equation}
p = -(x_1+x_2),\qquad q = x_1 x_2.
\end{equation}

\section{Komplexe Zahlen}\index{komplexe Zahl}
\subsection{Rechenoperationen}

\begin{gather}
\frac{z_1}{z_2}
= \frac{z_1\overline z_2}{z_2\overline z_2}
= \frac{z_1\overline z_2}{|z_2|^2},\\
\frac{1}{z} = \frac{\overline z}{z\overline z}
= \frac{\overline z}{|z|^2}.
\end{gather}

\subsection{Betrag}\index{Betrag!einer komplexen Zahl}
Für alle $z_1,z_2\in\C$ gilt:
\begin{gather}
|z_1z_2| = |z_1|\,|z_2|,\\
z_2\ne 0\implies \Big|\frac{z_1}{z_2}\Big|
= \frac{|z_1|}{|z_2|},\\
z\,\overline z = |z|^2.
\end{gather}

\subsection{Konjugation}\index{Konjugation!einer komplexen Zahl}
Für alle $z_1,z_2\in\C$ gilt:
\begin{gather}
\overline{z_1+z_2} = \overline z_1+\overline z_2,\qquad
\overline{z_1-z_2} = \overline z_1-\overline z_2,\\
\overline{z_1 z_2} = \overline z_1\,\overline z_2,\qquad
z_2\ne 0 \implies \overline{\Big(\frac{z_1}{z_2}\Big)}
= \frac{\overline z_1}{\overline z_2},\\
\overline{\overline z}=z,\qquad
|\overline{z}| = |z|,\qquad
z\,\overline z = |z|^2,\\
\real(z) = \frac{z+\overline z}{2},\qquad
\imag(z) = \frac{z-\overline z}{2\ui},\\
\overline{\cos(z)} = \cos(\overline z),\qquad
\overline{\sin(z)} = \sin(\overline z),\\
\overline{\exp(z)} = \exp(\overline z).
\end{gather}

\begin{table*}[t]
\caption{Rechnen mit komplexen Zahlen}
\bgroup
\def\arraystretch{1.4}
\begin{tabular}{|l|r|l|l|}
\hline
  \thbf{Name}
& \thbf{Operation}
& \thbf{Polarform}
& \thbf{kartesische Form}\\
\hline
  Identität
& $z$ & $=r\ee^{\ui\varphi}$
& $= a+b\ui$\\
\hline
  Addition
& $z_1+z_2$ &
& $= (a_1+a_2)+(b_1+b_2)\ui$\\
\hline
  Subtraktion
& $z_1-z_2$ &
& $= (a_1-a_2)+(b_1-b_2)\ui$\\
\hline
  Multiplikation
& $z_1 z_2$
& $= r_1 r_2 \ee^{\ui(\varphi_1+\varphi_2)}$
& $= (a_1 a_2 - b_1 b_2)+(a_1 b_2+a_2 b_1)\ui$\\
\hline
  Division
& $\displaystyle\frac{z_1}{z_2}$
& $\displaystyle =\frac{r_1}{r_2}\ee^{\ui(\varphi_1-\varphi_2)}$
& $\displaystyle =\frac{a_1 a_2 + b_1 b_2}{a_2^2+b_2^2}
   + \frac{a_2 b_1 - a_1 b_2}{a_2^2+b_2^2}\ui$\\
\hline
  Kehrwert
& $\displaystyle\frac{1}{z}$
& $\displaystyle =\frac{1}{r}\ee^{-\ui\varphi}$
& $\displaystyle =\frac{a}{a^2+b^2}-\frac{b}{a^2+b^2}\ui$\\
\hline
  Realteil
& $\real(z)$
& $=\cos\varphi$
& $=a$\\
\hline
  Imaginärteil
& $\imag(z)$
& $=\sin\varphi$
& $=b$\\
\hline
  Konjugation
& $\overline{z}$
& $=r\ee^{-\varphi\ui}$
& $=a-b\ui$\\
\hline
Betrag
& $|z|$
& $=r$
& $=\sqrt{a^2+b^2}$\\
\hline
  Argument
& $\arg(z)$
& $=\varphi$
& $\displaystyle = s(b)\arccos\Big(\frac{a}{r}\Big)$\\
\hline
\end{tabular}
\egroup\\
\\
$s(b):=\begin{cases}
+1 & \text{if}\;b\ge 0,\\
-1 & \text{if}\;b<0
\end{cases}$
\end{table*}

\section{Logik}
\subsection{Aussagenlogik}\index{Aussagenlogik}
\subsubsection{Boolesche Algebra}\index{boolesche Algebra}
\begin{table*}[t]
\caption{Boolesche Algebra}
\begin{tabular}{l|l|l}
\thbf{Disjunktion} & \thbf{Konjunktion} &\\
  $A\lor A \Leftrightarrow A$
& $A\land A \Leftrightarrow A$
& Idempotenzgesetze\\
  $A\lor 0 \Leftrightarrow A$
& $A\land 1 \Leftrightarrow A$
& Neutralitätsgesetze\\
  $A\lor 1 \Leftrightarrow 1$
& $A\land 0 = 0$
& Extremalgesetze\\
  $A\lor \overline A \Leftrightarrow 1$
& $A\land \overline A \Leftrightarrow 0$
& Komplementärgesetze\\
\noalign{\vspace{1em}}
  $A\lor B \Leftrightarrow B\lor A$
& $A\land B \Leftrightarrow B\land A$
& Kommutativgesetze\\
  $(A\lor B)\lor C \Leftrightarrow A\lor (B\lor C)$
& $(A\land B)\land C \Leftrightarrow A\land (B\land C)$
& Assoziativgesetze\\
  $\overline{A\lor B} \Leftrightarrow \overline A\land\overline B$
& $\overline{A\land B} \Leftrightarrow \overline A\lor\overline B$
& De Morgansche Regeln\\
  $A\lor (A\land B) \Leftrightarrow A$
& $A\land (A\lor B) \Leftrightarrow A$
& Absorptionsgesetze\\
\end{tabular}
\end{table*}

\noindent
\strong{Distributivgesetze}:
\begin{gather}
A\lor (B\land C) \iff (A\lor B)\land (A\lor C),\\
A\land (B\lor C) \iff (A\land B)\lor (A\land C).
\end{gather}

\subsubsection{Zweistellige Funktionen}
Es gibt 16 zweistellige boolesche\\
Funktionen.

\begin{tabular}{r|l}
\textbf{\texttt{AB}} & \thbf{Wert}\\
\texttt{00} & \texttt{a}\\
\texttt{01} & \texttt{b}\\
\texttt{10} & \texttt{c}\\
\texttt{11} & \texttt{d}
\end{tabular}

\begin{tabular}{r|l|l|l}
\thbf{Nr.}& \textbf{\texttt{dcba}} & \thbf{Fkt.} & \thbf{Name}\\
 0 & \texttt{0000} & 0 & Kontradiktion\\
 1 & \texttt{0001} & $\overline{A\lor B}$ & NOR\\
 2 & \texttt{0010} & $\overline{B\Rightarrow A}$\\
 3 & \texttt{0011} & $\overline A$\\
 4 & \texttt{0100} & $\overline{A\Rightarrow B}$\\
 5 & \texttt{0101} & $\overline{B}$\\
 6 & \texttt{0110} & $A\oplus B$ & Kontravalenz\index{Kontravalenz}\\
 7 & \texttt{0111} & $\overline{A\land B}$ & NAND\\
 8 & \texttt{1000} & $A\land B$ & Konjunktion\index{Konjunktion}\\
 9 & \texttt{1001} & $A\Leftrightarrow B$ & Äquivalenz\\
10 & \texttt{1010} & $B$ & Projektion\\
11 & \texttt{1011} & $A\Rightarrow B$ & Implikation\\
12 & \texttt{1100} & $A$ & Projektion\\
13 & \texttt{1101} & $B\Rightarrow A$ & Implikation\\
14 & \texttt{1110} & $A\lor B$ & Disjunktion\index{Disjunktion}\\
15 & \texttt{1111} & $1$ & Tautologie
\end{tabular}

\subsubsection[Darstellung mit Negation, Konjunktion und Disjunktion]%
{Darstellung mit Negation,\newline Konjunktion und Disjunktion}
\begin{gather}\label{eq:implication-definition}
A\Rightarrow B \iff \overline A\lor B,\\
(A\Leftrightarrow B) \iff
  (\overline A\land\overline B)\lor(A\land B),\\
A\oplus B \iff (\overline A\land B)\lor(A\land\overline B).
\end{gather}

\subsubsection{Tautologien}
Modus ponens:
\begin{equation}\label{eq:modus-ponens}
(A\Rightarrow B)\land A\implies B.
\end{equation}
Modus tollens:
\begin{equation}
(A\Rightarrow B)\land\overline B\implies\overline A.
\end{equation}
Modus tollendo ponens:
\begin{equation}
(A\lor B)\land\overline A \implies B.
\end{equation}
Modus ponendo tollens:
\begin{equation}
\overline{A\land B}\land A\implies\overline B.
\end{equation}
Kontraposition:\index{Kontraposition}
\begin{equation}
A\Rightarrow B \iff \overline B\Rightarrow \overline A.
\end{equation}
Beweis durch Widerspruch:\index{Widerspruch}
\begin{equation}
(\overline A\Rightarrow B\land\overline B)\implies A.
\end{equation}
Zerlegung einer Äquivalenz:
\begin{equation}
(A\Leftrightarrow B) \iff (A\Rightarrow B)\land(B\Rightarrow A).
\end{equation}
Kettenschluss:
\begin{equation}
(A\Rightarrow B)\land(B\Rightarrow C)\implies (A\Rightarrow C).
\end{equation}
Ringschluss:
\begin{equation}
\begin{split}
&(A\Rightarrow B)\land (B\Rightarrow C)\land(C\Rightarrow A)\\
&\implies (A\Leftrightarrow B)\land(A\Leftrightarrow C)\land(B\Leftrightarrow C).
\end{split}
\end{equation}
Ringschluss, allgemein:
\begin{equation}
\begin{split}
& (A_1{\Rightarrow }A_2)\land\ldots\land(A_{n-1}{\Rightarrow}A_n)
\land(A_n{\Rightarrow}A_1)\\
& \implies \forall i,j\,[A_i\Leftrightarrow A_j].
\end{split}
\end{equation}
Ersetzungsregel:

Für jede Funktion $P\colon\{0,1\}\to\{0,1\}$ gilt:
\begin{equation}
P(A)\land (A\Leftrightarrow B)\implies P(B).
\end{equation}
Regel zur Implikation:
\begin{equation}
A\land B\Rightarrow C \iff A\Rightarrow (B\Rightarrow C).
\end{equation}
Vollständige Fallunterscheidung:
\begin{gather}
(A\Rightarrow C)\land (B\Rightarrow C)\implies (A\oplus B\Rightarrow C),\\
(A\Rightarrow C)\land (B\Rightarrow C)\iff (A\lor B\Rightarrow C).
\end{gather}
Vollständige Fallunterscheidung, allgemein:
\begin{gather}
\textstyle \forall k[A_k\Rightarrow C]
\implies (\bigoplus_{k=1}^n A_k\Rightarrow C),\\
\forall k[A_k\Rightarrow C]
\iff (\exists k[A_k]\Rightarrow C).
\end{gather}

\subsubsection{Schlussregeln}
\strong{Ersetzungsregel.} Sei $p(\varphi)$ eine aussagenlogische
Formel in expliziter Abhängigkeit von der Formelvariablen $\varphi$.
Es gilt
\begin{equation}
\{p(\varphi),\varphi\leftrightarrow\psi\}\vdash p(\psi).
\end{equation}
\strong{Beispiel.} Betrachte $\varphi\land A\rightarrow B$ mit
$\varphi:=(A\rightarrow B)$, was expandiert wird zu
\[(A\rightarrow B)\land A\rightarrow B.\qquad\text{(s. \eqref{eq:modus-ponens})}\]
Nun gilt nach \eqref{eq:implication-definition} aber
\[A\rightarrow B\leftrightarrow \overline A\lor B.\]
Daher lässt sich folgern:
\[(\overline A\lor B)\land A\rightarrow B.\]

\subsubsection{Metatheoreme}
\strong{Korrektheit der Aussagenlogik.}

Für die Aussagenlogik gilt:
\begin{equation}
(\Gamma\vdash\psi)\implies (\Gamma\models\psi).
\end{equation}

\noindent
\strong{Vollständigkeit der Aussagenlogik.}

Für die Aussagenlogik gilt:
\begin{equation}
(\Gamma\models\psi)\implies (\Gamma\vdash\psi).
\end{equation}

\noindent
\strong{Deduktionstheorem (syntaktisch).}

Für die Aussagenlogik gilt:
\begin{equation}
(\Gamma\cup\{\varphi\}\vdash\psi)\iff (\Gamma\vdash\varphi\rightarrow\psi).
\end{equation}

Infolge gilt auch:
\begin{equation}
\begin{split}
&(\{\varphi_1,\ldots,\varphi_n\}\vdash\psi)\\
&\iff (\vdash \varphi_1\land\ldots\land\varphi_n\rightarrow\psi).
\end{split}
\end{equation}

\noindent
\strong{Deduktionstheorem (semantisch).}

Für die Aussagenlogik gilt:
\begin{equation}
(\Gamma\cup\{\varphi\}\models\psi)\iff (\Gamma\models\varphi\rightarrow\psi).
\end{equation}

Infolge gilt auch:
\begin{equation}
\begin{split}
&(\{\varphi_1,\ldots,\varphi_n\}\models\psi)\\
&\iff (\models \varphi_1\land\ldots\land\varphi_n\rightarrow\psi).
\end{split}
\end{equation}

\noindent
\strong{Einsetzungsregel.}

Sei $v$ eine metasprachliche Variable, die für eine beliebige
objektsprachliche Variable steht. Dann gilt:
\begin{equation}
(\models\varphi) \implies (\models\varphi[v:=\psi]).
\end{equation}
D.\,h. wenn in der tautologischen Formel $\varphi$ jedes auftreten
der Variable $v$ gegen die Formel $\psi$ ersetzt wird, ergibt
sich wieder eine tautologische Formel.

%\newpage
\subsection{Prädikatenlogik}
\subsubsection{Rechenregeln}
Verneinung (De Morgansche Regeln):
\begin{gather}
\overline{\forall x[P(x)]}\iff \exists x[\overline{P(x)}],\\
\overline{\exists x[P(x)]}\iff \forall x[\overline{P(x)}].
\end{gather}
Verallgemeinerte Distributivgesetze:
\begin{gather}
P\lor\forall x[Q(x)] \iff \forall x[P\lor Q(x)],\\
P\land\exists x[Q(x)] \iff \exists x[P\land Q(x)].
\end{gather}
Verallgemeinerte Idempotenzgesetze:
\begin{gather}
\begin{split}
\exists x{\in}M\,[P] & \iff
(M\ne\{\})\land P\\
& \iff\begin{cases}
P & \text{wenn}\; M\ne\{\},\\
0 & \text{wenn}\; M=\{\}.
\end{cases}
\end{split}\\
\begin{split}
\forall x{\in}M\,[P]& \iff
(M=\{\})\lor P\\
&\iff\begin{cases}
P & \text{wenn}\; M\ne\{\},\\
1 & \text{wenn}\; M=\{\}.
\end{cases}
\end{split}
\end{gather}
%\newpage\noindent
Äquivalenzen:
\begin{gather}
\hspace{-2em}\forall x\forall y[P(x,y)] \iff \forall y\forall x[P(x,y)],\\
\hspace{-2em}\exists x\exists y[P(x,y)] \iff \exists y\exists x[P(x,y)],\\
\hspace{-2em}\forall x[P(x)\land Q(x)] \iff \forall x[P(x)]\land\forall x[Q(x)],\\
\hspace{-2em}\exists x[P(x)\lor Q(x)] \iff \exists x[P(x)]\lor\exists x[Q(x)],\\
\hspace{-2em}\forall x[P(x)\Rightarrow Q] \iff \exists x[P(x)]\Rightarrow Q,\\
\hspace{-2em}\forall x[P\Rightarrow Q(x)] \iff P\Rightarrow\forall x[Q(x)],\\
\hspace{-2em}\exists x[P(x)\Rightarrow Q(x)]
  \iff\forall x[P(x)]\Rightarrow\exists x[Q(x)].
\end{gather}
% \newpage\noindent
Implikationen:
\begin{gather}
\hspace{-2em}\exists x\forall y[P(x,y)]\implies \forall y\exists x[P(x,y)],\\
\hspace{-2em}\forall x[P(x)]\lor\forall x[Q(x)]\implies\forall x[P(x)\lor Q(x)],\\
\hspace{-2em}\exists x[P(x)\land Q(x)]\implies
  \exists x[P(x)]\land \exists x[Q(x)],\\
\hspace{-2em}\forall x[P(x)\Rightarrow Q(x)]\implies
  (\forall x[P(x)]\Rightarrow\forall x[Q(x)]),\\
\hspace{-2em}\forall x[P(x)\Leftrightarrow Q(x)]\implies
  (\forall x[P(x)]\Leftrightarrow\forall x[Q(x)]).
\end{gather}

\newpage
\subsubsection{Endliche Mengen}
Sei $M=\{x_1,\ldots,x_n\}$. Es gilt:
\begin{gather}
\forall x{\in}M\,[P(x)]\iff P(x_1)\land\ldots\land P(x_n),\\
\exists x{\in}M\,[P(x)]\iff P(x_1)\lor\ldots\lor P(x_n).
\end{gather}

\subsubsection{Beschränkte Quantifizierung}
\begin{gather}
\begin{split}
& \forall x{\in}M\,[P(x)]\defiff\forall x[x\notin M\lor P(x)]\\
& \quad\iff\forall x[x\in M\Rightarrow P(x)],
\end{split}\\
\exists x{\in}M\,[P(x)]\defiff\exists x[x\in M\land P(x)],\\
\forall x{\in}M{\setminus}N\,[P(x)]\iff \forall x[x\notin N\Rightarrow P(x)].
\end{gather}

\subsubsection[Quantifizierung über Produktmengen]%
{Quantifizierung über\newline Produktmengen}
\begin{gather}
\forall(x,y)\,[P(x,y)]\iff \forall x\forall y[P(x,y)],\\
\exists(x,y)\,[P(x,y)]\iff \exists x\exists y[P(x,y)].
\end{gather}
Analog gilt
\begin{gather}
\forall(x,y,z)\,\iff \forall x\forall y\forall z,\\
\exists(x,y,z)\,\iff \exists x\exists y\exists z
\end{gather}
usw.

\subsubsection{Alternative Darstellung}
Sei $P\colon G\to\{0,1\}$ und $M\subseteq G$.
Mit $P(M)$ ist die Bildmenge von $P$ bezüglich $M$ gemeint.
Es gilt
\begin{equation}
\begin{split}
&\forall x{\in}M\,[P(x)] \iff P(M)=\{1\}\\
& \iff M\subseteq\{x{\in}G\mid P(x)\}
\end{split}
\end{equation}
und
\begin{equation}
\begin{split}
& \exists x{\in}M\,[P(x)] \iff \{1\}\subseteq P(M)\\
& \iff M\cap\{x{\in}G\mid P(x)\}\ne\{\}.
\end{split}
\end{equation}

\subsubsection{Eindeutigkeit}
Quantor für eindeutige Existenz:
\begin{equation}
\begin{split}
&\exists!x\,[P(x)]\\
&:\Longleftrightarrow\; \exists x\,[P(x)\land \forall y\,[P(y)\Rightarrow x=y]]\\
&\iff \exists x\,[P(x)]\land \forall x\forall y[P(x)\land P(y)\Rightarrow x=y].
\end{split}
\end{equation}

\newpage
\section{Mengenlehre}
\subsection{Definitionen}
Aufzählende Notation:
\begin{equation}
\hspace{-1em} a\in\{x_1,\ldots,x_n\} :\Leftrightarrow a=x_1\lor\ldots\lor a=x_n.
\end{equation}
Beschreibende Notation:
\begin{gather}
a\in\{x\mid P(x)\}\defiff P(a),\\
\{x\in M\mid P(x)\} := \{x\mid x\in M\land P(x)\},\\
\hspace{-1em}\{f(x)\mid P(x)\} := \{y\mid \exists x(y=f(x)\land P(x))\}.
\end{gather}
Teilmengenrelation:
\begin{equation}
A\subseteq B\defiff \forall x\,(x\in A\implies x\in B).
\end{equation}
Gleichheit:
\begin{equation}
A=B\defiff \forall x\,(x\in A\iff x\in B).
\end{equation}
Vereinigungsmenge:
\begin{equation}
A\cup B:=\{x\mid x\in A\lor x\in B\}.
\end{equation}
Schnittmenge:
\begin{equation}
A\cap B:=\{x\mid x\in A\land x\in B\}.
\end{equation}
Differenzmenge:
\begin{equation}
A\setminus B:=\{x\mid x\in A\land x\not\in B\}.
\end{equation}
Symmetrische Differenz:
\begin{equation}
A\triangle B:=\{x\mid x\in A\oplus x\in B\}.
\end{equation}
Komplementärmenge:
\begin{equation}
A^\comp := G\setminus A.\qquad (\text{$G$: Grundmenge})
\end{equation}
Vereinigung über indizierte Mengen:
\begin{equation}
\bigcup_{i\in I} A_i := \{x\mid\exists i{\in}I\,(x\in A_i)\}.
\end{equation}
Schnitt über indizierte Mengen:
\begin{equation}
\bigcap_{i\in I} A_i := \{x\mid\forall i{\in}I\,(x\in A_i)\}.
\end{equation}


\subsection{Boolesche Algebra}
\begin{table*}[t]
\caption{Boolesche Algebra}
\begin{tabular}{l|l|l}
\thbf{Vereinigung} & \thbf{Schnitt} &\\
  $A\cup A = A$
& $A\cap A = A$
& Idempotenzgesetze\\
  $A\cup \{\} = A$
& $A\cap G = A$
& Neutralitätsgesetze\\
  $A\cup G = G$
& $A\cap \{\} = \{\}$
& Extremalgesetze\\
  $A\cup \overline A = G$
& $A\cap \overline A = \{\}$
& Komplementärgesetze\\
\noalign{\vspace{1em}}
  $A\cup B = B\cup A$
& $A\cap B = B\cap A$
& Kommutativgesetze\\
  $(A\cup B)\cup C = A\cup (B\cup C)$
& $(A\cap B)\cap C = A\cap (B\cap C)$
& Assoziativgesetze\\
  $\overline{A\cup B} = \overline A\cap\overline B$
& $\overline{A\cap B} = \overline A\cup\overline B$
& De Morgansche Regeln\\
  $A\cup (A\cap B) = A$
& $A\cap (A\cup B) = A$
& Absorptionsgesetze\\
\end{tabular}\\
\\
$G$: Grundmenge
\end{table*}

\noindent
\strong{Distributivgesetze}:
\begin{gather}
M\cup (A\cap B) = (M\cup A)\cap (M\cup B),\\
M\cap (A\cup B) = (M\cap A)\cup (M\cap B).
\end{gather}

\subsection{Teilmengenrelation}
Zerlegung der Gleichheit:
\begin{equation}
A=B \iff A\subseteq B \land B\subseteq A.
\end{equation}
Umschreibung der Teilmengenrelation:
\begin{equation}
\begin{split}
A\subseteq B &\iff A\cap B=A\\
& \iff A\cup B=B\\
& \iff A\setminus B=\{\}.
\end{split}
\end{equation}
Kontraposition:
\begin{equation}
A\subseteq B = B^\comp\subseteq A^\comp.
\end{equation}

\subsection{Natürliche Zahlen}
\subsubsection{Von-Neumann-Modell}
Mengentheoretisches Modell der natürlichen Zahlen:
\begin{equation}
\begin{split}
& 0:=\{\},\quad 1:=\{0\},\quad 2:=\{0,1\},\\
& 3:=\{0,1,2\},\quad \text{usw.}
\end{split}
\end{equation}
Nachfolgerfunktion:
\begin{equation}
x' := x\cup\{x\}.
\end{equation}
\subsubsection{Vollständige Induktion}
Ist $A(n)$ mit $n\in\N$
eine Aussageform, so gilt:
\begin{equation}
\begin{split}
& A(n_0)\land \forall n\ge n_0\,[A(n)\Rightarrow A(n+1)]\\
& \implies \forall n\ge n_0\,[A(n)].
\end{split}
\end{equation}
Die Aussage $A(n_0)$ ist der \emph{Induktionsanfang}.

Die Implikation
\begin{equation}
A(n)\Rightarrow A(n+1)
\end{equation}
heißt \emph{Induktionsschritt}. Beim Induktionsschritt muss
$A(n+1)$ gezeigt werden, wobei $A(n)$ als gültig vorausgesetzt werden
darf.

% \newpage
\subsection{ZFC-Axiome}

Axiom der Bestimmtheit:
\begin{equation}
\forall A\forall B\,[A=B\iff\forall x\,[x\in A\Leftrightarrow x\in B]].
\end{equation}
Axiom der leeren Menge:
\begin{equation}
\exists M\forall x\,[x\notin M].
\end{equation}
Axiom der Paarung:
\begin{equation}
\forall x\forall y\exists M\forall a\,[a\in M\iff x=a\lor y=a].
\end{equation}
Axiom der Vereinigung:
\begin{equation}
\forall S\exists M\forall x\,[x\in M\iff\exists A{\in}S\,[x\in A]].
\end{equation}
Axiom der Aussonderung:
\begin{equation}
\forall A\exists M\forall x\,[x\in M\iff x\in A\land\varphi(x)].
\end{equation}
Axiom des Unendlichen:
\begin{equation}
\exists M\,[\{\}\in M\land\forall x{\in}M\,[x\cup\{x\}\in M]].
\end{equation}
Axiom der Potenzmenge:
\begin{equation}
\forall A\exists M\forall T\,[T\in M\iff T\subseteq A].
\end{equation}
Axiom der Ersetzung:
\begin{equation}
\begin{split}
&\forall a{\in}A\;\exists^{=1} b\,[\varphi(a,b)]\\
&\implies\exists B\,\forall b\,[b\in B\iff\exists a{\in}A\,[\varphi(a,b)]].
\end{split}
\end{equation}
Axiom der Fundierung:
\begin{equation}
\forall A\,[A\ne\{\}\implies\exists x{\in}A\,[x\cap A=\{\}]].
\end{equation}
Auswahlaxiom:
\begin{equation}
\begin{split}
&\forall x,y{\in}A\,[x\ne y\implies x\cap y=\{\}]\\
&\quad\land\forall x{\in}A\,[x\ne\{\}]\\
&\implies\exists M\;\forall x{\in}A\;\exists^{=1}u{\in}x\,[u\in M].
\end{split}
\end{equation}

\newpage
\subsection{Kardinalität}
\begin{definition}[Gleichmächtigkeit]
Zwei Mengen $M,N$ heißen \emdef{gleichmächtig}, notiert als
$|M|=|N|$, wenn es eine bijektive Abbildung $f\colon M\to N$ gibt.

Eine Menge $M$ heißt \emdef{weniger mächtig oder gleichmächtig},
notiert als $|M|\le|N|$, wenn es eine injektive Abbildung
$f\colon M\to N$ gibt. Äquivalent dazu ist, dass es eine
surjektive Abbildung $g\colon N\to M$ gibt.

Eine Menge heißt \emdef{abzählbar unendlich}, wenn sie gleichmächtig
zu den natürlichen Zahlen ist.
\end{definition}
Gleichmächtigkeit ist eine Äquivalenzrelation.
\begin{definition}[Kardinalzahl]
Die Äquivalenzklassen
\begin{equation}
|M| := \{N\mid\;{\scriptstyle |M|=|N|}\}
\end{equation}
heißen \emdef{Kardinalzahlen}.
\end{definition}

\strong{Satz von Cantor-Bernstein.}

Aus $|M|\le |N|$ und $|N|\le |M|$ folgt $|M|=|N|$.

\subsubsection{Potenzmengen}

\strong{Satz von Cantor.}
Für jede Menge gilt $|M|<|2^M|$.

Ist $M$ endlich, dann gilt $|M|=2^{|M|}$.


\section{Funktionen}
\subsection{Injektionen}\index{injektiv}
\begin{definition}[Injektion]
Eine Funktion $f\colon A\to B$ heißt \emdef{injektiv},
wenn
\begin{equation}
\forall x_1,x_2\in A\,[f(x_1)=f(x_2)\implies x_1=x_2]
\end{equation}
gilt.
\end{definition}

\begin{definition}[Linksinverse]
Sei $f\colon A\to B$. Eine Funktion $g\colon B\to A$ mit
\begin{equation}
g\circ f = \id_A
\end{equation}
heißt \emdef{Linksinverse} von $f$.
\end{definition}
Eine Funktion ist genau dann injektiv, wenn sie eine Linksinverse
besitzt. Zu einer Injektion kann es aber mehrere unterschiedliche
Linksinverse geben.

\subsection{Surjektionen}\index{surjektiv}
\begin{definition}[Surjektion]
Eine Funktion $f\colon A\to B$ heißt \emdef{surjektiv},\\
wenn $f(A)=B$ ist. Damit ist gemeint, dass jedes Element
der Zielmenge wenigstens einmal der Funktionswert von einem
Element der Definitionsmenge ist.
\end{definition}

\newpage
\begin{definition}[Rechtsinverse]
Sei $f\colon A\to B$. Eine Funktion $g\colon B\to A$ mit
\begin{equation}
f\circ g = \id_B
\end{equation}
heißt \emdef{Rechtsinverse} von $f$.
\end{definition}
Eine Funktion ist genau dann surjektiv, wenn sie eine Rechtsinverse
besitzt. Zu einer Surjektion kann es aber mehrere unterschiedliche
Rechtsinverse geben.

\subsection{Bijektionen}\index{bijektiv}
\begin{definition}[Bijektion]
Eine Funktion $f\colon A\to B$ heißt \emdef{bijektiv},
wenn sie injektiv und surjektiv ist.

Eine Funktion $f\colon A\to B$ ist genau dann bijektiv, wenn es
ein $g$ mit
\begin{equation}
g\circ f = \id_A\quad\text{und}\quad f\circ g = \id_B
\end{equation}
gibt. Wenn $f$ bijektiv ist, so gibt es $g$ genau einmal und
$g$ wird die \emph{Umkehrfunktion}\index{Umkehrfunktion}
oder \emph{Inverse}
von $f$ genannt und als $f^{-1}$ notiert.
\end{definition}

\subsection{Komposition}\index{Komposition}
\begin{definition}[Komposition]
Für zwei Funktionen $f\colon A\to B$
und $g\colon B\to C$ ist die \emdef{Komposition}
($g$ nach $f$)
durch
\begin{equation}\label{eq:composition}
g\circ f\colon A\to C,\quad (g\circ f)(x) := g(f(x))
\end{equation}
definiert.
\end{definition}
Für die Komposition gilt das Assozativgesetz:
\begin{equation}
(f\circ g)\circ h = f\circ(g\circ h).
\end{equation}

Die Komposition von Injektionen ist eine Injektion.

Die Komposition von Surjektionen ist eine Surjektion.

Die Komposition von Bijektionen ist eine Bijektion.

Sind $f,g$ Bijektionen, so gilt
\begin{equation}
(g\circ f)^{-1} = f^{-1}\circ g^{-1}.
\end{equation}

Ist $g\circ f$ injektiv, so ist $f$ injektiv.

Ist $g\circ f$ surjektiv, so ist $g$ surjektiv.

Ist $g\circ f$ bijektiv, so ist $f$ injektiv und $g$ surjektiv.

\begin{definition}[Iteration]
Für eine Funktion $\varphi\colon A\to A$ wird
\begin{equation}
\varphi^0:=\operatorname{id}_A,\quad \varphi^{n+1}:=\varphi^n\circ\varphi
\end{equation}
\emdef{Iteration}\index{Iteration} von $\varphi$ genannt.
\end{definition}

\newpage
\subsection{Einschränkung}\index{Einschränkung}
\begin{definition}[Einschränkung]
Sei $f\colon A\to B$ und $M\subseteq A$.
Die Funktion $g(x)=f(x)$ mit $g\colon M\to B$ wird \emdef{Einschränkung}
von $f$ genannt und mit $f|_M$ notiert.
\end{definition}
Sei $f\colon A\to B$ und $M\subseteq A$.
Mit der Inklusionsabbildung $i(x):=x$ mit $i\colon M\to A$ gilt:
\begin{equation}
f|_M = f\circ i.
\end{equation}
Es gilt
\begin{equation}
g\circ (f|_M) = (g\circ f)|_M.
\end{equation}

\subsection{Bild}\index{Bild}
\begin{definition}[Bild]
Ist $f\colon A\to B$ und $M\subseteq A$, so wird
\begin{equation}
f(M) := \{f(x)\mid x\in M\}
\end{equation}
das \emdef{Bild} von $M$ unter $f$ genannt.
\end{definition}
Es gilt
\begin{align}
&f(M\cup N) = f(M)\cup f(N),\\
&f(M\cap N) \subseteq f(M)\cap f(N),\\
&f\Big(\bigcup_{i\in I}M_i\Big) = \bigcup_{i\in I} f(M_i),\\
&I\ne\emptyset\implies f\Big(\bigcap_{i\in I} M_i\Big) \subseteq \bigcap_{i\in I} f(M_i),\\
&M\subseteq N\implies f(M)\subseteq f(N),\\
&f(\emptyset) = \emptyset,\\
&(g\circ f)(M) = g(f(M)),\\
&f(M) = \bigcup_{x\in M} f(\{x\}).
\end{align}

\subsection{Urbild}\index{Urbild}
\begin{definition}[Urbild]
Ist $f\colon A\to B$, so wird
\begin{equation}
f^{-1}(M) := \{x\in A\mid f(x)\in M\}.
\end{equation}
das \emdef{Urbild} von $M$ unter $f$ genannt.
\end{definition}
Es gilt
\begin{align}
& f^{-1}(M\cup N) = f^{-1}(M)\cup f^{-1}(N),\\
& f^{-1}(M\cap N) = f^{-1}(M)\cap f^{-1}(N),\\
& f^{-1}\Big(\bigcup_{i\in I}M_i\Big) = \bigcup_{i\in I} f^{-1}(M_i),\\
& I\ne\emptyset\implies f^{-1}\Big(\bigcap_{i\in I} M_i\Big) = \bigcap_{i\in I}f^{-1}(M_i),\\
& M\subseteq N\implies f^{-1}(M)\subseteq f^{-1}(N),\\
& f^{-1}(\emptyset) = \emptyset,\\
& f^{-1}(B) = A,\\
& f^{-1}(M\setminus N) = f^{-1}(M)\setminus f^{-1}(N),\\
& f^{-1}(B\setminus M) = B\setminus f^{-1}(M),\\
& (g\circ f)^{-1}(M) = f^{-1}(g^{-1}(M)),\\
& (f|_M)^{-1}(N) = M\cap f^{-1}(N).
\end{align}

\newpage
\phantom{x}

\newpage
\section{Formale Systeme}
\subsection{Formale Sprachen}
\begin{definition}[Formale Sprache]
Eine \emdef{formale Sprache} $L$ ist eine Teilmenge der kleenschen
Hülle über einer Menge $\Sigma$, kurz $L\subseteq\Sigma^*$.
Die Menge $\Sigma$ wird \emdef{Alphabet} genannt,
ihre Elemente heißen \emdef{Symbole}.

Die kleensche Hülle $\Sigma^*$ besteht aus allen möglichen
Konkatenationen von Symbolen aus $\Sigma$. Die Konkatenationen
von $\Sigma^*$ heißen \emdef{Wörter}. Die leere Konkatenation ist
zulässig und wird mit $\varepsilon$ notiert. Die Elemente von $L$ heißen
\emdef{wohlgeformte Wörter} oder \emdef{wohlgeformte Formeln},
engl. \emdef{well formed formulas}, kurz \emdef{wff}.
\end{definition}

Ein Wort $a$ ist ein Tupel
\begin{equation}
a = (a_1,\ldots, a_m).\qquad (a_k\in\Sigma)
\end{equation}
Sind $a,b$ zwei Wörter, dann ist mit $ab$ deren Konkatenation
gemeint:
\begin{equation}
ab := (a_1,\ldots,a_m,b_1,\ldots b_n).
\end{equation}
Es gilt $\varepsilon a=a$ und $a\varepsilon=a$.
Bei $\varepsilon$ handelt es sich um das leere Tupel.

\begin{definition}[Konkatenation von Sprachen]
\emdef{Konkatenation} von $L_1$ und $L_2$:
\begin{equation}
L_1\circ L_2 := \{ab\mid a\in L_1, b\in L_2\}.
\end{equation}
\end{definition}

\begin{definition}[Potenz einer Sprache]
\emdef{Potenzen} von $L$:
\begin{align}
L^0 &:= \{\varepsilon\},\\
L^n &:= L^{n-1}\circ L.
\end{align}
\end{definition}

\begin{definition}[Kleensche Hülle einer Sprache]
\emdef{Kleensche Hülle} von $L$:
\begin{equation}
L^* := \bigcup_{k\in\N_0} L^k.
\end{equation}

\emdef{Positive Hülle} von $L$:
\begin{equation}
L^+ := \bigcup_{k\in\N_1} L^k.
\end{equation}
\end{definition}

% \newpage
\subsection{Formale Grammatiken}
\begin{definition}[Formale Grammatik]
Eine \emdef{formale Grammatik} ist ein Tupel $(N,\Sigma,P,S)$,
wobei $N$ die \emdef{Nonterminalsymbolen}\index{Nonterminalsymbol},
$\Sigma$ die \emdef{Terminalsymbolen}\index{Terminalsymbol},
$P$ die \emdef{Produktionsregeln}\index{Produktionsregel} sind
und $S$ ein \emdef{Startsymbol}\index{Startsymbol} ist.
Die Mengen $N,\Sigma,P$ müssen endlich sein. Die Mengen $N$ und
$\Sigma$ müssen disjunkt sein. Bei $\Sigma$ handelt es sich um
ein Alphabet. Das Startsymbol ist ein Element $S\in N$.

Bei $P$ handelt es sich um eine Relation
\begin{equation}\label{eq:einfache-Produktionsregeln}
P\subseteq N\times (N\cup\Sigma)^*
\end{equation}
oder allgemeiner
\begin{equation}
P\subseteq (N\cup\Sigma)^*\setminus\Sigma^*\times (N\cup\Sigma)^*.
\end{equation}
Produktionsregeln werden in der Form $n\to w$ notiert und drücken aus,
dass in jedem Wort das Nonterminalsymbol $n$ durch das Wort $w$ ersetzt
werden darf. Allgemeiner bedeutet $t\to w$, dass ein Teilwort $t$
durch $w$ ersetzt werden darf.

Die Produktionsregeln werden ausgehend vom Startsymbol immer weiter
angewendet bis keine Nonterminalsymbole mehr vorhanden sind.
Die Menge aller möglichen Produktionen bildet
eine formale Sprache $L\subseteq\Sigma^*$.
\end{definition}

Für Produktionsregeln der Form \eqref{eq:einfache-Produktionsregeln}
wurde eine Kurznotation geschaffen, die EBNF:

\begin{tabular}{l|l}
\verb|Symbol| & Nonterminalsymbol\\
\verb|"Symbol"| & Terminalsymbol\\
\verb|w1, w2| & $w_1w_2$ (Konkatenation)\\
\verb/n = w1 | w2./ & $n\to w_1,\; n\to w_2$\\
\verb|n = {w}.| & $n\to \varepsilon,\; n\to wn$\\
\verb|n = [w].| & $n\to w,\; n\to wn$
\end{tabular}

\subsection{Formale Systeme}
\begin{definition}[Formales System]
Ein \emdef{formales System} ist ein Tupel $(\Sigma,L,A,R)$, wobei
$\Sigma$ ein Alphabet, $L$ eine formale Sprache über
dem Alphabet, $A$ eine Menge von Axiomen und $R$ eine Menge von
Ableitungsrelationen ist. Die Menge der \emdef{Axiome} ist eine
beliebige Teilmenge von $L$. 
Eine \emdef{Ableitungsrelation} ist eine zwei oder mehrstellige
Relation über $L$, die
\begin{equation}
a_1,\ldots,a_n\vdash b
\end{equation}
geschrieben wird. Eine wohlgeformte Formel wird $\emdef{Satz}$
genannt, wenn sie ein Axiom ist oder über eine Kette von
Ableitungen aus den Axiomen folgt.
\end{definition}

\subsection{Semantik}
\begin{definition}[Interpretation (Aussagenlogik)]
Eine \emph{Interpretation}\index{Interpretation}
$I\colon V\to\{0,1\}$ ist eine Abbildung,
welche jeder logischen Variablen einen Wahrheitswert zuordnet.

Eine \emph{Interpretation} $I\colon F\to\{0,1\}$ erweitert den
Definitionsbereich einer Interpretation wie folgt auf die
Menge aller wohlgeformten Formeln:
\begin{gather}
I(\varphi\land\psi) = (I(\varphi)\land I(\psi)),\\
I(\varphi\lor\psi) = (I(\varphi)\lor I(\psi)),\\
I(\varphi\rightarrow\psi) = (I(\varphi)\rightarrow I(\psi)),\\
I(\varphi\leftrightarrow\psi) = (I(\varphi)\leftrightarrow I(\psi)),\\
I(\neg\varphi) = (\neg I(\varphi)).
\end{gather}
Die rechten Seiten werden hierbei entsprechend den Wertetabellen
ausgewertet.
\end{definition}

\begin{definition}[Modellrelation]
Sei $\Gamma=\{\varphi_1,\ldots,\varphi_n\}$ eine endliche Menge
von Formeln und sei $\psi$ eine Formel. Die Formelmenge $\Gamma$
\emph{modelliert} $\psi$, wenn jede Interpretation, die alle
Formeln in $\Gamma$ erfüllt, auch $\psi$ erfüllt. Kurz:
\begin{equation}
(\Gamma\models\psi) \defiff \forall I[\forall\varphi{\in}\Gamma(I(\varphi))\Rightarrow I(\psi)].
\end{equation}
\end{definition}

\newpage
\section{Mathematische Strukturen}\label{sec:Strukturen}
\subsubsection*{Axiome}

\noindent\bsf{E:} Abgeschlossenheit.
\ibox{Die Verknüpfung führt nicht aus der Menge heraus.}

\noindent\bsf{A:} Assoziativgesetz.
\ibox{$\forall a,b,c\bright (a*b)*c = a*(b*c)\bleft$.}

\noindent\bsf{N:} Existenz des neutralen Elements.
\ibox{$\exists e\forall a\bright e*a=a*e=a\bleft$.}

\noindent\bsf{I:} Existenz der inversen Elemente.
\ibox{$\forall a\exists b\bright a*b=b*a=e\bleft$.}

\noindent\bsf{K:} Kommutativgesetz.
\ibox{$\forall a,b\bright a*b=b*a\bleft.$}

\noindent
\bsf{I*:} Existenz der multiplikativ inversen Elemente.
\ibox{$\forall a{\ne}0\;\exists b\bright a*b=b*a=1\bleft$.}

\noindent\bsf{Dl:} Linksdistributivgestz.
\ibox{$\forall a,x,y\bright a*(x+y) = a*x+a*y\bleft$.}

\noindent\bsf{Dr:} Rechtsdistributivgesetz.
\ibox{$\forall a,x,y\bright (x+y)*a = x*a+y*a\bleft$.}

\noindent\bsf{D:} Distributivgesetze.
\ibox{Dl und Dr.}

\noindent\bsf{T:} Nullteilerfreiheit.
\ibox{$\forall a,b\bright a\ne 0\land b\ne 0\implies a*b\ne 0\bleft$}
\ibox{bzw. die Kontraposition}
\ibox{$\forall a,b\bright a*b=0\implies a=0\lor b=0\bleft$.}

\noindent\bsf{U:} Unterscheibarkeit von Null- und Einselement.
\ibox{Die neutralen Elemente bezüglich Addition und}
\ibox{Multiplikation sind unterschiedlich.}

\subsubsection*{Strukturen}
Strukturen mit einer inneren Verknüpfung:\\
\begin{tabular}{l|l}
\bsf{EA} & Halbgruppe\\
\bsf{EAN} & Monoid\\
\bsf{EANI} & Gruppe\\
\bsf{EANIK} & abelsche Gruppe
\end{tabular}

\noindent
Strukturen mit zwei inneren Verknüpfungen:\\
\begin{tabular}{l|l}
\bsf{EANIK, EA, D}\dotfill & Ring\\
\bsf{EANIK, EAK, D}\dotfill & kommutativer Ring\\
\bsf{EANIK, EAN, D}\dotfill & unitärer Ring\\
\bsf{EANIK, EANK, DTU} & Integritätsring\\
\bsf{EANIK, EANI*K, DTU} & Körper
\end{tabular}

\newpage
\subsubsection*{Axiome für Relationen}

\noindent\bsf{R:} Reflexivität.
\ibox{$\forall a\,(a R a)$.}

\noindent\bsf{S:} Symmetrie.
\ibox{$\forall a,b\,(aRb\iff bRa)$.}

\noindent\bsf{T:} Transitivität.
\ibox{$\forall a,b,c\,(aRb\land bRc\implies aRc)$.}

\noindent\bsf{An:} Antisymmetrie.
\ibox{$\forall a,b\,(aRb\land bRa\implies a=b)$.}

\noindent\bsf{L:} Linearität.
\ibox{$\forall a,b\,(aRb\lor bRa)$.}

\noindent\bsf{Ri:} Irrreflexivität.
\ibox{$\forall a\,(\neg aRa)$.}

\noindent\bsf{A:} Asymmetrie.
\ibox{$\forall a,b\,(aRb\implies \neg bRa)$.}

\noindent\bsf{Min:} Existenz der Minimalelemente.
\ibox{$\forall T{\subseteq}M, T{\ne}\emptyset\;\exists x{\in}T\;\forall y{\in}T{\setminus}\{x\}\,(x<y)$.}

\subsubsection*{Relationen}
\begin{tabular}{l|l}
\bsf{RST}\dotfill & Äquivalenzrelation\\
\bsf{RAnT}\dotfill & Halbordnung\\
\bsf{RAnTL}\dotfill & Totalordnung\\
\bsf{RiAT}\dotfill & strenge Halbordnung\\
\bsf{RiATL}\dotfill & strenge Totalordnung\\
\bsf{RiATLMin} & Wohlordnung
\end{tabular}


\chapter{Anhang}
\section{Mathematische Konstanten}
\begin{enumerate}
\item Kreiszahl\\
$\pi = 3.14159\;26535\;89793\;23846\;26433\;83279\ldots$

\item Eulersche Zahl\\
$\ee = 2.71828\;18284\;59045\;23536\;02874\;71352\ldots$

\item Euler-Mascheroni-Konstante\\
$\gamma = 0.57721\;56649\;01532\;86060\;65120\;90082\ldots$

\item Goldener Schnitt, $(1+\sqrt{5})/2$\\
$\varphi = 1.61803\;39887\;49894\;84820\;45868\;34365\ldots$

\item 1. Feigenbaum-Konstante\\
$\delta = 4.66920\;16091\;02990\;67185\;32038\;20466\ldots$

\item 2. Feigenbaum-Konstante\\
$\alpha = 2.50290\;78750\;95892\;82228\;39028\;73218\ldots$
\end{enumerate}

\section{Physikalische Konstanten}

\begin{enumerate}
\item Lichtgeschwindigkeit im Vakuum\\
$c=299\:792\:458\:\unit{m/s}$

\item Elektrische Feldkonstante\\
$\varepsilon_0 = 8.854\:187\:817\:620\:39\times 10^{-12}\:\unit{F/m}$

\item Magnetische Feldkonstante\\
$\mu_0 = 4\pi\times 10^{-7}\:\unit{H/m}$

\item Elementarladung\\
$e = 1.602\:176\:6208(98)\times 10^{-19}\:\unit{C}$
\end{enumerate}

\newpage
\section{Griechisches Alphabet}

\begin{tabular}{l|l}
\begin{tabular}[t]{lll}
$\mathrm A$ & $\alpha$   & Alpha\\
$\mathrm B$ & $\beta$    & Beta\\
$\Gamma$    & $\gamma$   & Gamma\\
$\Delta$    & $\delta$   & Delta\\
\noalign{\vspace{1em}}
$\mathrm E$ & $\varepsilon$ & Epsilon\\
$\mathrm Z$ & $\zeta$    & Zeta\\
$\mathrm H$ & $\eta$     & Eta\\
$\Theta$    & $\theta$   & Theta\\
\noalign{\vspace{1em}}
$\mathrm I$ & $\iota$    & Jota\\
$\mathrm K$ & $\kappa$   & Kappa\\
$\Lambda$   & $\lambda$  & Lambda\\
$\mathrm M$ & $\mu$      & My
\end{tabular}
&
\begin{tabular}[t]{lll}
$\mathrm N$ & $\nu$      & Nu\\
$\Xi$       & $\xi$      & Xi\\
$\mathrm O$ & $o$        & Omikron\\
$\Pi$       & $\pi$      & Pi\\
\noalign{\vspace{1em}}
$\mathrm R$ & $\varrho$  & Rho\\
$\Sigma$    & $\sigma$   & Sigma\\
$\mathrm T$ & $\tau$     & Tau\\
$\mathrm Y$ & $y$        & Ypsilon\\
\noalign{\vspace{1em}}
$\Phi$      & $\varphi$  & Phi\\
$\mathrm X$ & $\chi$     & Chi\\
$\Psi$      & $\psi$     & Psi\\
$\Omega$    & $\omega$   & Omega 
\end{tabular}
\end{tabular}

\section{Frakturbuchstaben}
\begin{tabular}{l|l}
\begin{tabular}[t]{l@{\hskip 2pt}ll@{\hskip 2pt}l}
A & a & $\mathfrak A$ & $\mathfrak a$\\
B & b & $\mathfrak B$ & $\mathfrak b$ \\
C & c & $\mathfrak C$ & $\mathfrak c$\\
D & d & $\mathfrak D$ & $\mathfrak d$\\
\noalign{\vspace{1em}}
E & e & $\mathfrak E$ & $\mathfrak e$\\
F & f & $\mathfrak F$ & $\mathfrak f$\\
G & g & $\mathfrak G$ & $\mathfrak g$\\
H & h & $\mathfrak H$ & $\mathfrak h$\\
\noalign{\vspace{1em}}
I & i & $\mathfrak I$ & $\mathfrak i$\\
J & j & $\mathfrak J$ & $\mathfrak j$\\
K & k & $\mathfrak K$ & $\mathfrak k$\\
L & l & $\mathfrak L$ & $\mathfrak l$\\
\noalign{\vspace{1em}}
M & m & $\mathfrak M$ & $\mathfrak m$\\
N & n & $\mathfrak N$ & $\mathfrak n$
\end{tabular}
&
\begin{tabular}[t]{l@{\hskip 2pt}ll@{\hskip 2pt}l}
O & o & $\mathfrak O$ & $\mathfrak o$\\
P & p & $\mathfrak P$ & $\mathfrak p$\\
Q & q & $\mathfrak Q$ & $\mathfrak q$\\
R & r & $\mathfrak R$ & $\mathfrak r$\\
\noalign{\vspace{1em}}
S & s & $\mathfrak S$ & $\mathfrak s$\\
T & t & $\mathfrak T$ & $\mathfrak t$\\
U & u & $\mathfrak U$ & $\mathfrak u$\\
V & v & $\mathfrak V$ & $\mathfrak v$\\
\noalign{\vspace{1em}}
W & w & $\mathfrak W$ & $\mathfrak w$\\
X & x & $\mathfrak X$ & $\mathfrak x$\\
Y & y & $\mathfrak Y$ & $\mathfrak y$\\
Z & z & $\mathfrak Z$ & $\mathfrak z$
\end{tabular}
\end{tabular}




\end{document}


