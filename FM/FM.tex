\documentclass[a4paper,10pt,fleqn,onecolumn,twoside]{book}
\usepackage[utf8]{inputenc}
\usepackage[T1]{fontenc}
\usepackage{lmodern}
\usepackage{ngerman}
\usepackage{amsmath}
\usepackage{amssymb}
\usepackage{color}
\definecolor{c1}{RGB}{00,40,80}
\usepackage[colorlinks=true,linkcolor=c1]{hyperref}
\usepackage{geometry}
\geometry{a4paper,left=25mm,right=12mm,top=20mm,bottom=28mm}
\setlength{\columnsep}{6mm}
\usepackage{lipsum}
\usepackage{multicol}
\usepackage[toc]{multitoc}
\setcounter{secnumdepth}{4}
\usepackage{titlesec}
\titleformat{\chapter}[block]
  {\normalfont\huge\bfseries}{\thechapter}{1em}{\Huge}
\titlespacing*{\chapter}{0pt}{0pt}{10pt}
\titleformat{\section}[block]
  {\normalfont\Large\bfseries}{\thesection}{1em}{\Large}
\titleformat{\subsection}[block]
  {\normalfont\large\bfseries}{\thesubsection}{1em}{\large}
\titleformat{\subsubsection}[block]
  {\normalfont\large\bfseries}{\thesubsubsection}{1em}{\large}

\numberwithin{equation}{chapter}
\newenvironment{ttsection}{\ttfamily}{\par}
\newcommand{\strong}[1]{\textbf{#1}}

% \ui: imaginäre Einheit
% \ue: Einheitsvektor
\newcommand{\ui}{\mathrm i}
\newcommand{\ue}{e}

\begin{document}
% \thispagestyle{empty}

\begin{titlepage}
\centering
\phantom{x}

\vspace{20em}
{\noindent\Huge\textbf{Formelsammlung\\
Mathematik}}

\vspace{2em}
{\Large November 2016}\\
\end{titlepage}

\noindent
Dieses Buch ist unter der Lizenz\\
Creative Commons CC0 veröffentlicht.
\vspace{8em}

\noindent
$\sin(-x) = -\sin x$\\
$\cos(-x) = \cos x$
\vspace{1em}

\noindent
$\sin(x+y) = \sin x\cos y + \cos x\sin y$\\
$\sin(x-y) = \sin x\cos y - \cos x\sin y$\\
$\cos(x+y) = \cos x\cos y - \sin x\sin y$\\
$\cos(x-y) = \cos x\cos y + \sin x\sin y$
\vspace{2em}

\noindent
\begin{ttsection}
\begin{tabular}{r|r|r|r}
 0 & 0000 & 0 &  0\\
 1 & 0001 & 1 &  1\\
 2 & 0010 & 2 &  2\\
 3 & 0011 & 3 &  3\\
\noalign{\vspace{1em}}
 4 & 0100 & 4 &  4\\
 5 & 0101 & 5 &  5\\
 6 & 0110 & 6 &  6\\
 7 & 0111 & 7 &  7\\
\noalign{\vspace{1em}}
 8 & 1000 & 8 & 10\\
 9 & 1001 & 9 & 11\\
10 & 1010 & A & 12\\
11 & 1011 & B & 13\\
\noalign{\vspace{1em}}
12 & 1100 & C & 14\\
13 & 1101 & D & 15\\
14 & 1110 & E & 16\\
15 & 1111 & F & 17
\end{tabular}
\end{ttsection}



\tableofcontents

\begin{multicols}{2}
\chapter{Grundlagen}
\section{Komplexe Zahlen}
\subsection{Rechenoperationen}

\begin{gather}
z_1+z_2 = (a_1+a_2)+(b_1+b_2)\ui\\
z_1-z_2 = (a_1-a_2)+(b_1-b_2)\ui\\
z_1 z_2 = (a_1 a_2 - b_1 b_2)+(a_1 b_2+a_2 b_1)\ui\\
\frac{z_1}{z_2}
= \frac{a_1 a_2 + b_1 b_2}{a_2^2+b_2^2}
+ \frac{a_2 b_1 - a_1 b_2}{a_2^2+b_2^2}\ui\\
\frac{1}{z} = \frac{a}{a^2+b^2}-\frac{b}{a^2+b^2}\ui
\end{gather}

\end{multicols}
\section{Mengenlehre}
\subsection{Boolesche Algebra}
\begin{tabular}{l|l|l}
\strong{Vereinigung} & \strong{Schnitt} &\\
  $A\cup A = A$
& $A\cap A = A$
& Idempotenzgesetze\\
  $A\cup \{\} = A$
& $A\cap G = A$
& Neutralitätsgesetze\\
  $A\cup G = G$
& $A\cap \{\} = \{\}$
& Extremalgesetze\\
  $A\cup \overline A = G$
& $A\cap \overline A = \{\}$
& Komplementärgesetze\\
\noalign{\vspace{1em}}
  $A\cup B = B\cup A$
& $A\cap B = B\cap A$
& Kommutativgesetze\\
  $(A\cup B)\cup C = A\cup (B\cup C)$
& $(A\cap B)\cap C = A\cap (B\cap C)$
& Assoziativgesetze\\
  $\overline{A\cup B} = \overline A\cap\overline B$
& $\overline{A\cap B} = \overline A\cup\overline B$
& De Morgansche Regeln\\
  $A\cup (A\cap B) = A$
& $A\cap (A\cup B) = A$
& Absorptionsgesetze\\
\end{tabular}
\begin{multicols}{2}
\noindent
$G$: Grundmenge\\

\noindent
\strong{Distributivgesetze}:
\begin{gather}
M\cup (A\cap B) = (M\cup A)\cap (M\cup B)\\
M\cap (A\cup B) = (M\cap A)\cup (M\cap B)
\end{gather}
\subsection{Teilmengenrelation}
Zerlegung der Gleichheit:
\begin{equation}
A=B \iff A\subseteq B \land B\subseteq A
\end{equation}
Umschreibung der Teilmengenrelation:
\begin{equation}
\begin{split}
A\subseteq B &\iff A\cap B=A\\
& \iff A\cup B=B\\
& \iff A\setminus B=\{\}
\end{split}
\end{equation}
Kontraposition:
\begin{equation}
A\subseteq B = \overline B\subseteq \overline A
\end{equation}
\end{multicols}

\chapter{Anhang}
\section{Griechisches Alphabet}

\begin{tabular}{l|l}
\begin{tabular}[t]{lll}
$\mathrm A$ & $\alpha$   & Alpha\\
$\mathrm B$ & $\beta$    & Beta\\
$\Gamma$    & $\gamma$   & Gamma\\
$\Delta$    & $\delta$   & Delta\\
\noalign{\vspace{1em}}
$\mathrm E$ & $\varepsilon$ & Epsilon\\
$\mathrm Z$ & $\zeta$    & Zeta\\
$\mathrm H$ & $\eta$     & Eta\\
$\Theta$    & $\theta$   & Theta\\
\noalign{\vspace{1em}}
$\mathrm I$ & $\iota$    & Jota\\
$\mathrm K$ & $\kappa$   & Kappa\\
$\Lambda$   & $\lambda$  & Lambda\\
$\mathrm M$ & $\mu$      & My
\end{tabular}
&
\begin{tabular}[t]{lll}
$\mathrm N$ & $\nu$      & Nu\\
$\Xi$       & $\xi$      & Xi\\
$\mathrm O$ & $o$        & Omikron\\
$\Pi$       & $\pi$      & Pi\\
\noalign{\vspace{1em}}
$\mathrm R$ & $\rho$     & Rho\\
$\Sigma$    & $\sigma$   & Sigma\\
$\mathrm T$ & $\tau$     & Tau\\
$\mathrm Y$ & $y$        & Ypsilon\\
\noalign{\vspace{1em}}
$\Phi$      & $\varphi$  & Phi\\
$\mathrm X$ & $\chi$     & Chi\\
$\Psi$      & $\psi$     & Psi\\
$\Omega$    & $\omega$   & Omega 
\end{tabular}
\end{tabular}

\section{Frakturbuchstaben}
\begin{tabular}{l|l}
\begin{tabular}[t]{l@{\hskip 2pt}ll@{\hskip 2pt}l}
A & a & $\mathfrak A$ & $\mathfrak a$\\
B & b & $\mathfrak B$ & $\mathfrak b$ \\
C & c & $\mathfrak C$ & $\mathfrak c$\\
D & d & $\mathfrak D$ & $\mathfrak d$\\
\noalign{\vspace{1em}}
E & e & $\mathfrak E$ & $\mathfrak e$\\
F & f & $\mathfrak F$ & $\mathfrak f$\\
G & g & $\mathfrak G$ & $\mathfrak g$\\
H & h & $\mathfrak H$ & $\mathfrak h$\\
\noalign{\vspace{1em}}
I & i & $\mathfrak I$ & $\mathfrak i$\\
J & j & $\mathfrak J$ & $\mathfrak j$\\
K & k & $\mathfrak K$ & $\mathfrak k$\\
L & l & $\mathfrak L$ & $\mathfrak l$\\
\noalign{\vspace{1em}}
M & m & $\mathfrak M$ & $\mathfrak m$\\
N & n & $\mathfrak N$ & $\mathfrak n$
\end{tabular}
&
\begin{tabular}[t]{l@{\hskip 2pt}ll@{\hskip 2pt}l}
O & o & $\mathfrak O$ & $\mathfrak o$\\
P & p & $\mathfrak P$ & $\mathfrak p$\\
Q & q & $\mathfrak Q$ & $\mathfrak q$\\
R & r & $\mathfrak R$ & $\mathfrak r$\\
\noalign{\vspace{1em}}
S & s & $\mathfrak S$ & $\mathfrak s$\\
T & t & $\mathfrak T$ & $\mathfrak t$\\
U & u & $\mathfrak U$ & $\mathfrak u$\\
V & v & $\mathfrak V$ & $\mathfrak v$\\
\noalign{\vspace{1em}}
W & w & $\mathfrak W$ & $\mathfrak w$\\
X & x & $\mathfrak X$ & $\mathfrak x$\\
Y & y & $\mathfrak Y$ & $\mathfrak y$\\
Z & z & $\mathfrak Z$ & $\mathfrak z$
\end{tabular}
\end{tabular}

\end{document}


