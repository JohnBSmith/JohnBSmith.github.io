
\chapter{Lineare Algebra}
\section{Skalarprodukt}\index{Skalarprodukt}
\subsection{Axiome}
Axiome für $v,w$ aus einem reellen Vektorraum und $\lambda$ ein Skalar:
\begin{gather}
\langle v,w\rangle = \langle w,v\rangle,\\
\langle v,\lambda w\rangle = \lambda\langle v,w\rangle,\\
\langle v,w_1+w_2\rangle = \langle v,w_1\rangle +\langle v,w_2\rangle,\\
\langle v,v\rangle\ge 0,\\
\langle v,v\rangle=0 \iff v=0.
\end{gather}
Axiome für $v,w$ aus einem komplexen Vektorraum und $\lambda$ ein Skalar:
\begin{gather}
\langle v,w\rangle = \overline{\langle w,v\rangle},\\
\langle \lambda v,w\rangle = \overline{\lambda}\langle v,w\rangle,\\
\langle v,\lambda w\rangle = \lambda\langle v,w\rangle,\\
\langle v,w_1+w_2\rangle = \langle v,w_1\rangle +\langle v,w_2\rangle,\\
\langle v,v\rangle\ge 0,\\
\langle v,v\rangle=0 \iff v=0.
\end{gather}
\subsection{Rechenregeln}
Das Skalarprodukt ist eine bilineare Abbildung.

Ist $\varphi$ der Winkel zwishen $v$ und $w$, so gilt:
\begin{equation}
\langle v,w\rangle = \|v\|\,\|w\|\,\cos\varphi.
\end{equation}
Definition \textit{Orthogonal}:\index{Orthogonal}
\begin{equation}
v\perp w \;:\Longleftrightarrow\; \langle v,w\rangle=0.
\end{equation}

\section{Analytische Geometrie}
\subsection{Geraden}\index{Gerade}
\subsubsection{Darstellung}

Punktrichtungsform einer Geraden:
\begin{equation}
p(t) = p_0+t\underline v,
\end{equation}
$p_0$: Stützpunkt, $\underline v$: Richtungsvektor.

Der Vektor $\underline v$ repräsentiert außerdem die Geschwindigkeit,
mit der diese Parameterdarstellung durchlaufen wird: $p'(t)=v$.

\subsubsection{Gerade durch zwei Punkte}
Sind zwei Punkte $p_1,p_2$ mit $p_1\ne p_2$ gegeben, so ist
durch die beiden Punkte eine Gerade gegeben. Für diese Gerade ist
\begin{equation}
p(t) = p_1+t(p_2-p_1)
\end{equation}
eine Punktrichtungsform\index{Punktrichtungsform}.
Durch Umformung ergibt sich die Zweipunkteform:
\begin{equation}\label{eq:Zweipunkteform}
p(t) = (1-t)p_1+tp_2.
\end{equation}
Bei \eqref{eq:Zweipunkteform} handelt es sich um eine
Affinkombination. Gilt $t\in[0,1]$, so ist \eqref{eq:Zweipunkteform}
eine Konvexkombination: eine Parameterdarstellung für die Strecke
von $p_1$ nach $p_2$.

\subsubsection{Abstand Punkt zu Gerade}
Sei $p(t):=p_0+t\underline v$ die Punktrichtungsform einer Geraden und
sei $q$ ein weiterer Punkt. Bei $\underline d(t):=p(t)-q$ handelt
es sich um den Abstandsvektor in Abhängigkeit von $t$.

Ansatz: Es gibt genau ein $t$, so dass gilt:
\begin{equation}
\langle\underline d,\underline v\rangle=0.
\end{equation}
Lösung:
\begin{equation}
t = \frac
  {\langle\underline v,q{-}p_0\rangle}
  {\langle\underline v,\underline v\rangle}.
\end{equation}

\subsection{Ebenen}\index{Ebene}
\subsubsection{Darstellung}
Seien $\uv u, \uv v$ zwei nicht kollineare Vektoren.

Punktrichtungsform:
\begin{equation}
p(s,t) = p_0+s\uv u+t\uv v.
\end{equation}

\subsubsection{Abstand Punkt zu Ebene}
Sei $p(s,t):=p_0+s\uv u+t\uv v$ die Punktrichtungsform einer Ebene
und sei $q$ ein weiterer Punkt. Bei $\uv d(s,t):=p-q$ handelt es sich um
den Abstandsvektor in Abhängigkeit von $(s,t)$.

Ansatz: Es gibt genau ein Tupel $(s,t)$, so dass gilt:
\begin{equation}
\langle\uv d,\uv u\rangle=0  \;\land\; \langle\uv d,\uv v\rangle=0.
\end{equation}
Lösung: Es ergibt sich ein LGS:
\begin{equation}
\begin{bmatrix}
\langle\uv u,\uv v\rangle & \langle\uv v,\uv v\rangle\\
\langle\uv v,\uv v\rangle & \langle\uv u,\uv v\rangle
\end{bmatrix}
\begin{bmatrix}
s\\ t
\end{bmatrix}
= \begin{bmatrix}
\langle\uv v,q{-}p_0\rangle\\
\langle\uv u,q{-}p_0\rangle
\end{bmatrix}.
\end{equation}
Bemerkung: Die Systemmatrix $g_{ij}$ ist der metrische Tensor für die
Basis $B=(\uv u,\uv v)$. Die Lösung des LGS ist:
\begin{gather}
s = \frac
  {\langle g_{12}\uv v-g_{12}\uv u, q{-}p_0\rangle}
  {g_{11}^2-g_{12}^2},\\
t = \frac
  {\langle g_{12}\uv u-g_{12}\uv v, q{-}p_0\rangle}
  {g_{11}^2-g_{12}^2}.
\end{gather}

