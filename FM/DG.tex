
\chapter{Differentialgeometrie}
\section{Kurven}
\subsection{Parameterkurven}\index{Weg}\index{Kurve}
\strong{Definition.} Sei $X$ ein topologischer Raum und
$I$ ein reelles Intervall, auch offen oder halboffen, auch unbeschränkt.
Eine stetige Funktion
\begin{equation}
f\colon I\to X
\end{equation}
heißt \emdef{Parameterdarstellung einer Kurve}, kurz
\emdef{Parameterkurve}. Die Bildmenge $f(I)$ heißt \emdef{Kurve}.

Eine Parameterdarstellung mit einem kompakten Intervall $I=[a,b]$
heißt \emdef{Weg}.

Für einen Weg mit $I=[a,b]$ heißt $f(a)$ \emdef{Anfangspunkt}
und $f(b)$ \emdef{Endpunkt}. Ein Weg mit $f(a)=f(b)$
heißt \emdef{geschlossen}. Ein Weg dessen Einschränkung auf $[a,b)$
injektiv ist, heißt \emdef{einfach}, auch \emdef{doppelpunktfrei} oder
\emdef{Jordan-Weg}.

Bsp. für einen einfachen geschlossenen Weg:
\begin{equation}
f\colon [0,2\pi]\to \R^2,\quad
f(t):=\begin{bmatrix}
\cos t\\
\sin t
\end{bmatrix}.
\end{equation}
Die Kurve ist der Einheitskreis.

Bsp. für einen geschlossenen Weg mit Doppelpunkt:
\begin{equation}
f\colon [0,2\pi]\to \R^2,\quad
f(t):=\begin{bmatrix}
2\cos t\\
\sin(2t)
\end{bmatrix}.
\end{equation}
Die Kurve ist eine Achterschleife.

\subsection{Differenzierbare Parameterkurven}
\strong{Definition.} Eine Parameterkurve $f\colon (a,b)\to\R^n$ heißt
\emdef{differenzierbar}, wenn die Ableitung $f'(t)$ an jeder Stelle
$t$ existiert. Die Ableitung $f'(t)$ wird
\emdef{Tangentialvektor} an die Kurve an der Stelle $t$ genannt.

Ein \emdef{$C^k$-Kurve} ist ein Parameterkurve, dessen $k$-te Ableitung
eine stetige Funktion ist. Ein unendlich oft differenzierbare
Parameterkurve heißt \emdef{glatt}.

Eine Parameterkurve heißt \emdef{regulär}, wenn:
\begin{equation}
\forall t\colon f'(t)\ne 0.
\end{equation}






