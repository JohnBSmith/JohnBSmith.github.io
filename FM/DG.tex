
\chapter{Differentialgeometrie}
\section{Kurven}
\subsection{Parameterkurven}\index{Weg}\index{Kurve}
\begin{Definition}
Sei $X$ ein topologischer Raum und
$I$ ein reelles Intervall, auch offen oder halboffen, auch unbeschränkt.
Eine stetige Funktion
\begin{equation}
f\colon I\to X
\end{equation}
heißt \emdef{Parameterdarstellung einer Kurve}, kurz
\emdef{Parameterkurve}. Die Bildmenge $f(I)$ heißt \emdef{Kurve}.
\end{Definition}

Eine Parameterdarstellung mit einem kompakten Intervall $I=[a,b]$
heißt \emdef{Weg}.

Für einen Weg mit $I=[a,b]$ heißt $f(a)$ \emdef{Anfangspunkt}
und $f(b)$ \emdef{Endpunkt}. Ein Weg mit $f(a)=f(b)$
heißt \emdef{geschlossen}. Ein Weg, dessen Einschränkung auf $[a,b)$
injektiv ist, heißt \emdef{einfach}, auch \emdef{doppelpunktfrei} oder
\emdef{Jordan-Weg}.

Bsp. für einen einfachen geschlossenen Weg:
\begin{equation}
f\colon [0,2\pi]\to \R^2,\quad
f(t):=\begin{bmatrix}
\cos t\\
\sin t
\end{bmatrix}.
\end{equation}
Die Kurve ist der Einheitskreis.

Bsp. für einen geschlossenen Weg mit Doppelpunkt:
\begin{equation}
f\colon [0,2\pi]\to \R^2,\quad
f(t):=\begin{bmatrix}
2\cos t\\
\sin(2t)
\end{bmatrix}.
\end{equation}
Die Kurve ist eine Achterschleife.

\subsection{Differenzierbare Parameterkurven}
\begin{Definition}
Eine Parameterkurve $f\colon (a,b)\to\R^n$ heißt
\emdef{differenzierbar}, wenn die Ableitung $f'(t)$ an jeder Stelle
$t$ existiert. Die Ableitung $f'(t)$ wird
\emdef{Tangentialvektor} an die Kurve an der Stelle $t$ genannt.
\end{Definition}

Ein \emdef{$C^k$-Kurve} ist ein Parameterkurve, dessen $k$-te Ableitung
eine stetige Funktion ist. Ein unendlich oft differenzierbare
Parameterkurve heißt \emdef{glatt}.

Eine Parameterkurve heißt \emdef{regulär}, wenn:
\begin{equation}
\forall t\colon f'(t)\ne 0.
\end{equation}

\section{Koordinatensysteme}
\subsection{Polarkoordinaten}\index{Polarkoordinaten}
Polarkoordinaten $r,\varphi$ sind gegeben durch
die Abbildung%
\begin{equation}
\begin{bmatrix}x\\ y\end{bmatrix}
=f(r,\varphi)
:=\begin{bmatrix}
r\cos\varphi\\
r\sin\varphi
\end{bmatrix}
\end{equation}
mit $r>0$ und $0\le\varphi<2\pi$.

Umkehrabbildung für $(x,y)\ne (0,0)$:
\begin{equation}
\begin{bmatrix}r\\ \varphi\end{bmatrix}
= f^{-1}(x,y)
= \begin{bmatrix}
r\\
\displaystyle s(y)\arccos\Big(\frac{x}{r}\Big)
\end{bmatrix}
\end{equation}
mit $r=\sqrt{x^2+y^2}$\\
und $s(y)=\sgn(y)+1-|\sgn(y)|$.

Jacobi-Determinante:
\begin{equation}
\det J = \det((Df)(r,\varphi)) =r.
\end{equation}

\pagebreak[3]\noindent
Darstellung des metrischen Tensors in Polarkoordinaten:%
\begin{equation}
(g_{ij}) = J^T J = \begin{bmatrix}
1 & 0\\
0 & r^2
\end{bmatrix}.
\end{equation}

\section{Mannigfaltigkeiten}
\subsection{Grundbegriffe}
\begin{Definition}
Seien $U,V$ offene Mengen. Eine Abbildung
\begin{equation}
\varphi\colon (U\subseteq\R^m)\to(V\subseteq\R^n)
\end{equation}
heißt \emdef{regulär}, wenn
\begin{equation}
\forall u\in U\colon \operatorname{rg}((D\varphi)(u))=\min(m,n)
\end{equation}
gilt. Mit $(D\varphi)(u)$ ist dabei die Jacobi-Matrix an der Stelle
$u$ gemeint:
\begin{equation}
((D\varphi)(u))_{ij} := \frac{\partial\varphi_i(u)}{\partial u_j}.
\end{equation}
\end{Definition}
\noindent
Für $(D\varphi)(u)\colon\R^m\to\R^n$ gilt:
\begin{gather}
m{\ge}n\implies\forall u\colon (D\varphi)(u)\;\text{ist surjektiv},\\
m{<}n\implies\forall u\colon (D\varphi)(u)\;\text{ist injektiv}.
\end{gather}

\begin{Definition}
Sei $m,n\in\N, m<n$ und sei $M\subseteq\R^n$.
Eine Abbildung $\varphi$ von einer offenen Menge $U'\subseteq\R^m$
in eine offene Menge $U\subseteq M$ heißt \emdef{Karte},
wenn $\varphi$ ein Homöomorphismus und $\varphi\colon U'\to\R^n$
eine reguläre Abbildung ist. Ist $U$ eine offene Umgebung von
$p\in M$, so heißt $\varphi$ \emdef{lokale Karte} bezüglich $p$.
\end{Definition}
\pagebreak[1]
\begin{Definition}
Sei $m,n\in\N, m<n$. Eine Menge $M\subseteq\R^n$ heißt
\emdef{$m$-dimensionale Untermannigfaltigkeit} des $\R^n$, wenn
es zu jedem Punkt $p\in M$ eine lokale Karte
\begin{equation}
\varphi\colon (U'\subseteq R^m)\to (U\subseteq M\subseteq\R^n)
\end{equation}
gibt.
\end{Definition}
\begin{Definition} Ein \emdef{Atlas} für eine Mannigfaltigkeit $M$
ist eine Menge von Karten, deren Bildmengen $M$ überdecken.
\end{Definition}
Sei $M$ eine glatte Mannigfaltigkeit.
\begin{Definition}
Eine Abbildung $f\colon M\to\R$ ist ($k$ mal) (stetig)
\emdef{differenzierbar}
gdw. für jede Karte $\varphi\colon U'\to (U\subseteq M)$ das
Kompositum $f\circ\varphi$ ($k$ mal) (stetig) differenzierbar ist.
Es genügt der Nachweis für alle Karten aus einem Atlas.
\end{Definition}
Seien $M,N$ zwei glatte Mannigfaltigkeiten.
\begin{Definition} Eine Abbildung $f\colon M\to N$ heißt \emdef{glatt}
gdw. für alle Karten $\varphi\colon U'\to (U\subseteq M)$ und
$\psi\colon V'\to (V\subseteq N)$ das Kompositum
$\psi^{-1}\circ f\circ\varphi$ eine glatte Abbildung ist.
Es genügt bereits der Nachweis für alle Karten aus jeweils einem
Atlas für $M$ und $N$.
\end{Definition}


\subsection{Vektorfelder}
\subsubsection{Tangentialräume}
\begin{Definition} \emdef{Tangentialbündel}\index{Tangentialbündel}:
\begin{equation}
TM := \bigsqcup_{p\in M} T_p M = \bigcup_{p\in M} \{p\}\times T_p M.
\end{equation}
\emdef{Kotangentialbündel}\index{Kotangentialbündel}:
\begin{equation}
T^*M := \bigsqcup_{p\in M} T_p^* M.
\end{equation}
\emdef{Natürliche Projektion}\index{natürliche Projektion}:
\begin{equation}
\pi(p,v):=p,\quad\pi\colon TM\to M.
\end{equation}
\end{Definition}
\noindent
Das Tangentialbündel einer glatten Mannigfaltigkeit ist eine
glatte Mannigfaltigkeit.

\subsubsection{Christoffel-Symbole}\index{Christoffel-Symbole}
Sei $(M,g)$ eine pseudo-riemannsche Mannigfaltigkeit.

Es gilt:
\begin{align}
\Gamma_{ab}^k &= \frac{1}{2} g^{kc}
(\partial_a g_{bc}+\partial_b g_{ac}-\partial_c g_{ab}),\\
\Gamma_{cab} &= \frac{1}{2}
(\partial_a g_{bc}+\partial_b g_{ac}-\partial_c g_{ab}),\\
\partial_a g_{bc} &= \Gamma_{bac}+\Gamma_{cab},\\
\Gamma_{ab}^k &= \Gamma_{ba}^k.
\end{align}


