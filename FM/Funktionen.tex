
\chapter{Funktionen}
\section{Elementare Funktionen}
\subsection{Exponentialfunktion}
\begin{Definition}
$\exp\colon\C\to\C$ mit
\begin{equation}
\exp(x) := \sum_{k=0}^{\infty} \frac{x^k}{k!}.
\end{equation}
\end{Definition}
\noindent
Die Einschränkung von $\exp$ auf $\R$ ist injektiv und
hat die Bildmenge $\{x{\in}\R\mid x>0\}$.

Für alle $x,y\in\C$ gilt:
\begin{equation}
\exp(x+y) = \exp(x)\exp(y).
\end{equation}

\subsection{Winkelfunktionen}\index{Winkelfunktion}
\subsubsection{Additionstheoreme}
\index{Additionstheoreme}

Für alle $x,y\in\C$ gilt:
\begin{gather}
\sin(x+y) = \sin(x)\cos(y)+\cos(x)\sin(y),\\
\sin(x-y) = \sin(x)\cos(y)-\cos(x)\sin(y),\\
\cos(x+y) = \cos(x)\cos(y)-\sin(x)\sin(y),\\
\cos(x-y) = \cos(x)\cos(y)+\sin(x)\sin(y).
\end{gather}

\subsubsection{Produkte}
Für alle $x,y\in\C$ gilt:
\begin{gather}
2\sin x\sin y = \cos(x-y)-\cos(x+y),\\
2\cos x\cos y = \cos(x-y)+\cos(x+y),\\
2\sin x\cos y = \sin(x-y)+\sin(x+y).
\end{gather}
