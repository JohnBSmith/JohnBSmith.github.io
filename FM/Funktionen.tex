
\chapter{Funktionen}
\section{Elementare Funktionen}
\subsection{Exponentialfunktion}
\begin{Definition}
$\exp\colon\C\to\C$ mit
\begin{equation}
\exp(x) := \sum_{k=0}^{\infty} \frac{x^k}{k!}.
\end{equation}
\end{Definition}
\noindent
Die Einschränkung von $\exp$ auf $\R$ ist injektiv und
hat die Bildmenge $\{x{\in}\R\mid x>0\}$.

Für alle $x,y\in\C$ gilt:
\begin{gather}
\exp(x+y) = \exp(x)\exp(y),\\
\exp(x-y) = \frac{\exp(x)}{\exp(y)},\\
\exp(-x) = \frac{1}{\exp(x)}.
\end{gather}

\subsection{Winkelfunktionen}\index{Winkelfunktion}
\begin{Definition} Die Funktion
$\cos\colon\C\to\C$ mit
\begin{equation}
\cos(x) := \sum_{k=0}^\infty \frac{x^{2k}}{(2k)!}
\end{equation}
heißt $\emdef{Kosinus}$.

Die Funktion $\sin\colon\C\to\C$ mit
\begin{equation}
\sin(x) := \sum_{k=0}^\infty \frac{x^{2k+1}}{(2k+1)!}.
\end{equation}
heißt $\emdef{Sinus}$.
\end{Definition}
Die Einschränkungen auf $\R$ sind periodische Funktionen mit
Periodenlänge $2\pi$.

\subsubsection{Additionstheoreme}
\index{Additionstheoreme}

Für alle $x,y\in\C$ gilt:
\begin{gather}
\sin(x+y) = \sin x\cos y+\cos x\sin y,\\
\sin(x-y) = \sin x\cos y-\cos x\sin y,\\
\cos(x+y) = \cos x\cos y-\sin x\sin y,\\
\cos(x-y) = \cos x\cos y+\sin x\sin y.
\end{gather}

\subsubsection{Produkte}
Für alle $x,y\in\C$ gilt:
\begin{gather}
2\sin x\sin y = \cos(x-y)-\cos(x+y),\\
2\cos x\cos y = \cos(x-y)+\cos(x+y),\\
2\sin x\cos y = \sin(x-y)+\sin(x+y).
\end{gather}

\subsubsection{Summen und Differenzen}
Für alle $x,y\in\C$ gilt:
\begin{gather}
\sin x+\sin y = 2\sin\frac{x+y}{2}\cos\frac{x-y}{2},\\
\sin x-\sin y = 2\cos\frac{x+y}{2}\sin\frac{x-y}{2},\\
\cos x+\cos y = 2\cos\frac{x+y}{2}\cos\frac{x-y}{2},\\
\cos x-\cos y = 2\sin\frac{x+y}{2}\sin\frac{y-x}{2}.
\end{gather}

\subsubsection{Winkelvielfache}
Für alle $x\in\C$ gilt:
\begin{gather}
\sin(2x) = 2\sin x\cos x,\\
\cos(2x) = \cos^2 x-\sin^2 x,\\
\sin(3x) = 3\sin x-4\sin^3 x,\\
\cos(3x) = 4\cos^3 x-3\cos x.
\end{gather}
