
\chapter{Analysis}
\section{Ableitungen}
\subsection{Differentialquotient}\index{Differentialquotient}
Sei $U\subseteq\mathbb R$ ein offenes Intervall
und sei $f\colon U\to\mathbb R$. Die Funktion $f$ heißt
differenzierbar\index{differenzierbar}
an der Stelle $x_0\in U$, falls der Grenzwert
\begin{equation}
\begin{split}
&\lim_{x\to x_0} \frac{f(x)-f(x_0)}{x-x_0}
= \lim_{h\to 0}\frac{f(x_0+h)-f(x_0)}{h}
\end{split}
\end{equation}
existiert. Dieser Grenzwert heißt
Differentialquotient oder Ableitung
von $f$ an der Stelle $x_0$. Notation:
\begin{equation}
f'(x_0),\,\qquad (Df)(x_0),\qquad \frac{\mathrm df(x)}{\mathrm dx}\Big|_{x=x_0}.
\end{equation}

\subsection{Ableitungsregeln}
Sind $f,g$ differenzierbare Funktionen und ist $a$ eine reelle Zahl,
so gilt
\begin{gather}
(af)' = af',\\
(f+g)' = f'+g',\\
(f-g)' = f'-g',\\
(fg)' = f'g+g'f,\\
\Big(\frac{f}{g}\Big)'(x) = \frac{(f'g-g'f)(x)}{g(x)^2}.
\end{gather}
\subsubsection{Kettenregel}
Ist $g$ differenzierbar an der Stelle $x_0$ und
$f$ differenzierbar an der Stelle $g(x_0)$, so ist $f\circ g$
differenzierbar an der Stelle $x_0$ und es gilt
\begin{equation}
(f\circ g)'(x_0) = (f'\circ g)(x_0)\, g'(x_0).
\end{equation}

