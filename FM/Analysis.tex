
\chapter{Analysis}
\section{Ableitungen}
\subsection{Differentialquotient}
Sei $U\subseteq\mathbb R$ ein offenes Intervall
und sei $f\colon U\to\mathbb R$. Die Funktion $f$ heißt
differenzierbar an der Stelle $x_0\in U$, falls der Grenzwert
\begin{equation}
\begin{split}
&\lim_{x\to x_0} \frac{f(x)-f(x_0)}{x-x_0}
= \lim_{h\to 0}\frac{f(x+h)-f(x)}{h}
\end{split}
\end{equation}
existiert. Dieser Grenzwert heißt
Differentialquotient oder Ableitung
von $f$ an der Stelle $x_0$. Notation:
\begin{equation}
f'(x_0),\,\qquad (Df)(x_0),\qquad \frac{\mathrm df(x)}{\mathrm dx}\Big|_{x=x_0}.
\end{equation}

