
\chapter{Analysis}
\section{Konvergenz}
\subsection{Umgebungen}\index{Umgebung}
Sei $(X,d)$ ein metrischer Raum und $x\in X$.

\noindent
\strong{Definition.} $\varepsilon$-\emdef{Umgebung}:
\begin{equation}
U_\varepsilon(x) := \{y\in X\mid d(x,y)<\varepsilon\}.
\end{equation}
\emdef{Punktierte $\varepsilon$-Umgebung}:
\begin{equation}
\dot U_\varepsilon(x) := U_\varepsilon(x)\setminus\{x\}.
\end{equation}
Für einen normierten Raum ist durch $d(x,y):=\|x-y\|$ eine
Metrik gegeben. Speziell für $X=\R$ oder $X=\C$ wird fast immer
$d(x,y):=|x-y|$ verwendet.

Sei $(X,T)$ ein topologischer Raum und $x\in X$.

\noindent
\strong{Definition.} \emdef{Umgebungsfilter}:\index{Umgebungsfilter}
\begin{equation}
\mathfrak U(x) := \{U\subseteq X\mid x\in O\land O\in T
\land O\subseteq U\}.
\end{equation}
Ein $U\in\mathfrak U(x)$ wird Umgebung von $x$ genannt.

\subsection{Konvergente Folge}\index{konvergente Folge}
Eine Folge $(a_n)\colon \N\to X$ heißt \emdef{konvergent} gegen $x$, wenn%
\begin{equation}
\forall U{\in}\mathfrak B(x)\,\exists n_0\,\forall n{>}n_0\colon\;
a_n\in U.
\end{equation}
Man schreibt dann $\lim\limits_{n\to\infty} a_n=x$.

\section{Ableitungen}\index{Ableitung}
\subsection{Differentialquotient}\index{Differentialquotient}
Sei $U\subseteq\R$ ein offenes Intervall
und sei $f\colon U\to\R$. Die Funktion $f$ heißt
differenzierbar\index{differenzierbar}
an der Stelle $x_0\in U$, falls der Grenzwert
\begin{equation}
\begin{split}
&\lim_{x\to x_0} \frac{f(x)-f(x_0)}{x-x_0}
= \lim_{h\to 0}\frac{f(x_0+h)-f(x_0)}{h}
\end{split}
\end{equation}
existiert. Dieser Grenzwert heißt
Differentialquotient oder Ableitung
von $f$ an der Stelle $x_0$. Notation:
\begin{equation}
f'(x_0),\,\qquad (Df)(x_0),\qquad \frac{\mathrm df(x)}{\mathrm dx}\Big|_{x=x_0}.
\end{equation}

\subsection{Ableitungsregeln}
Sind $f,g$ differenzierbare Funktionen und ist $a$ eine reelle Zahl,
so gilt
\begin{gather}
(af)' = af',\\
(f+g)' = f'+g',\\
(f-g)' = f'-g',\\
(fg)' = f'g+g'f,\\
\Big(\frac{f}{g}\Big)'(x) = \frac{(f'g-g'f)(x)}{g(x)^2}.
\end{gather}
\subsubsection{Kettenregel}
Ist $g$ differenzierbar an der Stelle $x_0$ und
$f$ differenzierbar an der Stelle $g(x_0)$, so ist $f\circ g$
differenzierbar an der Stelle $x_0$ und es gilt
\begin{equation}
(f\circ g)'(x_0) = (f'\circ g)(x_0)\, g'(x_0).
\end{equation}

