
\chapter{Anhang}
\section{Griechisches Alphabet}

\begin{tabular}{l|l}
\begin{tabular}[t]{lll}
$\mathrm A$ & $\alpha$   & Alpha\\
$\mathrm B$ & $\beta$    & Beta\\
$\Gamma$    & $\gamma$   & Gamma\\
$\Delta$    & $\delta$   & Delta\\
\noalign{\vspace{1em}}
$\mathrm E$ & $\varepsilon$ & Epsilon\\
$\mathrm Z$ & $\zeta$    & Zeta\\
$\mathrm H$ & $\eta$     & Eta\\
$\Theta$    & $\theta$   & Theta\\
\noalign{\vspace{1em}}
$\mathrm I$ & $\iota$    & Jota\\
$\mathrm K$ & $\kappa$   & Kappa\\
$\Lambda$   & $\lambda$  & Lambda\\
$\mathrm M$ & $\mu$      & My
\end{tabular}
&
\begin{tabular}[t]{lll}
$\mathrm N$ & $\nu$      & Ny\\
$\Xi$       & $\xi$      & Xi\\
$\mathrm O$ & $o$        & Omikron\\
$\Pi$       & $\pi$      & Pi\\
\noalign{\vspace{1em}}
$\mathrm R$ & $\varrho$  & Rho\\
$\Sigma$    & $\sigma$   & Sigma\\
$\mathrm T$ & $\tau$     & Tau\\
$\mathrm Y$ & $\upsilon$ & Ypsilon\\
\noalign{\vspace{1em}}
$\Phi$      & $\varphi$  & Phi\\
$\mathrm X$ & $\chi$     & Chi\\
$\Psi$      & $\psi$     & Psi\\
$\Omega$    & $\omega$   & Omega 
\end{tabular}
\end{tabular}

\section{Frakturbuchstaben}
\begin{tabular}{l|l}
\begin{tabular}[t]{l@{\hskip 2pt}ll@{\hskip 2pt}l}
A & a & $\mathfrak A$ & $\mathfrak a$\\
B & b & $\mathfrak B$ & $\mathfrak b$ \\
C & c & $\mathfrak C$ & $\mathfrak c$\\
D & d & $\mathfrak D$ & $\mathfrak d$\\
\noalign{\vspace{1em}}
E & e & $\mathfrak E$ & $\mathfrak e$\\
F & f & $\mathfrak F$ & $\mathfrak f$\\
G & g & $\mathfrak G$ & $\mathfrak g$\\
H & h & $\mathfrak H$ & $\mathfrak h$\\
\noalign{\vspace{1em}}
I & i & $\mathfrak I$ & $\mathfrak i$\\
J & j & $\mathfrak J$ & $\mathfrak j$\\
K & k & $\mathfrak K$ & $\mathfrak k$\\
L & l & $\mathfrak L$ & $\mathfrak l$\\
\noalign{\vspace{1em}}
M & m & $\mathfrak M$ & $\mathfrak m$\\
N & n & $\mathfrak N$ & $\mathfrak n$
\end{tabular}
&
\begin{tabular}[t]{l@{\hskip 2pt}ll@{\hskip 2pt}l}
O & o & $\mathfrak O$ & $\mathfrak o$\\
P & p & $\mathfrak P$ & $\mathfrak p$\\
Q & q & $\mathfrak Q$ & $\mathfrak q$\\
R & r & $\mathfrak R$ & $\mathfrak r$\\
\noalign{\vspace{1em}}
S & s & $\mathfrak S$ & $\mathfrak s$\\
T & t & $\mathfrak T$ & $\mathfrak t$\\
U & u & $\mathfrak U$ & $\mathfrak u$\\
V & v & $\mathfrak V$ & $\mathfrak v$\\
\noalign{\vspace{1em}}
W & w & $\mathfrak W$ & $\mathfrak w$\\
X & x & $\mathfrak X$ & $\mathfrak x$\\
Y & y & $\mathfrak Y$ & $\mathfrak y$\\
Z & z & $\mathfrak Z$ & $\mathfrak z$
\end{tabular}
\end{tabular}

\section{Mathematische Konstanten}
\begin{enumerate}
\item Kreiszahl\\
$\pi = 3{,}14159\;26535\;89793\;23846\;26433\;83279\ldots$

\item Eulersche Zahl\\
$\ee = 2{,}71828\;18284\;59045\;23536\;02874\;71352\ldots$

\item Euler-Mascheroni-Konstante\\
$\gamma = 0{,}57721\;56649\;01532\;86060\;65120\;90082\ldots$

\item Goldener Schnitt, $(1+\sqrt{5})/2$\\
$\varphi = 1{,}61803\;39887\;49894\;84820\;45868\;34365\ldots$

\item 1. Feigenbaum-Konstante\\
$\delta = 4{,}66920\;16091\;02990\;67185\;32038\;20466\ldots$

\item 2. Feigenbaum-Konstante\\
$\alpha = 2{,}50290\;78750\;95892\;82228\;39028\;73218\ldots$
\end{enumerate}

\newpage
\section{Physikalische Konstanten}
\begin{enumerate}
\item Lichtgeschwindigkeit im Vakuum\\
$c=299\;792\;458\;\unit{m/s}$

\item Elektrische Feldkonstante\\
$\varepsilon_0 = 8{,}854\;187\;817\;620\;39\times 10^{-12}\:\unit{F/m}$

\item Magnetische Feldkonstante\\
$\mu_0 = 4\pi\times 10^{-7}\:\unit{H/m}$

\item Elementarladung\\
$e = 1{,}602\;176\;6208\;(98)\times 10^{-19}\,\unit{C}$

\item Gravitationskonstante\\
$G = 6{,}674\;08\;(31)\times 10^{-11}\,\unit{m^3/(kg\,s^2)}$

\item Avogadro-Konstante\\
$N_A = 6{,}022\;140\;857\;(74)\times 10^{23}/\unit{mol}$

\item Boltzmann-Konstante\\
$k_B = 1{,}380\;648\;52\;(79)\times 10^{-23}\,\unit{J/K}$

\item Universelle Gaskonstante\\
$R = 8{,}314\;4598\;(48)\:\unit{J/(mol\,K)}$

\item Plancksches Wirkungsquantum\\
$h = 6{,}626\;070\;040\;(81)\times 10^{-34}\,\unit{Js}$

\item Reduziertes planksches Wirkungsquantum\\
$\hbar = 1{,}054\;571\;800\;(13)\times 10^{-34}\,\unit{Js}$

\item Masse des Elektrons\\
$m_e = 9{,}109\;383\;56\;(11)\times 10^{-31}\,\unit{kg}$

\item Masse des Neutrons\\
$m_n = 1{,}674\;927\;471\;(21)\times 10^{-27}\,\unit{kg}$

\item Masse des Protons\\
$m_p = 1{,}672\;621\;898\;(21)\times 10^{-27}\,\unit{kg}$
\end{enumerate}

\newpage
\section{Einheiten}
\subsection{SI-System}
Newton (Kraft):
\begin{equation}
\unit{N}=\unit{kg\,m/s^2}.
\end{equation}
Watt (Leistung):
\begin{equation}
\unit{W}=\unit{kg\,m^2/s^3}=\unit{VA}.
\end{equation}
Joule (Energie):
\begin{equation}
\unit{J}=\unit{kg\,m^2/s^2}=\unit{Nm}=\unit{Ws}=\unit{VAs}.
\end{equation}
Pascal (Druck):
\begin{equation}
\unit{Pa}=\unit{N/m^2} = 10^{-5}\,\unit{bar}.
\end{equation}
Hertz (Frequenz):
\begin{equation}
\unit{Hz} = \unit{1/s}.
\end{equation}
Coulomb (Ladung):
\begin{equation}
\unit{C} = \unit{As}.
\end{equation}
Volt (Spannung):
\begin{equation}
\unit{V} = \unit{kg\,m^2/(A\,s^3)}
\end{equation}
Tesla (magnetische Flussdichte):
\begin{equation}
\unit{T} = \unit{N/(A\,m)} = \unit{Vs/m^2}.
\end{equation}

\subsection{Nicht-SI-Einheiten}
\begin{tabular}{l|r|l}
\hline
\thbf{Einheit} & \thbf{Symbol} & \thbf{Umrechnung}\pstrut{1pt}\\
\hline
\multicolumn{3}{l}{\thbf{Zeit:}\pstrut{4pt}}\\
\hline
Minute & min & = 60\,s\pstrut{2pt}\\
Stunde & h & = 60\,min = 3600\,s\\
Tag & d & = 24\,h = 86400\,s\\
Jahr & a & = 356,25\,d\\
\hline
\multicolumn{3}{l}{\thbf{Druck:}\pstrut{4pt}}\\
\hline
bar & bar & $= 10^5\,\unit{Pa}$\pstrut{2pt}\\
mmHg & mmHg & = 133,322 Pa\\
\hline
\multicolumn{3}{l}{\thbf{Fläche:}\pstrut{4pt}}\\
\hline
Ar & a & $= 100\,\unit{m^2}$\pstrut{2pt}\\
Hektar & ha & = 100\,a = $10\,000\,\unit{m^2}$\\
\hline
\multicolumn{3}{l}{\thbf{Masse:}\pstrut{4pt}}\\
\hline
Tonne & t & = 1000\,kg\pstrut{2pt}\\
\hline
\multicolumn{3}{l}{\thbf{Länge:}\pstrut{4pt}}\\
\hline
Liter & L & = $10^{-3}\,\unit{m^3}$\pstrut{2pt}
\end{tabular}

\subsection{Britische Einheiten}
\begin{tabular}{l|r|l}
\thbf{Einheit} & \thbf{Abk.} & \thbf{Umrechnung}\\
inch & in. & = 2,54\,cm\\
foot & ft. & = 12\,in. = 30,48\,cm\\
yard & yd. & = 3\,ft. = 91,44\,cm\\
chain & ch. & = 22\,yd. = 20,1168\,m\\
&\\[-4pt]
furlong & fur. & = 10\,ch. = 201,168\,m\\
mile & mi. & = 1760\,yd. = 1609,3440\,m
\end{tabular}


