
\chapter{Algebra}
\section{Gruppentheorie}
\subsection{Grundbegriffe}
\begin{Definition}
Sind $(G,*)$ und $(H,\bullet)$ zwei Gruppen, so
heißt $\varphi\colon G\to H$ \emdef{Gruppenhomomorphismus}
\index{Gruppenhomomorphismus}, wenn
\begin{gather}
\forall g_1,g_2\in G\colon
  \varphi(g_1*g_2) = \varphi(g_1)\bullet\varphi(g_2)
\end{gather}
gilt.
\end{Definition}
\begin{Definition}
\emdef{Direktes Produkt}\index{direktes Produkt}:
\begin{gather}
G\times H := \{(g,h)\mid g\in G, h\in H\},\\
(g_1,h_1)*(g_2,h_2) := (g_1*g_2, h_1*h_2).
\end{gather}
\end{Definition}

\subsection{Gruppenaktionen}
\begin{Definition}
Eine Funktion $f\colon G\times X\to X$ heißt
\emdef{Gruppenaktion}\index{Gruppenaktion}, wenn
\begin{gather}
\hspace{-1em}\forall g_1,g_2{\in}G, x{\in}X\colon f(g_1,f(g_2,x)) = f(g_1 g_2,x),\\
\hspace{-1em}\forall x\in X\colon f(e,x) = x
\end{gather}
gilt, wobei mit $e$ das neutrale Element von $G$ gemeint ist.
Anstelle von $f(g,x)$ wird üblicherweise kurz $gx$ geschrieben.
\end{Definition}
