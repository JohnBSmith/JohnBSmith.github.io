
\chapter{Kombinatorik}
\section{Kombinatorische Funktionen}
\subsection{Faktorielle}\index{Faktorielle}
\subsubsection{Fakultät}\index{Fakultät}
\strong{Definition.} Für $n\in\Z, n\ge 0$:
\begin{equation}
n! := \prod_{k=1}^n k.
\end{equation}
Rekursionsgleichung:
\begin{equation}
(n+1)! = n!\,(n+1)
\end{equation}
Die Gammafunktion ist eine Verallgemeinerung der Fakultät:
\begin{equation}
n! = \Gamma(n+1).
\end{equation}

\subsubsection{Fallende Faktorielle}
\strong{Definition.} Für $a\in\C$ und $k\ge 0$:
\begin{equation}\label{eq:FF}
a^{\underline k} := \prod_{j=0}^{k-1} (a-j).
\end{equation}
Für $a,k\in\C$:
\begin{equation}
a^{\underline k} := \lim_{x\to a}\frac{\Gamma(x+1)}{\Gamma(x-k+1)}.
\end{equation}
Für $n\ge k$ und $k\ge 0$ gilt:
\begin{equation}
n^{\underline k} = \frac{n!}{(n-k)!}.
\end{equation}

\subsubsection{Steigende Faktorielle}
\strong{Definition.} Für $a\in\C$ und $k\ge 0$:
\begin{equation}
a^{\overline k} := \prod_{j=0}^{k-1} (a+j).
\end{equation}
Für $a,k\in\C$:
\begin{equation}
a^{\overline k} := \lim_{x\to a}\frac{\Gamma(x+k)}{\Gamma(x)}.
\end{equation}
Für $n\ge 1$ und $n+k\ge 1$ gilt:
\begin{equation}
n^{\overline k} = \frac{(n+k-1)!}{(n-1)!}.
\end{equation}

\subsection{Binomialkoeffizienten}\index{Binomialkoeffizient}
\strong{Definition.} Für $a\in\C$
und $k\in\Z$:
\begin{equation}
\binom{a}{k} := \begin{cases}
\frac{a^{\underline k}}{k!} & \text{wenn}\;k>0,\\
1 & \text{wenn}\;k=0,\\
0 & \text{wenn}\;k<0.
\end{cases}
\end{equation}
Für $a,b\in\C$:
\begin{equation}\label{eq:bc-allg}
\binom{a}{b} := \lim_{x\to a}\lim_{y\to b}
\frac{\Gamma(x+1)}{\Gamma(y+1)\Gamma(x-y+1)}.
\end{equation}
Für $0\le k\le n$ gilt die Symmetriebeziehung
\begin{equation}
\binom{n}{k} = \binom{n}{n-k}
\end{equation}
und die Rekursionsgleichung
\begin{equation}
\binom{n+1}{k+1} = \binom{n}{k+1}+\binom{n}{k}.
\end{equation}
Für $a\in\C$ und $k\in\Z$ gilt:
\begin{equation}
\binom{-a}{k} = (-1)^k \binom{a+k-1}{k}.
\end{equation}

\section{Formale Potenzreihen}
\subsection{Binomische Reihe}
\strong{Definition.} Für $a\in\C$:
\begin{equation}
(1+X)^a := \sum_{k=0}^\infty \binom{a}{k} X^k
\end{equation}
Es gilt:
\begin{equation}
(1+X)^{a+b} = (1+X)^a (1+X)^b 
\end{equation}
und
\begin{equation}
(1+X)^{ab} = ((1+X)^a)^b.
\end{equation}
