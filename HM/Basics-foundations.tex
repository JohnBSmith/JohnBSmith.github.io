
\chapter{Basics and foundations}
\section{Complex numbers}
\subsection{Operations}

\begin{gather}
\frac{z_1}{z_2}
= \frac{z_1\overline z_2}{z_2\overline z_2}
= \frac{z_1\overline z_2}{|z_2|^2},\\
\frac{1}{z} = \frac{\overline z}{z\overline z}
= \frac{\overline z}{|z|^2}.
\end{gather}

\subsection{Absolut value}
For all $z_1,z_2\in\mathbb C$:
\begin{gather}
|z_1z_2| = |z_1|\,|z_2|,\\
z_2\ne 0\implies \Big|\frac{z_1}{z_2}\Big|
= \frac{|z_1|}{|z_2|},\\
z\,\overline z = |z|^2.
\end{gather}

\subsection{Conjugation}
For all $z_1,z_2\in\mathbb C$:
\begin{gather}
\overline{z_1+z_2} = \overline z_1+\overline z_2,\qquad
\overline{z_1-z_2} = \overline z_1-\overline z_2,\\
\overline{z_1 z_2} = \overline z_1\,\overline z_2,\qquad
z_2\ne 0 \implies \overline{\Big(\frac{z_1}{z_2}\Big)}
= \frac{\overline z_1}{\overline z_2},\\
\overline{\overline z}=z,\qquad
|\overline{z}| = |z|,\qquad
z\,\overline z = |z|^2,\\
\operatorname{Re}(z) = \frac{z+\overline z}{2},\qquad
\operatorname{Im}(z) = \frac{z-\overline z}{2\ui},\\
\overline{\cos(z)} = \cos(\overline z),\qquad
\overline{\sin(z)} = \sin(\overline z),\\
\overline{\exp(z)} = \exp(\overline z).
\end{gather}

\begin{table*}[t]
\caption{Operations}
\bgroup
\def\arraystretch{1.4}
\begin{tabular}{|l|r|l|l|}
\hline
  \thbf{Name}
& \thbf{Operation}
& \thbf{Polar form}
& \thbf{Cartesian form}\\
\hline
  Identity
& $z$ & $=r\ee^{\ui\varphi}$
& $= a+b\ui$\\
\hline
  Addition
& $z_1+z_2$ &
& $= (a_1+a_2)+(b_1+b_2)\ui$\\
\hline
  Subtraction
& $z_1-z_2$ &
& $= (a_1-a_2)+(b_1-b_2)\ui$\\
\hline
  Multiplication
& $z_1 z_2$
& $= r_1 r_2 \ee^{\ui(\varphi_1+\varphi_2)}$
& $= (a_1 a_2 - b_1 b_2)+(a_1 b_2+a_2 b_1)\ui$\\
\hline
  Division
& $\displaystyle\frac{z_1}{z_2}$
& $\displaystyle =\frac{r_1}{r_2}\ee^{\ui(\varphi_1-\varphi_2)}$
& $\displaystyle =\frac{a_1 a_2 + b_1 b_2}{a_2^2+b_2^2}
   + \frac{a_2 b_1 - a_1 b_2}{a_2^2+b_2^2}\ui$\\
\hline
  Reciprocal
& $\displaystyle\frac{1}{z}$
& $\displaystyle =\frac{1}{r}\ee^{-\ui\varphi}$
& $\displaystyle =\frac{a}{a^2+b^2}-\frac{b}{a^2+b^2}\ui$\\
\hline
  Real part
& $\operatorname{Re}(z)$
& $=\cos\varphi$
& $=a$\\
\hline
  Imaginary part
& $\operatorname{Im}(z)$
& $=\sin\varphi$
& $=b$\\
\hline
  Conjugation
& $\overline{z}$
& $=r\ee^{-\varphi\ui}$
& $=a-b\ui$\\
\hline
  Absolut value
& $|z|$
& $=r$
& $=\sqrt{a^2+b^2}$\\
\hline
  Argument
& $\arg(z)$
& $=\varphi$
& $\displaystyle =s(b)\arccos\Big(\frac{a}{r}\Big)$\\
\hline
\end{tabular}
\egroup\\
\\
$s(b):=\begin{cases}
+1 & \text{if}\;b\ge 0,\\
-1 & \text{if}\;b<0
\end{cases}$
\end{table*}



\section{Logic}
\subsection{Propositional logic}
\subsubsection{Boolean algebra}
\begin{table*}[t]
\caption{Boolean algebra}
\begin{tabular}{l|l|l}
\thbf{Disjunction} & \thbf{Conjunction} &\\
  $A\lor A \Leftrightarrow A$
& $A\land A \Leftrightarrow A$
& Laws of idempotence\\
  $A\lor 0 \Leftrightarrow A$
& $A\land 1 \Leftrightarrow A$
& Laws of neutrality\\
  $A\lor 1 \Leftrightarrow 1$
& $A\land 0 = 0$
& Laws of annihilation\\
  $A\lor \overline A \Leftrightarrow 1$
& $A\land \overline A \Leftrightarrow 0$
& Laws of complementation\\
\noalign{\vspace{1em}}
  $A\lor B \Leftrightarrow B\lor A$
& $A\land B \Leftrightarrow B\land A$
& Laws of commutativity\\
  $(A\lor B)\lor C \Leftrightarrow A\lor (B\lor C)$
& $(A\land B)\land C \Leftrightarrow A\land (B\land C)$
& Laws of associativity\\
  $\overline{A\lor B} \Leftrightarrow \overline A\land\overline B$
& $\overline{A\land B} \Leftrightarrow \overline A\lor\overline B$
& De Morgan's laws\\
  $A\lor (A\land B) \Leftrightarrow A$
& $A\land (A\lor B) \Leftrightarrow A$
& Laws of absorption\\
\end{tabular}
\end{table*}

\noindent
\strong{Laws of distributivity}:
\begin{gather}
A\lor (B\land C) = (A\lor B)\land (A\lor C),\\
A\land (B\lor C) = (A\land B)\lor (A\land C).
\end{gather}

\subsubsection{Functions in two arguments}
There are 16 boolean functions in two arguments.

\begin{tabular}{r|l}
\textbf{\texttt{AB}} & \thbf{Wert}\\
\texttt{00} & \texttt{a}\\
\texttt{01} & \texttt{b}\\
\texttt{10} & \texttt{c}\\
\texttt{11} & \texttt{d}
\end{tabular}

\begin{tabular}{r|l|l|l}
\thbf{No.}& \textbf{\texttt{dcba}} & \thbf{Function} & \thbf{Name}\\
 0 & \texttt{0000} & 0 & Contradiction\\
 1 & \texttt{0001} & $\overline{A\lor B}$ & NOR\\
 2 & \texttt{0010} & $\overline{B\Rightarrow A}$\\
 3 & \texttt{0011} & $\overline A$\\
 4 & \texttt{0100} & $\overline{A\Rightarrow B}$\\
 5 & \texttt{0101} & $\overline{B}$\\
 6 & \texttt{0110} & $A\oplus B$ & Contravalence\\
 7 & \texttt{0111} & $\overline{A\land B}$ & NAND\\
 8 & \texttt{1000} & $A\land B$ & Conjunction\\
 9 & \texttt{1001} & $A\Leftrightarrow B$ & Equivalence\\
10 & \texttt{1010} & $B$ & Projection\\
11 & \texttt{1011} & $A\Rightarrow B$ & Implication\\
12 & \texttt{1100} & $A$ & Projection\\
13 & \texttt{1101} & $B\Rightarrow A$ & Implication\\
14 & \texttt{1110} & $A\lor B$ & Disjunction\\
15 & \texttt{1111} & $1$ & Tautology
\end{tabular}

\subsection{Predicate logic}
\subsubsection{Basic laws}
Negation (De Morgan's laws):
\begin{gather}
\overline{\forall x[P(x)]}\iff \exists x[\overline{P(x)}],\\
\overline{\exists x[P(x)]}\iff \forall x[\overline{P(x)}].
\end{gather}
Generalized laws of distributivity:
\begin{gather}
P\lor\forall x[Q(x)] \iff \forall x[P\lor Q(x)],\\
P\land\exists x[Q(x)] \iff \exists x[P\land Q(x)].
\end{gather}
Generalized laws of idempotence:
\begin{gather}
\begin{split}
\exists x{\in}M\,[P] & \iff
(M\ne\{\})\land P\\
& \iff\begin{cases}
P & \text{if}\; M\ne\{\},\\
0 & \text{if}\; M=\{\}.
\end{cases}
\end{split}\\
\begin{split}
\forall x{\in}M\,[P]& \iff
(M=\{\})\lor P\\
&\iff\begin{cases}
P & \text{if}\; M\ne\{\},\\
1 & \text{if}\; M=\{\}.
\end{cases}
\end{split}
\end{gather}
Equivalences:
\begin{gather}
\hspace{-2em}\forall x\forall y[P(x,y)] \iff \forall y\forall x[P(x,y)],\\
\hspace{-2em}\exists x\exists y[P(x,y)] \iff \exists y\exists x[P(x,y)],\\
\hspace{-2em}\forall x[P(x)\land Q(x)] \iff \forall x[P(x)]\land\forall x[Q(x)],\\
\hspace{-2em}\exists x[P(x)]\lor Q(x)] \iff \forall x[P(x)]\lor\forall x[Q(x)],\\
\hspace{-2em}\forall x[P(x)\Rightarrow Q] \iff \exists x[P(x)]\Rightarrow Q,\\
\hspace{-2em}\forall x[P\Rightarrow Q(x)] \iff P\Rightarrow\forall x[Q(x)],\\
\hspace{-2em}\exists x[P(x)\Rightarrow Q(x)]
  \iff\forall x[P(x)]\Rightarrow\exists x[Q(x)].
\end{gather}
Implications:
\begin{gather}
\hspace{-2em}\exists x\forall y[P(x,y)]\implies \forall y\exists x[P(x,y)],\\
\hspace{-2em}\forall x[P(x)]\lor\forall x[Q(x)]\implies\forall x[P(x)\lor Q(x)],\\
\hspace{-2em}\exists x[P(x)\land Q(x)]\implies
  \exists x[P(x)]\land \exists x[Q(x)],\\
\hspace{-2em}\forall x[P(x)\Rightarrow Q(x)]\implies
  (\forall x[P(x)]\Rightarrow\forall x[Q(x)]),\\
\hspace{-2em}\forall x[P(x)\Leftrightarrow Q(x)]\implies
  (\forall x[P(x)]\Leftrightarrow\forall x[Q(x)]).
\end{gather}

\subsubsection{Finite sets}
Let $M=\{x_1,\ldots,x_n\}$. One has:
\begin{gather}
\forall x{\in}M\,[P(x)]\iff P(x_1)\land\ldots\land P(x_n),\\
\exists x{\in}M\,[P(x)]\iff P(x_1)\lor\ldots\lor P(x_n).
\end{gather}

\subsubsection{Restricted quantification}
\begin{gather}
\begin{split}
& \forall x{\in}M\,[P(x)]\;:\Longleftrightarrow\;\forall x[x\notin M\lor P(x)]\\
& \quad\iff\forall x[x\in M\Rightarrow P(x)],
\end{split}\\
\exists x{\in}M\,[P(x)]\;:\Longleftrightarrow\;\exists x[x\in M\land P(x)],\\
\forall x{\in}M{\setminus}N\,[P(x)]\iff \forall x[x\notin N\Rightarrow P(x)].
\end{gather}

\subsubsection{Product sets as\\
domains of discourse}
\begin{gather}
\forall(x,y)\,[P(x,y)]\iff \forall x\forall y[P(x,y)],\\
\exists(x,y)\,[P(x,y)]\iff \exists x\exists y[P(x,y)].
\end{gather}
By analogy:
\begin{gather}
\forall(x,y,z)\,\iff \forall x\forall y\forall z,\\
\exists(x,y,z)\,\iff \exists x\exists y\exists z
\end{gather}
etc.

\subsubsection{Alternative representation}
Let $P\colon G\to\{0,1\}$ and $M\subseteq G$.
Let $P(M)$ be the image of $M$ under $P$. One has
\begin{equation}
\begin{split}
&\forall x{\in}M\,[P(x)] \iff P(M)=\{1\}\\
& \iff M\subseteq\{x{\in}G\mid P(x)\}
\end{split}
\end{equation}
and
\begin{equation}
\begin{split}
& \exists x{\in}M\,[P(x)] \iff \{1\}\subseteq P(M)\\
& \iff M\cap\{x{\in}G\mid P(x)\}\ne\{\}.
\end{split}
\end{equation}

\subsubsection{Uniqueness}
Quantifier of unique existence:
\begin{equation}
\begin{split}
&\exists!x\,[P(x)]\\
&:\Longleftrightarrow\; \exists x\,[P(x)\land \forall y\,[P(y)\Rightarrow x=y]]\\
&\iff \exists x\,[P(x)]\land \forall x\forall y[P(x)\land P(y)\Rightarrow x=y].
\end{split}
\end{equation}



\section{Set theory}
\subsection{Definitions}
Subset relation:
\begin{equation}
A\subseteq B\;:\Longleftrightarrow\; \forall x\,[x\in A\implies x\in B].
\end{equation}
Equality:
\begin{equation}
A=B\;:\Longleftrightarrow\; \forall x\,[x\in A\iff x\in B].
\end{equation}
Union:
\begin{equation}
A\cup B:=\{x\mid x\in A\lor x\in B\}.
\end{equation}
Intersection:
\begin{equation}
A\cap B:=\{x\mid x\in A\land x\in B\}.
\end{equation}
Difference set:
\begin{equation}
A\setminus B:=\{x\mid x\in A\land x\not\in B\}.
\end{equation}
Symmetric difference:
\begin{equation}
A\triangle B:=\{x\mid x\in A\oplus x\in B\}.
\end{equation}

\subsection{Boolean algebra}
\begin{table*}[t]
\caption{Boolean algebra}
\begin{tabular}{l|l|l}
\thbf{Union} & \thbf{Intersection} &\\
  $A\cup A = A$
& $A\cap A = A$
& Laws of idempotence\\
  $A\cup \{\} = A$
& $A\cap G = A$
& Laws of neutrality\\
  $A\cup G = G$
& $A\cap \{\} = \{\}$
& Laws of annihilation\\
  $A\cup \overline A = G$
& $A\cap \overline A = \{\}$
& Laws of complementation\\
\noalign{\vspace{1em}}
  $A\cup B = B\cup A$
& $A\cap B = B\cap A$
& Laws of commutativity\\
  $(A\cup B)\cup C = A\cup (B\cup C)$
& $(A\cap B)\cap C = A\cap (B\cap C)$
& Laws of associativity\\
  $\overline{A\cup B} = \overline A\cap\overline B$
& $\overline{A\cap B} = \overline A\cup\overline B$
& De Morgan's laws\\
  $A\cup (A\cap B) = A$
& $A\cap (A\cup B) = A$
& Laws of absorption\\
\end{tabular}\\
\\
$G$: Universe
\end{table*}

\noindent
\strong{Laws of distributivity}:
\begin{gather}
M\cup (A\cap B) = (M\cup A)\cap (M\cup B),\\
M\cap (A\cup B) = (M\cap A)\cup (M\cap B).
\end{gather}

\subsection{Subset relation}
Decomposition of equality:
\begin{equation}
A=B \iff A\subseteq B \land B\subseteq A.
\end{equation}
Pharaphrasing of subset relations:
\begin{equation}
\begin{split}
A\subseteq B &\iff A\cap B=A\\
& \iff A\cup B=B\\
& \iff A\setminus B=\{\}.
\end{split}
\end{equation}
Law of contraposition:
\begin{equation}
A\subseteq B = \overline B\subseteq \overline A.
\end{equation}

\subsection{Inductive sets}
Set theoretical model of the natural numbers:
\begin{equation}
\begin{split}
& 0:=\{\},\quad 1:=\{0\},\quad 2:=\{0,1\},\\
& 3:=\{0,1,2\},\quad \text{usw.}
\end{split}
\end{equation}
Successor function:
\begin{equation}
x' := x\cup\{x\}.
\end{equation}
Proof by induction: For a predicate $A(n)$ with $n\in\mathbb N$
one has:
\begin{equation}
\begin{split}
& A(n_0)\land \forall n\ge n_0\,[A(n)\Rightarrow A(n+1)]\\
& \implies \forall n\ge n_0\,[A(n)].
\end{split}
\end{equation}

