\documentclass[a4paper,11pt,fleqn,twocolumn]{article}
\usepackage[utf8]{inputenc}
\usepackage[T1]{fontenc}
\usepackage{mathptmx}
\usepackage{ngerman}
\usepackage{amsmath}
\usepackage{amssymb}
\usepackage{color}
\definecolor{c1}{RGB}{60,0,40}

\usepackage{geometry}
\geometry{a4paper,left=25mm,right=14mm,top=20mm,bottom=28mm}
\setlength{\columnsep}{4mm}

\usepackage[colorlinks=true,linkcolor=black,citecolor=black]{hyperref}
\newcommand{\unit}[1]{\mathrm{#1}}

\begin{document}

\thispagestyle{empty}
\section*{Physikalische Chemie}
\section{Thermodynamik}
\subsection{Heizwert und Verbrennungsenthalpie}

In einem Tafelwerk für die Oberstufe \cite{Cornelsen}
sind Tabellen zum
(unteren) Heizwert $H$ und an anderer Stelle zur
molaren Standardverbrennungsenthalpie $\Delta_c H_m^0$ auszumachen.
Dabei stellt sich natürlich die Frage, ob und wie beides zusammenhängt.

Einfaches Umrechnen über die molare Masse $M$ bringt gewisse
Abweichungen. Tatsächlich handelt es sich dabei um den \emph{Brennwert}
$B$, welcher auch als \emph{oberer Heizwert} bezeichnet wird.
Es gilt also
\begin{equation}\label{eq:Brennwert}
mB = nMB = n\,|\Delta_c H_m^0|
\end{equation}
mit $m$: Masse, $n$: Stoffmenge.

Nun ist aber zu beachten, dass bei einer typischen Verbrennung
einer organischen Substanz Wasser entstehen wird.
Die Reaktionsgleichung für die Verbrennung von Ethanol ist z.\,B.
\begin{equation}
\mathrm{C_2H_5OH + 3O_2 \longrightarrow 3 H_2O + 2 CO_2.}
\end{equation}
Man muss nun beachten, dass das Wasser nach der Reaktion verdampft,
wofür Verdampfungswärme aufgebracht werden muss.

Als Ansatz macht man die Energiebilanz:
\begin{equation}\label{eq:Energiebilanz}
mH = mB-m(\mathrm{H_2O})\,q_v(\mathrm{H_2O})
\end{equation}
mit $q_v$: spezifische Verdampfungswärme.

Nun interessieren uns aber gerade keine Masse, sondern wir wollen
alle Rechnungen molar durchführen. Formt man
\eqref{eq:Energiebilanz} mit $M=m/n$ um und setzt \eqref{eq:Brennwert}
für $B$ ein, so ergibt sich
\begin{equation}
MH = |\Delta_c H_m^0| - \frac{n(\mathrm{H_2O})}{n}
M(\mathrm{H_2O})\,q_v(\mathrm{H_2O}).
\end{equation}
Hierbei sind $n(\mathrm{H_2O})$ und $n$ der Reaktionsgleichung
zu entnehmen.
Bei der Verbrennung von Ethanol gilt z.\,B.
\[\begin{split}
n(\mathrm{H_2O}) &= 3\,\unit{mol},\\
n(\mathrm{C_2H_5OH}) &= 1\,\unit{mol},\\
M(\mathrm{H_2O}) &= 0{,}018015\:\unit{kg/mol},\\
M(\mathrm{C_2H_5OH}) &= 0{,}046068\:\unit{kg/mol},\\
q_v(\mathrm{H_2O}) &= 2260\:\unit{kJ/kg},\;\cite{Cornelsen}\\
\Delta_c H_m^0 &= -1364\:\unit{kJ/mol}.\;\cite{Cornelsen}
\end{split}
\]
Damit ergibt sich $H\approx 26{,}96\:\unit{MJ/kg}$.
In Tabellen findet man Werte von $26{,}8$ bis $26{,}9$.

\subsection{Verbrennungsenthalpie}
Die Verbrennung ist eigentlich eine ganz gewöhnliche
chemische Reaktion. Tatsächlich stimmt die Verbrennungsenthalpie
mit der Reaktionsenthalpie überein:
\begin{equation}
\Delta_c H_m^0 = \Delta_R H_m^0.
\end{equation}
Die Reaktionsenthalpie lässt sich wiederum berechnen, wenn die
Standardbildungsenthalpien bekannt sind. Bezeichnet man mit
$P$ die Menge der Produkte und mit $E$ die Menge der Edukte, so gilt
\begin{equation}
\Delta_R H_m^0 = \sum_{\mathrm X\in P}\Delta_f H_m^0(\mathrm X)
- \sum_{\mathrm X\in E}\Delta_f H_m^0(\mathrm X).
\end{equation}
Diese Formel beruht auf dem Satz von Hess.

\subsection{Bildungsenthalpie}
Die Standardbildungsenthalpie von chemischen Elementen
in ihrem stabilsten Zustand ist als $0\,\unit{kJ/mol}$ definiert.

Die Standardbildungsenthalpie von einer beliebigen Substanz
ist dann die Standardreaktionsenthalpie für die Reaktion bei der
sich die Elemente in ihrem stabilsten Zustand
zu der Substanz zusammensetzen.

Betrachten wir zunächst die Verbrennung von Wasserstoff.
Per Definition ist $\Delta_f H_m^0(\mathrm{H_2})=0$
und $\Delta_f H_m^0(\mathrm{O_2})=0$. Daher stimmt die
Verbrennungsenthalpie von Wasserstoff mit der Bildungsenthalpie
von Wasser über ein. In Tafelwerk \cite{Cornelsen} findet man
\[\Delta_f H_m^0 (\mathrm{H_2O}) = -286\:\unit{kJ/mol}.\]
Man kann auch vor der Situation stehen, dass Reaktionsenthalpien
gegeben sind und eine Bildungsenthalpie ermittelt werden muss.
Hierzu beachtet man den Satz von Hess und stellt die Formeln,
die sich dabei ergeben, entsprechend um.

\begin{thebibliography}{xx}
\setlength{\itemsep}{0pt}
\bibitem{Cornelsen} Wolfgang Pfeil (Chemie), Willi Wörstenfeld (Physik)
et al:
»Das große Tafelwerk«.
Cornelsen/Volk und Wissen, Berlin, 1. Auflage, 13. Druck 2010.
\end{thebibliography}

\vfill

\noindent
{\small Dieser Text steht unter der Lizenz Creative Commons CC0.}

\end{document}
