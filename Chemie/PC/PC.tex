\documentclass[a4paper,11pt,fleqn,twocolumn]{article}
\usepackage[utf8]{inputenc}
\usepackage[T1]{fontenc}
\usepackage{mathptmx}
\usepackage{ngerman}
\usepackage{amsmath}
\usepackage{amssymb}
\usepackage{color}
\definecolor{c1}{RGB}{60,0,40}

\usepackage{geometry}
\geometry{a4paper,left=25mm,right=14mm,top=20mm,bottom=28mm}
\setlength{\columnsep}{4mm}

\usepackage[colorlinks=true,linkcolor=c1]{hyperref}
\newcommand{\unit}[1]{\mathrm{#1}}

\begin{document}

\thispagestyle{empty}
\section*{Physikalische Chemie}
\section{Thermodynamik}
\subsection{Heizwert und Verbrennungsenthalpie}

In einem Tafelwerk für die Oberstufe sind Tabellen zum
(unteren) Heizwert $H$ und an anderer Stelle zur
molaren Standardverbrennungsenthalpie $\Delta_c H_m^0$ auszumachen.
Dabei stellt sich natürlich die Frage, ob und wie beides zusammenhängt.

Einfaches Umrechnen über die molare Masse $M$ bringt gewisse
Abweichungen. Tatsächlich handelt es sich dabei um den \emph{Brennwert}
$B$, welcher auch als \emph{oberer Heizwert} bezeichnet wird.
Es gilt also
\begin{equation}\label{eq:Brennwert}
mB = nMB = n\,|\Delta_c H_m^0|
\end{equation}
mit $m$: Masse, $n$: Stoffmenge.

Nun ist aber zu beachten, dass bei einer typischen Verbrennung
einer organischen Substanz Wasser entstehen wird.
Die Reaktionsgleichung für die Verbrennung von Ethanol ist z.\,B.
\begin{equation}
\mathrm{C_2H_5OH + 3O_2 \longrightarrow 3 H_2O + 2 CO_2.}
\end{equation}
Man muss nun beachten, dass das Wasser nach der Reaktion verdampft,
wofür Verdampfungswärme aufgebracht werden muss.

Als Ansatz macht man die Energiebilanz:
\begin{equation}\label{eq:Energiebilanz}
mH = mB-m(\mathrm{H_2O})\,q_v(\mathrm{H_2O})
\end{equation}
mit $q_v$: spezifische Verdampfungswärme.

Nun interessieren uns aber gerade keine Masse, sondern wir wollen
alle Rechnungen molar durchführen. Formt man
\eqref{eq:Energiebilanz} mit $M=m/n$ um und setzt \eqref{eq:Brennwert}
für $B$ ein, so ergibt sich
\begin{equation}
MH = |\Delta_c H_m^0| - \frac{n(\mathrm{H_2O})}{n}
M(\mathrm{H_2O})\,q_v(\mathrm{H_2O}).
\end{equation}
Hierbei sind $n(\mathrm{H_2O})$ und $n$ der Reaktionsgleichung
zu entnehmen.
Bei der Verbrennung von Ethanol gilt z.\,B.
\[\begin{split}
n(\mathrm{H_2O}) &= 3\,\unit{mol},\\
n(\mathrm{C_2H_5OH}) &= 1\,\unit{mol},\\
M(\mathrm{H_2O}) &= 0{,}018015\:\unit{kg/mol},\\
M(\mathrm{C_2H_5OH}) &= 0{,}046068\:\unit{kg/mol},\\
q_v(\mathrm{H_2O}) &= 2260\:\unit{kJ/kg},\\
\Delta_c H_m^0 &= -1364\:\unit{kJ/mol}.
\end{split}
\]
Damit ergibt sich $H\approx 26{,}96\:\unit{MJ/kg}$.
In Tabellen findet man Werte von $26{,}8$ bis $26{,}9$.

\vfill

\noindent
{\small Dieser Text steht unter der Lizenz Creative Commons CC0.}

\end{document}
