\documentclass[a4paper,11pt,fleqn,twocolumn]{article}
\usepackage[utf8]{inputenc}
\usepackage[T1]{fontenc}
\usepackage[ngerman]{babel}
\usepackage{microtype}

\usepackage{mathptmx}
% \usepackage{libertine}
% \usepackage[libertine]{newtxmath}
% \renewcommand\ttdefault{lmtt}


\usepackage{amsmath}
\usepackage{amssymb}
\usepackage{color}
\definecolor{c1}{RGB}{60,0,40}

\usepackage{geometry}
\geometry{a4paper,left=25mm,right=14mm,top=20mm,bottom=28mm}
\setlength{\columnsep}{4mm}

\usepackage[colorlinks=true,linkcolor=c1]{hyperref}

\begin{document}

\thispagestyle{empty}
\section*{Praktikum Chemie}
\subsection*{Titration}

Bei der Titration versucht man zu bestimmen, mit welcher Konzentration
eine Substanz in einer Probe vorhanden ist. Die Titration ist
ein Verfahren zur sogenannten quantitativen Analyse. D.h. die
Erfassung von Informationen geschieht zahlenmäßig. Im Gegensatz zu
einer qualitativen Analyse ist man nicht nur an einer ja/nein-Antwort
interessiert, sondern fragt auch \textit{wie viel}.

Zunächst muss man zwischen Massenkonzentration $\beta$ und
Stoffmengenkonzentration $c$ unterscheiden. Welche davon Verwendung
findet,
ist eigentlich belanglos, denn mit der Formel $\beta = Mc$ lässt
sich die eine in die andere Umrechnen. Bei $M$ handelt es sich
um die molare Masse der betrachteten Substanz. Da man schon weiß,
nach welcher Substanz man sucht, kann $M$ natürlich immer
als bekannt vorausgesetzt werden.

Zunächst beschränkt man sich zur Vereinfachung auf die Titration
einer starken Säure mit einer starken Base. Als Beispiel zur
Einführung wird immer die Titration von HCL mit NaOH gewählt.
Bei HCL handelt sich dabei um die zu analysierende Substanz (Index A),
und bei NaOH handelt es sich um die Maßlösung (Index M).

Für die Messung ist es wesentlich, dass die Stoffmengenkonzentration
der Maßlösung möglichst genau bekannt ist. Für sehr genaue Messungen
wird man sich diese Konzentration eventuell über eine sogenannte
Titerbestimmung beschaffen, da sich die Konzentration im Laufe
der Zeit ändern kann.

Sei $c_0$ die genaue Stoffmengenkonzentration der Maßlösung. Es hat sich
eingebürgert, nur den näherungsweisen Wert $c$ der Konzentration
anzugeben. Man kann nun schreiben $c_0=ct$ wobei $t$ der
Korrekturfaktor ist. Dieser Korrekturfaktor $t$ wird \emph{Titer}
genannt. Der Titer liegt normalerweise nahe bei eins.

Bei der Durchführung der Titration lässt man mit einer Bürette Maßlösung
in ein Becherglas mit der Probe tropfen. Auf irgendeine Art und
Weise muss dabei der sogenannte Äquivalenzpunkt bestimmt werden.
Zur Bestimmung des Äquivalenzpunkts gibt es mehrere Möglichkeiten:
Verwendung eines Indikators, Konduktometrie oder Potenziometrie.

Man macht nun einen Stoffmengenansatz. Am Äquivalenzpunkt ist
\begin{equation}
n(\mathrm A)=n(\mathrm M),
\end{equation}
mit M: Maßlösung und A: Substanz in der Probe.
Dies funktioniert bei HCl mit Maßlösung NaOH.
Die Reaktionsgleichung ist
\[\mathrm{HCl + NaOH \longrightarrow H_2O + NaCl}.\]
Bei der Reaktion
\[\mathrm{2\,HCl + Mg(OH)_2 \longrightarrow 2\,H_2O + MgCl_2}\]
benötigt man für 2 HCl jedoch nur ein $\mathrm{Mg(OH)_2}$.
Man wird weniger mol Maßlösung $\mathrm{Mg(OH)_2}$ brauchen,
um die HCl zu neutralisieren. Der Äquivalenzpunkt liegt
dann also bei $n(\mathrm A)=2n(\mathrm M)$. Wir führen eine Zahl
$z$ ein. Unsere Gleichung lautet dann allgemein
\begin{equation}
z(\mathrm A)n(\mathrm A) = z(\mathrm M)n(\mathrm M).
\end{equation}
Mit $m=Mn$ und $n=cV$ erhält man
\begin{equation}
z(\mathrm A)\frac{m(\mathrm A)}{M(\mathrm A)}
= z(\mathrm M)c(\mathrm M)V(\mathrm M).
\end{equation}
Wir können die Indizes bei der Maßlösung entfernen, um
uns kurz zu halten. Die Gleichung lautet dann
\begin{equation}
z(\mathrm A)\frac{m(\mathrm A)}{M(\mathrm A)} = zcV.
\end{equation}
Umformen bringt
\begin{equation}
m(\mathrm A) = \frac{z}{z(\mathrm A)} M(\mathrm A) cV.
\end{equation}
Ersetzt man nun $c$ gegen $c_0$ und dann $c_0=ct$,
so lautet die Gleichung
\begin{equation}
m(\mathrm A) = \frac{z}{z(\mathrm A)} M(\mathrm A) ctV.
\end{equation}
Verwendet man nicht die gesamte zu untersuchende Lösung, sondern
nur einen Teil $V_1 = (1/10)V(\mathrm A)$, so errechnet man auch nur
$m_1 = (1/10)m(\mathrm A)$. Wir nennen $a=10$ den aliquoten Teil.
Sei also $a:=V(\mathrm A)/V_1$, wobei $V_1$ das Teilvolumen von
$V(\mathrm A)$ ist, das titriert wird.

Bei $V_1=V(\mathrm A)$ kann $a$ offensichtlich entfallen.
Wir berücksichtigen $a$ noch in der Gleichung und erhalten
die \emph{Titrationsgleichung}
\begin{equation}
m(\mathrm A) = \frac{z}{z(\mathrm A)} M(\mathrm A)ctVa.
\end{equation}
Anschließend lässt sich die Massenkonzentration $\beta(A)$ berechnen.
Es ist $\beta(\mathrm A) = m(\mathrm A)/V(\mathrm A)$.
Normalerweise wird $\beta(\mathrm A)$ in g/(100\,mL) angegeben.
Substituiert man $a$ in der Titrationsgleichung, so lässt sich
$V(\mathrm A)$ herauskürzen und es ergibt sich
\begin{equation}
\beta(\mathrm A) = \frac{z}{z(\mathrm A)} M(\mathrm A) ct\frac{V}{V_1}.
\end{equation}
{\small Dieser Text steht unter der Lizenz Creative Commons CC0.}

\end{document}
